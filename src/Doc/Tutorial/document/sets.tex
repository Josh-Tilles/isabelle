\chapter{Sets, Functions and Relations}

This chapter describes the formalization of typed set theory, which is
the basis of much else in HOL\@.  For example, an
inductive definition yields a set, and the abstract theories of relations
regard a relation as a set of pairs.  The chapter introduces the well-known
constants such as union and intersection, as well as the main operations on relations, such as converse, composition and
transitive closure.  Functions are also covered.  They are not sets in
HOL, but many of their properties concern sets: the range of a
function is a set, and the inverse image of a function maps sets to sets.

This chapter will be useful to anybody who plans to develop a substantial
proof.  Sets are convenient for formalizing  computer science concepts such
as grammars, logical calculi and state transition systems.  Isabelle can
prove many statements involving sets automatically.

This chapter ends with a case study concerning model checking for the
temporal logic CTL\@.  Most of the other examples are simple.  The
chapter presents a small selection of built-in theorems in order to point
out some key properties of the various constants and to introduce you to
the notation. 

Natural deduction rules are provided for the set theory constants, but they
are seldom used directly, so only a few are presented here.  


\section{Sets}

\index{sets|(}%
HOL's set theory should not be confused with traditional,  untyped set
theory, in which everything is a set.  Our sets are typed. In a given set,
all elements have the same type, say~$\tau$,  and the set itself has type
$\tau$~\tydx{set}. 

We begin with \textbf{intersection}, \textbf{union} and
\textbf{complement}. In addition to the
\textbf{membership relation}, there  is a symbol for its negation. These
points can be seen below.  

Here are the natural deduction rules for \rmindex{intersection}.  Note
the  resemblance to those for conjunction.  
\begin{isabelle}
\isasymlbrakk c\ \isasymin\ A;\ c\ \isasymin\ B\isasymrbrakk\ 
\isasymLongrightarrow\ c\ \isasymin\ A\ \isasyminter\ B%
\rulenamedx{IntI}\isanewline
c\ \isasymin\ A\ \isasyminter\ B\ \isasymLongrightarrow\ c\ \isasymin\ A
\rulenamedx{IntD1}\isanewline
c\ \isasymin\ A\ \isasyminter\ B\ \isasymLongrightarrow\ c\ \isasymin\ B
\rulenamedx{IntD2}
\end{isabelle}

Here are two of the many installed theorems concerning set
complement.\index{complement!of a set}
Note that it is denoted by a minus sign.
\begin{isabelle}
(c\ \isasymin\ -\ A)\ =\ (c\ \isasymnotin\ A)
\rulenamedx{Compl_iff}\isanewline
-\ (A\ \isasymunion\ B)\ =\ -\ A\ \isasyminter\ -\ B
\rulename{Compl_Un}
\end{isabelle}

Set \textbf{difference}\indexbold{difference!of sets} is the intersection
of a set with the  complement of another set. Here we also see the syntax
for the  empty set and for the universal set. 
\begin{isabelle}
A\ \isasyminter\ (B\ -\ A)\ =\ \isacharbraceleft\isacharbraceright
\rulename{Diff_disjoint}\isanewline
A\ \isasymunion\ -\ A\ =\ UNIV%
\rulename{Compl_partition}
\end{isabelle}

The \bfindex{subset relation} holds between two sets just if every element 
of one is also an element of the other. This relation is reflexive.  These
are its natural deduction rules:
\begin{isabelle}
({\isasymAnd}x.\ x\ \isasymin\ A\ \isasymLongrightarrow\ x\ \isasymin\ B)\ \isasymLongrightarrow\ A\ \isasymsubseteq\ B%
\rulenamedx{subsetI}%
\par\smallskip%          \isanewline didn't leave enough space
\isasymlbrakk A\ \isasymsubseteq\ B;\ c\ \isasymin\
A\isasymrbrakk\ \isasymLongrightarrow\ c\
\isasymin\ B%
\rulenamedx{subsetD}
\end{isabelle}
In harder proofs, you may need to apply \isa{subsetD} giving a specific term
for~\isa{c}.  However, \isa{blast} can instantly prove facts such as this
one: 
\begin{isabelle}
(A\ \isasymunion\ B\ \isasymsubseteq\ C)\ =\
(A\ \isasymsubseteq\ C\ \isasymand\ B\ \isasymsubseteq\ C)
\rulenamedx{Un_subset_iff}
\end{isabelle}
Here is another example, also proved automatically:
\begin{isabelle}
\isacommand{lemma}\ "(A\
\isasymsubseteq\ -B)\ =\ (B\ \isasymsubseteq\ -A)"\isanewline
\isacommand{by}\ blast
\end{isabelle}
%
This is the same example using \textsc{ascii} syntax, illustrating a pitfall: 
\begin{isabelle}
\isacommand{lemma}\ "(A\ <=\ -B)\ =\ (B\ <=\ -A)"
\end{isabelle}
%
The proof fails.  It is not a statement about sets, due to overloading;
the relation symbol~\isa{<=} can be any relation, not just  
subset. 
In this general form, the statement is not valid.  Putting
in a type constraint forces the variables to denote sets, allowing the
proof to succeed:

\begin{isabelle}
\isacommand{lemma}\ "((A::\ {\isacharprime}a\ set)\ <=\ -B)\ =\ (B\ <=\
-A)"
\end{isabelle}
Section~\ref{sec:axclass} below describes overloading.  Incidentally,
\isa{A~\isasymsubseteq~-B} asserts that the sets \isa{A} and \isa{B} are
disjoint.

\medskip
Two sets are \textbf{equal}\indexbold{equality!of sets} if they contain the
same elements.   This is
the principle of \textbf{extensionality}\indexbold{extensionality!for sets}
for sets. 
\begin{isabelle}
({\isasymAnd}x.\ (x\ {\isasymin}\ A)\ =\ (x\ {\isasymin}\ B))\
{\isasymLongrightarrow}\ A\ =\ B
\rulenamedx{set_ext}
\end{isabelle}
Extensionality can be expressed as 
$A=B\iff (A\subseteq B)\conj (B\subseteq A)$.  
The following rules express both
directions of this equivalence.  Proving a set equation using
\isa{equalityI} allows the two inclusions to be proved independently.
\begin{isabelle}
\isasymlbrakk A\ \isasymsubseteq\ B;\ B\ \isasymsubseteq\
A\isasymrbrakk\ \isasymLongrightarrow\ A\ =\ B
\rulenamedx{equalityI}
\par\smallskip%          \isanewline didn't leave enough space
\isasymlbrakk A\ =\ B;\ \isasymlbrakk A\ \isasymsubseteq\ B;\ B\
\isasymsubseteq\ A\isasymrbrakk\ \isasymLongrightarrow\ P\isasymrbrakk\
\isasymLongrightarrow\ P%
\rulenamedx{equalityE}
\end{isabelle}


\subsection{Finite Set Notation} 

\indexbold{sets!notation for finite}
Finite sets are expressed using the constant \cdx{insert}, which is
a form of union:
\begin{isabelle}
insert\ a\ A\ =\ \isacharbraceleft a\isacharbraceright\ \isasymunion\ A
\rulename{insert_is_Un}
\end{isabelle}
%
The finite set expression \isa{\isacharbraceleft
a,b\isacharbraceright} abbreviates
\isa{insert\ a\ (insert\ b\ \isacharbraceleft\isacharbraceright)}.
Many facts about finite sets can be proved automatically: 
\begin{isabelle}
\isacommand{lemma}\
"\isacharbraceleft a,b\isacharbraceright\
\isasymunion\ \isacharbraceleft c,d\isacharbraceright\ =\
\isacharbraceleft a,b,c,d\isacharbraceright"\isanewline
\isacommand{by}\ blast
\end{isabelle}


Not everything that we would like to prove is valid. 
Consider this attempt: 
\begin{isabelle}
\isacommand{lemma}\ "\isacharbraceleft a,b\isacharbraceright\ \isasyminter\ \isacharbraceleft b,c\isacharbraceright\ =\
\isacharbraceleft b\isacharbraceright"\isanewline
\isacommand{apply}\ auto
\end{isabelle}
%
The proof fails, leaving the subgoal \isa{b=c}. To see why it 
fails, consider a correct version: 
\begin{isabelle}
\isacommand{lemma}\ "\isacharbraceleft a,b\isacharbraceright\ \isasyminter\ 
\isacharbraceleft b,c\isacharbraceright\ =\ (if\ a=c\ then\
\isacharbraceleft a,b\isacharbraceright\ else\ \isacharbraceleft
b\isacharbraceright)"\isanewline
\isacommand{apply}\ simp\isanewline
\isacommand{by}\ blast
\end{isabelle}

Our mistake was to suppose that the various items were distinct.  Another
remark: this proof uses two methods, namely {\isa{simp}}  and
{\isa{blast}}. Calling {\isa{simp}} eliminates the
\isa{if}-\isa{then}-\isa{else} expression,  which {\isa{blast}}
cannot break down. The combined methods (namely {\isa{force}}  and
\isa{auto}) can prove this fact in one step. 


\subsection{Set Comprehension}

\index{set comprehensions|(}%
The set comprehension \isa{\isacharbraceleft x.\
P\isacharbraceright} expresses the set of all elements that satisfy the
predicate~\isa{P}.  Two laws describe the relationship between set 
comprehension and the membership relation:
\begin{isabelle}
(a\ \isasymin\
\isacharbraceleft x.\ P\ x\isacharbraceright)\ =\ P\ a
\rulename{mem_Collect_eq}\isanewline
\isacharbraceleft x.\ x\ \isasymin\ A\isacharbraceright\ =\ A
\rulename{Collect_mem_eq}
\end{isabelle}

\smallskip
Facts such as these have trivial proofs:
\begin{isabelle}
\isacommand{lemma}\ "\isacharbraceleft x.\ P\ x\ \isasymor\
x\ \isasymin\ A\isacharbraceright\ =\
\isacharbraceleft x.\ P\ x\isacharbraceright\ \isasymunion\ A"
\par\smallskip
\isacommand{lemma}\ "\isacharbraceleft x.\ P\ x\
\isasymlongrightarrow\ Q\ x\isacharbraceright\ =\
-\isacharbraceleft x.\ P\ x\isacharbraceright\
\isasymunion\ \isacharbraceleft x.\ Q\ x\isacharbraceright"
\end{isabelle}

\smallskip
Isabelle has a general syntax for comprehension, which is best 
described through an example: 
\begin{isabelle}
\isacommand{lemma}\ "\isacharbraceleft p*q\ \isacharbar\ p\ q.\ 
p{\isasymin}prime\ \isasymand\ q{\isasymin}prime\isacharbraceright\ =\ 
\isanewline
\ \ \ \ \ \ \ \ \isacharbraceleft z.\ \isasymexists p\ q.\ z\ =\ p*q\
\isasymand\ p{\isasymin}prime\ \isasymand\
q{\isasymin}prime\isacharbraceright"
\end{isabelle}
The left and right hand sides
of this equation are identical. The syntax used in the 
left-hand side abbreviates the right-hand side: in this case, all numbers
that are the product of two primes.  The syntax provides a neat
way of expressing any set given by an expression built up from variables
under specific constraints.  The drawback is that it hides the true form of
the expression, with its existential quantifiers. 

\smallskip
\emph{Remark}.  We do not need sets at all.  They are essentially equivalent
to predicate variables, which are allowed in  higher-order logic.  The main
benefit of sets is their notation;  we can write \isa{x{\isasymin}A}
and
\isa{\isacharbraceleft z.\ P\isacharbraceright} where predicates would
require writing
\isa{A(x)} and
\isa{{\isasymlambda}z.\ P}. 
\index{set comprehensions|)}


\subsection{Binding Operators}

\index{quantifiers!for sets|(}%
Universal and existential quantifications may range over sets, 
with the obvious meaning.  Here are the natural deduction rules for the
bounded universal quantifier.  Occasionally you will need to apply
\isa{bspec} with an explicit instantiation of the variable~\isa{x}:
%
\begin{isabelle}
({\isasymAnd}x.\ x\ \isasymin\ A\ \isasymLongrightarrow\ P\ x)\ \isasymLongrightarrow\ {\isasymforall}x\isasymin
A.\ P\ x%
\rulenamedx{ballI}%
\isanewline
\isasymlbrakk{\isasymforall}x\isasymin A.\
P\ x;\ x\ \isasymin\
A\isasymrbrakk\ \isasymLongrightarrow\ P\
x%
\rulenamedx{bspec}
\end{isabelle}
%
Dually, here are the natural deduction rules for the
bounded existential quantifier.  You may need to apply
\isa{bexI} with an explicit instantiation:
\begin{isabelle}
\isasymlbrakk P\ x;\
x\ \isasymin\ A\isasymrbrakk\
\isasymLongrightarrow\
\isasymexists x\isasymin A.\ P\
x%
\rulenamedx{bexI}%
\isanewline
\isasymlbrakk\isasymexists x\isasymin A.\
P\ x;\ {\isasymAnd}x.\
{\isasymlbrakk}x\ \isasymin\ A;\
P\ x\isasymrbrakk\ \isasymLongrightarrow\
Q\isasymrbrakk\ \isasymLongrightarrow\ Q%
\rulenamedx{bexE}
\end{isabelle}
\index{quantifiers!for sets|)}

\index{union!indexed}%
Unions can be formed over the values of a given  set.  The syntax is
\mbox{\isa{\isasymUnion x\isasymin A.\ B}} or 
\isa{UN x:A.\ B} in \textsc{ascii}. Indexed union satisfies this basic law:
\begin{isabelle}
(b\ \isasymin\
(\isasymUnion x\isasymin A. B\ x)) =\ (\isasymexists x\isasymin A.\
b\ \isasymin\ B\ x)
\rulenamedx{UN_iff}
\end{isabelle}
It has two natural deduction rules similar to those for the existential
quantifier.  Sometimes \isa{UN_I} must be applied explicitly:
\begin{isabelle}
\isasymlbrakk a\ \isasymin\
A;\ b\ \isasymin\
B\ a\isasymrbrakk\ \isasymLongrightarrow\
b\ \isasymin\
(\isasymUnion x\isasymin A. B\ x)
\rulenamedx{UN_I}%
\isanewline
\isasymlbrakk b\ \isasymin\
(\isasymUnion x\isasymin A. B\ x);\
{\isasymAnd}x.\ {\isasymlbrakk}x\ \isasymin\
A;\ b\ \isasymin\
B\ x\isasymrbrakk\ \isasymLongrightarrow\
R\isasymrbrakk\ \isasymLongrightarrow\ R%
\rulenamedx{UN_E}
\end{isabelle}
%
The following built-in abbreviation (see {\S}\ref{sec:abbreviations})
lets us express the union over a \emph{type}:
\begin{isabelle}
\ \ \ \ \
({\isasymUnion}x.\ B\ x)\ {\isasymequiv}\
({\isasymUnion}x{\isasymin}UNIV. B\ x)
\end{isabelle}

We may also express the union of a set of sets, written \isa{Union\ C} in
\textsc{ascii}: 
\begin{isabelle}
(A\ \isasymin\ \isasymUnion C)\ =\ (\isasymexists X\isasymin C.\ A\
\isasymin\ X)
\rulenamedx{Union_iff}
\end{isabelle}

\index{intersection!indexed}%
Intersections are treated dually, although they seem to be used less often
than unions.  The syntax below would be \isa{INT
x:\ A.\ B} and \isa{Inter\ C} in \textsc{ascii}.  Among others, these
theorems are available:
\begin{isabelle}
(b\ \isasymin\
(\isasymInter x\isasymin A. B\ x))\
=\
({\isasymforall}x\isasymin A.\
b\ \isasymin\ B\ x)
\rulenamedx{INT_iff}%
\isanewline
(A\ \isasymin\
\isasymInter C)\ =\
({\isasymforall}X\isasymin C.\
A\ \isasymin\ X)
\rulenamedx{Inter_iff}
\end{isabelle}

Isabelle uses logical equivalences such as those above in automatic proof. 
Unions, intersections and so forth are not simply replaced by their
definitions.  Instead, membership tests are simplified.  For example,  $x\in
A\cup B$ is replaced by $x\in A\lor x\in B$.

The internal form of a comprehension involves the constant  
\cdx{Collect},
which occasionally appears when a goal or theorem
is displayed.  For example, \isa{Collect\ P}  is the same term as
\isa{\isacharbraceleft x.\ P\ x\isacharbraceright}.  The same thing can
happen with quantifiers:   \hbox{\isa{All\ P}}\index{*All (constant)}
is 
\isa{{\isasymforall}x.\ P\ x} and 
\hbox{\isa{Ex\ P}}\index{*Ex (constant)} is \isa{\isasymexists x.\ P\ x}; 
also \isa{Ball\ A\ P}\index{*Ball (constant)} is 
\isa{{\isasymforall}x\isasymin A.\ P\ x} and 
\isa{Bex\ A\ P}\index{*Bex (constant)} is 
\isa{\isasymexists x\isasymin A.\ P\ x}.  For indexed unions and
intersections, you may see the constants 
\cdx{UNION} and  \cdx{INTER}\@.
The internal constant for $\varepsilon x.P(x)$ is~\cdx{Eps}.

We have only scratched the surface of Isabelle/HOL's set theory, which provides
hundreds of theorems for your use.


\subsection{Finiteness and Cardinality}

\index{sets!finite}%
The predicate \sdx{finite} holds of all finite sets.  Isabelle/HOL
includes many familiar theorems about finiteness and cardinality 
(\cdx{card}). For example, we have theorems concerning the
cardinalities of unions, intersections and the
powerset:\index{cardinality}
%
\begin{isabelle}
{\isasymlbrakk}finite\ A;\ finite\ B\isasymrbrakk\isanewline
\isasymLongrightarrow\ card\ A\ \isacharplus\ card\ B\ =\ card\ (A\ \isasymunion\ B)\ \isacharplus\ card\ (A\ \isasyminter\ B)
\rulenamedx{card_Un_Int}%
\isanewline
\isanewline
finite\ A\ \isasymLongrightarrow\ card\
(Pow\ A)\  =\ 2\ \isacharcircum\ card\ A%
\rulenamedx{card_Pow}%
\isanewline
\isanewline
finite\ A\ \isasymLongrightarrow\isanewline
card\ \isacharbraceleft B.\ B\ \isasymsubseteq\
A\ \isasymand\ card\ B\ =\
k\isacharbraceright\ =\ card\ A\ choose\ k%
\rulename{n_subsets}
\end{isabelle}
Writing $|A|$ as $n$, the last of these theorems says that the number of
$k$-element subsets of~$A$ is \index{binomial coefficients}
$\binom{n}{k}$.

%\begin{warn}
%The term \isa{finite\ A} is defined via a syntax translation as an
%abbreviation for \isa{A {\isasymin} Finites}, where the constant
%\cdx{Finites} denotes the set of all finite sets of a given type.  There
%is no constant \isa{finite}.
%\end{warn}
\index{sets|)}


\section{Functions}

\index{functions|(}%
This section describes a few concepts that involve
functions.  Some of the more important theorems are given along with the 
names. A few sample proofs appear. Unlike with set theory, however, 
we cannot simply state lemmas and expect them to be proved using
\isa{blast}. 

\subsection{Function Basics}

Two functions are \textbf{equal}\indexbold{equality!of functions}
if they yield equal results given equal
arguments.  This is the principle of
\textbf{extensionality}\indexbold{extensionality!for functions} for
functions:
\begin{isabelle}
({\isasymAnd}x.\ f\ x\ =\ g\ x)\ {\isasymLongrightarrow}\ f\ =\ g
\rulenamedx{ext}
\end{isabelle}

\indexbold{updating a function}%
Function \textbf{update} is useful for modelling machine states. It has 
the obvious definition and many useful facts are proved about 
it.  In particular, the following equation is installed as a simplification
rule:
\begin{isabelle}
(f(x:=y))\ z\ =\ (if\ z\ =\ x\ then\ y\ else\ f\ z)
\rulename{fun_upd_apply}
\end{isabelle}
Two syntactic points must be noted.  In
\isa{(f(x:=y))\ z} we are applying an updated function to an
argument; the outer parentheses are essential.  A series of two or more
updates can be abbreviated as shown on the left-hand side of this theorem:
\begin{isabelle}
f(x:=y,\ x:=z)\ =\ f(x:=z)
\rulename{fun_upd_upd}
\end{isabelle}
Note also that we can write \isa{f(x:=z)} with only one pair of parentheses
when it is not being applied to an argument.

\medskip
The \bfindex{identity function} and function 
\textbf{composition}\indexbold{composition!of functions} are
defined:
\begin{isabelle}%
id\ \isasymequiv\ {\isasymlambda}x.\ x%
\rulenamedx{id_def}\isanewline
f\ \isasymcirc\ g\ \isasymequiv\
{\isasymlambda}x.\ f\
(g\ x)%
\rulenamedx{o_def}
\end{isabelle}
%
Many familiar theorems concerning the identity and composition 
are proved. For example, we have the associativity of composition: 
\begin{isabelle}
f\ \isasymcirc\ (g\ \isasymcirc\ h)\ =\ f\ \isasymcirc\ g\ \isasymcirc\ h
\rulename{o_assoc}
\end{isabelle}

\subsection{Injections, Surjections, Bijections}

\index{injections}\index{surjections}\index{bijections}%
A function may be \textbf{injective}, \textbf{surjective} or \textbf{bijective}: 
\begin{isabelle}
inj_on\ f\ A\ \isasymequiv\ {\isasymforall}x\isasymin A.\
{\isasymforall}y\isasymin  A.\ f\ x\ =\ f\ y\ \isasymlongrightarrow\ x\
=\ y%
\rulenamedx{inj_on_def}\isanewline
surj\ f\ \isasymequiv\ {\isasymforall}y.\
\isasymexists x.\ y\ =\ f\ x%
\rulenamedx{surj_def}\isanewline
bij\ f\ \isasymequiv\ inj\ f\ \isasymand\ surj\ f
\rulenamedx{bij_def}
\end{isabelle}
The second argument
of \isa{inj_on} lets us express that a function is injective over a
given set. This refinement is useful in higher-order logic, where
functions are total; in some cases, a function's natural domain is a subset
of its domain type.  Writing \isa{inj\ f} abbreviates \isa{inj_on\ f\
UNIV}, for when \isa{f} is injective everywhere.

The operator \isa{inv} expresses the 
\textbf{inverse}\indexbold{inverse!of a function}
of a function. In 
general the inverse may not be well behaved.  We have the usual laws,
such as these: 
\begin{isabelle}
inj\ f\ \ \isasymLongrightarrow\ inv\ f\ (f\ x)\ =\ x%
\rulename{inv_f_f}\isanewline
surj\ f\ \isasymLongrightarrow\ f\ (inv\ f\ y)\ =\ y
\rulename{surj_f_inv_f}\isanewline
bij\ f\ \ \isasymLongrightarrow\ inv\ (inv\ f)\ =\ f
\rulename{inv_inv_eq}
\end{isabelle}
%
%Other useful facts are that the inverse of an injection 
%is a surjection and vice versa; the inverse of a bijection is 
%a bijection. 
%\begin{isabelle}
%inj\ f\ \isasymLongrightarrow\ surj\
%(inv\ f)
%\rulename{inj_imp_surj_inv}\isanewline
%surj\ f\ \isasymLongrightarrow\ inj\ (inv\ f)
%\rulename{surj_imp_inj_inv}\isanewline
%bij\ f\ \isasymLongrightarrow\ bij\ (inv\ f)
%\rulename{bij_imp_bij_inv}
%\end{isabelle}
%
%The converses of these results fail.  Unless a function is 
%well behaved, little can be said about its inverse. Here is another 
%law: 
%\begin{isabelle}
%{\isasymlbrakk}bij\ f;\ bij\ g\isasymrbrakk\ \isasymLongrightarrow\ inv\ (f\ \isasymcirc\ g)\ =\ inv\ g\ \isasymcirc\ inv\ f%
%\rulename{o_inv_distrib}
%\end{isabelle}

Theorems involving these concepts can be hard to prove. The following 
example is easy, but it cannot be proved automatically. To begin 
with, we need a law that relates the equality of functions to 
equality over all arguments: 
\begin{isabelle}
(f\ =\ g)\ =\ ({\isasymforall}x.\ f\ x\ =\ g\ x)
\rulename{fun_eq_iff}
\end{isabelle}
%
This is just a restatement of
extensionality.\indexbold{extensionality!for functions}
Our lemma
states  that an injection can be cancelled from the left  side of
function composition: 
\begin{isabelle}
\isacommand{lemma}\ "inj\ f\ \isasymLongrightarrow\ (f\ o\ g\ =\ f\ o\ h)\ =\ (g\ =\ h)"\isanewline
\isacommand{apply}\ (simp\ add:\ fun_eq_iff\ inj_on_def)\isanewline
\isacommand{apply}\ auto\isanewline
\isacommand{done}
\end{isabelle}

The first step of the proof invokes extensionality and the definitions 
of injectiveness and composition. It leaves one subgoal:
\begin{isabelle}
\ 1.\ {\isasymforall}x\ y.\ f\ x\ =\ f\ y\ \isasymlongrightarrow\ x\ =\ y\
\isasymLongrightarrow\isanewline
\ \ \ \ ({\isasymforall}x.\ f\ (g\ x)\ =\ f\ (h\ x))\ =\ ({\isasymforall}x.\ g\ x\ =\ h\ x)
\end{isabelle}
This can be proved using the \isa{auto} method. 


\subsection{Function Image}

The \textbf{image}\indexbold{image!under a function}
of a set under a function is a most useful notion.  It
has the obvious definition: 
\begin{isabelle}
f\ `\ A\ \isasymequiv\ \isacharbraceleft y.\ \isasymexists x\isasymin
A.\ y\ =\ f\ x\isacharbraceright
\rulenamedx{image_def}
\end{isabelle}
%
Here are some of the many facts proved about image: 
\begin{isabelle}
(f\ \isasymcirc\ g)\ `\ r\ =\ f\ `\ g\ `\ r
\rulename{image_compose}\isanewline
f`(A\ \isasymunion\ B)\ =\ f`A\ \isasymunion\ f`B
\rulename{image_Un}\isanewline
inj\ f\ \isasymLongrightarrow\ f`(A\ \isasyminter\
B)\ =\ f`A\ \isasyminter\ f`B
\rulename{image_Int}
%\isanewline
%bij\ f\ \isasymLongrightarrow\ f\ `\ (-\ A)\ =\ -\ f\ `\ A%
%\rulename{bij_image_Compl_eq}
\end{isabelle}


Laws involving image can often be proved automatically. Here 
are two examples, illustrating connections with indexed union and with the
general syntax for comprehension:
\begin{isabelle}
\isacommand{lemma}\ "f`A\ \isasymunion\ g`A\ =\ ({\isasymUnion}x{\isasymin}A.\ \isacharbraceleft f\ x,\ g\
x\isacharbraceright)"
\par\smallskip
\isacommand{lemma}\ "f\ `\ \isacharbraceleft(x,y){.}\ P\ x\ y\isacharbraceright\ =\ \isacharbraceleft f(x,y)\ \isacharbar\ x\ y.\ P\ x\
y\isacharbraceright"
\end{isabelle}

\medskip
\index{range!of a function}%
A function's \textbf{range} is the set of values that the function can 
take on. It is, in fact, the image of the universal set under 
that function. There is no constant \isa{range}.  Instead,
\sdx{range}  abbreviates an application of image to \isa{UNIV}: 
\begin{isabelle}
\ \ \ \ \ range\ f\
{\isasymrightleftharpoons}\ f`UNIV
\end{isabelle}
%
Few theorems are proved specifically 
for {\isa{range}}; in most cases, you should look for a more general
theorem concerning images.

\medskip
\textbf{Inverse image}\index{inverse image!of a function} is also  useful.
It is defined as follows: 
\begin{isabelle}
f\ -`\ B\ \isasymequiv\ \isacharbraceleft x.\ f\ x\ \isasymin\ B\isacharbraceright
\rulenamedx{vimage_def}
\end{isabelle}
%
This is one of the facts proved about it:
\begin{isabelle}
f\ -`\ (-\ A)\ =\ -\ f\ -`\ A%
\rulename{vimage_Compl}
\end{isabelle}
\index{functions|)}


\section{Relations}
\label{sec:Relations}

\index{relations|(}%
A \textbf{relation} is a set of pairs.  As such, the set operations apply
to them.  For instance, we may form the union of two relations.  Other
primitives are defined specifically for relations. 

\subsection{Relation Basics}

The \bfindex{identity relation}, also known as equality, has the obvious 
definition: 
\begin{isabelle}
Id\ \isasymequiv\ \isacharbraceleft p.\ \isasymexists x.\ p\ =\ (x,x){\isacharbraceright}%
\rulenamedx{Id_def}
\end{isabelle}

\indexbold{composition!of relations}%
\textbf{Composition} of relations (the infix \sdx{O}) is also
available: 
\begin{isabelle}
r\ O\ s\ = \isacharbraceleft(x,z).\ \isasymexists y.\ (x,y)\ \isasymin\ s\ \isasymand\ (y,z)\ \isasymin\ r\isacharbraceright
\rulenamedx{relcomp_unfold}
\end{isabelle}
%
This is one of the many lemmas proved about these concepts: 
\begin{isabelle}
R\ O\ Id\ =\ R
\rulename{R_O_Id}
\end{isabelle}
%
Composition is monotonic, as are most of the primitives appearing 
in this chapter.  We have many theorems similar to the following 
one: 
\begin{isabelle}
\isasymlbrakk r\isacharprime\ \isasymsubseteq\ r;\ s\isacharprime\
\isasymsubseteq\ s\isasymrbrakk\ \isasymLongrightarrow\ r\isacharprime\ O\
s\isacharprime\ \isasymsubseteq\ r\ O\ s%
\rulename{relcomp_mono}
\end{isabelle}

\indexbold{converse!of a relation}%
\indexbold{inverse!of a relation}%
The \textbf{converse} or inverse of a
relation exchanges the roles  of the two operands.  We use the postfix
notation~\isa{r\isasyminverse} or
\isa{r\isacharcircum-1} in ASCII\@.
\begin{isabelle}
((a,b)\ \isasymin\ r\isasyminverse)\ =\
((b,a)\ \isasymin\ r)
\rulenamedx{converse_iff}
\end{isabelle}
%
Here is a typical law proved about converse and composition: 
\begin{isabelle}
(r\ O\ s)\isasyminverse\ =\ s\isasyminverse\ O\ r\isasyminverse
\rulename{converse_relcomp}
\end{isabelle}

\indexbold{image!under a relation}%
The \textbf{image} of a set under a relation is defined
analogously  to image under a function: 
\begin{isabelle}
(b\ \isasymin\ r\ ``\ A)\ =\ (\isasymexists x\isasymin
A.\ (x,b)\ \isasymin\ r)
\rulenamedx{Image_iff}
\end{isabelle}
It satisfies many similar laws.

\index{domain!of a relation}%
\index{range!of a relation}%
The \textbf{domain} and \textbf{range} of a relation are defined in the 
standard way: 
\begin{isabelle}
(a\ \isasymin\ Domain\ r)\ =\ (\isasymexists y.\ (a,y)\ \isasymin\
r)
\rulenamedx{Domain_iff}%
\isanewline
(a\ \isasymin\ Range\ r)\
\ =\ (\isasymexists y.\
(y,a)\
\isasymin\ r)
\rulenamedx{Range_iff}
\end{isabelle}

Iterated composition of a relation is available.  The notation overloads 
that of exponentiation.  Two simplification rules are installed: 
\begin{isabelle}
R\ \isacharcircum\ \isadigit{0}\ =\ Id\isanewline
R\ \isacharcircum\ Suc\ n\ =\ R\ O\ R\isacharcircum n
\end{isabelle}

\subsection{The Reflexive and Transitive Closure}

\index{reflexive and transitive closure|(}%
The \textbf{reflexive and transitive closure} of the
relation~\isa{r} is written with a
postfix syntax.  In ASCII we write \isa{r\isacharcircum*} and in
symbol notation~\isa{r\isactrlsup *}.  It is the least solution of the
equation
\begin{isabelle}
r\isactrlsup *\ =\ Id\ \isasymunion \ (r\ O\ r\isactrlsup *)
\rulename{rtrancl_unfold}
\end{isabelle}
%
Among its basic properties are three that serve as introduction 
rules:
\begin{isabelle}
(a,\ a)\ \isasymin \ r\isactrlsup *
\rulenamedx{rtrancl_refl}\isanewline
p\ \isasymin \ r\ \isasymLongrightarrow \ p\ \isasymin \ r\isactrlsup *
\rulenamedx{r_into_rtrancl}\isanewline
\isasymlbrakk (a,b)\ \isasymin \ r\isactrlsup *;\ 
(b,c)\ \isasymin \ r\isactrlsup *\isasymrbrakk \ \isasymLongrightarrow \
(a,c)\ \isasymin \ r\isactrlsup *
\rulenamedx{rtrancl_trans}
\end{isabelle}
%
Induction over the reflexive transitive closure is available: 
\begin{isabelle}
\isasymlbrakk (a,\ b)\ \isasymin \ r\isactrlsup *;\ P\ a;\ \isasymAnd y\ z.\ \isasymlbrakk (a,\ y)\ \isasymin \ r\isactrlsup *;\ (y,\ z)\ \isasymin \ r;\ P\ y\isasymrbrakk \ \isasymLongrightarrow \ P\ z\isasymrbrakk \isanewline
\isasymLongrightarrow \ P\ b%
\rulename{rtrancl_induct}
\end{isabelle}
%
Idempotence is one of the laws proved about the reflexive transitive 
closure: 
\begin{isabelle}
(r\isactrlsup *)\isactrlsup *\ =\ r\isactrlsup *
\rulename{rtrancl_idemp}
\end{isabelle}

\smallskip
The transitive closure is similar.  The ASCII syntax is
\isa{r\isacharcircum+}.  It has two  introduction rules: 
\begin{isabelle}
p\ \isasymin \ r\ \isasymLongrightarrow \ p\ \isasymin \ r\isactrlsup +
\rulenamedx{r_into_trancl}\isanewline
\isasymlbrakk (a,\ b)\ \isasymin \ r\isactrlsup +;\ (b,\ c)\ \isasymin \ r\isactrlsup +\isasymrbrakk \ \isasymLongrightarrow \ (a,\ c)\ \isasymin \ r\isactrlsup +
\rulenamedx{trancl_trans}
\end{isabelle}
%
The induction rule resembles the one shown above. 
A typical lemma states that transitive closure commutes with the converse
operator: 
\begin{isabelle}
(r\isasyminverse )\isactrlsup +\ =\ (r\isactrlsup +)\isasyminverse 
\rulename{trancl_converse}
\end{isabelle}

\subsection{A Sample Proof}

The reflexive transitive closure also commutes with the converse
operator.  Let us examine the proof. Each direction of the equivalence
is  proved separately. The two proofs are almost identical. Here 
is the first one: 
\begin{isabelle}
\isacommand{lemma}\ rtrancl_converseD:\ "(x,y)\ \isasymin \
(r\isasyminverse)\isactrlsup *\ \isasymLongrightarrow \ (y,x)\ \isasymin
\ r\isactrlsup *"\isanewline
\isacommand{apply}\ (erule\ rtrancl_induct)\isanewline
\ \isacommand{apply}\ (rule\ rtrancl_refl)\isanewline
\isacommand{apply}\ (blast\ intro:\ rtrancl_trans)\isanewline
\isacommand{done}
\end{isabelle}
%
The first step of the proof applies induction, leaving these subgoals: 
\begin{isabelle}
\ 1.\ (x,\ x)\ \isasymin \ r\isactrlsup *\isanewline
\ 2.\ \isasymAnd y\ z.\ \isasymlbrakk (x,y)\ \isasymin \
(r\isasyminverse)\isactrlsup *;\ (y,z)\ \isasymin \ r\isasyminverse ;\
(y,x)\ \isasymin \ r\isactrlsup *\isasymrbrakk \isanewline
\ \ \ \ \ \ \ \ \ \ \isasymLongrightarrow \ (z,x)\ \isasymin \ r\isactrlsup *
\end{isabelle}
%
The first subgoal is trivial by reflexivity. The second follows 
by first eliminating the converse operator, yielding the
assumption \isa{(z,y)\
\isasymin\ r}, and then
applying the introduction rules shown above.  The same proof script handles
the other direction: 
\begin{isabelle}
\isacommand{lemma}\ rtrancl_converseI:\ "(y,x)\ \isasymin \ r\isactrlsup *\ \isasymLongrightarrow \ (x,y)\ \isasymin \ (r\isasyminverse)\isactrlsup *"\isanewline
\isacommand{apply}\ (erule\ rtrancl_induct)\isanewline
\ \isacommand{apply}\ (rule\ rtrancl_refl)\isanewline
\isacommand{apply}\ (blast\ intro:\ rtrancl_trans)\isanewline
\isacommand{done}
\end{isabelle}


Finally, we combine the two lemmas to prove the desired equation: 
\begin{isabelle}
\isacommand{lemma}\ rtrancl_converse:\ "(r\isasyminverse)\isactrlsup *\ =\ (r\isactrlsup *)\isasyminverse"\isanewline
\isacommand{by}\ (auto\ intro:\ rtrancl_converseI\ dest:\
rtrancl_converseD)
\end{isabelle}

\begin{warn}
This trivial proof requires \isa{auto} rather than \isa{blast} because
of a subtle issue involving ordered pairs.  Here is a subgoal that
arises internally after  the rules
\isa{equalityI} and \isa{subsetI} have been applied:
\begin{isabelle}
\ 1.\ \isasymAnd x.\ x\ \isasymin \ (r\isasyminverse )\isactrlsup *\ \isasymLongrightarrow \ x\ \isasymin \ (r\isactrlsup
*)\isasyminverse
%ignore subgoal 2
%\ 2.\ \isasymAnd x.\ x\ \isasymin \ (r\isactrlsup *)\isasyminverse \
%\isasymLongrightarrow \ x\ \isasymin \ (r\isasyminverse )\isactrlsup *
\end{isabelle}
\par\noindent
We cannot apply \isa{rtrancl_converseD}\@.  It refers to
ordered pairs, while \isa{x} is a variable of product type.
The \isa{simp} and \isa{blast} methods can do nothing, so let us try
\isa{clarify}:
\begin{isabelle}
\ 1.\ \isasymAnd a\ b.\ (a,b)\ \isasymin \ (r\isasyminverse )\isactrlsup *\ \isasymLongrightarrow \ (b,a)\ \isasymin \ r\isactrlsup
*
\end{isabelle}
Now that \isa{x} has been replaced by the pair \isa{(a,b)}, we can
proceed.  Other methods that split variables in this way are \isa{force},
\isa{auto}, \isa{fast} and \isa{best}.  Section~\ref{sec:products} will discuss proof
techniques for ordered pairs in more detail.
\end{warn}
\index{relations|)}\index{reflexive and transitive closure|)}


\section{Well-Founded Relations and Induction}
\label{sec:Well-founded}

\index{relations!well-founded|(}%
A well-founded relation captures the notion of a terminating
process. Complex recursive functions definitions must specify a
well-founded relation that justifies their
termination~\cite{isabelle-function}. Most of the forms of induction found
in mathematics are merely special cases of induction over a
well-founded relation.

Intuitively, the relation~$\prec$ is \textbf{well-founded} if it admits no
infinite  descending chains
\[ \cdots \prec a@2 \prec a@1 \prec a@0. \]
Well-foundedness can be hard to show. The various 
formulations are all complicated.  However,  often a relation
is well-founded by construction.  HOL provides
theorems concerning ways of constructing  a well-founded relation.  The
most familiar way is to specify a 
\index{measure functions}\textbf{measure function}~\isa{f} into
the natural numbers, when $\isa{x}\prec \isa{y}\iff \isa{f x} < \isa{f y}$;
we write this particular relation as
\isa{measure~f}.

\begin{warn}
You may want to skip the rest of this section until you need to perform a
complex recursive function definition or induction.  The induction rule
returned by
\isacommand{fun} is good enough for most purposes.  We use an explicit
well-founded induction only in {\S}\ref{sec:CTL-revisited}.
\end{warn}

Isabelle/HOL declares \cdx{less_than} as a relation object, 
that is, a set of pairs of natural numbers. Two theorems tell us that this
relation  behaves as expected and that it is well-founded: 
\begin{isabelle}
((x,y)\ \isasymin\ less_than)\ =\ (x\ <\ y)
\rulenamedx{less_than_iff}\isanewline
wf\ less_than
\rulenamedx{wf_less_than}
\end{isabelle}

The notion of measure generalizes to the 
\index{inverse image!of a relation}\textbf{inverse image} of
a relation. Given a relation~\isa{r} and a function~\isa{f}, we express  a
new relation using \isa{f} as a measure.  An infinite descending chain on
this new relation would give rise to an infinite descending chain
on~\isa{r}.  Isabelle/HOL defines this concept and proves a
theorem stating that it preserves well-foundedness: 
\begin{isabelle}
inv_image\ r\ f\ \isasymequiv\ \isacharbraceleft(x,y).\ (f\ x,\ f\ y)\
\isasymin\ r\isacharbraceright
\rulenamedx{inv_image_def}\isanewline
wf\ r\ \isasymLongrightarrow\ wf\ (inv_image\ r\ f)
\rulenamedx{wf_inv_image}
\end{isabelle}

A measure function involves the natural numbers.  The relation \isa{measure
size} justifies primitive recursion and structural induction over a
datatype.  Isabelle/HOL defines
\isa{measure} as shown: 
\begin{isabelle}
measure\ \isasymequiv\ inv_image\ less_than%
\rulenamedx{measure_def}\isanewline
wf\ (measure\ f)
\rulenamedx{wf_measure}
\end{isabelle}

Of the other constructions, the most important is the
\bfindex{lexicographic product} of two relations. It expresses the
standard dictionary  ordering over pairs.  We write \isa{ra\ <*lex*>\
rb}, where \isa{ra} and \isa{rb} are the two operands.  The
lexicographic product satisfies the usual  definition and it preserves
well-foundedness: 
\begin{isabelle}
ra\ <*lex*>\ rb\ \isasymequiv \isanewline
\ \ \isacharbraceleft ((a,b),(a',b')).\ (a,a')\ \isasymin \ ra\
\isasymor\isanewline
\ \ \ \ \ \ \ \ \ \ \ \ \ \ \ \ \ \ \ \ \,a=a'\ \isasymand \ (b,b')\
\isasymin \ rb\isacharbraceright 
\rulenamedx{lex_prod_def}%
\par\smallskip
\isasymlbrakk wf\ ra;\ wf\ rb\isasymrbrakk \ \isasymLongrightarrow \ wf\ (ra\ <*lex*>\ rb)
\rulenamedx{wf_lex_prod}
\end{isabelle}

%These constructions can be used in a
%\textbf{recdef} declaration ({\S}\ref{sec:recdef-simplification}) to define
%the well-founded relation used to prove termination.

The \bfindex{multiset ordering}, useful for hard termination proofs, is
available in the Library~\cite{HOL-Library}.
Baader and Nipkow \cite[{\S}2.5]{Baader-Nipkow} discuss it. 

\medskip
Induction\index{induction!well-founded|(}
comes in many forms,
including traditional mathematical  induction, structural induction on
lists and induction on size.  All are instances of the following rule,
for a suitable well-founded relation~$\prec$: 
\[ \infer{P(a)}{\infer*{P(x)}{[\forall y.\, y\prec x \imp P(y)]}} \]
To show $P(a)$ for a particular term~$a$, it suffices to show $P(x)$ for
arbitrary~$x$ under the assumption that $P(y)$ holds for $y\prec x$. 
Intuitively, the well-foundedness of $\prec$ ensures that the chains of
reasoning are finite.

\smallskip
In Isabelle, the induction rule is expressed like this:
\begin{isabelle}
{\isasymlbrakk}wf\ r;\ 
  {\isasymAnd}x.\ {\isasymforall}y.\ (y,x)\ \isasymin\ r\
\isasymlongrightarrow\ P\ y\ \isasymLongrightarrow\ P\ x\isasymrbrakk\
\isasymLongrightarrow\ P\ a
\rulenamedx{wf_induct}
\end{isabelle}
Here \isa{wf\ r} expresses that the relation~\isa{r} is well-founded.

Many familiar induction principles are instances of this rule. 
For example, the predecessor relation on the natural numbers 
is well-founded; induction over it is mathematical induction. 
The ``tail of'' relation on lists is well-founded; induction over 
it is structural induction.%
\index{induction!well-founded|)}%
\index{relations!well-founded|)}


\section{Fixed Point Operators}

\index{fixed points|(}%
Fixed point operators define sets
recursively.  They are invoked implicitly when making an inductive
definition, as discussed in Chap.\ts\ref{chap:inductive} below.  However,
they can be used directly, too. The
\emph{least}  or \emph{strongest} fixed point yields an inductive
definition;  the \emph{greatest} or \emph{weakest} fixed point yields a
coinductive  definition.  Mathematicians may wish to note that the
existence  of these fixed points is guaranteed by the Knaster-Tarski
theorem. 

\begin{warn}
Casual readers should skip the rest of this section.  We use fixed point
operators only in {\S}\ref{sec:VMC}.
\end{warn}

The theory applies only to monotonic functions.\index{monotone functions|bold} 
Isabelle's definition of monotone is overloaded over all orderings:
\begin{isabelle}
mono\ f\ \isasymequiv\ {\isasymforall}A\ B.\ A\ \isasymle\ B\ \isasymlongrightarrow\ f\ A\ \isasymle\ f\ B%
\rulenamedx{mono_def}
\end{isabelle}
%
For fixed point operators, the ordering will be the subset relation: if
$A\subseteq B$ then we expect $f(A)\subseteq f(B)$.  In addition to its
definition, monotonicity has the obvious introduction and destruction
rules:
\begin{isabelle}
({\isasymAnd}A\ B.\ A\ \isasymle\ B\ \isasymLongrightarrow\ f\ A\ \isasymle\ f\ B)\ \isasymLongrightarrow\ mono\ f%
\rulename{monoI}%
\par\smallskip%          \isanewline didn't leave enough space
{\isasymlbrakk}mono\ f;\ A\ \isasymle\ B\isasymrbrakk\
\isasymLongrightarrow\ f\ A\ \isasymle\ f\ B%
\rulename{monoD}
\end{isabelle}

The most important properties of the least fixed point are that 
it is a fixed point and that it enjoys an induction rule: 
\begin{isabelle}
mono\ f\ \isasymLongrightarrow\ lfp\ f\ =\ f\ (lfp\ f)
\rulename{lfp_unfold}%
\par\smallskip%          \isanewline didn't leave enough space
{\isasymlbrakk}a\ \isasymin\ lfp\ f;\ mono\ f;\isanewline
  \ {\isasymAnd}x.\ x\
\isasymin\ f\ (lfp\ f\ \isasyminter\ \isacharbraceleft x.\ P\
x\isacharbraceright)\ \isasymLongrightarrow\ P\ x\isasymrbrakk\
\isasymLongrightarrow\ P\ a%
\rulename{lfp_induct}
\end{isabelle}
%
The induction rule shown above is more convenient than the basic 
one derived from the minimality of {\isa{lfp}}.  Observe that both theorems
demand \isa{mono\ f} as a premise.

The greatest fixed point is similar, but it has a \bfindex{coinduction} rule: 
\begin{isabelle}
mono\ f\ \isasymLongrightarrow\ gfp\ f\ =\ f\ (gfp\ f)
\rulename{gfp_unfold}%
\isanewline
{\isasymlbrakk}mono\ f;\ a\ \isasymin\ X;\ X\ \isasymsubseteq\ f\ (X\
\isasymunion\ gfp\ f)\isasymrbrakk\ \isasymLongrightarrow\ a\ \isasymin\
gfp\ f%
\rulename{coinduct}
\end{isabelle}
A \textbf{bisimulation}\index{bisimulations} 
is perhaps the best-known concept defined as a
greatest fixed point.  Exhibiting a bisimulation to prove the equality of
two agents in a process algebra is an example of coinduction.
The coinduction rule can be strengthened in various ways.
\index{fixed points|)}

%The section "Case Study: Verified Model Checking" is part of this chapter
\index{model checking example|(}%
\index{lfp@{\texttt{lfp}}!applications of|see{CTL}}
%
\begin{isabellebody}%
\def\isabellecontext{Base}%
%
\isadelimtheory
%
\endisadelimtheory
%
\isatagtheory
%
\endisatagtheory
{\isafoldtheory}%
%
\isadelimtheory
%
\endisadelimtheory
%
\isamarkupsection{Case Study: Verified Model Checking%
}
\isamarkuptrue%
%
\begin{isamarkuptext}%
\label{sec:VMC}
This chapter ends with a case study concerning model checking for 
Computation Tree Logic (CTL), a temporal logic.
Model checking is a popular technique for the verification of finite
state systems (implementations) with respect to temporal logic formulae
(specifications) \cite{ClarkeGP-book,Huth-Ryan-book}. Its foundations are set theoretic
and this section will explore them in HOL\@. This is done in two steps.  First
we consider a simple modal logic called propositional dynamic
logic (PDL)\@.  We then proceed to the temporal logic CTL, which is
used in many real
model checkers. In each case we give both a traditional semantics (\isa{{\isaliteral{5C3C5475726E7374696C653E}{\isasymTurnstile}}}) and a
recursive function \isa{mc} that maps a formula into the set of all states of
the system where the formula is valid. If the system has a finite number of
states, \isa{mc} is directly executable: it is a model checker, albeit an
inefficient one. The main proof obligation is to show that the semantics
and the model checker agree.

\underscoreon

Our models are \emph{transition systems}:\index{transition systems}
sets of \emph{states} with
transitions between them.  Here is a simple example:
\begin{center}
\unitlength.5mm
\thicklines
\begin{picture}(100,60)
\put(50,50){\circle{20}}
\put(50,50){\makebox(0,0){$p,q$}}
\put(61,55){\makebox(0,0)[l]{$s_0$}}
\put(44,42){\vector(-1,-1){26}}
\put(16,18){\vector(1,1){26}}
\put(57,43){\vector(1,-1){26}}
\put(10,10){\circle{20}}
\put(10,10){\makebox(0,0){$q,r$}}
\put(-1,15){\makebox(0,0)[r]{$s_1$}}
\put(20,10){\vector(1,0){60}}
\put(90,10){\circle{20}}
\put(90,10){\makebox(0,0){$r$}}
\put(98, 5){\line(1,0){10}}
\put(108, 5){\line(0,1){10}}
\put(108,15){\vector(-1,0){10}}
\put(91,21){\makebox(0,0)[bl]{$s_2$}}
\end{picture}
\end{center}
Each state has a unique name or number ($s_0,s_1,s_2$), and in each state
certain \emph{atomic propositions} ($p,q,r$) hold.  The aim of temporal logic
is to formalize statements such as ``there is no path starting from $s_2$
leading to a state where $p$ or $q$ holds,'' which is true, and ``on all paths
starting from $s_0$, $q$ always holds,'' which is false.

Abstracting from this concrete example, we assume there is a type of
states:%
\end{isamarkuptext}%
\isamarkuptrue%
\isacommand{typedecl}\isamarkupfalse%
\ state%
\begin{isamarkuptext}%
\noindent
Command \commdx{typedecl} merely declares a new type but without
defining it (see \S\ref{sec:typedecl}). Thus we know nothing
about the type other than its existence. That is exactly what we need
because \isa{state} really is an implicit parameter of our model.  Of
course it would have been more generic to make \isa{state} a type
parameter of everything but declaring \isa{state} globally as above
reduces clutter.  Similarly we declare an arbitrary but fixed
transition system, i.e.\ a relation between states:%
\end{isamarkuptext}%
\isamarkuptrue%
\isacommand{consts}\isamarkupfalse%
\ M\ {\isaliteral{3A}{\isacharcolon}}{\isaliteral{3A}{\isacharcolon}}\ {\isaliteral{22}{\isachardoublequoteopen}}{\isaliteral{28}{\isacharparenleft}}state\ {\isaliteral{5C3C74696D65733E}{\isasymtimes}}\ state{\isaliteral{29}{\isacharparenright}}set{\isaliteral{22}{\isachardoublequoteclose}}%
\begin{isamarkuptext}%
\noindent
This is Isabelle's way of declaring a constant without defining it.
Finally we introduce a type of atomic propositions%
\end{isamarkuptext}%
\isamarkuptrue%
\isacommand{typedecl}\isamarkupfalse%
\ {\isaliteral{22}{\isachardoublequoteopen}}atom{\isaliteral{22}{\isachardoublequoteclose}}%
\begin{isamarkuptext}%
\noindent
and a \emph{labelling function}%
\end{isamarkuptext}%
\isamarkuptrue%
\isacommand{consts}\isamarkupfalse%
\ L\ {\isaliteral{3A}{\isacharcolon}}{\isaliteral{3A}{\isacharcolon}}\ {\isaliteral{22}{\isachardoublequoteopen}}state\ {\isaliteral{5C3C52696768746172726F773E}{\isasymRightarrow}}\ atom\ set{\isaliteral{22}{\isachardoublequoteclose}}%
\begin{isamarkuptext}%
\noindent
telling us which atomic propositions are true in each state.%
\end{isamarkuptext}%
\isamarkuptrue%
%
\isadelimtheory
%
\endisadelimtheory
%
\isatagtheory
%
\endisatagtheory
{\isafoldtheory}%
%
\isadelimtheory
%
\endisadelimtheory
\end{isabellebody}%
%%% Local Variables:
%%% mode: latex
%%% TeX-master: "root"
%%% End:

%
\begin{isabellebody}%
\def\isabellecontext{PDL}%
%
\isamarkupsubsection{Propositional dynamic logic---PDL%
}
%
\begin{isamarkuptext}%
\index{PDL|(}
The formulae of PDL are built up from atomic propositions via the customary
propositional connectives of negation and conjunction and the two temporal
connectives \isa{AX} and \isa{EF}. Since formulae are essentially
(syntax) trees, they are naturally modelled as a datatype:%
\end{isamarkuptext}%
\isacommand{datatype}\ formula\ {\isacharequal}\ Atom\ atom\isanewline
\ \ \ \ \ \ \ \ \ \ \ \ \ \ \ \ \ \ {\isacharbar}\ Neg\ formula\isanewline
\ \ \ \ \ \ \ \ \ \ \ \ \ \ \ \ \ \ {\isacharbar}\ And\ formula\ formula\isanewline
\ \ \ \ \ \ \ \ \ \ \ \ \ \ \ \ \ \ {\isacharbar}\ AX\ formula\isanewline
\ \ \ \ \ \ \ \ \ \ \ \ \ \ \ \ \ \ {\isacharbar}\ EF\ formula%
\begin{isamarkuptext}%
\noindent
This is almost the same as in the boolean expression case study in
\S\ref{sec:boolex}, except that what used to be called \isa{Var} is now
called \isa{Atom}.

The meaning of these formulae is given by saying which formula is true in
which state:%
\end{isamarkuptext}%
\isacommand{consts}\ valid\ {\isacharcolon}{\isacharcolon}\ {\isachardoublequote}state\ {\isasymRightarrow}\ formula\ {\isasymRightarrow}\ bool{\isachardoublequote}\ \ \ {\isacharparenleft}{\isachardoublequote}{\isacharparenleft}{\isacharunderscore}\ {\isasymTurnstile}\ {\isacharunderscore}{\isacharparenright}{\isachardoublequote}\ {\isacharbrackleft}{\isadigit{8}}{\isadigit{0}}{\isacharcomma}{\isadigit{8}}{\isadigit{0}}{\isacharbrackright}\ {\isadigit{8}}{\isadigit{0}}{\isacharparenright}%
\begin{isamarkuptext}%
\noindent
The concrete syntax annotation allows us to write \isa{s\ {\isasymTurnstile}\ f} instead of
\isa{valid\ s\ f}.

The definition of \isa{{\isasymTurnstile}} is by recursion over the syntax:%
\end{isamarkuptext}%
\isacommand{primrec}\isanewline
{\isachardoublequote}s\ {\isasymTurnstile}\ Atom\ a\ \ {\isacharequal}\ {\isacharparenleft}a\ {\isasymin}\ L\ s{\isacharparenright}{\isachardoublequote}\isanewline
{\isachardoublequote}s\ {\isasymTurnstile}\ Neg\ f\ \ \ {\isacharequal}\ {\isacharparenleft}{\isasymnot}{\isacharparenleft}s\ {\isasymTurnstile}\ f{\isacharparenright}{\isacharparenright}{\isachardoublequote}\isanewline
{\isachardoublequote}s\ {\isasymTurnstile}\ And\ f\ g\ {\isacharequal}\ {\isacharparenleft}s\ {\isasymTurnstile}\ f\ {\isasymand}\ s\ {\isasymTurnstile}\ g{\isacharparenright}{\isachardoublequote}\isanewline
{\isachardoublequote}s\ {\isasymTurnstile}\ AX\ f\ \ \ \ {\isacharequal}\ {\isacharparenleft}{\isasymforall}t{\isachardot}\ {\isacharparenleft}s{\isacharcomma}t{\isacharparenright}\ {\isasymin}\ M\ {\isasymlongrightarrow}\ t\ {\isasymTurnstile}\ f{\isacharparenright}{\isachardoublequote}\isanewline
{\isachardoublequote}s\ {\isasymTurnstile}\ EF\ f\ \ \ \ {\isacharequal}\ {\isacharparenleft}{\isasymexists}t{\isachardot}\ {\isacharparenleft}s{\isacharcomma}t{\isacharparenright}\ {\isasymin}\ M{\isacharcircum}{\isacharasterisk}\ {\isasymand}\ t\ {\isasymTurnstile}\ f{\isacharparenright}{\isachardoublequote}%
\begin{isamarkuptext}%
\noindent
The first three equations should be self-explanatory. The temporal formula
\isa{AX\ f} means that \isa{f} is true in all next states whereas
\isa{EF\ f} means that there exists some future state in which \isa{f} is
true. The future is expressed via \isa{{\isacharcircum}{\isacharasterisk}}, the transitive reflexive
closure. Because of reflexivity, the future includes the present.

Now we come to the model checker itself. It maps a formula into the set of
states where the formula is true and is defined by recursion over the syntax,
too:%
\end{isamarkuptext}%
\isacommand{consts}\ mc\ {\isacharcolon}{\isacharcolon}\ {\isachardoublequote}formula\ {\isasymRightarrow}\ state\ set{\isachardoublequote}\isanewline
\isacommand{primrec}\isanewline
{\isachardoublequote}mc{\isacharparenleft}Atom\ a{\isacharparenright}\ \ {\isacharequal}\ {\isacharbraceleft}s{\isachardot}\ a\ {\isasymin}\ L\ s{\isacharbraceright}{\isachardoublequote}\isanewline
{\isachardoublequote}mc{\isacharparenleft}Neg\ f{\isacharparenright}\ \ \ {\isacharequal}\ {\isacharminus}mc\ f{\isachardoublequote}\isanewline
{\isachardoublequote}mc{\isacharparenleft}And\ f\ g{\isacharparenright}\ {\isacharequal}\ mc\ f\ {\isasyminter}\ mc\ g{\isachardoublequote}\isanewline
{\isachardoublequote}mc{\isacharparenleft}AX\ f{\isacharparenright}\ \ \ \ {\isacharequal}\ {\isacharbraceleft}s{\isachardot}\ {\isasymforall}t{\isachardot}\ {\isacharparenleft}s{\isacharcomma}t{\isacharparenright}\ {\isasymin}\ M\ \ {\isasymlongrightarrow}\ t\ {\isasymin}\ mc\ f{\isacharbraceright}{\isachardoublequote}\isanewline
{\isachardoublequote}mc{\isacharparenleft}EF\ f{\isacharparenright}\ \ \ \ {\isacharequal}\ lfp{\isacharparenleft}{\isasymlambda}T{\isachardot}\ mc\ f\ {\isasymunion}\ M{\isacharcircum}{\isacharminus}{\isadigit{1}}\ {\isacharcircum}{\isacharcircum}\ T{\isacharparenright}{\isachardoublequote}%
\begin{isamarkuptext}%
\noindent
Only the equation for \isa{EF} deserves some comments. Remember that the
postfix \isa{{\isacharcircum}{\isacharminus}{\isadigit{1}}} and the infix \isa{{\isacharcircum}{\isacharcircum}} are predefined and denote the
converse of a relation and the application of a relation to a set. Thus
\isa{M{\isasyminverse}\ {\isacharcircum}{\isacharcircum}\ T} is the set of all predecessors of \isa{T} and the least
fixed point (\isa{lfp}) of \isa{{\isasymlambda}T{\isachardot}\ mc\ f\ {\isasymunion}\ M{\isasyminverse}\ {\isacharcircum}{\isacharcircum}\ T} is the least set
\isa{T} containing \isa{mc\ f} and all predecessors of \isa{T}. If you
find it hard to see that \isa{mc\ {\isacharparenleft}EF\ f{\isacharparenright}} contains exactly those states from
which there is a path to a state where \isa{f} is true, do not worry---that
will be proved in a moment.

First we prove monotonicity of the function inside \isa{lfp}%
\end{isamarkuptext}%
\isacommand{lemma}\ mono{\isacharunderscore}ef{\isacharcolon}\ {\isachardoublequote}mono{\isacharparenleft}{\isasymlambda}T{\isachardot}\ A\ {\isasymunion}\ M{\isacharcircum}{\isacharminus}{\isadigit{1}}\ {\isacharcircum}{\isacharcircum}\ T{\isacharparenright}{\isachardoublequote}\isanewline
\isacommand{apply}{\isacharparenleft}rule\ monoI{\isacharparenright}\isanewline
\isacommand{apply}\ blast\isanewline
\isacommand{done}%
\begin{isamarkuptext}%
\noindent
in order to make sure it really has a least fixed point.

Now we can relate model checking and semantics. For the \isa{EF} case we need
a separate lemma:%
\end{isamarkuptext}%
\isacommand{lemma}\ EF{\isacharunderscore}lemma{\isacharcolon}\isanewline
\ \ {\isachardoublequote}lfp{\isacharparenleft}{\isasymlambda}T{\isachardot}\ A\ {\isasymunion}\ M{\isacharcircum}{\isacharminus}{\isadigit{1}}\ {\isacharcircum}{\isacharcircum}\ T{\isacharparenright}\ {\isacharequal}\ {\isacharbraceleft}s{\isachardot}\ {\isasymexists}t{\isachardot}\ {\isacharparenleft}s{\isacharcomma}t{\isacharparenright}\ {\isasymin}\ M{\isacharcircum}{\isacharasterisk}\ {\isasymand}\ t\ {\isasymin}\ A{\isacharbraceright}{\isachardoublequote}%
\begin{isamarkuptxt}%
\noindent
The equality is proved in the canonical fashion by proving that each set
contains the other; the containment is shown pointwise:%
\end{isamarkuptxt}%
\isacommand{apply}{\isacharparenleft}rule\ equalityI{\isacharparenright}\isanewline
\ \isacommand{apply}{\isacharparenleft}rule\ subsetI{\isacharparenright}\isanewline
\ \isacommand{apply}{\isacharparenleft}simp{\isacharparenright}%
\begin{isamarkuptxt}%
\noindent
Simplification leaves us with the following first subgoal
\begin{isabelle}%
\ {\isadigit{1}}{\isachardot}\ {\isasymAnd}s{\isachardot}\ s\ {\isasymin}\ lfp\ {\isacharparenleft}{\isasymlambda}T{\isachardot}\ A\ {\isasymunion}\ M{\isasyminverse}\ {\isacharcircum}{\isacharcircum}\ T{\isacharparenright}\ {\isasymLongrightarrow}\ {\isasymexists}t{\isachardot}\ {\isacharparenleft}s{\isacharcomma}\ t{\isacharparenright}\ {\isasymin}\ M\isactrlsup {\isacharasterisk}\ {\isasymand}\ t\ {\isasymin}\ A%
\end{isabelle}
which is proved by \isa{lfp}-induction:%
\end{isamarkuptxt}%
\ \isacommand{apply}{\isacharparenleft}erule\ lfp{\isacharunderscore}induct{\isacharparenright}\isanewline
\ \ \isacommand{apply}{\isacharparenleft}rule\ mono{\isacharunderscore}ef{\isacharparenright}\isanewline
\ \isacommand{apply}{\isacharparenleft}simp{\isacharparenright}%
\begin{isamarkuptxt}%
\noindent
Having disposed of the monotonicity subgoal,
simplification leaves us with the following main goal
\begin{isabelle}
\ \isadigit{1}{\isachardot}\ {\isasymAnd}s{\isachardot}\ s\ {\isasymin}\ A\ {\isasymor}\isanewline
\ \ \ \ \ \ \ \ \ s\ {\isasymin}\ M{\isacharcircum}{\isacharminus}\isadigit{1}\ {\isacharcircum}{\isacharcircum}\ {\isacharparenleft}lfp\ {\isacharparenleft}{\dots}{\isacharparenright}\ {\isasyminter}\ {\isacharbraceleft}x{\isachardot}\ {\isasymexists}t{\isachardot}\ {\isacharparenleft}x{\isacharcomma}\ t{\isacharparenright}\ {\isasymin}\ M{\isacharcircum}{\isacharasterisk}\ {\isasymand}\ t\ {\isasymin}\ A{\isacharbraceright}{\isacharparenright}\isanewline
\ \ \ \ \ \ \ \ \ {\isasymLongrightarrow}\ {\isasymexists}t{\isachardot}\ {\isacharparenleft}s{\isacharcomma}\ t{\isacharparenright}\ {\isasymin}\ M{\isacharcircum}{\isacharasterisk}\ {\isasymand}\ t\ {\isasymin}\ A
\end{isabelle}
which is proved by \isa{blast} with the help of transitivity of \isa{{\isacharcircum}{\isacharasterisk}}:%
\end{isamarkuptxt}%
\ \isacommand{apply}{\isacharparenleft}blast\ intro{\isacharcolon}\ rtrancl{\isacharunderscore}trans{\isacharparenright}%
\begin{isamarkuptxt}%
We now return to the second set containment subgoal, which is again proved
pointwise:%
\end{isamarkuptxt}%
\isacommand{apply}{\isacharparenleft}rule\ subsetI{\isacharparenright}\isanewline
\isacommand{apply}{\isacharparenleft}simp{\isacharcomma}\ clarify{\isacharparenright}%
\begin{isamarkuptxt}%
\noindent
After simplification and clarification we are left with
\begin{isabelle}%
\ {\isadigit{1}}{\isachardot}\ {\isasymAnd}x\ t{\isachardot}\ {\isasymlbrakk}{\isacharparenleft}x{\isacharcomma}\ t{\isacharparenright}\ {\isasymin}\ M\isactrlsup {\isacharasterisk}{\isacharsemicolon}\ t\ {\isasymin}\ A{\isasymrbrakk}\ {\isasymLongrightarrow}\ x\ {\isasymin}\ lfp\ {\isacharparenleft}{\isasymlambda}T{\isachardot}\ A\ {\isasymunion}\ M{\isasyminverse}\ {\isacharcircum}{\isacharcircum}\ T{\isacharparenright}%
\end{isabelle}
This goal is proved by induction on \isa{{\isacharparenleft}s{\isacharcomma}\ t{\isacharparenright}\ {\isasymin}\ M\isactrlsup {\isacharasterisk}}. But since the model
checker works backwards (from \isa{t} to \isa{s}), we cannot use the
induction theorem \isa{rtrancl{\isacharunderscore}induct} because it works in the
forward direction. Fortunately the converse induction theorem
\isa{converse{\isacharunderscore}rtrancl{\isacharunderscore}induct} already exists:
\begin{isabelle}%
\ \ \ \ \ {\isasymlbrakk}{\isacharparenleft}a{\isacharcomma}\ b{\isacharparenright}\ {\isasymin}\ r\isactrlsup {\isacharasterisk}{\isacharsemicolon}\ P\ b{\isacharsemicolon}\isanewline
\ \ \ \ \ \ \ \ {\isasymAnd}y\ z{\isachardot}\ {\isasymlbrakk}{\isacharparenleft}y{\isacharcomma}\ z{\isacharparenright}\ {\isasymin}\ r{\isacharsemicolon}\ {\isacharparenleft}z{\isacharcomma}\ b{\isacharparenright}\ {\isasymin}\ r\isactrlsup {\isacharasterisk}{\isacharsemicolon}\ P\ z{\isasymrbrakk}\ {\isasymLongrightarrow}\ P\ y{\isasymrbrakk}\isanewline
\ \ \ \ \ {\isasymLongrightarrow}\ P\ a%
\end{isabelle}
It says that if \isa{{\isacharparenleft}a{\isacharcomma}\ b{\isacharparenright}\ {\isasymin}\ r\isactrlsup {\isacharasterisk}} and we know \isa{P\ b} then we can infer
\isa{P\ a} provided each step backwards from a predecessor \isa{z} of
\isa{b} preserves \isa{P}.%
\end{isamarkuptxt}%
\isacommand{apply}{\isacharparenleft}erule\ converse{\isacharunderscore}rtrancl{\isacharunderscore}induct{\isacharparenright}%
\begin{isamarkuptxt}%
\noindent
The base case
\begin{isabelle}%
\ {\isadigit{1}}{\isachardot}\ {\isasymAnd}x\ t{\isachardot}\ t\ {\isasymin}\ A\ {\isasymLongrightarrow}\ t\ {\isasymin}\ lfp\ {\isacharparenleft}{\isasymlambda}T{\isachardot}\ A\ {\isasymunion}\ M{\isasyminverse}\ {\isacharcircum}{\isacharcircum}\ T{\isacharparenright}%
\end{isabelle}
is solved by unrolling \isa{lfp} once%
\end{isamarkuptxt}%
\ \isacommand{apply}{\isacharparenleft}rule\ ssubst{\isacharbrackleft}OF\ lfp{\isacharunderscore}unfold{\isacharbrackleft}OF\ mono{\isacharunderscore}ef{\isacharbrackright}{\isacharbrackright}{\isacharparenright}%
\begin{isamarkuptxt}%
\begin{isabelle}%
\ {\isadigit{1}}{\isachardot}\ {\isasymAnd}x\ t{\isachardot}\ t\ {\isasymin}\ A\ {\isasymLongrightarrow}\ t\ {\isasymin}\ A\ {\isasymunion}\ M{\isasyminverse}\ {\isacharcircum}{\isacharcircum}\ lfp\ {\isacharparenleft}{\isasymlambda}T{\isachardot}\ A\ {\isasymunion}\ M{\isasyminverse}\ {\isacharcircum}{\isacharcircum}\ T{\isacharparenright}%
\end{isabelle}
and disposing of the resulting trivial subgoal automatically:%
\end{isamarkuptxt}%
\ \isacommand{apply}{\isacharparenleft}blast{\isacharparenright}%
\begin{isamarkuptxt}%
\noindent
The proof of the induction step is identical to the one for the base case:%
\end{isamarkuptxt}%
\isacommand{apply}{\isacharparenleft}rule\ ssubst{\isacharbrackleft}OF\ lfp{\isacharunderscore}unfold{\isacharbrackleft}OF\ mono{\isacharunderscore}ef{\isacharbrackright}{\isacharbrackright}{\isacharparenright}\isanewline
\isacommand{apply}{\isacharparenleft}blast{\isacharparenright}\isanewline
\isacommand{done}%
\begin{isamarkuptext}%
The main theorem is proved in the familiar manner: induction followed by
\isa{auto} augmented with the lemma as a simplification rule.%
\end{isamarkuptext}%
\isacommand{theorem}\ {\isachardoublequote}mc\ f\ {\isacharequal}\ {\isacharbraceleft}s{\isachardot}\ s\ {\isasymTurnstile}\ f{\isacharbraceright}{\isachardoublequote}\isanewline
\isacommand{apply}{\isacharparenleft}induct{\isacharunderscore}tac\ f{\isacharparenright}\isanewline
\isacommand{apply}{\isacharparenleft}auto\ simp\ add{\isacharcolon}EF{\isacharunderscore}lemma{\isacharparenright}\isanewline
\isacommand{done}%
\begin{isamarkuptext}%
\begin{exercise}
\isa{AX} has a dual operator \isa{EN}\footnote{We cannot use the customary \isa{EX}
as that is the ASCII equivalent of \isa{{\isasymexists}}}
(``there exists a next state such that'') with the intended semantics
\begin{isabelle}%
\ \ \ \ \ s\ {\isasymTurnstile}\ EN\ f\ {\isacharequal}\ {\isacharparenleft}{\isasymexists}t{\isachardot}\ {\isacharparenleft}s{\isacharcomma}\ t{\isacharparenright}\ {\isasymin}\ M\ {\isasymand}\ t\ {\isasymTurnstile}\ f{\isacharparenright}%
\end{isabelle}
Fortunately, \isa{EN\ f} can already be expressed as a PDL formula. How?

Show that the semantics for \isa{EF} satisfies the following recursion equation:
\begin{isabelle}%
\ \ \ \ \ s\ {\isasymTurnstile}\ EF\ f\ {\isacharequal}\ {\isacharparenleft}s\ {\isasymTurnstile}\ f\ {\isasymor}\ s\ {\isasymTurnstile}\ EN\ {\isacharparenleft}EF\ f{\isacharparenright}{\isacharparenright}%
\end{isabelle}
\end{exercise}
\index{PDL|)}%
\end{isamarkuptext}%
\end{isabellebody}%
%%% Local Variables:
%%% mode: latex
%%% TeX-master: "root"
%%% End:

%
\begin{isabellebody}%
\def\isabellecontext{CTL}%
%
\isamarkupsubsection{Computation Tree Logic --- CTL%
}
%
\begin{isamarkuptext}%
\label{sec:CTL}
\index{CTL|(}%
The semantics of PDL only needs reflexive transitive closure.
Let us be adventurous and introduce a more expressive temporal operator.
We extend the datatype
\isa{formula} by a new constructor%
\end{isamarkuptext}%
\ \ \ \ \ \ \ \ \ \ \ \ \ \ \ \ \ \ {\isacharbar}\ AF\ formula%
\begin{isamarkuptext}%
\noindent
which stands for ``\emph{A}lways in the \emph{F}uture'':
on all infinite paths, at some point the formula holds.
Formalizing the notion of an infinite path is easy
in HOL: it is simply a function from \isa{nat} to \isa{state}.%
\end{isamarkuptext}%
\isacommand{constdefs}\ Paths\ {\isacharcolon}{\isacharcolon}\ {\isachardoublequote}state\ {\isasymRightarrow}\ {\isacharparenleft}nat\ {\isasymRightarrow}\ state{\isacharparenright}set{\isachardoublequote}\isanewline
\ \ \ \ \ \ \ \ \ {\isachardoublequote}Paths\ s\ {\isasymequiv}\ {\isacharbraceleft}p{\isachardot}\ s\ {\isacharequal}\ p\ {\isadigit{0}}\ {\isasymand}\ {\isacharparenleft}{\isasymforall}i{\isachardot}\ {\isacharparenleft}p\ i{\isacharcomma}\ p{\isacharparenleft}i{\isacharplus}{\isadigit{1}}{\isacharparenright}{\isacharparenright}\ {\isasymin}\ M{\isacharparenright}{\isacharbraceright}{\isachardoublequote}%
\begin{isamarkuptext}%
\noindent
This definition allows a succinct statement of the semantics of \isa{AF}:
\footnote{Do not be misled: neither datatypes nor recursive functions can be
extended by new constructors or equations. This is just a trick of the
presentation. In reality one has to define a new datatype and a new function.}%
\end{isamarkuptext}%
{\isachardoublequote}s\ {\isasymTurnstile}\ AF\ f\ \ \ \ {\isacharequal}\ {\isacharparenleft}{\isasymforall}p\ {\isasymin}\ Paths\ s{\isachardot}\ {\isasymexists}i{\isachardot}\ p\ i\ {\isasymTurnstile}\ f{\isacharparenright}{\isachardoublequote}%
\begin{isamarkuptext}%
\noindent
Model checking \isa{AF} involves a function which
is just complicated enough to warrant a separate definition:%
\end{isamarkuptext}%
\isacommand{constdefs}\ af\ {\isacharcolon}{\isacharcolon}\ {\isachardoublequote}state\ set\ {\isasymRightarrow}\ state\ set\ {\isasymRightarrow}\ state\ set{\isachardoublequote}\isanewline
\ \ \ \ \ \ \ \ \ {\isachardoublequote}af\ A\ T\ {\isasymequiv}\ A\ {\isasymunion}\ {\isacharbraceleft}s{\isachardot}\ {\isasymforall}t{\isachardot}\ {\isacharparenleft}s{\isacharcomma}\ t{\isacharparenright}\ {\isasymin}\ M\ {\isasymlongrightarrow}\ t\ {\isasymin}\ T{\isacharbraceright}{\isachardoublequote}%
\begin{isamarkuptext}%
\noindent
Now we define \isa{mc\ {\isacharparenleft}AF\ f{\isacharparenright}} as the least set \isa{T} that includes
\isa{mc\ f} and all states all of whose direct successors are in \isa{T}:%
\end{isamarkuptext}%
{\isachardoublequote}mc{\isacharparenleft}AF\ f{\isacharparenright}\ \ \ \ {\isacharequal}\ lfp{\isacharparenleft}af{\isacharparenleft}mc\ f{\isacharparenright}{\isacharparenright}{\isachardoublequote}%
\begin{isamarkuptext}%
\noindent
Because \isa{af} is monotone in its second argument (and also its first, but
that is irrelevant), \isa{af\ A} has a least fixed point:%
\end{isamarkuptext}%
\isacommand{lemma}\ mono{\isacharunderscore}af{\isacharcolon}\ {\isachardoublequote}mono{\isacharparenleft}af\ A{\isacharparenright}{\isachardoublequote}\isanewline
\isacommand{apply}{\isacharparenleft}simp\ add{\isacharcolon}\ mono{\isacharunderscore}def\ af{\isacharunderscore}def{\isacharparenright}\isanewline
\isacommand{apply}\ blast\isanewline
\isacommand{done}%
\begin{isamarkuptext}%
All we need to prove now is  \isa{mc\ {\isacharparenleft}AF\ f{\isacharparenright}\ {\isacharequal}\ {\isacharbraceleft}s{\isachardot}\ s\ {\isasymTurnstile}\ AF\ f{\isacharbraceright}}, which states
that \isa{mc} and \isa{{\isasymTurnstile}} agree for \isa{AF}\@.
This time we prove the two inclusions separately, starting
with the easy one:%
\end{isamarkuptext}%
\isacommand{theorem}\ AF{\isacharunderscore}lemma{\isadigit{1}}{\isacharcolon}\isanewline
\ \ {\isachardoublequote}lfp{\isacharparenleft}af\ A{\isacharparenright}\ {\isasymsubseteq}\ {\isacharbraceleft}s{\isachardot}\ {\isasymforall}\ p\ {\isasymin}\ Paths\ s{\isachardot}\ {\isasymexists}\ i{\isachardot}\ p\ i\ {\isasymin}\ A{\isacharbraceright}{\isachardoublequote}%
\begin{isamarkuptxt}%
\noindent
In contrast to the analogous proof for \isa{EF}, and just
for a change, we do not use fixed point induction.  Park-induction,
named after David Park, is weaker but sufficient for this proof:
\begin{center}
\isa{f\ S\ {\isasymsubseteq}\ S\ {\isasymLongrightarrow}\ lfp\ f\ {\isasymsubseteq}\ S} \hfill (\isa{lfp{\isacharunderscore}lowerbound})
\end{center}
The instance of the premise \isa{f\ S\ {\isasymsubseteq}\ S} is proved pointwise,
a decision that clarification takes for us:%
\end{isamarkuptxt}%
\isacommand{apply}{\isacharparenleft}rule\ lfp{\isacharunderscore}lowerbound{\isacharparenright}\isanewline
\isacommand{apply}{\isacharparenleft}clarsimp\ simp\ add{\isacharcolon}\ af{\isacharunderscore}def\ Paths{\isacharunderscore}def{\isacharparenright}%
\begin{isamarkuptxt}%
\begin{isabelle}%
\ {\isadigit{1}}{\isachardot}\ {\isasymAnd}p{\isachardot}\ {\isasymlbrakk}p\ {\isadigit{0}}\ {\isasymin}\ A\ {\isasymor}\isanewline
\isaindent{\ {\isadigit{1}}{\isachardot}\ {\isasymAnd}p{\isachardot}\ {\isasymlbrakk}}{\isacharparenleft}{\isasymforall}t{\isachardot}\ {\isacharparenleft}p\ {\isadigit{0}}{\isacharcomma}\ t{\isacharparenright}\ {\isasymin}\ M\ {\isasymlongrightarrow}\isanewline
\isaindent{\ {\isadigit{1}}{\isachardot}\ {\isasymAnd}p{\isachardot}\ {\isasymlbrakk}{\isacharparenleft}{\isasymforall}t{\isachardot}\ }{\isacharparenleft}{\isasymforall}p{\isachardot}\ t\ {\isacharequal}\ p\ {\isadigit{0}}\ {\isasymand}\ {\isacharparenleft}{\isasymforall}i{\isachardot}\ {\isacharparenleft}p\ i{\isacharcomma}\ p\ {\isacharparenleft}Suc\ i{\isacharparenright}{\isacharparenright}\ {\isasymin}\ M{\isacharparenright}\ {\isasymlongrightarrow}\isanewline
\isaindent{\ {\isadigit{1}}{\isachardot}\ {\isasymAnd}p{\isachardot}\ {\isasymlbrakk}{\isacharparenleft}{\isasymforall}t{\isachardot}\ {\isacharparenleft}{\isasymforall}p{\isachardot}\ }{\isacharparenleft}{\isasymexists}i{\isachardot}\ p\ i\ {\isasymin}\ A{\isacharparenright}{\isacharparenright}{\isacharparenright}{\isacharsemicolon}\isanewline
\isaindent{\ {\isadigit{1}}{\isachardot}\ {\isasymAnd}p{\isachardot}\ \ \ \ }{\isasymforall}i{\isachardot}\ {\isacharparenleft}p\ i{\isacharcomma}\ p\ {\isacharparenleft}Suc\ i{\isacharparenright}{\isacharparenright}\ {\isasymin}\ M{\isasymrbrakk}\isanewline
\isaindent{\ {\isadigit{1}}{\isachardot}\ {\isasymAnd}p{\isachardot}\ }{\isasymLongrightarrow}\ {\isasymexists}i{\isachardot}\ p\ i\ {\isasymin}\ A%
\end{isabelle}
Now we eliminate the disjunction. The case \isa{p\ {\isadigit{0}}\ {\isasymin}\ A} is trivial:%
\end{isamarkuptxt}%
\isacommand{apply}{\isacharparenleft}erule\ disjE{\isacharparenright}\isanewline
\ \isacommand{apply}{\isacharparenleft}blast{\isacharparenright}%
\begin{isamarkuptxt}%
\noindent
In the other case we set \isa{t} to \isa{p\ {\isadigit{1}}} and simplify matters:%
\end{isamarkuptxt}%
\isacommand{apply}{\isacharparenleft}erule{\isacharunderscore}tac\ x\ {\isacharequal}\ {\isachardoublequote}p\ {\isadigit{1}}{\isachardoublequote}\ \isakeyword{in}\ allE{\isacharparenright}\isanewline
\isacommand{apply}{\isacharparenleft}clarsimp{\isacharparenright}%
\begin{isamarkuptxt}%
\begin{isabelle}%
\ {\isadigit{1}}{\isachardot}\ {\isasymAnd}p{\isachardot}\ {\isasymlbrakk}{\isasymforall}i{\isachardot}\ {\isacharparenleft}p\ i{\isacharcomma}\ p\ {\isacharparenleft}Suc\ i{\isacharparenright}{\isacharparenright}\ {\isasymin}\ M{\isacharsemicolon}\isanewline
\isaindent{\ {\isadigit{1}}{\isachardot}\ {\isasymAnd}p{\isachardot}\ \ \ \ }{\isasymforall}pa{\isachardot}\ p\ {\isadigit{1}}{\isacharprime}\ {\isacharequal}\ pa\ {\isadigit{0}}\ {\isasymand}\ {\isacharparenleft}{\isasymforall}i{\isachardot}\ {\isacharparenleft}pa\ i{\isacharcomma}\ pa\ {\isacharparenleft}Suc\ i{\isacharparenright}{\isacharparenright}\ {\isasymin}\ M{\isacharparenright}\ {\isasymlongrightarrow}\isanewline
\isaindent{\ {\isadigit{1}}{\isachardot}\ {\isasymAnd}p{\isachardot}\ \ \ \ {\isasymforall}pa{\isachardot}\ }{\isacharparenleft}{\isasymexists}i{\isachardot}\ pa\ i\ {\isasymin}\ A{\isacharparenright}{\isasymrbrakk}\isanewline
\isaindent{\ {\isadigit{1}}{\isachardot}\ {\isasymAnd}p{\isachardot}\ }{\isasymLongrightarrow}\ {\isasymexists}i{\isachardot}\ p\ i\ {\isasymin}\ A%
\end{isabelle}
It merely remains to set \isa{pa} to \isa{{\isasymlambda}i{\isachardot}\ p\ {\isacharparenleft}i\ {\isacharplus}\ {\isadigit{1}}{\isacharparenright}}, that is, 
\isa{p} without its first element.  The rest is automatic:%
\end{isamarkuptxt}%
\isacommand{apply}{\isacharparenleft}erule{\isacharunderscore}tac\ x\ {\isacharequal}\ {\isachardoublequote}{\isasymlambda}i{\isachardot}\ p{\isacharparenleft}i{\isacharplus}{\isadigit{1}}{\isacharparenright}{\isachardoublequote}\ \isakeyword{in}\ allE{\isacharparenright}\isanewline
\isacommand{apply}\ force\isanewline
\isacommand{done}%
\begin{isamarkuptext}%
The opposite inclusion is proved by contradiction: if some state
\isa{s} is not in \isa{lfp\ {\isacharparenleft}af\ A{\isacharparenright}}, then we can construct an
infinite \isa{A}-avoiding path starting from~\isa{s}. The reason is
that by unfolding \isa{lfp} we find that if \isa{s} is not in
\isa{lfp\ {\isacharparenleft}af\ A{\isacharparenright}}, then \isa{s} is not in \isa{A} and there is a
direct successor of \isa{s} that is again not in \mbox{\isa{lfp\ {\isacharparenleft}af\ A{\isacharparenright}}}. Iterating this argument yields the promised infinite
\isa{A}-avoiding path. Let us formalize this sketch.

The one-step argument in the sketch above
is proved by a variant of contraposition:%
\end{isamarkuptext}%
\isacommand{lemma}\ not{\isacharunderscore}in{\isacharunderscore}lfp{\isacharunderscore}afD{\isacharcolon}\isanewline
\ {\isachardoublequote}s\ {\isasymnotin}\ lfp{\isacharparenleft}af\ A{\isacharparenright}\ {\isasymLongrightarrow}\ s\ {\isasymnotin}\ A\ {\isasymand}\ {\isacharparenleft}{\isasymexists}\ t{\isachardot}\ {\isacharparenleft}s{\isacharcomma}t{\isacharparenright}\ {\isasymin}\ M\ {\isasymand}\ t\ {\isasymnotin}\ lfp{\isacharparenleft}af\ A{\isacharparenright}{\isacharparenright}{\isachardoublequote}\isanewline
\isacommand{apply}{\isacharparenleft}erule\ contrapos{\isacharunderscore}np{\isacharparenright}\isanewline
\isacommand{apply}{\isacharparenleft}subst\ lfp{\isacharunderscore}unfold{\isacharbrackleft}OF\ mono{\isacharunderscore}af{\isacharbrackright}{\isacharparenright}\isanewline
\isacommand{apply}{\isacharparenleft}simp\ add{\isacharcolon}af{\isacharunderscore}def{\isacharparenright}\isanewline
\isacommand{done}%
\begin{isamarkuptext}%
\noindent
We assume the negation of the conclusion and prove \isa{s\ {\isasymin}\ lfp\ {\isacharparenleft}af\ A{\isacharparenright}}.
Unfolding \isa{lfp} once and
simplifying with the definition of \isa{af} finishes the proof.

Now we iterate this process. The following construction of the desired
path is parameterized by a predicate \isa{Q} that should hold along the path:%
\end{isamarkuptext}%
\isacommand{consts}\ path\ {\isacharcolon}{\isacharcolon}\ {\isachardoublequote}state\ {\isasymRightarrow}\ {\isacharparenleft}state\ {\isasymRightarrow}\ bool{\isacharparenright}\ {\isasymRightarrow}\ {\isacharparenleft}nat\ {\isasymRightarrow}\ state{\isacharparenright}{\isachardoublequote}\isanewline
\isacommand{primrec}\isanewline
{\isachardoublequote}path\ s\ Q\ {\isadigit{0}}\ {\isacharequal}\ s{\isachardoublequote}\isanewline
{\isachardoublequote}path\ s\ Q\ {\isacharparenleft}Suc\ n{\isacharparenright}\ {\isacharequal}\ {\isacharparenleft}SOME\ t{\isachardot}\ {\isacharparenleft}path\ s\ Q\ n{\isacharcomma}t{\isacharparenright}\ {\isasymin}\ M\ {\isasymand}\ Q\ t{\isacharparenright}{\isachardoublequote}%
\begin{isamarkuptext}%
\noindent
Element \isa{n\ {\isacharplus}\ {\isadigit{1}}} on this path is some arbitrary successor
\isa{t} of element \isa{n} such that \isa{Q\ t} holds.  Remember that \isa{SOME\ t{\isachardot}\ R\ t}
is some arbitrary but fixed \isa{t} such that \isa{R\ t} holds (see \S\ref{sec:SOME}). Of
course, such a \isa{t} need not exist, but that is of no
concern to us since we will only use \isa{path} when a
suitable \isa{t} does exist.

Let us show that if each state \isa{s} that satisfies \isa{Q}
has a successor that again satisfies \isa{Q}, then there exists an infinite \isa{Q}-path:%
\end{isamarkuptext}%
\isacommand{lemma}\ infinity{\isacharunderscore}lemma{\isacharcolon}\isanewline
\ \ {\isachardoublequote}{\isasymlbrakk}\ Q\ s{\isacharsemicolon}\ {\isasymforall}s{\isachardot}\ Q\ s\ {\isasymlongrightarrow}\ {\isacharparenleft}{\isasymexists}\ t{\isachardot}\ {\isacharparenleft}s{\isacharcomma}t{\isacharparenright}\ {\isasymin}\ M\ {\isasymand}\ Q\ t{\isacharparenright}\ {\isasymrbrakk}\ {\isasymLongrightarrow}\isanewline
\ \ \ {\isasymexists}p{\isasymin}Paths\ s{\isachardot}\ {\isasymforall}i{\isachardot}\ Q{\isacharparenleft}p\ i{\isacharparenright}{\isachardoublequote}%
\begin{isamarkuptxt}%
\noindent
First we rephrase the conclusion slightly because we need to prove simultaneously
both the path property and the fact that \isa{Q} holds:%
\end{isamarkuptxt}%
\isacommand{apply}{\isacharparenleft}subgoal{\isacharunderscore}tac\ {\isachardoublequote}{\isasymexists}p{\isachardot}\ s\ {\isacharequal}\ p\ {\isadigit{0}}\ {\isasymand}\ {\isacharparenleft}{\isasymforall}i{\isachardot}\ {\isacharparenleft}p\ i{\isacharcomma}p{\isacharparenleft}i{\isacharplus}{\isadigit{1}}{\isacharparenright}{\isacharparenright}\ {\isasymin}\ M\ {\isasymand}\ Q{\isacharparenleft}p\ i{\isacharparenright}{\isacharparenright}{\isachardoublequote}{\isacharparenright}%
\begin{isamarkuptxt}%
\noindent
From this proposition the original goal follows easily:%
\end{isamarkuptxt}%
\ \isacommand{apply}{\isacharparenleft}simp\ add{\isacharcolon}Paths{\isacharunderscore}def{\isacharcomma}\ blast{\isacharparenright}%
\begin{isamarkuptxt}%
\noindent
The new subgoal is proved by providing the witness \isa{path\ s\ Q} for \isa{p}:%
\end{isamarkuptxt}%
\isacommand{apply}{\isacharparenleft}rule{\isacharunderscore}tac\ x\ {\isacharequal}\ {\isachardoublequote}path\ s\ Q{\isachardoublequote}\ \isakeyword{in}\ exI{\isacharparenright}\isanewline
\isacommand{apply}{\isacharparenleft}clarsimp{\isacharparenright}%
\begin{isamarkuptxt}%
\noindent
After simplification and clarification, the subgoal has the following form:
\begin{isabelle}%
\ {\isadigit{1}}{\isachardot}\ {\isasymAnd}i{\isachardot}\ {\isasymlbrakk}Q\ s{\isacharsemicolon}\ {\isasymforall}s{\isachardot}\ Q\ s\ {\isasymlongrightarrow}\ {\isacharparenleft}{\isasymexists}t{\isachardot}\ {\isacharparenleft}s{\isacharcomma}\ t{\isacharparenright}\ {\isasymin}\ M\ {\isasymand}\ Q\ t{\isacharparenright}{\isasymrbrakk}\isanewline
\isaindent{\ {\isadigit{1}}{\isachardot}\ {\isasymAnd}i{\isachardot}\ }{\isasymLongrightarrow}\ {\isacharparenleft}path\ s\ Q\ i{\isacharcomma}\ SOME\ t{\isachardot}\ {\isacharparenleft}path\ s\ Q\ i{\isacharcomma}\ t{\isacharparenright}\ {\isasymin}\ M\ {\isasymand}\ Q\ t{\isacharparenright}\ {\isasymin}\ M\ {\isasymand}\isanewline
\isaindent{\ {\isadigit{1}}{\isachardot}\ {\isasymAnd}i{\isachardot}\ {\isasymLongrightarrow}\ }Q\ {\isacharparenleft}path\ s\ Q\ i{\isacharparenright}%
\end{isabelle}
It invites a proof by induction on \isa{i}:%
\end{isamarkuptxt}%
\isacommand{apply}{\isacharparenleft}induct{\isacharunderscore}tac\ i{\isacharparenright}\isanewline
\ \isacommand{apply}{\isacharparenleft}simp{\isacharparenright}%
\begin{isamarkuptxt}%
\noindent
After simplification, the base case boils down to
\begin{isabelle}%
\ {\isadigit{1}}{\isachardot}\ {\isasymlbrakk}Q\ s{\isacharsemicolon}\ {\isasymforall}s{\isachardot}\ Q\ s\ {\isasymlongrightarrow}\ {\isacharparenleft}{\isasymexists}t{\isachardot}\ {\isacharparenleft}s{\isacharcomma}\ t{\isacharparenright}\ {\isasymin}\ M\ {\isasymand}\ Q\ t{\isacharparenright}{\isasymrbrakk}\isanewline
\isaindent{\ {\isadigit{1}}{\isachardot}\ }{\isasymLongrightarrow}\ {\isacharparenleft}s{\isacharcomma}\ SOME\ t{\isachardot}\ {\isacharparenleft}s{\isacharcomma}\ t{\isacharparenright}\ {\isasymin}\ M\ {\isasymand}\ Q\ t{\isacharparenright}\ {\isasymin}\ M%
\end{isabelle}
The conclusion looks exceedingly trivial: after all, \isa{t} is chosen such that \isa{{\isacharparenleft}s{\isacharcomma}\ t{\isacharparenright}\ {\isasymin}\ M}
holds. However, we first have to show that such a \isa{t} actually exists! This reasoning
is embodied in the theorem \isa{someI{\isadigit{2}}{\isacharunderscore}ex}:
\begin{isabelle}%
\ \ \ \ \ {\isasymlbrakk}{\isasymexists}a{\isachardot}\ {\isacharquery}P\ a{\isacharsemicolon}\ {\isasymAnd}x{\isachardot}\ {\isacharquery}P\ x\ {\isasymLongrightarrow}\ {\isacharquery}Q\ x{\isasymrbrakk}\ {\isasymLongrightarrow}\ {\isacharquery}Q\ {\isacharparenleft}SOME\ x{\isachardot}\ {\isacharquery}P\ x{\isacharparenright}%
\end{isabelle}
When we apply this theorem as an introduction rule, \isa{{\isacharquery}P\ x} becomes
\isa{{\isacharparenleft}s{\isacharcomma}\ x{\isacharparenright}\ {\isasymin}\ M\ {\isasymand}\ Q\ x} and \isa{{\isacharquery}Q\ x} becomes \isa{{\isacharparenleft}s{\isacharcomma}\ x{\isacharparenright}\ {\isasymin}\ M} and we have to prove
two subgoals: \isa{{\isasymexists}a{\isachardot}\ {\isacharparenleft}s{\isacharcomma}\ a{\isacharparenright}\ {\isasymin}\ M\ {\isasymand}\ Q\ a}, which follows from the assumptions, and
\isa{{\isacharparenleft}s{\isacharcomma}\ x{\isacharparenright}\ {\isasymin}\ M\ {\isasymand}\ Q\ x\ {\isasymLongrightarrow}\ {\isacharparenleft}s{\isacharcomma}\ x{\isacharparenright}\ {\isasymin}\ M}, which is trivial. Thus it is not surprising that
\isa{fast} can prove the base case quickly:%
\end{isamarkuptxt}%
\ \isacommand{apply}{\isacharparenleft}fast\ intro{\isacharcolon}someI{\isadigit{2}}{\isacharunderscore}ex{\isacharparenright}%
\begin{isamarkuptxt}%
\noindent
What is worth noting here is that we have used \methdx{fast} rather than
\isa{blast}.  The reason is that \isa{blast} would fail because it cannot
cope with \isa{someI{\isadigit{2}}{\isacharunderscore}ex}: unifying its conclusion with the current
subgoal is non-trivial because of the nested schematic variables. For
efficiency reasons \isa{blast} does not even attempt such unifications.
Although \isa{fast} can in principle cope with complicated unification
problems, in practice the number of unifiers arising is often prohibitive and
the offending rule may need to be applied explicitly rather than
automatically. This is what happens in the step case.

The induction step is similar, but more involved, because now we face nested
occurrences of \isa{SOME}. As a result, \isa{fast} is no longer able to
solve the subgoal and we apply \isa{someI{\isadigit{2}}{\isacharunderscore}ex} by hand.  We merely
show the proof commands but do not describe the details:%
\end{isamarkuptxt}%
\isacommand{apply}{\isacharparenleft}simp{\isacharparenright}\isanewline
\isacommand{apply}{\isacharparenleft}rule\ someI{\isadigit{2}}{\isacharunderscore}ex{\isacharparenright}\isanewline
\ \isacommand{apply}{\isacharparenleft}blast{\isacharparenright}\isanewline
\isacommand{apply}{\isacharparenleft}rule\ someI{\isadigit{2}}{\isacharunderscore}ex{\isacharparenright}\isanewline
\ \isacommand{apply}{\isacharparenleft}blast{\isacharparenright}\isanewline
\isacommand{apply}{\isacharparenleft}blast{\isacharparenright}\isanewline
\isacommand{done}%
\begin{isamarkuptext}%
Function \isa{path} has fulfilled its purpose now and can be forgotten.
It was merely defined to provide the witness in the proof of the
\isa{infinity{\isacharunderscore}lemma}. Aficionados of minimal proofs might like to know
that we could have given the witness without having to define a new function:
the term
\begin{isabelle}%
\ \ \ \ \ nat{\isacharunderscore}rec\ s\ {\isacharparenleft}{\isasymlambda}n\ t{\isachardot}\ SOME\ u{\isachardot}\ {\isacharparenleft}t{\isacharcomma}\ u{\isacharparenright}\ {\isasymin}\ M\ {\isasymand}\ Q\ u{\isacharparenright}%
\end{isabelle}
is extensionally equal to \isa{path\ s\ Q},
where \isa{nat{\isacharunderscore}rec} is the predefined primitive recursor on \isa{nat}.%
\end{isamarkuptext}%
%
\begin{isamarkuptext}%
At last we can prove the opposite direction of \isa{AF{\isacharunderscore}lemma{\isadigit{1}}}:%
\end{isamarkuptext}%
\isacommand{theorem}\ AF{\isacharunderscore}lemma{\isadigit{2}}{\isacharcolon}\ {\isachardoublequote}{\isacharbraceleft}s{\isachardot}\ {\isasymforall}\ p\ {\isasymin}\ Paths\ s{\isachardot}\ {\isasymexists}\ i{\isachardot}\ p\ i\ {\isasymin}\ A{\isacharbraceright}\ {\isasymsubseteq}\ lfp{\isacharparenleft}af\ A{\isacharparenright}{\isachardoublequote}%
\begin{isamarkuptxt}%
\noindent
The proof is again pointwise and then by contraposition:%
\end{isamarkuptxt}%
\isacommand{apply}{\isacharparenleft}rule\ subsetI{\isacharparenright}\isanewline
\isacommand{apply}{\isacharparenleft}erule\ contrapos{\isacharunderscore}pp{\isacharparenright}\isanewline
\isacommand{apply}\ simp%
\begin{isamarkuptxt}%
\begin{isabelle}%
\ {\isadigit{1}}{\isachardot}\ {\isasymAnd}x{\isachardot}\ x\ {\isasymnotin}\ lfp\ {\isacharparenleft}af\ A{\isacharparenright}\ {\isasymLongrightarrow}\ {\isasymexists}p{\isasymin}Paths\ x{\isachardot}\ {\isasymforall}i{\isachardot}\ p\ i\ {\isasymnotin}\ A%
\end{isabelle}
Applying the \isa{infinity{\isacharunderscore}lemma} as a destruction rule leaves two subgoals, the second
premise of \isa{infinity{\isacharunderscore}lemma} and the original subgoal:%
\end{isamarkuptxt}%
\isacommand{apply}{\isacharparenleft}drule\ infinity{\isacharunderscore}lemma{\isacharparenright}%
\begin{isamarkuptxt}%
\begin{isabelle}%
\ {\isadigit{1}}{\isachardot}\ {\isasymAnd}x{\isachardot}\ {\isasymforall}s{\isachardot}\ s\ {\isasymnotin}\ lfp\ {\isacharparenleft}af\ A{\isacharparenright}\ {\isasymlongrightarrow}\ {\isacharparenleft}{\isasymexists}t{\isachardot}\ {\isacharparenleft}s{\isacharcomma}\ t{\isacharparenright}\ {\isasymin}\ M\ {\isasymand}\ t\ {\isasymnotin}\ lfp\ {\isacharparenleft}af\ A{\isacharparenright}{\isacharparenright}\isanewline
\ {\isadigit{2}}{\isachardot}\ {\isasymAnd}x{\isachardot}\ {\isasymexists}p{\isasymin}Paths\ x{\isachardot}\ {\isasymforall}i{\isachardot}\ p\ i\ {\isasymnotin}\ lfp\ {\isacharparenleft}af\ A{\isacharparenright}\ {\isasymLongrightarrow}\isanewline
\isaindent{\ {\isadigit{2}}{\isachardot}\ {\isasymAnd}x{\isachardot}\ }{\isasymexists}p{\isasymin}Paths\ x{\isachardot}\ {\isasymforall}i{\isachardot}\ p\ i\ {\isasymnotin}\ A%
\end{isabelle}
Both are solved automatically:%
\end{isamarkuptxt}%
\ \isacommand{apply}{\isacharparenleft}auto\ dest{\isacharcolon}not{\isacharunderscore}in{\isacharunderscore}lfp{\isacharunderscore}afD{\isacharparenright}\isanewline
\isacommand{done}%
\begin{isamarkuptext}%
If you find these proofs too complicated, we recommend that you read
\S\ref{sec:CTL-revisited}, where we show how inductive definitions lead to
simpler arguments.

The main theorem is proved as for PDL, except that we also derive the
necessary equality \isa{lfp{\isacharparenleft}af\ A{\isacharparenright}\ {\isacharequal}\ {\isachardot}{\isachardot}{\isachardot}} by combining
\isa{AF{\isacharunderscore}lemma{\isadigit{1}}} and \isa{AF{\isacharunderscore}lemma{\isadigit{2}}} on the spot:%
\end{isamarkuptext}%
\isacommand{theorem}\ {\isachardoublequote}mc\ f\ {\isacharequal}\ {\isacharbraceleft}s{\isachardot}\ s\ {\isasymTurnstile}\ f{\isacharbraceright}{\isachardoublequote}\isanewline
\isacommand{apply}{\isacharparenleft}induct{\isacharunderscore}tac\ f{\isacharparenright}\isanewline
\isacommand{apply}{\isacharparenleft}auto\ simp\ add{\isacharcolon}\ EF{\isacharunderscore}lemma\ equalityI{\isacharbrackleft}OF\ AF{\isacharunderscore}lemma{\isadigit{1}}\ AF{\isacharunderscore}lemma{\isadigit{2}}{\isacharbrackright}{\isacharparenright}\isanewline
\isacommand{done}%
\begin{isamarkuptext}%
The language defined above is not quite CTL\@. The latter also includes an
until-operator \isa{EU\ f\ g} with semantics ``there \emph{E}xists a path
where \isa{f} is true \emph{U}ntil \isa{g} becomes true''.  We need
an auxiliary function:%
\end{isamarkuptext}%
\isacommand{consts}\ until{\isacharcolon}{\isacharcolon}\ {\isachardoublequote}state\ set\ {\isasymRightarrow}\ state\ set\ {\isasymRightarrow}\ state\ {\isasymRightarrow}\ state\ list\ {\isasymRightarrow}\ bool{\isachardoublequote}\isanewline
\isacommand{primrec}\isanewline
{\isachardoublequote}until\ A\ B\ s\ {\isacharbrackleft}{\isacharbrackright}\ \ \ \ {\isacharequal}\ {\isacharparenleft}s\ {\isasymin}\ B{\isacharparenright}{\isachardoublequote}\isanewline
{\isachardoublequote}until\ A\ B\ s\ {\isacharparenleft}t{\isacharhash}p{\isacharparenright}\ {\isacharequal}\ {\isacharparenleft}s\ {\isasymin}\ A\ {\isasymand}\ {\isacharparenleft}s{\isacharcomma}t{\isacharparenright}\ {\isasymin}\ M\ {\isasymand}\ until\ A\ B\ t\ p{\isacharparenright}{\isachardoublequote}%
\begin{isamarkuptext}%
\noindent
Expressing the semantics of \isa{EU} is now straightforward:
\begin{isabelle}%
\ \ \ \ \ s\ {\isasymTurnstile}\ EU\ f\ g\ {\isacharequal}\ {\isacharparenleft}{\isasymexists}p{\isachardot}\ until\ {\isacharbraceleft}t{\isachardot}\ t\ {\isasymTurnstile}\ f{\isacharbraceright}\ {\isacharbraceleft}t{\isachardot}\ t\ {\isasymTurnstile}\ g{\isacharbraceright}\ s\ p{\isacharparenright}%
\end{isabelle}
Note that \isa{EU} is not definable in terms of the other operators!

Model checking \isa{EU} is again a least fixed point construction:
\begin{isabelle}%
\ \ \ \ \ mc{\isacharparenleft}EU\ f\ g{\isacharparenright}\ {\isacharequal}\ lfp{\isacharparenleft}{\isasymlambda}T{\isachardot}\ mc\ g\ {\isasymunion}\ mc\ f\ {\isasyminter}\ {\isacharparenleft}M{\isasyminverse}\ {\isacharbackquote}{\isacharbackquote}\ T{\isacharparenright}{\isacharparenright}%
\end{isabelle}

\begin{exercise}
Extend the datatype of formulae by the above until operator
and prove the equivalence between semantics and model checking, i.e.\ that
\begin{isabelle}%
\ \ \ \ \ mc\ {\isacharparenleft}EU\ f\ g{\isacharparenright}\ {\isacharequal}\ {\isacharbraceleft}s{\isachardot}\ s\ {\isasymTurnstile}\ EU\ f\ g{\isacharbraceright}%
\end{isabelle}
%For readability you may want to annotate {term EU} with its customary syntax
%{text[display]"| EU formula formula    E[_ U _]"}
%which enables you to read and write {text"E[f U g]"} instead of {term"EU f g"}.
\end{exercise}
For more CTL exercises see, for example, Huth and Ryan \cite{Huth-Ryan-book}.%
\end{isamarkuptext}%
%
\begin{isamarkuptext}%
Let us close this section with a few words about the executability of our model checkers.
It is clear that if all sets are finite, they can be represented as lists and the usual
set operations are easily implemented. Only \isa{lfp} requires a little thought.
Fortunately, the HOL Library%
\footnote{See theory \isa{While_Combinator}.}
provides a theorem stating that 
in the case of finite sets and a monotone function~\isa{F},
the value of \mbox{\isa{lfp\ F}} can be computed by iterated application of \isa{F} to~\isa{{\isacharbraceleft}{\isacharbraceright}} until
a fixed point is reached. It is actually possible to generate executable functional programs
from HOL definitions, but that is beyond the scope of the tutorial.%
\index{CTL|)}%
\end{isamarkuptext}%
\end{isabellebody}%
%%% Local Variables:
%%% mode: latex
%%% TeX-master: "root"
%%% End:

\index{model checking example|)}

\endinput
