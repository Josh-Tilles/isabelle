
\chapter{Theorems and Forward Proof}

\section{*Sort hypotheses} \label{sec:sort-hyps}

\begin{ttbox} 
strip_shyps         : thm -> thm
strip_shyps_warning : thm -> thm
\end{ttbox}

Isabelle's type variables are decorated with sorts, constraining them to
certain ranges of types.  This has little impact when sorts only serve for
syntactic classification of types --- for example, FOL distinguishes between
terms and other types.  But when type classes are introduced through axioms,
this may result in some sorts becoming {\em empty\/}: where one cannot exhibit
a type belonging to it because certain sets of axioms are unsatisfiable.

If a theorem contains a type variable that is constrained by an empty
sort, then that theorem has no instances.  It is basically an instance
of {\em ex falso quodlibet}.  But what if it is used to prove another
theorem that no longer involves that sort?  The latter theorem holds
only if under an additional non-emptiness assumption.

Therefore, Isabelle's theorems carry around sort hypotheses.  The {\tt
shyps} field is a list of sorts occurring in type variables in the current
{\tt prop} and {\tt hyps} fields.  It may also includes sorts used in the
theorem's proof that no longer appear in the {\tt prop} or {\tt hyps}
fields --- so-called {\em dangling\/} sort constraints.  These are the
critical ones, asserting non-emptiness of the corresponding sorts.
 
Isabelle automatically removes extraneous sorts from the {\tt shyps} field at
the end of a proof, provided that non-emptiness can be established by looking
at the theorem's signature: from the {\tt classes} and {\tt arities}
information.  This operation is performed by \texttt{strip_shyps} and
\texttt{strip_shyps_warning}.

\begin{ttdescription}
  
\item[\ttindexbold{strip_shyps} $thm$] removes any extraneous sort hypotheses
  that can be witnessed from the type signature.
  
\item[\ttindexbold{strip_shyps_warning}] is like \texttt{strip_shyps}, but
  issues a warning message of any pending sort hypotheses that do not have a
  (syntactic) witness.

\end{ttdescription}


\section{Proof terms}\label{sec:proofObjects}
\index{proof terms|(} Isabelle can record the full meta-level proof of each
theorem.  The proof term contains all logical inferences in detail.
%while
%omitting bookkeeping steps that have no logical meaning to an outside
%observer.  Rewriting steps are recorded in similar detail as the output of
%simplifier tracing. 
Resolution and rewriting steps are broken down to primitive rules of the
meta-logic. The proof term can be inspected by a separate proof-checker,
for example.

According to the well-known {\em Curry-Howard isomorphism}, a proof can
be viewed as a $\lambda$-term. Following this idea, proofs
in Isabelle are internally represented by a datatype similar to the one for
terms described in \S\ref{sec:terms}.
\begin{ttbox}
infix 8 % %%;

datatype proof =
   PBound of int
 | Abst of string * typ option * proof
 | AbsP of string * term option * proof
 | op % of proof * term option
 | op %% of proof * proof
 | Hyp of term
 | PThm of (string * (string * string list) list) *
           proof * term * typ list option
 | PAxm of string * term * typ list option
 | Oracle of string * term * typ list option
 | MinProof of proof list;
\end{ttbox}

\begin{ttdescription}
\item[\ttindexbold{Abst} ($a$, $\tau$, $prf$)] is the abstraction over
a {\it term variable} of type $\tau$ in the body $prf$. Logically, this
corresponds to $\bigwedge$ introduction. The name $a$ is used only for
parsing and printing.
\item[\ttindexbold{AbsP} ($a$, $\varphi$, $prf$)] is the abstraction
over a {\it proof variable} standing for a proof of proposition $\varphi$
in the body $prf$. This corresponds to $\Longrightarrow$ introduction.
\item[$prf$ \% $t$] \index{\%@{\tt\%}|bold}
is the application of proof $prf$ to term $t$
which corresponds to $\bigwedge$ elimination.
\item[$prf@1$ \%\% $prf@2$] \index{\%\%@{\tt\%\%}|bold}
is the application of proof $prf@1$ to
proof $prf@2$ which corresponds to $\Longrightarrow$ elimination.
\item[\ttindexbold{PBound} $i$] is a {\em proof variable} with de Bruijn
\cite{debruijn72} index $i$.
\item[\ttindexbold{Hyp} $\varphi$] corresponds to the use of a meta level
hypothesis $\varphi$.
\item[\ttindexbold{PThm} (($name$, $tags$), $prf$, $\varphi$, $\overline{\tau}$)]
stands for a pre-proved theorem, where $name$ is the name of the theorem,
$prf$ is its actual proof, $\varphi$ is the proven proposition,
and $\overline{\tau}$ is
a type assignment for the type variables occurring in the proposition.
\item[\ttindexbold{PAxm} ($name$, $\varphi$, $\overline{\tau}$)]
corresponds to the use of an axiom with name $name$ and proposition
$\varphi$, where $\overline{\tau}$ is a type assignment for the type
variables occurring in the proposition.
\item[\ttindexbold{Oracle} ($name$, $\varphi$, $\overline{\tau}$)]
denotes the invocation of an oracle with name $name$ which produced
a proposition $\varphi$, where $\overline{\tau}$ is a type assignment
for the type variables occurring in the proposition.
\item[\ttindexbold{MinProof} $prfs$]
represents a {\em minimal proof} where $prfs$ is a list of theorems,
axioms or oracles.
\end{ttdescription}
Note that there are no separate constructors
for abstraction and application on the level of {\em types}, since
instantiation of type variables is accomplished via the type assignments
attached to {\tt Thm}, {\tt Axm} and {\tt Oracle}.

Each theorem's derivation is stored as the {\tt der} field of its internal
record: 
\begin{ttbox} 
#2 (#der (rep_thm conjI));
{\out PThm (("HOL.conjI", []),}
{\out   AbsP ("H", None, AbsP ("H", None, \dots)), \dots, None) %}
{\out     None % None : Proofterm.proof}
\end{ttbox}
This proof term identifies a labelled theorem, {\tt conjI} of theory
\texttt{HOL}, whose underlying proof is
{\tt AbsP ("H", None, AbsP ("H", None, $\dots$))}. 
The theorem is applied to two (implicit) term arguments, which correspond
to the two variables occurring in its proposition.

Isabelle's inference kernel can produce proof objects with different
levels of detail. This is controlled via the global reference variable
\ttindexbold{proofs}:
\begin{ttdescription}
\item[proofs := 0;] only record uses of oracles
\item[proofs := 1;] record uses of oracles as well as dependencies
  on other theorems and axioms
\item[proofs := 2;] record inferences in full detail
\end{ttdescription}
Reconstruction and checking of proofs as described in \S\ref{sec:reconstruct_proofs}
will not work for proofs constructed with {\tt proofs} set to
{\tt 0} or {\tt 1}.
Theorems involving oracles will be printed with a
suffixed \verb|[!]| to point out the different quality of confidence achieved.

\medskip

The dependencies of theorems can be viewed using the function
\ttindexbold{thm_deps}\index{theorems!dependencies}:
\begin{ttbox}
thm_deps [\(thm@1\), \(\ldots\), \(thm@n\)];
\end{ttbox}
generates the dependency graph of the theorems $thm@1$, $\ldots$, $thm@n$ and
displays it using Isabelle's graph browser. For this to work properly,
the theorems in question have to be proved with {\tt proofs} set to a value
greater than {\tt 0}. You can use
\begin{ttbox}
ThmDeps.enable : unit -> unit
ThmDeps.disable : unit -> unit
\end{ttbox}
to set \texttt{proofs} appropriately.

\subsection{Reconstructing and checking proof terms}\label{sec:reconstruct_proofs}
\index{proof terms!reconstructing}
\index{proof terms!checking}

When looking at the above datatype of proofs more closely, one notices that
some arguments of constructors are {\it optional}. The reason for this is that
keeping a full proof term for each theorem would result in enormous memory
requirements. Fortunately, typical proof terms usually contain quite a lot of
redundant information that can be reconstructed from the context. Therefore,
Isabelle's inference kernel creates only {\em partial} (or {\em implicit})
\index{proof terms!partial} proof terms, in which
all typing information in terms, all term and type labels of abstractions
{\tt AbsP} and {\tt Abst}, and (if possible) some argument terms of
\verb!%! are omitted. The following functions are available for
reconstructing and checking proof terms:
\begin{ttbox}
Reconstruct.reconstruct_proof :
  Sign.sg -> term -> Proofterm.proof -> Proofterm.proof
Reconstruct.expand_proof :
  Sign.sg -> string list -> Proofterm.proof -> Proofterm.proof
ProofChecker.thm_of_proof : theory -> Proofterm.proof -> thm
\end{ttbox}

\begin{ttdescription}
\item[Reconstruct.reconstruct_proof $sg$ $t$ $prf$]
turns the partial proof $prf$ into a full proof of the
proposition denoted by $t$, with respect to signature $sg$.
Reconstruction will fail with an error message if $prf$
is not a proof of $t$, is ill-formed, or does not contain
sufficient information for reconstruction by
{\em higher order pattern unification}
\cite{nipkow-patterns, Berghofer-Nipkow:2000:TPHOL}.
The latter may only happen for proofs
built up ``by hand'' but not for those produced automatically
by Isabelle's inference kernel.
\item[Reconstruct.expand_proof $sg$
  \ttlbrack$name@1$, $\ldots$, $name@n${\ttrbrack} $prf$]
expands and reconstructs the proofs of all theorems with names
$name@1$, $\ldots$, $name@n$ in the (full) proof $prf$.
\item[ProofChecker.thm_of_proof $thy$ $prf$] turns the (full) proof
$prf$ into a theorem with respect to theory $thy$ by replaying
it using only primitive rules from Isabelle's inference kernel.
\end{ttdescription}

\subsection{Parsing and printing proof terms}
\index{proof terms!parsing}
\index{proof terms!printing}

Isabelle offers several functions for parsing and printing
proof terms. The concrete syntax for proof terms is described
in Fig.\ts\ref{fig:proof_gram}.
Implicit term arguments in partial proofs are indicated
by ``{\tt _}''.
Type arguments for theorems and axioms may be specified using
\verb!%! or ``$\cdot$'' with an argument of the form {\tt TYPE($type$)}
(see \S\ref{sec:basic_syntax}).
They must appear before any other term argument of a theorem
or axiom. In contrast to term arguments, type arguments may
be completely omitted.
\begin{ttbox}
ProofSyntax.read_proof : theory -> bool -> string -> Proofterm.proof
ProofSyntax.pretty_proof : Sign.sg -> Proofterm.proof -> Pretty.T
ProofSyntax.pretty_proof_of : bool -> thm -> Pretty.T
ProofSyntax.print_proof_of : bool -> thm -> unit
\end{ttbox}
\begin{figure}
\begin{center}
\begin{tabular}{rcl}
$proof$  & $=$ & {\tt Lam} $params${\tt .} $proof$ ~~$|$~~
                 $\Lambda params${\tt .} $proof$ \\
         & $|$ & $proof$ \verb!%! $any$ ~~$|$~~
                 $proof$ $\cdot$ $any$ \\
         & $|$ & $proof$ \verb!%%! $proof$ ~~$|$~~
                 $proof$ {\boldmath$\cdot$} $proof$ \\
         & $|$ & $id$ ~~$|$~~ $longid$ \\\\
$param$  & $=$ & $idt$ ~~$|$~~ $idt$ {\tt :} $prop$ ~~$|$~~
                 {\tt (} $param$ {\tt )} \\\\
$params$ & $=$ & $param$ ~~$|$~~ $param$ $params$
\end{tabular}
\end{center}
\caption{Proof term syntax}\label{fig:proof_gram}
\end{figure}
The function {\tt read_proof} reads in a proof term with
respect to a given theory. The boolean flag indicates whether
the proof term to be parsed contains explicit typing information
to be taken into account.
Usually, typing information is left implicit and
is inferred during proof reconstruction. The pretty printing
functions operating on theorems take a boolean flag as an
argument which indicates whether the proof term should
be reconstructed before printing.

The following example (based on Isabelle/HOL) illustrates how
to parse and check proof terms. We start by parsing a partial
proof term
\begin{ttbox}
val prf = ProofSyntax.read_proof Main.thy false
  "impI % _ % _ %% (Lam H : _. conjE % _ % _ % _ %% H %%
     (Lam (H1 : _) H2 : _. conjI % _ % _ %% H2 %% H1))";
{\out val prf = PThm (("HOL.impI", []), \dots, \dots, None) % None % None %%}
{\out   AbsP ("H", None, PThm (("HOL.conjE", []), \dots, \dots, None) %}
{\out     None % None % None %% PBound 0 %%}
{\out     AbsP ("H1", None, AbsP ("H2", None, \dots))) : Proofterm.proof}
\end{ttbox}
The statement to be established by this proof is
\begin{ttbox}
val t = term_of
  (read_cterm (sign_of Main.thy) ("A & B --> B & A", propT));
{\out val t = Const ("Trueprop", "bool => prop") $}
{\out   (Const ("op -->", "[bool, bool] => bool") $}
{\out     \dots $ \dots : Term.term}
\end{ttbox}
Using {\tt t} we can reconstruct the full proof
\begin{ttbox}
val prf' = Reconstruct.reconstruct_proof (sign_of Main.thy) t prf;
{\out val prf' = PThm (("HOL.impI", []), \dots, \dots, Some []) %}
{\out   Some (Const ("op &", \dots) $ Free ("A", \dots) $ Free ("B", \dots)) %}
{\out   Some (Const ("op &", \dots) $ Free ("B", \dots) $ Free ("A", \dots)) %%}
{\out   AbsP ("H", Some (Const ("Trueprop", \dots) $ \dots), \dots)}
{\out     : Proofterm.proof}
\end{ttbox}
This proof can finally be turned into a theorem
\begin{ttbox}
val thm = ProofChecker.thm_of_proof Main.thy prf';
{\out val thm = "A & B --> B & A" : Thm.thm}
\end{ttbox}

\index{proof terms|)}
\index{theorems|)}


%%% Local Variables: 
%%% mode: latex
%%% TeX-master: "ref"
%%% End: 
