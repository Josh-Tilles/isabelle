\documentclass[12pt,a4paper]{report}
\usepackage{supertabular}
\usepackage{graphicx}
\usepackage{iman,extra,isar,ttbox}
\usepackage[nohyphen,strings]{underscore}
\usepackage{isabelle,isabellesym}
\usepackage{railsetup}
\usepackage{style}
\usepackage{pdfsetup}

\hyphenation{Isabelle}
\hyphenation{Isar}

\isadroptag{theory}

\isabellestyle{literal}

\title{\includegraphics[scale=0.5]{isabelle_jedit} \\[4ex] Isabelle/jEdit}

\author{\emph{Makarius Wenzel}}

\makeindex


\begin{document}

\maketitle

\begin{abstract}
  Isabelle/jEdit is a fully-featured Prover IDE, based on Isabelle/Scala and
  the jEdit text editor. This document provides an overview of general
  principles and its main IDE functionality.
\end{abstract}

\vspace*{2.5cm}

\begin{quote}
  {\small\em Isabelle's user interface is no advance over LCF's, which is
  widely condemned as ``user-unfriendly'': hard to use, bewildering to
  beginners. Hence the interest in proof editors, where a proof can be
  constructed and modified rule-by-rule using windows, mouse, and menus. But
  Edinburgh LCF was invented because real proofs require millions of
  inferences. Sophisticated tools --- rules, tactics and tacticals, the
  language ML, the logics themselves --- are hard to learn, yet they are
  essential. We may demand a mouse, but we need better education and
  training.}

  Lawrence C. Paulson, ``Isabelle: The Next 700 Theorem Provers''
\end{quote}


\vspace*{2.5cm}


\subsubsection*{Acknowledgements}

Research and implementation of concepts around PIDE and Isabelle/jEdit has
started around 2008 and was kindly supported by:
\begin{itemize}
\item TU M\"unchen \url{http://www.in.tum.de}
\item BMBF \url{http://www.bmbf.de}
\item Universit\'e Paris-Sud \url{http://www.u-psud.fr}
\item Digiteo \url{http://www.digiteo.fr}
\item ANR \url{http://www.agence-nationale-recherche.fr}
\end{itemize}


\pagenumbering{roman}
\tableofcontents
\listoffigures
\clearfirst

\input{JEdit.tex}

\begingroup
  \tocentry{\bibname}
  \bibliographystyle{abbrv} \small\raggedright\frenchspacing
  \bibliography{manual}
\endgroup

\tocentry{\indexname}
\printindex

\end{document}
