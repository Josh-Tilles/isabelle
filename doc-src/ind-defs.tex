\documentstyle[a4,proof,iman,extra,times]{llncs}
\newif\ifCADE
\CADEtrue

\title{A Fixedpoint Approach to Implementing\\ 
  (Co)Inductive Definitions\thanks{J. Grundy and S. Thompson made detailed
    comments; the referees were also helpful.  Research funded by
    SERC grants GR/G53279, GR/H40570 and by the ESPRIT Project 6453
    `Types'.}}

\author{Lawrence C. Paulson\\{\tt lcp@cl.cam.ac.uk}}
\institute{Computer Laboratory, University of Cambridge, England}
\date{\today} 
\setcounter{secnumdepth}{2} \setcounter{tocdepth}{2}

\newcommand\sbs{\subseteq}
\let\To=\Rightarrow


\newcommand\pow{{\cal P}}
%%%\let\pow=\wp
\newcommand\RepFun{\hbox{\tt RepFun}}
\newcommand\cons{\hbox{\tt cons}}
\def\succ{\hbox{\tt succ}}
\newcommand\split{\hbox{\tt split}}
\newcommand\fst{\hbox{\tt fst}}
\newcommand\snd{\hbox{\tt snd}}
\newcommand\converse{\hbox{\tt converse}}
\newcommand\domain{\hbox{\tt domain}}
\newcommand\range{\hbox{\tt range}}
\newcommand\field{\hbox{\tt field}}
\newcommand\lfp{\hbox{\tt lfp}}
\newcommand\gfp{\hbox{\tt gfp}}
\newcommand\id{\hbox{\tt id}}
\newcommand\trans{\hbox{\tt trans}}
\newcommand\wf{\hbox{\tt wf}}
\newcommand\nat{\hbox{\tt nat}}
\newcommand\rank{\hbox{\tt rank}}
\newcommand\univ{\hbox{\tt univ}}
\newcommand\Vrec{\hbox{\tt Vrec}}
\newcommand\Inl{\hbox{\tt Inl}}
\newcommand\Inr{\hbox{\tt Inr}}
\newcommand\case{\hbox{\tt case}}
\newcommand\lst{\hbox{\tt list}}
\newcommand\Nil{\hbox{\tt Nil}}
\newcommand\Cons{\hbox{\tt Cons}}
\newcommand\lstcase{\hbox{\tt list\_case}}
\newcommand\lstrec{\hbox{\tt list\_rec}}
\newcommand\length{\hbox{\tt length}}
\newcommand\listn{\hbox{\tt listn}}
\newcommand\acc{\hbox{\tt acc}}
\newcommand\primrec{\hbox{\tt primrec}}
\newcommand\SC{\hbox{\tt SC}}
\newcommand\CONST{\hbox{\tt CONST}}
\newcommand\PROJ{\hbox{\tt PROJ}}
\newcommand\COMP{\hbox{\tt COMP}}
\newcommand\PREC{\hbox{\tt PREC}}

\newcommand\quniv{\hbox{\tt quniv}}
\newcommand\llist{\hbox{\tt llist}}
\newcommand\LNil{\hbox{\tt LNil}}
\newcommand\LCons{\hbox{\tt LCons}}
\newcommand\lconst{\hbox{\tt lconst}}
\newcommand\lleq{\hbox{\tt lleq}}
\newcommand\map{\hbox{\tt map}}
\newcommand\term{\hbox{\tt term}}
\newcommand\Apply{\hbox{\tt Apply}}
\newcommand\termcase{\hbox{\tt term\_case}}
\newcommand\rev{\hbox{\tt rev}}
\newcommand\reflect{\hbox{\tt reflect}}
\newcommand\tree{\hbox{\tt tree}}
\newcommand\forest{\hbox{\tt forest}}
\newcommand\Part{\hbox{\tt Part}}
\newcommand\TF{\hbox{\tt tree\_forest}}
\newcommand\Tcons{\hbox{\tt Tcons}}
\newcommand\Fcons{\hbox{\tt Fcons}}
\newcommand\Fnil{\hbox{\tt Fnil}}
\newcommand\TFcase{\hbox{\tt TF\_case}}
\newcommand\Fin{\hbox{\tt Fin}}
\newcommand\QInl{\hbox{\tt QInl}}
\newcommand\QInr{\hbox{\tt QInr}}
\newcommand\qsplit{\hbox{\tt qsplit}}
\newcommand\qcase{\hbox{\tt qcase}}
\newcommand\Con{\hbox{\tt Con}}
\newcommand\data{\hbox{\tt data}}

\binperiod     %%%treat . like a binary operator

\begin{document}
%CADE%\pagestyle{empty}
%CADE%\begin{titlepage}
\maketitle 
\begin{abstract}
  This paper presents a fixedpoint approach to inductive definitions.
  Instead of using a syntactic test such as `strictly positive,' the
  approach lets definitions involve any operators that have been proved
  monotone.  It is conceptually simple, which has allowed the easy
  implementation of mutual recursion and other conveniences.  It also
  handles coinductive definitions: simply replace the least fixedpoint by a
  greatest fixedpoint.  This represents the first automated support for
  coinductive definitions.

  The method has been implemented in Isabelle's formalization of ZF set
  theory.  It should be applicable to any logic in which the Knaster-Tarski
  Theorem can be proved.  Examples include lists of $n$ elements, the
  accessible part of a relation and the set of primitive recursive
  functions.  One example of a coinductive definition is bisimulations for
  lazy lists.  \ifCADE\else Recursive datatypes are examined in detail, as
  well as one example of a {\bf codatatype}: lazy lists.  The appendices
  are simple user's manuals for this Isabelle/ZF package.\fi
\end{abstract}
%
%CADE%\bigskip\centerline{Copyright \copyright{} \number\year{} by Lawrence C. Paulson}
%CADE%\thispagestyle{empty} 
%CADE%\end{titlepage}
%CADE%\tableofcontents\cleardoublepage\pagestyle{headings}

\section{Introduction}
Several theorem provers provide commands for formalizing recursive data
structures, like lists and trees.  Examples include Boyer and Moore's shell
principle~\cite{bm79} and Melham's recursive type package for the HOL
system~\cite{melham89}.  Such data structures are called {\bf datatypes}
below, by analogy with {\tt datatype} definitions in Standard~ML\@.

A datatype is but one example of an {\bf inductive definition}.  This
specifies the least set closed under given rules~\cite{aczel77}.  The
collection of theorems in a logic is inductively defined.  A structural
operational semantics~\cite{hennessy90} is an inductive definition of a
reduction or evaluation relation on programs.  A few theorem provers
provide commands for formalizing inductive definitions; these include
Coq~\cite{paulin92} and again the HOL system~\cite{camilleri92}.

The dual notion is that of a {\bf coinductive definition}.  This specifies
the greatest set closed under given rules.  Important examples include
using bisimulation relations to formalize equivalence of
processes~\cite{milner89} or lazy functional programs~\cite{abramsky90}.
Other examples include lazy lists and other infinite data structures; these
are called {\bf codatatypes} below.

Not all inductive definitions are meaningful.  {\bf Monotone} inductive
definitions are a large, well-behaved class.  Monotonicity can be enforced
by syntactic conditions such as `strictly positive,' but this could lead to
monotone definitions being rejected on the grounds of their syntactic form.
More flexible is to formalize monotonicity within the logic and allow users
to prove it.

This paper describes a package based on a fixedpoint approach.  Least
fixedpoints yield inductive definitions; greatest fixedpoints yield
coinductive definitions.  The package has several advantages:
\begin{itemize}
\item It allows reference to any operators that have been proved monotone.
  Thus it accepts all provably monotone inductive definitions, including
  iterated definitions.
\item It accepts a wide class of datatype definitions, though at present
  restricted to finite branching.
\item It handles coinductive and codatatype definitions.  Most of
  the discussion below applies equally to inductive and coinductive
  definitions, and most of the code is shared.  To my knowledge, this is
  the only package supporting coinductive definitions.
\item Definitions may be mutually recursive.
\end{itemize}
The package is implemented in Isabelle~\cite{isabelle-intro}, using ZF set
theory \cite{paulson-set-I,paulson-set-II}.  However, the fixedpoint
approach is independent of Isabelle.  The recursion equations are specified
as introduction rules for the mutually recursive sets.  The package
transforms these rules into a mapping over sets, and attempts to prove that
the mapping is monotonic and well-typed.  If successful, the package
makes fixedpoint definitions and proves the introduction, elimination and
(co)induction rules.  The package consists of several Standard ML
functors~\cite{paulson91}; it accepts its argument and returns its result
as ML structures.\footnote{This use of ML modules is not essential; the
  package could also be implemented as a function on records.}

Most datatype packages equip the new datatype with some means of expressing
recursive functions.  This is the main omission from my package.  Its
fixedpoint operators define only recursive sets.  To define recursive
functions, the Isabelle/ZF theory provides well-founded recursion and other
logical tools~\cite{paulson-set-II}.

{\bf Outline.} Section~2 introduces the least and greatest fixedpoint
operators.  Section~3 discusses the form of introduction rules, mutual
recursion and other points common to inductive and coinductive definitions.
Section~4 discusses induction and coinduction rules separately.  Section~5
presents several examples, including a coinductive definition.  Section~6
describes datatype definitions.  Section~7 presents related work.
Section~8 draws brief conclusions.  \ifCADE\else The appendices are simple
user's manuals for this Isabelle/ZF package.\fi

Most of the definitions and theorems shown below have been generated by the
package.  I have renamed some variables to improve readability.
 
\section{Fixedpoint operators}
In set theory, the least and greatest fixedpoint operators are defined as
follows:
\begin{eqnarray*}
   \lfp(D,h)  & \equiv & \inter\{X\sbs D. h(X)\sbs X\} \\
   \gfp(D,h)  & \equiv & \union\{X\sbs D. X\sbs h(X)\}
\end{eqnarray*}   
Let $D$ be a set.  Say that $h$ is {\bf bounded by}~$D$ if $h(D)\sbs D$, and
{\bf monotone below~$D$} if
$h(A)\sbs h(B)$ for all $A$ and $B$ such that $A\sbs B\sbs D$.  If $h$ is
bounded by~$D$ and monotone then both operators yield fixedpoints:
\begin{eqnarray*}
   \lfp(D,h)  & = & h(\lfp(D,h)) \\
   \gfp(D,h)  & = & h(\gfp(D,h)) 
\end{eqnarray*}   
These equations are instances of the Knaster-Tarski Theorem, which states
that every monotonic function over a complete lattice has a
fixedpoint~\cite{davey&priestley}.  It is obvious from their definitions
that  $\lfp$ must be the least fixedpoint, and $\gfp$ the greatest.

This fixedpoint theory is simple.  The Knaster-Tarski Theorem is easy to
prove.  Showing monotonicity of~$h$ is trivial, in typical cases.  We must
also exhibit a bounding set~$D$ for~$h$.  Frequently this is trivial, as
when a set of `theorems' is (co)inductively defined over some previously
existing set of `formulae.'  Isabelle/ZF provides a suitable bounding set
for finitely branching (co)datatype definitions; see~\S\ref{univ-sec}
below.  Bounding sets are also called {\bf domains}.

The powerset operator is monotone, but by Cantor's Theorem there is no
set~$A$ such that $A=\pow(A)$.  We cannot put $A=\lfp(D,\pow)$ because
there is no suitable domain~$D$.  But \S\ref{acc-sec} demonstrates
that~$\pow$ is still useful in inductive definitions.

\section{Elements of an inductive or coinductive definition}\label{basic-sec}
Consider a (co)inductive definition of the sets $R_1$, \ldots,~$R_n$, in
mutual recursion.  They will be constructed from domains $D_1$,
\ldots,~$D_n$, respectively.  The construction yields not $R_i\sbs D_i$ but
$R_i\sbs D_1+\cdots+D_n$, where $R_i$ is contained in the image of~$D_i$
under an injection.  Reasons for this are discussed
elsewhere~\cite[\S4.5]{paulson-set-II}.

The definition may involve arbitrary parameters $\vec{p}=p_1$,
\ldots,~$p_k$.  Each recursive set then has the form $R_i(\vec{p})$.  The
parameters must be identical every time they occur within a definition.  This
would appear to be a serious restriction compared with other systems such as
Coq~\cite{paulin92}.  For instance, we cannot define the lists of
$n$ elements as the set $\listn(A,n)$ using rules where the parameter~$n$
varies.  Section~\ref{listn-sec} describes how to express this set using the
inductive definition package.

To avoid clutter below, the recursive sets are shown as simply $R_i$
instead of $R_i(\vec{p})$.

\subsection{The form of the introduction rules}\label{intro-sec}
The body of the definition consists of the desired introduction rules,
specified as strings.  The conclusion of each rule must have the form $t\in
R_i$, where $t$ is any term.  Premises typically have the same form, but
they can have the more general form $t\in M(R_i)$ or express arbitrary
side-conditions.

The premise $t\in M(R_i)$ is permitted if $M$ is a monotonic operator on
sets, satisfying the rule 
\[ \infer{M(A)\sbs M(B)}{A\sbs B} \]
The user must supply the package with monotonicity rules for all such premises.

The ability to introduce new monotone operators makes the approach
flexible.  A suitable choice of~$M$ and~$t$ can express a lot.  The
powerset operator $\pow$ is monotone, and the premise $t\in\pow(R)$
expresses $t\sbs R$; see \S\ref{acc-sec} for an example.  The `list of'
operator is monotone, as is easily proved by induction.  The premise
$t\in\lst(R)$ avoids having to encode the effect of~$\lst(R)$ using mutual
recursion; see \S\ref{primrec-sec} and also my earlier
paper~\cite[\S4.4]{paulson-set-II}.

Introduction rules may also contain {\bf side-conditions}.  These are
premises consisting of arbitrary formulae not mentioning the recursive
sets. Side-conditions typically involve type-checking.  One example is the
premise $a\in A$ in the following rule from the definition of lists:
\[ \infer{\Cons(a,l)\in\lst(A)}{a\in A & l\in\lst(A)} \]

\subsection{The fixedpoint definitions}
The package translates the list of desired introduction rules into a fixedpoint
definition.  Consider, as a running example, the finite powerset operator
$\Fin(A)$: the set of all finite subsets of~$A$.  It can be
defined as the least set closed under the rules
\[  \emptyset\in\Fin(A)  \qquad 
    \infer{\{a\}\un b\in\Fin(A)}{a\in A & b\in\Fin(A)} 
\]

The domain in a (co)inductive definition must be some existing set closed
under the rules.  A suitable domain for $\Fin(A)$ is $\pow(A)$, the set of all
subsets of~$A$.  The package generates the definition
\begin{eqnarray*}
  \Fin(A) & \equiv &  \lfp(\pow(A), \;
  \begin{array}[t]{r@{\,}l}
      \lambda X. \{z\in\pow(A). & z=\emptyset \disj{} \\
                  &(\exists a\,b. z=\{a\}\un b\conj a\in A\conj b\in X)\})
  \end{array}
\end{eqnarray*} 
The contribution of each rule to the definition of $\Fin(A)$ should be
obvious.  A coinductive definition is similar but uses $\gfp$ instead
of~$\lfp$.

The package must prove that the fixedpoint operator is applied to a
monotonic function.  If the introduction rules have the form described
above, and if the package is supplied a monotonicity theorem for every
$t\in M(R_i)$ premise, then this proof is trivial.\footnote{Due to the
  presence of logical connectives in the fixedpoint's body, the
  monotonicity proof requires some unusual rules.  These state that the
  connectives $\conj$, $\disj$ and $\exists$ preserve monotonicity with respect
  to the partial ordering on unary predicates given by $P\sqsubseteq Q$ if and
  only if $\forall x.P(x)\imp Q(x)$.}

The package returns its result as an ML structure, which consists of named
components; we may regard it as a record.  The result structure contains
the definitions of the recursive sets as a theorem list called {\tt defs}.
It also contains, as the theorem {\tt unfold}, a fixedpoint equation such
as
\begin{eqnarray*}
  \Fin(A) & = &
  \begin{array}[t]{r@{\,}l}
     \{z\in\pow(A). & z=\emptyset \disj{} \\
             &(\exists a\,b. z=\{a\}\un b\conj a\in A\conj b\in \Fin(A))\}
  \end{array}
\end{eqnarray*}
It also contains, as the theorem {\tt dom\_subset}, an inclusion such as 
$\Fin(A)\sbs\pow(A)$.


\subsection{Mutual recursion} \label{mutual-sec}
In a mutually recursive definition, the domain of the fixedpoint construction
is the disjoint sum of the domain~$D_i$ of each~$R_i$, for $i=1$,
\ldots,~$n$.  The package uses the injections of the
binary disjoint sum, typically $\Inl$ and~$\Inr$, to express injections
$h_{1n}$, \ldots, $h_{nn}$ for the $n$-ary disjoint sum $D_1+\cdots+D_n$.

As discussed elsewhere \cite[\S4.5]{paulson-set-II}, Isabelle/ZF defines the
operator $\Part$ to support mutual recursion.  The set $\Part(A,h)$
contains those elements of~$A$ having the form~$h(z)$:
\begin{eqnarray*}
   \Part(A,h)  & \equiv & \{x\in A. \exists z. x=h(z)\}.
\end{eqnarray*}   
For mutually recursive sets $R_1$, \ldots,~$R_n$ with
$n>1$, the package makes $n+1$ definitions.  The first defines a set $R$ using
a fixedpoint operator. The remaining $n$ definitions have the form
\begin{eqnarray*}
  R_i & \equiv & \Part(R,h_{in}), \qquad i=1,\ldots, n.
\end{eqnarray*} 
It follows that $R=R_1\un\cdots\un R_n$, where the $R_i$ are pairwise disjoint.


\subsection{Proving the introduction rules}
The user supplies the package with the desired form of the introduction
rules.  Once it has derived the theorem {\tt unfold}, it attempts
to prove those rules.  From the user's point of view, this is the
trickiest stage; the proofs often fail.  The task is to show that the domain 
$D_1+\cdots+D_n$ of the combined set $R_1\un\cdots\un R_n$ is
closed under all the introduction rules.  This essentially involves replacing
each~$R_i$ by $D_1+\cdots+D_n$ in each of the introduction rules and
attempting to prove the result.

Consider the $\Fin(A)$ example.  After substituting $\pow(A)$ for $\Fin(A)$
in the rules, the package must prove
\[  \emptyset\in\pow(A)  \qquad 
    \infer{\{a\}\un b\in\pow(A)}{a\in A & b\in\pow(A)} 
\]
Such proofs can be regarded as type-checking the definition.  The user
supplies the package with type-checking rules to apply.  Usually these are
general purpose rules from the ZF theory.  They could however be rules
specifically proved for a particular inductive definition; sometimes this is
the easiest way to get the definition through!

The result structure contains the introduction rules as the theorem list {\tt
intrs}.

\subsection{The case analysis rule}
The elimination rule, called {\tt elim}, performs case analysis.  There is one
case for each introduction rule.  The elimination rule
for $\Fin(A)$ is
\[ \infer{Q}{x\in\Fin(A) & \infer*{Q}{[x=\emptyset]}
                 & \infer*{Q}{[x=\{a\}\un b & a\in A &b\in\Fin(A)]_{a,b}} }
\]
The subscripted variables $a$ and~$b$ above the third premise are
eigenvariables, subject to the usual `not free in \ldots' proviso.
The rule states that if $x\in\Fin(A)$ then either $x=\emptyset$ or else
$x=\{a\}\un b$ for some $a\in A$ and $b\in\Fin(A)$; it is a simple consequence
of {\tt unfold}.

The package also returns a function for generating simplified instances of
the case analysis rule.  It works for datatypes and for inductive
definitions involving datatypes, such as an inductively defined relation
between lists.  It instantiates {\tt elim} with a user-supplied term then
simplifies the cases using freeness of the underlying datatype.  The
simplified rules perform `rule inversion' on the inductive definition.
Section~\S\ref{mkcases} presents an example.


\section{Induction and coinduction rules}
Here we must consider inductive and coinductive definitions separately.
For an inductive definition, the package returns an induction rule derived
directly from the properties of least fixedpoints, as well as a modified
rule for mutual recursion and inductively defined relations.  For a
coinductive definition, the package returns a basic coinduction rule.

\subsection{The basic induction rule}\label{basic-ind-sec}
The basic rule, called {\tt induct}, is appropriate in most situations.
For inductive definitions, it is strong rule induction~\cite{camilleri92}; for
datatype definitions (see below), it is just structural induction.  

The induction rule for an inductively defined set~$R$ has the following form.
The major premise is $x\in R$.  There is a minor premise for each
introduction rule:
\begin{itemize}
\item If the introduction rule concludes $t\in R_i$, then the minor premise
is~$P(t)$.

\item The minor premise's eigenvariables are precisely the introduction
rule's free variables that are not parameters of~$R$.  For instance, the
eigenvariables in the $\Fin(A)$ rule below are $a$ and $b$, but not~$A$.

\item If the introduction rule has a premise $t\in R_i$, then the minor
premise discharges the assumption $t\in R_i$ and the induction
hypothesis~$P(t)$.  If the introduction rule has a premise $t\in M(R_i)$
then the minor premise discharges the single assumption
\[ t\in M(\{z\in R_i. P(z)\}). \] 
Because $M$ is monotonic, this assumption implies $t\in M(R_i)$.  The
occurrence of $P$ gives the effect of an induction hypothesis, which may be
exploited by appealing to properties of~$M$.
\end{itemize}
The induction rule for $\Fin(A)$ resembles the elimination rule shown above,
but includes an induction hypothesis:
\[ \infer{P(x)}{x\in\Fin(A) & P(\emptyset)
        & \infer*{P(\{a\}\un b)}{[a\in A & b\in\Fin(A) & P(b)]_{a,b}} }
\] 
Stronger induction rules often suggest themselves.  We can derive a rule
for $\Fin(A)$ whose third premise discharges the extra assumption $a\not\in
b$.  The Isabelle/ZF theory defines the {\bf rank} of a
set~\cite[\S3.4]{paulson-set-II}, which supports well-founded induction and
recursion over datatypes.  The package proves a rule for mutual induction
and inductive relations.

\subsection{Mutual induction}
The mutual induction rule is called {\tt
mutual\_induct}.  It differs from the basic rule in several respects:
\begin{itemize}
\item Instead of a single predicate~$P$, it uses $n$ predicates $P_1$,
\ldots,~$P_n$: one for each recursive set.

\item There is no major premise such as $x\in R_i$.  Instead, the conclusion
refers to all the recursive sets:
\[ (\forall z.z\in R_1\imp P_1(z))\conj\cdots\conj
   (\forall z.z\in R_n\imp P_n(z))
\]
Proving the premises establishes $P_i(z)$ for $z\in R_i$ and $i=1$,
\ldots,~$n$.

\item If the domain of some $R_i$ is the Cartesian product
$A_1\times\cdots\times A_m$, then the corresponding predicate $P_i$ takes $m$
arguments and the corresponding conjunct of the conclusion is
\[ (\forall z_1\ldots z_m.\pair{z_1,\ldots,z_m}\in R_i\imp P_i(z_1,\ldots,z_m))
\]
\end{itemize}
The last point above simplifies reasoning about inductively defined
relations.  It eliminates the need to express properties of $z_1$,
\ldots,~$z_m$ as properties of the tuple $\pair{z_1,\ldots,z_m}$.

\subsection{Coinduction}\label{coind-sec}
A coinductive definition yields a primitive coinduction rule, with no
refinements such as those for the induction rules.  (Experience may suggest
refinements later.)  Consider the codatatype of lazy lists as an example.  For
suitable definitions of $\LNil$ and $\LCons$, lazy lists may be defined as the
greatest fixedpoint satisfying the rules
\[  \LNil\in\llist(A)  \qquad 
    \infer[(-)]{\LCons(a,l)\in\llist(A)}{a\in A & l\in\llist(A)}
\]
The $(-)$ tag stresses that this is a coinductive definition.  A suitable
domain for $\llist(A)$ is $\quniv(A)$, a set closed under variant forms of
sum and product for representing infinite data structures
(see~\S\ref{univ-sec}).  Coinductive definitions use these variant sums and
products.

The package derives an {\tt unfold} theorem similar to that for $\Fin(A)$. 
Then it proves the theorem {\tt coinduct}, which expresses that $\llist(A)$
is the greatest solution to this equation contained in $\quniv(A)$:
\[ \infer{x\in\llist(A)}{x\in X & X\sbs \quniv(A) &
    \infer*{z=\LNil\disj \bigl(\exists a\,l.\,
            z=\LCons(a,l) \conj a\in A \conj l\in X\un\llist(A) \bigr)}
           {[z\in X]_z}}
%     \begin{array}[t]{@{}l}
%       z=\LCons(a,l) \conj a\in A \conj{}\\
%       l\in X\un\llist(A) \bigr)
%     \end{array}  }{[z\in X]_z}}
\]
This rule complements the introduction rules; it provides a means of showing
$x\in\llist(A)$ when $x$ is infinite.  For instance, if $x=\LCons(0,x)$ then
applying the rule with $X=\{x\}$ proves $x\in\llist(\nat)$.  (Here $\nat$
is the set of natural numbers.)

Having $X\un\llist(A)$ instead of simply $X$ in the third premise above
represents a slight strengthening of the greatest fixedpoint property.  I
discuss several forms of coinduction rules elsewhere~\cite{paulson-coind}.


\section{Examples of inductive and coinductive definitions}\label{ind-eg-sec}
This section presents several examples: the finite powerset operator,
lists of $n$ elements, bisimulations on lazy lists, the well-founded part
of a relation, and the primitive recursive functions.

\subsection{The finite powerset operator}
This operator has been discussed extensively above.  Here
is the corresponding ML invocation (note that $\cons(a,b)$ abbreviates
$\{a\}\un b$ in Isabelle/ZF):
\begin{ttbox}
structure Fin = Inductive_Fun
 (val thy        = Arith.thy addconsts [(["Fin"],"i=>i")]
  val rec_doms   = [("Fin","Pow(A)")]
  val sintrs     = ["0 : Fin(A)",
                    "[| a: A;  b: Fin(A) |] ==> cons(a,b) : Fin(A)"]
  val monos      = []
  val con_defs   = []
  val type_intrs = [empty_subsetI, cons_subsetI, PowI]
  val type_elims = [make_elim PowD]);
\end{ttbox}
We apply the functor {\tt Inductive\_Fun} to a structure describing the
desired inductive definition.  The parent theory~{\tt thy} is obtained from
{\tt Arith.thy} by adding the unary function symbol~$\Fin$.  Its domain is
specified as $\pow(A)$, where $A$ is the parameter appearing in the
introduction rules.  For type-checking, the structure supplies introduction
rules:
\[ \emptyset\sbs A              \qquad
   \infer{\{a\}\un B\sbs C}{a\in C & B\sbs C}
\]
A further introduction rule and an elimination rule express the two
directions of the equivalence $A\in\pow(B)\bimp A\sbs B$.  Type-checking
involves mostly introduction rules.  

ML is Isabelle's top level, so such functor invocations can take place at
any time.  The result structure is declared with the name~{\tt Fin}; we can
refer to the $\Fin(A)$ introduction rules as {\tt Fin.intrs}, the induction
rule as {\tt Fin.induct} and so forth.  There are plans to integrate the
package better into Isabelle so that users can place inductive definitions
in Isabelle theory files instead of applying functors.


\subsection{Lists of $n$ elements}\label{listn-sec}
This has become a standard example of an inductive definition.  Following
Paulin-Mohring~\cite{paulin92}, we could attempt to define a new datatype
$\listn(A,n)$, for lists of length~$n$, as an $n$-indexed family of sets.
But her introduction rules
\[ \hbox{\tt Niln}\in\listn(A,0)  \qquad
   \infer{\hbox{\tt Consn}(n,a,l)\in\listn(A,\succ(n))}
         {n\in\nat & a\in A & l\in\listn(A,n)}
\]
are not acceptable to the inductive definition package:
$\listn$ occurs with three different parameter lists in the definition.

\begin{figure}
\begin{ttbox}
structure ListN = Inductive_Fun
 (val thy        = ListFn.thy addconsts [(["listn"],"i=>i")]
  val rec_doms   = [("listn", "nat*list(A)")]
  val sintrs     = 
        ["<0,Nil>: listn(A)",
         "[| a: A;  <n,l>: listn(A) |] ==> <succ(n), Cons(a,l)>: listn(A)"]
  val monos      = []
  val con_defs   = []
  val type_intrs = nat_typechecks @ List.intrs @ [SigmaI]
  val type_elims = [SigmaE2]);
\end{ttbox}
\hrule
\caption{Defining lists of $n$ elements} \label{listn-fig}
\end{figure} 

The Isabelle/ZF version of this example suggests a general treatment of
varying parameters.  Here, we use the existing datatype definition of
$\lst(A)$, with constructors $\Nil$ and~$\Cons$.  Then incorporate the
parameter~$n$ into the inductive set itself, defining $\listn(A)$ as a
relation.  It consists of pairs $\pair{n,l}$ such that $n\in\nat$
and~$l\in\lst(A)$ and $l$ has length~$n$.  In fact, $\listn(A)$ is the
converse of the length function on~$\lst(A)$.  The Isabelle/ZF introduction
rules are
\[ \pair{0,\Nil}\in\listn(A)  \qquad
   \infer{\pair{\succ(n),\Cons(a,l)}\in\listn(A)}
         {a\in A & \pair{n,l}\in\listn(A)}
\]
Figure~\ref{listn-fig} presents the ML invocation.  A theory of lists,
extended with a declaration of $\listn$, is the parent theory.  The domain
is specified as $\nat\times\lst(A)$.  The type-checking rules include those
for 0, $\succ$, $\Nil$ and $\Cons$.  Because $\listn(A)$ is a set of pairs,
type-checking also requires introduction and elimination rules to express
both directions of the equivalence $\pair{a,b}\in A\times B \bimp a\in A
\conj b\in B$. 

The package returns introduction, elimination and induction rules for
$\listn$.  The basic induction rule, {\tt ListN.induct}, is
\[ \infer{P(x)}{x\in\listn(A) & P(\pair{0,\Nil}) &
             \infer*{P(\pair{\succ(n),\Cons(a,l)})}
                {[a\in A & \pair{n,l}\in\listn(A) & P(\pair{n,l})]_{a,l,n}}}
\]
This rule requires the induction formula to be a 
unary property of pairs,~$P(\pair{n,l})$.  The alternative rule, {\tt
ListN.mutual\_induct}, uses a binary property instead:
\[ \infer{\forall n\,l. \pair{n,l}\in\listn(A) \imp P(n,l)}
         {P(0,\Nil) &
          \infer*{P(\succ(n),\Cons(a,l))}
                {[a\in A & \pair{n,l}\in\listn(A) & P(n,l)]_{a,l,n}}}
\]
It is now a simple matter to prove theorems about $\listn(A)$, such as
\[ \forall l\in\lst(A). \pair{\length(l),\, l}\in\listn(A) \]
\[ \listn(A)``\{n\} = \{l\in\lst(A). \length(l)=n\} \]
This latter result --- here $r``X$ denotes the image of $X$ under $r$
--- asserts that the inductive definition agrees with the obvious notion of
$n$-element list.  

Unlike in Coq, the definition does not declare a new datatype.  A `list of
$n$ elements' really is a list and is subject to list operators such
as append (concatenation).  For example, a trivial induction on
$\pair{m,l}\in\listn(A)$ yields
\[ \infer{\pair{m\mathbin{+} m,\, l@l'}\in\listn(A)}
         {\pair{m,l}\in\listn(A) & \pair{m',l'}\in\listn(A)} 
\]
where $+$ here denotes addition on the natural numbers and @ denotes append.

\subsection{A demonstration of rule inversion}\label{mkcases}
The elimination rule, {\tt ListN.elim}, is cumbersome:
\[ \infer{Q}{x\in\listn(A) & 
          \infer*{Q}{[x = \pair{0,\Nil}]} &
          \infer*{Q}
             {\left[\begin{array}{l}
               x = \pair{\succ(n),\Cons(a,l)} \\
               a\in A \\
               \pair{n,l}\in\listn(A)
               \end{array} \right]_{a,l,n}}}
\]
The ML function {\tt ListN.mk\_cases} generates simplified instances of
this rule.  It works by freeness reasoning on the list constructors:
$\Cons(a,l)$ is injective in its two arguments and differs from~$\Nil$.  If
$x$ is $\pair{i,\Nil}$ or $\pair{i,\Cons(a,l)}$ then {\tt ListN.mk\_cases}
deduces the corresponding form of~$i$;  this is called rule inversion.  For
example, 
\begin{ttbox}
ListN.mk_cases List.con_defs "<i,Cons(a,l)> : listn(A)"
\end{ttbox}
yields a rule with only two premises:
\[ \infer{Q}{\pair{i, \Cons(a,l)}\in\listn(A) & 
          \infer*{Q}
             {\left[\begin{array}{l}
               i = \succ(n) \\ a\in A \\ \pair{n,l}\in\listn(A)
               \end{array} \right]_{n}}}
\]
The package also has built-in rules for freeness reasoning about $0$
and~$\succ$.  So if $x$ is $\pair{0,l}$ or $\pair{\succ(i),l}$, then {\tt
ListN.mk\_cases} can similarly deduce the corresponding form of~$l$. 

The function {\tt mk\_cases} is also useful with datatype definitions.  The
instance from the definition of lists, namely {\tt List.mk\_cases}, can
prove the rule
\[ \infer{Q}{\Cons(a,l)\in\lst(A) & 
                 & \infer*{Q}{[a\in A &l\in\lst(A)]} }
\]
A typical use of {\tt mk\_cases} concerns inductive definitions of
evaluation relations.  Then rule inversion yields case analysis on possible
evaluations.  For example, the Isabelle/ZF theory includes a short proof
of the diamond property for parallel contraction on combinators.

\subsection{A coinductive definition: bisimulations on lazy lists}
This example anticipates the definition of the codatatype $\llist(A)$, which
consists of finite and infinite lists over~$A$.  Its constructors are $\LNil$
and
$\LCons$, satisfying the introduction rules shown in~\S\ref{coind-sec}.  
Because $\llist(A)$ is defined as a greatest fixedpoint and uses the variant
pairing and injection operators, it contains non-well-founded elements such as
solutions to $\LCons(a,l)=l$.

The next step in the development of lazy lists is to define a coinduction
principle for proving equalities.  This is done by showing that the equality
relation on lazy lists is the greatest fixedpoint of some monotonic
operation.  The usual approach~\cite{pitts94} is to define some notion of 
bisimulation for lazy lists, define equivalence to be the greatest
bisimulation, and finally to prove that two lazy lists are equivalent if and
only if they are equal.  The coinduction rule for equivalence then yields a
coinduction principle for equalities.

A binary relation $R$ on lazy lists is a {\bf bisimulation} provided $R\sbs
R^+$, where $R^+$ is the relation
\[ \{\pair{\LNil,\LNil}\} \un 
   \{\pair{\LCons(a,l),\LCons(a,l')} . a\in A \conj \pair{l,l'}\in R\}.
\]

A pair of lazy lists are {\bf equivalent} if they belong to some bisimulation. 
Equivalence can be coinductively defined as the greatest fixedpoint for the
introduction rules
\[  \pair{\LNil,\LNil} \in\lleq(A)  \qquad 
    \infer[(-)]{\pair{\LCons(a,l),\LCons(a,l')} \in\lleq(A)}
          {a\in A & \pair{l,l'}\in \lleq(A)}
\]
To make this coinductive definition, we invoke \verb|CoInductive_Fun|:
\begin{ttbox}
structure LList_Eq = CoInductive_Fun
 (val thy = LList.thy addconsts [(["lleq"],"i=>i")]
  val rec_doms   = [("lleq", "llist(A) * llist(A)")]
  val sintrs     = 
       ["<LNil, LNil> : lleq(A)",
        "[| a:A; <l,l'>: lleq(A) |] ==> <LCons(a,l),LCons(a,l')>: lleq(A)"]
  val monos      = []
  val con_defs   = []
  val type_intrs = LList.intrs @ [SigmaI]
  val type_elims = [SigmaE2]);
\end{ttbox}
Again, {\tt addconsts} declares a constant for $\lleq$ in the parent theory. 
The domain of $\lleq(A)$ is $\llist(A)\times\llist(A)$.  The type-checking
rules include the introduction rules for lazy lists as well as rules
for both directions of the equivalence
$\pair{a,b}\in A\times B \bimp a\in A \conj b\in B$.

The package returns the introduction rules and the elimination rule, as
usual.  But instead of induction rules, it returns a coinduction rule.
The rule is too big to display in the usual notation; its conclusion is
$x\in\lleq(A)$ and its premises are $x\in X$, 
${X\sbs\llist(A)\times\llist(A)}$ and
\[ \infer*{z=\pair{\LNil,\LNil}\disj \bigl(\exists a\,l\,l'.\,
      z=\pair{\LCons(a,l),\LCons(a,l')} \conj 
      a\in A \conj\pair{l,l'}\in X\un\lleq(A) \bigr)
%     \begin{array}[t]{@{}l}
%       z=\pair{\LCons(a,l),\LCons(a,l')} \conj a\in A \conj{}\\
%       \pair{l,l'}\in X\un\lleq(A) \bigr)
%     \end{array}  
    }{[z\in X]_z}
\]
Thus if $x\in X$, where $X$ is a bisimulation contained in the
domain of $\lleq(A)$, then $x\in\lleq(A)$.  It is easy to show that
$\lleq(A)$ is reflexive: the equality relation is a bisimulation.  And
$\lleq(A)$ is symmetric: its converse is a bisimulation.  But showing that
$\lleq(A)$ coincides with the equality relation takes some work.

\subsection{The accessible part of a relation}\label{acc-sec}
Let $\prec$ be a binary relation on~$D$; in short, $(\prec)\sbs D\times D$.
The {\bf accessible} or {\bf well-founded} part of~$\prec$, written
$\acc(\prec)$, is essentially that subset of~$D$ for which $\prec$ admits
no infinite decreasing chains~\cite{aczel77}.  Formally, $\acc(\prec)$ is
inductively defined to be the least set that contains $a$ if it contains
all $\prec$-predecessors of~$a$, for $a\in D$.  Thus we need an
introduction rule of the form 
\[ \infer{a\in\acc(\prec)}{\forall y.y\prec a\imp y\in\acc(\prec)} \]
Paulin-Mohring treats this example in Coq~\cite{paulin92}, but it causes
difficulties for other systems.  Its premise does not conform to 
the structure of introduction rules for HOL's inductive definition
package~\cite{camilleri92}.  It is also unacceptable to Isabelle package
(\S\ref{intro-sec}), but fortunately can be transformed into the acceptable
form $t\in M(R)$.

The powerset operator is monotonic, and $t\in\pow(R)$ is equivalent to
$t\sbs R$.  This in turn is equivalent to $\forall y\in t. y\in R$.  To
express $\forall y.y\prec a\imp y\in\acc(\prec)$ we need only find a
term~$t$ such that $y\in t$ if and only if $y\prec a$.  A suitable $t$ is
the inverse image of~$\{a\}$ under~$\prec$.

The ML invocation below follows this approach.  Here $r$ is~$\prec$ and
$\field(r)$ refers to~$D$, the domain of $\acc(r)$.  (The field of a
relation is the union of its domain and range.)  Finally
$r^{-}``\{a\}$ denotes the inverse image of~$\{a\}$ under~$r$.  The package is
supplied the theorem {\tt Pow\_mono}, which asserts that $\pow$ is monotonic.
\begin{ttbox}
structure Acc = Inductive_Fun
 (val thy        = WF.thy addconsts [(["acc"],"i=>i")]
  val rec_doms   = [("acc", "field(r)")]
  val sintrs     = ["[| r-``\{a\}:\,Pow(acc(r)); a:\,field(r) |] ==> a:\,acc(r)"]
  val monos      = [Pow_mono]
  val con_defs   = []
  val type_intrs = []
  val type_elims = []);
\end{ttbox}
The Isabelle theory proceeds to prove facts about $\acc(\prec)$.  For
instance, $\prec$ is well-founded if and only if its field is contained in
$\acc(\prec)$.  

As mentioned in~\S\ref{basic-ind-sec}, a premise of the form $t\in M(R)$
gives rise to an unusual induction hypothesis.  Let us examine the
induction rule, {\tt Acc.induct}:
\[ \infer{P(x)}{x\in\acc(r) &
     \infer*{P(a)}{[r^{-}``\{a\}\in\pow(\{z\in\acc(r).P(z)\}) & 
                   a\in\field(r)]_a}}
\]
The strange induction hypothesis is equivalent to
$\forall y. \pair{y,a}\in r\imp y\in\acc(r)\conj P(y)$.
Therefore the rule expresses well-founded induction on the accessible part
of~$\prec$.

The use of inverse image is not essential.  The Isabelle package can accept
introduction rules with arbitrary premises of the form $\forall
\vec{y}.P(\vec{y})\imp f(\vec{y})\in R$.  The premise can be expressed
equivalently as 
\[ \{z\in D. P(\vec{y}) \conj z=f(\vec{y})\} \in \pow(R) \] 
provided $f(\vec{y})\in D$ for all $\vec{y}$ such that~$P(\vec{y})$.  The
following section demonstrates another use of the premise $t\in M(R)$,
where $M=\lst$. 

\subsection{The primitive recursive functions}\label{primrec-sec}
The primitive recursive functions are traditionally defined inductively, as
a subset of the functions over the natural numbers.  One difficulty is that
functions of all arities are taken together, but this is easily
circumvented by regarding them as functions on lists.  Another difficulty,
the notion of composition, is less easily circumvented.

Here is a more precise definition.  Letting $\vec{x}$ abbreviate
$x_0,\ldots,x_{n-1}$, we can write lists such as $[\vec{x}]$,
$[y+1,\vec{x}]$, etc.  A function is {\bf primitive recursive} if it
belongs to the least set of functions in $\lst(\nat)\to\nat$ containing
\begin{itemize}
\item The {\bf successor} function $\SC$, such that $\SC[y,\vec{x}]=y+1$.
\item All {\bf constant} functions $\CONST(k)$, such that
  $\CONST(k)[\vec{x}]=k$. 
\item All {\bf projection} functions $\PROJ(i)$, such that
  $\PROJ(i)[\vec{x}]=x_i$ if $0\leq i<n$. 
\item All {\bf compositions} $\COMP(g,[f_0,\ldots,f_{m-1}])$, 
where $g$ and $f_0$, \ldots, $f_{m-1}$ are primitive recursive,
such that
\begin{eqnarray*}
  \COMP(g,[f_0,\ldots,f_{m-1}])[\vec{x}] & = & 
  g[f_0[\vec{x}],\ldots,f_{m-1}[\vec{x}]].
\end{eqnarray*} 

\item All {\bf recursions} $\PREC(f,g)$, where $f$ and $g$ are primitive
  recursive, such that
\begin{eqnarray*}
  \PREC(f,g)[0,\vec{x}] & = & f[\vec{x}] \\
  \PREC(f,g)[y+1,\vec{x}] & = & g[\PREC(f,g)[y,\vec{x}],\, y,\, \vec{x}].
\end{eqnarray*} 
\end{itemize}
Composition is awkward because it combines not two functions, as is usual,
but $m+1$ functions.  In her proof that Ackermann's function is not
primitive recursive, Nora Szasz was unable to formalize this definition
directly~\cite{szasz93}.  So she generalized primitive recursion to
tuple-valued functions.  This modified the inductive definition such that
each operation on primitive recursive functions combined just two functions.

\begin{figure}
\begin{ttbox}
structure Primrec = Inductive_Fun
 (val thy        = Primrec0.thy
  val rec_doms   = [("primrec", "list(nat)->nat")]
  val sintrs     = 
        ["SC : primrec",
         "k: nat ==> CONST(k) : primrec",
         "i: nat ==> PROJ(i) : primrec",
         "[| g: primrec; fs: list(primrec) |] ==> COMP(g,fs): primrec",
         "[| f: primrec; g: primrec |] ==> PREC(f,g): primrec"]
  val monos      = [list_mono]
  val con_defs   = [SC_def,CONST_def,PROJ_def,COMP_def,PREC_def]
  val type_intrs = pr0_typechecks
  val type_elims = []);
\end{ttbox}
\hrule
\caption{Inductive definition of the primitive recursive functions} 
\label{primrec-fig}
\end{figure}
\def\fs{{\it fs}} 
Szasz was using ALF, but Coq and HOL would also have problems accepting
this definition.  Isabelle's package accepts it easily since
$[f_0,\ldots,f_{m-1}]$ is a list of primitive recursive functions and
$\lst$ is monotonic.  There are five introduction rules, one for each of
the five forms of primitive recursive function.  Let us examine the one for
$\COMP$: 
\[ \infer{\COMP(g,\fs)\in\primrec}{g\in\primrec & \fs\in\lst(\primrec)} \]
The induction rule for $\primrec$ has one case for each introduction rule.
Due to the use of $\lst$ as a monotone operator, the composition case has
an unusual induction hypothesis:
 \[ \infer*{P(\COMP(g,\fs))}
          {[g\in\primrec & \fs\in\lst(\{z\in\primrec.P(z)\})]_{\fs,g}} \]
The hypothesis states that $\fs$ is a list of primitive recursive functions
satisfying the induction formula.  Proving the $\COMP$ case typically requires
structural induction on lists, yielding two subcases: either $\fs=\Nil$ or
else $\fs=\Cons(f,\fs')$, where $f\in\primrec$, $P(f)$, and $\fs'$ is
another list of primitive recursive functions satisfying~$P$.

Figure~\ref{primrec-fig} presents the ML invocation.  Theory {\tt
  Primrec0.thy} defines the constants $\SC$, $\CONST$, etc.  These are not
constructors of a new datatype, but functions over lists of numbers.  Their
definitions, which are omitted, consist of routine list programming.  In
Isabelle/ZF, the primitive recursive functions are defined as a subset of
the function set $\lst(\nat)\to\nat$.

The Isabelle theory goes on to formalize Ackermann's function and prove
that it is not primitive recursive, using the induction rule {\tt
  Primrec.induct}.  The proof follows Szasz's excellent account.


\section{Datatypes and codatatypes}\label{data-sec}
A (co)datatype definition is a (co)inductive definition with automatically
defined constructors and a case analysis operator.  The package proves that
the case operator inverts the constructors and can prove freeness theorems
involving any pair of constructors.


\subsection{Constructors and their domain}\label{univ-sec}
Conceptually, our two forms of definition are distinct.  A (co)inductive
definition selects a subset of an existing set; a (co)datatype definition
creates a new set.  But the package reduces the latter to the former.  A
set having strong closure properties must serve as the domain of the
(co)inductive definition.  Constructing this set requires some theoretical
effort, which must be done anyway to show that (co)datatypes exist.  It is
not obvious that standard set theory is suitable for defining codatatypes.

Isabelle/ZF defines the standard notion of Cartesian product $A\times B$,
containing ordered pairs $\pair{a,b}$.  Now the $m$-tuple
$\pair{x_1,\ldots,x_m}$ is the empty set~$\emptyset$ if $m=0$, simply
$x_1$ if $m=1$ and $\pair{x_1,\pair{x_2,\ldots,x_m}}$ if $m\geq2$.
Isabelle/ZF also defines the disjoint sum $A+B$, containing injections
$\Inl(a)\equiv\pair{0,a}$ and $\Inr(b)\equiv\pair{1,b}$.

A datatype constructor $\Con(x_1,\ldots,x_m)$ is defined to be
$h(\pair{x_1,\ldots,x_m})$, where $h$ is composed of $\Inl$ and~$\Inr$.
In a mutually recursive definition, all constructors for the set~$R_i$ have
the outer form~$h_{in}$, where $h_{in}$ is the injection described
in~\S\ref{mutual-sec}.  Further nested injections ensure that the
constructors for~$R_i$ are pairwise distinct.  

Isabelle/ZF defines the set $\univ(A)$, which contains~$A$ and
furthermore contains $\pair{a,b}$, $\Inl(a)$ and $\Inr(b)$ for $a$,
$b\in\univ(A)$.  In a typical datatype definition with set parameters
$A_1$, \ldots, $A_k$, a suitable domain for all the recursive sets is
$\univ(A_1\un\cdots\un A_k)$.  This solves the problem for
datatypes~\cite[\S4.2]{paulson-set-II}.

The standard pairs and injections can only yield well-founded
constructions.  This eases the (manual!) definition of recursive functions
over datatypes.  But they are unsuitable for codatatypes, which typically
contain non-well-founded objects.

To support codatatypes, Isabelle/ZF defines a variant notion of ordered
pair, written~$\pair{a;b}$.  It also defines the corresponding variant
notion of Cartesian product $A\otimes B$, variant injections $\QInl(a)$
and~$\QInr(b)$ and variant disjoint sum $A\oplus B$.  Finally it defines
the set $\quniv(A)$, which contains~$A$ and furthermore contains
$\pair{a;b}$, $\QInl(a)$ and $\QInr(b)$ for $a$, $b\in\quniv(A)$.  In a
typical codatatype definition with set parameters $A_1$, \ldots, $A_k$, a
suitable domain is $\quniv(A_1\un\cdots\un A_k)$.  This approach using
standard ZF set theory~\cite{paulson-final} is an alternative to adopting
Aczel's Anti-Foundation Axiom~\cite{aczel88}.

\subsection{The case analysis operator}
The (co)datatype package automatically defines a case analysis operator,
called {\tt$R$\_case}.  A mutually recursive definition still has only one
operator, whose name combines those of the recursive sets: it is called
{\tt$R_1$\_\ldots\_$R_n$\_case}.  The case operator is analogous to those
for products and sums.

Datatype definitions employ standard products and sums, whose operators are
$\split$ and $\case$ and satisfy the equations
\begin{eqnarray*}
  \split(f,\pair{x,y})  & = &  f(x,y) \\
  \case(f,g,\Inl(x))    & = &  f(x)   \\
  \case(f,g,\Inr(y))    & = &  g(y)
\end{eqnarray*}
Suppose the datatype has $k$ constructors $\Con_1$, \ldots,~$\Con_k$.  Then
its case operator takes $k+1$ arguments and satisfies an equation for each
constructor:
\begin{eqnarray*}
  R\hbox{\_case}(f_1,\ldots,f_k, {\tt Con}_i(\vec{x})) & = & f_i(\vec{x}),
    \qquad i = 1, \ldots, k
\end{eqnarray*}
The case operator's definition takes advantage of Isabelle's representation
of syntax in the typed $\lambda$-calculus; it could readily be adapted to a
theorem prover for higher-order logic.  If $f$ and~$g$ have meta-type
$i\To i$ then so do $\split(f)$ and
$\case(f,g)$.  This works because $\split$ and $\case$ operate on their last
argument.  They are easily combined to make complex case analysis
operators.  Here are two examples:
\begin{itemize}
\item $\split(\lambda x.\split(f(x)))$ performs case analysis for
$A\times (B\times C)$, as is easily verified:
\begin{eqnarray*}
  \split(\lambda x.\split(f(x)), \pair{a,b,c}) 
    & = & (\lambda x.\split(f(x))(a,\pair{b,c}) \\
    & = & \split(f(a), \pair{b,c}) \\
    & = & f(a,b,c)
\end{eqnarray*}

\item $\case(f,\case(g,h))$ performs case analysis for $A+(B+C)$; let us
verify one of the three equations:
\begin{eqnarray*}
  \case(f,\case(g,h), \Inr(\Inl(b))) 
    & = & \case(g,h,\Inl(b)) \\
    & = & g(b)
\end{eqnarray*}
\end{itemize}
Codatatype definitions are treated in precisely the same way.  They express
case operators using those for the variant products and sums, namely
$\qsplit$ and~$\qcase$.

\medskip

\ifCADE The package has processed all the datatypes discussed in
my earlier paper~\cite{paulson-set-II} and the codatatype of lazy lists.
Space limitations preclude discussing these examples here, but they are
distributed with Isabelle.  \typeout{****Omitting datatype examples from
  CADE version!} \else

To see how constructors and the case analysis operator are defined, let us
examine some examples.  These include lists and trees/forests, which I have
discussed extensively in another paper~\cite{paulson-set-II}.

\begin{figure}
\begin{ttbox} 
structure List = Datatype_Fun
 (val thy        = Univ.thy
  val rec_specs  = [("list", "univ(A)",
                      [(["Nil"],    "i"), 
                       (["Cons"],   "[i,i]=>i")])]
  val rec_styp   = "i=>i"
  val ext        = None
  val sintrs     = ["Nil : list(A)",
                    "[| a: A;  l: list(A) |] ==> Cons(a,l) : list(A)"]
  val monos      = []
  val type_intrs = datatype_intrs
  val type_elims = datatype_elims);
\end{ttbox}
\hrule
\caption{Defining the datatype of lists} \label{list-fig}

\medskip
\begin{ttbox}
structure LList = CoDatatype_Fun
 (val thy        = QUniv.thy
  val rec_specs  = [("llist", "quniv(A)",
                      [(["LNil"],   "i"), 
                       (["LCons"],  "[i,i]=>i")])]
  val rec_styp   = "i=>i"
  val ext        = None
  val sintrs     = ["LNil : llist(A)",
                    "[| a: A;  l: llist(A) |] ==> LCons(a,l) : llist(A)"]
  val monos      = []
  val type_intrs = codatatype_intrs
  val type_elims = codatatype_elims);
\end{ttbox}
\hrule
\caption{Defining the codatatype of lazy lists} \label{llist-fig}
\end{figure}

\subsection{Example: lists and lazy lists}
Figures \ref{list-fig} and~\ref{llist-fig} present the ML definitions of
lists and lazy lists, respectively.  They highlight the (many) similarities
and (few) differences between datatype and codatatype definitions.

Each form of list has two constructors, one for the empty list and one for
adding an element to a list.  Each takes a parameter, defining the set of
lists over a given set~$A$.  Each uses the appropriate domain from a
Isabelle/ZF theory:
\begin{itemize}
\item $\lst(A)$ specifies domain $\univ(A)$ and parent theory {\tt Univ.thy}.

\item $\llist(A)$ specifies domain $\quniv(A)$ and parent theory {\tt
QUniv.thy}.
\end{itemize}

Since $\lst(A)$ is a datatype, it enjoys a structural induction rule, {\tt
  List.induct}:
\[ \infer{P(x)}{x\in\lst(A) & P(\Nil)
        & \infer*{P(\Cons(a,l))}{[a\in A & l\in\lst(A) & P(l)]_{a,l}} }
\] 
Induction and freeness yield the law $l\not=\Cons(a,l)$.  To strengthen this,
Isabelle/ZF defines the rank of a set and proves that the standard pairs and
injections have greater rank than their components.  An immediate consequence,
which justifies structural recursion on lists \cite[\S4.3]{paulson-set-II},
is
\[ \rank(l) < \rank(\Cons(a,l)). \]

Since $\llist(A)$ is a codatatype, it has no induction rule.  Instead it has
the coinduction rule shown in \S\ref{coind-sec}.  Since variant pairs and
injections are monotonic and need not have greater rank than their
components, fixedpoint operators can create cyclic constructions.  For
example, the definition
\begin{eqnarray*}
  \lconst(a) & \equiv & \lfp(\univ(a), \lambda l. \LCons(a,l))
\end{eqnarray*}
yields $\lconst(a) = \LCons(a,\lconst(a))$.

\medskip
It may be instructive to examine the definitions of the constructors and
case operator for $\lst(A)$.  The definitions for $\llist(A)$ are similar.
The list constructors are defined as follows:
\begin{eqnarray*}
  \Nil       & = & \Inl(\emptyset) \\
  \Cons(a,l) & = & \Inr(\pair{a,l})
\end{eqnarray*}
The operator $\lstcase$ performs case analysis on these two alternatives:
\begin{eqnarray*}
  \lstcase(c,h) & \equiv & \case(\lambda u.c, \split(h)) 
\end{eqnarray*}
Let us verify the two equations:
\begin{eqnarray*}
    \lstcase(c, h, \Nil) & = & 
       \case(\lambda u.c, \split(h), \Inl(\emptyset)) \\
     & = & (\lambda u.c)(\emptyset) \\
     & = & c\\[1ex]
    \lstcase(c, h, \Cons(x,y)) & = & 
       \case(\lambda u.c, \split(h), \Inr(\pair{x,y})) \\
     & = & \split(h, \pair{x,y}) \\
     & = & h(x,y)
\end{eqnarray*} 

\begin{figure}
\begin{ttbox}
structure TF = Datatype_Fun
 (val thy        = Univ.thy
  val rec_specs  = [("tree", "univ(A)",
                       [(["Tcons"],  "[i,i]=>i")]),
                    ("forest", "univ(A)",
                       [(["Fnil"],   "i"),
                        (["Fcons"],  "[i,i]=>i")])]
  val rec_styp   = "i=>i"
  val ext        = None
  val sintrs     = 
        ["[| a:A;  f: forest(A) |] ==> Tcons(a,f) : tree(A)",
         "Fnil : forest(A)",
         "[| t: tree(A);  f: forest(A) |] ==> Fcons(t,f) : forest(A)"]
  val monos      = []
  val type_intrs = datatype_intrs
  val type_elims = datatype_elims);
\end{ttbox}
\hrule
\caption{Defining the datatype of trees and forests} \label{tf-fig}
\end{figure}


\subsection{Example: mutual recursion}
In mutually recursive trees and forests~\cite[\S4.5]{paulson-set-II}, trees
have the one constructor $\Tcons$, while forests have the two constructors
$\Fnil$ and~$\Fcons$.  Figure~\ref{tf-fig} presents the ML
definition.  It has much in common with that of $\lst(A)$, including its
use of $\univ(A)$ for the domain and {\tt Univ.thy} for the parent theory.
The three introduction rules define the mutual recursion.  The
distinguishing feature of this example is its two induction rules.

The basic induction rule is called {\tt TF.induct}:
\[ \infer{P(x)}{x\in\TF(A) & 
     \infer*{P(\Tcons(a,f))}
        {\left[\begin{array}{l} a\in A \\ 
                                f\in\forest(A) \\ P(f)
               \end{array}
         \right]_{a,f}}
     & P(\Fnil)
     & \infer*{P(\Fcons(t,f))}
        {\left[\begin{array}{l} t\in\tree(A)   \\ P(t) \\
                                f\in\forest(A) \\ P(f)
                \end{array}
         \right]_{t,f}} }
\] 
This rule establishes a single predicate for $\TF(A)$, the union of the
recursive sets.  

Although such reasoning is sometimes useful
\cite[\S4.5]{paulson-set-II}, a proper mutual induction rule should establish
separate predicates for $\tree(A)$ and $\forest(A)$.   The package calls this
rule {\tt TF.mutual\_induct}.  Observe the usage of $P$ and $Q$ in the
induction hypotheses:
\[ \infer{(\forall z. z\in\tree(A)\imp P(z)) \conj
          (\forall z. z\in\forest(A)\imp Q(z))}
     {\infer*{P(\Tcons(a,f))}
        {\left[\begin{array}{l} a\in A \\ 
                                f\in\forest(A) \\ Q(f)
               \end{array}
         \right]_{a,f}}
     & Q(\Fnil)
     & \infer*{Q(\Fcons(t,f))}
        {\left[\begin{array}{l} t\in\tree(A)   \\ P(t) \\
                                f\in\forest(A) \\ Q(f)
                \end{array}
         \right]_{t,f}} }
\] 
As mentioned above, the package does not define a structural recursion
operator.  I have described elsewhere how this is done
\cite[\S4.5]{paulson-set-II}.

Both forest constructors have the form $\Inr(\cdots)$,
while the tree constructor has the form $\Inl(\cdots)$.  This pattern would
hold regardless of how many tree or forest constructors there were.
\begin{eqnarray*}
  \Tcons(a,l)  & = & \Inl(\pair{a,l}) \\
  \Fnil        & = & \Inr(\Inl(\emptyset)) \\
  \Fcons(a,l)  & = & \Inr(\Inr(\pair{a,l}))
\end{eqnarray*} 
There is only one case operator; it works on the union of the trees and
forests:
\begin{eqnarray*}
  {\tt tree\_forest\_case}(f,c,g) & \equiv & 
    \case(\split(f),\, \case(\lambda u.c, \split(g)))
\end{eqnarray*}

\begin{figure}
\begin{ttbox}
structure Data = Datatype_Fun
 (val thy        = Univ.thy
  val rec_specs  = [("data", "univ(A Un B)",
                       [(["Con0"],   "i"),
                        (["Con1"],   "i=>i"),
                        (["Con2"],   "[i,i]=>i"),
                        (["Con3"],   "[i,i,i]=>i")])]
  val rec_styp   = "[i,i]=>i"
  val ext        = None
  val sintrs     = 
        ["Con0 : data(A,B)",
         "[| a: A |] ==> Con1(a) : data(A,B)",
         "[| a: A; b: B |] ==> Con2(a,b) : data(A,B)",
         "[| a: A; b: B;  d: data(A,B) |] ==> Con3(a,b,d) : data(A,B)"]
  val monos      = []
  val type_intrs = datatype_intrs
  val type_elims = datatype_elims);
\end{ttbox}
\hrule
\caption{Defining the four-constructor sample datatype} \label{data-fig}
\end{figure}

\subsection{A four-constructor datatype}
Finally let us consider a fairly general datatype.  It has four
constructors $\Con_0$, \ldots, $\Con_3$, with the
corresponding arities.  Figure~\ref{data-fig} presents the ML definition. 
Because this datatype has two set parameters, $A$ and~$B$, it specifies
$\univ(A\un B)$ as its domain.  The structural induction rule has four
minor premises, one per constructor:
\[ \infer{P(x)}{x\in\data(A,B) & 
    P(\Con_0) &
    \infer*{P(\Con_1(a))}{[a\in A]_a} &
    \infer*{P(\Con_2(a,b))}
      {\left[\begin{array}{l} a\in A \\ b\in B \end{array}
       \right]_{a,b}} &
    \infer*{P(\Con_3(a,b,d))}
      {\left[\begin{array}{l} a\in A \\ b\in B \\
                              d\in\data(A,B) \\ P(d)
              \end{array}
       \right]_{a,b,d}} }
\] 

The constructor definitions are
\begin{eqnarray*}
  \Con_0         & = & \Inl(\Inl(\emptyset)) \\
  \Con_1(a)      & = & \Inl(\Inr(a)) \\
  \Con_2(a,b)    & = & \Inr(\Inl(\pair{a,b})) \\
  \Con_3(a,b,c)  & = & \Inr(\Inr(\pair{a,b,c})).
\end{eqnarray*} 
The case operator is
\begin{eqnarray*}
  {\tt data\_case}(f_0,f_1,f_2,f_3) & \equiv & 
    \case(\begin{array}[t]{@{}l}
          \case(\lambda u.f_0,\; f_1),\, \\
          \case(\split(f_2),\; \split(\lambda v.\split(f_3(v)))) )
   \end{array} 
\end{eqnarray*}
This may look cryptic, but the case equations are trivial to verify.

In the constructor definitions, the injections are balanced.  A more naive
approach is to define $\Con_3(a,b,c)$ as
$\Inr(\Inr(\Inr(\pair{a,b,c})))$; instead, each constructor has two
injections.  The difference here is small.  But the ZF examples include a
60-element enumeration type, where each constructor has 5 or~6 injections.
The naive approach would require 1 to~59 injections; the definitions would be
quadratic in size.  It is like the difference between the binary and unary
numeral systems. 

The result structure contains the case operator and constructor definitions as
the theorem list \verb|con_defs|. It contains the case equations, such as 
\begin{eqnarray*}
  {\tt data\_case}(f_0,f_1,f_2,f_3,\Con_3(a,b,c)) & = & f_3(a,b,c),
\end{eqnarray*}
as the theorem list \verb|case_eqns|.  There is one equation per constructor.

\subsection{Proving freeness theorems}
There are two kinds of freeness theorems:
\begin{itemize}
\item {\bf injectiveness} theorems, such as
\[ \Con_2(a,b) = \Con_2(a',b') \bimp a=a' \conj b=b' \]

\item {\bf distinctness} theorems, such as
\[ \Con_1(a) \not= \Con_2(a',b')  \]
\end{itemize}
Since the number of such theorems is quadratic in the number of constructors,
the package does not attempt to prove them all.  Instead it returns tools for
proving desired theorems --- either explicitly or `on the fly' during
simplification or classical reasoning.

The theorem list \verb|free_iffs| enables the simplifier to perform freeness
reasoning.  This works by incremental unfolding of constructors that appear in
equations.  The theorem list contains logical equivalences such as
\begin{eqnarray*}
  \Con_0=c      & \bimp &  c=\Inl(\Inl(\emptyset))     \\
  \Con_1(a)=c   & \bimp &  c=\Inl(\Inr(a))             \\
                & \vdots &                             \\
  \Inl(a)=\Inl(b)   & \bimp &  a=b                     \\
  \Inl(a)=\Inr(b)   & \bimp &  {\tt False}             \\
  \pair{a,b} = \pair{a',b'} & \bimp & a=a' \conj b=b'
\end{eqnarray*}
For example, these rewrite $\Con_1(a)=\Con_1(b)$ to $a=b$ in four steps.

The theorem list \verb|free_SEs| enables the classical
reasoner to perform similar replacements.  It consists of elimination rules
to replace $\Con_0=c$ by $c=\Inl(\Inl(\emptyset))$ and so forth, in the
assumptions.

Such incremental unfolding combines freeness reasoning with other proof
steps.  It has the unfortunate side-effect of unfolding definitions of
constructors in contexts such as $\exists x.\Con_1(a)=x$, where they should
be left alone.  Calling the Isabelle tactic {\tt fold\_tac con\_defs}
restores the defined constants.
\fi  %CADE

\section{Related work}\label{related}
The use of least fixedpoints to express inductive definitions seems
obvious.  Why, then, has this technique so seldom been implemented?

Most automated logics can only express inductive definitions by asserting
new axioms.  Little would be left of Boyer and Moore's logic~\cite{bm79} if
their shell principle were removed.  With ALF the situation is more
complex; earlier versions of Martin-L\"of's type theory could (using
wellordering types) express datatype definitions, but the version
underlying ALF requires new rules for each definition~\cite{dybjer91}.
With Coq the situation is subtler still; its underlying Calculus of
Constructions can express inductive definitions~\cite{huet88}, but cannot
quite handle datatype definitions~\cite{paulin92}.  It seems that
researchers tried hard to circumvent these problems before finally
extending the Calculus with rule schemes for strictly positive operators.

Higher-order logic can express inductive definitions through quantification
over unary predicates.  The following formula expresses that~$i$ belongs to the
least set containing~0 and closed under~$\succ$:
\[ \forall P. P(0)\conj (\forall x.P(x)\imp P(\succ(x))) \imp P(i) \] 
This technique can be used to prove the Knaster-Tarski Theorem, but it is
little used in the HOL system.  Melham~\cite{melham89} clearly describes
the development.  The natural numbers are defined as shown above, but lists
are defined as functions over the natural numbers.  Unlabelled
trees are defined using G\"odel numbering; a labelled tree consists of an
unlabelled tree paired with a list of labels.  Melham's datatype package
expresses the user's datatypes in terms of labelled trees.  It has been
highly successful, but a fixedpoint approach would have yielded greater
functionality with less effort.

Melham's inductive definition package~\cite{camilleri92} uses
quantification over predicates, which is implicitly a fixedpoint approach.
Instead of formalizing the notion of monotone function, it requires
definitions to consist of finitary rules, a syntactic form that excludes
many monotone inductive definitions.

The earliest use of least fixedpoints is probably Robin Milner's datatype
package for Edinburgh LCF~\cite{milner-ind}.  Brian Monahan extended this
package considerably~\cite{monahan84}, as did I in unpublished
work.\footnote{The datatype package described in my LCF
  book~\cite{paulson87} does {\it not\/} make definitions, but merely
  asserts axioms.  I justified this shortcut on grounds of efficiency:
  existing packages took tens of minutes to run.  Such an explanation would
  not do today.}
LCF is a first-order logic of domain theory; the relevant fixedpoint
theorem is not Knaster-Tarski but concerns fixedpoints of continuous
functions over domains.  LCF is too weak to express recursive predicates.
Thus it would appear that the Isabelle/ZF package is the first to be based
on the Knaster-Tarski Theorem.


\section{Conclusions and future work}
Higher-order logic and set theory are both powerful enough to express
inductive definitions.  A growing number of theorem provers implement one
of these~\cite{IMPS,saaltink-fme}.  The easiest sort of inductive
definition package to write is one that asserts new axioms, not one that
makes definitions and proves theorems about them.  But asserting axioms
could introduce unsoundness.

The fixedpoint approach makes it fairly easy to implement a package for
(co)inductive definitions that does not assert axioms.  It is efficient: it
processes most definitions in seconds and even a 60-constructor datatype
requires only two minutes.  It is also simple: the package consists of
under 1100 lines (35K bytes) of Standard ML code.  The first working
version took under a week to code.

In set theory, care is required to ensure that the inductive definition
yields a set (rather than a proper class).  This problem is inherent to set
theory, whether or not the Knaster-Tarski Theorem is employed.  We must
exhibit a bounding set (called a domain above).  For inductive definitions,
this is often trivial.  For datatype definitions, I have had to formalize
much set theory.  I intend to formalize cardinal arithmetic and the
$\aleph$-sequence to handle datatype definitions that have infinite
branching.  The need for such efforts is not a drawback of the fixedpoint
approach, for the alternative is to take such definitions on faith.

The approach is not restricted to set theory.  It should be suitable for
any logic that has some notion of set and the Knaster-Tarski Theorem.  I
intend to use the Isabelle/ZF package as the basis for a higher-order logic
one, using Isabelle/HOL\@.  The necessary theory is already
mechanized~\cite{paulson-coind}.  HOL represents sets by unary predicates;
defining the corresponding types may cause complications.


\bibliographystyle{springer}
\bibliography{string-abbrv,atp,theory,funprog,isabelle}
%%%%%\doendnotes

\ifCADE\typeout{****Omitting appendices from CADE version!}
\else
\newpage
\appendix
\section{Inductive and coinductive definitions: users guide}
The ML functors \verb|Inductive_Fun| and \verb|CoInductive_Fun| build
inductive and coinductive definitions, respectively.  This section describes
how to invoke them.  

\subsection{The result structure}
Many of the result structure's components have been discussed
in~\S\ref{basic-sec}; others are self-explanatory.
\begin{description}
\item[\tt thy] is the new theory containing the recursive sets.

\item[\tt defs] is the list of definitions of the recursive sets.

\item[\tt bnd\_mono] is a monotonicity theorem for the fixedpoint operator.

\item[\tt unfold] is a fixedpoint equation for the recursive set (the union of
the recursive sets, in the case of mutual recursion).

\item[\tt dom\_subset] is a theorem stating inclusion in the domain.

\item[\tt intrs] is the list of introduction rules, now proved as theorems, for
the recursive sets.

\item[\tt elim] is the elimination rule.

\item[\tt mk\_cases] is a function to create simplified instances of {\tt
elim}, using freeness reasoning on some underlying datatype.
\end{description}

For an inductive definition, the result structure contains two induction rules,
{\tt induct} and \verb|mutual_induct|.  For a coinductive definition, it
contains the rule \verb|coinduct|.

Figure~\ref{def-result-fig} summarizes the two result signatures,
specifying the types of all these components.

\begin{figure}
\begin{ttbox}
sig
val thy          : theory
val defs         : thm list
val bnd_mono     : thm
val unfold       : thm
val dom_subset   : thm
val intrs        : thm list
val elim         : thm
val mk_cases     : thm list -> string -> thm
{\it(Inductive definitions only)} 
val induct       : thm
val mutual_induct: thm
{\it(Coinductive definitions only)}
val coinduct    : thm
end
\end{ttbox}
\hrule
\caption{The result of a (co)inductive definition} \label{def-result-fig}

\medskip
\begin{ttbox}
sig  
val thy          : theory
val rec_doms     : (string*string) list
val sintrs       : string list
val monos        : thm list
val con_defs     : thm list
val type_intrs   : thm list
val type_elims   : thm list
end
\end{ttbox}
\hrule
\caption{The argument of a (co)inductive definition} \label{def-arg-fig}
\end{figure}

\subsection{The argument structure}
Both \verb|Inductive_Fun| and \verb|CoInductive_Fun| take the same argument
structure (Figure~\ref{def-arg-fig}).  Its components are as follows:
\begin{description}
\item[\tt thy] is the definition's parent theory, which {\it must\/}
declare constants for the recursive sets.

\item[\tt rec\_doms] is a list of pairs, associating the name of each recursive
set with its domain.

\item[\tt sintrs] specifies the desired introduction rules as strings.

\item[\tt monos] consists of monotonicity theorems for each operator applied
to a recursive set in the introduction rules.

\item[\tt con\_defs] contains definitions of constants appearing in the
introduction rules.  The (co)datatype package supplies the constructors'
definitions here.  Most direct calls of \verb|Inductive_Fun| or
\verb|CoInductive_Fun| pass the empty list; one exception is the primitive
recursive functions example (\S\ref{primrec-sec}).

\item[\tt type\_intrs] consists of introduction rules for type-checking the
  definition, as discussed in~\S\ref{basic-sec}.  They are applied using
  depth-first search; you can trace the proof by setting
  \verb|trace_DEPTH_FIRST := true|.

\item[\tt type\_elims] consists of elimination rules for type-checking the
definition.  They are presumed to be `safe' and are applied as much as
possible, prior to the {\tt type\_intrs} search.
\end{description}
The package has a few notable restrictions:
\begin{itemize}
\item The parent theory, {\tt thy}, must declare the recursive sets as
  constants.  You can extend a theory with new constants using {\tt
    addconsts}, as illustrated in~\S\ref{ind-eg-sec}.  If the inductive
  definition also requires new concrete syntax, then it is simpler to
  express the parent theory using a theory file.  It is often convenient to
  define an infix syntax for relations, say $a\prec b$ for $\pair{a,b}\in
  R$.

\item The names of the recursive sets must be identifiers, not infix
operators.  

\item Side-conditions must not be conjunctions.  However, an introduction rule
may contain any number of side-conditions.
\end{itemize}


\section{Datatype and codatatype definitions: users guide}
The ML functors \verb|Datatype_Fun| and \verb|CoDatatype_Fun| define datatypes
and codatatypes, invoking \verb|Datatype_Fun| and
\verb|CoDatatype_Fun| to make the underlying (co)inductive definitions. 


\subsection{The result structure}
The result structure extends that of (co)inductive definitions
(Figure~\ref{def-result-fig}) with several additional items:
\begin{ttbox}
val con_thy   : theory
val con_defs  : thm list
val case_eqns : thm list
val free_iffs : thm list
val free_SEs  : thm list
val mk_free   : string -> thm
\end{ttbox}
Most of these have been discussed in~\S\ref{data-sec}.  Here is a summary:
\begin{description}
\item[\tt con\_thy] is a new theory containing definitions of the
(co)datatype's constructors and case operator.  It also declares the
recursive sets as constants, so that it may serve as the parent
theory for the (co)inductive definition.

\item[\tt con\_defs] is a list of definitions: the case operator followed by
the constructors.  This theorem list can be supplied to \verb|mk_cases|, for
example.

\item[\tt case\_eqns] is a list of equations, stating that the case operator
inverts each constructor.

\item[\tt free\_iffs] is a list of logical equivalences to perform freeness
reasoning by rewriting.  A typical application has the form
\begin{ttbox}
by (asm_simp_tac (ZF_ss addsimps free_iffs) 1);
\end{ttbox}

\item[\tt free\_SEs] is a list of `safe' elimination rules to perform freeness
reasoning.  It can be supplied to \verb|eresolve_tac| or to the classical
reasoner:
\begin{ttbox} 
by (fast_tac (ZF_cs addSEs free_SEs) 1);
\end{ttbox}

\item[\tt mk\_free] is a function to prove freeness properties, specified as
strings.  The theorems can be expressed in various forms, such as logical
equivalences or elimination rules.
\end{description}

The result structure also inherits everything from the underlying
(co)inductive definition, such as the introduction rules, elimination rule,
and (co)induction rule.


\begin{figure}
\begin{ttbox}
sig
val thy       : theory
val rec_specs : (string * string * (string list*string)list) list
val rec_styp  : string
val ext       : Syntax.sext option
val sintrs    : string list
val monos     : thm list
val type_intrs: thm list
val type_elims: thm list
end
\end{ttbox}
\hrule
\caption{The argument of a (co)datatype definition} \label{data-arg-fig}
\end{figure}

\subsection{The argument structure}
Both (co)datatype functors take the same argument structure
(Figure~\ref{data-arg-fig}).  It does not extend that for (co)inductive
definitions, but shares several components  and passes them uninterpreted to
\verb|Datatype_Fun| or
\verb|CoDatatype_Fun|.  The new components are as follows:
\begin{description}
\item[\tt thy] is the (co)datatype's parent theory. It {\it must not\/}
declare constants for the recursive sets.  Recall that (co)inductive
definitions have the opposite restriction.

\item[\tt rec\_specs] is a list of triples of the form ({\it recursive set\/},
{\it domain\/}, {\it constructors\/}) for each mutually recursive set.  {\it
Constructors\/} is a list of the form (names, type).  See the discussion and
examples in~\S\ref{data-sec}.

\item[\tt rec\_styp] is the common meta-type of the mutually recursive sets,
specified as a string.  They must all have the same type because all must
take the same parameters.

\item[\tt ext] is an optional syntax extension, usually omitted by writing
{\tt None}.  You can supply mixfix syntax for the constructors by supplying
\begin{ttbox}
Some (Syntax.simple_sext [{\it mixfix declarations\/}])
\end{ttbox}
\end{description}
The choice of domain is usually simple.  Isabelle/ZF defines the set
$\univ(A)$, which contains~$A$ and is closed under the standard Cartesian
products and disjoint sums \cite[\S4.2]{paulson-set-II}.  In a typical
datatype definition with set parameters $A_1$, \ldots, $A_k$, a suitable
domain for all the recursive sets is $\univ(A_1\un\cdots\un A_k)$.  For a
codatatype definition, the set
$\quniv(A)$ contains~$A$ and is closed under the variant Cartesian products
and disjoint sums; the appropriate domain is
$\quniv(A_1\un\cdots\un A_k)$.

The {\tt sintrs} specify the introduction rules, which govern the recursive
structure of the datatype.  Introduction rules may involve monotone
operators and side-conditions to express things that go beyond the usual
notion of datatype.  The theorem lists {\tt monos}, {\tt type\_intrs} and
{\tt type\_elims} should contain precisely what is needed for the
underlying (co)inductive definition.  Isabelle/ZF defines lists of
type-checking rules that can be supplied for the latter two components:
\begin{itemize}
\item {\tt datatype\_intrs} and {\tt datatype\_elims} are rules
for $\univ(A)$.
\item {\tt codatatype\_intrs} and {\tt codatatype\_elims} are
rules for $\quniv(A)$.
\end{itemize}
In typical definitions, these theorem lists need not be supplemented with
other theorems.

The constructor definitions' right-hand sides can overlap.  A
simple example is the datatype for the combinators, whose constructors are 
\begin{eqnarray*}
  {\tt K} & \equiv & \Inl(\emptyset) \\
  {\tt S} & \equiv & \Inr(\Inl(\emptyset)) \\
  p{\tt\#}q & \equiv & \Inr(\Inl(\pair{p,q})) 
\end{eqnarray*}
Unlike in previous versions of Isabelle, \verb|fold_tac| now ensures that the
longest right-hand sides are folded first.

\fi
\end{document}
