
\chapter{The Classical Reasoner}\label{chap:classical}
\index{classical reasoner|(}
\newcommand\ainfer[2]{\begin{array}{r@{\,}l}#2\\ \hline#1\end{array}}

\section{Classical rule sets}
\index{classical sets}

For elimination and destruction rules there are variants of the add operations
adding a rule in a way such that it is applied only if also its second premise
can be unified with an assumption of the current proof state:
\indexbold{*addSE2}\indexbold{*addSD2}\indexbold{*addE2}\indexbold{*addD2}
\begin{ttbox}
addSE2      : claset * (string * thm) -> claset           \hfill{\bf infix 4}
addSD2      : claset * (string * thm) -> claset           \hfill{\bf infix 4}
addE2       : claset * (string * thm) -> claset           \hfill{\bf infix 4}
addD2       : claset * (string * thm) -> claset           \hfill{\bf infix 4}
\end{ttbox}
\begin{warn}
  A rule to be added in this special way must be given a name, which is used 
  to delete it again -- when desired -- using \texttt{delSWrappers} or 
  \texttt{delWrappers}, respectively. This is because these add operations
  are implemented as wrappers (see \ref{sec:modifying-search} below).
\end{warn}


\subsection{Modifying the search step}
\label{sec:modifying-search}
For a given classical set, the proof strategy is simple.  Perform as many safe
inferences as possible; or else, apply certain safe rules, allowing
instantiation of unknowns; or else, apply an unsafe rule.  The tactics also
eliminate assumptions of the form $x=t$ by substitution if they have been set
up to do so (see \texttt{hyp_subst_tacs} in~{\S}\ref{sec:classical-setup} below).
They may perform a form of Modus Ponens: if there are assumptions $P\imp Q$
and~$P$, then replace $P\imp Q$ by~$Q$.

The classical reasoning tactics --- except \texttt{blast_tac}! --- allow
you to modify this basic proof strategy by applying two lists of arbitrary 
{\bf wrapper tacticals} to it. 
The first wrapper list, which is considered to contain safe wrappers only, 
affects \ttindex{safe_step_tac} and all the tactics that call it.  
The second one, which may contain unsafe wrappers, affects the unsafe parts
of \ttindex{step_tac}, \ttindex{slow_step_tac}, and the tactics that call them.
A wrapper transforms each step of the search, for example 
by attempting other tactics before or after the original step tactic. 
All members of a wrapper list are applied in turn to the respective step tactic.

Initially the two wrapper lists are empty, which means no modification of the
step tactics. Safe and unsafe wrappers are added to a claset 
with the functions given below, supplying them with wrapper names. 
These names may be used to selectively delete wrappers.

\begin{ttbox} 
type wrapper = (int -> tactic) -> (int -> tactic);

addSWrapper  : claset * (string *  wrapper       ) -> claset \hfill{\bf infix 4}
addSbefore   : claset * (string * (int -> tactic)) -> claset \hfill{\bf infix 4}
addSafter    : claset * (string * (int -> tactic)) -> claset \hfill{\bf infix 4}
delSWrapper  : claset *  string                    -> claset \hfill{\bf infix 4}

addWrapper   : claset * (string *  wrapper       ) -> claset \hfill{\bf infix 4}
addbefore    : claset * (string * (int -> tactic)) -> claset \hfill{\bf infix 4}
addafter     : claset * (string * (int -> tactic)) -> claset \hfill{\bf infix 4}
delWrapper   : claset *  string                    -> claset \hfill{\bf infix 4}

addSss       : claset * simpset -> claset                 \hfill{\bf infix 4}
addss        : claset * simpset -> claset                 \hfill{\bf infix 4}
\end{ttbox}
%

\begin{ttdescription}
\item[$cs$ addSWrapper $(name,wrapper)$] \indexbold{*addSWrapper}
adds a new wrapper, which should yield a safe tactic, 
to modify the existing safe step tactic.

\item[$cs$ addSbefore $(name,tac)$] \indexbold{*addSbefore}
adds the given tactic as a safe wrapper, such that it is tried 
{\em before} each safe step of the search.

\item[$cs$ addSafter $(name,tac)$] \indexbold{*addSafter}
adds the given tactic as a safe wrapper, such that it is tried 
when a safe step of the search would fail.

\item[$cs$ delSWrapper $name$] \indexbold{*delSWrapper}
deletes the safe wrapper with the given name.

\item[$cs$ addWrapper $(name,wrapper)$] \indexbold{*addWrapper}
adds a new wrapper to modify the existing (unsafe) step tactic.

\item[$cs$ addbefore $(name,tac)$] \indexbold{*addbefore}
adds the given tactic as an unsafe wrapper, such that it its result is 
concatenated {\em before} the result of each unsafe step.

\item[$cs$ addafter $(name,tac)$] \indexbold{*addafter}
adds the given tactic as an unsafe wrapper, such that it its result is 
concatenated {\em after} the result of each unsafe step.

\item[$cs$ delWrapper $name$] \indexbold{*delWrapper}
deletes the unsafe wrapper with the given name.

\item[$cs$ addSss $ss$] \indexbold{*addss}
adds the simpset~$ss$ to the classical set.  The assumptions and goal will be
simplified, in a rather safe way, after each safe step of the search.

\item[$cs$ addss $ss$] \indexbold{*addss}
adds the simpset~$ss$ to the classical set.  The assumptions and goal will be
simplified, before the each unsafe step of the search.

\end{ttdescription}

\index{simplification!from classical reasoner} 
Strictly speaking, the operators \texttt{addss} and \texttt{addSss}
are not part of the classical reasoner.
, which are used as primitives 
for the automatic tactics described in {\S}\ref{sec:automatic-tactics}, are
implemented as wrapper tacticals.
they  
\begin{warn}
Being defined as wrappers, these operators are inappropriate for adding more 
than one simpset at a time: the simpset added last overwrites any earlier ones.
When a simpset combined with a claset is to be augmented, this should done 
{\em before} combining it with the claset.
\end{warn}


\section{The classical tactics}

\subsection{Other classical tactics}
\begin{ttbox} 
slow_best_tac : claset -> int -> tactic
\end{ttbox}

\begin{ttdescription}
\item[\ttindexbold{slow_best_tac} $cs$ $i$] applies \texttt{slow_step_tac} with
best-first search to prove subgoal~$i$.
\end{ttdescription}


\subsection{Depth-limited automatic tactics}
\begin{ttbox} 
depth_tac  : claset -> int -> int -> tactic
deepen_tac : claset -> int -> int -> tactic
\end{ttbox}
These work by exhaustive search up to a specified depth.  Unsafe rules are
modified to preserve the formula they act on, so that it be used repeatedly.
They can prove more goals than \texttt{fast_tac} can but are much
slower, for example if the assumptions have many universal quantifiers.

The depth limits the number of unsafe steps.  If you can estimate the minimum
number of unsafe steps needed, supply this value as~$m$ to save time.
\begin{ttdescription}
\item[\ttindexbold{depth_tac} $cs$ $m$ $i$] 
tries to prove subgoal~$i$ by exhaustive search up to depth~$m$.

\item[\ttindexbold{deepen_tac} $cs$ $m$ $i$] 
tries to prove subgoal~$i$ by iterative deepening.  It calls \texttt{depth_tac}
repeatedly with increasing depths, starting with~$m$.
\end{ttdescription}


\subsection{Other useful tactics}
\index{tactics!for contradiction}
\index{tactics!for Modus Ponens}
\begin{ttbox} 
contr_tac    :             int -> tactic
mp_tac       :             int -> tactic
eq_mp_tac    :             int -> tactic
swap_res_tac : thm list -> int -> tactic
\end{ttbox}
These can be used in the body of a specialized search.
\begin{ttdescription}
\item[\ttindexbold{contr_tac} {\it i}]\index{assumptions!contradictory}
  solves subgoal~$i$ by detecting a contradiction among two assumptions of
  the form $P$ and~$\neg P$, or fail.  It may instantiate unknowns.  The
  tactic can produce multiple outcomes, enumerating all possible
  contradictions.

\item[\ttindexbold{mp_tac} {\it i}] 
is like \texttt{contr_tac}, but also attempts to perform Modus Ponens in
subgoal~$i$.  If there are assumptions $P\imp Q$ and~$P$, then it replaces
$P\imp Q$ by~$Q$.  It may instantiate unknowns.  It fails if it can do
nothing.

\item[\ttindexbold{eq_mp_tac} {\it i}] 
is like \texttt{mp_tac} {\it i}, but may not instantiate unknowns --- thus, it
is safe.

\item[\ttindexbold{swap_res_tac} {\it thms} {\it i}] refines subgoal~$i$ of
the proof state using {\it thms}, which should be a list of introduction
rules.  First, it attempts to prove the goal using \texttt{assume_tac} or
\texttt{contr_tac}.  It then attempts to apply each rule in turn, attempting
resolution and also elim-resolution with the swapped form.
\end{ttdescription}


\section{Setting up the classical reasoner}\label{sec:classical-setup}
\index{classical reasoner!setting up}
Isabelle's classical object-logics, including \texttt{FOL} and \texttt{HOL}, 
have the classical reasoner already set up.  
When defining a new classical logic, you should set up the reasoner yourself.  
It consists of the \ML{} functor \ttindex{ClassicalFun}, which takes the 
argument signature \texttt{CLASSICAL_DATA}:
\begin{ttbox} 
signature CLASSICAL_DATA =
  sig
  val mp             : thm
  val not_elim       : thm
  val swap           : thm
  val sizef          : thm -> int
  val hyp_subst_tacs : (int -> tactic) list
  end;
\end{ttbox}
Thus, the functor requires the following items:
\begin{ttdescription}
\item[\tdxbold{mp}] should be the Modus Ponens rule
$\List{\Var{P}\imp\Var{Q};\; \Var{P}} \Imp \Var{Q}$.

\item[\tdxbold{not_elim}] should be the contradiction rule
$\List{\neg\Var{P};\; \Var{P}} \Imp \Var{R}$.

\item[\tdxbold{swap}] should be the swap rule
$\List{\neg \Var{P}; \; \neg \Var{R}\Imp \Var{P}} \Imp \Var{R}$.

\item[\ttindexbold{sizef}] is the heuristic function used for best-first
search.  It should estimate the size of the remaining subgoals.  A good
heuristic function is \ttindex{size_of_thm}, which measures the size of the
proof state.  Another size function might ignore certain subgoals (say,
those concerned with type-checking).  A heuristic function might simply
count the subgoals.

\item[\ttindexbold{hyp_subst_tacs}] is a list of tactics for substitution in
the hypotheses, typically created by \ttindex{HypsubstFun} (see
Chapter~\ref{substitution}).  This list can, of course, be empty.  The
tactics are assumed to be safe!
\end{ttdescription}
The functor is not at all sensitive to the formalization of the
object-logic.  It does not even examine the rules, but merely applies
them according to its fixed strategy.  The functor resides in {\tt
  Provers/classical.ML} in the Isabelle sources.

\index{classical reasoner|)}

%%% Local Variables: 
%%% mode: latex
%%% TeX-master: "ref"
%%% End: 
