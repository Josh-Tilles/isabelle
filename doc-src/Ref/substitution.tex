%% $Id$
\chapter{Substitution Tactics} \label{substitution}
\index{substitution|(}
Replacing equals by equals is a basic form of reasoning.  Isabelle supports
several kinds of equality reasoning.  {\bf Substitution} means to replace
free occurrences of~$t$ by~$u$ in a subgoal.  This is easily done, given an
equality $t=u$, provided the logic possesses the appropriate rule ---
unless you want to substitute even in the assumptions.  The tactic
\ttindex{hyp_subst_tac} performs substitution in the assumptions, but it
works via object-level implication, and therefore must be specially set up
for each suitable object-logic.

Substitution should not be confused with object-level {\bf rewriting}.
Given equalities of the form $t=u$, rewriting replaces instances of~$t$ by
corresponding instances of~$u$, and continues until it reaches a normal
form.  Substitution handles `one-off' replacements by particular
equalities, while rewriting handles general equalities.
Chapter~\ref{simp-chap} discusses Isabelle's rewriting tactics.


\section{Simple substitution}
\index{substitution!simple}
Many logics include a substitution rule of the form\indexbold{*subst}
$$ \List{\Var{a}=\Var{b}; \Var{P}(\Var{a})} \Imp 
   \Var{P}(\Var{b})  \eqno(subst)$$
In backward proof, this may seem difficult to use: the conclusion
$\Var{P}(\Var{b})$ admits far too many unifiers.  But, if the theorem {\tt
eqth} asserts $t=u$, then \hbox{\tt eqth RS subst} is the derived rule
\[ \Var{P}(t) \Imp \Var{P}(u). \]
Provided $u$ is not an unknown, resolution with this rule is
well-behaved.\footnote{Unifying $\Var{P}(u)$ with a formula~$Q$
expresses~$Q$ in terms of its dependence upon~$u$.  There are still $2^k$
unifiers, if $Q$ has $k$ occurrences of~$u$, but Isabelle ensures that
the first unifier includes all the occurrences.}  To replace $u$ by~$t$ in
subgoal~$i$, use
\begin{ttbox} 
resolve_tac [eqth RS subst] \(i\) {\it.}
\end{ttbox}
To replace $t$ by~$u$ in
subgoal~$i$, use
\begin{ttbox} 
resolve_tac [eqth RS ssubst] \(i\) {\it,}
\end{ttbox}
where \ttindexbold{ssubst} is the `swapped' substitution rule
$$ \List{\Var{a}=\Var{b}; \Var{P}(\Var{b})} \Imp 
   \Var{P}(\Var{a}).  \eqno(ssubst)$$
If \ttindex{sym} denotes the symmetry rule
\(\Var{a}=\Var{b}\Imp\Var{b}=\Var{a}\), then {\tt ssubst} is just
\hbox{\tt sym RS subst}.  Many logics with equality include the rules {\tt
subst} and {\tt ssubst}, as well as {\tt refl}, {\tt sym} and {\tt trans}
(for the usual equality laws).

Elim-resolution is well-behaved with assumptions of the form $t=u$.
To replace $u$ by~$t$ or $t$ by~$u$ in subgoal~$i$, use
\begin{ttbox} 
eresolve_tac [subst] \(i\)    {\it or}    eresolve_tac [ssubst] \(i\) {\it.}
\end{ttbox}


\section{Substitution in the hypotheses}
\index{substitution!in hypotheses}
Substitution rules, like other rules of natural deduction, do not affect
the assumptions.  This can be inconvenient.  Consider proving the subgoal
\[ \List{c=a; c=b} \Imp a=b. \]
Calling \hbox{\tt eresolve_tac [ssubst] \(i\)} simply discards the
assumption~$c=a$, since $c$ does not occur in~$a=b$.  Of course, we can
work out a solution.  First apply \hbox{\tt eresolve_tac [subst] \(i\)},
replacing~$a$ by~$c$:
\[ \List{c=b} \Imp c=b \]
Equality reasoning can be difficult, but this trivial proof requires
nothing more sophisticated than substitution in the assumptions.
Object-logics that include the rule~$(subst)$ provide a tactic for this
purpose:
\begin{ttbox} 
hyp_subst_tac : int -> tactic
\end{ttbox}
\begin{description}
\item[\ttindexbold{hyp_subst_tac} {\it i}] 
selects an equality assumption of the form $t=u$ or $u=t$, where $t$ is a
free variable or parameter.  Deleting this assumption, it replaces $t$
by~$u$ throughout subgoal~$i$, including the other assumptions.
\end{description}
The term being replaced must be a free variable or parameter.  Substitution
for constants is usually unhelpful, since they may appear in other
theorems.  For instance, the best way to use the assumption $0=1$ is to
contradict a theorem that states $0\not=1$, rather than to replace 0 by~1
in the subgoal!

Replacing a free variable causes similar problems if they appear in the
premises of a rule being derived --- the substitution affects object-level
assumptions, not meta-level assumptions.  For instance, replacing~$a$
by~$b$ could make the premise~$P(a)$ worthless.  To avoid this problem, call
\ttindex{cut_facts_tac} to insert the atomic premises as object-level
assumptions.


\section{Setting up {\tt hyp_subst_tac}} 
Many Isabelle object-logics, such as {\tt FOL}, {\tt HOL} and their
descendants, come with {\tt hyp_subst_tac} already defined.  A few others,
such as {\tt CTT}, do not support this tactic because they lack the
rule~$(subst)$.  When defining a new logic that includes a substitution
rule and implication, you must set up {\tt hyp_subst_tac} yourself.  It
is packaged as the \ML{} functor \ttindex{HypsubstFun}, which takes the
argument signature~\ttindexbold{HYPSUBST_DATA}:
\begin{ttbox} 
signature HYPSUBST_DATA =
  sig
  val subst      : thm
  val sym        : thm
  val rev_cut_eq : thm
  val imp_intr   : thm
  val rev_mp     : thm
  val dest_eq    : term -> term*term
  end;
\end{ttbox}
Thus, the functor requires the following items:
\begin{description}
\item[\ttindexbold{subst}] should be the substitution rule
$\List{\Var{a}=\Var{b};\; \Var{P}(\Var{a})} \Imp \Var{P}(\Var{b})$.

\item[\ttindexbold{sym}] should be the symmetry rule
$\Var{a}=\Var{b}\Imp\Var{b}=\Var{a}$.

\item[\ttindexbold{rev_cut_eq}] should have the form
$\List{\Var{a}=\Var{b};\; \Var{a}=\Var{b}\Imp\Var{R}} \Imp \Var{R}$.

\item[\ttindexbold{imp_intr}] should be the implies introduction
rule $(\Var{P}\Imp\Var{Q})\Imp \Var{P}\imp\Var{Q}$.

\item[\ttindexbold{rev_mp}] should be the ``reversed'' implies elimination
rule $\List{\Var{P};  \;\Var{P}\imp\Var{Q}} \Imp \Var{Q}$.

\item[\ttindexbold{dest_eq}] should return the pair~$(t,u)$ when
applied to the \ML{} term that represents~$t=u$.  For other terms, it
should raise exception~\ttindex{Match}.
\end{description}
The functor resides in {\tt Provers/hypsubst.ML} on the Isabelle
distribution directory.  It is not sensitive to the precise formalization
of the object-logic.  It is not concerned with the names of the equality
and implication symbols, or the types of formula and terms.  Coding the
function {\tt dest_eq} requires knowledge of Isabelle's representation of
terms.  For {\tt FOL} it is defined by
\begin{ttbox} 
fun dest_eq (Const("Trueprop",_) $ (Const("op =",_)$t$u)) = (t,u);
\end{ttbox}
Here {\tt Trueprop} is the coercion from type~$o$ to type~$prop$, while
\hbox{\tt op =} is the internal name of the infix operator~{\tt=}.
Pattern-matching expresses the function concisely, using wildcards~({\tt_})
to hide the types.

Given a subgoal of the form
\[ \List{P@1; \cdots ; t=u; \cdots ; P@n} \Imp Q, \]
\ttindexbold{hyp_subst_tac} locates a suitable equality
assumption and moves it to the last position using elim-resolution on {\tt
rev_cut_eq} (possibly re-orienting it using~{\tt sym}):
\[ \List{P@1; \cdots ; P@n; t=u} \Imp Q \]
Using $n$ calls of \hbox{\tt eresolve_tac [rev_mp]}, it creates the subgoal
\[ \List{t=u} \Imp P@1\imp \cdots \imp P@n \imp Q \]
By \hbox{\tt eresolve_tac [ssubst]}, it replaces~$t$ by~$u$ throughout:
\[ P'@1\imp \cdots \imp P'@n \imp Q' \]
Finally, using $n$ calls of \hbox{\tt resolve_tac [imp_intr]}, it restores
$P'@1$, \ldots, $P'@n$ as assumptions:
\[ \List{P'@n; \cdots ; P'@1} \Imp Q' \]

\index{substitution|)}
