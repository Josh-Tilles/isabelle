%% $Id$
\maketitle 
\pagenumbering{roman} \tableofcontents \clearfirst

\chapter{Introduction} 
\index{sessions|(}
This manual is a comprehensive description of Isabelle, including all
commands, functions and packages.  Please do not read it: it is intended
for reference.  It is not a tutorial!  The manual assumes
familiarity with the basic concepts explained in {\em Introduction to
Isabelle}.

Functions are organized by their purpose, by their operands (subgoals,
tactics, theorems), and by their usefulness.  In each section, basic
functions appear first, then advanced functions, and finally esoteric
functions.  When you are looking for a way of performing some task, scan
the Table of Contents for a relevant heading.

The Index provides an alphabetical listing.  Page numbers of definitions
are printed in {\bf bold}, passing references in Roman type.  Use the Index
when you are looking for the definition of a particular Isabelle function.

This manual contains few examples.  Many files of examples are distributed
with Isabelle, however; please experiment interactively.


\section{Basic interaction with Isabelle}
\index{sessions!saving|bold}\index{saving your work|bold}
Isabelle provides no means of storing theorems or proofs on files.
Theorems are simply part of the \ML{} state and are named by \ML{}
identifiers.  To save your work between sessions, you must save a copy of
the \ML{} image.  The procedure for doing so is compiler-dependent:
\begin{itemize}
\item At the end of a session, Poly/\ML{} saves the state, provided you have
created a database for your own use.  You can create a database by copying
an existing one, or by calling the Poly/\ML{} function {\tt make_database};
the latter course uses much less disc space.  Note that a Poly/\ML{}
database {\bf does not} save the contents of references, such as the
current state of a backward proof.

\item With New Jersey \ML{} you must save the state explicitly before
ending the session.  While Poly/\ML{} database can be small, a New Jersey
image occupies several megabytes.
\end{itemize}
See your \ML{} compiler's documentation for full instructions on saving the
state.

Saving the state is not enough.  Record, on a file, the top-level commands
that generate your theories and proofs.  Such a record allows you to replay
the proofs whenever required, for instance after making minor changes to
the axioms.  Ideally, your record will be intelligible to others as a
formal description of your work.

Since Isabelle's user interface is the \ML{} top level, some kind of window
support is essential.  One window displays the Isabelle session, while the
other displays a file --- your proof record --- being edited.  Some people
use a screen editor such as Emacs, which supports windows and can manage
interactive sessions.  Others prefer to use a workstation running the X Window
System.


\section{Ending a session}
\index{sessions!ending|bold}
\begin{ttbox} 
quit     : unit -> unit
commit   : unit -> unit \hfill{\bf Poly/ML only}
exportML : string -> bool \hfill{\bf New Jersey ML only}
\end{ttbox}
\begin{description}
\item[\ttindexbold{quit}();]  
aborts the Isabelle session, without saving the state.

\item[\ttindexbold{commit}();]  saves the current state in your
Poly/\ML{} database without finishing the session.  Values of reference
variables are lost, so never do this during an interactive proof!

\item[\ttindexbold{exportML} \tt"{\it file}";]  saves an
image of your session to the given {\it file}.
\end{description}

\begin{warn}
Typing control-D also finishes the session, but its effect is
compiler-dependent.  Poly/\ML{} will then save the state, if you have a
private database.  New Jersey \ML{} will discard the state!
\end{warn}


\section{Reading files of proofs and theories}
\index{files, reading|bold}
\begin{ttbox} 
cd           : string -> unit
use          : string -> unit
use_thy      : string -> unit
time_use     : string -> unit
time_use_thy : string -> unit
\end{ttbox}
\begin{description}
\item[\ttindexbold{cd} \tt"{\it dir}";]  changes to {\it dir\/} the default
directory for reading files.

\item[\ttindexbold{use} \tt"$file$";]  
reads the given {\it file} as input to the \ML{} session.  Reading a file
of Isabelle commands is the usual way of replaying a proof.

\item[\ttindexbold{use_thy} \tt"$tname$";]  
reads a theory definition from file {\it tname}{\tt.thy} and also reads
{\ML} commands from the file {\it tname}{\tt.ML}, if it exists.  See
Chapter~\ref{theories} for details.

\item[\ttindexbold{time_use} \tt"$file$";]  
performs {\tt use~"$file$"} and prints the total execution time.

\item[\ttindexbold{time_use_thy} \tt"$tname$";]  
performs {\tt use_thy "$tname$"} and prints the total execution time.
\end{description}


\section{Printing of terms and theorems}
\index{printing!flags|bold}
Isabelle's pretty printer is controlled by a number of parameters.

\subsection{Printing limits}
\begin{ttbox} 
Pretty.setdepth  : int -> unit
Pretty.setmargin : int -> unit
print_depth      : int -> unit
\end{ttbox}
These set limits for terminal output.

\begin{description}
\item[\ttindexbold{Pretty.setdepth} \(d\);]  tells
Isabelle's pretty printer to limit the printing depth to~$d$.  This affects
Isabelle's display of theorems and terms.  The default value is~0, which
permits printing to an arbitrary depth.  Useful values for $d$ are~10 and~20.

\item[\ttindexbold{Pretty.setmargin} \(m\);]  tells
Isabelle's pretty printer to assume a right margin (page width) of~$m$.
The initial margin is~80.

\item[\ttindexbold{print_depth} \(n\);]  limits
the printing depth of complex \ML{} values, such as theorems and terms.
This command affects the \ML{} top level and its effect is
compiler-dependent.  Typically $n$ should be less than~10.
\end{description}


\subsection{Printing of meta-level hypotheses}
\index{hypotheses!meta-level!printing of|bold}
\begin{ttbox} 
show_hyps: bool ref \hfill{\bf initially true}
\end{ttbox}
A theorem's hypotheses are normally displayed, since such dependence is
important.  If this information becomes too verbose, however, it can be
switched off;  each hypothesis is then displayed as a dot.
\begin{description}
\item[\ttindexbold{show_hyps} \tt:= true;]   
makes Isabelle show meta-level hypotheses when printing a theorem; setting
it to {\tt false} suppresses them.
\end{description}


\subsection{Printing of types and sorts}
\begin{ttbox} 
show_types: bool ref \hfill{\bf initially false}
show_sorts: bool ref \hfill{\bf initially false}
\end{ttbox}
These control Isabelle's display of types and sorts.  Normally terms are
printed without type and sorts because they are verbose.  Occasionally you
may require this information, say to discover why a polymorphic inference rule
fails to resolve with some goal.

\begin{description}
\item[\ttindexbold{show_types} \tt:= true;]\index{types}
makes Isabelle show types when printing a term or theorem.

\item[\ttindexbold{show_sorts} \tt:= true;]\index{sorts}
makes Isabelle show the sorts of type variables.  It has no effect unless
{\tt show_types} is~{\tt true}. 
\end{description}


\subsection{$\eta$-contraction before printing}
\begin{ttbox} 
eta_contract: bool ref \hfill{\bf initially false}
\end{ttbox}
The {\bf $\eta$-contraction law} asserts $(\lambda x.f(x))\equiv f$,
provided $x$ is not free in ~$f$.  It asserts {\bf extensionality} of
functions: $f\equiv g$ if $f(x)\equiv g(x)$ for all~$x$.  Higher-order
unification puts terms into a fully $\eta$-expanded form.  For example, if
$F$ has type $(\tau\To\tau)\To\tau$ then its expanded form is $\lambda
h.F(\lambda x.h(x))$.  By default, the user sees this expanded form.

\begin{description}
\item[\ttindexbold{eta_contract} \tt:= true;]
makes Isabelle perform $\eta$-contractions before printing, so that
$\lambda h.F(\lambda x.h(x))$ appears simply as~$F$.  The
distinction between a term and its $\eta$-expanded form occasionally
matters.
\end{description}


\section{Displaying exceptions as error messages}
\index{printing!exceptions|bold}\index{exceptions|bold}
\begin{ttbox} 
print_exn: exn -> 'a
\end{ttbox}
Certain Isabelle primitives, such as the forward proof functions {\tt RS}
and {\tt RSN}, are called both interactively and from programs.  They
indicate errors not by printing messages, but by raising exceptions.  For
interactive use, \ML's reporting of an uncaught exception is most
uninformative.

\begin{description}
\item[\ttindexbold{print_exn} $e$] 
displays the exception~$e$ in a readable manner, and then re-raises~$e$.
Typical usage is~\hbox{\tt \ldots{} handle e => print_exn e;}, where
\ldots{} is an expression that may raise an exception.

{\tt print_exn} can display the following common exceptions, which concern
types, terms, theorems and theories, respectively.  Each carries a message
and related information.
\begin{ttbox} 
exception TYPE   of string * typ list * term list
exception TERM   of string * term list
exception THM    of string * int * thm list
exception THEORY of string * theory list
\end{ttbox}
{\tt print_exn} calls \ttindex{prin} to print terms.  This obtains pretty
printing information from the proof state last stored in the subgoal
module, and will fail if no interactive proof has begun in the current
session.
\end{description}


\section{Shell scripts}
\index{shell scripts|bold}
The following files are distributed with Isabelle, and work under
Unix$^{\rm TM}$.  They can be executed as commands to the Unix shell.  Some
of them depend upon shell environment variables.
\begin{description}
\item[\ttindexbold{make-all} $switches$] 
compiles the Isabelle system, when executed on the source directory.
Environment variables specify which \ML{} compiler (and {\tt Makefile}s) to
use.  These variables, and the {\it switches}, are documented on the file
itself.

\item[\ttindexbold{teeinput} $program$] 
executes the {\it program}, while piping the standard input to a log file
designated by the \verb|$LISTEN| environment variable.  Normally the
program is Isabelle, and the log file receives a copy of all the Isabelle
commands.

\item[\ttindexbold{xlisten} $program$] 
  is a trivial `user interface' for the X Window System.  It creates two
  windows using {\tt xterm}.  One executes an interactive session via
  \hbox{\tt teeinput $program$}, while the other displays the log file.  To
  make a proof record, simply paste lines from the log file into an editor
  window.

\item[\ttindexbold{expandshort} $files$] 
  reads the {\it files\/} and replaces all occurrences of the shorthand
  commands {\tt br}, {\tt be}, {\tt brs}, {\tt bes}, etc., with commands
  like \hbox{\tt by (resolve_tac \ldots)}.  The commands should appear one
  per line.  The old versions of the files
  are renamed to have the suffix~\verb'~~'.
\end{description}

\index{sessions|)}
