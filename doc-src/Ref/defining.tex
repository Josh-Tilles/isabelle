%% $Id$
\chapter{Defining Logics} \label{Defining-Logics}
This chapter explains how to define new formal systems --- in particular,
their concrete syntax.  While Isabelle can be regarded as a theorem prover
for set theory, higher-order logic or the sequent calculus, its
distinguishing feature is support for the definition of new logics.

Isabelle logics are hierarchies of theories, which are described and
illustrated in
\iflabelundefined{sec:defining-theories}{{\em Introduction to Isabelle}}%
{\S\ref{sec:defining-theories}}.  That material, together with the theory
files provided in the examples directories, should suffice for all simple
applications.  The easiest way to define a new theory is by modifying a
copy of an existing theory.

This chapter documents the meta-logic syntax, mixfix declarations and
pretty printing.  The extended examples in \S\ref{sec:min_logics}
demonstrate the logical aspects of the definition of theories.


\section{Priority grammars} \label{sec:priority_grammars}
\index{priority grammars|(}

A context-free grammar contains a set of {\bf nonterminal symbols}, a set of
{\bf terminal symbols} and a set of {\bf productions}\index{productions}.
Productions have the form ${A=\gamma}$, where $A$ is a nonterminal and
$\gamma$ is a string of terminals and nonterminals.  One designated
nonterminal is called the {\bf start symbol}.  The language defined by the
grammar consists of all strings of terminals that can be derived from the
start symbol by applying productions as rewrite rules.

The syntax of an Isabelle logic is specified by a {\bf priority
  grammar}.\index{priorities} Each nonterminal is decorated by an integer
priority, as in~$A^{(p)}$.  A nonterminal $A^{(p)}$ in a derivation may be
rewritten using a production $A^{(q)} = \gamma$ only if~$p \le q$.  Any
priority grammar can be translated into a normal context free grammar by
introducing new nonterminals and productions.

Formally, a set of context free productions $G$ induces a derivation
relation $\longrightarrow@G$.  Let $\alpha$ and $\beta$ denote strings of
terminal or nonterminal symbols.  Then
\[ \alpha\, A^{(p)}\, \beta ~\longrightarrow@G~ \alpha\,\gamma\,\beta \]
if and only if $G$ contains some production $A^{(q)}=\gamma$ for~$p \le q$.

The following simple grammar for arithmetic expressions demonstrates how
binding power and associativity of operators can be enforced by priorities.
\begin{center}
\begin{tabular}{rclr}
  $A^{(9)}$ & = & {\tt0} \\
  $A^{(9)}$ & = & {\tt(} $A^{(0)}$ {\tt)} \\
  $A^{(0)}$ & = & $A^{(0)}$ {\tt+} $A^{(1)}$ \\
  $A^{(2)}$ & = & $A^{(3)}$ {\tt*} $A^{(2)}$ \\
  $A^{(3)}$ & = & {\tt-} $A^{(3)}$
\end{tabular}
\end{center}
The choice of priorities determines that {\tt -} binds tighter than {\tt *},
which binds tighter than {\tt +}.  Furthermore {\tt +} associates to the
left and {\tt *} to the right.

For clarity, grammars obey these conventions:
\begin{itemize}
\item All priorities must lie between~0 and \ttindex{max_pri}, which is a
  some fixed integer.  Sometimes {\tt max_pri} is written as $\infty$.
\item Priority 0 on the right-hand side and priority \ttindex{max_pri} on
  the left-hand side may be omitted.
\item The production $A^{(p)} = \alpha$ is written as $A = \alpha~(p)$; the
  priority of the left-hand side actually appears in a column on the far
  right.
\item Alternatives are separated by~$|$.
\item Repetition is indicated by dots~(\dots) in an informal but obvious
  way.
\end{itemize}

Using these conventions and assuming $\infty=9$, the grammar
takes the form
\begin{center}
\begin{tabular}{rclc}
$A$ & = & {\tt0} & \hspace*{4em} \\
 & $|$ & {\tt(} $A$ {\tt)} \\
 & $|$ & $A$ {\tt+} $A^{(1)}$ & (0) \\
 & $|$ & $A^{(3)}$ {\tt*} $A^{(2)}$ & (2) \\
 & $|$ & {\tt-} $A^{(3)}$ & (3)
\end{tabular}
\end{center}
\index{priority grammars|)}


\begin{figure}
\begin{center}
\begin{tabular}{rclc}
$any$ &=& $prop$ ~~$|$~~ $logic$ \\\\
$prop$ &=& {\tt(} $prop$ {\tt)} \\
     &$|$& $prop^{(4)}$ {\tt::} $type$ & (3) \\
     &$|$& {\tt PROP} $aprop$ \\
     &$|$& $any^{(3)}$ {\tt ==} $any^{(2)}$ & (2) \\
     &$|$& $any^{(3)}$ {\tt =?=} $any^{(2)}$ & (2) \\
     &$|$& $prop^{(2)}$ {\tt ==>} $prop^{(1)}$ & (1) \\
     &$|$& {\tt[|} $prop$ {\tt;} \dots {\tt;} $prop$ {\tt|]} {\tt==>} $prop^{(1)}$ & (1) \\
     &$|$& {\tt!!} $idts$ {\tt.} $prop$ & (0) \\
     &$|$& {\tt OFCLASS} {\tt(} $type$ {\tt,} $logic$ {\tt)} \\\\
$aprop$ &=& $id$ ~~$|$~~ $var$
    ~~$|$~~ $logic^{(\infty)}$ {\tt(} $any$ {\tt,} \dots {\tt,} $any$ {\tt)} \\\\
$logic$ &=& {\tt(} $logic$ {\tt)} \\
      &$|$& $logic^{(4)}$ {\tt::} $type$ & (3) \\
      &$|$& $id$ ~~$|$~~ $var$
    ~~$|$~~ $logic^{(\infty)}$ {\tt(} $any$ {\tt,} \dots {\tt,} $any$ {\tt)} \\
      &$|$& {\tt \%} $idts$ {\tt.} $any$ & (0) \\\\
$idts$ &=& $idt$ ~~$|$~~ $idt^{(1)}$ $idts$ \\\\
$idt$ &=& $id$ ~~$|$~~ {\tt(} $idt$ {\tt)} \\
    &$|$& $id$ {\tt ::} $type$ & (0) \\\\
$type$ &=& {\tt(} $type$ {\tt)} \\
     &$|$& $tid$ ~~$|$~~ $tvar$ ~~$|$~~ $tid$ {\tt::} $sort$
       ~~$|$~~ $tvar$ {\tt::} $sort$ \\
     &$|$& $id$ ~~$|$~~ $type^{(\infty)}$ $id$
                ~~$|$~~ {\tt(} $type$ {\tt,} \dots {\tt,} $type$ {\tt)} $id$ \\
     &$|$& $type^{(1)}$ {\tt =>} $type$ & (0) \\
     &$|$& {\tt[}  $type$ {\tt,} \dots {\tt,} $type$ {\tt]} {\tt=>} $type$&(0) \\\\
$sort$ &=& $id$ ~~$|$~~ {\tt\ttlbrace\ttrbrace}
                ~~$|$~~ {\tt\ttlbrace} $id$ {\tt,} \dots {\tt,} $id$ {\tt\ttrbrace}
\end{tabular}
\index{*PROP symbol}
\index{*== symbol}\index{*=?= symbol}\index{*==> symbol}
\index{*:: symbol}\index{*=> symbol}
\index{sort constraints}
%the index command: a percent is permitted, but braces must match!
\index symbol}
\index{{}@{\tt\ttlbrace} symbol}\index{{}@{\tt\ttrbrace} symbol}
\index{*[ symbol}\index{*] symbol}
\index{*"!"! symbol}
\index{*"["| symbol}
\index{*"|"] symbol}
\end{center}
\caption{Meta-logic syntax}\label{fig:pure_gram}
\end{figure}


\section{The Pure syntax} \label{sec:basic_syntax}
\index{syntax!Pure|(}

At the root of all object-logics lies the theory \thydx{Pure}.  It
contains, among many other things, the Pure syntax.  An informal account of
this basic syntax (types, terms and formulae) appears in
\iflabelundefined{sec:forward}{{\em Introduction to Isabelle}}%
{\S\ref{sec:forward}}.  A more precise description using a priority grammar
appears in Fig.\ts\ref{fig:pure_gram}.  It defines the following
nonterminals:
\begin{ttdescription}
  \item[\ndxbold{any}] denotes any term.

  \item[\ndxbold{prop}] denotes terms of type {\tt prop}.  These are formulae
    of the meta-logic.  Note that user constants of result type {\tt prop}
    (i.e.\ $c :: \ldots \To prop$) should always provide concrete syntax.
    Otherwise atomic propositions with head $c$ may be printed incorrectly.

  \item[\ndxbold{aprop}] denotes atomic propositions.

%% FIXME huh!?
%  These typically
%  include the judgement forms of the object-logic; its definition
%  introduces a meta-level predicate for each judgement form.

  \item[\ndxbold{logic}] denotes terms whose type belongs to class
    \cldx{logic}, excluding type \tydx{prop}.

  \item[\ndxbold{idts}] denotes a list of identifiers, possibly constrained
    by types.

  \item[\ndxbold{type}] denotes types of the meta-logic.

  \item[\ndxbold{sort}] denotes meta-level sorts.
\end{ttdescription}

\begin{warn}
  In {\tt idts}, note that \verb|x::nat y| is parsed as \verb|x::(nat y)|,
  treating {\tt y} like a type constructor applied to {\tt nat}.  The
  likely result is an error message.  To avoid this interpretation, use
  parentheses and write \verb|(x::nat) y|.
  \index{type constraints}\index{*:: symbol}

  Similarly, \verb|x::nat y::nat| is parsed as \verb|x::(nat y::nat)| and
  yields an error.  The correct form is \verb|(x::nat) (y::nat)|.
\end{warn}

\begin{warn}
  Type constraints bind very weakly. For example, \verb!x<y::nat! is normally
  parsed as \verb!(x<y)::nat!, unless \verb$<$ has priority of 3 or less, in
  which case the string is likely to be ambiguous. The correct form is
  \verb!x<(y::nat)!.
\end{warn}

\subsection{Logical types and default syntax}\label{logical-types}
\index{lambda calc@$\lambda$-calculus}

Isabelle's representation of mathematical languages is based on the
simply typed $\lambda$-calculus.  All logical types, namely those of
class \cldx{logic}, are automatically equipped with a basic syntax of
types, identifiers, variables, parentheses, $\lambda$-abstraction and
application.
\begin{warn}
  Isabelle combines the syntaxes for all types of class \cldx{logic} by
  mapping all those types to the single nonterminal $logic$.  Thus all
  productions of $logic$, in particular $id$, $var$ etc, become available.
\end{warn}


\subsection{Lexical matters}
The parser does not process input strings directly.  It operates on token
lists provided by Isabelle's \bfindex{lexer}.  There are two kinds of
tokens: \bfindex{delimiters} and \bfindex{name tokens}.

\index{reserved words}
Delimiters can be regarded as reserved words of the syntax.  You can
add new ones when extending theories.  In Fig.\ts\ref{fig:pure_gram} they
appear in typewriter font, for example {\tt ==}, {\tt =?=} and
{\tt PROP}\@.

Name tokens have a predefined syntax.  The lexer distinguishes six disjoint
classes of names: \rmindex{identifiers}, \rmindex{unknowns}, type
identifiers\index{type identifiers}, type unknowns\index{type unknowns},
\rmindex{numerals}, \rmindex{strings}. They are denoted by \ndxbold{id},
\ndxbold{var}, \ndxbold{tid}, \ndxbold{tvar}, \ndxbold{xnum}, \ndxbold{xstr},
respectively.  Typical examples are {\tt x}, {\tt ?x7}, {\tt 'a}, {\tt ?'a3},
{\tt \#42}, {\tt ''foo bar''}. Here is the precise syntax:
\begin{eqnarray*}
id        & =   & letter~quasiletter^* \\
var       & =   & \mbox{\tt ?}id ~~|~~ \mbox{\tt ?}id\mbox{\tt .}nat \\
tid       & =   & \mbox{\tt '}id \\
tvar      & =   & \mbox{\tt ?}tid ~~|~~
                  \mbox{\tt ?}tid\mbox{\tt .}nat \\
xnum      & =   & \mbox{\tt \#}nat ~~|~~ \mbox{\tt \#\char`\~}nat \\
xstr      & =   & \mbox{\tt ''\rm text\tt ''} \\[1ex]
letter    & =   & \mbox{one of {\tt a}\dots {\tt z} {\tt A}\dots {\tt Z}} \\
digit     & =   & \mbox{one of {\tt 0}\dots {\tt 9}} \\
quasiletter & =  & letter ~~|~~ digit ~~|~~ \mbox{\tt _} ~~|~~ \mbox{\tt '} \\
nat       & =   & digit^+
\end{eqnarray*}
The lexer repeatedly takes the maximal prefix of the input string that forms
a valid token.  A maximal prefix that is both a delimiter and a name is
treated as a delimiter.  Spaces, tabs, newlines and formfeeds are separators;
they never occur within tokens, except those of class $xstr$.

\medskip
Delimiters need not be separated by white space.  For example, if {\tt -}
is a delimiter but {\tt --} is not, then the string {\tt --} is treated as
two consecutive occurrences of the token~{\tt -}.  In contrast, \ML\
treats {\tt --} as a single symbolic name.  The consequence of Isabelle's
more liberal scheme is that the same string may be parsed in different ways
after extending the syntax: after adding {\tt --} as a delimiter, the input
{\tt --} is treated as a single token.

A \ndxbold{var} or \ndxbold{tvar} describes an unknown, which is internally
a pair of base name and index (\ML\ type \mltydx{indexname}).  These
components are either separated by a dot as in {\tt ?x.1} or {\tt ?x7.3} or
run together as in {\tt ?x1}.  The latter form is possible if the base name
does not end with digits.  If the index is 0, it may be dropped altogether:
{\tt ?x} abbreviates both {\tt ?x0} and {\tt ?x.0}.

Tokens of class $xnum$ or $xstr$ are not used by the meta-logic.
Object-logics may provide numerals and string constants by adding appropriate
productions and translation functions.

\medskip
Although name tokens are returned from the lexer rather than the parser, it
is more logical to regard them as nonterminals.  Delimiters, however, are
terminals; they are just syntactic sugar and contribute nothing to the
abstract syntax tree.


\subsection{*Inspecting the syntax}
\begin{ttbox}
syn_of              : theory -> Syntax.syntax
print_syntax        : theory -> unit
Syntax.print_syntax : Syntax.syntax -> unit
Syntax.print_gram   : Syntax.syntax -> unit
Syntax.print_trans  : Syntax.syntax -> unit
\end{ttbox}
The abstract type \mltydx{Syntax.syntax} allows manipulation of syntaxes
in \ML.  You can display values of this type by calling the following
functions:
\begin{ttdescription}
\item[\ttindexbold{syn_of} {\it thy}] returns the syntax of the Isabelle
  theory~{\it thy} as an \ML\ value.

\item[\ttindexbold{print_syntax} $thy$] displays the syntax part of $thy$
  using {\tt Syntax.print_syntax}.

\item[\ttindexbold{Syntax.print_syntax} {\it syn}] shows virtually all
  information contained in the syntax {\it syn}.  The displayed output can
  be large.  The following two functions are more selective.

\item[\ttindexbold{Syntax.print_gram} {\it syn}] shows the grammar part
  of~{\it syn}, namely the lexicon, logical types and productions.  These are
  discussed below.

\item[\ttindexbold{Syntax.print_trans} {\it syn}] shows the translation
  part of~{\it syn}, namely the constants, parse/print macros and
  parse/print translations.
\end{ttdescription}

Let us demonstrate these functions by inspecting Pure's syntax.  Even that
is too verbose to display in full.
\begin{ttbox}\index{*Pure theory}
Syntax.print_syntax (syn_of Pure.thy);
{\out lexicon: "!!" "\%" "(" ")" "," "." "::" ";" "==" "==>" \dots}
{\out logtypes: fun itself}
{\out prods:}
{\out   type = tid  (1000)}
{\out   type = tvar  (1000)}
{\out   type = id  (1000)}
{\out   type = tid "::" sort[0]  => "_ofsort" (1000)}
{\out   type = tvar "::" sort[0]  => "_ofsort" (1000)}
{\out   \vdots}
\ttbreak
{\out consts: "_K" "_appl" "_aprop" "_args" "_asms" "_bigimpl" \dots}
{\out parse_ast_translation: "_appl" "_bigimpl" "_bracket"}
{\out   "_idtyp" "_lambda" "_tapp" "_tappl"}
{\out parse_rules:}
{\out parse_translation: "!!" "_K" "_abs" "_aprop"}
{\out print_translation: "all"}
{\out print_rules:}
{\out print_ast_translation: "==>" "_abs" "_idts" "fun"}
\end{ttbox}

As you can see, the output is divided into labelled sections.  The grammar
is represented by {\tt lexicon}, {\tt logtypes} and {\tt prods}.  The rest
refers to syntactic translations and macro expansion.  Here is an
explanation of the various sections.
\begin{description}
  \item[{\tt lexicon}] lists the delimiters used for lexical
    analysis.\index{delimiters}

  \item[{\tt logtypes}] lists the types that are regarded the same as {\tt
    logic} syntactically. Thus types of object-logics (e.g.\ {\tt nat}, say)
    will be automatically equipped with the standard syntax of
    $\lambda$-calculus.

  \item[{\tt prods}] lists the \rmindex{productions} of the priority grammar.
    The nonterminal $A^{(n)}$ is rendered in {\sc ascii} as {\tt $A$[$n$]}.
    Each delimiter is quoted.  Some productions are shown with {\tt =>} and
    an attached string.  These strings later become the heads of parse
    trees; they also play a vital role when terms are printed (see
    \S\ref{sec:asts}).

    Productions with no strings attached are called {\bf copy
      productions}\indexbold{productions!copy}.  Their right-hand side must
    have exactly one nonterminal symbol (or name token).  The parser does
    not create a new parse tree node for copy productions, but simply
    returns the parse tree of the right-hand symbol.

    If the right-hand side consists of a single nonterminal with no
    delimiters, then the copy production is called a {\bf chain
      production}.  Chain productions act as abbreviations:
    conceptually, they are removed from the grammar by adding new
    productions.  Priority information attached to chain productions is
    ignored; only the dummy value $-1$ is displayed.

  \item[{\tt consts}, {\tt parse_rules}, {\tt print_rules}]
    relate to macros (see \S\ref{sec:macros}).

  \item[{\tt parse_ast_translation}, {\tt print_ast_translation}]
    list sets of constants that invoke translation functions for abstract
    syntax trees.  Section \S\ref{sec:asts} below discusses this obscure
    matter.\index{constants!for translations}

  \item[{\tt parse_translation}, {\tt print_translation}] list sets
    of constants that invoke translation functions for terms (see
    \S\ref{sec:tr_funs}).
\end{description}
\index{syntax!Pure|)}


\section{Mixfix declarations} \label{sec:mixfix}
\index{mixfix declarations|(}

When defining a theory, you declare new constants by giving their names,
their type, and an optional {\bf mixfix annotation}.  Mixfix annotations
allow you to extend Isabelle's basic $\lambda$-calculus syntax with
readable notation.  They can express any context-free priority grammar.
Isabelle syntax definitions are inspired by \OBJ~\cite{OBJ}; they are more
general than the priority declarations of \ML\ and Prolog.

A mixfix annotation defines a production of the priority grammar.  It
describes the concrete syntax, the translation to abstract syntax, and the
pretty printing.  Special case annotations provide a simple means of
specifying infix operators and binders.

\subsection{The general mixfix form}
Here is a detailed account of mixfix declarations.  Suppose the following
line occurs within a {\tt consts} or {\tt syntax} section of a {\tt .thy}
file:
\begin{center}
  {\tt $c$ ::\ "$\sigma$" ("$template$" $ps$ $p$)}
\end{center}
This constant declaration and mixfix annotation are interpreted as follows:
\begin{itemize}\index{productions}
\item The string {\tt $c$} is the name of the constant associated with the
  production; unless it is a valid identifier, it must be enclosed in
  quotes.  If $c$ is empty (given as~{\tt ""}) then this is a copy
  production.\index{productions!copy} Otherwise, parsing an instance of the
  phrase $template$ generates the \AST{} {\tt ("$c$" $a@1$ $\ldots$
    $a@n$)}, where $a@i$ is the \AST{} generated by parsing the $i$-th
  argument.

  \item The constant $c$, if non-empty, is declared to have type $\sigma$
    ({\tt consts} section only).

  \item The string $template$ specifies the right-hand side of
    the production.  It has the form
    \[ w@0 \;_\; w@1 \;_\; \ldots \;_\; w@n, \]
    where each occurrence of {\tt_} denotes an argument position and
    the~$w@i$ do not contain~{\tt _}.  (If you want a literal~{\tt _} in
    the concrete syntax, you must escape it as described below.)  The $w@i$
    may consist of \rmindex{delimiters}, spaces or
    \rmindex{pretty printing} annotations (see below).

  \item The type $\sigma$ specifies the production's nonterminal symbols
    (or name tokens).  If $template$ is of the form above then $\sigma$
    must be a function type with at least~$n$ argument positions, say
    $\sigma = [\tau@1, \dots, \tau@n] \To \tau$.  Nonterminal symbols are
    derived from the types $\tau@1$, \ldots,~$\tau@n$, $\tau$ as described
    below.  Any of these may be function types.

  \item The optional list~$ps$ may contain at most $n$ integers, say {\tt
      [$p@1$, $\ldots$, $p@m$]}, where $p@i$ is the minimal
    priority\indexbold{priorities} required of any phrase that may appear
    as the $i$-th argument.  Missing priorities default to~0.

  \item The integer $p$ is the priority of this production.  If omitted, it
    defaults to the maximal priority.
    Priorities range between 0 and \ttindexbold{max_pri} (= 1000).
\end{itemize}
%
The resulting production is \[ A^{(p)}= w@0\, A@1^{(p@1)}\, w@1\,
A@2^{(p@2)}\, \dots\, A@n^{(p@n)}\, w@n \] where $A$ and the $A@i$ are the
nonterminals corresponding to the types $\tau$ and $\tau@i$ respectively.
The nonterminal symbol associated with a type $(\ldots)ty$ is {\tt logic}, if
this is a logical type (namely one of class {\tt logic} excluding {\tt
prop}).  Otherwise it is $ty$ (note that only the outermost type constructor
is taken into account).  Finally, the nonterminal of a type variable is {\tt
any}.

\begin{warn}
  Theories must sometimes declare types for purely syntactic purposes ---
  merely playing the role of nonterminals. One example is \tydx{type}, the
  built-in type of types.  This is a `type of all types' in the syntactic
  sense only.  Do not declare such types under {\tt arities} as belonging to
  class {\tt logic}\index{*logic class}, for that would make them useless as
  separate nonterminal symbols.
\end{warn}

Associating nonterminals with types allows a constant's type to specify
syntax as well.  We can declare the function~$f$ to have type $[\tau@1,
\ldots, \tau@n]\To \tau$ and, through a mixfix annotation, specify the layout
of the function's $n$ arguments.  The constant's name, in this case~$f$, will
also serve as the label in the abstract syntax tree.

You may also declare mixfix syntax without adding constants to the theory's
signature, by using a {\tt syntax} section instead of {\tt consts}.  Thus a
production need not map directly to a logical function (this typically
requires additional syntactic translations, see also
Chapter~\ref{chap:syntax}).


\medskip
As a special case of the general mixfix declaration, the form
\begin{center}
  {\tt $c$ ::\ "$\sigma$" ("$template$")}
\end{center}
specifies no priorities.  The resulting production puts no priority
constraints on any of its arguments and has maximal priority itself.
Omitting priorities in this manner is prone to syntactic ambiguities unless
the production's right-hand side is fully bracketed, as in \verb|"if _ then _
else _ fi"|.

Omitting the mixfix annotation completely, as in {\tt $c$ ::\ "$\sigma$"},
is sensible only if~$c$ is an identifier.  Otherwise you will be unable to
write terms involving~$c$.


\subsection{Example: arithmetic expressions}
\index{examples!of mixfix declarations}
This theory specification contains a {\tt syntax} section with mixfix
declarations encoding the priority grammar from
\S\ref{sec:priority_grammars}:
\begin{ttbox}
EXP = Pure +
types
  exp
syntax
  "0" :: exp                 ("0"      9)
  "+" :: [exp, exp] => exp   ("_ + _"  [0, 1] 0)
  "*" :: [exp, exp] => exp   ("_ * _"  [3, 2] 2)
  "-" :: exp => exp          ("- _"    [3] 3)
end
\end{ttbox}
If you put this into a file {\tt EXP.thy} and load it via {\tt use_thy"EXP"},
you can run some tests:
\begin{ttbox}
val read_exp = Syntax.test_read (syn_of EXP.thy) "exp";
{\out val it = fn : string -> unit}
read_exp "0 * 0 * 0 * 0 + 0 + 0 + 0";
{\out tokens: "0" "*" "0" "*" "0" "*" "0" "+" "0" "+" "0" "+" "0"}
{\out raw: ("+" ("+" ("+" ("*" "0" ("*" "0" ("*" "0" "0"))) "0") "0") "0")}
{\out \vdots}
read_exp "0 + - 0 + 0";
{\out tokens: "0" "+" "-" "0" "+" "0"}
{\out raw: ("+" ("+" "0" ("-" "0")) "0")}
{\out \vdots}
\end{ttbox}
The output of \ttindex{Syntax.test_read} includes the token list ({\tt
  tokens}) and the raw \AST{} directly derived from the parse tree,
ignoring parse \AST{} translations.  The rest is tracing information
provided by the macro expander (see \S\ref{sec:macros}).

Executing {\tt Syntax.print_gram} reveals the productions derived from the
above mixfix declarations (lots of additional information deleted):
\begin{ttbox}
Syntax.print_gram (syn_of EXP.thy);
{\out exp = "0"  => "0" (9)}
{\out exp = exp[0] "+" exp[1]  => "+" (0)}
{\out exp = exp[3] "*" exp[2]  => "*" (2)}
{\out exp = "-" exp[3]  => "-" (3)}
\end{ttbox}

Note that because {\tt exp} is not of class {\tt logic}, it has been retained
as a separate nonterminal. This also entails that the syntax does not provide
for identifiers or paranthesized expressions. Normally you would also want to
add the declaration {\tt arities exp :: logic} and use {\tt consts} instead
of {\tt syntax}. Try this as an exercise and study the changes in the
grammar.

\subsection{The mixfix template}
Let us now take a closer look at the string $template$ appearing in mixfix
annotations.  This string specifies a list of parsing and printing
directives: delimiters\index{delimiters}, arguments, spaces, blocks of
indentation and line breaks.  These are encoded by the following character
sequences:
\index{pretty printing|(}
\begin{description}
\item[~$d$~] is a delimiter, namely a non-empty sequence of characters
  other than the special characters {\tt _}, {\tt(}, {\tt)} and~{\tt/}.
  Even these characters may appear if escaped; this means preceding it with
  a~{\tt '} (single quote).  Thus you have to write {\tt ''} if you really
  want a single quote.  Furthermore, a~{\tt '} followed by a space separates
  delimiters without extra white space being added for printing.

\item[~{\tt_}~] is an argument position, which stands for a nonterminal symbol
  or name token.

\item[~$s$~] is a non-empty sequence of spaces for printing.  This and the
  following specifications do not affect parsing at all.

\item[~{\tt(}$n$~] opens a pretty printing block.  The optional number $n$
  specifies how much indentation to add when a line break occurs within the
  block.  If {\tt(} is not followed by digits, the indentation defaults
  to~0.

\item[~{\tt)}~] closes a pretty printing block.

\item[~{\tt//}~] forces a line break.

\item[~{\tt/}$s$~] allows a line break.  Here $s$ stands for the string of
  spaces (zero or more) right after the {\tt /} character.  These spaces
  are printed if the break is not taken.
\end{description}
For example, the template {\tt"(_ +/ _)"} specifies an infix operator.
There are two argument positions; the delimiter~{\tt+} is preceded by a
space and followed by a space or line break; the entire phrase is a pretty
printing block.  Other examples appear in Fig.\ts\ref{fig:set_trans} below.
Isabelle's pretty printer resembles the one described in
Paulson~\cite{paulson91}.

\index{pretty printing|)}


\subsection{Infixes}
\indexbold{infixes}

Infix operators associating to the left or right can be declared
using {\tt infixl} or {\tt infixr}.
Roughly speaking, the form {\tt $c$ ::\ $\sigma$ (infixl $p$)}
abbreviates the mixfix declarations
\begin{ttbox}
"op \(c\)" :: \(\sigma\)   ("(_ \(c\)/ _)" [\(p\), \(p+1\)] \(p\))
"op \(c\)" :: \(\sigma\)   ("op \(c\)")
\end{ttbox}
and {\tt $c$ ::\ $\sigma$ (infixr $p$)} abbreviates the mixfix declarations
\begin{ttbox}
"op \(c\)" :: \(\sigma\)   ("(_ \(c\)/ _)" [\(p+1\), \(p\)] \(p\))
"op \(c\)" :: \(\sigma\)   ("op \(c\)")
\end{ttbox}
The infix operator is declared as a constant with the prefix {\tt op}.
Thus, prefixing infixes with \sdx{op} makes them behave like ordinary
function symbols, as in \ML.  Special characters occurring in~$c$ must be
escaped, as in delimiters, using a single quote.


\subsection{Binders}
\indexbold{binders}
\begingroup
\def\Q{{\cal Q}}
A {\bf binder} is a variable-binding construct such as a quantifier.  The
constant declaration
\begin{ttbox}
\(c\) :: \(\sigma\)   (binder "\(\Q\)" [\(pb\)] \(p\))
\end{ttbox}
introduces a constant~$c$ of type~$\sigma$, which must have the form
$(\tau@1 \To \tau@2) \To \tau@3$.  Its concrete syntax is $\Q~x.P$, where
$x$ is a bound variable of type~$\tau@1$, the body~$P$ has type $\tau@2$
and the whole term has type~$\tau@3$. The optional integer $pb$
specifies the body's priority, by default~$p$.  Special characters
in $\Q$ must be escaped using a single quote.

The declaration is expanded internally to something like
\begin{ttbox}
\(c\)    :: (\(\tau@1\) => \(\tau@2\)) => \(\tau@3\)
"\(\Q\)"\hskip-3pt  :: [idts, \(\tau@2\)] => \(\tau@3\)   ("(3\(\Q\)_./ _)" [0, \(pb\)] \(p\))
\end{ttbox}
Here \ndx{idts} is the nonterminal symbol for a list of identifiers with
\index{type constraints}
optional type constraints (see Fig.\ts\ref{fig:pure_gram}).  The
declaration also installs a parse translation\index{translations!parse}
for~$\Q$ and a print translation\index{translations!print} for~$c$ to
translate between the internal and external forms.

A binder of type $(\sigma \To \tau) \To \tau$ can be nested by giving a
list of variables.  The external form $\Q~x@1~x@2 \ldots x@n. P$
corresponds to the internal form
\[ c(\lambda x@1. c(\lambda x@2. \ldots c(\lambda x@n. P) \ldots)). \]

\medskip
For example, let us declare the quantifier~$\forall$:\index{quantifiers}
\begin{ttbox}
All :: ('a => o) => o   (binder "ALL " 10)
\end{ttbox}
This lets us write $\forall x.P$ as either {\tt All(\%$x$.$P$)} or {\tt ALL
  $x$.$P$}.  When printing, Isabelle prefers the latter form, but must fall
back on ${\tt All}(P)$ if $P$ is not an abstraction.  Both $P$ and {\tt ALL
  $x$.$P$} have type~$o$, the type of formulae, while the bound variable
can be polymorphic.
\endgroup

\index{mixfix declarations|)}

\section{Ambiguity of parsed expressions} \label{sec:ambiguity}
\index{ambiguity!of parsed expressions}

To keep the grammar small and allow common productions to be shared
all logical types (except {\tt prop}) are internally represented
by one nonterminal, namely {\tt logic}. This and omitted or too freely
chosen priorities may lead to ways of parsing an expression that were
not intended by the theory's maker. In most cases Isabelle is able to
select one of multiple parse trees that an expression has lead
to by checking which of them can be typed correctly. But this may not
work in every case and always slows down parsing.
The warning and error messages that can be produced during this process are
as follows:

If an ambiguity can be resolved by type inference the following
warning is shown to remind the user that parsing is (unnecessarily)
slowed down. In cases where it's not easily possible to eliminate the
ambiguity the frequency of the warning can be controlled by changing
the value of {\tt Syntax.ambiguity_level} which has type {\tt int
ref}. Its default value is 1 and by increasing it one can control how
many parse trees are necessary to generate the warning.

\begin{ttbox}
{\out Warning: Ambiguous input "..."}
{\out produces the following parse trees:}
{\out ...}
{\out Fortunately, only one parse tree is type correct.}
{\out It helps (speed!) if you disambiguate your grammar or your input.}
\end{ttbox}

The following message is normally caused by using the same
syntax in two different productions:

\begin{ttbox}
{\out Warning: Ambiguous input "..."}
{\out produces the following parse trees:}
{\out ...}
{\out Error: More than one term is type correct:}
{\out ...}
\end{ttbox}

Ambiguities occuring in syntax translation rules cannot be resolved by
type inference because it is not necessary for these rules to be type
correct. Therefore Isabelle always generates an error message and the
ambiguity should be eliminated by changing the grammar or the rule.


\section{Example: some minimal logics} \label{sec:min_logics}
\index{examples!of logic definitions}

This section presents some examples that have a simple syntax.  They
demonstrate how to define new object-logics from scratch.

First we must define how an object-logic syntax is embedded into the
meta-logic.  Since all theorems must conform to the syntax for~\ndx{prop}
(see Fig.\ts\ref{fig:pure_gram}), that syntax has to be extended with the
object-level syntax.  Assume that the syntax of your object-logic defines a
meta-type~\tydx{o} of formulae which refers to the nonterminal {\tt logic}.
These formulae can now appear in axioms and theorems wherever \ndx{prop} does
if you add the production
\[ prop ~=~ logic. \]
This is not supposed to be a copy production but an implicit coercion from
formulae to propositions:
\begin{ttbox}
Base = Pure +
types
  o
arities
  o :: logic
consts
  Trueprop :: o => prop   ("_" 5)
end
\end{ttbox}
The constant \cdx{Trueprop} (the name is arbitrary) acts as an invisible
coercion function.  Assuming this definition resides in a file {\tt Base.thy},
you have to load it with the command {\tt use_thy "Base"}.

One of the simplest nontrivial logics is {\bf minimal logic} of
implication.  Its definition in Isabelle needs no advanced features but
illustrates the overall mechanism nicely:
\begin{ttbox}
Hilbert = Base +
consts
  "-->" :: [o, o] => o   (infixr 10)
rules
  K     "P --> Q --> P"
  S     "(P --> Q --> R) --> (P --> Q) --> P --> R"
  MP    "[| P --> Q; P |] ==> Q"
end
\end{ttbox}
After loading this definition from the file {\tt Hilbert.thy}, you can
start to prove theorems in the logic:
\begin{ttbox}
goal Hilbert.thy "P --> P";
{\out Level 0}
{\out P --> P}
{\out  1.  P --> P}
\ttbreak
by (resolve_tac [Hilbert.MP] 1);
{\out Level 1}
{\out P --> P}
{\out  1.  ?P --> P --> P}
{\out  2.  ?P}
\ttbreak
by (resolve_tac [Hilbert.MP] 1);
{\out Level 2}
{\out P --> P}
{\out  1.  ?P1 --> ?P --> P --> P}
{\out  2.  ?P1}
{\out  3.  ?P}
\ttbreak
by (resolve_tac [Hilbert.S] 1);
{\out Level 3}
{\out P --> P}
{\out  1.  P --> ?Q2 --> P}
{\out  2.  P --> ?Q2}
\ttbreak
by (resolve_tac [Hilbert.K] 1);
{\out Level 4}
{\out P --> P}
{\out  1.  P --> ?Q2}
\ttbreak
by (resolve_tac [Hilbert.K] 1);
{\out Level 5}
{\out P --> P}
{\out No subgoals!}
\end{ttbox}
As we can see, this Hilbert-style formulation of minimal logic is easy to
define but difficult to use.  The following natural deduction formulation is
better:
\begin{ttbox}
MinI = Base +
consts
  "-->" :: [o, o] => o   (infixr 10)
rules
  impI  "(P ==> Q) ==> P --> Q"
  impE  "[| P --> Q; P |] ==> Q"
end
\end{ttbox}
Note, however, that although the two systems are equivalent, this fact
cannot be proved within Isabelle.  Axioms {\tt S} and {\tt K} can be
derived in {\tt MinI} (exercise!), but {\tt impI} cannot be derived in {\tt
  Hilbert}.  The reason is that {\tt impI} is only an {\bf admissible} rule
in {\tt Hilbert}, something that can only be shown by induction over all
possible proofs in {\tt Hilbert}.

We may easily extend minimal logic with falsity:
\begin{ttbox}
MinIF = MinI +
consts
  False :: o
rules
  FalseE "False ==> P"
end
\end{ttbox}
On the other hand, we may wish to introduce conjunction only:
\begin{ttbox}
MinC = Base +
consts
  "&" :: [o, o] => o   (infixr 30)
\ttbreak
rules
  conjI  "[| P; Q |] ==> P & Q"
  conjE1 "P & Q ==> P"
  conjE2 "P & Q ==> Q"
end
\end{ttbox}
And if we want to have all three connectives together, we create and load a
theory file consisting of a single line:\footnote{We can combine the
  theories without creating a theory file using the ML declaration
\begin{ttbox}
val MinIFC_thy = merge_theories(MinIF,MinC)
\end{ttbox}
\index{*merge_theories|fnote}}
\begin{ttbox}
MinIFC = MinIF + MinC
\end{ttbox}
Now we can prove mixed theorems like
\begin{ttbox}
goal MinIFC.thy "P & False --> Q";
by (resolve_tac [MinI.impI] 1);
by (dresolve_tac [MinC.conjE2] 1);
by (eresolve_tac [MinIF.FalseE] 1);
\end{ttbox}
Try this as an exercise!
