%% $Id$
\part{Advanced Methods}
Before continuing, it might be wise to try some of your own examples in
Isabelle, reinforcing your knowledge of the basic functions.

Look through {\em Isabelle's Object-Logics\/} and try proving some simple
theorems.  You probably should begin with first-order logic ({\tt FOL}
or~{\tt LK}).  Try working some of the examples provided, and others from
the literature.  Set theory~({\tt ZF}) and Constructive Type Theory~({\tt
  CTT}) form a richer world for mathematical reasoning and, again, many
examples are in the literature.  Higher-order logic~({\tt HOL}) is
Isabelle's most sophisticated logic because its types and functions are
identified with those of the meta-logic.

Choose a logic that you already understand.  Isabelle is a proof
tool, not a teaching tool; if you do not know how to do a particular proof
on paper, then you certainly will not be able to do it on the machine.
Even experienced users plan large proofs on paper.

We have covered only the bare essentials of Isabelle, but enough to perform
substantial proofs.  By occasionally dipping into the {\em Reference
Manual}, you can learn additional tactics, subgoal commands and tacticals.


\section{Deriving rules in Isabelle}
\index{rules!derived}
A mathematical development goes through a progression of stages.  Each
stage defines some concepts and derives rules about them.  We shall see how
to derive rules, perhaps involving definitions, using Isabelle.  The
following section will explain how to declare types, constants, rules and
definitions.


\subsection{Deriving a rule using tactics and meta-level assumptions} 
\label{deriving-example}
\index{examples!of deriving rules}\index{assumptions!of main goal}

The subgoal module supports the derivation of rules, as discussed in
\S\ref{deriving}.  The \ttindex{goal} command, when supplied a goal of the
form $\List{\theta@1; \ldots; \theta@k} \Imp \phi$, creates $\phi\Imp\phi$
as the initial proof state and returns a list consisting of the theorems
${\theta@i\;[\theta@i]}$, for $i=1$, \ldots,~$k$.  These meta-assumptions
are also recorded internally, allowing {\tt result} to discharge them
in the original order.

Let us derive $\conj$ elimination using Isabelle.
Until now, calling {\tt goal} has returned an empty list, which we have
thrown away.  In this example, the list contains the two premises of the
rule.  We bind them to the \ML\ identifiers {\tt major} and {\tt
minor}:\footnote{Some ML compilers will print a message such as {\em
binding not exhaustive}.  This warns that {\tt goal} must return a
2-element list.  Otherwise, the pattern-match will fail; ML will
raise exception \xdx{Match}.}
\begin{ttbox}
val [major,minor] = goal FOL.thy
    "[| P&Q;  [| P; Q |] ==> R |] ==> R";
{\out Level 0}
{\out R}
{\out  1. R}
{\out val major = "P & Q  [P & Q]" : thm}
{\out val minor = "[| P; Q |] ==> R  [[| P; Q |] ==> R]" : thm}
\end{ttbox}
Look at the minor premise, recalling that meta-level assumptions are
shown in brackets.  Using {\tt minor}, we reduce $R$ to the subgoals
$P$ and~$Q$:
\begin{ttbox}
by (resolve_tac [minor] 1);
{\out Level 1}
{\out R}
{\out  1. P}
{\out  2. Q}
\end{ttbox}
Deviating from~\S\ref{deriving}, we apply $({\conj}E1)$ forwards from the
assumption $P\conj Q$ to obtain the theorem~$P\;[P\conj Q]$.
\begin{ttbox}
major RS conjunct1;
{\out val it = "P  [P & Q]" : thm}
\ttbreak
by (resolve_tac [major RS conjunct1] 1);
{\out Level 2}
{\out R}
{\out  1. Q}
\end{ttbox}
Similarly, we solve the subgoal involving~$Q$.
\begin{ttbox}
major RS conjunct2;
{\out val it = "Q  [P & Q]" : thm}
by (resolve_tac [major RS conjunct2] 1);
{\out Level 3}
{\out R}
{\out No subgoals!}
\end{ttbox}
Calling \ttindex{topthm} returns the current proof state as a theorem.
Note that it contains assumptions.  Calling \ttindex{result} discharges the
assumptions --- both occurrences of $P\conj Q$ are discharged as one ---
and makes the variables schematic.
\begin{ttbox}
topthm();
{\out val it = "R  [P & Q, P & Q, [| P; Q |] ==> R]" : thm}
val conjE = result();
{\out val conjE = "[| ?P & ?Q; [| ?P; ?Q |] ==> ?R |] ==> ?R" : thm}
\end{ttbox}


\subsection{Definitions and derived rules} \label{definitions}
\index{rules!derived}\index{definitions!and derived rules|(}

Definitions are expressed as meta-level equalities.  Let us define negation
and the if-and-only-if connective:
\begin{eqnarray*}
  \neg \Var{P}          & \equiv & \Var{P}\imp\bot \\
  \Var{P}\bimp \Var{Q}  & \equiv & 
                (\Var{P}\imp \Var{Q}) \conj (\Var{Q}\imp \Var{P})
\end{eqnarray*}
\index{meta-rewriting}%
Isabelle permits {\bf meta-level rewriting} using definitions such as
these.  {\bf Unfolding} replaces every instance
of $\neg \Var{P}$ by the corresponding instance of ${\Var{P}\imp\bot}$.  For
example, $\forall x.\neg (P(x)\conj \neg R(x,0))$ unfolds to
\[ \forall x.(P(x)\conj R(x,0)\imp\bot)\imp\bot.  \]
{\bf Folding} a definition replaces occurrences of the right-hand side by
the left.  The occurrences need not be free in the entire formula.

When you define new concepts, you should derive rules asserting their
abstract properties, and then forget their definitions.  This supports
modularity: if you later change the definitions without affecting their
abstract properties, then most of your proofs will carry through without
change.  Indiscriminate unfolding makes a subgoal grow exponentially,
becoming unreadable.

Taking this point of view, Isabelle does not unfold definitions
automatically during proofs.  Rewriting must be explicit and selective.
Isabelle provides tactics and meta-rules for rewriting, and a version of
the {\tt goal} command that unfolds the conclusion and premises of the rule
being derived.

For example, the intuitionistic definition of negation given above may seem
peculiar.  Using Isabelle, we shall derive pleasanter negation rules:
\[  \infer[({\neg}I)]{\neg P}{\infer*{\bot}{[P]}}   \qquad
    \infer[({\neg}E)]{Q}{\neg P & P}  \]
This requires proving the following meta-formulae:
$$ (P\Imp\bot)    \Imp \neg P   \eqno(\neg I)$$
$$ \List{\neg P; P} \Imp Q.       \eqno(\neg E)$$


\subsection{Deriving the $\neg$ introduction rule}
To derive $(\neg I)$, we may call {\tt goal} with the appropriate
formula.  Again, {\tt goal} returns a list consisting of the rule's
premises.  We bind this one-element list to the \ML\ identifier {\tt
  prems}.
\begin{ttbox}
val prems = goal FOL.thy "(P ==> False) ==> ~P";
{\out Level 0}
{\out ~P}
{\out  1. ~P}
{\out val prems = ["P ==> False  [P ==> False]"] : thm list}
\end{ttbox}
Calling \ttindex{rewrite_goals_tac} with \tdx{not_def}, which is the
definition of negation, unfolds that definition in the subgoals.  It leaves
the main goal alone.
\begin{ttbox}
not_def;
{\out val it = "~?P == ?P --> False" : thm}
by (rewrite_goals_tac [not_def]);
{\out Level 1}
{\out ~P}
{\out  1. P --> False}
\end{ttbox}
Using \tdx{impI} and the premise, we reduce subgoal~1 to a triviality:
\begin{ttbox}
by (resolve_tac [impI] 1);
{\out Level 2}
{\out ~P}
{\out  1. P ==> False}
\ttbreak
by (resolve_tac prems 1);
{\out Level 3}
{\out ~P}
{\out  1. P ==> P}
\end{ttbox}
The rest of the proof is routine.  Note the form of the final result.
\begin{ttbox}
by (assume_tac 1);
{\out Level 4}
{\out ~P}
{\out No subgoals!}
\ttbreak
val notI = result();
{\out val notI = "(?P ==> False) ==> ~?P" : thm}
\end{ttbox}
\indexbold{*notI theorem}

There is a simpler way of conducting this proof.  The \ttindex{goalw}
command starts a backward proof, as does {\tt goal}, but it also
unfolds definitions.  Thus there is no need to call
\ttindex{rewrite_goals_tac}:
\begin{ttbox}
val prems = goalw FOL.thy [not_def]
    "(P ==> False) ==> ~P";
{\out Level 0}
{\out ~P}
{\out  1. P --> False}
{\out val prems = ["P ==> False  [P ==> False]"] : thm list}
\end{ttbox}


\subsection{Deriving the $\neg$ elimination rule}
Let us derive the rule $(\neg E)$.  The proof follows that of~{\tt conjE}
above, with an additional step to unfold negation in the major premise.
Although the {\tt goalw} command is best for this, let us
try~{\tt goal} to see another way of unfolding definitions.  After
binding the premises to \ML\ identifiers, we apply \tdx{FalseE}:
\begin{ttbox}
val [major,minor] = goal FOL.thy "[| ~P;  P |] ==> R";
{\out Level 0}
{\out R}
{\out  1. R}
{\out val major = "~ P  [~ P]" : thm}
{\out val minor = "P  [P]" : thm}
\ttbreak
by (resolve_tac [FalseE] 1);
{\out Level 1}
{\out R}
{\out  1. False}
\end{ttbox}
Everything follows from falsity.  And we can prove falsity using the
premises and Modus Ponens:
\begin{ttbox}
by (resolve_tac [mp] 1);
{\out Level 2}
{\out R}
{\out  1. ?P1 --> False}
{\out  2. ?P1}
\end{ttbox}
For subgoal~1, we transform the major premise from~$\neg P$
to~${P\imp\bot}$.  The function \ttindex{rewrite_rule}, given a list of
definitions, unfolds them in a theorem.  Rewriting does not
affect the theorem's hypothesis, which remains~$\neg P$:
\begin{ttbox}
rewrite_rule [not_def] major;
{\out val it = "P --> False  [~P]" : thm}
by (resolve_tac [it] 1);
{\out Level 3}
{\out R}
{\out  1. P}
\end{ttbox}
The subgoal {\tt?P1} has been instantiated to~{\tt P}, which we can prove
using the minor premise:
\begin{ttbox}
by (resolve_tac [minor] 1);
{\out Level 4}
{\out R}
{\out No subgoals!}
val notE = result();
{\out val notE = "[| ~?P; ?P |] ==> ?R" : thm}
\end{ttbox}
\indexbold{*notE theorem}

\medskip
Again, there is a simpler way of conducting this proof.  Recall that
the \ttindex{goalw} command unfolds definitions the conclusion; it also
unfolds definitions in the premises:
\begin{ttbox}
val [major,minor] = goalw FOL.thy [not_def]
    "[| ~P;  P |] ==> R";
{\out val major = "P --> False  [~ P]" : thm}
{\out val minor = "P  [P]" : thm}
\end{ttbox}
Observe the difference in {\tt major}; the premises are unfolded without
calling~\ttindex{rewrite_rule}.  Incidentally, the four calls to
\ttindex{resolve_tac} above can be collapsed to one, with the help
of~\ttindex{RS}; this is a typical example of forward reasoning from a
complex premise.
\begin{ttbox}
minor RS (major RS mp RS FalseE);
{\out val it = "?P  [P, ~P]" : thm}
by (resolve_tac [it] 1);
{\out Level 1}
{\out R}
{\out No subgoals!}
\end{ttbox}
\index{definitions!and derived rules|)}

\goodbreak\medskip\index{*"!"! symbol!in main goal}
Finally, here is a trick that is sometimes useful.  If the goal
has an outermost meta-quantifier, then \ttindex{goal} and \ttindex{goalw}
do not return the rule's premises in the list of theorems;  instead, the
premises become assumptions in subgoal~1.  
%%%It does not matter which variables are quantified over.
\begin{ttbox}
goalw FOL.thy [not_def] "!!P R. [| ~P;  P |] ==> R";
{\out Level 0}
{\out !!P R. [| ~ P; P |] ==> R}
{\out  1. !!P R. [| P --> False; P |] ==> R}
val it = [] : thm list
\end{ttbox}
The proof continues as before.  But instead of referring to \ML\
identifiers, we refer to assumptions using {\tt eresolve_tac} or
{\tt assume_tac}: 
\begin{ttbox}
by (resolve_tac [FalseE] 1);
{\out Level 1}
{\out !!P R. [| ~ P; P |] ==> R}
{\out  1. !!P R. [| P --> False; P |] ==> False}
\ttbreak
by (eresolve_tac [mp] 1);
{\out Level 2}
{\out !!P R. [| ~ P; P |] ==> R}
{\out  1. !!P R. P ==> P}
\ttbreak
by (assume_tac 1);
{\out Level 3}
{\out !!P R. [| ~ P; P |] ==> R}
{\out No subgoals!}
\end{ttbox}
Calling \ttindex{result} strips the meta-quantifiers, so the resulting
theorem is the same as before.
\begin{ttbox}
val notE = result();
{\out val notE = "[| ~?P; ?P |] ==> ?R" : thm}
\end{ttbox}
Do not use the {\tt!!}\ trick if the premises contain meta-level
connectives, because \ttindex{eresolve_tac} and \ttindex{assume_tac} would
not be able to handle the resulting assumptions.  The trick is not suitable
for deriving the introduction rule~$(\neg I)$.


\section{Defining theories}\label{sec:defining-theories}
\index{theories!defining|(}

Isabelle makes no distinction between simple extensions of a logic --- like
defining a type~$bool$ with constants~$true$ and~$false$ --- and defining
an entire logic.  A theory definition has the form
\begin{ttbox}
\(T\) = \(S@1\) + \(\cdots\) + \(S@n\) +
classes      {\it class declarations}
default      {\it sort}
types        {\it type declarations and synonyms}
arities      {\it arity declarations}
consts       {\it constant declarations}
translations {\it translation declarations}
defs         {\it definitions}
rules        {\it rule declarations}
end
ML           {\it ML code}
\end{ttbox}
This declares the theory $T$ to extend the existing theories
$S@1$,~\ldots,~$S@n$.  It may declare new classes, types, arities
(overloadings of existing types), constants and rules; it can specify the
default sort for type variables.  A constant declaration can specify an
associated concrete syntax.  The translations section specifies rewrite
rules on abstract syntax trees, for defining notations and abbreviations.
\index{*ML section}
The {\tt ML} section contains code to perform arbitrary syntactic
transformations.  The main declaration forms are discussed below.
The full syntax can be found in \iflabelundefined{app:TheorySyntax}{the
  appendix of the {\it Reference Manual}}{App.\ts\ref{app:TheorySyntax}}.

All the declaration parts can be omitted or repeated and may appear in any
order, except that the {\ML} section must be last.  In the simplest case, $T$
is just the union of $S@1$,~\ldots,~$S@n$.  New theories always extend one or
more other theories, inheriting their types, constants, syntax, etc.  The
theory \thydx{Pure} contains nothing but Isabelle's meta-logic.

Each theory definition must reside in a separate file, whose name is the
theory's with {\tt.thy} appended.  For example, theory {\tt ListFn} resides
on a file named {\tt ListFn.thy}.  Isabelle uses this convention to locate the
file containing a given theory; \ttindexbold{use_thy} automatically loads a
theory's parents before loading the theory itself.

Calling \ttindexbold{use_thy}~{\tt"{\it T\/}"} reads a theory from the
file {\it T}{\tt.thy}, writes the corresponding {\ML} code to the file
{\tt.{\it T}.thy.ML}, reads the latter file, and deletes it if no errors
occurred.  This declares the {\ML} structure~$T$, which contains a component
{\tt thy} denoting the new theory, a component for each rule, and everything
declared in {\it ML code}.

Errors may arise during the translation to {\ML} (say, a misspelled keyword)
or during creation of the new theory (say, a type error in a rule).  But if
all goes well, {\tt use_thy} will finally read the file {\it T}{\tt.ML}, if
it exists.  This file typically begins with the {\ML} declaration {\tt
open}~$T$ and contains proofs that refer to the components of~$T$.

When a theory file is modified, many theories may have to be reloaded.
Isabelle records the modification times and dependencies of theory files.
See 
\iflabelundefined{sec:reloading-theories}{the {\em Reference Manual\/}}%
                 {\S\ref{sec:reloading-theories}}
for more details.


\subsection{Declaring constants, definitions and rules}
\indexbold{constants!declaring}\index{rules!declaring}

Most theories simply declare constants, definitions and rules.  The {\bf
  constant declaration part} has the form
\begin{ttbox}
consts  \(c@1\) :: \(\tau@1\)
        \vdots
        \(c@n\) :: \(\tau@n\)
\end{ttbox}
where $c@1$, \ldots, $c@n$ are constants and $\tau@1$, \ldots, $\tau@n$ are
types.  The types must be enclosed in quotation marks if they contain
user-declared infix type constructors like {\tt *}.  Each
constant must be enclosed in quotation marks unless it is a valid
identifier.  To declare $c@1$, \ldots, $c@n$ as constants of type $\tau$,
the $n$ declarations may be abbreviated to a single line:
\begin{ttbox}
        \(c@1\), \ldots, \(c@n\) :: \(\tau\)
\end{ttbox}
The {\bf rule declaration part} has the form
\begin{ttbox}
rules   \(id@1\) "\(rule@1\)"
        \vdots
        \(id@n\) "\(rule@n\)"
\end{ttbox}
where $id@1$, \ldots, $id@n$ are \ML{} identifiers and $rule@1$, \ldots,
$rule@n$ are expressions of type~$prop$.  Each rule {\em must\/} be
enclosed in quotation marks.

\indexbold{definitions} The {\bf definition part} is similar, but with the
keyword {\tt defs} instead of {\tt rules}.  {\bf Definitions} are rules of the
form $t\equiv u$, and should serve only as abbreviations.  Isabelle checks for
common errors in definitions, such as extra variables on the right-hard side.
Determined users can write non-conservative `definitions' by using mutual
recursion, for example; the consequences of such actions are their
responsibility.


\index{examples!of theories} 
This theory extends first-order logic by declaring and defining two constants,
{\em nand} and {\em xor}:
\begin{ttbox}
Gate = FOL +
consts  nand,xor :: [o,o] => o
defs    nand_def "nand(P,Q) == ~(P & Q)"
        xor_def  "xor(P,Q)  == P & ~Q | ~P & Q"
end
\end{ttbox}

\begin{warn}
A common mistake when writing definitions is to introduce extra free variables
on the right-hard side as in the following fictitious definition:
\begin{ttbox}
defs  prime_def "prime(p) == (m divides p) --> (m=1 | m=p)"
\end{ttbox}
Isabelle rejects this ``definition'' because of the extra {\tt m} on the
right-hard side, which would introduce an inconsistency. What you should have
written is
\begin{ttbox}
defs  prime_def "prime(p) == ALL m. (m divides p) --> (m=1 | m=p)"
\end{ttbox}
\end{warn}

\subsection{Declaring type constructors}
\indexbold{types!declaring}\indexbold{arities!declaring}
%
Types are composed of type variables and {\bf type constructors}.  Each
type constructor takes a fixed number of arguments.  They are declared
with an \ML-like syntax.  If $list$ takes one type argument, $tree$ takes
two arguments and $nat$ takes no arguments, then these type constructors
can be declared by
\begin{ttbox}
types 'a list
      ('a,'b) tree
      nat
\end{ttbox}

The {\bf type declaration part} has the general form
\begin{ttbox}
types   \(tids@1\) \(id@1\)
        \vdots
        \(tids@n\) \(id@n\)
\end{ttbox}
where $id@1$, \ldots, $id@n$ are identifiers and $tids@1$, \ldots, $tids@n$
are type argument lists as shown in the example above.  It declares each
$id@i$ as a type constructor with the specified number of argument places.

The {\bf arity declaration part} has the form
\begin{ttbox}
arities \(tycon@1\) :: \(arity@1\)
        \vdots
        \(tycon@n\) :: \(arity@n\)
\end{ttbox}
where $tycon@1$, \ldots, $tycon@n$ are identifiers and $arity@1$, \ldots,
$arity@n$ are arities.  Arity declarations add arities to existing
types; they do not declare the types themselves.
In the simplest case, for an 0-place type constructor, an arity is simply
the type's class.  Let us declare a type~$bool$ of class $term$, with
constants $tt$ and~$ff$.  (In first-order logic, booleans are
distinct from formulae, which have type $o::logic$.)
\index{examples!of theories}
\begin{ttbox}
Bool = FOL +
types   bool
arities bool    :: term
consts  tt,ff   :: bool
end
\end{ttbox}
A $k$-place type constructor may have arities of the form
$(s@1,\ldots,s@k)c$, where $s@1,\ldots,s@n$ are sorts and $c$ is a class.
Each sort specifies a type argument; it has the form $\{c@1,\ldots,c@m\}$,
where $c@1$, \dots,~$c@m$ are classes.  Mostly we deal with singleton
sorts, and may abbreviate them by dropping the braces.  The arity
$(term)term$ is short for $(\{term\})term$.  Recall the discussion in
\S\ref{polymorphic}.

A type constructor may be overloaded (subject to certain conditions) by
appearing in several arity declarations.  For instance, the function type
constructor~$fun$ has the arity $(logic,logic)logic$; in higher-order
logic, it is declared also to have arity $(term,term)term$.

Theory {\tt List} declares the 1-place type constructor $list$, gives
it arity $(term)term$, and declares constants $Nil$ and $Cons$ with
polymorphic types:%
\footnote{In the {\tt consts} part, type variable {\tt'a} has the default
  sort, which is {\tt term}.  See the {\em Reference Manual\/}
\iflabelundefined{sec:ref-defining-theories}{}%
{(\S\ref{sec:ref-defining-theories})} for more information.}
\index{examples!of theories}
\begin{ttbox}
List = FOL +
types   'a list
arities list    :: (term)term
consts  Nil     :: 'a list
        Cons    :: ['a, 'a list] => 'a list
end
\end{ttbox}
Multiple arity declarations may be abbreviated to a single line:
\begin{ttbox}
arities \(tycon@1\), \ldots, \(tycon@n\) :: \(arity\)
\end{ttbox}

\begin{warn}
Arity declarations resemble constant declarations, but there are {\it no\/}
quotation marks!  Types and rules must be quoted because the theory
translator passes them verbatim to the {\ML} output file.
\end{warn}

\subsection{Type synonyms}\indexbold{type synonyms}
Isabelle supports {\bf type synonyms} ({\bf abbreviations}) which are similar
to those found in \ML.  Such synonyms are defined in the type declaration part
and are fairly self explanatory:
\begin{ttbox}
types gate       = [o,o] => o
      'a pred    = 'a => o
      ('a,'b)nuf = 'b => 'a
\end{ttbox}
Type declarations and synonyms can be mixed arbitrarily:
\begin{ttbox}
types nat
      'a stream = nat => 'a
      signal    = nat stream
      'a list
\end{ttbox}
A synonym is merely an abbreviation for some existing type expression.  Hence
synonyms may not be recursive!  Internally all synonyms are fully expanded.  As
a consequence Isabelle output never contains synonyms.  Their main purpose is
to improve the readability of theories.  Synonyms can be used just like any
other type:
\begin{ttbox}
consts and,or :: gate
       negate :: signal => signal
\end{ttbox}

\subsection{Infix and mixfix operators}
\index{infixes}\index{examples!of theories}

Infix or mixfix syntax may be attached to constants.  Consider the
following theory:
\begin{ttbox}
Gate2 = FOL +
consts  "~&"     :: [o,o] => o         (infixl 35)
        "#"      :: [o,o] => o         (infixl 30)
defs    nand_def "P ~& Q == ~(P & Q)"    
        xor_def  "P # Q  == P & ~Q | ~P & Q"
end
\end{ttbox}
The constant declaration part declares two left-associating infix operators
with their priorities, or precedences; they are $\nand$ of priority~35 and
$\xor$ of priority~30.  Hence $P \xor Q \xor R$ is parsed as $(P\xor Q)
\xor R$ and $P \xor Q \nand R$ as $P \xor (Q \nand R)$.  Note the quotation
marks in \verb|"~&"| and \verb|"#"|.

The constants \hbox{\verb|op ~&|} and \hbox{\verb|op #|} are declared
automatically, just as in \ML.  Hence you may write propositions like
\verb|op #(True) == op ~&(True)|, which asserts that the functions $\lambda
Q.True \xor Q$ and $\lambda Q.True \nand Q$ are identical.

\bigskip\index{mixfix declarations}
{\bf Mixfix} operators may have arbitrary context-free syntaxes.  Let us
add a line to the constant declaration part:
\begin{ttbox}
        If :: [o,o,o] => o       ("if _ then _ else _")
\end{ttbox}
This declares a constant $If$ of type $[o,o,o] \To o$ with concrete syntax {\tt
  if~$P$ then~$Q$ else~$R$} as well as {\tt If($P$,$Q$,$R$)}.  Underscores
denote argument positions.  

The declaration above does not allow the {\tt if}-{\tt then}-{\tt else}
construct to be split across several lines, even if it is too long to fit
on one line.  Pretty-printing information can be added to specify the
layout of mixfix operators.  For details, see
\iflabelundefined{Defining-Logics}%
    {the {\it Reference Manual}, chapter `Defining Logics'}%
    {Chap.\ts\ref{Defining-Logics}}.

Mixfix declarations can be annotated with priorities, just like
infixes.  The example above is just a shorthand for
\begin{ttbox}
        If :: [o,o,o] => o       ("if _ then _ else _" [0,0,0] 1000)
\end{ttbox}
The numeric components determine priorities.  The list of integers
defines, for each argument position, the minimal priority an expression
at that position must have.  The final integer is the priority of the
construct itself.  In the example above, any argument expression is
acceptable because priorities are non-negative, and conditionals may
appear everywhere because 1000 is the highest priority.  On the other
hand, the declaration
\begin{ttbox}
        If :: [o,o,o] => o       ("if _ then _ else _" [100,0,0] 99)
\end{ttbox}
defines concrete syntax for a conditional whose first argument cannot have
the form {\tt if~$P$ then~$Q$ else~$R$} because it must have a priority
of at least~100.  We may of course write
\begin{quote}\tt
if (if $P$ then $Q$ else $R$) then $S$ else $T$
\end{quote}
because expressions in parentheses have maximal priority.  

Binary type constructors, like products and sums, may also be declared as
infixes.  The type declaration below introduces a type constructor~$*$ with
infix notation $\alpha*\beta$, together with the mixfix notation
${<}\_,\_{>}$ for pairs.  We also see a rule declaration part.
\index{examples!of theories}\index{mixfix declarations}
\begin{ttbox}
Prod = FOL +
types   ('a,'b) "*"                           (infixl 20)
arities "*"     :: (term,term)term
consts  fst     :: "'a * 'b => 'a"
        snd     :: "'a * 'b => 'b"
        Pair    :: "['a,'b] => 'a * 'b"       ("(1<_,/_>)")
rules   fst     "fst(<a,b>) = a"
        snd     "snd(<a,b>) = b"
end
\end{ttbox}

\begin{warn}
The name of the type constructor is~{\tt *} and not {\tt op~*}, as it would
be in the case of an infix constant.  Only infix type constructors can have
symbolic names like~{\tt *}.  There is no general mixfix syntax for types.
\end{warn}


\subsection{Overloading}
\index{overloading}\index{examples!of theories}
The {\bf class declaration part} has the form
\begin{ttbox}
classes \(id@1\) < \(c@1\)
        \vdots
        \(id@n\) < \(c@n\)
\end{ttbox}
where $id@1$, \ldots, $id@n$ are identifiers and $c@1$, \ldots, $c@n$ are
existing classes.  It declares each $id@i$ as a new class, a subclass
of~$c@i$.  In the general case, an identifier may be declared to be a
subclass of $k$ existing classes:
\begin{ttbox}
        \(id\) < \(c@1\), \ldots, \(c@k\)
\end{ttbox}
Type classes allow constants to be overloaded.  As suggested in
\S\ref{polymorphic}, let us define the class $arith$ of arithmetic
types with the constants ${+} :: [\alpha,\alpha]\To \alpha$ and $0,1 {::}
\alpha$, for $\alpha{::}arith$.  We introduce $arith$ as a subclass of
$term$ and add the three polymorphic constants of this class.
\index{examples!of theories}\index{constants!overloaded}
\begin{ttbox}
Arith = FOL +
classes arith < term
consts  "0"     :: 'a::arith                  ("0")
        "1"     :: 'a::arith                  ("1")
        "+"     :: ['a::arith,'a] => 'a       (infixl 60)
end
\end{ttbox}
No rules are declared for these constants: we merely introduce their
names without specifying properties.  On the other hand, classes
with rules make it possible to prove {\bf generic} theorems.  Such
theorems hold for all instances, all types in that class.

We can now obtain distinct versions of the constants of $arith$ by
declaring certain types to be of class $arith$.  For example, let us
declare the 0-place type constructors $bool$ and $nat$:
\index{examples!of theories}
\begin{ttbox}
BoolNat = Arith +
types   bool  nat
arities bool, nat   :: arith
consts  Suc         :: nat=>nat
\ttbreak
rules   add0        "0 + n = n::nat"
        addS        "Suc(m)+n = Suc(m+n)"
        nat1        "1 = Suc(0)"
        or0l        "0 + x = x::bool"
        or0r        "x + 0 = x::bool"
        or1l        "1 + x = 1::bool"
        or1r        "x + 1 = 1::bool"
end
\end{ttbox}
Because $nat$ and $bool$ have class $arith$, we can use $0$, $1$ and $+$ at
either type.  The type constraints in the axioms are vital.  Without
constraints, the $x$ in $1+x = x$ would have type $\alpha{::}arith$
and the axiom would hold for any type of class $arith$.  This would
collapse $nat$ to a trivial type:
\[ Suc(1) = Suc(0+1) = Suc(0)+1 = 1+1 = 1! \]


\section{Theory example: the natural numbers}

We shall now work through a small example of formalized mathematics
demonstrating many of the theory extension features.


\subsection{Extending first-order logic with the natural numbers}
\index{examples!of theories}

Section\ts\ref{sec:logical-syntax} has formalized a first-order logic,
including a type~$nat$ and the constants $0::nat$ and $Suc::nat\To nat$.
Let us introduce the Peano axioms for mathematical induction and the
freeness of $0$ and~$Suc$:\index{axioms!Peano}
\[ \vcenter{\infer[(induct)]{P[n/x]}{P[0/x] & \infer*{P[Suc(x)/x]}{[P]}}}
 \qquad \parbox{4.5cm}{provided $x$ is not free in any assumption except~$P$}
\]
\[ \infer[(Suc\_inject)]{m=n}{Suc(m)=Suc(n)} \qquad
   \infer[(Suc\_neq\_0)]{R}{Suc(m)=0}
\]
Mathematical induction asserts that $P(n)$ is true, for any $n::nat$,
provided $P(0)$ holds and that $P(x)$ implies $P(Suc(x))$ for all~$x$.
Some authors express the induction step as $\forall x. P(x)\imp P(Suc(x))$.
To avoid making induction require the presence of other connectives, we
formalize mathematical induction as
$$ \List{P(0); \Forall x. P(x)\Imp P(Suc(x))} \Imp P(n). \eqno(induct) $$

\noindent
Similarly, to avoid expressing the other rules using~$\forall$, $\imp$
and~$\neg$, we take advantage of the meta-logic;\footnote
{On the other hand, the axioms $Suc(m)=Suc(n) \bimp m=n$
and $\neg(Suc(m)=0)$ are logically equivalent to those given, and work
better with Isabelle's simplifier.} 
$(Suc\_neq\_0)$ is
an elimination rule for $Suc(m)=0$:
$$ Suc(m)=Suc(n) \Imp m=n  \eqno(Suc\_inject) $$
$$ Suc(m)=0      \Imp R    \eqno(Suc\_neq\_0) $$

\noindent
We shall also define a primitive recursion operator, $rec$.  Traditionally,
primitive recursion takes a natural number~$a$ and a 2-place function~$f$,
and obeys the equations
\begin{eqnarray*}
  rec(0,a,f)            & = & a \\
  rec(Suc(m),a,f)       & = & f(m, rec(m,a,f))
\end{eqnarray*}
Addition, defined by $m+n \equiv rec(m,n,\lambda x\,y.Suc(y))$,
should satisfy
\begin{eqnarray*}
  0+n      & = & n \\
  Suc(m)+n & = & Suc(m+n)
\end{eqnarray*}
Primitive recursion appears to pose difficulties: first-order logic has no
function-valued expressions.  We again take advantage of the meta-logic,
which does have functions.  We also generalise primitive recursion to be
polymorphic over any type of class~$term$, and declare the addition
function:
\begin{eqnarray*}
  rec   & :: & [nat, \alpha{::}term, [nat,\alpha]\To\alpha] \To\alpha \\
  +     & :: & [nat,nat]\To nat 
\end{eqnarray*}


\subsection{Declaring the theory to Isabelle}
\index{examples!of theories}
Let us create the theory \thydx{Nat} starting from theory~\verb$FOL$,
which contains only classical logic with no natural numbers.  We declare
the 0-place type constructor $nat$ and the associated constants.  Note that
the constant~0 requires a mixfix annotation because~0 is not a legal
identifier, and could not otherwise be written in terms:
\begin{ttbox}\index{mixfix declarations}
Nat = FOL +
types   nat
arities nat         :: term
consts  "0"         :: nat                              ("0")
        Suc         :: nat=>nat
        rec         :: [nat, 'a, [nat,'a]=>'a] => 'a
        "+"         :: [nat, nat] => nat                (infixl 60)
rules   Suc_inject  "Suc(m)=Suc(n) ==> m=n"
        Suc_neq_0   "Suc(m)=0      ==> R"
        induct      "[| P(0);  !!x. P(x) ==> P(Suc(x)) |]  ==> P(n)"
        rec_0       "rec(0,a,f) = a"
        rec_Suc     "rec(Suc(m), a, f) = f(m, rec(m,a,f))"
        add_def     "m+n == rec(m, n, \%x y. Suc(y))"
end
\end{ttbox}
In axiom {\tt add_def}, recall that \verb|%| stands for~$\lambda$.
Loading this theory file creates the \ML\ structure {\tt Nat}, which
contains the theory and axioms.  Opening structure {\tt Nat} lets us write
{\tt induct} instead of {\tt Nat.induct}, and so forth.
\begin{ttbox}
open Nat;
\end{ttbox}

\subsection{Proving some recursion equations}
File {\tt FOL/ex/Nat.ML} contains proofs involving this theory of the
natural numbers.  As a trivial example, let us derive recursion equations
for \verb$+$.  Here is the zero case:
\begin{ttbox}
goalw Nat.thy [add_def] "0+n = n";
{\out Level 0}
{\out 0 + n = n}
{\out  1. rec(0,n,\%x y. Suc(y)) = n}
\ttbreak
by (resolve_tac [rec_0] 1);
{\out Level 1}
{\out 0 + n = n}
{\out No subgoals!}
val add_0 = result();
\end{ttbox}
And here is the successor case:
\begin{ttbox}
goalw Nat.thy [add_def] "Suc(m)+n = Suc(m+n)";
{\out Level 0}
{\out Suc(m) + n = Suc(m + n)}
{\out  1. rec(Suc(m),n,\%x y. Suc(y)) = Suc(rec(m,n,\%x y. Suc(y)))}
\ttbreak
by (resolve_tac [rec_Suc] 1);
{\out Level 1}
{\out Suc(m) + n = Suc(m + n)}
{\out No subgoals!}
val add_Suc = result();
\end{ttbox}
The induction rule raises some complications, which are discussed next.
\index{theories!defining|)}


\section{Refinement with explicit instantiation}
\index{resolution!with instantiation}
\index{instantiation|(}

In order to employ mathematical induction, we need to refine a subgoal by
the rule~$(induct)$.  The conclusion of this rule is $\Var{P}(\Var{n})$,
which is highly ambiguous in higher-order unification.  It matches every
way that a formula can be regarded as depending on a subterm of type~$nat$.
To get round this problem, we could make the induction rule conclude
$\forall n.\Var{P}(n)$ --- but putting a subgoal into this form requires
refinement by~$(\forall E)$, which is equally hard!

The tactic {\tt res_inst_tac}, like {\tt resolve_tac}, refines a subgoal by
a rule.  But it also accepts explicit instantiations for the rule's
schematic variables.  
\begin{description}
\item[\ttindex{res_inst_tac} {\it insts} {\it thm} {\it i}]
instantiates the rule {\it thm} with the instantiations {\it insts}, and
then performs resolution on subgoal~$i$.

\item[\ttindex{eres_inst_tac}] 
and \ttindex{dres_inst_tac} are similar, but perform elim-resolution
and destruct-resolution, respectively.
\end{description}
The list {\it insts} consists of pairs $[(v@1,e@1), \ldots, (v@n,e@n)]$,
where $v@1$, \ldots, $v@n$ are names of schematic variables in the rule ---
with no leading question marks! --- and $e@1$, \ldots, $e@n$ are
expressions giving their instantiations.  The expressions are type-checked
in the context of a particular subgoal: free variables receive the same
types as they have in the subgoal, and parameters may appear.  Type
variable instantiations may appear in~{\it insts}, but they are seldom
required: {\tt res_inst_tac} instantiates type variables automatically
whenever the type of~$e@i$ is an instance of the type of~$\Var{v@i}$.

\subsection{A simple proof by induction}
\index{examples!of induction}
Let us prove that no natural number~$k$ equals its own successor.  To
use~$(induct)$, we instantiate~$\Var{n}$ to~$k$; Isabelle finds a good
instantiation for~$\Var{P}$.
\begin{ttbox}
goal Nat.thy "~ (Suc(k) = k)";
{\out Level 0}
{\out Suc(k) ~= k}
{\out  1. Suc(k) ~= k}
\ttbreak
by (res_inst_tac [("n","k")] induct 1);
{\out Level 1}
{\out Suc(k) ~= k}
{\out  1. Suc(0) ~= 0}
{\out  2. !!x. Suc(x) ~= x ==> Suc(Suc(x)) ~= Suc(x)}
\end{ttbox}
We should check that Isabelle has correctly applied induction.  Subgoal~1
is the base case, with $k$ replaced by~0.  Subgoal~2 is the inductive step,
with $k$ replaced by~$Suc(x)$ and with an induction hypothesis for~$x$.
The rest of the proof demonstrates~\tdx{notI}, \tdx{notE} and the
other rules of theory {\tt Nat}.  The base case holds by~\ttindex{Suc_neq_0}:
\begin{ttbox}
by (resolve_tac [notI] 1);
{\out Level 2}
{\out Suc(k) ~= k}
{\out  1. Suc(0) = 0 ==> False}
{\out  2. !!x. Suc(x) ~= x ==> Suc(Suc(x)) ~= Suc(x)}
\ttbreak
by (eresolve_tac [Suc_neq_0] 1);
{\out Level 3}
{\out Suc(k) ~= k}
{\out  1. !!x. Suc(x) ~= x ==> Suc(Suc(x)) ~= Suc(x)}
\end{ttbox}
The inductive step holds by the contrapositive of~\ttindex{Suc_inject}.
Negation rules transform the subgoal into that of proving $Suc(x)=x$ from
$Suc(Suc(x)) = Suc(x)$:
\begin{ttbox}
by (resolve_tac [notI] 1);
{\out Level 4}
{\out Suc(k) ~= k}
{\out  1. !!x. [| Suc(x) ~= x; Suc(Suc(x)) = Suc(x) |] ==> False}
\ttbreak
by (eresolve_tac [notE] 1);
{\out Level 5}
{\out Suc(k) ~= k}
{\out  1. !!x. Suc(Suc(x)) = Suc(x) ==> Suc(x) = x}
\ttbreak
by (eresolve_tac [Suc_inject] 1);
{\out Level 6}
{\out Suc(k) ~= k}
{\out No subgoals!}
\end{ttbox}


\subsection{An example of ambiguity in {\tt resolve_tac}}
\index{examples!of induction}\index{unification!higher-order}
If you try the example above, you may observe that {\tt res_inst_tac} is
not actually needed.  Almost by chance, \ttindex{resolve_tac} finds the right
instantiation for~$(induct)$ to yield the desired next state.  With more
complex formulae, our luck fails.  
\begin{ttbox}
goal Nat.thy "(k+m)+n = k+(m+n)";
{\out Level 0}
{\out k + m + n = k + (m + n)}
{\out  1. k + m + n = k + (m + n)}
\ttbreak
by (resolve_tac [induct] 1);
{\out Level 1}
{\out k + m + n = k + (m + n)}
{\out  1. k + m + n = 0}
{\out  2. !!x. k + m + n = x ==> k + m + n = Suc(x)}
\end{ttbox}
This proof requires induction on~$k$.  The occurrence of~0 in subgoal~1
indicates that induction has been applied to the term~$k+(m+n)$; this
application is sound but will not lead to a proof here.  Fortunately,
Isabelle can (lazily!) generate all the valid applications of induction.
The \ttindex{back} command causes backtracking to an alternative outcome of
the tactic.
\begin{ttbox}
back();
{\out Level 1}
{\out k + m + n = k + (m + n)}
{\out  1. k + m + n = k + 0}
{\out  2. !!x. k + m + n = k + x ==> k + m + n = k + Suc(x)}
\end{ttbox}
Now induction has been applied to~$m+n$.  This is equally useless.  Let us
call \ttindex{back} again.
\begin{ttbox}
back();
{\out Level 1}
{\out k + m + n = k + (m + n)}
{\out  1. k + m + 0 = k + (m + 0)}
{\out  2. !!x. k + m + x = k + (m + x) ==>}
{\out          k + m + Suc(x) = k + (m + Suc(x))}
\end{ttbox}
Now induction has been applied to~$n$.  What is the next alternative?
\begin{ttbox}
back();
{\out Level 1}
{\out k + m + n = k + (m + n)}
{\out  1. k + m + n = k + (m + 0)}
{\out  2. !!x. k + m + n = k + (m + x) ==> k + m + n = k + (m + Suc(x))}
\end{ttbox}
Inspecting subgoal~1 reveals that induction has been applied to just the
second occurrence of~$n$.  This perfectly legitimate induction is useless
here.  

The main goal admits fourteen different applications of induction.  The
number is exponential in the size of the formula.

\subsection{Proving that addition is associative}
Let us invoke the induction rule properly, using~{\tt
  res_inst_tac}.  At the same time, we shall have a glimpse at Isabelle's
simplification tactics, which are described in 
\iflabelundefined{simp-chap}%
    {the {\em Reference Manual}}{Chap.\ts\ref{simp-chap}}.

\index{simplification}\index{examples!of simplification} 

Isabelle's simplification tactics repeatedly apply equations to a subgoal,
perhaps proving it.  For efficiency, the rewrite rules must be
packaged into a {\bf simplification set},\index{simplification sets} 
or {\bf simpset}.  We take the standard simpset for first-order logic and
insert the equations proved in the previous section, namely
$0+n=n$ and ${\tt Suc}(m)+n={\tt Suc}(m+n)$:
\begin{ttbox}
val add_ss = FOL_ss addsimps [add_0, add_Suc];
\end{ttbox}
We state the goal for associativity of addition, and
use \ttindex{res_inst_tac} to invoke induction on~$k$:
\begin{ttbox}
goal Nat.thy "(k+m)+n = k+(m+n)";
{\out Level 0}
{\out k + m + n = k + (m + n)}
{\out  1. k + m + n = k + (m + n)}
\ttbreak
by (res_inst_tac [("n","k")] induct 1);
{\out Level 1}
{\out k + m + n = k + (m + n)}
{\out  1. 0 + m + n = 0 + (m + n)}
{\out  2. !!x. x + m + n = x + (m + n) ==>}
{\out          Suc(x) + m + n = Suc(x) + (m + n)}
\end{ttbox}
The base case holds easily; both sides reduce to $m+n$.  The
tactic~\ttindex{simp_tac} rewrites with respect to the given simplification
set, applying the rewrite rules for addition:
\begin{ttbox}
by (simp_tac add_ss 1);
{\out Level 2}
{\out k + m + n = k + (m + n)}
{\out  1. !!x. x + m + n = x + (m + n) ==>}
{\out          Suc(x) + m + n = Suc(x) + (m + n)}
\end{ttbox}
The inductive step requires rewriting by the equations for addition
together the induction hypothesis, which is also an equation.  The
tactic~\ttindex{asm_simp_tac} rewrites using a simplification set and any
useful assumptions:
\begin{ttbox}
by (asm_simp_tac add_ss 1);
{\out Level 3}
{\out k + m + n = k + (m + n)}
{\out No subgoals!}
\end{ttbox}
\index{instantiation|)}


\section{A Prolog interpreter}
\index{Prolog interpreter|bold}
To demonstrate the power of tacticals, let us construct a Prolog
interpreter and execute programs involving lists.\footnote{To run these
examples, see the file {\tt FOL/ex/Prolog.ML}.} The Prolog program
consists of a theory.  We declare a type constructor for lists, with an
arity declaration to say that $(\tau)list$ is of class~$term$
provided~$\tau$ is:
\begin{eqnarray*}
  list  & :: & (term)term
\end{eqnarray*}
We declare four constants: the empty list~$Nil$; the infix list
constructor~{:}; the list concatenation predicate~$app$; the list reverse
predicate~$rev$.  (In Prolog, functions on lists are expressed as
predicates.)
\begin{eqnarray*}
    Nil         & :: & \alpha list \\
    {:}         & :: & [\alpha,\alpha list] \To \alpha list \\
    app & :: & [\alpha list,\alpha list,\alpha list] \To o \\
    rev & :: & [\alpha list,\alpha list] \To o 
\end{eqnarray*}
The predicate $app$ should satisfy the Prolog-style rules
\[ {app(Nil,ys,ys)} \qquad
   {app(xs,ys,zs) \over app(x:xs, ys, x:zs)} \]
We define the naive version of $rev$, which calls~$app$:
\[ {rev(Nil,Nil)} \qquad
   {rev(xs,ys)\quad  app(ys, x:Nil, zs) \over
    rev(x:xs, zs)} 
\]

\index{examples!of theories}
Theory \thydx{Prolog} extends first-order logic in order to make use
of the class~$term$ and the type~$o$.  The interpreter does not use the
rules of~{\tt FOL}.
\begin{ttbox}
Prolog = FOL +
types   'a list
arities list    :: (term)term
consts  Nil     :: 'a list
        ":"     :: ['a, 'a list]=> 'a list            (infixr 60)
        app     :: ['a list, 'a list, 'a list] => o
        rev     :: ['a list, 'a list] => o
rules   appNil  "app(Nil,ys,ys)"
        appCons "app(xs,ys,zs) ==> app(x:xs, ys, x:zs)"
        revNil  "rev(Nil,Nil)"
        revCons "[| rev(xs,ys); app(ys,x:Nil,zs) |] ==> rev(x:xs,zs)"
end
\end{ttbox}
\subsection{Simple executions}
Repeated application of the rules solves Prolog goals.  Let us
append the lists $[a,b,c]$ and~$[d,e]$.  As the rules are applied, the
answer builds up in~{\tt ?x}.
\begin{ttbox}
goal Prolog.thy "app(a:b:c:Nil, d:e:Nil, ?x)";
{\out Level 0}
{\out app(a : b : c : Nil, d : e : Nil, ?x)}
{\out  1. app(a : b : c : Nil, d : e : Nil, ?x)}
\ttbreak
by (resolve_tac [appNil,appCons] 1);
{\out Level 1}
{\out app(a : b : c : Nil, d : e : Nil, a : ?zs1)}
{\out  1. app(b : c : Nil, d : e : Nil, ?zs1)}
\ttbreak
by (resolve_tac [appNil,appCons] 1);
{\out Level 2}
{\out app(a : b : c : Nil, d : e : Nil, a : b : ?zs2)}
{\out  1. app(c : Nil, d : e : Nil, ?zs2)}
\end{ttbox}
At this point, the first two elements of the result are~$a$ and~$b$.
\begin{ttbox}
by (resolve_tac [appNil,appCons] 1);
{\out Level 3}
{\out app(a : b : c : Nil, d : e : Nil, a : b : c : ?zs3)}
{\out  1. app(Nil, d : e : Nil, ?zs3)}
\ttbreak
by (resolve_tac [appNil,appCons] 1);
{\out Level 4}
{\out app(a : b : c : Nil, d : e : Nil, a : b : c : d : e : Nil)}
{\out No subgoals!}
\end{ttbox}

Prolog can run functions backwards.  Which list can be appended
with $[c,d]$ to produce $[a,b,c,d]$?
Using \ttindex{REPEAT}, we find the answer at once, $[a,b]$:
\begin{ttbox}
goal Prolog.thy "app(?x, c:d:Nil, a:b:c:d:Nil)";
{\out Level 0}
{\out app(?x, c : d : Nil, a : b : c : d : Nil)}
{\out  1. app(?x, c : d : Nil, a : b : c : d : Nil)}
\ttbreak
by (REPEAT (resolve_tac [appNil,appCons] 1));
{\out Level 1}
{\out app(a : b : Nil, c : d : Nil, a : b : c : d : Nil)}
{\out No subgoals!}
\end{ttbox}


\subsection{Backtracking}\index{backtracking!Prolog style}
Prolog backtracking can answer questions that have multiple solutions.
Which lists $x$ and $y$ can be appended to form the list $[a,b,c,d]$?  This
question has five solutions.  Using \ttindex{REPEAT} to apply the rules, we
quickly find the first solution, namely $x=[]$ and $y=[a,b,c,d]$:
\begin{ttbox}
goal Prolog.thy "app(?x, ?y, a:b:c:d:Nil)";
{\out Level 0}
{\out app(?x, ?y, a : b : c : d : Nil)}
{\out  1. app(?x, ?y, a : b : c : d : Nil)}
\ttbreak
by (REPEAT (resolve_tac [appNil,appCons] 1));
{\out Level 1}
{\out app(Nil, a : b : c : d : Nil, a : b : c : d : Nil)}
{\out No subgoals!}
\end{ttbox}
Isabelle can lazily generate all the possibilities.  The \ttindex{back}
command returns the tactic's next outcome, namely $x=[a]$ and $y=[b,c,d]$:
\begin{ttbox}
back();
{\out Level 1}
{\out app(a : Nil, b : c : d : Nil, a : b : c : d : Nil)}
{\out No subgoals!}
\end{ttbox}
The other solutions are generated similarly.
\begin{ttbox}
back();
{\out Level 1}
{\out app(a : b : Nil, c : d : Nil, a : b : c : d : Nil)}
{\out No subgoals!}
\ttbreak
back();
{\out Level 1}
{\out app(a : b : c : Nil, d : Nil, a : b : c : d : Nil)}
{\out No subgoals!}
\ttbreak
back();
{\out Level 1}
{\out app(a : b : c : d : Nil, Nil, a : b : c : d : Nil)}
{\out No subgoals!}
\end{ttbox}


\subsection{Depth-first search}
\index{search!depth-first}
Now let us try $rev$, reversing a list.
Bundle the rules together as the \ML{} identifier {\tt rules}.  Naive
reverse requires 120 inferences for this 14-element list, but the tactic
terminates in a few seconds.
\begin{ttbox}
goal Prolog.thy "rev(a:b:c:d:e:f:g:h:i:j:k:l:m:n:Nil, ?w)";
{\out Level 0}
{\out rev(a : b : c : d : e : f : g : h : i : j : k : l : m : n : Nil, ?w)}
{\out  1. rev(a : b : c : d : e : f : g : h : i : j : k : l : m : n : Nil,}
{\out         ?w)}
\ttbreak
val rules = [appNil,appCons,revNil,revCons];
\ttbreak
by (REPEAT (resolve_tac rules 1));
{\out Level 1}
{\out rev(a : b : c : d : e : f : g : h : i : j : k : l : m : n : Nil,}
{\out     n : m : l : k : j : i : h : g : f : e : d : c : b : a : Nil)}
{\out No subgoals!}
\end{ttbox}
We may execute $rev$ backwards.  This, too, should reverse a list.  What
is the reverse of $[a,b,c]$?
\begin{ttbox}
goal Prolog.thy "rev(?x, a:b:c:Nil)";
{\out Level 0}
{\out rev(?x, a : b : c : Nil)}
{\out  1. rev(?x, a : b : c : Nil)}
\ttbreak
by (REPEAT (resolve_tac rules 1));
{\out Level 1}
{\out rev(?x1 : Nil, a : b : c : Nil)}
{\out  1. app(Nil, ?x1 : Nil, a : b : c : Nil)}
\end{ttbox}
The tactic has failed to find a solution!  It reached a dead end at
subgoal~1: there is no~$\Var{x@1}$ such that [] appended with~$[\Var{x@1}]$
equals~$[a,b,c]$.  Backtracking explores other outcomes.
\begin{ttbox}
back();
{\out Level 1}
{\out rev(?x1 : a : Nil, a : b : c : Nil)}
{\out  1. app(Nil, ?x1 : Nil, b : c : Nil)}
\end{ttbox}
This too is a dead end, but the next outcome is successful.
\begin{ttbox}
back();
{\out Level 1}
{\out rev(c : b : a : Nil, a : b : c : Nil)}
{\out No subgoals!}
\end{ttbox}
\ttindex{REPEAT} goes wrong because it is only a repetition tactical, not a
search tactical.  {\tt REPEAT} stops when it cannot continue, regardless of
which state is reached.  The tactical \ttindex{DEPTH_FIRST} searches for a
satisfactory state, as specified by an \ML{} predicate.  Below,
\ttindex{has_fewer_prems} specifies that the proof state should have no
subgoals.
\begin{ttbox}
val prolog_tac = DEPTH_FIRST (has_fewer_prems 1) 
                             (resolve_tac rules 1);
\end{ttbox}
Since Prolog uses depth-first search, this tactic is a (slow!) 
Prolog interpreter.  We return to the start of the proof using
\ttindex{choplev}, and apply {\tt prolog_tac}:
\begin{ttbox}
choplev 0;
{\out Level 0}
{\out rev(?x, a : b : c : Nil)}
{\out  1. rev(?x, a : b : c : Nil)}
\ttbreak
by (DEPTH_FIRST (has_fewer_prems 1) (resolve_tac rules 1));
{\out Level 1}
{\out rev(c : b : a : Nil, a : b : c : Nil)}
{\out No subgoals!}
\end{ttbox}
Let us try {\tt prolog_tac} on one more example, containing four unknowns:
\begin{ttbox}
goal Prolog.thy "rev(a:?x:c:?y:Nil, d:?z:b:?u)";
{\out Level 0}
{\out rev(a : ?x : c : ?y : Nil, d : ?z : b : ?u)}
{\out  1. rev(a : ?x : c : ?y : Nil, d : ?z : b : ?u)}
\ttbreak
by prolog_tac;
{\out Level 1}
{\out rev(a : b : c : d : Nil, d : c : b : a : Nil)}
{\out No subgoals!}
\end{ttbox}
Although Isabelle is much slower than a Prolog system, Isabelle
tactics can exploit logic programming techniques.  

