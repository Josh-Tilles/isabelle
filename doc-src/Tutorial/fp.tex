\chapter{Functional Programming in HOL}

Although on the surface this chapter is mainly concerned with how to write
functional programs in HOL and how to verify them, most of the
constructs and proof procedures introduced are general purpose and recur in
any specification or verification task.

The dedicated functional programmer should be warned: HOL offers only what
could be called {\em total functional programming} --- all functions in HOL
must be total; lazy data structures are not directly available. On the
positive side, functions in HOL need not be computable: HOL is a
specification language that goes well beyond what can be expressed as a
program. However, for the time being we concentrate on the computable.

\section{An introductory theory}
\label{sec:intro-theory}

Functional programming needs datatypes and functions. Both of them can be
defined in a theory with a syntax reminiscent of languages like ML or
Haskell. As an example consider the theory in Fig.~\ref{fig:ToyList}.

\begin{figure}[htbp]
\begin{ttbox}\makeatother
%
\begin{isabellebody}%
\def\isabellecontext{ToyList}%
%
\isadelimtheory
%
\endisadelimtheory
%
\isatagtheory
\isacommand{theory}\isamarkupfalse%
\ ToyList\isanewline
\isakeyword{imports}\ PreList\isanewline
\isakeyword{begin}%
\endisatagtheory
{\isafoldtheory}%
%
\isadelimtheory
%
\endisadelimtheory
%
\begin{isamarkuptext}%
\noindent
HOL already has a predefined theory of lists called \isa{List} ---
\isa{ToyList} is merely a small fragment of it chosen as an example. In
contrast to what is recommended in \S\ref{sec:Basic:Theories},
\isa{ToyList} is not based on \isa{Main} but on \isa{PreList}, a
theory that contains pretty much everything but lists, thus avoiding
ambiguities caused by defining lists twice.%
\end{isamarkuptext}%
\isamarkuptrue%
\isacommand{datatype}\isamarkupfalse%
\ {\isacharprime}a\ list\ {\isacharequal}\ Nil\ \ \ \ \ \ \ \ \ \ \ \ \ \ \ \ \ \ \ \ \ \ \ \ \ \ {\isacharparenleft}{\isachardoublequoteopen}{\isacharbrackleft}{\isacharbrackright}{\isachardoublequoteclose}{\isacharparenright}\isanewline
\ \ \ \ \ \ \ \ \ \ \ \ \ \ \ \ \ {\isacharbar}\ Cons\ {\isacharprime}a\ {\isachardoublequoteopen}{\isacharprime}a\ list{\isachardoublequoteclose}\ \ \ \ \ \ \ \ \ \ \ \ {\isacharparenleft}\isakeyword{infixr}\ {\isachardoublequoteopen}{\isacharhash}{\isachardoublequoteclose}\ {\isadigit{6}}{\isadigit{5}}{\isacharparenright}%
\begin{isamarkuptext}%
\noindent
The datatype\index{datatype@\isacommand {datatype} (command)}
\tydx{list} introduces two
constructors \cdx{Nil} and \cdx{Cons}, the
empty~list and the operator that adds an element to the front of a list. For
example, the term \isa{Cons True (Cons False Nil)} is a value of
type \isa{bool\ list}, namely the list with the elements \isa{True} and
\isa{False}. Because this notation quickly becomes unwieldy, the
datatype declaration is annotated with an alternative syntax: instead of
\isa{Nil} and \isa{Cons x xs} we can write
\isa{{\isacharbrackleft}{\isacharbrackright}}\index{$HOL2list@\isa{[]}|bold} and
\isa{x\ {\isacharhash}\ xs}\index{$HOL2list@\isa{\#}|bold}. In fact, this
alternative syntax is the familiar one.  Thus the list \isa{Cons True
(Cons False Nil)} becomes \isa{True\ {\isacharhash}\ False\ {\isacharhash}\ {\isacharbrackleft}{\isacharbrackright}}. The annotation
\isacommand{infixr}\index{infixr@\isacommand{infixr} (annotation)} 
means that \isa{{\isacharhash}} associates to
the right: the term \isa{x\ {\isacharhash}\ y\ {\isacharhash}\ z} is read as \isa{x\ {\isacharhash}\ {\isacharparenleft}y\ {\isacharhash}\ z{\isacharparenright}}
and not as \isa{{\isacharparenleft}x\ {\isacharhash}\ y{\isacharparenright}\ {\isacharhash}\ z}.
The \isa{{\isadigit{6}}{\isadigit{5}}} is the priority of the infix \isa{{\isacharhash}}.

\begin{warn}
  Syntax annotations can be powerful, but they are difficult to master and 
  are never necessary.  You
  could drop them from theory \isa{ToyList} and go back to the identifiers
  \isa{Nil} and \isa{Cons}.
  Novices should avoid using
  syntax annotations in their own theories.
\end{warn}
Next, two functions \isa{app} and \cdx{rev} are declared:%
\end{isamarkuptext}%
\isamarkuptrue%
\isacommand{consts}\isamarkupfalse%
\ app\ {\isacharcolon}{\isacharcolon}\ {\isachardoublequoteopen}{\isacharprime}a\ list\ {\isasymRightarrow}\ {\isacharprime}a\ list\ {\isasymRightarrow}\ {\isacharprime}a\ list{\isachardoublequoteclose}\ \ \ {\isacharparenleft}\isakeyword{infixr}\ {\isachardoublequoteopen}{\isacharat}{\isachardoublequoteclose}\ {\isadigit{6}}{\isadigit{5}}{\isacharparenright}\isanewline
\ \ \ \ \ \ \ rev\ {\isacharcolon}{\isacharcolon}\ {\isachardoublequoteopen}{\isacharprime}a\ list\ {\isasymRightarrow}\ {\isacharprime}a\ list{\isachardoublequoteclose}%
\begin{isamarkuptext}%
\noindent
In contrast to many functional programming languages,
Isabelle insists on explicit declarations of all functions
(keyword \commdx{consts}).  Apart from the declaration-before-use
restriction, the order of items in a theory file is unconstrained. Function
\isa{app} is annotated with concrete syntax too. Instead of the
prefix syntax \isa{app\ xs\ ys} the infix
\isa{xs\ {\isacharat}\ ys}\index{$HOL2list@\isa{\at}|bold} becomes the preferred
form. Both functions are defined recursively:%
\end{isamarkuptext}%
\isamarkuptrue%
\isacommand{primrec}\isamarkupfalse%
\isanewline
{\isachardoublequoteopen}{\isacharbrackleft}{\isacharbrackright}\ {\isacharat}\ ys\ \ \ \ \ \ \ {\isacharequal}\ ys{\isachardoublequoteclose}\isanewline
{\isachardoublequoteopen}{\isacharparenleft}x\ {\isacharhash}\ xs{\isacharparenright}\ {\isacharat}\ ys\ {\isacharequal}\ x\ {\isacharhash}\ {\isacharparenleft}xs\ {\isacharat}\ ys{\isacharparenright}{\isachardoublequoteclose}\isanewline
\isanewline
\isacommand{primrec}\isamarkupfalse%
\isanewline
{\isachardoublequoteopen}rev\ {\isacharbrackleft}{\isacharbrackright}\ \ \ \ \ \ \ \ {\isacharequal}\ {\isacharbrackleft}{\isacharbrackright}{\isachardoublequoteclose}\isanewline
{\isachardoublequoteopen}rev\ {\isacharparenleft}x\ {\isacharhash}\ xs{\isacharparenright}\ \ {\isacharequal}\ {\isacharparenleft}rev\ xs{\isacharparenright}\ {\isacharat}\ {\isacharparenleft}x\ {\isacharhash}\ {\isacharbrackleft}{\isacharbrackright}{\isacharparenright}{\isachardoublequoteclose}%
\begin{isamarkuptext}%
\noindent\index{*rev (constant)|(}\index{append function|(}
The equations for \isa{app} and \isa{rev} hardly need comments:
\isa{app} appends two lists and \isa{rev} reverses a list.  The
keyword \commdx{primrec} indicates that the recursion is
of a particularly primitive kind where each recursive call peels off a datatype
constructor from one of the arguments.  Thus the
recursion always terminates, i.e.\ the function is \textbf{total}.
\index{functions!total}

The termination requirement is absolutely essential in HOL, a logic of total
functions. If we were to drop it, inconsistencies would quickly arise: the
``definition'' $f(n) = f(n)+1$ immediately leads to $0 = 1$ by subtracting
$f(n)$ on both sides.
% However, this is a subtle issue that we cannot discuss here further.

\begin{warn}
  As we have indicated, the requirement for total functions is an essential characteristic of HOL\@. It is only
  because of totality that reasoning in HOL is comparatively easy.  More
  generally, the philosophy in HOL is to refrain from asserting arbitrary axioms (such as
  function definitions whose totality has not been proved) because they
  quickly lead to inconsistencies. Instead, fixed constructs for introducing
  types and functions are offered (such as \isacommand{datatype} and
  \isacommand{primrec}) which are guaranteed to preserve consistency.
\end{warn}

\index{syntax}%
A remark about syntax.  The textual definition of a theory follows a fixed
syntax with keywords like \isacommand{datatype} and \isacommand{end}.
% (see Fig.~\ref{fig:keywords} in Appendix~\ref{sec:Appendix} for a full list).
Embedded in this syntax are the types and formulae of HOL, whose syntax is
extensible (see \S\ref{sec:concrete-syntax}), e.g.\ by new user-defined infix operators.
To distinguish the two levels, everything
HOL-specific (terms and types) should be enclosed in
\texttt{"}\dots\texttt{"}. 
To lessen this burden, quotation marks around a single identifier can be
dropped, unless the identifier happens to be a keyword, as in%
\end{isamarkuptext}%
\isamarkuptrue%
\isacommand{consts}\isamarkupfalse%
\ {\isachardoublequoteopen}end{\isachardoublequoteclose}\ {\isacharcolon}{\isacharcolon}\ {\isachardoublequoteopen}{\isacharprime}a\ list\ {\isasymRightarrow}\ {\isacharprime}a{\isachardoublequoteclose}%
\begin{isamarkuptext}%
\noindent
When Isabelle prints a syntax error message, it refers to the HOL syntax as
the \textbf{inner syntax} and the enclosing theory language as the \textbf{outer syntax}.


\section{An Introductory Proof}
\label{sec:intro-proof}

Assuming you have processed the declarations and definitions of
\texttt{ToyList} presented so far, we are ready to prove a few simple
theorems. This will illustrate not just the basic proof commands but
also the typical proof process.

\subsubsection*{Main Goal.}

Our goal is to show that reversing a list twice produces the original
list.%
\end{isamarkuptext}%
\isamarkuptrue%
\isacommand{theorem}\isamarkupfalse%
\ rev{\isacharunderscore}rev\ {\isacharbrackleft}simp{\isacharbrackright}{\isacharcolon}\ {\isachardoublequoteopen}rev{\isacharparenleft}rev\ xs{\isacharparenright}\ {\isacharequal}\ xs{\isachardoublequoteclose}%
\isadelimproof
%
\endisadelimproof
%
\isatagproof
%
\begin{isamarkuptxt}%
\index{theorem@\isacommand {theorem} (command)|bold}%
\noindent
This \isacommand{theorem} command does several things:
\begin{itemize}
\item
It establishes a new theorem to be proved, namely \isa{rev\ {\isacharparenleft}rev\ xs{\isacharparenright}\ {\isacharequal}\ xs}.
\item
It gives that theorem the name \isa{rev{\isacharunderscore}rev}, for later reference.
\item
It tells Isabelle (via the bracketed attribute \attrdx{simp}) to take the eventual theorem as a simplification rule: future proofs involving
simplification will replace occurrences of \isa{rev\ {\isacharparenleft}rev\ xs{\isacharparenright}} by
\isa{xs}.
\end{itemize}
The name and the simplification attribute are optional.
Isabelle's response is to print the initial proof state consisting
of some header information (like how many subgoals there are) followed by
\begin{isabelle}%
\ {\isadigit{1}}{\isachardot}\ rev\ {\isacharparenleft}rev\ xs{\isacharparenright}\ {\isacharequal}\ xs%
\end{isabelle}
For compactness reasons we omit the header in this tutorial.
Until we have finished a proof, the \rmindex{proof state} proper
always looks like this:
\begin{isabelle}
~1.~$G\sb{1}$\isanewline
~~\vdots~~\isanewline
~$n$.~$G\sb{n}$
\end{isabelle}
The numbered lines contain the subgoals $G\sb{1}$, \dots, $G\sb{n}$
that we need to prove to establish the main goal.\index{subgoals}
Initially there is only one subgoal, which is identical with the
main goal. (If you always want to see the main goal as well,
set the flag \isa{Proof.show_main_goal}\index{*show_main_goal (flag)}
--- this flag used to be set by default.)

Let us now get back to \isa{rev\ {\isacharparenleft}rev\ xs{\isacharparenright}\ {\isacharequal}\ xs}. Properties of recursively
defined functions are best established by induction. In this case there is
nothing obvious except induction on \isa{xs}:%
\end{isamarkuptxt}%
\isamarkuptrue%
\isacommand{apply}\isamarkupfalse%
{\isacharparenleft}induct{\isacharunderscore}tac\ xs{\isacharparenright}%
\begin{isamarkuptxt}%
\noindent\index{*induct_tac (method)}%
This tells Isabelle to perform induction on variable \isa{xs}. The suffix
\isa{tac} stands for \textbf{tactic},\index{tactics}
a synonym for ``theorem proving function''.
By default, induction acts on the first subgoal. The new proof state contains
two subgoals, namely the base case (\isa{Nil}) and the induction step
(\isa{Cons}):
\begin{isabelle}%
\ {\isadigit{1}}{\isachardot}\ rev\ {\isacharparenleft}rev\ {\isacharbrackleft}{\isacharbrackright}{\isacharparenright}\ {\isacharequal}\ {\isacharbrackleft}{\isacharbrackright}\isanewline
\ {\isadigit{2}}{\isachardot}\ {\isasymAnd}a\ list{\isachardot}\isanewline
\isaindent{\ {\isadigit{2}}{\isachardot}\ \ \ \ }rev\ {\isacharparenleft}rev\ list{\isacharparenright}\ {\isacharequal}\ list\ {\isasymLongrightarrow}\ rev\ {\isacharparenleft}rev\ {\isacharparenleft}a\ {\isacharhash}\ list{\isacharparenright}{\isacharparenright}\ {\isacharequal}\ a\ {\isacharhash}\ list%
\end{isabelle}

The induction step is an example of the general format of a subgoal:\index{subgoals}
\begin{isabelle}
~$i$.~{\isasymAnd}$x\sb{1}$~\dots$x\sb{n}$.~{\it assumptions}~{\isasymLongrightarrow}~{\it conclusion}
\end{isabelle}\index{$IsaAnd@\isasymAnd|bold}
The prefix of bound variables \isasymAnd$x\sb{1}$~\dots~$x\sb{n}$ can be
ignored most of the time, or simply treated as a list of variables local to
this subgoal. Their deeper significance is explained in Chapter~\ref{chap:rules}.
The {\it assumptions}\index{assumptions!of subgoal}
are the local assumptions for this subgoal and {\it
  conclusion}\index{conclusion!of subgoal} is the actual proposition to be proved. 
Typical proof steps
that add new assumptions are induction and case distinction. In our example
the only assumption is the induction hypothesis \isa{rev\ {\isacharparenleft}rev\ list{\isacharparenright}\ {\isacharequal}\ list}, where \isa{list} is a variable name chosen by Isabelle. If there
are multiple assumptions, they are enclosed in the bracket pair
\indexboldpos{\isasymlbrakk}{$Isabrl} and
\indexboldpos{\isasymrbrakk}{$Isabrr} and separated by semicolons.

Let us try to solve both goals automatically:%
\end{isamarkuptxt}%
\isamarkuptrue%
\isacommand{apply}\isamarkupfalse%
{\isacharparenleft}auto{\isacharparenright}%
\begin{isamarkuptxt}%
\noindent
This command tells Isabelle to apply a proof strategy called
\isa{auto} to all subgoals. Essentially, \isa{auto} tries to
simplify the subgoals.  In our case, subgoal~1 is solved completely (thanks
to the equation \isa{rev\ {\isacharbrackleft}{\isacharbrackright}\ {\isacharequal}\ {\isacharbrackleft}{\isacharbrackright}}) and disappears; the simplified version
of subgoal~2 becomes the new subgoal~1:
\begin{isabelle}%
\ {\isadigit{1}}{\isachardot}\ {\isasymAnd}a\ list{\isachardot}\isanewline
\isaindent{\ {\isadigit{1}}{\isachardot}\ \ \ \ }rev\ {\isacharparenleft}rev\ list{\isacharparenright}\ {\isacharequal}\ list\ {\isasymLongrightarrow}\ rev\ {\isacharparenleft}rev\ list\ {\isacharat}\ a\ {\isacharhash}\ {\isacharbrackleft}{\isacharbrackright}{\isacharparenright}\ {\isacharequal}\ a\ {\isacharhash}\ list%
\end{isabelle}
In order to simplify this subgoal further, a lemma suggests itself.%
\end{isamarkuptxt}%
\isamarkuptrue%
%
\endisatagproof
{\isafoldproof}%
%
\isadelimproof
%
\endisadelimproof
%
\isamarkupsubsubsection{First Lemma%
}
\isamarkuptrue%
%
\begin{isamarkuptext}%
\indexbold{abandoning a proof}\indexbold{proofs!abandoning}
After abandoning the above proof attempt (at the shell level type
\commdx{oops}) we start a new proof:%
\end{isamarkuptext}%
\isamarkuptrue%
\isacommand{lemma}\isamarkupfalse%
\ rev{\isacharunderscore}app\ {\isacharbrackleft}simp{\isacharbrackright}{\isacharcolon}\ {\isachardoublequoteopen}rev{\isacharparenleft}xs\ {\isacharat}\ ys{\isacharparenright}\ {\isacharequal}\ {\isacharparenleft}rev\ ys{\isacharparenright}\ {\isacharat}\ {\isacharparenleft}rev\ xs{\isacharparenright}{\isachardoublequoteclose}%
\isadelimproof
%
\endisadelimproof
%
\isatagproof
%
\begin{isamarkuptxt}%
\noindent The keywords \commdx{theorem} and
\commdx{lemma} are interchangeable and merely indicate
the importance we attach to a proposition.  Therefore we use the words
\emph{theorem} and \emph{lemma} pretty much interchangeably, too.

There are two variables that we could induct on: \isa{xs} and
\isa{ys}. Because \isa{{\isacharat}} is defined by recursion on
the first argument, \isa{xs} is the correct one:%
\end{isamarkuptxt}%
\isamarkuptrue%
\isacommand{apply}\isamarkupfalse%
{\isacharparenleft}induct{\isacharunderscore}tac\ xs{\isacharparenright}%
\begin{isamarkuptxt}%
\noindent
This time not even the base case is solved automatically:%
\end{isamarkuptxt}%
\isamarkuptrue%
\isacommand{apply}\isamarkupfalse%
{\isacharparenleft}auto{\isacharparenright}%
\begin{isamarkuptxt}%
\begin{isabelle}%
\ {\isadigit{1}}{\isachardot}\ rev\ ys\ {\isacharequal}\ rev\ ys\ {\isacharat}\ {\isacharbrackleft}{\isacharbrackright}%
\end{isabelle}
Again, we need to abandon this proof attempt and prove another simple lemma
first. In the future the step of abandoning an incomplete proof before
embarking on the proof of a lemma usually remains implicit.%
\end{isamarkuptxt}%
\isamarkuptrue%
%
\endisatagproof
{\isafoldproof}%
%
\isadelimproof
%
\endisadelimproof
%
\isamarkupsubsubsection{Second Lemma%
}
\isamarkuptrue%
%
\begin{isamarkuptext}%
We again try the canonical proof procedure:%
\end{isamarkuptext}%
\isamarkuptrue%
\isacommand{lemma}\isamarkupfalse%
\ app{\isacharunderscore}Nil{\isadigit{2}}\ {\isacharbrackleft}simp{\isacharbrackright}{\isacharcolon}\ {\isachardoublequoteopen}xs\ {\isacharat}\ {\isacharbrackleft}{\isacharbrackright}\ {\isacharequal}\ xs{\isachardoublequoteclose}\isanewline
%
\isadelimproof
%
\endisadelimproof
%
\isatagproof
\isacommand{apply}\isamarkupfalse%
{\isacharparenleft}induct{\isacharunderscore}tac\ xs{\isacharparenright}\isanewline
\isacommand{apply}\isamarkupfalse%
{\isacharparenleft}auto{\isacharparenright}%
\begin{isamarkuptxt}%
\noindent
It works, yielding the desired message \isa{No\ subgoals{\isacharbang}}:
\begin{isabelle}%
xs\ {\isacharat}\ {\isacharbrackleft}{\isacharbrackright}\ {\isacharequal}\ xs\isanewline
No\ subgoals{\isacharbang}%
\end{isabelle}
We still need to confirm that the proof is now finished:%
\end{isamarkuptxt}%
\isamarkuptrue%
\isacommand{done}\isamarkupfalse%
%
\endisatagproof
{\isafoldproof}%
%
\isadelimproof
%
\endisadelimproof
%
\begin{isamarkuptext}%
\noindent
As a result of that final \commdx{done}, Isabelle associates the lemma just proved
with its name. In this tutorial, we sometimes omit to show that final \isacommand{done}
if it is obvious from the context that the proof is finished.

% Instead of \isacommand{apply} followed by a dot, you can simply write
% \isacommand{by}\indexbold{by}, which we do most of the time.
Notice that in lemma \isa{app{\isacharunderscore}Nil{\isadigit{2}}},
as printed out after the final \isacommand{done}, the free variable \isa{xs} has been
replaced by the unknown \isa{{\isacharquery}xs}, just as explained in
\S\ref{sec:variables}.

Going back to the proof of the first lemma%
\end{isamarkuptext}%
\isamarkuptrue%
\isacommand{lemma}\isamarkupfalse%
\ rev{\isacharunderscore}app\ {\isacharbrackleft}simp{\isacharbrackright}{\isacharcolon}\ {\isachardoublequoteopen}rev{\isacharparenleft}xs\ {\isacharat}\ ys{\isacharparenright}\ {\isacharequal}\ {\isacharparenleft}rev\ ys{\isacharparenright}\ {\isacharat}\ {\isacharparenleft}rev\ xs{\isacharparenright}{\isachardoublequoteclose}\isanewline
%
\isadelimproof
%
\endisadelimproof
%
\isatagproof
\isacommand{apply}\isamarkupfalse%
{\isacharparenleft}induct{\isacharunderscore}tac\ xs{\isacharparenright}\isanewline
\isacommand{apply}\isamarkupfalse%
{\isacharparenleft}auto{\isacharparenright}%
\begin{isamarkuptxt}%
\noindent
we find that this time \isa{auto} solves the base case, but the
induction step merely simplifies to
\begin{isabelle}%
\ {\isadigit{1}}{\isachardot}\ {\isasymAnd}a\ list{\isachardot}\isanewline
\isaindent{\ {\isadigit{1}}{\isachardot}\ \ \ \ }rev\ {\isacharparenleft}list\ {\isacharat}\ ys{\isacharparenright}\ {\isacharequal}\ rev\ ys\ {\isacharat}\ rev\ list\ {\isasymLongrightarrow}\isanewline
\isaindent{\ {\isadigit{1}}{\isachardot}\ \ \ \ }{\isacharparenleft}rev\ ys\ {\isacharat}\ rev\ list{\isacharparenright}\ {\isacharat}\ a\ {\isacharhash}\ {\isacharbrackleft}{\isacharbrackright}\ {\isacharequal}\ rev\ ys\ {\isacharat}\ rev\ list\ {\isacharat}\ a\ {\isacharhash}\ {\isacharbrackleft}{\isacharbrackright}%
\end{isabelle}
Now we need to remember that \isa{{\isacharat}} associates to the right, and that
\isa{{\isacharhash}} and \isa{{\isacharat}} have the same priority (namely the \isa{{\isadigit{6}}{\isadigit{5}}}
in their \isacommand{infixr} annotation). Thus the conclusion really is
\begin{isabelle}
~~~~~(rev~ys~@~rev~list)~@~(a~\#~[])~=~rev~ys~@~(rev~list~@~(a~\#~[]))
\end{isabelle}
and the missing lemma is associativity of \isa{{\isacharat}}.%
\end{isamarkuptxt}%
\isamarkuptrue%
%
\endisatagproof
{\isafoldproof}%
%
\isadelimproof
%
\endisadelimproof
%
\isamarkupsubsubsection{Third Lemma%
}
\isamarkuptrue%
%
\begin{isamarkuptext}%
Abandoning the previous attempt, the canonical proof procedure
succeeds without further ado.%
\end{isamarkuptext}%
\isamarkuptrue%
\isacommand{lemma}\isamarkupfalse%
\ app{\isacharunderscore}assoc\ {\isacharbrackleft}simp{\isacharbrackright}{\isacharcolon}\ {\isachardoublequoteopen}{\isacharparenleft}xs\ {\isacharat}\ ys{\isacharparenright}\ {\isacharat}\ zs\ {\isacharequal}\ xs\ {\isacharat}\ {\isacharparenleft}ys\ {\isacharat}\ zs{\isacharparenright}{\isachardoublequoteclose}\isanewline
%
\isadelimproof
%
\endisadelimproof
%
\isatagproof
\isacommand{apply}\isamarkupfalse%
{\isacharparenleft}induct{\isacharunderscore}tac\ xs{\isacharparenright}\isanewline
\isacommand{apply}\isamarkupfalse%
{\isacharparenleft}auto{\isacharparenright}\isanewline
\isacommand{done}\isamarkupfalse%
%
\endisatagproof
{\isafoldproof}%
%
\isadelimproof
%
\endisadelimproof
%
\begin{isamarkuptext}%
\noindent
Now we can prove the first lemma:%
\end{isamarkuptext}%
\isamarkuptrue%
\isacommand{lemma}\isamarkupfalse%
\ rev{\isacharunderscore}app\ {\isacharbrackleft}simp{\isacharbrackright}{\isacharcolon}\ {\isachardoublequoteopen}rev{\isacharparenleft}xs\ {\isacharat}\ ys{\isacharparenright}\ {\isacharequal}\ {\isacharparenleft}rev\ ys{\isacharparenright}\ {\isacharat}\ {\isacharparenleft}rev\ xs{\isacharparenright}{\isachardoublequoteclose}\isanewline
%
\isadelimproof
%
\endisadelimproof
%
\isatagproof
\isacommand{apply}\isamarkupfalse%
{\isacharparenleft}induct{\isacharunderscore}tac\ xs{\isacharparenright}\isanewline
\isacommand{apply}\isamarkupfalse%
{\isacharparenleft}auto{\isacharparenright}\isanewline
\isacommand{done}\isamarkupfalse%
%
\endisatagproof
{\isafoldproof}%
%
\isadelimproof
%
\endisadelimproof
%
\begin{isamarkuptext}%
\noindent
Finally, we prove our main theorem:%
\end{isamarkuptext}%
\isamarkuptrue%
\isacommand{theorem}\isamarkupfalse%
\ rev{\isacharunderscore}rev\ {\isacharbrackleft}simp{\isacharbrackright}{\isacharcolon}\ {\isachardoublequoteopen}rev{\isacharparenleft}rev\ xs{\isacharparenright}\ {\isacharequal}\ xs{\isachardoublequoteclose}\isanewline
%
\isadelimproof
%
\endisadelimproof
%
\isatagproof
\isacommand{apply}\isamarkupfalse%
{\isacharparenleft}induct{\isacharunderscore}tac\ xs{\isacharparenright}\isanewline
\isacommand{apply}\isamarkupfalse%
{\isacharparenleft}auto{\isacharparenright}\isanewline
\isacommand{done}\isamarkupfalse%
%
\endisatagproof
{\isafoldproof}%
%
\isadelimproof
%
\endisadelimproof
%
\begin{isamarkuptext}%
\noindent
The final \commdx{end} tells Isabelle to close the current theory because
we are finished with its development:%
\index{*rev (constant)|)}\index{append function|)}%
\end{isamarkuptext}%
\isamarkuptrue%
%
\isadelimtheory
%
\endisadelimtheory
%
\isatagtheory
\isacommand{end}\isamarkupfalse%
%
\endisatagtheory
{\isafoldtheory}%
%
\isadelimtheory
%
\endisadelimtheory
\isanewline
\end{isabellebody}%
%%% Local Variables:
%%% mode: latex
%%% TeX-master: "root"
%%% End:
\end{ttbox}
\caption{A theory of lists}
\label{fig:ToyList}
\end{figure}

HOL already has a predefined theory of lists called \texttt{List} ---
\texttt{ToyList} is merely a small fragment of it chosen as an example. In
contrast to what is recommended in \S\ref{sec:Basic:Theories},
\texttt{ToyList} is not based on \texttt{Main} but on \texttt{Datatype}, a
theory that contains everything required for datatype definitions but does
not have \texttt{List} as a parent, thus avoiding ambiguities caused by
defining lists twice.

The \ttindexbold{datatype} \texttt{list} introduces two constructors
\texttt{Nil} and \texttt{Cons}, the empty list and the operator that adds an
element to the front of a list. For example, the term \texttt{Cons True (Cons
  False Nil)} is a value of type \texttt{bool~list}, namely the list with the
elements \texttt{True} and \texttt{False}. Because this notation becomes
unwieldy very quickly, the datatype declaration is annotated with an
alternative syntax: instead of \texttt{Nil} and \texttt{Cons}~$x$~$xs$ we can
write \index{#@{\tt[]}|bold}\texttt{[]} and
\texttt{$x$~\#~$xs$}\index{#@{\tt\#}|bold}. In fact, this alternative syntax
is the standard syntax. Thus the list \texttt{Cons True (Cons False Nil)}
becomes \texttt{True \# False \# []}. The annotation \ttindexbold{infixr}
means that \texttt{\#} associates to the right, i.e.\ the term \texttt{$x$ \#
  $y$ \# $z$} is read as \texttt{$x$ \# ($y$ \# $z$)} and not as \texttt{($x$
  \# $y$) \# $z$}.

\begin{warn}
  Syntax annotations are a powerful but completely optional feature. You
  could drop them from theory \texttt{ToyList} and go back to the identifiers
  \texttt{Nil} and \texttt{Cons}. However, lists are such a central datatype
  that their syntax is highly customized. We recommend that novices should
  not use syntax annotations in their own theories.
\end{warn}

Next, the functions \texttt{app} and \texttt{rev} are declared. In contrast
to ML, Isabelle insists on explicit declarations of all functions (keyword
\ttindexbold{consts}).  (Apart from the declaration-before-use restriction,
the order of items in a theory file is unconstrained.) Function \texttt{app}
is annotated with concrete syntax too. Instead of the prefix syntax
\texttt{app}~$xs$~$ys$ the infix $xs$~\texttt{\at}~$ys$ becomes the preferred
form.

Both functions are defined recursively. The equations for \texttt{app} and
\texttt{rev} hardly need comments: \texttt{app} appends two lists and
\texttt{rev} reverses a list.  The keyword \texttt{primrec} indicates that
the recursion is of a particularly primitive kind where each recursive call
peels off a datatype constructor from one of the arguments (see
\S\ref{sec:datatype}).  Thus the recursion always terminates, i.e.\ the
function is \bfindex{total}.

The termination requirement is absolutely essential in HOL, a logic of total
functions. If we were to drop it, inconsistencies could quickly arise: the
``definition'' $f(n) = f(n)+1$ immediately leads to $0 = 1$ by subtracting
$f(n)$ on both sides.
% However, this is a subtle issue that we cannot discuss here further.

\begin{warn}
  As we have indicated, the desire for total functions is not a gratuitously
  imposed restriction but an essential characteristic of HOL. It is only
  because of totality that reasoning in HOL is comparatively easy.  More
  generally, the philosophy in HOL is not to allow arbitrary axioms (such as
  function definitions whose totality has not been proved) because they
  quickly lead to inconsistencies. Instead, fixed constructs for introducing
  types and functions are offered (such as \texttt{datatype} and
  \texttt{primrec}) which are guaranteed to preserve consistency.
\end{warn}

A remark about syntax.  The textual definition of a theory follows a fixed
syntax with keywords like \texttt{datatype} and \texttt{end} (see
Fig.~\ref{fig:keywords} in Appendix~\ref{sec:Appendix} for a full list).
Embedded in this syntax are the types and formulae of HOL, whose syntax is
extensible, e.g.\ by new user-defined infix operators
(see~\ref{sec:infix-syntax}). To distinguish the two levels, everything
HOL-specific should be enclosed in \texttt{"}\dots\texttt{"}. The same holds
for identifiers that happen to be keywords, as in
\begin{ttbox}
consts "end" :: 'a list => 'a
\end{ttbox}
To lessen this burden, quotation marks around types can be dropped,
provided their syntax does not go beyond what is described in
\S\ref{sec:TypesTermsForms}. Types containing further operators, e.g.\
\texttt{*} for Cartesian products, need quotation marks.

When Isabelle prints a syntax error message, it refers to the HOL syntax as
the \bfindex{inner syntax}.

\section{An introductory proof}
\label{sec:intro-proof}

Having defined \texttt{ToyList}, we load it with the ML command
\begin{ttbox}
use_thy "ToyList";
\end{ttbox}
and are ready to prove a few simple theorems. This will illustrate not just
the basic proof commands but also the typical proof process.

\subsubsection*{Main goal: \texttt{rev(rev xs) = xs}}

Our goal is to show that reversing a list twice produces the original
list. Typing
\begin{ttbox}
%% $Id$
\chapter{Theorems and Forward Proof}
\index{theorems|(}

Theorems, which represent the axioms, theorems and rules of object-logics,
have type \mltydx{thm}.  This chapter begins by describing operations that
print theorems and that join them in forward proof.  Most theorem
operations are intended for advanced applications, such as programming new
proof procedures.  Many of these operations refer to signatures, certified
terms and certified types, which have the \ML{} types {\tt Sign.sg}, {\tt
  Sign.cterm} and {\tt Sign.ctyp} and are discussed in
Chapter~\ref{theories}.  Beginning users should ignore such complexities
--- and skip all but the first section of this chapter.

The theorem operations do not print error messages.  Instead, they raise
exception~\xdx{THM}\@.  Use \ttindex{print_exn} to display
exceptions nicely:
\begin{ttbox} 
allI RS mp  handle e => print_exn e;
{\out Exception THM raised:}
{\out RSN: no unifiers -- premise 1}
{\out (!!x. ?P(x)) ==> ALL x. ?P(x)}
{\out [| ?P --> ?Q; ?P |] ==> ?Q}
{\out}
{\out uncaught exception THM}
\end{ttbox}


\section{Basic operations on theorems}
\subsection{Pretty-printing a theorem}
\index{theorems!printing of}
\begin{ttbox} 
prth          : thm -> thm
prths         : thm list -> thm list
prthq         : thm Sequence.seq -> thm Sequence.seq
print_thm     : thm -> unit
print_goals   : int -> thm -> unit
string_of_thm : thm -> string
\end{ttbox}
The first three commands are for interactive use.  They are identity
functions that display, then return, their argument.  The \ML{} identifier
{\tt it} will refer to the value just displayed.

The others are for use in programs.  Functions with result type {\tt unit}
are convenient for imperative programming.

\begin{ttdescription}
\item[\ttindexbold{prth} {\it thm}]  
prints {\it thm\/} at the terminal.

\item[\ttindexbold{prths} {\it thms}]  
prints {\it thms}, a list of theorems.

\item[\ttindexbold{prthq} {\it thmq}]  
prints {\it thmq}, a sequence of theorems.  It is useful for inspecting
the output of a tactic.

\item[\ttindexbold{print_thm} {\it thm}]  
prints {\it thm\/} at the terminal.

\item[\ttindexbold{print_goals} {\it limit\/} {\it thm}]  
prints {\it thm\/} in goal style, with the premises as subgoals.  It prints
at most {\it limit\/} subgoals.  The subgoal module calls {\tt print_goals}
to display proof states.

\item[\ttindexbold{string_of_thm} {\it thm}]  
converts {\it thm\/} to a string.
\end{ttdescription}


\subsection{Forward proof: joining rules by resolution}
\index{theorems!joining by resolution}
\index{resolution}\index{forward proof}
\begin{ttbox} 
RSN : thm * (int * thm) -> thm                 \hfill{\bf infix}
RS  : thm * thm -> thm                         \hfill{\bf infix}
MRS : thm list * thm -> thm                    \hfill{\bf infix}
RLN : thm list * (int * thm list) -> thm list  \hfill{\bf infix}
RL  : thm list * thm list -> thm list          \hfill{\bf infix}
MRL : thm list list * thm list -> thm list     \hfill{\bf infix}
\end{ttbox}
Joining rules together is a simple way of deriving new rules.  These
functions are especially useful with destruction rules.  To store
the result in the theorem database, use \ttindex{bind_thm}
(\S\ref{ExtractingAndStoringTheProvedTheorem}). 
\begin{ttdescription}
\item[\tt$thm@1$ RSN $(i,thm@2)$] \indexbold{*RSN} 
  resolves the conclusion of $thm@1$ with the $i$th premise of~$thm@2$.
  Unless there is precisely one resolvent it raises exception
  \xdx{THM}; in that case, use {\tt RLN}.

\item[\tt$thm@1$ RS $thm@2$] \indexbold{*RS} 
abbreviates \hbox{\tt$thm@1$ RSN $(1,thm@2)$}.  Thus, it resolves the
conclusion of $thm@1$ with the first premise of~$thm@2$.

\item[\tt {$[thm@1,\ldots,thm@n]$} MRS $thm$] \indexbold{*MRS} 
  uses {\tt RSN} to resolve $thm@i$ against premise~$i$ of $thm$, for
  $i=n$, \ldots,~1.  This applies $thm@n$, \ldots, $thm@1$ to the first $n$
  premises of $thm$.  Because the theorems are used from right to left, it
  does not matter if the $thm@i$ create new premises.  {\tt MRS} is useful
  for expressing proof trees.

\item[\tt$thms@1$ RLN $(i,thms@2)$] \indexbold{*RLN} 
  joins lists of theorems.  For every $thm@1$ in $thms@1$ and $thm@2$ in
  $thms@2$, it resolves the conclusion of $thm@1$ with the $i$th premise
  of~$thm@2$, accumulating the results. 

\item[\tt$thms@1$ RL $thms@2$] \indexbold{*RL} 
abbreviates \hbox{\tt$thms@1$ RLN $(1,thms@2)$}. 

\item[\tt {$[thms@1,\ldots,thms@n]$} MRL $thms$] \indexbold{*MRL} 
is analogous to {\tt MRS}, but combines theorem lists rather than theorems.
It too is useful for expressing proof trees.
\end{ttdescription}


\subsection{Expanding definitions in theorems}
\index{meta-rewriting!in theorems}
\begin{ttbox} 
rewrite_rule       : thm list -> thm -> thm
rewrite_goals_rule : thm list -> thm -> thm
\end{ttbox}
\begin{ttdescription}
\item[\ttindexbold{rewrite_rule} {\it defs} {\it thm}]  
unfolds the {\it defs} throughout the theorem~{\it thm}.

\item[\ttindexbold{rewrite_goals_rule} {\it defs} {\it thm}]  
unfolds the {\it defs} in the premises of~{\it thm}, but leaves the
conclusion unchanged.  This rule underlies \ttindex{rewrite_goals_tac}, but 
serves little purpose in forward proof.
\end{ttdescription}


\subsection{Instantiating a theorem}
\index{instantiation}
\begin{ttbox}
read_instantiate    :            (string*string)list -> thm -> thm
read_instantiate_sg : Sign.sg -> (string*string)list -> thm -> thm
cterm_instantiate   :    (Sign.cterm*Sign.cterm)list -> thm -> thm
\end{ttbox}
These meta-rules instantiate type and term unknowns in a theorem.  They are
occasionally useful.  They can prevent difficulties with higher-order
unification, and define specialized versions of rules.
\begin{ttdescription}
\item[\ttindexbold{read_instantiate} {\it insts} {\it thm}] 
processes the instantiations {\it insts} and instantiates the rule~{\it
thm}.  The processing of instantiations is described
in \S\ref{res_inst_tac}, under {\tt res_inst_tac}.  

Use {\tt res_inst_tac}, not {\tt read_instantiate}, to instantiate a rule
and refine a particular subgoal.  The tactic allows instantiation by the
subgoal's parameters, and reads the instantiations using the signature
associated with the proof state.

Use {\tt read_instantiate_sg} below if {\it insts\/} appears to be treated
incorrectly.

\item[\ttindexbold{read_instantiate_sg} {\it sg} {\it insts} {\it thm}]
  resembles \hbox{\tt read_instantiate {\it insts} {\it thm}}, but reads
  the instantiations under signature~{\it sg}.  This is necessary to
  instantiate a rule from a general theory, such as first-order logic,
  using the notation of some specialized theory.  Use the function {\tt
    sign_of} to get a theory's signature.

\item[\ttindexbold{cterm_instantiate} {\it ctpairs} {\it thm}] 
is similar to {\tt read_instantiate}, but the instantiations are provided
as pairs of certified terms, not as strings to be read.
\end{ttdescription}


\subsection{Miscellaneous forward rules}\label{MiscellaneousForwardRules}
\index{theorems!standardizing}
\begin{ttbox} 
standard         :           thm -> thm
zero_var_indexes :           thm -> thm
make_elim        :           thm -> thm
rule_by_tactic   : tactic -> thm -> thm
\end{ttbox}
\begin{ttdescription}
\item[\ttindexbold{standard} $thm$]  
puts $thm$ into the standard form of object-rules.  It discharges all
meta-assumptions, replaces free variables by schematic variables, and
renames schematic variables to have subscript zero.

\item[\ttindexbold{zero_var_indexes} $thm$] 
makes all schematic variables have subscript zero, renaming them to avoid
clashes. 

\item[\ttindexbold{make_elim} $thm$] 
\index{rules!converting destruction to elimination}
converts $thm$, a destruction rule of the form $\List{P@1;\ldots;P@m}\Imp
Q$, to the elimination rule $\List{P@1; \ldots; P@m; Q\Imp R}\Imp R$.  This
is the basis for destruct-resolution: {\tt dresolve_tac}, etc.

\item[\ttindexbold{rule_by_tactic} {\it tac} {\it thm}] 
  applies {\it tac\/} to the {\it thm}, freezing its variables first, then
  yields the proof state returned by the tactic.  In typical usage, the
  {\it thm\/} represents an instance of a rule with several premises, some
  with contradictory assumptions (because of the instantiation).  The
  tactic proves those subgoals and does whatever else it can, and returns
  whatever is left.
\end{ttdescription}


\subsection{Taking a theorem apart}
\index{theorems!taking apart}
\index{flex-flex constraints}
\begin{ttbox} 
concl_of      : thm -> term
prems_of      : thm -> term list
nprems_of     : thm -> int
tpairs_of     : thm -> (term*term)list
stamps_of_thy : thm -> string ref list
theory_of_thm : thm -> theory
dest_state    : thm*int -> (term*term)list*term list*term*term
rep_thm       : thm -> \{prop:term, hyps:term list, 
                        maxidx:int, sign:Sign.sg\}
\end{ttbox}
\begin{ttdescription}
\item[\ttindexbold{concl_of} $thm$] 
returns the conclusion of $thm$ as a term.

\item[\ttindexbold{prems_of} $thm$] 
returns the premises of $thm$ as a list of terms.

\item[\ttindexbold{nprems_of} $thm$] 
returns the number of premises in $thm$, and is equivalent to {\tt
  length(prems_of~$thm$)}.

\item[\ttindexbold{tpairs_of} $thm$] 
returns the flex-flex constraints of $thm$.

\item[\ttindexbold{stamps_of_thm} $thm$] 
returns the \rmindex{stamps} of the signature associated with~$thm$.

\item[\ttindexbold{theory_of_thm} $thm$]
returns the theory associated with $thm$.

\item[\ttindexbold{dest_state} $(thm,i)$] 
decomposes $thm$ as a tuple containing a list of flex-flex constraints, a
list of the subgoals~1 to~$i-1$, subgoal~$i$, and the rest of the theorem
(this will be an implication if there are more than $i$ subgoals).

\item[\ttindexbold{rep_thm} $thm$] 
decomposes $thm$ as a record containing the statement of~$thm$, its list of
meta-assumptions, the maximum subscript of its unknowns, and its signature.
\end{ttdescription}


\subsection{Tracing flags for unification}
\index{tracing!of unification}
\begin{ttbox} 
Unify.trace_simp   : bool ref \hfill{\bf initially false}
Unify.trace_types  : bool ref \hfill{\bf initially false}
Unify.trace_bound  : int ref \hfill{\bf initially 10}
Unify.search_bound : int ref \hfill{\bf initially 20}
\end{ttbox}
Tracing the search may be useful when higher-order unification behaves
unexpectedly.  Letting {\tt res_inst_tac} circumvent the problem is easier,
though.
\begin{ttdescription}
\item[Unify.trace_simp := true;] 
causes tracing of the simplification phase.

\item[Unify.trace_types := true;] 
generates warnings of incompleteness, when unification is not considering
all possible instantiations of type unknowns.

\item[Unify.trace_bound := $n$;] 
causes unification to print tracing information once it reaches depth~$n$.
Use $n=0$ for full tracing.  At the default value of~10, tracing
information is almost never printed.

\item[Unify.search_bound := $n$;] 
causes unification to limit its search to depth~$n$.  Because of this
bound, higher-order unification cannot return an infinite sequence, though
it can return a very long one.  The search rarely approaches the default
value of~20.  If the search is cut off, unification prints {\tt
***Unification bound exceeded}.
\end{ttdescription}


\section{Primitive meta-level inference rules}
\index{meta-rules|(}
These implement the meta-logic in {\sc lcf} style, as functions from theorems
to theorems.  They are, rarely, useful for deriving results in the pure
theory.  Mainly, they are included for completeness, and most users should
not bother with them.  The meta-rules raise exception \xdx{THM} to signal
malformed premises, incompatible signatures and similar errors.

\index{meta-assumptions}
The meta-logic uses natural deduction.  Each theorem may depend on
meta-level assumptions.  Certain rules, such as $({\Imp}I)$,
discharge assumptions; in most other rules, the conclusion depends on all
of the assumptions of the premises.  Formally, the system works with
assertions of the form
\[ \phi \quad [\phi@1,\ldots,\phi@n], \]
where $\phi@1$,~\ldots,~$\phi@n$ are the assumptions.  Do not confuse
meta-level assumptions with the object-level assumptions in a subgoal,
which are represented in the meta-logic using~$\Imp$.

Each theorem has a signature.  Certified terms have a signature.  When a
rule takes several premises and certified terms, it merges the signatures
to make a signature for the conclusion.  This fails if the signatures are
incompatible. 

\index{meta-implication}
The {\bf implication} rules are $({\Imp}I)$
and $({\Imp}E)$:
\[ \infer[({\Imp}I)]{\phi\Imp \psi}{\infer*{\psi}{[\phi]}}  \qquad
   \infer[({\Imp}E)]{\psi}{\phi\Imp \psi & \phi}  \]

\index{meta-equality}
Equality of truth values means logical equivalence:
\[ \infer[({\equiv}I)]{\phi\equiv\psi}{\infer*{\psi}{[\phi]} &
                                       \infer*{\phi}{[\psi]}}  
   \qquad
   \infer[({\equiv}E)]{\psi}{\phi\equiv \psi & \phi}   \]

The {\bf equality} rules are reflexivity, symmetry, and transitivity:
\[ {a\equiv a}\,(refl)  \qquad
   \infer[(sym)]{b\equiv a}{a\equiv b}  \qquad
   \infer[(trans)]{a\equiv c}{a\equiv b & b\equiv c}   \]

\index{lambda calc@$\lambda$-calculus}
The $\lambda$-conversions are $\alpha$-conversion, $\beta$-conversion, and
extensionality:\footnote{$\alpha$-conversion holds if $y$ is not free
in~$a$; $(ext)$ holds if $x$ is not free in the assumptions, $f$, or~$g$.}
\[ {(\lambda x.a) \equiv (\lambda y.a[y/x])}    \qquad
   {((\lambda x.a)(b)) \equiv a[b/x]}           \qquad
   \infer[(ext)]{f\equiv g}{f(x) \equiv g(x)}   \]

The {\bf abstraction} and {\bf combination} rules let conversions be
applied to subterms:\footnote{Abstraction holds if $x$ is not free in the
assumptions.}
\[  \infer[(abs)]{(\lambda x.a) \equiv (\lambda x.b)}{a\equiv b}   \qquad
    \infer[(comb)]{f(a)\equiv g(b)}{f\equiv g & a\equiv b}   \]

\index{meta-quantifiers}
The {\bf universal quantification} rules are $(\Forall I)$ and $(\Forall
E)$:\footnote{$(\Forall I)$ holds if $x$ is not free in the assumptions.}
\[ \infer[(\Forall I)]{\Forall x.\phi}{\phi}        \qquad
   \infer[(\Forall E)]{\phi[b/x]}{\Forall x.\phi}   \]


\subsection{Assumption rule}
\index{meta-assumptions}
\begin{ttbox} 
assume: Sign.cterm -> thm
\end{ttbox}
\begin{ttdescription}
\item[\ttindexbold{assume} $ct$] 
makes the theorem \(\phi \;[\phi]\), where $\phi$ is the value of~$ct$.
The rule checks that $ct$ has type $prop$ and contains no unknowns, which
are not allowed in assumptions.
\end{ttdescription}

\subsection{Implication rules}
\index{meta-implication}
\begin{ttbox} 
implies_intr      : Sign.cterm -> thm -> thm
implies_intr_list : Sign.cterm list -> thm -> thm
implies_intr_hyps : thm -> thm
implies_elim      : thm -> thm -> thm
implies_elim_list : thm -> thm list -> thm
\end{ttbox}
\begin{ttdescription}
\item[\ttindexbold{implies_intr} $ct$ $thm$] 
is $({\Imp}I)$, where $ct$ is the assumption to discharge, say~$\phi$.  It
maps the premise~$\psi$ to the conclusion $\phi\Imp\psi$, removing all
occurrences of~$\phi$ from the assumptions.  The rule checks that $ct$ has
type $prop$. 

\item[\ttindexbold{implies_intr_list} $cts$ $thm$] 
applies $({\Imp}I)$ repeatedly, on every element of the list~$cts$.

\item[\ttindexbold{implies_intr_hyps} $thm$] 
applies $({\Imp}I)$ to discharge all the hypotheses (assumptions) of~$thm$.
It maps the premise $\phi \; [\phi@1,\ldots,\phi@n]$ to the conclusion
$\List{\phi@1,\ldots,\phi@n}\Imp\phi$.

\item[\ttindexbold{implies_elim} $thm@1$ $thm@2$] 
applies $({\Imp}E)$ to $thm@1$ and~$thm@2$.  It maps the premises $\phi\Imp
\psi$ and $\phi$ to the conclusion~$\psi$.

\item[\ttindexbold{implies_elim_list} $thm$ $thms$] 
applies $({\Imp}E)$ repeatedly to $thm$, using each element of~$thms$ in
turn.  It maps the premises $\List{\phi@1,\ldots,\phi@n}\Imp\psi$ and
$\phi@1$,\ldots,$\phi@n$ to the conclusion~$\psi$.
\end{ttdescription}

\subsection{Logical equivalence rules}
\index{meta-equality}
\begin{ttbox} 
equal_intr : thm -> thm -> thm 
equal_elim : thm -> thm -> thm
\end{ttbox}
\begin{ttdescription}
\item[\ttindexbold{equal_intr} $thm@1$ $thm@2$] 
applies $({\equiv}I)$ to $thm@1$ and~$thm@2$.  It maps the premises~$\psi$
and~$\phi$ to the conclusion~$\phi\equiv\psi$; the assumptions are those of
the first premise with~$\phi$ removed, plus those of
the second premise with~$\psi$ removed.

\item[\ttindexbold{equal_elim} $thm@1$ $thm@2$] 
applies $({\equiv}E)$ to $thm@1$ and~$thm@2$.  It maps the premises
$\phi\equiv\psi$ and $\phi$ to the conclusion~$\psi$.
\end{ttdescription}


\subsection{Equality rules}
\index{meta-equality}
\begin{ttbox} 
reflexive  : Sign.cterm -> thm
symmetric  : thm -> thm
transitive : thm -> thm -> thm
\end{ttbox}
\begin{ttdescription}
\item[\ttindexbold{reflexive} $ct$] 
makes the theorem \(ct\equiv ct\). 

\item[\ttindexbold{symmetric} $thm$] 
maps the premise $a\equiv b$ to the conclusion $b\equiv a$.

\item[\ttindexbold{transitive} $thm@1$ $thm@2$] 
maps the premises $a\equiv b$ and $b\equiv c$ to the conclusion~${a\equiv c}$.
\end{ttdescription}


\subsection{The $\lambda$-conversion rules}
\index{lambda calc@$\lambda$-calculus}
\begin{ttbox} 
beta_conversion : Sign.cterm -> thm
extensional     : thm -> thm
abstract_rule   : string -> Sign.cterm -> thm -> thm
combination     : thm -> thm -> thm
\end{ttbox} 
There is no rule for $\alpha$-conversion because Isabelle regards
$\alpha$-convertible theorems as equal.
\begin{ttdescription}
\item[\ttindexbold{beta_conversion} $ct$] 
makes the theorem $((\lambda x.a)(b)) \equiv a[b/x]$, where $ct$ is the
term $(\lambda x.a)(b)$.

\item[\ttindexbold{extensional} $thm$] 
maps the premise $f(x) \equiv g(x)$ to the conclusion $f\equiv g$.
Parameter~$x$ is taken from the premise.  It may be an unknown or a free
variable (provided it does not occur in the assumptions); it must not occur
in $f$ or~$g$.

\item[\ttindexbold{abstract_rule} $v$ $x$ $thm$] 
maps the premise $a\equiv b$ to the conclusion $(\lambda x.a) \equiv
(\lambda x.b)$, abstracting over all occurrences (if any!) of~$x$.
Parameter~$x$ is supplied as a cterm.  It may be an unknown or a free
variable (provided it does not occur in the assumptions).  In the
conclusion, the bound variable is named~$v$.

\item[\ttindexbold{combination} $thm@1$ $thm@2$] 
maps the premises $f\equiv g$ and $a\equiv b$ to the conclusion~$f(a)\equiv
g(b)$.
\end{ttdescription}


\subsection{Forall introduction rules}
\index{meta-quantifiers}
\begin{ttbox} 
forall_intr       : Sign.cterm      -> thm -> thm
forall_intr_list  : Sign.cterm list -> thm -> thm
forall_intr_frees :                    thm -> thm
\end{ttbox}

\begin{ttdescription}
\item[\ttindexbold{forall_intr} $x$ $thm$] 
applies $({\Forall}I)$, abstracting over all occurrences (if any!) of~$x$.
The rule maps the premise $\phi$ to the conclusion $\Forall x.\phi$.
Parameter~$x$ is supplied as a cterm.  It may be an unknown or a free
variable (provided it does not occur in the assumptions).

\item[\ttindexbold{forall_intr_list} $xs$ $thm$] 
applies $({\Forall}I)$ repeatedly, on every element of the list~$xs$.

\item[\ttindexbold{forall_intr_frees} $thm$] 
applies $({\Forall}I)$ repeatedly, generalizing over all the free variables
of the premise.
\end{ttdescription}


\subsection{Forall elimination rules}
\begin{ttbox} 
forall_elim       : Sign.cterm      -> thm -> thm
forall_elim_list  : Sign.cterm list -> thm -> thm
forall_elim_var   :             int -> thm -> thm
forall_elim_vars  :             int -> thm -> thm
\end{ttbox}

\begin{ttdescription}
\item[\ttindexbold{forall_elim} $ct$ $thm$] 
applies $({\Forall}E)$, mapping the premise $\Forall x.\phi$ to the conclusion
$\phi[ct/x]$.  The rule checks that $ct$ and $x$ have the same type.

\item[\ttindexbold{forall_elim_list} $cts$ $thm$] 
applies $({\Forall}E)$ repeatedly, on every element of the list~$cts$.

\item[\ttindexbold{forall_elim_var} $k$ $thm$] 
applies $({\Forall}E)$, mapping the premise $\Forall x.\phi$ to the conclusion
$\phi[\Var{x@k}/x]$.  Thus, it replaces the outermost $\Forall$-bound
variable by an unknown having subscript~$k$.

\item[\ttindexbold{forall_elim_vars} $ks$ $thm$] 
applies {\tt forall_elim_var} repeatedly, for every element of the list~$ks$.
\end{ttdescription}

\subsection{Instantiation of unknowns}
\index{instantiation}
\begin{ttbox} 
instantiate: (indexname*Sign.ctyp)list * 
             (Sign.cterm*Sign.cterm)list  -> thm -> thm
\end{ttbox}
\begin{ttdescription}
\item[\ttindexbold{instantiate} ($tyinsts$, $insts$) $thm$] 
simultaneously substitutes types for type unknowns (the
$tyinsts$) and terms for term unknowns (the $insts$).  Instantiations are
given as $(v,t)$ pairs, where $v$ is an unknown and $t$ is a term (of the
same type as $v$) or a type (of the same sort as~$v$).  All the unknowns
must be distinct.  The rule normalizes its conclusion.
\end{ttdescription}


\subsection{Freezing/thawing type unknowns}
\index{type unknowns!freezing/thawing of}
\begin{ttbox} 
freezeT: thm -> thm
varifyT: thm -> thm
\end{ttbox}
\begin{ttdescription}
\item[\ttindexbold{freezeT} $thm$] 
converts all the type unknowns in $thm$ to free type variables.

\item[\ttindexbold{varifyT} $thm$] 
converts all the free type variables in $thm$ to type unknowns.
\end{ttdescription}


\section{Derived rules for goal-directed proof}
Most of these rules have the sole purpose of implementing particular
tactics.  There are few occasions for applying them directly to a theorem.

\subsection{Proof by assumption}
\index{meta-assumptions}
\begin{ttbox} 
assumption    : int -> thm -> thm Sequence.seq
eq_assumption : int -> thm -> thm
\end{ttbox}
\begin{ttdescription}
\item[\ttindexbold{assumption} {\it i} $thm$] 
attempts to solve premise~$i$ of~$thm$ by assumption.

\item[\ttindexbold{eq_assumption}] 
is like {\tt assumption} but does not use unification.
\end{ttdescription}


\subsection{Resolution}
\index{resolution}
\begin{ttbox} 
biresolution : bool -> (bool*thm)list -> int -> thm
               -> thm Sequence.seq
\end{ttbox}
\begin{ttdescription}
\item[\ttindexbold{biresolution} $match$ $rules$ $i$ $state$] 
performs bi-resolution on subgoal~$i$ of $state$, using the list of $\it
(flag,rule)$ pairs.  For each pair, it applies resolution if the flag
is~{\tt false} and elim-resolution if the flag is~{\tt true}.  If $match$
is~{\tt true}, the $state$ is not instantiated.
\end{ttdescription}


\subsection{Composition: resolution without lifting}
\index{resolution!without lifting}
\begin{ttbox}
compose   : thm * int * thm -> thm list
COMP      : thm * thm -> thm
bicompose : bool -> bool * thm * int -> int -> thm
            -> thm Sequence.seq
\end{ttbox}
In forward proof, a typical use of composition is to regard an assertion of
the form $\phi\Imp\psi$ as atomic.  Schematic variables are not renamed, so
beware of clashes!
\begin{ttdescription}
\item[\ttindexbold{compose} ($thm@1$, $i$, $thm@2$)] 
uses $thm@1$, regarded as an atomic formula, to solve premise~$i$
of~$thm@2$.  Let $thm@1$ and $thm@2$ be $\psi$ and $\List{\phi@1; \ldots;
\phi@n} \Imp \phi$.  For each $s$ that unifies~$\psi$ and $\phi@i$, the
result list contains the theorem
\[ (\List{\phi@1; \ldots; \phi@{i-1}; \phi@{i+1}; \ldots; \phi@n} \Imp \phi)s.
\]

\item[\tt $thm@1$ COMP $thm@2$] 
calls \hbox{\tt compose ($thm@1$, 1, $thm@2$)} and returns the result, if
unique; otherwise, it raises exception~\xdx{THM}\@.  It is
analogous to {\tt RS}\@.  

For example, suppose that $thm@1$ is $a=b\Imp b=a$, a symmetry rule, and
that $thm@2$ is $\List{P\Imp Q; \neg Q} \Imp\neg P$, which is the
principle of contrapositives.  Then the result would be the
derived rule $\neg(b=a)\Imp\neg(a=b)$.

\item[\ttindexbold{bicompose} $match$ ($flag$, $rule$, $m$) $i$ $state$]
refines subgoal~$i$ of $state$ using $rule$, without lifting.  The $rule$
is taken to have the form $\List{\psi@1; \ldots; \psi@m} \Imp \psi$, where
$\psi$ need not be atomic; thus $m$ determines the number of new
subgoals.  If $flag$ is {\tt true} then it performs elim-resolution --- it
solves the first premise of~$rule$ by assumption and deletes that
assumption.  If $match$ is~{\tt true}, the $state$ is not instantiated.
\end{ttdescription}


\subsection{Other meta-rules}
\begin{ttbox} 
trivial            : Sign.cterm -> thm
lift_rule          : (thm * int) -> thm -> thm
rename_params_rule : string list * int -> thm -> thm
rewrite_cterm      : thm list -> Sign.cterm -> thm
flexflex_rule      : thm -> thm Sequence.seq
\end{ttbox}
\begin{ttdescription}
\item[\ttindexbold{trivial} $ct$] 
makes the theorem \(\phi\Imp\phi\), where $\phi$ is the value of~$ct$.
This is the initial state for a goal-directed proof of~$\phi$.  The rule
checks that $ct$ has type~$prop$.

\item[\ttindexbold{lift_rule} ($state$, $i$) $rule$] \index{lifting}
prepares $rule$ for resolution by lifting it over the parameters and
assumptions of subgoal~$i$ of~$state$.

\item[\ttindexbold{rename_params_rule} ({\it names}, {\it i}) $thm$] 
uses the $names$ to rename the parameters of premise~$i$ of $thm$.  The
names must be distinct.  If there are fewer names than parameters, then the
rule renames the innermost parameters and may modify the remaining ones to
ensure that all the parameters are distinct.
\index{parameters!renaming}

\item[\ttindexbold{rewrite_cterm} $defs$ $ct$]
transforms $ct$ to $ct'$ by repeatedly applying $defs$ as rewrite rules; it
returns the conclusion~$ct\equiv ct'$.  This underlies the meta-rewriting
tactics and rules.
\index{meta-rewriting!in terms}

\item[\ttindexbold{flexflex_rule} $thm$]  \index{flex-flex constraints}
removes all flex-flex pairs from $thm$ using the trivial unifier.
\end{ttdescription}
\index{theorems|)}
\index{meta-rules|)}
\end{ttbox}
establishes a new goal to be proved in the context of the current theory,
which is the one we just loaded. Isabelle's response is to print the current proof state:
\begin{ttbox}
{\out Level 0}
{\out rev (rev xs) = xs}
{\out  1. rev (rev xs) = xs}
\end{ttbox}
Until we have finished a proof, the proof state always looks like this:
\begin{ttbox}
{\out Level \(i\)}
{\out \(G\)}
{\out  1. \(G@1\)}
{\out  \(\vdots\)}
{\out  \(n\). \(G@n\)}
\end{ttbox}
where \texttt{Level}~$i$ indicates that we are $i$ steps into the proof, $G$
is the overall goal that we are trying to prove, and the numbered lines
contain the subgoals $G@1$, \dots, $G@n$ that we need to prove to establish
$G$. At \texttt{Level 0} there is only one subgoal, which is identical with
the overall goal.  Normally $G$ is constant and only serves as a reminder.
Hence we rarely show it in this tutorial.

Let us now get back to \texttt{rev(rev xs) = xs}. Properties of recursively
defined functions are best established by induction. In this case there is
not much choice except to induct on \texttt{xs}:
\begin{ttbox}
\input{ToyList/inductxs}\end{ttbox}
This tells Isabelle to perform induction on variable \texttt{xs} in subgoal
1. The new proof state contains two subgoals, namely the base case
(\texttt{Nil}) and the induction step (\texttt{Cons}):
\begin{ttbox}
{\out 1. rev (rev []) = []}
{\out 2. !!a list. rev (rev list) = list ==> rev (rev (a # list)) = a # list}
\end{ttbox}
The induction step is an example of the general format of a subgoal:
\begin{ttbox}
{\out  \(i\). !!\(x@1 \dots x@n\). {\it assumptions} ==> {\it conclusion}}
\end{ttbox}\index{==>@{\tt==>}|bold}
The prefix of bound variables \texttt{!!\(x@1 \dots x@n\)} can be ignored
most of the time, or simply treated as a list of variables local to this
subgoal. Their deeper significance is explained in \S\ref{sec:PCproofs}.  The
{\it assumptions} are the local assumptions for this subgoal and {\it
  conclusion} is the actual proposition to be proved. Typical proof steps
that add new assumptions are induction or case distinction. In our example
the only assumption is the induction hypothesis \texttt{rev (rev list) =
  list}, where \texttt{list} is a variable name chosen by Isabelle. If there
are multiple assumptions, they are enclosed in the bracket pair
\texttt{[|}\index{==>@\ttlbr|bold} and \texttt{|]}\index{==>@\ttrbr|bold}
and separated by semicolons.

Let us try to solve both goals automatically:
\begin{ttbox}
\input{ToyList/autotac}\end{ttbox}
This command tells Isabelle to apply a proof strategy called
\texttt{Auto_tac} to all subgoals. Essentially, \texttt{Auto_tac} tries to
`simplify' the subgoals.  In our case, subgoal~1 is solved completely (thanks
to the equation \texttt{rev [] = []}) and disappears; the simplified version
of subgoal~2 becomes the new subgoal~1:
\begin{ttbox}\makeatother
{\out 1. !!a list. rev(rev list) = list ==> rev(rev list @ a # []) = a # list}
\end{ttbox}
In order to simplify this subgoal further, a lemma suggests itself.

\subsubsection*{First lemma: \texttt{rev(xs \at~ys) = (rev ys) \at~(rev xs)}}

We start the proof as usual:
\begin{ttbox}\makeatother
\input{ToyList/lemma1}\end{ttbox}
There are two variables that we could induct on: \texttt{xs} and
\texttt{ys}. Because \texttt{\at} is defined by recursion on
the first argument, \texttt{xs} is the correct one:
\begin{ttbox}
\input{ToyList/inductxs}\end{ttbox}
This time not even the base case is solved automatically:
\begin{ttbox}\makeatother
by(Auto_tac);
{\out 1. rev ys = rev ys @ []}
{\out 2. \dots}
\end{ttbox}
We need another lemma.

\subsubsection*{Second lemma: \texttt{xs \at~[] = xs}}

This time the canonical proof procedure
\begin{ttbox}\makeatother
\input{ToyList/lemma2}\input{ToyList/inductxs}\input{ToyList/autotac}\end{ttbox}
leads to the desired message \texttt{No subgoals!}:
\begin{ttbox}\makeatother
{\out Level 2}
{\out xs @ [] = xs}
{\out No subgoals!}
\end{ttbox}
Now we can give the lemma just proved a suitable name
\begin{ttbox}
\input{ToyList/qed2}\end{ttbox}
and tell Isabelle to use this lemma in all future proofs by simplification:
\begin{ttbox}
\input{ToyList/addsimps2}\end{ttbox}
Note that in the theorem \texttt{app_Nil2} the free variable \texttt{xs} has
been replaced by the unknown \texttt{?xs}, just as explained in
\S\ref{sec:variables}.

Going back to the proof of the first lemma
\begin{ttbox}\makeatother
\input{ToyList/lemma1}\input{ToyList/inductxs}\input{ToyList/autotac}\end{ttbox}
we find that this time \texttt{Auto_tac} solves the base case, but the
induction step merely simplifies to
\begin{ttbox}\makeatother
{\out 1. !!a list.}
{\out       rev (list @ ys) = rev ys @ rev list}
{\out       ==> (rev ys @ rev list) @ a # [] = rev ys @ rev list @ a # []}
\end{ttbox}
Now we need to remember that \texttt{\at} associates to the right, and that
\texttt{\#} and \texttt{\at} have the same priority (namely the \texttt{65}
in the definition of \texttt{ToyList}). Thus the conclusion really is
\begin{ttbox}\makeatother
{\out     ==> (rev ys @ rev list) @ (a # []) = rev ys @ (rev list @ (a # []))}
\end{ttbox}
and the missing lemma is associativity of \texttt{\at}.

\subsubsection*{Third lemma: \texttt{(xs \at~ys) \at~zs = xs \at~(ys \at~zs)}}

This time the canonical proof procedure
\begin{ttbox}\makeatother
\input{ToyList/lemma3}\end{ttbox}
succeeds without further ado. Again we name the lemma and add it to
the set of lemmas used during simplification:
\begin{ttbox}
\input{ToyList/qed3}\end{ttbox}
Now we can go back and prove the first lemma
\begin{ttbox}\makeatother
\input{ToyList/lemma1}\input{ToyList/inductxs}\input{ToyList/autotac}\end{ttbox}
add it to the simplification lemmas
\begin{ttbox}
\input{ToyList/qed1}\end{ttbox}
and then solve our main theorem:
\begin{ttbox}\makeatother
%% $Id$
\chapter{Theorems and Forward Proof}
\index{theorems|(}

Theorems, which represent the axioms, theorems and rules of object-logics,
have type \mltydx{thm}.  This chapter begins by describing operations that
print theorems and that join them in forward proof.  Most theorem
operations are intended for advanced applications, such as programming new
proof procedures.  Many of these operations refer to signatures, certified
terms and certified types, which have the \ML{} types {\tt Sign.sg}, {\tt
  Sign.cterm} and {\tt Sign.ctyp} and are discussed in
Chapter~\ref{theories}.  Beginning users should ignore such complexities
--- and skip all but the first section of this chapter.

The theorem operations do not print error messages.  Instead, they raise
exception~\xdx{THM}\@.  Use \ttindex{print_exn} to display
exceptions nicely:
\begin{ttbox} 
allI RS mp  handle e => print_exn e;
{\out Exception THM raised:}
{\out RSN: no unifiers -- premise 1}
{\out (!!x. ?P(x)) ==> ALL x. ?P(x)}
{\out [| ?P --> ?Q; ?P |] ==> ?Q}
{\out}
{\out uncaught exception THM}
\end{ttbox}


\section{Basic operations on theorems}
\subsection{Pretty-printing a theorem}
\index{theorems!printing of}
\begin{ttbox} 
prth          : thm -> thm
prths         : thm list -> thm list
prthq         : thm Sequence.seq -> thm Sequence.seq
print_thm     : thm -> unit
print_goals   : int -> thm -> unit
string_of_thm : thm -> string
\end{ttbox}
The first three commands are for interactive use.  They are identity
functions that display, then return, their argument.  The \ML{} identifier
{\tt it} will refer to the value just displayed.

The others are for use in programs.  Functions with result type {\tt unit}
are convenient for imperative programming.

\begin{ttdescription}
\item[\ttindexbold{prth} {\it thm}]  
prints {\it thm\/} at the terminal.

\item[\ttindexbold{prths} {\it thms}]  
prints {\it thms}, a list of theorems.

\item[\ttindexbold{prthq} {\it thmq}]  
prints {\it thmq}, a sequence of theorems.  It is useful for inspecting
the output of a tactic.

\item[\ttindexbold{print_thm} {\it thm}]  
prints {\it thm\/} at the terminal.

\item[\ttindexbold{print_goals} {\it limit\/} {\it thm}]  
prints {\it thm\/} in goal style, with the premises as subgoals.  It prints
at most {\it limit\/} subgoals.  The subgoal module calls {\tt print_goals}
to display proof states.

\item[\ttindexbold{string_of_thm} {\it thm}]  
converts {\it thm\/} to a string.
\end{ttdescription}


\subsection{Forward proof: joining rules by resolution}
\index{theorems!joining by resolution}
\index{resolution}\index{forward proof}
\begin{ttbox} 
RSN : thm * (int * thm) -> thm                 \hfill{\bf infix}
RS  : thm * thm -> thm                         \hfill{\bf infix}
MRS : thm list * thm -> thm                    \hfill{\bf infix}
RLN : thm list * (int * thm list) -> thm list  \hfill{\bf infix}
RL  : thm list * thm list -> thm list          \hfill{\bf infix}
MRL : thm list list * thm list -> thm list     \hfill{\bf infix}
\end{ttbox}
Joining rules together is a simple way of deriving new rules.  These
functions are especially useful with destruction rules.  To store
the result in the theorem database, use \ttindex{bind_thm}
(\S\ref{ExtractingAndStoringTheProvedTheorem}). 
\begin{ttdescription}
\item[\tt$thm@1$ RSN $(i,thm@2)$] \indexbold{*RSN} 
  resolves the conclusion of $thm@1$ with the $i$th premise of~$thm@2$.
  Unless there is precisely one resolvent it raises exception
  \xdx{THM}; in that case, use {\tt RLN}.

\item[\tt$thm@1$ RS $thm@2$] \indexbold{*RS} 
abbreviates \hbox{\tt$thm@1$ RSN $(1,thm@2)$}.  Thus, it resolves the
conclusion of $thm@1$ with the first premise of~$thm@2$.

\item[\tt {$[thm@1,\ldots,thm@n]$} MRS $thm$] \indexbold{*MRS} 
  uses {\tt RSN} to resolve $thm@i$ against premise~$i$ of $thm$, for
  $i=n$, \ldots,~1.  This applies $thm@n$, \ldots, $thm@1$ to the first $n$
  premises of $thm$.  Because the theorems are used from right to left, it
  does not matter if the $thm@i$ create new premises.  {\tt MRS} is useful
  for expressing proof trees.

\item[\tt$thms@1$ RLN $(i,thms@2)$] \indexbold{*RLN} 
  joins lists of theorems.  For every $thm@1$ in $thms@1$ and $thm@2$ in
  $thms@2$, it resolves the conclusion of $thm@1$ with the $i$th premise
  of~$thm@2$, accumulating the results. 

\item[\tt$thms@1$ RL $thms@2$] \indexbold{*RL} 
abbreviates \hbox{\tt$thms@1$ RLN $(1,thms@2)$}. 

\item[\tt {$[thms@1,\ldots,thms@n]$} MRL $thms$] \indexbold{*MRL} 
is analogous to {\tt MRS}, but combines theorem lists rather than theorems.
It too is useful for expressing proof trees.
\end{ttdescription}


\subsection{Expanding definitions in theorems}
\index{meta-rewriting!in theorems}
\begin{ttbox} 
rewrite_rule       : thm list -> thm -> thm
rewrite_goals_rule : thm list -> thm -> thm
\end{ttbox}
\begin{ttdescription}
\item[\ttindexbold{rewrite_rule} {\it defs} {\it thm}]  
unfolds the {\it defs} throughout the theorem~{\it thm}.

\item[\ttindexbold{rewrite_goals_rule} {\it defs} {\it thm}]  
unfolds the {\it defs} in the premises of~{\it thm}, but leaves the
conclusion unchanged.  This rule underlies \ttindex{rewrite_goals_tac}, but 
serves little purpose in forward proof.
\end{ttdescription}


\subsection{Instantiating a theorem}
\index{instantiation}
\begin{ttbox}
read_instantiate    :            (string*string)list -> thm -> thm
read_instantiate_sg : Sign.sg -> (string*string)list -> thm -> thm
cterm_instantiate   :    (Sign.cterm*Sign.cterm)list -> thm -> thm
\end{ttbox}
These meta-rules instantiate type and term unknowns in a theorem.  They are
occasionally useful.  They can prevent difficulties with higher-order
unification, and define specialized versions of rules.
\begin{ttdescription}
\item[\ttindexbold{read_instantiate} {\it insts} {\it thm}] 
processes the instantiations {\it insts} and instantiates the rule~{\it
thm}.  The processing of instantiations is described
in \S\ref{res_inst_tac}, under {\tt res_inst_tac}.  

Use {\tt res_inst_tac}, not {\tt read_instantiate}, to instantiate a rule
and refine a particular subgoal.  The tactic allows instantiation by the
subgoal's parameters, and reads the instantiations using the signature
associated with the proof state.

Use {\tt read_instantiate_sg} below if {\it insts\/} appears to be treated
incorrectly.

\item[\ttindexbold{read_instantiate_sg} {\it sg} {\it insts} {\it thm}]
  resembles \hbox{\tt read_instantiate {\it insts} {\it thm}}, but reads
  the instantiations under signature~{\it sg}.  This is necessary to
  instantiate a rule from a general theory, such as first-order logic,
  using the notation of some specialized theory.  Use the function {\tt
    sign_of} to get a theory's signature.

\item[\ttindexbold{cterm_instantiate} {\it ctpairs} {\it thm}] 
is similar to {\tt read_instantiate}, but the instantiations are provided
as pairs of certified terms, not as strings to be read.
\end{ttdescription}


\subsection{Miscellaneous forward rules}\label{MiscellaneousForwardRules}
\index{theorems!standardizing}
\begin{ttbox} 
standard         :           thm -> thm
zero_var_indexes :           thm -> thm
make_elim        :           thm -> thm
rule_by_tactic   : tactic -> thm -> thm
\end{ttbox}
\begin{ttdescription}
\item[\ttindexbold{standard} $thm$]  
puts $thm$ into the standard form of object-rules.  It discharges all
meta-assumptions, replaces free variables by schematic variables, and
renames schematic variables to have subscript zero.

\item[\ttindexbold{zero_var_indexes} $thm$] 
makes all schematic variables have subscript zero, renaming them to avoid
clashes. 

\item[\ttindexbold{make_elim} $thm$] 
\index{rules!converting destruction to elimination}
converts $thm$, a destruction rule of the form $\List{P@1;\ldots;P@m}\Imp
Q$, to the elimination rule $\List{P@1; \ldots; P@m; Q\Imp R}\Imp R$.  This
is the basis for destruct-resolution: {\tt dresolve_tac}, etc.

\item[\ttindexbold{rule_by_tactic} {\it tac} {\it thm}] 
  applies {\it tac\/} to the {\it thm}, freezing its variables first, then
  yields the proof state returned by the tactic.  In typical usage, the
  {\it thm\/} represents an instance of a rule with several premises, some
  with contradictory assumptions (because of the instantiation).  The
  tactic proves those subgoals and does whatever else it can, and returns
  whatever is left.
\end{ttdescription}


\subsection{Taking a theorem apart}
\index{theorems!taking apart}
\index{flex-flex constraints}
\begin{ttbox} 
concl_of      : thm -> term
prems_of      : thm -> term list
nprems_of     : thm -> int
tpairs_of     : thm -> (term*term)list
stamps_of_thy : thm -> string ref list
theory_of_thm : thm -> theory
dest_state    : thm*int -> (term*term)list*term list*term*term
rep_thm       : thm -> \{prop:term, hyps:term list, 
                        maxidx:int, sign:Sign.sg\}
\end{ttbox}
\begin{ttdescription}
\item[\ttindexbold{concl_of} $thm$] 
returns the conclusion of $thm$ as a term.

\item[\ttindexbold{prems_of} $thm$] 
returns the premises of $thm$ as a list of terms.

\item[\ttindexbold{nprems_of} $thm$] 
returns the number of premises in $thm$, and is equivalent to {\tt
  length(prems_of~$thm$)}.

\item[\ttindexbold{tpairs_of} $thm$] 
returns the flex-flex constraints of $thm$.

\item[\ttindexbold{stamps_of_thm} $thm$] 
returns the \rmindex{stamps} of the signature associated with~$thm$.

\item[\ttindexbold{theory_of_thm} $thm$]
returns the theory associated with $thm$.

\item[\ttindexbold{dest_state} $(thm,i)$] 
decomposes $thm$ as a tuple containing a list of flex-flex constraints, a
list of the subgoals~1 to~$i-1$, subgoal~$i$, and the rest of the theorem
(this will be an implication if there are more than $i$ subgoals).

\item[\ttindexbold{rep_thm} $thm$] 
decomposes $thm$ as a record containing the statement of~$thm$, its list of
meta-assumptions, the maximum subscript of its unknowns, and its signature.
\end{ttdescription}


\subsection{Tracing flags for unification}
\index{tracing!of unification}
\begin{ttbox} 
Unify.trace_simp   : bool ref \hfill{\bf initially false}
Unify.trace_types  : bool ref \hfill{\bf initially false}
Unify.trace_bound  : int ref \hfill{\bf initially 10}
Unify.search_bound : int ref \hfill{\bf initially 20}
\end{ttbox}
Tracing the search may be useful when higher-order unification behaves
unexpectedly.  Letting {\tt res_inst_tac} circumvent the problem is easier,
though.
\begin{ttdescription}
\item[Unify.trace_simp := true;] 
causes tracing of the simplification phase.

\item[Unify.trace_types := true;] 
generates warnings of incompleteness, when unification is not considering
all possible instantiations of type unknowns.

\item[Unify.trace_bound := $n$;] 
causes unification to print tracing information once it reaches depth~$n$.
Use $n=0$ for full tracing.  At the default value of~10, tracing
information is almost never printed.

\item[Unify.search_bound := $n$;] 
causes unification to limit its search to depth~$n$.  Because of this
bound, higher-order unification cannot return an infinite sequence, though
it can return a very long one.  The search rarely approaches the default
value of~20.  If the search is cut off, unification prints {\tt
***Unification bound exceeded}.
\end{ttdescription}


\section{Primitive meta-level inference rules}
\index{meta-rules|(}
These implement the meta-logic in {\sc lcf} style, as functions from theorems
to theorems.  They are, rarely, useful for deriving results in the pure
theory.  Mainly, they are included for completeness, and most users should
not bother with them.  The meta-rules raise exception \xdx{THM} to signal
malformed premises, incompatible signatures and similar errors.

\index{meta-assumptions}
The meta-logic uses natural deduction.  Each theorem may depend on
meta-level assumptions.  Certain rules, such as $({\Imp}I)$,
discharge assumptions; in most other rules, the conclusion depends on all
of the assumptions of the premises.  Formally, the system works with
assertions of the form
\[ \phi \quad [\phi@1,\ldots,\phi@n], \]
where $\phi@1$,~\ldots,~$\phi@n$ are the assumptions.  Do not confuse
meta-level assumptions with the object-level assumptions in a subgoal,
which are represented in the meta-logic using~$\Imp$.

Each theorem has a signature.  Certified terms have a signature.  When a
rule takes several premises and certified terms, it merges the signatures
to make a signature for the conclusion.  This fails if the signatures are
incompatible. 

\index{meta-implication}
The {\bf implication} rules are $({\Imp}I)$
and $({\Imp}E)$:
\[ \infer[({\Imp}I)]{\phi\Imp \psi}{\infer*{\psi}{[\phi]}}  \qquad
   \infer[({\Imp}E)]{\psi}{\phi\Imp \psi & \phi}  \]

\index{meta-equality}
Equality of truth values means logical equivalence:
\[ \infer[({\equiv}I)]{\phi\equiv\psi}{\infer*{\psi}{[\phi]} &
                                       \infer*{\phi}{[\psi]}}  
   \qquad
   \infer[({\equiv}E)]{\psi}{\phi\equiv \psi & \phi}   \]

The {\bf equality} rules are reflexivity, symmetry, and transitivity:
\[ {a\equiv a}\,(refl)  \qquad
   \infer[(sym)]{b\equiv a}{a\equiv b}  \qquad
   \infer[(trans)]{a\equiv c}{a\equiv b & b\equiv c}   \]

\index{lambda calc@$\lambda$-calculus}
The $\lambda$-conversions are $\alpha$-conversion, $\beta$-conversion, and
extensionality:\footnote{$\alpha$-conversion holds if $y$ is not free
in~$a$; $(ext)$ holds if $x$ is not free in the assumptions, $f$, or~$g$.}
\[ {(\lambda x.a) \equiv (\lambda y.a[y/x])}    \qquad
   {((\lambda x.a)(b)) \equiv a[b/x]}           \qquad
   \infer[(ext)]{f\equiv g}{f(x) \equiv g(x)}   \]

The {\bf abstraction} and {\bf combination} rules let conversions be
applied to subterms:\footnote{Abstraction holds if $x$ is not free in the
assumptions.}
\[  \infer[(abs)]{(\lambda x.a) \equiv (\lambda x.b)}{a\equiv b}   \qquad
    \infer[(comb)]{f(a)\equiv g(b)}{f\equiv g & a\equiv b}   \]

\index{meta-quantifiers}
The {\bf universal quantification} rules are $(\Forall I)$ and $(\Forall
E)$:\footnote{$(\Forall I)$ holds if $x$ is not free in the assumptions.}
\[ \infer[(\Forall I)]{\Forall x.\phi}{\phi}        \qquad
   \infer[(\Forall E)]{\phi[b/x]}{\Forall x.\phi}   \]


\subsection{Assumption rule}
\index{meta-assumptions}
\begin{ttbox} 
assume: Sign.cterm -> thm
\end{ttbox}
\begin{ttdescription}
\item[\ttindexbold{assume} $ct$] 
makes the theorem \(\phi \;[\phi]\), where $\phi$ is the value of~$ct$.
The rule checks that $ct$ has type $prop$ and contains no unknowns, which
are not allowed in assumptions.
\end{ttdescription}

\subsection{Implication rules}
\index{meta-implication}
\begin{ttbox} 
implies_intr      : Sign.cterm -> thm -> thm
implies_intr_list : Sign.cterm list -> thm -> thm
implies_intr_hyps : thm -> thm
implies_elim      : thm -> thm -> thm
implies_elim_list : thm -> thm list -> thm
\end{ttbox}
\begin{ttdescription}
\item[\ttindexbold{implies_intr} $ct$ $thm$] 
is $({\Imp}I)$, where $ct$ is the assumption to discharge, say~$\phi$.  It
maps the premise~$\psi$ to the conclusion $\phi\Imp\psi$, removing all
occurrences of~$\phi$ from the assumptions.  The rule checks that $ct$ has
type $prop$. 

\item[\ttindexbold{implies_intr_list} $cts$ $thm$] 
applies $({\Imp}I)$ repeatedly, on every element of the list~$cts$.

\item[\ttindexbold{implies_intr_hyps} $thm$] 
applies $({\Imp}I)$ to discharge all the hypotheses (assumptions) of~$thm$.
It maps the premise $\phi \; [\phi@1,\ldots,\phi@n]$ to the conclusion
$\List{\phi@1,\ldots,\phi@n}\Imp\phi$.

\item[\ttindexbold{implies_elim} $thm@1$ $thm@2$] 
applies $({\Imp}E)$ to $thm@1$ and~$thm@2$.  It maps the premises $\phi\Imp
\psi$ and $\phi$ to the conclusion~$\psi$.

\item[\ttindexbold{implies_elim_list} $thm$ $thms$] 
applies $({\Imp}E)$ repeatedly to $thm$, using each element of~$thms$ in
turn.  It maps the premises $\List{\phi@1,\ldots,\phi@n}\Imp\psi$ and
$\phi@1$,\ldots,$\phi@n$ to the conclusion~$\psi$.
\end{ttdescription}

\subsection{Logical equivalence rules}
\index{meta-equality}
\begin{ttbox} 
equal_intr : thm -> thm -> thm 
equal_elim : thm -> thm -> thm
\end{ttbox}
\begin{ttdescription}
\item[\ttindexbold{equal_intr} $thm@1$ $thm@2$] 
applies $({\equiv}I)$ to $thm@1$ and~$thm@2$.  It maps the premises~$\psi$
and~$\phi$ to the conclusion~$\phi\equiv\psi$; the assumptions are those of
the first premise with~$\phi$ removed, plus those of
the second premise with~$\psi$ removed.

\item[\ttindexbold{equal_elim} $thm@1$ $thm@2$] 
applies $({\equiv}E)$ to $thm@1$ and~$thm@2$.  It maps the premises
$\phi\equiv\psi$ and $\phi$ to the conclusion~$\psi$.
\end{ttdescription}


\subsection{Equality rules}
\index{meta-equality}
\begin{ttbox} 
reflexive  : Sign.cterm -> thm
symmetric  : thm -> thm
transitive : thm -> thm -> thm
\end{ttbox}
\begin{ttdescription}
\item[\ttindexbold{reflexive} $ct$] 
makes the theorem \(ct\equiv ct\). 

\item[\ttindexbold{symmetric} $thm$] 
maps the premise $a\equiv b$ to the conclusion $b\equiv a$.

\item[\ttindexbold{transitive} $thm@1$ $thm@2$] 
maps the premises $a\equiv b$ and $b\equiv c$ to the conclusion~${a\equiv c}$.
\end{ttdescription}


\subsection{The $\lambda$-conversion rules}
\index{lambda calc@$\lambda$-calculus}
\begin{ttbox} 
beta_conversion : Sign.cterm -> thm
extensional     : thm -> thm
abstract_rule   : string -> Sign.cterm -> thm -> thm
combination     : thm -> thm -> thm
\end{ttbox} 
There is no rule for $\alpha$-conversion because Isabelle regards
$\alpha$-convertible theorems as equal.
\begin{ttdescription}
\item[\ttindexbold{beta_conversion} $ct$] 
makes the theorem $((\lambda x.a)(b)) \equiv a[b/x]$, where $ct$ is the
term $(\lambda x.a)(b)$.

\item[\ttindexbold{extensional} $thm$] 
maps the premise $f(x) \equiv g(x)$ to the conclusion $f\equiv g$.
Parameter~$x$ is taken from the premise.  It may be an unknown or a free
variable (provided it does not occur in the assumptions); it must not occur
in $f$ or~$g$.

\item[\ttindexbold{abstract_rule} $v$ $x$ $thm$] 
maps the premise $a\equiv b$ to the conclusion $(\lambda x.a) \equiv
(\lambda x.b)$, abstracting over all occurrences (if any!) of~$x$.
Parameter~$x$ is supplied as a cterm.  It may be an unknown or a free
variable (provided it does not occur in the assumptions).  In the
conclusion, the bound variable is named~$v$.

\item[\ttindexbold{combination} $thm@1$ $thm@2$] 
maps the premises $f\equiv g$ and $a\equiv b$ to the conclusion~$f(a)\equiv
g(b)$.
\end{ttdescription}


\subsection{Forall introduction rules}
\index{meta-quantifiers}
\begin{ttbox} 
forall_intr       : Sign.cterm      -> thm -> thm
forall_intr_list  : Sign.cterm list -> thm -> thm
forall_intr_frees :                    thm -> thm
\end{ttbox}

\begin{ttdescription}
\item[\ttindexbold{forall_intr} $x$ $thm$] 
applies $({\Forall}I)$, abstracting over all occurrences (if any!) of~$x$.
The rule maps the premise $\phi$ to the conclusion $\Forall x.\phi$.
Parameter~$x$ is supplied as a cterm.  It may be an unknown or a free
variable (provided it does not occur in the assumptions).

\item[\ttindexbold{forall_intr_list} $xs$ $thm$] 
applies $({\Forall}I)$ repeatedly, on every element of the list~$xs$.

\item[\ttindexbold{forall_intr_frees} $thm$] 
applies $({\Forall}I)$ repeatedly, generalizing over all the free variables
of the premise.
\end{ttdescription}


\subsection{Forall elimination rules}
\begin{ttbox} 
forall_elim       : Sign.cterm      -> thm -> thm
forall_elim_list  : Sign.cterm list -> thm -> thm
forall_elim_var   :             int -> thm -> thm
forall_elim_vars  :             int -> thm -> thm
\end{ttbox}

\begin{ttdescription}
\item[\ttindexbold{forall_elim} $ct$ $thm$] 
applies $({\Forall}E)$, mapping the premise $\Forall x.\phi$ to the conclusion
$\phi[ct/x]$.  The rule checks that $ct$ and $x$ have the same type.

\item[\ttindexbold{forall_elim_list} $cts$ $thm$] 
applies $({\Forall}E)$ repeatedly, on every element of the list~$cts$.

\item[\ttindexbold{forall_elim_var} $k$ $thm$] 
applies $({\Forall}E)$, mapping the premise $\Forall x.\phi$ to the conclusion
$\phi[\Var{x@k}/x]$.  Thus, it replaces the outermost $\Forall$-bound
variable by an unknown having subscript~$k$.

\item[\ttindexbold{forall_elim_vars} $ks$ $thm$] 
applies {\tt forall_elim_var} repeatedly, for every element of the list~$ks$.
\end{ttdescription}

\subsection{Instantiation of unknowns}
\index{instantiation}
\begin{ttbox} 
instantiate: (indexname*Sign.ctyp)list * 
             (Sign.cterm*Sign.cterm)list  -> thm -> thm
\end{ttbox}
\begin{ttdescription}
\item[\ttindexbold{instantiate} ($tyinsts$, $insts$) $thm$] 
simultaneously substitutes types for type unknowns (the
$tyinsts$) and terms for term unknowns (the $insts$).  Instantiations are
given as $(v,t)$ pairs, where $v$ is an unknown and $t$ is a term (of the
same type as $v$) or a type (of the same sort as~$v$).  All the unknowns
must be distinct.  The rule normalizes its conclusion.
\end{ttdescription}


\subsection{Freezing/thawing type unknowns}
\index{type unknowns!freezing/thawing of}
\begin{ttbox} 
freezeT: thm -> thm
varifyT: thm -> thm
\end{ttbox}
\begin{ttdescription}
\item[\ttindexbold{freezeT} $thm$] 
converts all the type unknowns in $thm$ to free type variables.

\item[\ttindexbold{varifyT} $thm$] 
converts all the free type variables in $thm$ to type unknowns.
\end{ttdescription}


\section{Derived rules for goal-directed proof}
Most of these rules have the sole purpose of implementing particular
tactics.  There are few occasions for applying them directly to a theorem.

\subsection{Proof by assumption}
\index{meta-assumptions}
\begin{ttbox} 
assumption    : int -> thm -> thm Sequence.seq
eq_assumption : int -> thm -> thm
\end{ttbox}
\begin{ttdescription}
\item[\ttindexbold{assumption} {\it i} $thm$] 
attempts to solve premise~$i$ of~$thm$ by assumption.

\item[\ttindexbold{eq_assumption}] 
is like {\tt assumption} but does not use unification.
\end{ttdescription}


\subsection{Resolution}
\index{resolution}
\begin{ttbox} 
biresolution : bool -> (bool*thm)list -> int -> thm
               -> thm Sequence.seq
\end{ttbox}
\begin{ttdescription}
\item[\ttindexbold{biresolution} $match$ $rules$ $i$ $state$] 
performs bi-resolution on subgoal~$i$ of $state$, using the list of $\it
(flag,rule)$ pairs.  For each pair, it applies resolution if the flag
is~{\tt false} and elim-resolution if the flag is~{\tt true}.  If $match$
is~{\tt true}, the $state$ is not instantiated.
\end{ttdescription}


\subsection{Composition: resolution without lifting}
\index{resolution!without lifting}
\begin{ttbox}
compose   : thm * int * thm -> thm list
COMP      : thm * thm -> thm
bicompose : bool -> bool * thm * int -> int -> thm
            -> thm Sequence.seq
\end{ttbox}
In forward proof, a typical use of composition is to regard an assertion of
the form $\phi\Imp\psi$ as atomic.  Schematic variables are not renamed, so
beware of clashes!
\begin{ttdescription}
\item[\ttindexbold{compose} ($thm@1$, $i$, $thm@2$)] 
uses $thm@1$, regarded as an atomic formula, to solve premise~$i$
of~$thm@2$.  Let $thm@1$ and $thm@2$ be $\psi$ and $\List{\phi@1; \ldots;
\phi@n} \Imp \phi$.  For each $s$ that unifies~$\psi$ and $\phi@i$, the
result list contains the theorem
\[ (\List{\phi@1; \ldots; \phi@{i-1}; \phi@{i+1}; \ldots; \phi@n} \Imp \phi)s.
\]

\item[\tt $thm@1$ COMP $thm@2$] 
calls \hbox{\tt compose ($thm@1$, 1, $thm@2$)} and returns the result, if
unique; otherwise, it raises exception~\xdx{THM}\@.  It is
analogous to {\tt RS}\@.  

For example, suppose that $thm@1$ is $a=b\Imp b=a$, a symmetry rule, and
that $thm@2$ is $\List{P\Imp Q; \neg Q} \Imp\neg P$, which is the
principle of contrapositives.  Then the result would be the
derived rule $\neg(b=a)\Imp\neg(a=b)$.

\item[\ttindexbold{bicompose} $match$ ($flag$, $rule$, $m$) $i$ $state$]
refines subgoal~$i$ of $state$ using $rule$, without lifting.  The $rule$
is taken to have the form $\List{\psi@1; \ldots; \psi@m} \Imp \psi$, where
$\psi$ need not be atomic; thus $m$ determines the number of new
subgoals.  If $flag$ is {\tt true} then it performs elim-resolution --- it
solves the first premise of~$rule$ by assumption and deletes that
assumption.  If $match$ is~{\tt true}, the $state$ is not instantiated.
\end{ttdescription}


\subsection{Other meta-rules}
\begin{ttbox} 
trivial            : Sign.cterm -> thm
lift_rule          : (thm * int) -> thm -> thm
rename_params_rule : string list * int -> thm -> thm
rewrite_cterm      : thm list -> Sign.cterm -> thm
flexflex_rule      : thm -> thm Sequence.seq
\end{ttbox}
\begin{ttdescription}
\item[\ttindexbold{trivial} $ct$] 
makes the theorem \(\phi\Imp\phi\), where $\phi$ is the value of~$ct$.
This is the initial state for a goal-directed proof of~$\phi$.  The rule
checks that $ct$ has type~$prop$.

\item[\ttindexbold{lift_rule} ($state$, $i$) $rule$] \index{lifting}
prepares $rule$ for resolution by lifting it over the parameters and
assumptions of subgoal~$i$ of~$state$.

\item[\ttindexbold{rename_params_rule} ({\it names}, {\it i}) $thm$] 
uses the $names$ to rename the parameters of premise~$i$ of $thm$.  The
names must be distinct.  If there are fewer names than parameters, then the
rule renames the innermost parameters and may modify the remaining ones to
ensure that all the parameters are distinct.
\index{parameters!renaming}

\item[\ttindexbold{rewrite_cterm} $defs$ $ct$]
transforms $ct$ to $ct'$ by repeatedly applying $defs$ as rewrite rules; it
returns the conclusion~$ct\equiv ct'$.  This underlies the meta-rewriting
tactics and rules.
\index{meta-rewriting!in terms}

\item[\ttindexbold{flexflex_rule} $thm$]  \index{flex-flex constraints}
removes all flex-flex pairs from $thm$ using the trivial unifier.
\end{ttdescription}
\index{theorems|)}
\index{meta-rules|)}
\input{ToyList/inductxs}\input{ToyList/autotac}\end{ttbox}

\subsubsection*{Review}

This is the end of our toy proof. It should have familiarized you with
\begin{itemize}
\item the standard theorem proving procedure:
state a goal; proceed with proof until a new lemma is required; prove that
lemma; come back to the original goal.
\item a specific procedure that works well for functional programs:
induction followed by all-out simplification via \texttt{Auto_tac}.
\item a basic repertoire of proof commands.
\end{itemize}


\section{Some helpful commands}
\label{sec:commands-and-hints}

This section discusses a few basic commands for manipulating the proof state
and can be skipped by casual readers.

There are two kinds of commands used during a proof: the actual proof
commands and auxiliary commands for examining the proof state and controlling
the display. Proof commands are always of the form
\texttt{by(\textit{tactic});}\indexbold{tactic} where \textbf{tactic} is a
synonym for ``theorem proving function''. Typical examples are
\texttt{induct_tac} and \texttt{Auto_tac} --- the suffix \texttt{_tac} is
merely a mnemonic. Further tactics are introduced throughout the tutorial.

%Tactics can also be modified. For example,
%\begin{ttbox}
%by(ALLGOALS Asm_simp_tac);
%\end{ttbox}
%tells Isabelle to apply \texttt{Asm_simp_tac} to all subgoals. For more on
%tactics and how to combine them see~\S\ref{sec:Tactics}.

The most useful auxiliary commands are:
\begin{description}
\item[Printing the current state]
Type \texttt{pr();} to redisplay the current proof state, for example when it
has disappeared off the screen.
\item[Limiting the number of subgoals]
Typing \texttt{prlim $k$;} tells Isabelle to print only the first $k$
subgoals from now on and redisplays the current proof state. This is helpful
when there are many subgoals.
\item[Undoing] Typing \texttt{undo();} undoes the effect of the last
tactic.
\item[Context switch] Every proof happens in the context of a
  \bfindex{current theory}. By default, this is the last theory loaded. If
  you want to prove a theorem in the context of a different theory
  \texttt{T}, you need to type \texttt{context T.thy;}\index{*context|bold}
  first. Of course you need to change the context again if you want to go
  back to your original theory.
\item[Displaying types] We have already mentioned the flag
  \ttindex{show_types} above. It can also be useful for detecting typos in
  formulae early on. For example, if \texttt{show_types} is set and the goal
  \texttt{rev(rev xs) = xs} is started, Isabelle prints the additional output
\begin{ttbox}
{\out Variables:}
{\out   xs :: 'a list}
\end{ttbox}
which tells us that Isabelle has correctly inferred that
\texttt{xs} is a variable of list type. On the other hand, had we
made a typo as in \texttt{rev(re xs) = xs}, the response
\begin{ttbox}
Variables:
  re :: 'a list => 'a list
  xs :: 'a list
\end{ttbox}
would have alerted us because of the unexpected variable \texttt{re}.
\item[(Re)loading theories]\index{loading theories}\index{reloading theories}
Initially you load theory \texttt{T} by typing \ttindex{use_thy}~\texttt{"T";},
which loads all parent theories of \texttt{T} automatically, if they are not
loaded already. If you modify \texttt{T.thy} or \texttt{T.ML}, you can
reload it by typing \texttt{use_thy~"T";} again. This time, however, only
\texttt{T} is reloaded. If some of \texttt{T}'s parents have changed as well,
type \ttindexbold{update}\texttt{();} to reload all theories that have
changed.
\end{description}
Further commands are found in the Reference Manual.


\section{Datatypes}
\label{sec:datatype}

Inductive datatypes are part of almost every non-trivial application of HOL.
First we take another look at a very important example, the datatype of
lists, before we turn to datatypes in general. The section closes with a
case study.


\subsection{Lists}

Lists are one of the essential datatypes in computing. Readers of this tutorial
and users of HOL need to be familiar with their basic operations. Theory
\texttt{ToyList} is only a small fragment of HOL's predefined theory
\texttt{List}\footnote{\texttt{http://www.in.tum.de/\~\relax
    isabelle/library/HOL/List.html}}.
The latter contains many further operations. For example, the functions
\ttindexbold{hd} (`head') and \ttindexbold{tl} (`tail') return the first
element and the remainder of a list. (However, pattern-matching is usually
preferable to \texttt{hd} and \texttt{tl}.)
Theory \texttt{List} also contains more syntactic sugar:
\texttt{[}$x@1$\texttt{,}\dots\texttt{,}$x@n$\texttt{]} abbreviates
$x@1$\texttt{\#}\dots\texttt{\#}$x@n$\texttt{\#[]}.
In the rest of the tutorial we always use HOL's predefined lists.


\subsection{The general format}
\label{sec:general-datatype}

The general HOL \texttt{datatype} definition is of the form
\[
\mathtt{datatype}~(\alpha@1, \dots, \alpha@n) \, t ~=~
C@1~\tau@{11}~\dots~\tau@{1k@1} ~\mid~ \dots ~\mid~
C@m~\tau@{m1}~\dots~\tau@{mk@m}
\]
where $\alpha@i$ are type variables (the parameters), $C@i$ are distinct
constructor names and $\tau@{ij}$ are types; it is customary to capitalize
the first letter in constructor names. There are a number of
restrictions (such as the type should not be empty) detailed
elsewhere~\cite{Isa-Logics-Man}. Isabelle notifies you if you violate them.

Laws about datatypes, such as \verb$[] ~= x#xs$ and \texttt{(x\#xs = y\#ys) =
  (x=y \& xs=ys)}, are used automatically during proofs by simplification.
The same is true for the equations in primitive recursive function
definitions.

\subsection{Primitive recursion}

Functions on datatypes are usually defined by recursion. In fact, most of the
time they are defined by what is called \bfindex{primitive recursion}.
The keyword \texttt{primrec} is followed by a list of equations
\[ f \, x@1 \, \dots \, (C \, y@1 \, \dots \, y@k)\, \dots \, x@n = r \]
such that $C$ is a constructor of the datatype $t$ and all recursive calls of
$f$ in $r$ are of the form $f \, \dots \, y@i \, \dots$ for some $i$. Thus
Isabelle immediately sees that $f$ terminates because one (fixed!) argument
becomes smaller with every recursive call. There must be exactly one equation
for each constructor.  Their order is immaterial.
A more general method for defining total recursive functions is explained in
\S\ref{sec:recdef}.

\begin{exercise}
Given the datatype of binary trees
\begin{ttbox}
\input{Misc/tree}\end{ttbox}
define a function \texttt{mirror} that mirrors the structure of a binary tree
by swapping subtrees (recursively). Prove \texttt{mirror(mirror(t)) = t}.
\end{exercise}

\subsection{\texttt{case}-expressions}
\label{sec:case-expressions}

HOL also features \ttindexbold{case}-expressions for analyzing elements of a
datatype. For example,
\begin{ttbox}
case xs of [] => 0 | y#ys => y
\end{ttbox}
evaluates to \texttt{0} if \texttt{xs} is \texttt{[]} and to \texttt{y} if 
\texttt{xs} is \texttt{y\#ys}. (Since the result in both branches must be of
the same type, it follows that \texttt{y::nat} and hence
\texttt{xs::(nat)list}.)

In general, if $e$ is a term of the datatype $t$ defined in
\S\ref{sec:general-datatype} above, the corresponding
\texttt{case}-expression analyzing $e$ is
\[
\begin{array}{rrcl}
\mbox{\tt case}~e~\mbox{\tt of} & C@1~x@{11}~\dots~x@{1k@1} & \To & e@1 \\
                           \vdots \\
                           \mid & C@m~x@{m1}~\dots~x@{mk@m} & \To & e@m
\end{array}
\]

\begin{warn}
{\em All} constructors must be present, their order is fixed, and nested
patterns are not supported.  Violating these restrictions results in strange
error messages.
\end{warn}
\noindent
Nested patterns can be simulated by nested \texttt{case}-expressions: instead
of
\begin{ttbox}
case xs of [] => 0 | [x] => x | x#(y#zs) => y
\end{ttbox}
write
\begin{ttbox}
case xs of [] => 0 | x#ys => (case ys of [] => x | y#zs => y)
\end{ttbox}
Note that \texttt{case}-expressions should be enclosed in parentheses to
indicate their scope.

\subsection{Structural induction}

Almost all the basic laws about a datatype are applied automatically during
simplification. Only induction is invoked by hand via \texttt{induct_tac},
which works for any datatype. In some cases, induction is overkill and a case
distinction over all constructors of the datatype suffices. This is performed
by \ttindexbold{exhaust_tac}. A trivial example:
\begin{ttbox}
\input{Misc/exhaust.ML}{\out1. xs = [] ==> (case xs of [] => [] | y # ys => xs) = xs}
{\out2. !!a list. xs = a # list ==> (case xs of [] => [] | y # ys => xs) = xs}
\input{Misc/autotac.ML}\end{ttbox}
Note that this particular case distinction could have been automated
completely. See~\S\ref{sec:SimpFeatures}.

\begin{warn}
  Induction is only allowed on a free variable that should not occur among
  the assumptions of the subgoal.  Exhaustion works for arbitrary terms.
\end{warn}

\subsection{Case study: boolean expressions}
\label{sec:boolex}

The aim of this case study is twofold: it shows how to model boolean
expressions and some algorithms for manipulating them, and it demonstrates
the constructs introduced above.

\subsubsection{How can we model boolean expressions?}

We want to represent boolean expressions built up from variables and
constants by negation and conjunction. The following datatype serves exactly
that purpose:
\begin{ttbox}
\input{Ifexpr/boolex}\end{ttbox}
The two constants are represented by the terms \texttt{Const~True} and
\texttt{Const~False}. Variables are represented by terms of the form
\texttt{Var}~$n$, where $n$ is a natural number (type \texttt{nat}).
For example, the formula $P@0 \land \neg P@1$ is represented by the term
\texttt{And~(Var~0)~(Neg(Var~1))}.

\subsubsection{What is the value of boolean expressions?}

The value of a boolean expressions depends on the value of its variables.
Hence the function \texttt{value} takes an additional parameter, an {\em
  environment} of type \texttt{nat~=>~bool}, which maps variables to their
values:
\begin{ttbox}
\input{Ifexpr/value}\end{ttbox}

\subsubsection{If-expressions}

An alternative and often more efficient (because in a certain sense
canonical) representation are so-called \textit{If-expressions\/} built up
from constants (\texttt{CIF}), variables (\texttt{VIF}) and conditionals
(\texttt{IF}):
\begin{ttbox}
\input{Ifexpr/ifex}\end{ttbox}
The evaluation if If-expressions proceeds as for \texttt{boolex}:
\begin{ttbox}
\input{Ifexpr/valif}\end{ttbox}

\subsubsection{Transformation into and of If-expressions}

The type \texttt{boolex} is close to the customary representation of logical
formulae, whereas \texttt{ifex} is designed for efficiency. Thus we need to
translate from \texttt{boolex} into \texttt{ifex}:
\begin{ttbox}
\input{Ifexpr/bool2if}\end{ttbox}
At last, we have something we can verify: that \texttt{bool2if} preserves the
value of its argument.
\begin{ttbox}
\input{Ifexpr/bool2if.ML}\end{ttbox}
The proof is canonical:
\begin{ttbox}
%
\begin{isabellebody}%
\def\isabellecontext{Proof}%
%
\isadelimtheory
%
\endisadelimtheory
%
\isatagtheory
\isacommand{theory}\isamarkupfalse%
\ Proof\isanewline
\isakeyword{imports}\ Base\isanewline
\isakeyword{begin}%
\endisatagtheory
{\isafoldtheory}%
%
\isadelimtheory
%
\endisadelimtheory
%
\isamarkupchapter{Structured proofs%
}
\isamarkuptrue%
%
\isamarkupsection{Variables \label{sec:variables}%
}
\isamarkuptrue%
%
\begin{isamarkuptext}%
Any variable that is not explicitly bound by \isa{{\isasymlambda}}-abstraction
  is considered as ``free''.  Logically, free variables act like
  outermost universal quantification at the sequent level: \isa{A\isactrlisub {\isadigit{1}}{\isacharparenleft}x{\isacharparenright}{\isacharcomma}\ {\isasymdots}{\isacharcomma}\ A\isactrlisub n{\isacharparenleft}x{\isacharparenright}\ {\isasymturnstile}\ B{\isacharparenleft}x{\isacharparenright}} means that the result
  holds \emph{for all} values of \isa{x}.  Free variables for
  terms (not types) can be fully internalized into the logic: \isa{{\isasymturnstile}\ B{\isacharparenleft}x{\isacharparenright}} and \isa{{\isasymturnstile}\ {\isasymAnd}x{\isachardot}\ B{\isacharparenleft}x{\isacharparenright}} are interchangeable, provided
  that \isa{x} does not occur elsewhere in the context.
  Inspecting \isa{{\isasymturnstile}\ {\isasymAnd}x{\isachardot}\ B{\isacharparenleft}x{\isacharparenright}} more closely, we see that inside the
  quantifier, \isa{x} is essentially ``arbitrary, but fixed'',
  while from outside it appears as a place-holder for instantiation
  (thanks to \isa{{\isasymAnd}} elimination).

  The Pure logic represents the idea of variables being either inside
  or outside the current scope by providing separate syntactic
  categories for \emph{fixed variables} (e.g.\ \isa{x}) vs.\
  \emph{schematic variables} (e.g.\ \isa{{\isacharquery}x}).  Incidently, a
  universal result \isa{{\isasymturnstile}\ {\isasymAnd}x{\isachardot}\ B{\isacharparenleft}x{\isacharparenright}} has the HHF normal form \isa{{\isasymturnstile}\ B{\isacharparenleft}{\isacharquery}x{\isacharparenright}}, which represents its generality nicely without requiring
  an explicit quantifier.  The same principle works for type
  variables: \isa{{\isasymturnstile}\ B{\isacharparenleft}{\isacharquery}{\isasymalpha}{\isacharparenright}} represents the idea of ``\isa{{\isasymturnstile}\ {\isasymforall}{\isasymalpha}{\isachardot}\ B{\isacharparenleft}{\isasymalpha}{\isacharparenright}}'' without demanding a truly polymorphic framework.

  \medskip Additional care is required to treat type variables in a
  way that facilitates type-inference.  In principle, term variables
  depend on type variables, which means that type variables would have
  to be declared first.  For example, a raw type-theoretic framework
  would demand the context to be constructed in stages as follows:
  \isa{{\isasymGamma}\ {\isacharequal}\ {\isasymalpha}{\isacharcolon}\ type{\isacharcomma}\ x{\isacharcolon}\ {\isasymalpha}{\isacharcomma}\ a{\isacharcolon}\ A{\isacharparenleft}x\isactrlisub {\isasymalpha}{\isacharparenright}}.

  We allow a slightly less formalistic mode of operation: term
  variables \isa{x} are fixed without specifying a type yet
  (essentially \emph{all} potential occurrences of some instance
  \isa{x\isactrlisub {\isasymtau}} are fixed); the first occurrence of \isa{x}
  within a specific term assigns its most general type, which is then
  maintained consistently in the context.  The above example becomes
  \isa{{\isasymGamma}\ {\isacharequal}\ x{\isacharcolon}\ term{\isacharcomma}\ {\isasymalpha}{\isacharcolon}\ type{\isacharcomma}\ A{\isacharparenleft}x\isactrlisub {\isasymalpha}{\isacharparenright}}, where type \isa{{\isasymalpha}} is fixed \emph{after} term \isa{x}, and the constraint
  \isa{x\ {\isacharcolon}{\isacharcolon}\ {\isasymalpha}} is an implicit consequence of the occurrence of
  \isa{x\isactrlisub {\isasymalpha}} in the subsequent proposition.

  This twist of dependencies is also accommodated by the reverse
  operation of exporting results from a context: a type variable
  \isa{{\isasymalpha}} is considered fixed as long as it occurs in some fixed
  term variable of the context.  For example, exporting \isa{x{\isacharcolon}\ term{\isacharcomma}\ {\isasymalpha}{\isacharcolon}\ type\ {\isasymturnstile}\ x\isactrlisub {\isasymalpha}\ {\isacharequal}\ x\isactrlisub {\isasymalpha}} produces in the first step
  \isa{x{\isacharcolon}\ term\ {\isasymturnstile}\ x\isactrlisub {\isasymalpha}\ {\isacharequal}\ x\isactrlisub {\isasymalpha}} for fixed \isa{{\isasymalpha}},
  and only in the second step \isa{{\isasymturnstile}\ {\isacharquery}x\isactrlisub {\isacharquery}\isactrlisub {\isasymalpha}\ {\isacharequal}\ {\isacharquery}x\isactrlisub {\isacharquery}\isactrlisub {\isasymalpha}} for schematic \isa{{\isacharquery}x} and \isa{{\isacharquery}{\isasymalpha}}.

  \medskip The Isabelle/Isar proof context manages the gory details of
  term vs.\ type variables, with high-level principles for moving the
  frontier between fixed and schematic variables.

  The \isa{add{\isacharunderscore}fixes} operation explictly declares fixed
  variables; the \isa{declare{\isacharunderscore}term} operation absorbs a term into
  a context by fixing new type variables and adding syntactic
  constraints.

  The \isa{export} operation is able to perform the main work of
  generalizing term and type variables as sketched above, assuming
  that fixing variables and terms have been declared properly.

  There \isa{import} operation makes a generalized fact a genuine
  part of the context, by inventing fixed variables for the schematic
  ones.  The effect can be reversed by using \isa{export} later,
  potentially with an extended context; the result is equivalent to
  the original modulo renaming of schematic variables.

  The \isa{focus} operation provides a variant of \isa{import}
  for nested propositions (with explicit quantification): \isa{{\isasymAnd}x\isactrlisub {\isadigit{1}}\ {\isasymdots}\ x\isactrlisub n{\isachardot}\ B{\isacharparenleft}x\isactrlisub {\isadigit{1}}{\isacharcomma}\ {\isasymdots}{\isacharcomma}\ x\isactrlisub n{\isacharparenright}} is
  decomposed by inventing fixed variables \isa{x\isactrlisub {\isadigit{1}}{\isacharcomma}\ {\isasymdots}{\isacharcomma}\ x\isactrlisub n} for the body.%
\end{isamarkuptext}%
\isamarkuptrue%
%
\isadelimmlref
%
\endisadelimmlref
%
\isatagmlref
%
\begin{isamarkuptext}%
\begin{mldecls}
  \indexml{Variable.add\_fixes}\verb|Variable.add_fixes: |\isasep\isanewline%
\verb|  string list -> Proof.context -> string list * Proof.context| \\
  \indexml{Variable.variant\_fixes}\verb|Variable.variant_fixes: |\isasep\isanewline%
\verb|  string list -> Proof.context -> string list * Proof.context| \\
  \indexml{Variable.declare\_term}\verb|Variable.declare_term: term -> Proof.context -> Proof.context| \\
  \indexml{Variable.declare\_constraints}\verb|Variable.declare_constraints: term -> Proof.context -> Proof.context| \\
  \indexml{Variable.export}\verb|Variable.export: Proof.context -> Proof.context -> thm list -> thm list| \\
  \indexml{Variable.polymorphic}\verb|Variable.polymorphic: Proof.context -> term list -> term list| \\
  \indexml{Variable.import\_thms}\verb|Variable.import_thms: bool -> thm list -> Proof.context ->|\isasep\isanewline%
\verb|  ((ctyp list * cterm list) * thm list) * Proof.context| \\
  \indexml{Variable.focus}\verb|Variable.focus: cterm -> Proof.context -> (cterm list * cterm) * Proof.context| \\
  \end{mldecls}

  \begin{description}

  \item \verb|Variable.add_fixes|~\isa{xs\ ctxt} fixes term
  variables \isa{xs}, returning the resulting internal names.  By
  default, the internal representation coincides with the external
  one, which also means that the given variables must not be fixed
  already.  There is a different policy within a local proof body: the
  given names are just hints for newly invented Skolem variables.

  \item \verb|Variable.variant_fixes| is similar to \verb|Variable.add_fixes|, but always produces fresh variants of the given
  names.

  \item \verb|Variable.declare_term|~\isa{t\ ctxt} declares term
  \isa{t} to belong to the context.  This automatically fixes new
  type variables, but not term variables.  Syntactic constraints for
  type and term variables are declared uniformly, though.

  \item \verb|Variable.declare_constraints|~\isa{t\ ctxt} declares
  syntactic constraints from term \isa{t}, without making it part
  of the context yet.

  \item \verb|Variable.export|~\isa{inner\ outer\ thms} generalizes
  fixed type and term variables in \isa{thms} according to the
  difference of the \isa{inner} and \isa{outer} context,
  following the principles sketched above.

  \item \verb|Variable.polymorphic|~\isa{ctxt\ ts} generalizes type
  variables in \isa{ts} as far as possible, even those occurring
  in fixed term variables.  The default policy of type-inference is to
  fix newly introduced type variables, which is essentially reversed
  with \verb|Variable.polymorphic|: here the given terms are detached
  from the context as far as possible.

  \item \verb|Variable.import_thms|~\isa{open\ thms\ ctxt} invents fixed
  type and term variables for the schematic ones occurring in \isa{thms}.  The \isa{open} flag indicates whether the fixed names
  should be accessible to the user, otherwise newly introduced names
  are marked as ``internal'' (\secref{sec:names}).

  \item \verb|Variable.focus|~\isa{B} decomposes the outermost \isa{{\isasymAnd}} prefix of proposition \isa{B}.

  \end{description}%
\end{isamarkuptext}%
\isamarkuptrue%
%
\endisatagmlref
{\isafoldmlref}%
%
\isadelimmlref
%
\endisadelimmlref
%
\isamarkupsection{Assumptions \label{sec:assumptions}%
}
\isamarkuptrue%
%
\begin{isamarkuptext}%
An \emph{assumption} is a proposition that it is postulated in the
  current context.  Local conclusions may use assumptions as
  additional facts, but this imposes implicit hypotheses that weaken
  the overall statement.

  Assumptions are restricted to fixed non-schematic statements, i.e.\
  all generality needs to be expressed by explicit quantifiers.
  Nevertheless, the result will be in HHF normal form with outermost
  quantifiers stripped.  For example, by assuming \isa{{\isasymAnd}x\ {\isacharcolon}{\isacharcolon}\ {\isasymalpha}{\isachardot}\ P\ x} we get \isa{{\isasymAnd}x\ {\isacharcolon}{\isacharcolon}\ {\isasymalpha}{\isachardot}\ P\ x\ {\isasymturnstile}\ P\ {\isacharquery}x} for schematic \isa{{\isacharquery}x}
  of fixed type \isa{{\isasymalpha}}.  Local derivations accumulate more and
  more explicit references to hypotheses: \isa{A\isactrlisub {\isadigit{1}}{\isacharcomma}\ {\isasymdots}{\isacharcomma}\ A\isactrlisub n\ {\isasymturnstile}\ B} where \isa{A\isactrlisub {\isadigit{1}}{\isacharcomma}\ {\isasymdots}{\isacharcomma}\ A\isactrlisub n} needs to
  be covered by the assumptions of the current context.

  \medskip The \isa{add{\isacharunderscore}assms} operation augments the context by
  local assumptions, which are parameterized by an arbitrary \isa{export} rule (see below).

  The \isa{export} operation moves facts from a (larger) inner
  context into a (smaller) outer context, by discharging the
  difference of the assumptions as specified by the associated export
  rules.  Note that the discharged portion is determined by the
  difference contexts, not the facts being exported!  There is a
  separate flag to indicate a goal context, where the result is meant
  to refine an enclosing sub-goal of a structured proof state.

  \medskip The most basic export rule discharges assumptions directly
  by means of the \isa{{\isasymLongrightarrow}} introduction rule:
  \[
  \infer[(\isa{{\isasymLongrightarrow}{\isacharunderscore}intro})]{\isa{{\isasymGamma}\ {\isacharbackslash}\ A\ {\isasymturnstile}\ A\ {\isasymLongrightarrow}\ B}}{\isa{{\isasymGamma}\ {\isasymturnstile}\ B}}
  \]

  The variant for goal refinements marks the newly introduced
  premises, which causes the canonical Isar goal refinement scheme to
  enforce unification with local premises within the goal:
  \[
  \infer[(\isa{{\isacharhash}{\isasymLongrightarrow}{\isacharunderscore}intro})]{\isa{{\isasymGamma}\ {\isacharbackslash}\ A\ {\isasymturnstile}\ {\isacharhash}A\ {\isasymLongrightarrow}\ B}}{\isa{{\isasymGamma}\ {\isasymturnstile}\ B}}
  \]

  \medskip Alternative versions of assumptions may perform arbitrary
  transformations on export, as long as the corresponding portion of
  hypotheses is removed from the given facts.  For example, a local
  definition works by fixing \isa{x} and assuming \isa{x\ {\isasymequiv}\ t},
  with the following export rule to reverse the effect:
  \[
  \infer[(\isa{{\isasymequiv}{\isacharminus}expand})]{\isa{{\isasymGamma}\ {\isacharbackslash}\ x\ {\isasymequiv}\ t\ {\isasymturnstile}\ B\ t}}{\isa{{\isasymGamma}\ {\isasymturnstile}\ B\ x}}
  \]
  This works, because the assumption \isa{x\ {\isasymequiv}\ t} was introduced in
  a context with \isa{x} being fresh, so \isa{x} does not
  occur in \isa{{\isasymGamma}} here.%
\end{isamarkuptext}%
\isamarkuptrue%
%
\isadelimmlref
%
\endisadelimmlref
%
\isatagmlref
%
\begin{isamarkuptext}%
\begin{mldecls}
  \indexmltype{Assumption.export}\verb|type Assumption.export| \\
  \indexml{Assumption.assume}\verb|Assumption.assume: cterm -> thm| \\
  \indexml{Assumption.add\_assms}\verb|Assumption.add_assms: Assumption.export ->|\isasep\isanewline%
\verb|  cterm list -> Proof.context -> thm list * Proof.context| \\
  \indexml{Assumption.add\_assumes}\verb|Assumption.add_assumes: |\isasep\isanewline%
\verb|  cterm list -> Proof.context -> thm list * Proof.context| \\
  \indexml{Assumption.export}\verb|Assumption.export: bool -> Proof.context -> Proof.context -> thm -> thm| \\
  \end{mldecls}

  \begin{description}

  \item \verb|Assumption.export| represents arbitrary export
  rules, which is any function of type \verb|bool -> cterm list -> thm -> thm|,
  where the \verb|bool| indicates goal mode, and the \verb|cterm list| the collection of assumptions to be discharged
  simultaneously.

  \item \verb|Assumption.assume|~\isa{A} turns proposition \isa{A} into a raw assumption \isa{A\ {\isasymturnstile}\ A{\isacharprime}}, where the conclusion
  \isa{A{\isacharprime}} is in HHF normal form.

  \item \verb|Assumption.add_assms|~\isa{r\ As} augments the context
  by assumptions \isa{As} with export rule \isa{r}.  The
  resulting facts are hypothetical theorems as produced by the raw
  \verb|Assumption.assume|.

  \item \verb|Assumption.add_assumes|~\isa{As} is a special case of
  \verb|Assumption.add_assms| where the export rule performs \isa{{\isasymLongrightarrow}{\isacharunderscore}intro} or \isa{{\isacharhash}{\isasymLongrightarrow}{\isacharunderscore}intro}, depending on goal mode.

  \item \verb|Assumption.export|~\isa{is{\isacharunderscore}goal\ inner\ outer\ thm}
  exports result \isa{thm} from the the \isa{inner} context
  back into the \isa{outer} one; \isa{is{\isacharunderscore}goal\ {\isacharequal}\ true} means
  this is a goal context.  The result is in HHF normal form.  Note
  that \verb|ProofContext.export| combines \verb|Variable.export|
  and \verb|Assumption.export| in the canonical way.

  \end{description}%
\end{isamarkuptext}%
\isamarkuptrue%
%
\endisatagmlref
{\isafoldmlref}%
%
\isadelimmlref
%
\endisadelimmlref
%
\isamarkupsection{Results \label{sec:results}%
}
\isamarkuptrue%
%
\begin{isamarkuptext}%
Local results are established by monotonic reasoning from facts
  within a context.  This allows common combinations of theorems,
  e.g.\ via \isa{{\isasymAnd}{\isacharslash}{\isasymLongrightarrow}} elimination, resolution rules, or equational
  reasoning, see \secref{sec:thms}.  Unaccounted context manipulations
  should be avoided, notably raw \isa{{\isasymAnd}{\isacharslash}{\isasymLongrightarrow}} introduction or ad-hoc
  references to free variables or assumptions not present in the proof
  context.

  \medskip The \isa{SUBPROOF} combinator allows to structure a
  tactical proof recursively by decomposing a selected sub-goal:
  \isa{{\isacharparenleft}{\isasymAnd}x{\isachardot}\ A{\isacharparenleft}x{\isacharparenright}\ {\isasymLongrightarrow}\ B{\isacharparenleft}x{\isacharparenright}{\isacharparenright}\ {\isasymLongrightarrow}\ {\isasymdots}} is turned into \isa{B{\isacharparenleft}x{\isacharparenright}\ {\isasymLongrightarrow}\ {\isasymdots}}
  after fixing \isa{x} and assuming \isa{A{\isacharparenleft}x{\isacharparenright}}.  This means
  the tactic needs to solve the conclusion, but may use the premise as
  a local fact, for locally fixed variables.

  The \isa{prove} operation provides an interface for structured
  backwards reasoning under program control, with some explicit sanity
  checks of the result.  The goal context can be augmented by
  additional fixed variables (cf.\ \secref{sec:variables}) and
  assumptions (cf.\ \secref{sec:assumptions}), which will be available
  as local facts during the proof and discharged into implications in
  the result.  Type and term variables are generalized as usual,
  according to the context.

  The \isa{obtain} operation produces results by eliminating
  existing facts by means of a given tactic.  This acts like a dual
  conclusion: the proof demonstrates that the context may be augmented
  by certain fixed variables and assumptions.  See also
  \cite{isabelle-isar-ref} for the user-level \isa{{\isasymOBTAIN}} and
  \isa{{\isasymGUESS}} elements.  Final results, which may not refer to
  the parameters in the conclusion, need to exported explicitly into
  the original context.%
\end{isamarkuptext}%
\isamarkuptrue%
%
\isadelimmlref
%
\endisadelimmlref
%
\isatagmlref
%
\begin{isamarkuptext}%
\begin{mldecls}
  \indexml{SUBPROOF}\verb|SUBPROOF: ({context: Proof.context, schematics: ctyp list * cterm list,|\isasep\isanewline%
\verb|    params: cterm list, asms: cterm list, concl: cterm,|\isasep\isanewline%
\verb|    prems: thm list} -> tactic) -> Proof.context -> int -> tactic| \\
  \end{mldecls}
  \begin{mldecls}
  \indexml{Goal.prove}\verb|Goal.prove: Proof.context -> string list -> term list -> term ->|\isasep\isanewline%
\verb|  ({prems: thm list, context: Proof.context} -> tactic) -> thm| \\
  \indexml{Goal.prove\_multi}\verb|Goal.prove_multi: Proof.context -> string list -> term list -> term list ->|\isasep\isanewline%
\verb|  ({prems: thm list, context: Proof.context} -> tactic) -> thm list| \\
  \end{mldecls}
  \begin{mldecls}
  \indexml{Obtain.result}\verb|Obtain.result: (Proof.context -> tactic) ->|\isasep\isanewline%
\verb|  thm list -> Proof.context -> (cterm list * thm list) * Proof.context| \\
  \end{mldecls}

  \begin{description}

  \item \verb|SUBPROOF|~\isa{tac\ ctxt\ i} decomposes the structure
  of the specified sub-goal, producing an extended context and a
  reduced goal, which needs to be solved by the given tactic.  All
  schematic parameters of the goal are imported into the context as
  fixed ones, which may not be instantiated in the sub-proof.

  \item \verb|Goal.prove|~\isa{ctxt\ xs\ As\ C\ tac} states goal \isa{C} in the context augmented by fixed variables \isa{xs} and
  assumptions \isa{As}, and applies tactic \isa{tac} to solve
  it.  The latter may depend on the local assumptions being presented
  as facts.  The result is in HHF normal form.

  \item \verb|Goal.prove_multi| is simular to \verb|Goal.prove|, but
  states several conclusions simultaneously.  The goal is encoded by
  means of Pure conjunction; \verb|Goal.conjunction_tac| will turn this
  into a collection of individual subgoals.

  \item \verb|Obtain.result|~\isa{tac\ thms\ ctxt} eliminates the
  given facts using a tactic, which results in additional fixed
  variables and assumptions in the context.  Final results need to be
  exported explicitly.

  \end{description}%
\end{isamarkuptext}%
\isamarkuptrue%
%
\endisatagmlref
{\isafoldmlref}%
%
\isadelimmlref
%
\endisadelimmlref
%
\isadelimtheory
%
\endisadelimtheory
%
\isatagtheory
\isacommand{end}\isamarkupfalse%
%
\endisatagtheory
{\isafoldtheory}%
%
\isadelimtheory
%
\endisadelimtheory
\isanewline
\end{isabellebody}%
%%% Local Variables:
%%% mode: latex
%%% TeX-master: "root"
%%% End:
\end{ttbox}
In fact, all proofs in this case study look exactly like this. Hence we do
not show them below.

More interesting is the transformation of If-expressions into a normal form
where the first argument of \texttt{IF} cannot be another \texttt{IF} but
must be a constant or variable. Such a normal form can be computed by
repeatedly replacing a subterm of the form \texttt{IF~(IF~b~x~y)~z~u} by
\texttt{IF b (IF x z u) (IF y z u)}, which has the same value. The following
primitive recursive functions perform this task:
\begin{ttbox}
\input{Ifexpr/normif}
\input{Ifexpr/norm}\end{ttbox}
Their interplay is a bit tricky, and we leave it to the reader to develop an
intuitive understanding. Fortunately, Isabelle can help us to verify that the
transformation preserves the value of the expression:
\begin{ttbox}
\input{Ifexpr/norm.ML}\end{ttbox}
The proof is canonical, provided we first show the following lemma (which
also helps to understand what \texttt{normif} does) and make it available
for simplification via \texttt{Addsimps}:
\begin{ttbox}
\input{Ifexpr/normif.ML}\end{ttbox}

But how can we be sure that \texttt{norm} really produces a normal form in
the above sense? We have to prove
\begin{ttbox}
\input{Ifexpr/normal_norm.ML}\end{ttbox}
where \texttt{normal} expresses that an If-expression is in normal form:
\begin{ttbox}
\input{Ifexpr/normal}\end{ttbox}
Of course, this requires a lemma about normality of \texttt{normif}
\begin{ttbox}
\input{Ifexpr/normal_normif.ML}\end{ttbox}
that has to be made available for simplification via \texttt{Addsimps}.

How does one come up with the required lemmas? Try to prove the main theorems
without them and study carefully what \texttt{Auto_tac} leaves unproved. This
has to provide the clue.
The necessity of universal quantification (\texttt{!t e}) in the two lemmas
is explained in \S\ref{sec:InductionHeuristics}

\begin{exercise}
  We strengthen the definition of a {\em normal\/} If-expression as follows:
  the first argument of all \texttt{IF}s must be a variable. Adapt the above
  development to this changed requirement. (Hint: you may need to formulate
  some of the goals as implications (\texttt{-->}) rather than equalities
  (\texttt{=}).)
\end{exercise}

\section{Some basic types}

\subsection{Natural numbers}

The type \ttindexbold{nat} of natural numbers is predefined and behaves like
\begin{ttbox}
datatype nat = 0 | Suc nat
\end{ttbox}
In particular, there are \texttt{case}-expressions, for example
\begin{ttbox}
case n of 0 => 0 | Suc m => m
\end{ttbox}
primitive recursion, for example
\begin{ttbox}
%
\begin{isabellebody}%
\def\isabellecontext{natsum}%
\isamarkupfalse%
%
\begin{isamarkuptext}%
\noindent
In particular, there are \isa{case}-expressions, for example
\begin{isabelle}%
\ \ \ \ \ case\ n\ of\ {\isadigit{0}}\ {\isasymRightarrow}\ {\isadigit{0}}\ {\isacharbar}\ Suc\ m\ {\isasymRightarrow}\ m%
\end{isabelle}
primitive recursion, for example%
\end{isamarkuptext}%
\isamarkuptrue%
\isacommand{consts}\ sum\ {\isacharcolon}{\isacharcolon}\ {\isachardoublequote}nat\ {\isasymRightarrow}\ nat{\isachardoublequote}\isanewline
\isamarkupfalse%
\isacommand{primrec}\ {\isachardoublequote}sum\ {\isadigit{0}}\ {\isacharequal}\ {\isadigit{0}}{\isachardoublequote}\isanewline
\ \ \ \ \ \ \ \ {\isachardoublequote}sum\ {\isacharparenleft}Suc\ n{\isacharparenright}\ {\isacharequal}\ Suc\ n\ {\isacharplus}\ sum\ n{\isachardoublequote}\isamarkupfalse%
%
\begin{isamarkuptext}%
\noindent
and induction, for example%
\end{isamarkuptext}%
\isamarkuptrue%
\isacommand{lemma}\ {\isachardoublequote}sum\ n\ {\isacharplus}\ sum\ n\ {\isacharequal}\ n{\isacharasterisk}{\isacharparenleft}Suc\ n{\isacharparenright}{\isachardoublequote}\isanewline
\isamarkupfalse%
\isacommand{apply}{\isacharparenleft}induct{\isacharunderscore}tac\ n{\isacharparenright}\isanewline
\isamarkupfalse%
\isacommand{apply}{\isacharparenleft}auto{\isacharparenright}\isanewline
\isamarkupfalse%
\isacommand{done}\isamarkupfalse%
%
\begin{isamarkuptext}%
\newcommand{\mystar}{*%
}
\index{arithmetic operations!for \protect\isa{nat}}%
The arithmetic operations \isadxboldpos{+}{$HOL2arithfun},
\isadxboldpos{-}{$HOL2arithfun}, \isadxboldpos{\mystar}{$HOL2arithfun},
\sdx{div}, \sdx{mod}, \cdx{min} and
\cdx{max} are predefined, as are the relations
\isadxboldpos{\isasymle}{$HOL2arithrel} and
\isadxboldpos{<}{$HOL2arithrel}. As usual, \isa{m\ {\isacharminus}\ n\ {\isacharequal}\ {\isadigit{0}}} if
\isa{m\ {\isacharless}\ n}. There is even a least number operation
\sdx{LEAST}\@.  For example, \isa{{\isacharparenleft}LEAST\ n{\isachardot}\ {\isadigit{0}}\ {\isacharless}\ n{\isacharparenright}\ {\isacharequal}\ Suc\ {\isadigit{0}}}.
\begin{warn}\index{overloading}
  The constants \cdx{0} and \cdx{1} and the operations
  \isadxboldpos{+}{$HOL2arithfun}, \isadxboldpos{-}{$HOL2arithfun},
  \isadxboldpos{\mystar}{$HOL2arithfun}, \cdx{min},
  \cdx{max}, \isadxboldpos{\isasymle}{$HOL2arithrel} and
  \isadxboldpos{<}{$HOL2arithrel} are overloaded: they are available
  not just for natural numbers but for other types as well.
  For example, given the goal \isa{x\ {\isacharplus}\ {\isadigit{0}}\ {\isacharequal}\ x}, there is nothing to indicate
  that you are talking about natural numbers. Hence Isabelle can only infer
  that \isa{x} is of some arbitrary type where \isa{{\isadigit{0}}} and \isa{{\isacharplus}} are
  declared. As a consequence, you will be unable to prove the
  goal. To alert you to such pitfalls, Isabelle flags numerals without a
  fixed type in its output: \isa{x\ {\isacharplus}\ {\isacharparenleft}{\isadigit{0}}{\isasymColon}{\isacharprime}a{\isacharparenright}\ {\isacharequal}\ x}. (In the absence of a numeral,
  it may take you some time to realize what has happened if \pgmenu{Show
  Types} is not set).  In this particular example, you need to include
  an explicit type constraint, for example \isa{x{\isacharplus}{\isadigit{0}}\ {\isacharequal}\ {\isacharparenleft}x{\isacharcolon}{\isacharcolon}nat{\isacharparenright}}. If there
  is enough contextual information this may not be necessary: \isa{Suc\ x\ {\isacharequal}\ x} automatically implies \isa{x{\isacharcolon}{\isacharcolon}nat} because \isa{Suc} is not
  overloaded.

  For details on overloading see \S\ref{sec:overloading}.
  Table~\ref{tab:overloading} in the appendix shows the most important
  overloaded operations.
\end{warn}
\begin{warn}
  The symbols \isadxboldpos{>}{$HOL2arithrel} and
  \isadxboldpos{\isasymge}{$HOL2arithrel} are merely syntax: \isa{x\ {\isachargreater}\ y}
  stands for \isa{y\ {\isacharless}\ x} and similary for \isa{{\isasymge}} and
  \isa{{\isasymle}}.
\end{warn}
\begin{warn}
  Constant \isa{{\isadigit{1}}{\isacharcolon}{\isacharcolon}nat} is defined to equal \isa{Suc\ {\isadigit{0}}}. This definition
  (see \S\ref{sec:ConstDefinitions}) is unfolded automatically by some
  tactics (like \isa{auto}, \isa{simp} and \isa{arith}) but not by
  others (especially the single step tactics in Chapter~\ref{chap:rules}).
  If you need the full set of numerals, see~\S\ref{sec:numerals}.
  \emph{Novices are advised to stick to \isa{{\isadigit{0}}} and \isa{Suc}.}
\end{warn}

Both \isa{auto} and \isa{simp}
(a method introduced below, \S\ref{sec:Simplification}) prove 
simple arithmetic goals automatically:%
\end{isamarkuptext}%
\isamarkuptrue%
\isacommand{lemma}\ {\isachardoublequote}{\isasymlbrakk}\ {\isasymnot}\ m\ {\isacharless}\ n{\isacharsemicolon}\ m\ {\isacharless}\ n\ {\isacharplus}\ {\isacharparenleft}{\isadigit{1}}{\isacharcolon}{\isacharcolon}nat{\isacharparenright}\ {\isasymrbrakk}\ {\isasymLongrightarrow}\ m\ {\isacharequal}\ n{\isachardoublequote}\isamarkupfalse%
\isamarkupfalse%
%
\begin{isamarkuptext}%
\noindent
For efficiency's sake, this built-in prover ignores quantified formulae,
many logical connectives, and all arithmetic operations apart from addition.
In consequence, \isa{auto} and \isa{simp} cannot prove this slightly more complex goal:%
\end{isamarkuptext}%
\isamarkuptrue%
\isacommand{lemma}\ {\isachardoublequote}m\ {\isasymnoteq}\ {\isacharparenleft}n{\isacharcolon}{\isacharcolon}nat{\isacharparenright}\ {\isasymLongrightarrow}\ m\ {\isacharless}\ n\ {\isasymor}\ n\ {\isacharless}\ m{\isachardoublequote}\isamarkupfalse%
\isamarkupfalse%
%
\begin{isamarkuptext}%
\noindent The method \methdx{arith} is more general.  It attempts to
prove the first subgoal provided it is a \textbf{linear arithmetic} formula.
Such formulas may involve the usual logical connectives (\isa{{\isasymnot}},
\isa{{\isasymand}}, \isa{{\isasymor}}, \isa{{\isasymlongrightarrow}}, \isa{{\isacharequal}},
\isa{{\isasymforall}}, \isa{{\isasymexists}}), the relations \isa{{\isacharequal}},
\isa{{\isasymle}} and \isa{{\isacharless}}, and the operations \isa{{\isacharplus}}, \isa{{\isacharminus}},
\isa{min} and \isa{max}.  For example,%
\end{isamarkuptext}%
\isamarkuptrue%
\isacommand{lemma}\ {\isachardoublequote}min\ i\ {\isacharparenleft}max\ j\ {\isacharparenleft}k{\isacharasterisk}k{\isacharparenright}{\isacharparenright}\ {\isacharequal}\ max\ {\isacharparenleft}min\ {\isacharparenleft}k{\isacharasterisk}k{\isacharparenright}\ i{\isacharparenright}\ {\isacharparenleft}min\ i\ {\isacharparenleft}j{\isacharcolon}{\isacharcolon}nat{\isacharparenright}{\isacharparenright}{\isachardoublequote}\isanewline
\isamarkupfalse%
\isacommand{apply}{\isacharparenleft}arith{\isacharparenright}\isamarkupfalse%
\isamarkupfalse%
%
\begin{isamarkuptext}%
\noindent
succeeds because \isa{k\ {\isacharasterisk}\ k} can be treated as atomic. In contrast,%
\end{isamarkuptext}%
\isamarkuptrue%
\isacommand{lemma}\ {\isachardoublequote}n{\isacharasterisk}n\ {\isacharequal}\ n\ {\isasymLongrightarrow}\ n{\isacharequal}{\isadigit{0}}\ {\isasymor}\ n{\isacharequal}{\isadigit{1}}{\isachardoublequote}\isamarkupfalse%
\isamarkupfalse%
%
\begin{isamarkuptext}%
\noindent
is not proved even by \isa{arith} because the proof relies 
on properties of multiplication. Only multiplication by numerals (which is
the same as iterated addition) is allowed.

\begin{warn} The running time of \isa{arith} is exponential in the number
  of occurrences of \ttindexboldpos{-}{$HOL2arithfun}, \cdx{min} and
  \cdx{max} because they are first eliminated by case distinctions.

If \isa{k} is a numeral, \sdx{div}~\isa{k}, \sdx{mod}~\isa{k} and
\isa{k}~\sdx{dvd} are also supported, where the former two are eliminated
by case distinctions, again blowing up the running time.

If the formula involves quantifiers, \isa{arith} may take
super-exponential time and space.
\end{warn}%
\end{isamarkuptext}%
\isamarkuptrue%
\isamarkupfalse%
\end{isabellebody}%
%%% Local Variables:
%%% mode: latex
%%% TeX-master: "root"
%%% End:
\end{ttbox}
and induction, for example
\begin{ttbox}
\input{Misc/NatSum.ML}\ttbreak
{\out sum n + sum n = n * Suc n}
{\out No subgoals!}
\end{ttbox}

The usual arithmetic operations \ttindexbold{+}, \ttindexbold{-},
\ttindexbold{*}, \ttindexbold{div} and \ttindexbold{mod} are predefined, as
are the relations \ttindexbold{<=} and \ttindexbold{<}. There is even a least
number operation \ttindexbold{LEAST}. For example, \texttt{(LEAST n.$\,$1 <
  n) = 2} (HOL does not prove this completely automatically).

\begin{warn}
  The operations \ttindexbold{+}, \ttindexbold{-}, \ttindexbold{*} are
  overloaded, i.e.\ they are available not just for natural numbers but at
  other types as well (see \S\ref{sec:TypeClasses}). For example, given
  the goal \texttt{x+y = y+x}, there is nothing to indicate that you are
  talking about natural numbers. Hence Isabelle can only infer that
  \texttt{x} and \texttt{y} are of some arbitrary type where \texttt{+} is
  declared. As a consequence, you will be unable to prove the goal (although
  it may take you some time to realize what has happened if
  \texttt{show_types} is not set).  In this particular example, you need to
  include an explicit type constraint, for example \texttt{x+y = y+(x::nat)}.
  If there is enough contextual information this may not be necessary:
  \texttt{x+0 = x} automatically implies \texttt{x::nat}.
\end{warn}


\subsection{Products}

HOL also has pairs: \texttt{($a@1$,$a@2$)} is of type \texttt{$\tau@1$ *
$\tau@2$} provided each $a@i$ is of type $\tau@i$. The components of a pair
are extracted by \texttt{fst} and \texttt{snd}:
\texttt{fst($x$,$y$) = $x$} and \texttt{snd($x$,$y$) = $y$}. Tuples
are simulated by pairs nested to the right: 
\texttt{($a@1$,$a@2$,$a@3$)} and \texttt{$\tau@1$ * $\tau@2$ * $\tau@3$}
stand for \texttt{($a@1$,($a@2$,$a@3$))} and \texttt{$\tau@1$ * ($\tau@2$ *
$\tau@3$)}. Therefore \texttt{fst(snd($a@1$,$a@2$,$a@3$)) = $a@2$}.

It is possible to use (nested) tuples as patterns in abstractions, for
example \texttt{\%(x,y,z).x+y+z} and \texttt{\%((x,y),z).x+y+z}.

In addition to explicit $\lambda$-abstractions, tuple patterns can be used in
most variable binding constructs. Typical examples are
\begin{ttbox}
let (x,y) = f z in (y,x)

case xs of [] => 0 | (x,y)\#zs => x+y
\end{ttbox}
Further important examples are quantifiers and sets.

\begin{warn}
Abstraction over pairs and tuples is merely a convenient shorthand for a more
complex internal representation.  Thus the internal and external form of a
term may differ, which can affect proofs. If you want to avoid this
complication, use \texttt{fst} and \texttt{snd}, i.e.\ write
\texttt{\%p.~fst p + snd p} instead of \texttt{\%(x,y).~x + y}.
See~\S\ref{} for theorem proving with tuple patterns.
\end{warn}


\section{Definitions}
\label{sec:Definitions}

A definition is simply an abbreviation, i.e.\ a new name for an existing
construction. In particular, definitions cannot be recursive. Isabelle offers
definitions on the level of types and terms. Those on the type level are
called type synonyms, those on the term level are called (constant)
definitions.


\subsection{Type synonyms}
\indexbold{type synonyms}

Type synonyms are similar to those found in ML. Their syntax is fairly self
explanatory:
\begin{ttbox}
\begin{isabelle}%
\isacommand{types}~number~~~~~~~=~nat\isanewline
~~~~~~gate~~~~~~~~~=~{"}bool~{\isasymRightarrow}~bool~{\isasymRightarrow}~bool{"}\isanewline
~~~~~~('a,'b)alist~=~{"}('a~*~'b)list{"}%
\begin{isamarkuptext}%
\noindent\indexbold{*types}%
Internally all synonyms are fully expanded.  As a consequence Isabelle's
output never contains synonyms.  Their main purpose is to improve the
readability of theory definitions.  Synonyms can be used just like any other
type:%
\end{isamarkuptext}%
\isacommand{consts}~nand~::~gate\isanewline
~~~~~~~exor~::~gate%
\begin{isamarkuptext}%
\subsection{Constant definitions}
\label{sec:ConstDefinitions}
\indexbold{definition}

The above constants \isa{nand} and \isa{exor} are non-recursive and can
therefore be defined directly by%
\end{isamarkuptext}%
\isacommand{defs}~nand\_def:~{"}nand~A~B~{\isasymequiv}~{\isasymnot}(A~{\isasymand}~B){"}\isanewline
~~~~~exor\_def:~{"}exor~A~B~{\isasymequiv}~A~{\isasymand}~{\isasymnot}B~{\isasymor}~{\isasymnot}A~{\isasymand}~B{"}%
\begin{isamarkuptext}%
\noindent%
where \isacommand{defs}\indexbold{*defs} is a keyword and \isa{nand_def} and
\isa{exor_def} are arbitrary user-supplied names.
The symbol \indexboldpos{\isasymequiv}{$IsaEq} is a special form of equality
that should only be used in constant definitions.
Declarations and definitions can also be merged%
\end{isamarkuptext}%
\isacommand{constdefs}~nor~::~gate\isanewline
~~~~~~~~~{"}nor~A~B~{\isasymequiv}~{\isasymnot}(A~{\isasymor}~B){"}\isanewline
~~~~~~~~~~exor2~::~gate\isanewline
~~~~~~~~~{"}exor2~A~B~{\isasymequiv}~(A~{\isasymor}~B)~{\isasymand}~({\isasymnot}A~{\isasymor}~{\isasymnot}B){"}%
\begin{isamarkuptext}%
\noindent\indexbold{*constdefs}%
in which case the default name of each definition is \isa{$f$_def}, where
$f$ is the name of the defined constant.%
\end{isamarkuptext}%
\end{isabelle}%
\end{ttbox}\indexbold{*types}
The synonym \texttt{alist} shows that in general the type on the right-hand
side needs to be enclosed in double quotation marks
(see the end of~\S\ref{sec:intro-theory}).

Internally all synonyms are fully expanded.  As a consequence Isabelle's
output never contains synonyms.  Their main purpose is to improve the
readability of theory definitions.  Synonyms can be used just like any other
type:
\begin{ttbox}
\input{Misc/consts}\end{ttbox}

\subsection{Constant definitions}
\label{sec:ConstDefinitions}

The above constants \texttt{nand} and \texttt{exor} are non-recursive and can
therefore be defined directly by
\begin{ttbox}
\input{Misc/defs}\end{ttbox}\indexbold{*defs}
where \texttt{defs} is a keyword and \texttt{nand_def} and \texttt{exor_def}
are arbitrary user-supplied names.
The symbol \texttt{==}\index{==>@{\tt==}|bold} is a special form of equality
that should only be used in constant definitions.
Declarations and definitions can also be merged
\begin{ttbox}
\input{Misc/constdefs}\end{ttbox}\indexbold{*constdefs}
in which case the default name of each definition is $f$\texttt{_def}, where
$f$ is the name of the defined constant.

Note that pattern-matching is not allowed, i.e.\ each definition must be of
the form $f\,x@1\,\dots\,x@n$\texttt{~==~}$t$.

Section~\S\ref{sec:Simplification} explains how definitions are used
in proofs.

\begin{warn}
A common mistake when writing definitions is to introduce extra free variables
on the right-hand side as in the following fictitious definition:
\begin{ttbox}
defs  prime_def "prime(p) == (m divides p) --> (m=1 | m=p)"
\end{ttbox}
Isabelle rejects this `definition' because of the extra {\tt m} on the
right-hand side, which would introduce an inconsistency.  What you should have
written is
\begin{ttbox}
defs  prime_def "prime(p) == !m. (m divides p) --> (m=1 | m=p)"
\end{ttbox}
\end{warn}




\chapter{More Functional Programming}

The purpose of this chapter is to deepen the reader's understanding of the
concepts encountered so far and to introduce an advanced method for defining
recursive functions. The first two sections give a structured presentation of
theorem proving by simplification (\S\ref{sec:Simplification}) and
discuss important heuristics for induction (\S\ref{sec:InductionHeuristics}). They
can be skipped by readers less interested in proofs. They are followed by a
case study, a compiler for expressions (\S\ref{sec:ExprCompiler}).
Finally we present a very general method for defining recursive functions
that goes well beyond what \texttt{primrec} allows (\S\ref{sec:recdef}).


\section{Simplification}
\label{sec:Simplification}

So far we have proved our theorems by \texttt{Auto_tac}, which
`simplifies' all subgoals. In fact, \texttt{Auto_tac} can do much more than
that, except that it did not need to so far. However, when you go beyond toy
examples, you need to understand the ingredients of \texttt{Auto_tac}.
This section covers the tactic that \texttt{Auto_tac} always applies first,
namely simplification.

Simplification is one of the central theorem proving tools in Isabelle and
many other systems. The tool itself is called the \bfindex{simplifier}. The
purpose of this section is twofold: to introduce the many features of the
simplifier (\S\ref{sec:SimpFeatures}) and to explain a little bit how the
simplifier works (\S\ref{sec:SimpHow}).  Anybody intending to use HOL should
read \S\ref{sec:SimpFeatures}, and the serious student should read
\S\ref{sec:SimpHow} as well in order to understand what happened in case
things do not simplify as expected.


\subsection{Using the simplifier}
\label{sec:SimpFeatures}

In its most basic form, simplification means repeated application of
equations from left to right. For example, taking the rules for \texttt{\at}
and applying them to the term \texttt{[0,1] \at\ []} results in a sequence of
simplification steps:
\begin{ttbox}\makeatother
(0#1#[]) @ []  \(\leadsto\)  0#((1#[]) @ [])  \(\leadsto\)  0#(1#([] @ []))  \(\leadsto\)  0#1#[]
\end{ttbox}
This is also known as {\em term rewriting} and the equations are referred
to as {\em rewrite rules}. This is more honest than `simplification' because
the terms do not necessarily become simpler in the process.

\subsubsection{Simpsets}

To facilitate simplification, each theory has an associated set of
simplification rules, known as a \bfindex{simpset}. Within a theory,
proofs by simplification refer to the associated simpset by default.
The simpset of a theory is built up as follows: starting with the union of
the simpsets of the parent theories, each occurrence of a \texttt{datatype}
or \texttt{primrec} construct augments the simpset. Explicit definitions are
not added automatically. Users can add new theorems via \texttt{Addsimps} and
delete them again later by \texttt{Delsimps}.

You may augment a simpset not just by equations but by pretty much any
theorem. The simplifier will try to make sense of it.  For example, a theorem
\verb$~$$P$ is automatically turned into \texttt{$P$ = False}. The details are
explained in \S\ref{sec:SimpHow}.

As a rule of thumb, rewrite rules that really simplify a term (like
\texttt{xs \at\ [] = xs} and \texttt{rev(rev xs) = xs}) should be added to the
current simpset right after they have been proved.  Those of a more specific
nature (e.g.\ distributivity laws, which alter the structure of terms
considerably) should only be added for specific proofs and deleted again
afterwards.  Conversely, it may also happen that a generally useful rule
needs to be removed for a certain proof and is added again afterwards.  The
need of frequent temporary additions or deletions may indicate a badly
designed simpset.
\begin{warn}
  Simplification may not terminate, for example if both $f(x) = g(x)$ and
  $g(x) = f(x)$ are in the simpset. It is the user's responsibility not to
  include rules that can lead to nontermination, either on their own or in
  combination with other rules.
\end{warn}

\subsubsection{Simplification tactics}

There are four main simplification tactics:
\begin{ttdescription}
\item[\ttindexbold{Simp_tac} $i$] simplifies the conclusion of subgoal~$i$
  using the theory's simpset.  It may solve the subgoal completely if it has
  become trivial. For example:
\begin{ttbox}\makeatother
{\out 1. [] @ [] = []}
by(Simp_tac 1);
{\out No subgoals!}
\end{ttbox}

\item[\ttindexbold{Asm_simp_tac}]
  is like \verb$Simp_tac$, but extracts additional rewrite rules from
  the assumptions of the subgoal. For example, it solves
\begin{ttbox}\makeatother
{\out 1. xs = [] ==> xs @ xs = xs}
\end{ttbox}
which \texttt{Simp_tac} does not do.
  
\item[\ttindexbold{Full_simp_tac}] is like \verb$Simp_tac$, but also
  simplifies the assumptions (without using the assumptions to
  simplify each other or the actual goal).

\item[\ttindexbold{Asm_full_simp_tac}] is like \verb$Asm_simp_tac$,
  but also simplifies the assumptions. In particular, assumptions can
  simplify each other. For example:
\begin{ttbox}\makeatother
\out{ 1. [| xs @ zs = ys @ xs; [] @ xs = [] @ [] |] ==> ys = zs}
by(Asm_full_simp_tac 1);
{\out No subgoals!}
\end{ttbox}
The second assumption simplifies to \texttt{xs = []}, which in turn
simplifies the first assumption to \texttt{zs = ys}, thus reducing the
conclusion to \texttt{ys = ys} and hence to \texttt{True}.
(See also the paragraph on tracing below.)
\end{ttdescription}
\texttt{Asm_full_simp_tac} is the most powerful of this quartet of
tactics. In fact, \texttt{Auto_tac} starts by applying
\texttt{Asm_full_simp_tac} to all subgoals. The only reason for the existence
of the other three tactics is that sometimes one wants to limit the amount of
simplification, for example to avoid nontermination:
\begin{ttbox}\makeatother
{\out  1. ! x. f x = g (f (g x)) ==> f [] = f [] @ []}
\end{ttbox}
is solved by \texttt{Simp_tac}, but \texttt{Asm_simp_tac} and
\texttt{Asm_full_simp_tac} loop because the rewrite rule \texttt{f x = g(f(g
x))} extracted from the assumption does not terminate.  Isabelle notices
certain simple forms of nontermination, but not this one.
 
\subsubsection{Modifying simpsets locally}

If a certain theorem is merely needed in one proof by simplification, the
pattern
\begin{ttbox}
Addsimps [\(rare_theorem\)];
by(Simp_tac 1);
Delsimps [\(rare_theorem\)];
\end{ttbox}
is awkward. Therefore there are lower-case versions of the simplification
tactics (\ttindexbold{simp_tac}, \ttindexbold{asm_simp_tac},
\ttindexbold{full_simp_tac}, \ttindexbold{asm_full_simp_tac}) and of the
simpset modifiers (\ttindexbold{addsimps}, \ttindexbold{delsimps})
that do not access or modify the implicit simpset but explicitly take a
simpset as an argument. For example, the above three lines become
\begin{ttbox}
by(simp_tac (simpset() addsimps [\(rare_theorem\)]) 1);
\end{ttbox}
where the result of the function call \texttt{simpset()} is the simpset of
the current theory and \texttt{addsimps} is an infix function. The implicit
simpset is read once but not modified.
This is far preferable to pairs of \texttt{Addsimps} and \texttt{Delsimps}.
Local modifications can be stacked as in
\begin{ttbox}
by(simp_tac (simpset() addsimps [\(rare_theorem\)] delsimps [\(some_thm\)]) 1);
\end{ttbox}

\subsubsection{Rewriting with definitions}

Constant definitions (\S\ref{sec:ConstDefinitions}) are not automatically
included in the simpset of a theory. Hence such definitions are not expanded
automatically either, just as it should be: definitions are introduced for
the purpose of abbreviating complex concepts. Of course we need to expand the
definitions initially to derive enough lemmas that characterize the concept
sufficiently for us to forget the original definition completely. For
example, given the theory
\begin{ttbox}
\input{Misc/Exor.thy}\end{ttbox}
we may want to prove \verb$exor A (~A)$. Instead of \texttt{Goal} we use
\begin{ttbox}
\input{Misc/exorgoal.ML}\end{ttbox}
which tells Isabelle to expand the definition of \texttt{exor}---the first
argument of \texttt{Goalw} can be a list of definitions---in the initial goal:
\begin{ttbox}
{\out exor A (~ A)}
{\out  1. A & ~ ~ A | ~ A & ~ A}
\end{ttbox}
In this simple example, the goal is proved by \texttt{Simp_tac}.
Of course the resulting theorem is insufficient to characterize \texttt{exor}
completely.

In case we want to expand a definition in the middle of a proof, we can
simply add the definition locally to the simpset:
\begin{ttbox}
\input{Misc/exorproof.ML}\end{ttbox}
You should normally not add the definition permanently to the simpset
using \texttt{Addsimps} because this defeats the whole purpose of an
abbreviation.

\begin{warn}
If you have defined $f\,x\,y$\texttt{~==~}$t$ then you can only expand
occurrences of $f$ with at least two arguments. Thus it is safer to define
$f$\texttt{~==~\%$x\,y$.}$\;t$.
\end{warn}

\subsubsection{Simplifying \texttt{let}-expressions}

Proving a goal containing \ttindex{let}-expressions invariably requires the
\texttt{let}-constructs to be expanded at some point. Since
\texttt{let}-\texttt{in} is just syntactic sugar for a defined constant
(called \texttt{Let}), expanding \texttt{let}-constructs means rewriting with
\texttt{Let_def}:
%context List.thy;
%Goal "(let xs = [] in xs@xs) = ys";
\begin{ttbox}\makeatother
{\out  1. (let xs = [] in xs @ xs) = ys}
by(simp_tac (simpset() addsimps [Let_def]) 1);
{\out  1. [] = ys}
\end{ttbox}
If, in a particular context, there is no danger of a combinatorial explosion
of nested \texttt{let}s one could even add \texttt{Let_def} permanently via
\texttt{Addsimps}.

\subsubsection{Conditional equations}

So far all examples of rewrite rules were equations. The simplifier also
accepts {\em conditional\/} equations, for example
\begin{ttbox}
xs ~= []  ==>  hd xs # tl xs = xs \hfill \((*)\)
\end{ttbox}
(which is proved by \texttt{exhaust_tac} on \texttt{xs} followed by
\texttt{Asm_full_simp_tac} twice). Assuming that this theorem together with
%\begin{ttbox}\makeatother
\texttt{(rev xs = []) = (xs = [])}
%\end{ttbox}
are part of the simpset, the subgoal
\begin{ttbox}\makeatother
{\out 1. xs ~= [] ==> hd(rev xs) # tl(rev xs) = rev xs}
\end{ttbox}
is proved by simplification:
the conditional equation $(*)$ above
can simplify \texttt{hd(rev~xs)~\#~tl(rev~xs)} to \texttt{rev xs}
because the corresponding precondition \verb$rev xs ~= []$ simplifies to
\verb$xs ~= []$, which is exactly the local assumption of the subgoal.


\subsubsection{Automatic case splits}

Goals containing \ttindex{if}-expressions are usually proved by case
distinction on the condition of the \texttt{if}. For example the goal
\begin{ttbox}
{\out 1. ! xs. if xs = [] then rev xs = [] else rev xs ~= []}
\end{ttbox}
can be split into
\begin{ttbox}
{\out 1. ! xs. (xs = [] --> rev xs = []) \& (xs ~= [] --> rev xs ~= [])}
\end{ttbox}
by typing
\begin{ttbox}
\input{Misc/splitif.ML}\end{ttbox}
Because this is almost always the right proof strategy, the simplifier
performs case-splitting on \texttt{if}s automatically. Try \texttt{Simp_tac}
on the initial goal above.

This splitting idea generalizes from \texttt{if} to \ttindex{case}:
\begin{ttbox}\makeatother
{\out 1. (case xs of [] => zs | y#ys => y#(ys@zs)) = xs@zs}
\end{ttbox}
becomes
\begin{ttbox}\makeatother
{\out 1. (xs = [] --> zs = xs @ zs) &}
{\out    (! a list. xs = a # list --> a # list @ zs = xs @ zs)}
\end{ttbox}
by typing
\begin{ttbox}
\input{Misc/splitlist.ML}\end{ttbox}
In contrast to \texttt{if}-expressions, the simplifier does not split
\texttt{case}-expressions by default because this can lead to nontermination
in case of recursive datatypes.
Nevertheless the simplifier can be instructed to perform \texttt{case}-splits
by adding the appropriate rule to the simpset:
\begin{ttbox}
by(simp_tac (simpset() addsplits [split_list_case]) 1);
\end{ttbox}\indexbold{*addsplits}
solves the initial goal outright, which \texttt{Simp_tac} alone will not do.

In general, every datatype $t$ comes with a rule
\texttt{$t$.split} that can be added to the simpset either
locally via \texttt{addsplits} (see above), or permanently via
\begin{ttbox}
Addsplits [\(t\).split];
\end{ttbox}\indexbold{*Addsplits}
Split-rules can be removed globally via \ttindexbold{Delsplits} and locally
via \ttindexbold{delsplits} as, for example, in
\begin{ttbox}
by(simp_tac (simpset() addsimps [\(\dots\)] delsplits [split_if]) 1);
\end{ttbox}


\subsubsection{Permutative rewrite rules}

A rewrite rule is {\bf permutative} if the left-hand side and right-hand side
are the same up to renaming of variables.  The most common permutative rule
is commutativity: $x+y = y+x$.  Another example is $(x-y)-z = (x-z)-y$.  Such
rules are problematic because once they apply, they can be used forever.
The simplifier is aware of this danger and treats permutative rules
separately. For details see~\cite{Isa-Ref-Man}.

\subsubsection{Tracing}
\indexbold{tracing the simplifier}

Using the simplifier effectively may take a bit of experimentation.  Set the
\verb$trace_simp$ flag to get a better idea of what is going on:
\begin{ttbox}\makeatother
{\out  1. rev [x] = []}
\ttbreak
set trace_simp;
by(Simp_tac 1);
\ttbreak\makeatother
{\out Applying instance of rewrite rule:}
{\out rev (?x # ?xs) == rev ?xs @ [?x]}
{\out Rewriting:}
{\out rev [x] == rev [] @ [x]}
\ttbreak
{\out Applying instance of rewrite rule:}
{\out rev [] == []}
{\out Rewriting:}
{\out rev [] == []}
\ttbreak\makeatother
{\out Applying instance of rewrite rule:}
{\out [] @ ?y == ?y}
{\out Rewriting:}
{\out [] @ [x] == [x]}
\ttbreak
{\out Applying instance of rewrite rule:}
{\out ?x # ?t = ?t == False}
{\out Rewriting:}
{\out [x] = [] == False}
\ttbreak
{\out Level 1}
{\out rev [x] = []}
{\out  1. False}
\end{ttbox}
In more complicated cases, the trace can be enormous, especially since
invocations of the simplifier are often nested (e.g.\ when solving conditions
of rewrite rules).

\subsection{How it works}
\label{sec:SimpHow}

\subsubsection{Higher-order patterns}

\subsubsection{Local assumptions}

\subsubsection{The preprocessor}

\section{Induction heuristics}
\label{sec:InductionHeuristics}

The purpose of this section is to illustrate some simple heuristics for
inductive proofs. The first one we have already mentioned in our initial
example:
\begin{quote}
{\em 1. Theorems about recursive functions are proved by induction.}
\end{quote}
In case the function has more than one argument
\begin{quote}
{\em 2. Do induction on argument number $i$ if the function is defined by
recursion in argument number $i$.}
\end{quote}
When we look at the proof of
\begin{ttbox}\makeatother
(xs @ ys) @ zs = xs @ (ys @ zs)
\end{ttbox}
in \S\ref{sec:intro-proof} we find (a) \texttt{\at} is recursive in the first
argument, (b) \texttt{xs} occurs only as the first argument of \texttt{\at},
and (c) both \texttt{ys} and \texttt{zs} occur at least once as the second
argument of \texttt{\at}. Hence it is natural to perform induction on
\texttt{xs}.

The key heuristic, and the main point of this section, is to
generalize the goal before induction. The reason is simple: if the goal is
too specific, the induction hypothesis is too weak to allow the induction
step to go through. Let us now illustrate the idea with an example.

We define a tail-recursive version of list-reversal,
i.e.\ one that can be compiled into a loop:
\begin{ttbox}
%
\begin{isabellebody}%
\def\isabellecontext{Itrev}%
\isamarkupfalse%
%
\isamarkupsection{Induction Heuristics%
}
\isamarkuptrue%
%
\begin{isamarkuptext}%
\label{sec:InductionHeuristics}
\index{induction heuristics|(}%
The purpose of this section is to illustrate some simple heuristics for
inductive proofs. The first one we have already mentioned in our initial
example:
\begin{quote}
\emph{Theorems about recursive functions are proved by induction.}
\end{quote}
In case the function has more than one argument
\begin{quote}
\emph{Do induction on argument number $i$ if the function is defined by
recursion in argument number $i$.}
\end{quote}
When we look at the proof of \isa{{\isacharparenleft}xs{\isacharat}ys{\isacharparenright}\ {\isacharat}\ zs\ {\isacharequal}\ xs\ {\isacharat}\ {\isacharparenleft}ys{\isacharat}zs{\isacharparenright}}
in \S\ref{sec:intro-proof} we find
\begin{itemize}
\item \isa{{\isacharat}} is recursive in
the first argument
\item \isa{xs}  occurs only as the first argument of
\isa{{\isacharat}}
\item both \isa{ys} and \isa{zs} occur at least once as
the second argument of \isa{{\isacharat}}
\end{itemize}
Hence it is natural to perform induction on~\isa{xs}.

The key heuristic, and the main point of this section, is to
\emph{generalize the goal before induction}.
The reason is simple: if the goal is
too specific, the induction hypothesis is too weak to allow the induction
step to go through. Let us illustrate the idea with an example.

Function \cdx{rev} has quadratic worst-case running time
because it calls function \isa{{\isacharat}} for each element of the list and
\isa{{\isacharat}} is linear in its first argument.  A linear time version of
\isa{rev} reqires an extra argument where the result is accumulated
gradually, using only~\isa{{\isacharhash}}:%
\end{isamarkuptext}%
\isamarkuptrue%
\isacommand{consts}\ itrev\ {\isacharcolon}{\isacharcolon}\ {\isachardoublequote}{\isacharprime}a\ list\ {\isasymRightarrow}\ {\isacharprime}a\ list\ {\isasymRightarrow}\ {\isacharprime}a\ list{\isachardoublequote}\isanewline
\isamarkupfalse%
\isacommand{primrec}\isanewline
{\isachardoublequote}itrev\ {\isacharbrackleft}{\isacharbrackright}\ \ \ \ \ ys\ {\isacharequal}\ ys{\isachardoublequote}\isanewline
{\isachardoublequote}itrev\ {\isacharparenleft}x{\isacharhash}xs{\isacharparenright}\ ys\ {\isacharequal}\ itrev\ xs\ {\isacharparenleft}x{\isacharhash}ys{\isacharparenright}{\isachardoublequote}\isamarkupfalse%
%
\begin{isamarkuptext}%
\noindent
The behaviour of \cdx{itrev} is simple: it reverses
its first argument by stacking its elements onto the second argument,
and returning that second argument when the first one becomes
empty. Note that \isa{itrev} is tail-recursive: it can be
compiled into a loop.

Naturally, we would like to show that \isa{itrev} does indeed reverse
its first argument provided the second one is empty:%
\end{isamarkuptext}%
\isamarkuptrue%
\isacommand{lemma}\ {\isachardoublequote}itrev\ xs\ {\isacharbrackleft}{\isacharbrackright}\ {\isacharequal}\ rev\ xs{\isachardoublequote}\isamarkupfalse%
%
\begin{isamarkuptxt}%
\noindent
There is no choice as to the induction variable, and we immediately simplify:%
\end{isamarkuptxt}%
\isamarkuptrue%
\isacommand{apply}{\isacharparenleft}induct{\isacharunderscore}tac\ xs{\isacharcomma}\ simp{\isacharunderscore}all{\isacharparenright}\isamarkupfalse%
%
\begin{isamarkuptxt}%
\noindent
Unfortunately, this attempt does not prove
the induction step:
\begin{isabelle}%
\ {\isadigit{1}}{\isachardot}\ {\isasymAnd}a\ list{\isachardot}\isanewline
\isaindent{\ {\isadigit{1}}{\isachardot}\ \ \ \ }itrev\ list\ {\isacharbrackleft}{\isacharbrackright}\ {\isacharequal}\ rev\ list\ {\isasymLongrightarrow}\ itrev\ list\ {\isacharbrackleft}a{\isacharbrackright}\ {\isacharequal}\ rev\ list\ {\isacharat}\ {\isacharbrackleft}a{\isacharbrackright}%
\end{isabelle}
The induction hypothesis is too weak.  The fixed
argument,~\isa{{\isacharbrackleft}{\isacharbrackright}}, prevents it from rewriting the conclusion.  
This example suggests a heuristic:
\begin{quote}\index{generalizing induction formulae}%
\emph{Generalize goals for induction by replacing constants by variables.}
\end{quote}
Of course one cannot do this na\"{\i}vely: \isa{itrev\ xs\ ys\ {\isacharequal}\ rev\ xs} is
just not true.  The correct generalization is%
\end{isamarkuptxt}%
\isamarkuptrue%
\isamarkupfalse%
\isacommand{lemma}\ {\isachardoublequote}itrev\ xs\ ys\ {\isacharequal}\ rev\ xs\ {\isacharat}\ ys{\isachardoublequote}\isamarkupfalse%
\isamarkupfalse%
%
\begin{isamarkuptxt}%
\noindent
If \isa{ys} is replaced by \isa{{\isacharbrackleft}{\isacharbrackright}}, the right-hand side simplifies to
\isa{rev\ xs}, as required.

In this instance it was easy to guess the right generalization.
Other situations can require a good deal of creativity.  

Although we now have two variables, only \isa{xs} is suitable for
induction, and we repeat our proof attempt. Unfortunately, we are still
not there:
\begin{isabelle}%
\ {\isadigit{1}}{\isachardot}\ {\isasymAnd}a\ list{\isachardot}\isanewline
\isaindent{\ {\isadigit{1}}{\isachardot}\ \ \ \ }itrev\ list\ ys\ {\isacharequal}\ rev\ list\ {\isacharat}\ ys\ {\isasymLongrightarrow}\isanewline
\isaindent{\ {\isadigit{1}}{\isachardot}\ \ \ \ }itrev\ list\ {\isacharparenleft}a\ {\isacharhash}\ ys{\isacharparenright}\ {\isacharequal}\ rev\ list\ {\isacharat}\ a\ {\isacharhash}\ ys%
\end{isabelle}
The induction hypothesis is still too weak, but this time it takes no
intuition to generalize: the problem is that \isa{ys} is fixed throughout
the subgoal, but the induction hypothesis needs to be applied with
\isa{a\ {\isacharhash}\ ys} instead of \isa{ys}. Hence we prove the theorem
for all \isa{ys} instead of a fixed one:%
\end{isamarkuptxt}%
\isamarkuptrue%
\isamarkupfalse%
\isacommand{lemma}\ {\isachardoublequote}{\isasymforall}ys{\isachardot}\ itrev\ xs\ ys\ {\isacharequal}\ rev\ xs\ {\isacharat}\ ys{\isachardoublequote}\isamarkupfalse%
\isamarkupfalse%
%
\begin{isamarkuptext}%
\noindent
This time induction on \isa{xs} followed by simplification succeeds. This
leads to another heuristic for generalization:
\begin{quote}
\emph{Generalize goals for induction by universally quantifying all free
variables {\em(except the induction variable itself!)}.}
\end{quote}
This prevents trivial failures like the one above and does not affect the
validity of the goal.  However, this heuristic should not be applied blindly.
It is not always required, and the additional quantifiers can complicate
matters in some cases, The variables that should be quantified are typically
those that change in recursive calls.

A final point worth mentioning is the orientation of the equation we just
proved: the more complex notion (\isa{itrev}) is on the left-hand
side, the simpler one (\isa{rev}) on the right-hand side. This constitutes
another, albeit weak heuristic that is not restricted to induction:
\begin{quote}
  \emph{The right-hand side of an equation should (in some sense) be simpler
    than the left-hand side.}
\end{quote}
This heuristic is tricky to apply because it is not obvious that
\isa{rev\ xs\ {\isacharat}\ ys} is simpler than \isa{itrev\ xs\ ys}. But see what
happens if you try to prove \isa{rev\ xs\ {\isacharat}\ ys\ {\isacharequal}\ itrev\ xs\ ys}!

If you have tried these heuristics and still find your
induction does not go through, and no obvious lemma suggests itself, you may
need to generalize your proposition even further. This requires insight into
the problem at hand and is beyond simple rules of thumb.  
Additionally, you can read \S\ref{sec:advanced-ind}
to learn about some advanced techniques for inductive proofs.%
\index{induction heuristics|)}%
\end{isamarkuptext}%
\isamarkuptrue%
\isamarkupfalse%
\end{isabellebody}%
%%% Local Variables:
%%% mode: latex
%%% TeX-master: "root"
%%% End:
\end{ttbox}
The behaviour of \texttt{itrev} is simple: it reverses its first argument by
stacking its elements onto the second argument, and returning that second
argument when the first one becomes empty.
We need to show that \texttt{itrev} does indeed reverse its first argument
provided the second one is empty:
\begin{ttbox}
\input{Misc/itrev1.ML}\end{ttbox}
There is no choice as to the induction variable, and we immediately simplify:
\begin{ttbox}
\input{Misc/induct_auto.ML}\ttbreak\makeatother
{\out1. !!a list. itrev list [] = rev list\(\;\)==> itrev list [a] = rev list @ [a]}
\end{ttbox}
Just as predicted above, the overall goal, and hence the induction
hypothesis, is too weak to solve the induction step because of the fixed
\texttt{[]}. The corresponding heuristic:
\begin{quote}
{\em 3. Generalize goals for induction by replacing constants by variables.}
\end{quote}
Of course one cannot do this na\"{\i}vely: \texttt{itrev xs ys = rev xs} is
just not true --- the correct generalization is
\begin{ttbox}\makeatother
\input{Misc/itrev2.ML}\end{ttbox}
If \texttt{ys} is replaced by \texttt{[]}, the right-hand side simplifies to
\texttt{rev xs}, just as required.

In this particular instance it is easy to guess the right generalization,
but in more complex situations a good deal of creativity is needed. This is
the main source of complications in inductive proofs.

Although we now have two variables, only \texttt{xs} is suitable for
induction, and we repeat our above proof attempt. Unfortunately, we are still
not there:
\begin{ttbox}\makeatother
{\out 1. !!a list.}
{\out       itrev list ys = rev list @ ys}
{\out       ==> itrev list (a # ys) = rev list @ a # ys}
\end{ttbox}
The induction hypothesis is still too weak, but this time it takes no
intuition to generalize: the problem is that \texttt{ys} is fixed throughout
the subgoal, but the induction hypothesis needs to be applied with
\texttt{a \# ys} instead of \texttt{ys}. Hence we prove the theorem
for all \texttt{ys} instead of a fixed one:
\begin{ttbox}\makeatother
\input{Misc/itrev3.ML}\end{ttbox}
This time induction on \texttt{xs} followed by simplification succeeds. This
leads to another heuristic for generalization:
\begin{quote}
{\em 4. Generalize goals for induction by universally quantifying all free
variables {\em(except the induction variable itself!)}.}
\end{quote}
This prevents trivial failures like the above and does not change the
provability of the goal. Because it is not always required, and may even
complicate matters in some cases, this heuristic is often not
applied blindly.

A final point worth mentioning is the orientation of the equation we just
proved: the more complex notion (\texttt{itrev}) is on the left-hand
side, the simpler \texttt{rev} on the right-hand side. This constitutes
another, albeit weak heuristic that is not restricted to induction:
\begin{quote}
  {\em 5. The right-hand side of an equation should (in some sense) be
    simpler than the left-hand side.}
\end{quote}
The heuristic is tricky to apply because it is not obvious that
\texttt{rev xs \at\ ys} is simpler than \texttt{itrev xs ys}. But see what
happens if you try to prove the symmetric equation!


\section{Case study: compiling expressions}
\label{sec:ExprCompiler}

The task is to develop a compiler from a generic type of expressions (built
up from variables, constants and binary operations) to a stack machine.  This
generic type of expressions is a generalization of the boolean expressions in
\S\ref{sec:boolex}.  This time we do not commit ourselves to a particular
type of variables or values but make them type parameters.  Neither is there
a fixed set of binary operations: instead the expression contains the
appropriate function itself.
\begin{ttbox}
\input{CodeGen/expr}\end{ttbox}
The three constructors represent constants, variables and the combination of
two subexpressions with a binary operation.

The value of an expression w.r.t.\ an environment that maps variables to
values is easily defined:
\begin{ttbox}
\input{CodeGen/value}\end{ttbox}

The stack machine has three instructions: load a constant value onto the
stack, load the contents of a certain address onto the stack, and apply a
binary operation to the two topmost elements of the stack, replacing them by
the result. As for \texttt{expr}, addresses and values are type parameters:
\begin{ttbox}
\input{CodeGen/instr}\end{ttbox}

The execution of the stack machine is modelled by a function \texttt{exec}
that takes a store (modelled as a function from addresses to values, just
like the environment for evaluating expressions), a stack (modelled as a
list) of values and a list of instructions and returns the stack at the end
of the execution --- the store remains unchanged:
\begin{ttbox}
\input{CodeGen/exec}\end{ttbox}
Recall that \texttt{hd} and \texttt{tl}
return the first element and the remainder of a list.

Because all functions are total, \texttt{hd} is defined even for the empty
list, although we do not know what the result is. Thus our model of the
machine always terminates properly, although the above definition does not
tell us much about the result in situations where \texttt{Apply} was executed
with fewer than two elements on the stack.

The compiler is a function from expressions to a list of instructions. Its
definition is pretty much obvious:
\begin{ttbox}\makeatother
\input{CodeGen/comp}\end{ttbox}

Now we have to prove the correctness of the compiler, i.e.\ that the
execution of a compiled expression results in the value of the expression:
\begin{ttbox}
exec s [] (comp e) = [value s e]
\end{ttbox}
This is generalized to
\begin{ttbox}
\input{CodeGen/goal2.ML}\end{ttbox}
and proved by induction on \texttt{e} followed by simplification, once we
have the following lemma about executing the concatenation of two instruction
sequences:
\begin{ttbox}\makeatother
\input{CodeGen/goal2.ML}\end{ttbox}
This requires induction on \texttt{xs} and ordinary simplification for the
base cases. In the induction step, simplification leaves us with a formula
that contains two \texttt{case}-expressions over instructions. Thus we add
automatic case splitting as well, which finishes the proof:
\begin{ttbox}
\input{CodeGen/simpsplit.ML}\end{ttbox}

We could now go back and prove \texttt{exec s [] (comp e) = [value s e]}
merely by simplification with the generalized version we just proved.
However, this is unnecessary because the generalized version fully subsumes
its instance.

\section{Total recursive functions}
\label{sec:recdef}
\index{*recdef|(}


Although many total functions have a natural primitive recursive definition,
this is not always the case. Arbitrary total recursive functions can be
defined by means of \texttt{recdef}: you can use full pattern-matching,
recursion need not involve datatypes, and termination is proved by showing
that each recursive call makes the argument smaller in a suitable (user
supplied) sense.

\subsection{Defining recursive functions}

Here is a simple example, the Fibonacci function:
\begin{ttbox}
consts fib  :: nat => nat
recdef fib "measure(\%n. n)"
    "fib 0 = 0"
    "fib 1 = 1"
    "fib (Suc(Suc x)) = fib x + fib (Suc x)"
\end{ttbox}
The definition of \texttt{fib} is accompanied by a \bfindex{measure function}
\texttt{\%n.$\;$n} that maps the argument of \texttt{fib} to a natural
number. The requirement is that in each equation the measure of the argument
on the left-hand side is strictly greater than the measure of the argument of
each recursive call. In the case of \texttt{fib} this is obviously true
because the measure function is the identity and \texttt{Suc(Suc~x)} is
strictly greater than both \texttt{x} and \texttt{Suc~x}.

Slightly more interesting is the insertion of a fixed element
between any two elements of a list:
\begin{ttbox}
consts sep :: "'a * 'a list => 'a list"
recdef sep "measure (\%(a,xs). length xs)"
    "sep(a, [])     = []"
    "sep(a, [x])    = [x]"
    "sep(a, x#y#zs) = x # a # sep(a,y#zs)"
\end{ttbox}
This time the measure is the length of the list, which decreases with the
recursive call; the first component of the argument tuple is irrelevant.

Pattern matching need not be exhaustive:
\begin{ttbox}
consts last :: 'a list => bool
recdef last "measure (\%xs. length xs)"
    "last [x]      = x"
    "last (x#y#zs) = last (y#zs)"
\end{ttbox}

Overlapping patterns are disambiguated by taking the order of equations into
account, just as in functional programming:
\begin{ttbox}
recdef sep "measure (\%(a,xs). length xs)"
    "sep(a, x#y#zs) = x # a # sep(a,y#zs)"
    "sep(a, xs)     = xs"
\end{ttbox}
This defines exactly the same function \texttt{sep} as further above.

\begin{warn}
Currently \texttt{recdef} only accepts functions with a single argument,
possibly of tuple type.
\end{warn}

When loading a theory containing a \texttt{recdef} of a function $f$,
Isabelle proves the recursion equations and stores the result as a list of
theorems $f$.\texttt{rules}. It can be viewed by typing
\begin{ttbox}
prths \(f\).rules;
\end{ttbox}
All of the above examples are simple enough that Isabelle can determine
automatically that the measure of the argument goes down in each recursive
call. In that case $f$.\texttt{rules} contains precisely the defining
equations.

In general, Isabelle may not be able to prove all termination conditions
automatically. For example, termination of
\begin{ttbox}
consts gcd :: "nat*nat => nat"
recdef gcd "measure ((\%(m,n).n))"
    "gcd (m, n) = (if n=0 then m else gcd(n, m mod n))"
\end{ttbox}
relies on the lemma \texttt{mod_less_divisor}
\begin{ttbox}
0 < n ==> m mod n < n
\end{ttbox}
that is not part of the default simpset. As a result, Isabelle prints a
warning and \texttt{gcd.rules} contains a precondition:
\begin{ttbox}
(! m n. 0 < n --> m mod n < n) ==> gcd (m, n) = (if n=0 \dots)
\end{ttbox}
We need to instruct \texttt{recdef} to use an extended simpset to prove the
termination condition:
\begin{ttbox}
recdef gcd "measure ((\%(m,n).n))"
  simpset "simpset() addsimps [mod_less_divisor]"
    "gcd (m, n) = (if n=0 then m else gcd(n, m mod n))"
\end{ttbox}
This time everything works fine and \texttt{gcd.rules} contains precisely the
stated recursion equation for \texttt{gcd}.

When defining some nontrivial total recursive function, the first attempt
will usually generate a number of termination conditions, some of which may
require new lemmas to be proved in some of the parent theories. Those lemmas
can then be added to the simpset used by \texttt{recdef} for its
proofs, as shown for \texttt{gcd}.

Although all the above examples employ measure functions, \texttt{recdef}
allows arbitrary wellfounded relations. For example, termination of
Ackermann's function requires the lexicographic product \texttt{**}:
\begin{ttbox}
recdef ack "measure(\%m. m) ** measure(\%n. n)"
    "ack(0,n)         = Suc n"
    "ack(Suc m,0)     = ack(m, 1)"
    "ack(Suc m,Suc n) = ack(m,ack(Suc m,n))"
\end{ttbox}
For details see~\cite{Isa-Logics-Man} and the examples in the library.


\subsection{Deriving simplification rules}

Once we have succeeded to prove all termination conditions, we can start to
derive some theorems. In contrast to \texttt{primrec} definitions, which are
automatically added to the simpset, \texttt{recdef} rules must be included
explicitly, for example as in
\begin{ttbox}
Addsimps fib.rules;
\end{ttbox}
However, some care is necessary now, in contrast to \texttt{primrec}.
Although \texttt{gcd} is a total function, its defining equation leads to
nontermination of the simplifier, because the subterm \texttt{gcd(n, m mod
  n)} on the right-hand side can again be simplified by the same equation,
and so on. The reason: the simplifier rewrites the \texttt{then} and
\texttt{else} branches of a conditional if the condition simplifies to
neither \texttt{True} nor \texttt{False}.  Therefore it is recommended to
derive an alternative formulation that replaces case distinctions on the
right-hand side by conditional equations. For \texttt{gcd} it means we have
to prove
\begin{ttbox}
           gcd (m, 0) = m
n ~= 0 ==> gcd (m, n) = gcd(n, m mod n)
\end{ttbox}
To avoid nontermination during those proofs, we have to resort to some low
level tactics:
\begin{ttbox}
Goal "gcd(m,0) = m";
by(resolve_tac [trans] 1);
by(resolve_tac gcd.rules 1);
by(Simp_tac 1);
\end{ttbox}
At this point it is not necessary to understand what exactly
\texttt{resolve_tac} is doing. The main point is that the above proof works
not just for this one example but in general (except that we have to use
\texttt{Asm_simp_tac} and $f$\texttt{.rules} in general). Try the second
\texttt{gcd}-equation above!

\subsection{Induction}

Assuming we have added the recursion equations (or some suitable derived
equations) to the simpset, we might like to prove something about our
function. Since the function is recursive, the natural proof principle is
again induction. But this time the structural form of induction that comes
with datatypes is unlikely to work well---otherwise we could have defined the
function by \texttt{primrec}. Therefore \texttt{recdef} automatically proves
a suitable induction rule $f$\texttt{.induct} that follows the recursion
pattern of the particular function $f$. Roughly speaking, it requires you to
prove for each \texttt{recdef} equation that the property you are trying to
establish holds for the left-hand side provided it holds for all recursive
calls on the right-hand side. Applying $f$\texttt{.induct} requires its
explicit instantiation. See \S\ref{sec:explicit-inst} for details.

\index{*recdef|)}
