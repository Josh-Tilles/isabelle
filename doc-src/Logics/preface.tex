%% $Id$
\chapter*{Preface}
Several logics come with Isabelle.  Many of them are sufficiently developed
to serve as comfortable reasoning environments.  They are also good
starting points for defining new logics.  Each logic is distributed with
sample proofs, some of which are described in this document.

The logics \texttt{FOL} (first-order logic) and \texttt{ZF} (axiomatic set
theory) are described in a separate manual~\cite{isabelle-ZF}.  Here are the
others:

\begin{ttdescription}
\item[\thydx{CCL}] is Martin Coen's Classical Computational Logic,
  which is the basis of a preliminary method for deriving programs from
  proofs~\cite{coen92}.  It is built upon classical~\FOL{}.
 
\item[\thydx{LCF}] is a version of Scott's Logic for Computable
  Functions, which is also implemented by the~{\sc lcf}
  system~\cite{paulson87}.  It is built upon classical~\FOL{}.

\item[\thydx{HOL}] is the higher-order logic of Church~\cite{church40},
which is also implemented by Gordon's~{\sc hol} system~\cite{mgordon-hol}.
This object-logic should not be confused with Isabelle's meta-logic, which is
also a form of higher-order logic.

\item[\thydx{HOLCF}] is a version of {\sc lcf}, defined as an
  extension of \texttt{HOL}\@.
 
\item[\thydx{CTT}] is a version of Martin-L\"of's Constructive Type
Theory~\cite{nordstrom90}, with extensional equality.  Universes are not
included.

\item[\thydx{Cube}] is Barendregt's $\lambda$-cube.
 \end{ttdescription}

The directory \texttt{Sequents} contains several logics based
  upon the sequent calculus.  Sequents have the form $A@1,\ldots,A@m\turn
B@1,\ldots,B@n$; rules are applied using associative matching.
\begin{ttdescription}
\item[\thydx{LK}] is classical first-order logic as a sequent calculus.

\item[\thydx{Modal}] implements the modal logics $T$, $S4$, and~$S43$.  

\item[\thydx{ILL}] implements intuitionistic linear logic.
\end{ttdescription}

The logics \texttt{CCL}, \texttt{LCF}, \texttt{HOLCF}, \texttt{Modal}, \texttt{ILL} and {\tt
  Cube} are undocumented.  All object-logics' sources are
distributed with Isabelle (see the directory \texttt{src}).  They are
also available for browsing on the WWW at
\begin{ttbox}
http://www4.informatik.tu-muenchen.de/~nipkow/isabelle/
\end{ttbox}
Note that this is not necessarily consistent with your local sources!

\medskip Do not read this manual before reading \emph{Introduction to
  Isabelle} and performing some Isabelle proofs.  Consult the {\em Reference
  Manual} for more information on tactics, packages, etc.


