%% $Id$
\chapter*{Preface}
Several logics come with Isabelle.  Many of them are sufficiently developed
to serve as comfortable reasoning environments.  They are also good
starting points for defining new logics.  Each logic is distributed with
sample proofs, some of which are described in this document.

\texttt{HOL} is currently the best developed Isabelle object-logic, including
an extensive library of (concrete) mathematics, and various packages for
advanced definitional concepts (like (co-)inductive sets and types,
well-founded recursion etc.). The distribution also includes some large
applications.  See the separate manual \emph{Isabelle's Logics: HOL}.  There
is also a comprehensive tutorial on Isabelle/HOL available.

\texttt{ZF} provides another starting point for applications, with a slightly
less developed library than \texttt{HOL}.  \texttt{ZF}'s definitional packages
are similar to those of \texttt{HOL}. Untyped \texttt{ZF} set theory provides
more advanced constructions for sets than simply-typed \texttt{HOL}.
\texttt{ZF} is built on \texttt{FOL} (first-order logic), both are described
in a separate manual \emph{Isabelle's Logics: FOL and ZF}~\cite{isabelle-ZF}.

\medskip There are some further logics distributed with Isabelle:
\begin{ttdescription}
\item[\thydx{CCL}] is Martin Coen's Classical Computational Logic,
  which is the basis of a preliminary method for deriving programs from
  proofs~\cite{coen92}.  It is built upon classical~\FOL{}.
 
\item[\thydx{LCF}] is a version of Scott's Logic for Computable
  Functions, which is also implemented by the~{\sc lcf}
  system~\cite{paulson87}.  It is built upon classical~\FOL{}.
  
\item[\thydx{HOLCF}] is a version of {\sc lcf}, defined as an extension of
  \texttt{HOL}\@. %FIXME See \cite{MNOS98} for more details on \texttt{HOLCF}.
 
\item[\thydx{CTT}] is a version of Martin-L\"of's Constructive Type
Theory~\cite{nordstrom90}, with extensional equality.  Universes are not
included.

\item[\thydx{Cube}] is Barendregt's $\lambda$-cube.
 \end{ttdescription}

The directory \texttt{Sequents} contains several logics based
  upon the sequent calculus.  Sequents have the form $A@1,\ldots,A@m\turn
B@1,\ldots,B@n$; rules are applied using associative matching.
\begin{ttdescription}
\item[\thydx{LK}] is classical first-order logic as a sequent calculus.

\item[\thydx{Modal}] implements the modal logics $T$, $S4$, and~$S43$.  

\item[\thydx{ILL}] implements intuitionistic linear logic.
\end{ttdescription}

The logics \texttt{CCL}, \texttt{LCF}, \texttt{Modal}, \texttt{ILL} and {\tt
  Cube} are undocumented.  All object-logics' sources are distributed with
Isabelle (see the directory \texttt{src}).  They are also available for
browsing on the WWW at
\begin{ttbox}
http://www.cl.cam.ac.uk/Research/HVG/Isabelle/library/
http://isabelle.in.tum.de/library/
\end{ttbox}
Note that this is not necessarily consistent with your local sources!

\medskip Do not read the \emph{Isabelle's Logics} manuals before reading
\emph{Isabelle/HOL --- The Tutorial} or \emph{Introduction to Isabelle}, and
performing some Isabelle proofs.  Consult the {\em Reference Manual} for more
information on tactics, packages, etc.


%%% Local Variables: 
%%% mode: latex
%%% TeX-master: "logics"
%%% End: 
