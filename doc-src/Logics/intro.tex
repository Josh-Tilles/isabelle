%% $Id$
\chapter{Basic Concepts}
Several logics come with Isabelle.  Many of them are sufficiently developed
to serve as comfortable reasoning environments.  They are also good
starting points for defining new logics.  Each logic is distributed with
sample proofs, some of which are described in this document.

\begin{ttdescription}
\item[\thydx{FOL}] is many-sorted first-order logic with natural
deduction.  It comes in both constructive and classical versions.

\item[\thydx{ZF}] is axiomatic set theory, using the Zermelo-Fraenkel
axioms~\cite{suppes72}.  It is built upon classical~\FOL{}.

\item[\thydx{CCL}] is Martin Coen's Classical Computational Logic,
  which is the basis of a preliminary method for deriving programs from
  proofs~\cite{coen92}.  It is built upon classical~\FOL{}.
 
\item[\thydx{LCF}] is a version of Scott's Logic for Computable
  Functions, which is also implemented by the~{\sc lcf}
  system~\cite{paulson87}.  It is built upon classical~\FOL{}.

\item[\thydx{HOL}] is the higher-order logic of Church~\cite{church40},
which is also implemented by Gordon's~{\sc hol} system~\cite{mgordon-hol}.
This object-logic should not be confused with Isabelle's meta-logic, which is
also a form of higher-order logic.

\item[\thydx{HOLCF}] is a version of {\sc lcf}, defined as an
  extension of {\tt HOL}\@.
 
\item[\thydx{CTT}] is a version of Martin-L\"of's Constructive Type
Theory~\cite{nordstrom90}, with extensional equality.  Universes are not
included.
 
\item[\thydx{LK}] is another version of first-order logic, a classical
sequent calculus.  Sequents have the form $A@1,\ldots,A@m\turn
B@1,\ldots,B@n$; rules are applied using associative matching.

\item[\thydx{Modal}] implements the modal logics $T$, $S4$,
  and~$S43$.  It is built upon~\LK{}.

\item[\thydx{Cube}] is Barendregt's $\lambda$-cube.
\end{ttdescription}
The logics {\tt CCL}, {\tt LCF}, {\tt HOLCF}, {\tt Modal} and {\tt Cube}
are currently undocumented.

You should not read this before reading {\em Introduction to Isabelle\/}
and performing some Isabelle proofs.  Consult the {\em Reference Manual}
for more information on tactics, packages, etc.


\section{Syntax definitions}
The syntax of each logic is presented using a context-free grammar.
These grammars obey the following conventions:
\begin{itemize}
\item identifiers denote nonterminal symbols
\item {\tt typewriter} font denotes terminal symbols
\item parentheses $(\ldots)$ express grouping
\item constructs followed by a Kleene star, such as $id^*$ and $(\ldots)^*$
can be repeated~0 or more times 
\item alternatives are separated by a vertical bar,~$|$
\item the symbol for alphanumeric identifiers is~{\it id\/} 
\item the symbol for scheme variables is~{\it var}
\end{itemize}
To reduce the number of nonterminals and grammar rules required, Isabelle's
syntax module employs {\bf priorities},\index{priorities} or precedences.
Each grammar rule is given by a mixfix declaration, which has a priority,
and each argument place has a priority.  This general approach handles
infix operators that associate either to the left or to the right, as well
as prefix and binding operators.

In a syntactically valid expression, an operator's arguments never involve
an operator of lower priority unless brackets are used.  Consider
first-order logic, where $\exists$ has lower priority than $\disj$,
which has lower priority than $\conj$.  There, $P\conj Q \disj R$
abbreviates $(P\conj Q) \disj R$ rather than $P\conj (Q\disj R)$.  Also,
$\exists x.P\disj Q$ abbreviates $\exists x.(P\disj Q)$ rather than
$(\exists x.P)\disj Q$.  Note especially that $P\disj(\exists x.Q)$
becomes syntactically invalid if the brackets are removed.

A {\bf binder} is a symbol associated with a constant of type
$(\sigma\To\tau)\To\tau'$.  For instance, we may declare~$\forall$ as
a binder for the constant~$All$, which has type $(\alpha\To o)\To o$.
This defines the syntax $\forall x.t$ to mean $All(\lambda x.t)$.  We
can also write $\forall x@1\ldots x@m.t$ to abbreviate $\forall x@1.
\ldots \forall x@m.t$; this is possible for any constant provided that
$\tau$ and $\tau'$ are the same type.  \HOL's description operator
$\varepsilon x.P\,x$ has type $(\alpha\To bool)\To\alpha$ and can bind
only one variable, except when $\alpha$ is $bool$.  \ZF's bounded
quantifier $\forall x\in A.P(x)$ cannot be declared as a binder
because it has type $[i, i\To o]\To o$.  The syntax for binders allows
type constraints on bound variables, as in
\[ \forall (x{::}\alpha) \; y{::}\beta. R(x,y) \]

To avoid excess detail, the logic descriptions adopt a semi-formal style.
Infix operators and binding operators are listed in separate tables, which
include their priorities.  Grammar descriptions do not include numeric
priorities; instead, the rules appear in order of decreasing priority.
This should suffice for most purposes; for full details, please consult the
actual syntax definitions in the {\tt.thy} files.

Each nonterminal symbol is associated with some Isabelle type.  For
example, the formulae of first-order logic have type~$o$.  Every
Isabelle expression of type~$o$ is therefore a formula.  These include
atomic formulae such as $P$, where $P$ is a variable of type~$o$, and more
generally expressions such as $P(t,u)$, where $P$, $t$ and~$u$ have
suitable types.  Therefore, `expression of type~$o$' is listed as a
separate possibility in the grammar for formulae.


\section{Proof procedures}\label{sec:safe}
Most object-logics come with simple proof procedures.  These are reasonably
powerful for interactive use, though often simplistic and incomplete.  You
can do single-step proofs using \verb|resolve_tac| and
\verb|assume_tac|, referring to the inference rules of the logic by {\sc
ml} identifiers.

For theorem proving, rules can be classified as {\bf safe} or {\bf unsafe}.
A rule is safe if applying it to a provable goal always yields provable
subgoals.  If a rule is safe then it can be applied automatically to a goal
without destroying our chances of finding a proof.  For instance, all the
rules of the classical sequent calculus {\sc lk} are safe.  Universal
elimination is unsafe if the formula $\all{x}P(x)$ is deleted after use.
Other unsafe rules include the following:
\[ \infer[({\disj}I1)]{P\disj Q}{P} \qquad 
   \infer[({\imp}E)]{Q}{P\imp Q & P} \qquad
   \infer[({\exists}I)]{\exists x.P}{P[t/x]} 
\]

Proof procedures use safe rules whenever possible, delaying the application
of unsafe rules. Those safe rules are preferred that generate the fewest
subgoals. Safe rules are (by definition) deterministic, while the unsafe
rules require search. The design of a suitable set of rules can be as
important as the strategy for applying them.

Many of the proof procedures use backtracking.  Typically they attempt to
solve subgoal~$i$ by repeatedly applying a certain tactic to it.  This
tactic, which is known as a {\bf step tactic}, resolves a selection of
rules with subgoal~$i$. This may replace one subgoal by many;  the
search persists until there are fewer subgoals in total than at the start.
Backtracking happens when the search reaches a dead end: when the step
tactic fails.  Alternative outcomes are then searched by a depth-first or
best-first strategy.  
