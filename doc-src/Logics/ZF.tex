%% $Id$
%%%See grant/bra/Lib/ZF.tex for lfp figure
\chapter{Zermelo-Fraenkel set theory}
The directory~\ttindexbold{ZF} implements Zermelo-Fraenkel set
theory~\cite{halmos60,suppes72} as an extension of~\ttindex{FOL}, classical
first-order logic.  The theory includes a collection of derived natural
deduction rules, for use with Isabelle's classical reasoning module.  Much
of it is based on the work of No\"el~\cite{noel}.  The theory has the {\ML}
identifier \ttindexbold{ZF.thy}.  However, many further theories
are defined, introducing the natural numbers, etc.

A tremendous amount of set theory has been formally developed, including
the basic properties of relations, functions and ordinals.  Significant
results have been proved, such as the Schr\"oder-Bernstein Theorem and the
Recursion Theorem.  General methods have been developed for solving
recursion equations over monotonic functors; these have been applied to
yield constructions of lists and trees.  Thus, we may even regard set
theory as a computational logic.  It admits recursive definitions of
functions and types.  It has much in common with Martin-L\"of type theory,
although of course it is classical.

Because {\ZF} is an extension of {\FOL}, it provides the same packages,
namely \ttindex{hyp_subst_tac}, the simplifier, and the classical reasoning
module.  The main simplification set is called \ttindexbold{ZF_ss}.
Several classical rule sets are defined, including \ttindexbold{lemmas_cs},
\ttindexbold{upair_cs} and~\ttindexbold{ZF_cs}.  See the files on directory
{\tt ZF} for details.

Isabelle/ZF now has a flexible package for handling inductive definitions,
such as inference systems, and datatype definitions, such as lists and
trees.  Moreover it also handles co-inductive definitions, such as
bisimulation relations, and co-datatype definitions, such as streams.  A
recent paper describes the package~\cite{paulson-fixedpt}.  

Recent reports describe Isabelle/ZF less formally than this
chapter~\cite{paulson-set-I,paulson-set-II}.  Isabelle/ZF employs a novel
treatment of non-well-founded data structures within the standard ZF axioms
including the Axiom of Foundation, but no report describes this treatment.


\section{Which version of axiomatic set theory?}
Resolution theorem provers can work in set theory, using the
Bernays-G\"odel axiom system~(BG) because it is
finite~\cite{boyer86,quaife92}.  {\ZF} does not have a finite axiom system
(because of its Axiom Scheme of Replacement) and is therefore unsuitable
for classical resolution.  Since Isabelle has no difficulty with axiom
schemes, we may adopt either axiom system.

These two theories differ in their treatment of {\bf classes}, which are
collections that are ``too big'' to be sets.  The class of all sets,~$V$,
cannot be a set without admitting Russell's Paradox.  In BG, both classes
and sets are individuals; $x\in V$ expresses that $x$ is a set.  In {\ZF}, all
variables denote sets; classes are identified with unary predicates.  The
two systems define essentially the same sets and classes, with similar
properties.  In particular, a class cannot belong to another class (let
alone a set).

Modern set theorists tend to prefer {\ZF} because they are mainly concerned
with sets, rather than classes.  BG requires tiresome proofs that various
collections are sets; for instance, showing $x\in\{x\}$ requires showing that
$x$ is a set.  {\ZF} does not have this problem.


\begin{figure} 
\begin{center}
\begin{tabular}{rrr} 
  \it name      &\it meta-type  & \it description \\ 
  \idx{0}       & $i$           & empty set\\
  \idx{cons}    & $[i,i]\To i$  & finite set constructor\\
  \idx{Upair}   & $[i,i]\To i$  & unordered pairing\\
  \idx{Pair}    & $[i,i]\To i$  & ordered pairing\\
  \idx{Inf}     & $i$   & infinite set\\
  \idx{Pow}     & $i\To i$      & powerset\\
  \idx{Union} \idx{Inter} & $i\To i$    & set union/intersection \\
  \idx{split}   & $[[i,i]\To i, i] \To i$ & generalized projection\\
  \idx{fst} \idx{snd}   & $i\To i$      & projections\\
  \idx{converse}& $i\To i$      & converse of a relation\\
  \idx{succ}    & $i\To i$      & successor\\
  \idx{Collect} & $[i,i\To o]\To i$     & separation\\
  \idx{Replace} & $[i, [i,i]\To o] \To i$       & replacement\\
  \idx{PrimReplace} & $[i, [i,i]\To o] \To i$   & primitive replacement\\
  \idx{RepFun}  & $[i, i\To i] \To i$   & functional replacement\\
  \idx{Pi} \idx{Sigma}  & $[i,i\To i]\To i$     & general product/sum\\
  \idx{domain}  & $i\To i$      & domain of a relation\\
  \idx{range}   & $i\To i$      & range of a relation\\
  \idx{field}   & $i\To i$      & field of a relation\\
  \idx{Lambda}  & $[i, i\To i]\To i$    & $\lambda$-abstraction\\
  \idx{restrict}& $[i, i] \To i$        & restriction of a function\\
  \idx{The}     & $[i\To o]\To i$       & definite description\\
  \idx{if}      & $[o,i,i]\To i$        & conditional\\
  \idx{Ball} \idx{Bex}  & $[i, i\To o]\To o$    & bounded quantifiers
\end{tabular}
\end{center}
\subcaption{Constants}

\begin{center}
\indexbold{*"`"`}
\indexbold{*"-"`"`}
\indexbold{*"`}
\indexbold{*"-}
\indexbold{*":}
\indexbold{*"<"=}
\begin{tabular}{rrrr} 
  \it symbol  & \it meta-type & \it precedence & \it description \\ 
  \tt ``        & $[i,i]\To i$  &  Left 90      & image \\
  \tt -``       & $[i,i]\To i$  &  Left 90      & inverse image \\
  \tt `         & $[i,i]\To i$  &  Left 90      & application \\
  \idx{Int}     & $[i,i]\To i$  &  Left 70      & intersection ($\inter$) \\
  \idx{Un}      & $[i,i]\To i$  &  Left 65      & union ($\union$) \\
  \tt -         & $[i,i]\To i$  &  Left 65      & set difference ($-$) \\[1ex]
  \tt:          & $[i,i]\To o$  &  Left 50      & membership ($\in$) \\
  \tt <=        & $[i,i]\To o$  &  Left 50      & subset ($\subseteq$) 
\end{tabular}
\end{center}
\subcaption{Infixes}
\caption{Constants of {\ZF}} \label{ZF-constants}
\end{figure} 


\section{The syntax of set theory}
The language of set theory, as studied by logicians, has no constants.  The
traditional axioms merely assert the existence of empty sets, unions,
powersets, etc.; this would be intolerable for practical reasoning.  The
Isabelle theory declares constants for primitive sets.  It also extends
{\tt FOL} with additional syntax for finite sets, ordered pairs,
comprehension, general union/intersection, general sums/products, and
bounded quantifiers.  In most other respects, Isabelle implements precisely
Zermelo-Fraenkel set theory.

Figure~\ref{ZF-constants} lists the constants and infixes of~\ZF, while
Figure~\ref{ZF-trans} presents the syntax translations.  Finally,
Figure~\ref{ZF-syntax} presents the full grammar for set theory, including
the constructs of \FOL.

Set theory does not use polymorphism.  All terms in {\ZF} have type~{\it
i}, which is the type of individuals and lies in class {\it logic}.
The type of first-order formulae,
remember, is~{\it o}.

Infix operators include union and intersection ($A\union B$ and $A\inter
B$), and the subset and membership relations.  Note that $a$\verb|~:|$b$ is
translated to \verb|~(|$a$:$b$\verb|)|.  The union and intersection
operators ($\bigcup A$ and $\bigcap A$) form the union or intersection of a
set of sets; $\bigcup A$ means the same as $\bigcup@{x\in A}x$.  Of these
operators, only $\bigcup A$ is primitive.

The constant \ttindexbold{Upair} constructs unordered pairs; thus {\tt
Upair($A$,$B$)} denotes the set~$\{A,B\}$ and {\tt Upair($A$,$A$)} denotes
the singleton~$\{A\}$.  As usual in {\ZF}, general union is used to define
binary union.  The Isabelle version goes on to define the constant
\ttindexbold{cons}:
\begin{eqnarray*}
   A\cup B              & \equiv &       \bigcup({\tt Upair}(A,B)) \\
   {\tt cons}(a,B)      & \equiv &        {\tt Upair}(a,a) \union B
\end{eqnarray*}
The {\tt\{\ldots\}} notation abbreviates finite sets constructed in the
obvious manner using~{\tt cons} and~$\emptyset$ (the empty set):
\begin{eqnarray*}
 \{a,b,c\} & \equiv & {\tt cons}(a,{\tt cons}(b,{\tt cons}(c,\emptyset)))
\end{eqnarray*}

The constant \ttindexbold{Pair} constructs ordered pairs, as in {\tt
Pair($a$,$b$)}.  Ordered pairs may also be written within angle brackets,
as {\tt<$a$,$b$>}.  The $n$-tuple {\tt<$a@1$,\ldots,$a@{n-1}$,$a@n$>}
abbreviates the nest of pairs {\tt
  Pair($a@1$,\ldots,Pair($a@{n-1}$,$a@n$)\ldots)}. 

In {\ZF}, a function is a set of pairs.  A {\ZF} function~$f$ is simply an
individual as far as Isabelle is concerned: its Isabelle type is~$i$, not
say $i\To i$.  The infix operator~{\tt`} denotes the application of a
function set to its argument; we must write~$f{\tt`}x$, not~$f(x)$.  The
syntax for image is~$f{\tt``}A$ and that for inverse image is~$f{\tt-``}A$.


\begin{figure} 
\indexbold{*"-">}
\indexbold{*"*}
\begin{center} \tt\frenchspacing
\begin{tabular}{rrr} 
  \it external          & \it internal  & \it description \\ 
  $a$ \ttilde: $b$      & \ttilde($a$ : $b$)    & \rm negated membership\\
  \{$a@1$, $\ldots$, $a@n$\}  &  cons($a@1$,$\cdots$,cons($a@n$,0)) &
        \rm finite set \\
  <$a@1$, $\ldots$, $a@{n-1}$, $a@n$> & 
        Pair($a@1$,\ldots,Pair($a@{n-1}$,$a@n$)\ldots) &
        \rm ordered $n$-tuple \\
  \{$x$:$A . P[x]$\}    &  Collect($A$,$\lambda x.P[x]$) &
        \rm separation \\
  \{$y . x$:$A$, $Q[x,y]$\}  &  Replace($A$,$\lambda x\,y.Q[x,y]$) &
        \rm replacement \\
  \{$b[x] . x$:$A$\}  &  RepFun($A$,$\lambda x.b[x]$) &
        \rm functional replacement \\
  \idx{INT} $x$:$A . B[x]$      & Inter(\{$B[x] . x$:$A$\}) &
        \rm general intersection \\
  \idx{UN}  $x$:$A . B[x]$      & Union(\{$B[x] . x$:$A$\}) &
        \rm general union \\
  \idx{PROD} $x$:$A . B[x]$     & Pi($A$,$\lambda x.B[x]$) & 
        \rm general product \\
  \idx{SUM}  $x$:$A . B[x]$     & Sigma($A$,$\lambda x.B[x]$) & 
        \rm general sum \\
  $A$ -> $B$            & Pi($A$,$\lambda x.B$) & 
        \rm function space \\
  $A$ * $B$             & Sigma($A$,$\lambda x.B$) & 
        \rm binary product \\
  \idx{THE}  $x . P[x]$ & The($\lambda x.P[x]$) & 
        \rm definite description \\
  \idx{lam}  $x$:$A . b[x]$     & Lambda($A$,$\lambda x.b[x]$) & 
        \rm $\lambda$-abstraction\\[1ex]
  \idx{ALL} $x$:$A . P[x]$      & Ball($A$,$\lambda x.P[x]$) & 
        \rm bounded $\forall$ \\
  \idx{EX}  $x$:$A . P[x]$      & Bex($A$,$\lambda x.P[x]$) & 
        \rm bounded $\exists$
\end{tabular}
\end{center}
\caption{Translations for {\ZF}} \label{ZF-trans}
\end{figure} 


\begin{figure} 
\dquotes
\[\begin{array}{rcl}
    term & = & \hbox{expression of type~$i$} \\
         & | & "\{ " term\; ("," term)^* " \}" \\
         & | & "< " term ", " term " >" \\
         & | & "\{ " id ":" term " . " formula " \}" \\
         & | & "\{ " id " . " id ":" term "," formula " \}" \\
         & | & "\{ " term " . " id ":" term " \}" \\
         & | & term " `` " term \\
         & | & term " -`` " term \\
         & | & term " ` " term \\
         & | & term " * " term \\
         & | & term " Int " term \\
         & | & term " Un " term \\
         & | & term " - " term \\
         & | & term " -> " term \\
         & | & "THE~~"  id  " . " formula\\
         & | & "lam~~"  id ":" term " . " term \\
         & | & "INT~~"  id ":" term " . " term \\
         & | & "UN~~~"  id ":" term " . " term \\
         & | & "PROD~"  id ":" term " . " term \\
         & | & "SUM~~"  id ":" term " . " term \\[2ex]
 formula & = & \hbox{expression of type~$o$} \\
         & | & term " : " term \\
         & | & term " <= " term \\
         & | & term " = " term \\
         & | & "\ttilde\ " formula \\
         & | & formula " \& " formula \\
         & | & formula " | " formula \\
         & | & formula " --> " formula \\
         & | & formula " <-> " formula \\
         & | & "ALL " id ":" term " . " formula \\
         & | & "EX~~" id ":" term " . " formula \\
         & | & "ALL~" id~id^* " . " formula \\
         & | & "EX~~" id~id^* " . " formula \\
         & | & "EX!~" id~id^* " . " formula
  \end{array}
\]
\caption{Full grammar for {\ZF}} \label{ZF-syntax}
\end{figure} 


\section{Binding operators}
The constant \ttindexbold{Collect} constructs sets by the principle of {\bf
  separation}.  The syntax for separation is \hbox{\tt\{$x$:$A$.$P[x]$\}},
where $P[x]$ is a formula that may contain free occurrences of~$x$.  It
abbreviates the set {\tt Collect($A$,$\lambda x.P$[x])}, which consists of
all $x\in A$ that satisfy~$P[x]$.  Note that {\tt Collect} is an
unfortunate choice of name: some set theories adopt a set-formation
principle, related to replacement, called collection.

The constant \ttindexbold{Replace} constructs sets by the principle of {\bf
  replacement}.  The syntax for replacement is
\hbox{\tt\{$y$.$x$:$A$,$Q[x,y]$\}}.  It denotes the set {\tt
  Replace($A$,$\lambda x\,y.Q$[x,y])} consisting of all $y$ such that there
exists $x\in A$ satisfying~$Q[x,y]$.  The Replacement Axiom has the
condition that $Q$ must be single-valued over~$A$: for all~$x\in A$ there
exists at most one $y$ satisfying~$Q[x,y]$.  A single-valued binary
predicate is also called a {\bf class function}.

The constant \ttindexbold{RepFun} expresses a special case of replacement,
where $Q[x,y]$ has the form $y=b[x]$.  Such a $Q$ is trivially
single-valued, since it is just the graph of the meta-level
function~$\lambda x.b[x]$.  The syntax is \hbox{\tt\{$b[x]$.$x$:$A$\}},
denoting set {\tt RepFun($A$,$\lambda x.b[x]$)} of all $b[x]$ for~$x\in A$.
This is analogous to the \ML{} functional {\tt map}, since it applies a
function to every element of a set.

\indexbold{*INT}\indexbold{*UN}
General unions and intersections of families, namely $\bigcup@{x\in A}B[x]$ and
$\bigcap@{x\in A}B[x]$, are written \hbox{\tt UN $x$:$A$.$B[x]$} and
\hbox{\tt INT $x$:$A$.$B[x]$}.  Their meaning is expressed using {\tt
RepFun} as
\[ \bigcup(\{B[x]. x\in A\}) \qquad\hbox{and}\qquad 
   \bigcap(\{B[x]. x\in A\}). 
\]
General sums $\sum@{x\in A}B[x]$ and products $\prod@{x\in A}B[x]$ can be
constructed in set theory, where $B[x]$ is a family of sets over~$A$.  They
have as special cases $A\times B$ and $A\to B$, where $B$ is simply a set.
This is similar to the situation in Constructive Type Theory (set theory
has ``dependent sets'') and calls for similar syntactic conventions.  The
constants~\ttindexbold{Sigma} and~\ttindexbold{Pi} construct general sums and
products.  Instead of {\tt Sigma($A$,$B$)} and {\tt Pi($A$,$B$)} we may write
\hbox{\tt SUM $x$:$A$.$B[x]$} and \hbox{\tt PROD $x$:$A$.$B[x]$}.  
\indexbold{*SUM}\indexbold{*PROD}%
The special cases as \hbox{\tt$A$*$B$} and \hbox{\tt$A$->$B$} abbreviate
general sums and products over a constant family.\footnote{Unlike normal
infix operators, {\tt*} and {\tt->} merely define abbreviations; there are
no constants~{\tt op~*} and~\hbox{\tt op~->}.} Isabelle accepts these
abbreviations in parsing and uses them whenever possible for printing.

\indexbold{*THE} 
As mentioned above, whenever the axioms assert the existence and uniqueness
of a set, Isabelle's set theory declares a constant for that set.  These
constants can express the {\bf definite description} operator~$\iota
x.P[x]$, which stands for the unique~$a$ satisfying~$P[a]$, if such exists.
Since all terms in {\ZF} denote something, a description is always
meaningful, but we do not know its value unless $P[x]$ defines it uniquely.
Using the constant~\ttindexbold{The}, we may write descriptions as {\tt
  The($\lambda x.P[x]$)} or use the syntax \hbox{\tt THE $x$.$P[x]$}.

\indexbold{*lam}
Function sets may be written in $\lambda$-notation; $\lambda x\in A.b[x]$
stands for the set of all pairs $\pair{x,b[x]}$ for $x\in A$.  In order for
this to be a set, the function's domain~$A$ must be given.  Using the
constant~\ttindexbold{Lambda}, we may express function sets as {\tt
Lambda($A$,$\lambda x.b[x]$)} or use the syntax \hbox{\tt lam $x$:$A$.$b[x]$}.

Isabelle's set theory defines two {\bf bounded quantifiers}:
\begin{eqnarray*}
   \forall x\in A.P[x] &\hbox{which abbreviates}& \forall x. x\in A\imp P[x] \\
   \exists x\in A.P[x] &\hbox{which abbreviates}& \exists x. x\in A\conj P[x]
\end{eqnarray*}
The constants~\ttindexbold{Ball} and~\ttindexbold{Bex} are defined
accordingly.  Instead of {\tt Ball($A$,$P$)} and {\tt Bex($A$,$P$)} we may
write
\hbox{\tt ALL $x$:$A$.$P[x]$} and \hbox{\tt EX $x$:$A$.$P[x]$}.


%%%% zf.thy

\begin{figure}
\begin{ttbox}
\idx{Ball_def}           Ball(A,P) == ALL x. x:A --> P(x)
\idx{Bex_def}            Bex(A,P)  == EX x. x:A & P(x)

\idx{subset_def}         A <= B  == ALL x:A. x:B
\idx{extension}          A = B  <->  A <= B & B <= A

\idx{union_iff}          A : Union(C) <-> (EX B:C. A:B)
\idx{power_set}          A : Pow(B) <-> A <= B
\idx{foundation}         A=0 | (EX x:A. ALL y:x. ~ y:A)

\idx{replacement}        (ALL x:A. ALL y z. P(x,y) & P(x,z) --> y=z) ==>
                   b : PrimReplace(A,P) <-> (EX x:A. P(x,b))
\subcaption{The Zermelo-Fraenkel Axioms}

\idx{Replace_def}  Replace(A,P) == 
                   PrimReplace(A, %x y. (EX!z.P(x,z)) & P(x,y))
\idx{RepFun_def}   RepFun(A,f)  == \{y . x:A, y=f(x)\}
\idx{the_def}      The(P)       == Union(\{y . x:\{0\}, P(y)\})
\idx{if_def}       if(P,a,b)    == THE z. P & z=a | ~P & z=b
\idx{Collect_def}  Collect(A,P) == \{y . x:A, x=y & P(x)\}
\idx{Upair_def}    Upair(a,b)   == 
                 \{y. x:Pow(Pow(0)), (x=0 & y=a) | (x=Pow(0) & y=b)\}
\subcaption{Consequences of replacement}

\idx{Inter_def}    Inter(A) == \{ x:Union(A) . ALL y:A. x:y\}
\idx{Un_def}       A Un  B  == Union(Upair(A,B))
\idx{Int_def}      A Int B  == Inter(Upair(A,B))
\idx{Diff_def}     A - B    == \{ x:A . ~(x:B) \}
\subcaption{Union, intersection, difference}
\end{ttbox}
\caption{Rules and axioms of {\ZF}} \label{ZF-rules}
\end{figure}


\begin{figure}
\begin{ttbox}
\idx{cons_def}     cons(a,A) == Upair(a,a) Un A
\idx{succ_def}     succ(i) == cons(i,i)
\idx{infinity}     0:Inf & (ALL y:Inf. succ(y): Inf)
\subcaption{Finite and infinite sets}

\idx{Pair_def}       <a,b>      == \{\{a,a\}, \{a,b\}\}
\idx{split_def}      split(c,p) == THE y. EX a b. p=<a,b> & y=c(a,b)
\idx{fst_def}        fst(A)     == split(%x y.x, p)
\idx{snd_def}        snd(A)     == split(%x y.y, p)
\idx{Sigma_def}      Sigma(A,B) == UN x:A. UN y:B(x). \{<x,y>\}
\subcaption{Ordered pairs and Cartesian products}

\idx{converse_def}   converse(r) == \{z. w:r, EX x y. w=<x,y> & z=<y,x>\}
\idx{domain_def}     domain(r)   == \{x. w:r, EX y. w=<x,y>\}
\idx{range_def}      range(r)    == domain(converse(r))
\idx{field_def}      field(r)    == domain(r) Un range(r)
\idx{image_def}      r `` A      == \{y : range(r) . EX x:A. <x,y> : r\}
\idx{vimage_def}     r -`` A     == converse(r)``A
\subcaption{Operations on relations}

\idx{lam_def}    Lambda(A,b) == \{<x,b(x)> . x:A\}
\idx{apply_def}  f`a         == THE y. <a,y> : f
\idx{Pi_def}     Pi(A,B) == \{f: Pow(Sigma(A,B)). ALL x:A. EX! y. <x,y>: f\}
\idx{restrict_def}   restrict(f,A) == lam x:A.f`x
\subcaption{Functions and general product}
\end{ttbox}
\caption{Further definitions of {\ZF}} \label{ZF-defs}
\end{figure}


%%%% zf.ML

\begin{figure}
\begin{ttbox}
\idx{ballI}       [| !!x. x:A ==> P(x) |] ==> ALL x:A. P(x)
\idx{bspec}       [| ALL x:A. P(x);  x: A |] ==> P(x)
\idx{ballE}       [| ALL x:A. P(x);  P(x) ==> Q;  ~ x:A ==> Q |] ==> Q

\idx{ball_cong}   [| A=A';  !!x. x:A' ==> P(x) <-> P'(x) |] ==> 
            (ALL x:A. P(x)) <-> (ALL x:A'. P'(x))

\idx{bexI}        [| P(x);  x: A |] ==> EX x:A. P(x)
\idx{bexCI}       [| ALL x:A. ~P(x) ==> P(a);  a: A |] ==> EX x:A.P(x)
\idx{bexE}        [| EX x:A. P(x);  !!x. [| x:A; P(x) |] ==> Q |] ==> Q

\idx{bex_cong}    [| A=A';  !!x. x:A' ==> P(x) <-> P'(x) |] ==> 
            (EX x:A. P(x)) <-> (EX x:A'. P'(x))
\subcaption{Bounded quantifiers}

\idx{subsetI}       (!!x.x:A ==> x:B) ==> A <= B
\idx{subsetD}       [| A <= B;  c:A |] ==> c:B
\idx{subsetCE}      [| A <= B;  ~(c:A) ==> P;  c:B ==> P |] ==> P
\idx{subset_refl}   A <= A
\idx{subset_trans}  [| A<=B;  B<=C |] ==> A<=C

\idx{equalityI}     [| A <= B;  B <= A |] ==> A = B
\idx{equalityD1}    A = B ==> A<=B
\idx{equalityD2}    A = B ==> B<=A
\idx{equalityE}     [| A = B;  [| A<=B; B<=A |] ==> P |]  ==>  P
\subcaption{Subsets and extensionality}

\idx{emptyE}          a:0 ==> P
\idx{empty_subsetI}   0 <= A
\idx{equals0I}        [| !!y. y:A ==> False |] ==> A=0
\idx{equals0D}        [| A=0;  a:A |] ==> P

\idx{PowI}            A <= B ==> A : Pow(B)
\idx{PowD}            A : Pow(B)  ==>  A<=B
\subcaption{The empty set; power sets}
\end{ttbox}
\caption{Basic derived rules for {\ZF}} \label{ZF-lemmas1}
\end{figure}



\begin{figure}
\begin{ttbox}
\idx{ReplaceI}      [| x: A;  P(x,b);  !!y. P(x,y) ==> y=b |] ==> 
              b : \{y. x:A, P(x,y)\}

\idx{ReplaceE}      [| b : \{y. x:A, P(x,y)\};  
                 !!x. [| x: A;  P(x,b);  ALL y. P(x,y)-->y=b |] ==> R 
              |] ==> R

\idx{RepFunI}       [| a : A |] ==> f(a) : \{f(x). x:A\}
\idx{RepFunE}       [| b : \{f(x). x:A\};  
                 !!x.[| x:A;  b=f(x) |] ==> P |] ==> P

\idx{separation}     a : \{x:A. P(x)\} <-> a:A & P(a)
\idx{CollectI}       [| a:A;  P(a) |] ==> a : \{x:A. P(x)\}
\idx{CollectE}       [| a : \{x:A. P(x)\};  [| a:A; P(a) |] ==> R |] ==> R
\idx{CollectD1}      a : \{x:A. P(x)\} ==> a:A
\idx{CollectD2}      a : \{x:A. P(x)\} ==> P(a)
\end{ttbox}
\caption{Replacement and separation} \label{ZF-lemmas2}
\end{figure}


\begin{figure}
\begin{ttbox}
\idx{UnionI}    [| B: C;  A: B |] ==> A: Union(C)
\idx{UnionE}    [| A : Union(C);  !!B.[| A: B;  B: C |] ==> R |] ==> R

\idx{InterI}    [| !!x. x: C ==> A: x;  c:C |] ==> A : Inter(C)
\idx{InterD}    [| A : Inter(C);  B : C |] ==> A : B
\idx{InterE}    [| A : Inter(C);  A:B ==> R;  ~ B:C ==> R |] ==> R

\idx{UN_I}      [| a: A;  b: B(a) |] ==> b: (UN x:A. B(x))
\idx{UN_E}      [| b : (UN x:A. B(x));  !!x.[| x: A;  b: B(x) |] ==> R 
          |] ==> R

\idx{INT_I}     [| !!x. x: A ==> b: B(x);  a: A |] ==> b: (INT x:A. B(x))
\idx{INT_E}     [| b : (INT x:A. B(x));  a: A |] ==> b : B(a)
\end{ttbox}
\caption{General Union and Intersection} \label{ZF-lemmas3}
\end{figure}


\section{The Zermelo-Fraenkel axioms}
The axioms appear in Figure~\ref{ZF-rules}.  They resemble those
presented by Suppes~\cite{suppes72}.  Most of the theory consists of
definitions.  In particular, bounded quantifiers and the subset relation
appear in other axioms.  Object-level quantifiers and implications have
been replaced by meta-level ones wherever possible, to simplify use of the
axioms.  See the file \ttindexbold{ZF/zf.thy} for details.

The traditional replacement axiom asserts
\[ y \in {\tt PrimReplace}(A,P) \bimp (\exists x\in A. P(x,y)) \]
subject to the condition that $P(x,y)$ is single-valued for all~$x\in A$.
The Isabelle theory defines \ttindex{Replace} to apply
\ttindexbold{PrimReplace} to the single-valued part of~$P$, namely
\[ (\exists!z.P(x,z)) \conj P(x,y). \]
Thus $y\in {\tt Replace}(A,P)$ if and only if there is some~$x$ such that
$P(x,-)$ holds uniquely for~$y$.  Because the equivalence is unconditional,
{\tt Replace} is much easier to use than {\tt PrimReplace}; it defines the
same set, if $P(x,y)$ is single-valued.  The nice syntax for replacement
expands to {\tt Replace}.

Other consequences of replacement include functional replacement
(\ttindexbold{RepFun}) and definite descriptions (\ttindexbold{The}).
Axioms for separation (\ttindexbold{Collect}) and unordered pairs
(\ttindexbold{Upair}) are traditionally assumed, but they actually follow
from replacement~\cite[pages 237--8]{suppes72}.

The definitions of general intersection, etc., are straightforward.  Note
the definition of \ttindex{cons}, which underlies the finite set notation.
The axiom of infinity gives us a set that contains~0 and is closed under
successor (\ttindexbold{succ}).  Although this set is not uniquely defined,
the theory names it (\ttindexbold{Inf}) in order to simplify the
construction of the natural numbers.
                                             
Further definitions appear in Figure~\ref{ZF-defs}.  Ordered pairs are
defined in the standard way, $\pair{a,b}\equiv\{\{a\},\{a,b\}\}$.  Recall
that \ttindexbold{Sigma}$(A,B)$ generalizes the Cartesian product of two
sets.  It is defined to be the union of all singleton sets
$\{\pair{x,y}\}$, for $x\in A$ and $y\in B(x)$.  This is a typical usage of
general union.

The projections involve definite descriptions.  The \ttindex{split}
operation is like the similar operation in Martin-L\"of Type Theory, and is
often easier to use than \ttindex{fst} and~\ttindex{snd}.  It is defined
using a description for convenience, but could equivalently be defined by
\begin{ttbox}
split(c,p) == c(fst(p),snd(p))
\end{ttbox}  
Operations on relations include converse, domain, range, and image.  The
set ${\tt Pi}(A,B)$ generalizes the space of functions between two sets.
Note the simple definitions of $\lambda$-abstraction (using
\ttindex{RepFun}) and application (using a definite description).  The
function \ttindex{restrict}$(f,A)$ has the same values as~$f$, but only
over the domain~$A$.

No axiom of choice is provided.  It is traditional to include this axiom
only where it is needed --- mainly in the theory of cardinal numbers, which
Isabelle does not formalize at present.


\section{From basic lemmas to function spaces}
Faced with so many definitions, it is essential to prove lemmas.  Even
trivial theorems like $A\inter B=B\inter A$ would be difficult to prove
from the definitions alone.  Isabelle's set theory derives many rules using
a natural deduction style.  Ideally, a natural deduction rule should
introduce or eliminate just one operator, but this is not always practical.
For most operators, we may forget its definition and use its derived rules
instead.

\subsection{Fundamental lemmas}
Figure~\ref{ZF-lemmas1} presents the derived rules for the most basic
operators.  The rules for the bounded quantifiers resemble those for the
ordinary quantifiers, but note that \ttindex{BallE} uses a negated
assumption in the style of Isabelle's classical module.  The congruence rules
\ttindex{ball_cong} and \ttindex{bex_cong} are required by Isabelle's
simplifier, but have few other uses.  Congruence rules must be specially
derived for all binding operators, and henceforth will not be shown.

Figure~\ref{ZF-lemmas1} also shows rules for the subset and equality
relations (proof by extensionality), and rules about the empty set and the
power set operator.

Figure~\ref{ZF-lemmas2} presents rules for replacement and separation.
The rules for \ttindex{Replace} and \ttindex{RepFun} are much simpler than
comparable rules for {\tt PrimReplace} would be.  The principle of
separation is proved explicitly, although most proofs should use the
natural deduction rules for \ttindex{Collect}.  The elimination rule
\ttindex{CollectE} is equivalent to the two destruction rules
\ttindex{CollectD1} and \ttindex{CollectD2}, but each rule is suited to
particular circumstances.  Although too many rules can be confusing, there
is no reason to aim for a minimal set of rules.  See the file
\ttindexbold{ZF/zf.ML} for a complete listing.

Figure~\ref{ZF-lemmas3} presents rules for general union and intersection.
The empty intersection should be undefined.  We cannot have
$\bigcap(\emptyset)=V$ because $V$, the universal class, is not a set.  All
expressions denote something in {\ZF} set theory; the definition of
intersection implies $\bigcap(\emptyset)=\emptyset$, but this value is
arbitrary.  The rule \ttindexbold{InterI} must have a premise to exclude
the empty intersection.  Some of the laws governing intersections require
similar premises.


%%% upair.ML

\begin{figure}
\begin{ttbox}
\idx{pairing}      a:Upair(b,c) <-> (a=b | a=c)
\idx{UpairI1}      a : Upair(a,b)
\idx{UpairI2}      b : Upair(a,b)
\idx{UpairE}       [| a : Upair(b,c);  a = b ==> P;  a = c ==> P |] ==> P
\subcaption{Unordered pairs}

\idx{UnI1}         c : A ==> c : A Un B
\idx{UnI2}         c : B ==> c : A Un B
\idx{UnCI}         (~c : B ==> c : A) ==> c : A Un B
\idx{UnE}          [| c : A Un B;  c:A ==> P;  c:B ==> P |] ==> P

\idx{IntI}         [| c : A;  c : B |] ==> c : A Int B
\idx{IntD1}        c : A Int B ==> c : A
\idx{IntD2}        c : A Int B ==> c : B
\idx{IntE}         [| c : A Int B;  [| c:A; c:B |] ==> P |] ==> P

\idx{DiffI}        [| c : A;  ~ c : B |] ==> c : A - B
\idx{DiffD1}       c : A - B ==> c : A
\idx{DiffD2}       [| c : A - B;  c : B |] ==> P
\idx{DiffE}        [| c : A - B;  [| c:A; ~ c:B |] ==> P |] ==> P
\subcaption{Union, intersection, difference}
\end{ttbox}
\caption{Unordered pairs and their consequences} \label{ZF-upair1}
\end{figure}


\begin{figure}
\begin{ttbox}
\idx{consI1}       a : cons(a,B)
\idx{consI2}       a : B ==> a : cons(b,B)
\idx{consCI}       (~ a:B ==> a=b) ==> a: cons(b,B)
\idx{consE}        [| a : cons(b,A);  a=b ==> P;  a:A ==> P |] ==> P

\idx{singletonI}   a : \{a\}
\idx{singletonE}   [| a : \{b\}; a=b ==> P |] ==> P
\subcaption{Finite and singleton sets}

\idx{succI1}       i : succ(i)
\idx{succI2}       i : j ==> i : succ(j)
\idx{succCI}       (~ i:j ==> i=j) ==> i: succ(j)
\idx{succE}        [| i : succ(j);  i=j ==> P;  i:j ==> P |] ==> P
\idx{succ_neq_0}   [| succ(n)=0 |] ==> P
\idx{succ_inject}  succ(m) = succ(n) ==> m=n
\subcaption{The successor function}

\idx{the_equality}     [| P(a);  !!x. P(x) ==> x=a |] ==> (THE x. P(x)) = a
\idx{theI}             EX! x. P(x) ==> P(THE x. P(x))

\idx{if_P}             P ==> if(P,a,b) = a
\idx{if_not_P}        ~P ==> if(P,a,b) = b

\idx{mem_anti_sym}     [| a:b;  b:a |] ==> P
\idx{mem_anti_refl}    a:a ==> P
\subcaption{Descriptions; non-circularity}
\end{ttbox}
\caption{Finite sets and their consequences} \label{ZF-upair2}
\end{figure}


\subsection{Unordered pairs and finite sets}
Figure~\ref{ZF-upair1} presents the principle of unordered pairing, along
with its derived rules.  Binary union and intersection are defined in terms
of ordered pairs, and set difference is included for completeness.  The
rule \ttindexbold{UnCI} is useful for classical reasoning about unions,
like {\tt disjCI}\@; it supersedes \ttindexbold{UnI1} and
\ttindexbold{UnI2}, but these rules are often easier to work with.  For
intersection and difference we have both elimination and destruction rules.
Again, there is no reason to provide a minimal rule set.

Figure~\ref{ZF-upair2} is concerned with finite sets.  It presents rules
for~\ttindex{cons}, the finite set constructor, and rules for singleton
sets.  Because the successor function is defined in terms of~{\tt cons},
its derived rules appear here.

Definite descriptions (\ttindex{THE}) are defined in terms of the singleton
set $\{0\}$, but their derived rules fortunately hide this.  The
rule~\ttindex{theI} can be difficult to apply, because $\Var{P}$ must be
instantiated correctly.  However, \ttindex{the_equality} does not have this
problem and the files contain many examples of its use.

Finally, the impossibility of having both $a\in b$ and $b\in a$
(\ttindex{mem_anti_sym}) is proved by applying the axiom of foundation to
the set $\{a,b\}$.  The impossibility of $a\in a$ is a trivial consequence.

See the file \ttindexbold{ZF/upair.ML} for full details.


%%% subset.ML

\begin{figure}
\begin{ttbox}
\idx{Union_upper}       B:A ==> B <= Union(A)
\idx{Union_least}       [| !!x. x:A ==> x<=C |] ==> Union(A) <= C

\idx{Inter_lower}       B:A ==> Inter(A) <= B
\idx{Inter_greatest}    [| a:A;  !!x. x:A ==> C<=x |] ==> C <= Inter(A)

\idx{Un_upper1}         A <= A Un B
\idx{Un_upper2}         B <= A Un B
\idx{Un_least}          [| A<=C;  B<=C |] ==> A Un B <= C

\idx{Int_lower1}        A Int B <= A
\idx{Int_lower2}        A Int B <= B
\idx{Int_greatest}      [| C<=A;  C<=B |] ==> C <= A Int B

\idx{Diff_subset}       A-B <= A
\idx{Diff_contains}     [| C<=A;  C Int B = 0 |] ==> C <= A-B

\idx{Collect_subset}    Collect(A,P) <= A
\end{ttbox}
\caption{Subset and lattice properties} \label{ZF-subset}
\end{figure}


\subsection{Subset and lattice properties}
Figure~\ref{ZF-subset} shows that the subset relation is a complete
lattice.  Unions form least upper bounds; non-empty intersections form
greatest lower bounds.  A few other laws involving subsets are included.
See the file \ttindexbold{ZF/subset.ML}.

%%% pair.ML

\begin{figure}
\begin{ttbox}
\idx{Pair_inject1}    <a,b> = <c,d> ==> a=c
\idx{Pair_inject2}    <a,b> = <c,d> ==> b=d
\idx{Pair_inject}     [| <a,b> = <c,d>;  [| a=c; b=d |] ==> P |] ==> P
\idx{Pair_neq_0}      <a,b>=0 ==> P

\idx{fst}       fst(<a,b>) = a
\idx{snd}       snd(<a,b>) = b
\idx{split}     split(%x y.c(x,y), <a,b>) = c(a,b)

\idx{SigmaI}    [| a:A;  b:B(a) |] ==> <a,b> : Sigma(A,B)

\idx{SigmaE}    [| c: Sigma(A,B);  
             !!x y.[| x:A; y:B(x); c=<x,y> |] ==> P |] ==> P

\idx{SigmaE2}   [| <a,b> : Sigma(A,B);    
             [| a:A;  b:B(a) |] ==> P   |] ==> P
\end{ttbox}
\caption{Ordered pairs; projections; general sums} \label{ZF-pair}
\end{figure}


\subsection{Ordered pairs}
Figure~\ref{ZF-pair} presents the rules governing ordered pairs,
projections and general sums.  File \ttindexbold{ZF/pair.ML} contains the
full (and tedious) proof that $\{\{a\},\{a,b\}\}$ functions as an ordered
pair.  This property is expressed as two destruction rules,
\ttindexbold{Pair_inject1} and \ttindexbold{Pair_inject2}, and equivalently
as the elimination rule \ttindexbold{Pair_inject}.

The rule \ttindexbold{Pair_neq_0} asserts $\pair{a,b}\neq\emptyset$.  This
is a property of $\{\{a\},\{a,b\}\}$, and need not hold for other 
encoding of ordered pairs.  The non-standard ordered pairs mentioned below
satisfy $\pair{\emptyset;\emptyset}=\emptyset$.

The natural deduction rules \ttindexbold{SigmaI} and \ttindexbold{SigmaE}
assert that \ttindex{Sigma}$(A,B)$ consists of all pairs of the form
$\pair{x,y}$, for $x\in A$ and $y\in B(x)$.  The rule \ttindexbold{SigmaE2}
merely states that $\pair{a,b}\in {\tt Sigma}(A,B)$ implies $a\in A$ and
$b\in B(a)$.


%%% domrange.ML

\begin{figure}
\begin{ttbox}
\idx{domainI}        <a,b>: r ==> a : domain(r)
\idx{domainE}        [| a : domain(r);  !!y. <a,y>: r ==> P |] ==> P
\idx{domain_subset}  domain(Sigma(A,B)) <= A

\idx{rangeI}         <a,b>: r ==> b : range(r)
\idx{rangeE}         [| b : range(r);  !!x. <x,b>: r ==> P |] ==> P
\idx{range_subset}   range(A*B) <= B

\idx{fieldI1}        <a,b>: r ==> a : field(r)
\idx{fieldI2}        <a,b>: r ==> b : field(r)
\idx{fieldCI}        (~ <c,a>:r ==> <a,b>: r) ==> a : field(r)

\idx{fieldE}         [| a : field(r);  
                  !!x. <a,x>: r ==> P;  
                  !!x. <x,a>: r ==> P      
               |] ==> P

\idx{field_subset}   field(A*A) <= A
\subcaption{Domain, range and field of a Relation}

\idx{imageI}         [| <a,b>: r;  a:A |] ==> b : r``A
\idx{imageE}         [| b: r``A;  !!x.[| <x,b>: r;  x:A |] ==> P |] ==> P

\idx{vimageI}        [| <a,b>: r;  b:B |] ==> a : r-``B
\idx{vimageE}        [| a: r-``B;  !!x.[| <a,x>: r;  x:B |] ==> P |] ==> P
\subcaption{Image and inverse image}
\end{ttbox}
\caption{Relations} \label{ZF-domrange}
\end{figure}


\subsection{Relations}
Figure~\ref{ZF-domrange} presents rules involving relations, which are sets
of ordered pairs.  The converse of a relation~$r$ is the set of all pairs
$\pair{y,x}$ such that $\pair{x,y}\in r$; if $r$ is a function, then
{\ttindex{converse}$(r)$} is its inverse.  The rules for the domain
operation, \ttindex{domainI} and~\ttindex{domainE}, assert that
\ttindex{domain}$(r)$ consists of every element~$a$ such that $r$ contains
some pair of the form~$\pair{x,y}$.  The range operation is similar, and
the field of a relation is merely the union of its domain and range.  Note
that image and inverse image are generalizations of range and domain,
respectively.  See the file
\ttindexbold{ZF/domrange.ML} for derivations of the rules.


%%% func.ML

\begin{figure}
\begin{ttbox}
\idx{fun_is_rel}      f: Pi(A,B) ==> f <= Sigma(A,B)

\idx{apply_equality}  [| <a,b>: f;  f: Pi(A,B) |] ==> f`a = b
\idx{apply_equality2} [| <a,b>: f;  <a,c>: f;  f: Pi(A,B) |] ==> b=c

\idx{apply_type}      [| f: Pi(A,B);  a:A |] ==> f`a : B(a)
\idx{apply_Pair}      [| f: Pi(A,B);  a:A |] ==> <a,f`a>: f
\idx{apply_iff}       f: Pi(A,B) ==> <a,b>: f <-> a:A & f`a = b

\idx{fun_extension}   [| f : Pi(A,B);  g: Pi(A,D);
                   !!x. x:A ==> f`x = g`x     |] ==> f=g

\idx{domain_type}     [| <a,b> : f;  f: Pi(A,B) |] ==> a : A
\idx{range_type}      [| <a,b> : f;  f: Pi(A,B) |] ==> b : B(a)

\idx{Pi_type}         [| f: A->C;  !!x. x:A ==> f`x: B(x) |] ==> f: Pi(A,B)
\idx{domain_of_fun}   f: Pi(A,B) ==> domain(f)=A
\idx{range_of_fun}    f: Pi(A,B) ==> f: A->range(f)

\idx{restrict}   a : A ==> restrict(f,A) ` a = f`a
\idx{restrict_type}   [| !!x. x:A ==> f`x: B(x) |] ==> 
                restrict(f,A) : Pi(A,B)

\idx{lamI}       a:A ==> <a,b(a)> : (lam x:A. b(x))
\idx{lamE}       [| p: (lam x:A. b(x));  !!x.[| x:A; p=<x,b(x)> |] ==> P 
           |] ==>  P

\idx{lam_type}   [| !!x. x:A ==> b(x): B(x) |] ==> (lam x:A.b(x)) : Pi(A,B)

\idx{beta}       a : A ==> (lam x:A.b(x)) ` a = b(a)
\idx{eta}        f : Pi(A,B) ==> (lam x:A. f`x) = f

\idx{lam_theI}   (!!x. x:A ==> EX! y. Q(x,y)) ==> EX h. ALL x:A. Q(x, h`x)
\end{ttbox}
\caption{Functions and $\lambda$-abstraction} \label{ZF-func1}
\end{figure}


\begin{figure}
\begin{ttbox}
\idx{fun_empty}            0: 0->0
\idx{fun_single}           \{<a,b>\} : \{a\} -> \{b\}

\idx{fun_disjoint_Un}      [| f: A->B;  g: C->D;  A Int C = 0  |] ==>  
                     (f Un g) : (A Un C) -> (B Un D)

\idx{fun_disjoint_apply1}  [| a:A;  f: A->B;  g: C->D;  A Int C = 0 |] ==>  
                     (f Un g)`a = f`a

\idx{fun_disjoint_apply2}  [| c:C;  f: A->B;  g: C->D;  A Int C = 0 |] ==>  
                     (f Un g)`c = g`c
\end{ttbox}
\caption{Constructing functions from smaller sets} \label{ZF-func2}
\end{figure}


\subsection{Functions}
Functions, represented by graphs, are notoriously difficult to reason
about.  The file \ttindexbold{ZF/func.ML} derives many rules, which overlap
more than they ought.  One day these rules will be tidied up; this section
presents only the more important ones.

Figure~\ref{ZF-func1} presents the basic properties of \ttindex{Pi}$(A,B)$,
the generalized function space.  For example, if $f$ is a function and
$\pair{a,b}\in f$, then $f`a=b$ (\ttindex{apply_equality}).  Two functions
are equal provided they have equal domains and deliver equals results
(\ttindex{fun_extension}).

By \ttindex{Pi_type}, a function typing of the form $f\in A\to C$ can be
refined to the dependent typing $f\in\prod@{x\in A}B(x)$, given a suitable
family of sets $\{B(x)\}@{x\in A}$.  Conversely, by \ttindex{range_of_fun},
any dependent typing can be flattened to yield a function type of the form
$A\to C$; here, $C={\tt range}(f)$.

Among the laws for $\lambda$-abstraction, \ttindex{lamI} and \ttindex{lamE}
describe the graph of the generated function, while \ttindex{beta} and
\ttindex{eta} are the standard conversions.  We essentially have a
dependently-typed $\lambda$-calculus.

Figure~\ref{ZF-func2} presents some rules that can be used to construct
functions explicitly.  We start with functions consisting of at most one
pair, and may form the union of two functions provided their domains are
disjoint.  


\begin{figure} 
\begin{center}
\begin{tabular}{rrr} 
  \it name      &\it meta-type  & \it description \\ 
  \idx{id}      & $i$           & identity function \\
  \idx{inj}     & $[i,i]\To i$  & injective function space\\
  \idx{surj}    & $[i,i]\To i$  & surjective function space\\
  \idx{bij}     & $[i,i]\To i$  & bijective function space
        \\[1ex]
  \idx{1}       & $i$           & $\{\emptyset\}$       \\
  \idx{bool}    & $i$           & the set $\{\emptyset,1\}$     \\
  \idx{cond}    & $[i,i,i]\To i$        & conditional for {\tt bool}
        \\[1ex]
  \idx{Inl}~~\idx{Inr}  & $i\To i$      & injections\\
  \idx{case}    & $[i\To i,i\To i, i]\To i$      & conditional for $+$
        \\[1ex]
  \idx{nat}     & $i$           & set of natural numbers \\
  \idx{nat_case}& $[i,i\To i,i]\To i$   & conditional for $nat$\\
  \idx{rec}     & $[i,i,[i,i]\To i]\To i$ & recursor for $nat$
        \\[1ex]
  \idx{list}    & $i\To i$      & lists over some set\\
  \idx{list_case} & $[i, [i,i]\To i, i] \To i$  & conditional for $list(A)$ \\
  \idx{list_rec} & $[i, i, [i,i,i]\To i] \To i$ & recursor for $list(A)$ \\
  \idx{map}     & $[i\To i, i] \To i$   & mapping functional\\
  \idx{length}  & $i\To i$              & length of a list\\
  \idx{rev}     & $i\To i$              & reverse of a list\\
  \idx{flat}    & $i\To i$              & flatting a list of lists\\
\end{tabular}
\end{center}
\subcaption{Constants}

\begin{center}
\indexbold{*"+}
\index{#*@{\tt\#*}|bold}
\index{*div|bold}
\index{*mod|bold}
\index{#+@{\tt\#+}|bold}
\index{#-@{\tt\#-}|bold}
\begin{tabular}{rrrr} 
  \idx{O}       & $[i,i]\To i$  &  Right 60     & composition ($\circ$) \\
  \tt +         & $[i,i]\To i$  &  Right 65     & disjoint union \\
  \tt \#*       & $[i,i]\To i$  &  Left 70      & multiplication \\
  \tt div       & $[i,i]\To i$  &  Left 70      & division\\
  \tt mod       & $[i,i]\To i$  &  Left 70      & modulus\\
  \tt \#+       & $[i,i]\To i$  &  Left 65      & addition\\
  \tt \#-       & $[i,i]\To i$  &  Left 65      & subtraction\\
  \tt \@        & $[i,i]\To i$  &  Right 60     & append for lists
\end{tabular}
\end{center}
\subcaption{Infixes}
\caption{Further constants for {\ZF}} \label{ZF-further-constants}
\end{figure} 


\begin{figure}
\begin{ttbox}
\idx{Int_absorb}         A Int A = A
\idx{Int_commute}        A Int B = B Int A
\idx{Int_assoc}          (A Int B) Int C  =  A Int (B Int C)
\idx{Int_Un_distrib}     (A Un B) Int C  =  (A Int C) Un (B Int C)

\idx{Un_absorb}          A Un A = A
\idx{Un_commute}         A Un B = B Un A
\idx{Un_assoc}           (A Un B) Un C  =  A Un (B Un C)
\idx{Un_Int_distrib}     (A Int B) Un C  =  (A Un C) Int (B Un C)

\idx{Diff_cancel}        A-A = 0
\idx{Diff_disjoint}      A Int (B-A) = 0
\idx{Diff_partition}     A<=B ==> A Un (B-A) = B
\idx{double_complement}  [| A<=B; B<= C |] ==> (B - (C-A)) = A
\idx{Diff_Un}            A - (B Un C) = (A-B) Int (A-C)
\idx{Diff_Int}           A - (B Int C) = (A-B) Un (A-C)

\idx{Union_Un_distrib}   Union(A Un B) = Union(A) Un Union(B)
\idx{Inter_Un_distrib}   [| a:A;  b:B |] ==> 
                   Inter(A Un B) = Inter(A) Int Inter(B)

\idx{Int_Union_RepFun}   A Int Union(B) = (UN C:B. A Int C)

\idx{Un_Inter_RepFun}    b:B ==> 
                   A Un Inter(B) = (INT C:B. A Un C)

\idx{SUM_Un_distrib1}    (SUM x:A Un B. C(x)) = 
                   (SUM x:A. C(x)) Un (SUM x:B. C(x))

\idx{SUM_Un_distrib2}    (SUM x:C. A(x) Un B(x)) =
                   (SUM x:C. A(x))  Un  (SUM x:C. B(x))

\idx{SUM_Int_distrib1}   (SUM x:A Int B. C(x)) =
                   (SUM x:A. C(x)) Int (SUM x:B. C(x))

\idx{SUM_Int_distrib2}   (SUM x:C. A(x) Int B(x)) =
                   (SUM x:C. A(x)) Int (SUM x:C. B(x))
\end{ttbox}
\caption{Equalities} \label{zf-equalities}
\end{figure}


\begin{figure}
\begin{ttbox}
\idx{bnd_mono_def}   bnd_mono(D,h) == 
                 h(D)<=D & (ALL W X. W<=X --> X<=D --> h(W) <= h(X))

\idx{lfp_def}        lfp(D,h) == Inter({X: Pow(D). h(X) <= X})
\idx{gfp_def}        gfp(D,h) == Union({X: Pow(D). X <= h(X)})
\subcaption{Definitions}

\idx{lfp_lowerbound} [| h(A) <= A;  A<=D |] ==> lfp(D,h) <= A

\idx{lfp_subset}     lfp(D,h) <= D

\idx{lfp_greatest}   [| bnd_mono(D,h);  
                  !!X. [| h(X) <= X;  X<=D |] ==> A<=X 
               |] ==> A <= lfp(D,h)

\idx{lfp_Tarski}     bnd_mono(D,h) ==> lfp(D,h) = h(lfp(D,h))

\idx{induct}         [| a : lfp(D,h);  bnd_mono(D,h);
                  !!x. x : h(Collect(lfp(D,h),P)) ==> P(x)
               |] ==> P(a)

\idx{lfp_mono}       [| bnd_mono(D,h);  bnd_mono(E,i);
                  !!X. X<=D ==> h(X) <= i(X)  
               |] ==> lfp(D,h) <= lfp(E,i)

\idx{gfp_upperbound} [| A <= h(A);  A<=D |] ==> A <= gfp(D,h)

\idx{gfp_subset}     gfp(D,h) <= D

\idx{gfp_least}      [| bnd_mono(D,h);  
                  !!X. [| X <= h(X);  X<=D |] ==> X<=A
               |] ==> gfp(D,h) <= A

\idx{gfp_Tarski}     bnd_mono(D,h) ==> gfp(D,h) = h(gfp(D,h))

\idx{coinduct}       [| bnd_mono(D,h); a: X; X <= h(X Un gfp(D,h)); X <= D 
               |] ==> a : gfp(D,h)

\idx{gfp_mono}       [| bnd_mono(D,h);  D <= E;
                  !!X. X<=D ==> h(X) <= i(X)  
               |] ==> gfp(D,h) <= gfp(E,i)
\end{ttbox}
\caption{Least and greatest fixedpoints} \label{zf-fixedpt}
\end{figure}


\begin{figure}
\begin{ttbox}
\idx{comp_def}  r O s     == \{xz : domain(s)*range(r) . 
                        EX x y z. xz=<x,z> & <x,y>:s & <y,z>:r\}
\idx{id_def}    id(A)     == (lam x:A. x)
\idx{inj_def}   inj(A,B)  == \{ f: A->B. ALL w:A. ALL x:A. f`w=f`x --> w=x\}
\idx{surj_def}  surj(A,B) == \{ f: A->B . ALL y:B. EX x:A. f`x=y\}
\idx{bij_def}   bij(A,B)  == inj(A,B) Int surj(A,B)
\subcaption{Definitions}

\idx{left_inverse}     [| f: inj(A,B);  a: A |] ==> converse(f)`(f`a) = a
\idx{right_inverse}    [| f: inj(A,B);  b: range(f) |] ==> 
                 f`(converse(f)`b) = b

\idx{inj_converse_inj} f: inj(A,B) ==> converse(f): inj(range(f), A)
\idx{bij_converse_bij} f: bij(A,B) ==> converse(f): bij(B,A)

\idx{comp_type}        [| s<=A*B;  r<=B*C |] ==> (r O s) <= A*C
\idx{comp_assoc}       (r O s) O t = r O (s O t)

\idx{left_comp_id}     r<=A*B ==> id(B) O r = r
\idx{right_comp_id}    r<=A*B ==> r O id(A) = r

\idx{comp_func}        [| g:A->B; f:B->C |] ==> (f O g):A->C
\idx{comp_func_apply}  [| g:A->B; f:B->C; a:A |] ==> (f O g)`a = f`(g`a)

\idx{comp_inj}         [| g:inj(A,B);  f:inj(B,C)  |] ==> (f O g):inj(A,C)
\idx{comp_surj}        [| g:surj(A,B); f:surj(B,C) |] ==> (f O g):surj(A,C)
\idx{comp_bij}         [| g:bij(A,B); f:bij(B,C) |] ==> (f O g):bij(A,C)

\idx{left_comp_inverse}     f: inj(A,B) ==> converse(f) O f = id(A)
\idx{right_comp_inverse}    f: surj(A,B) ==> f O converse(f) = id(B)

\idx{bij_disjoint_Un}   
    [| f: bij(A,B);  g: bij(C,D);  A Int C = 0;  B Int D = 0 |] ==> 
    (f Un g) : bij(A Un C, B Un D)

\idx{restrict_bij}  [| f:inj(A,B);  C<=A |] ==> restrict(f,C): bij(C, f``C)
\end{ttbox}
\caption{Permutations} \label{zf-perm}
\end{figure}

\begin{figure}
\begin{ttbox}
\idx{one_def}        1    == succ(0)
\idx{bool_def}       bool == {0,1}
\idx{cond_def}       cond(b,c,d) == if(b=1,c,d)

\idx{sum_def}        A+B == \{0\}*A Un \{1\}*B
\idx{Inl_def}        Inl(a) == <0,a>
\idx{Inr_def}        Inr(b) == <1,b>
\idx{case_def}       case(c,d,u) == split(%y z. cond(y, d(z), c(z)), u)
\subcaption{Definitions}

\idx{bool_1I}        1 : bool
\idx{bool_0I}        0 : bool

\idx{boolE}          [| c: bool;  P(1);  P(0) |] ==> P(c)
\idx{cond_1}         cond(1,c,d) = c
\idx{cond_0}         cond(0,c,d) = d

\idx{sum_InlI}       a : A ==> Inl(a) : A+B
\idx{sum_InrI}       b : B ==> Inr(b) : A+B

\idx{Inl_inject}     Inl(a)=Inl(b) ==> a=b
\idx{Inr_inject}     Inr(a)=Inr(b) ==> a=b
\idx{Inl_neq_Inr}    Inl(a)=Inr(b) ==> P

\idx{sumE2}   u: A+B ==> (EX x. x:A & u=Inl(x)) | (EX y. y:B & u=Inr(y))

\idx{case_Inl}       case(c,d,Inl(a)) = c(a)
\idx{case_Inr}       case(c,d,Inr(b)) = d(b)
\end{ttbox}
\caption{Booleans and disjoint unions} \label{zf-sum}
\end{figure}

\begin{figure}
\begin{ttbox}
\idx{QPair_def}       <a;b> == a+b
\idx{qsplit_def}      qsplit(c,p)  == THE y. EX a b. p=<a;b> & y=c(a,b)
\idx{qfsplit_def}     qfsplit(R,z) == EX x y. z=<x;y> & R(x,y)
\idx{qconverse_def}   qconverse(r) == {z. w:r, EX x y. w=<x;y> & z=<y;x>}
\idx{QSigma_def}      QSigma(A,B)  ==  UN x:A. UN y:B(x). {<x;y>}

\idx{qsum_def}        A <+> B      == (\{0\} <*> A) Un (\{1\} <*> B)
\idx{QInl_def}        QInl(a)      == <0;a>
\idx{QInr_def}        QInr(b)      == <1;b>
\idx{qcase_def}       qcase(c,d)   == qsplit(%y z. cond(y, d(z), c(z)))
\end{ttbox}
\caption{Non-standard pairs, products and sums} \label{zf-qpair}
\end{figure}


\begin{figure}
\begin{ttbox}
\idx{nat_def}       nat == lfp(lam r: Pow(Inf). \{0\} Un \{succ(x). x:r\}

\idx{nat_case_def}  nat_case(a,b,k) == 
              THE y. k=0 & y=a | (EX x. k=succ(x) & y=b(x))

\idx{rec_def}       rec(k,a,b) ==  
              transrec(k, %n f. nat_case(a, %m. b(m, f`m), n))

\idx{add_def}       m#+n == rec(m, n, %u v.succ(v))
\idx{diff_def}      m#-n == rec(n, m, %u v. rec(v, 0, %x y.x))
\idx{mult_def}      m#*n == rec(m, 0, %u v. n #+ v)
\idx{mod_def}       m mod n == transrec(m, %j f. if(j:n, j, f`(j#-n)))
\idx{div_def}       m div n == transrec(m, %j f. if(j:n, 0, succ(f`(j#-n))))
\subcaption{Definitions}

\idx{nat_0I}        0 : nat
\idx{nat_succI}     n : nat ==> succ(n) : nat

\idx{nat_induct}        
    [| n: nat;  P(0);  !!x. [| x: nat;  P(x) |] ==> P(succ(x)) 
    |] ==> P(n)

\idx{nat_case_0}    nat_case(a,b,0) = a
\idx{nat_case_succ} nat_case(a,b,succ(m)) = b(m)

\idx{rec_0}         rec(0,a,b) = a
\idx{rec_succ}      rec(succ(m),a,b) = b(m, rec(m,a,b))

\idx{mult_type}     [| m:nat;  n:nat |] ==> m #* n : nat
\idx{mult_0}        0 #* n = 0
\idx{mult_succ}     succ(m) #* n = n #+ (m #* n)
\idx{mult_commute}  [| m:nat;  n:nat |] ==> m #* n = n #* m
\idx{add_mult_dist}
    [| m:nat;  k:nat |] ==> (m #+ n) #* k = (m #* k) #+ (n #* k)
\idx{mult_assoc}
    [| m:nat;  n:nat;  k:nat |] ==> (m #* n) #* k = m #* (n #* k)

\idx{mod_quo_equality}
    [| 0:n;  m:nat;  n:nat |] ==> (m div n)#*n #+ m mod n = m
\end{ttbox}
\caption{The natural numbers} \label{zf-nat}
\end{figure}

\begin{figure}
\begin{ttbox}
\idx{Fin_0I}          0 : Fin(A)
\idx{Fin_consI}       [| a: A;  b: Fin(A) |] ==> cons(a,b) : Fin(A)

\idx{Fin_induct}
    [| b: Fin(A);
       P(0);
       !!x y. [| x: A;  y: Fin(A);  x~:y;  P(y) |] ==> P(cons(x,y))
    |] ==> P(b)

\idx{Fin_mono}        A<=B ==> Fin(A) <= Fin(B)
\idx{Fin_UnI}         [| b: Fin(A);  c: Fin(A) |] ==> b Un c : Fin(A)
\idx{Fin_UnionI}      C : Fin(Fin(A)) ==> Union(C) : Fin(A)
\idx{Fin_subset}      [| c<=b;  b: Fin(A) |] ==> c: Fin(A)
\end{ttbox}
\caption{The finite set operator} \label{zf-fin}
\end{figure}

\begin{figure}\underscoreon %%because @ is used here
\begin{ttbox}
\idx{list_rec_def}    list_rec(l,c,h) == 
                Vrec(l, %l g.list_case(c, %x xs. h(x, xs, g`xs), l))

\idx{map_def}         map(f,l)  == list_rec(l,  0,  %x xs r. <f(x), r>)
\idx{length_def}      length(l) == list_rec(l,  0,  %x xs r. succ(r))
\idx{app_def}         xs@ys     == list_rec(xs, ys, %x xs r. <x,r>)
\idx{rev_def}         rev(l)    == list_rec(l,  0,  %x xs r. r @ <x,0>)
\idx{flat_def}        flat(ls)  == list_rec(ls, 0,  %l ls r. l @ r)
\subcaption{Definitions}

\idx{NilI}            Nil : list(A)
\idx{ConsI}           [| a: A;  l: list(A) |] ==> Cons(a,l) : list(A)

\idx{List.induct}
    [| l: list(A);
       P(Nil);
       !!x y. [| x: A;  y: list(A);  P(y) |] ==> P(Cons(x,y))
    |] ==> P(l)

\idx{Cons_iff}        Cons(a,l)=Cons(a',l') <-> a=a' & l=l'
\idx{Nil_Cons_iff}    ~ Nil=Cons(a,l)

\idx{list_mono}       A<=B ==> list(A) <= list(B)

\idx{list_rec_Nil}    list_rec(Nil,c,h) = c
\idx{list_rec_Cons}   list_rec(Cons(a,l), c, h) = h(a, l, list_rec(l,c,h))

\idx{map_ident}       l: list(A) ==> map(%u.u, l) = l
\idx{map_compose}     l: list(A) ==> map(h, map(j,l)) = map(%u.h(j(u)), l)
\idx{map_app_distrib} xs: list(A) ==> map(h, xs@ys) = map(h,xs) @ map(h,ys)
\idx{map_type}
    [| l: list(A);  !!x. x: A ==> h(x): B |] ==> map(h,l) : list(B)
\idx{map_flat}
    ls: list(list(A)) ==> map(h, flat(ls)) = flat(map(map(h),ls))
\end{ttbox}
\caption{Lists} \label{zf-list}
\end{figure}

\section{Further developments}
The next group of developments is complex and extensive, and only
highlights can be covered here.  Figure~\ref{ZF-further-constants} lists
some of the further constants and infixes that are declared in the various
theory extensions.

Figure~\ref{zf-equalities} presents commutative, associative, distributive,
and idempotency laws of union and intersection, along with other equations.
See file \ttindexbold{ZF/equalities.ML}.

Figure~\ref{zf-sum} defines $\{0,1\}$ as a set of booleans, with a
conditional operator.  It also defines the disjoint union of two sets, with
injections and a case analysis operator.  See files
\ttindexbold{ZF/bool.ML} and \ttindexbold{ZF/sum.ML}.

Figure~\ref{zf-qpair} defines a notion of ordered pair that admits
non-well-founded tupling.  Such pairs are written {\tt<$a$;$b$>}.  It also
defines the eliminator \ttindexbold{qsplit}, the converse operator
\ttindexbold{qconverse}, and the summation operator \ttindexbold{QSigma}.
These are completely analogous to the corresponding versions for standard
ordered pairs.  The theory goes on to define a non-standard notion of
disjoint sum using non-standard pairs.  This will support co-inductive
definitions, for example of infinite lists.  See file \ttindexbold{qpair.thy}.

Monotonicity properties of most of the set-forming operations are proved.
These are useful for applying the Knaster-Tarski Fixedpoint Theorem.
See file \ttindexbold{ZF/mono.ML}.

Figure~\ref{zf-fixedpt} presents the Knaster-Tarski Fixedpoint Theorem, proved
for the lattice of subsets of a set.  The theory defines least and greatest
fixedpoint operators with corresponding induction and co-induction rules.
Later definitions use these, such as the natural numbers and
the transitive closure operator.  The (co-)inductive definition
package also uses them.    See file \ttindexbold{ZF/fixedpt.ML}.

Figure~\ref{zf-perm} defines composition and injective, surjective and
bijective function spaces, with proofs of many of their properties.
See file \ttindexbold{ZF/perm.ML}.

Figure~\ref{zf-nat} presents the natural numbers, with induction and a
primitive recursion operator.  See file \ttindexbold{ZF/nat.ML}.  File
\ttindexbold{ZF/arith.ML} develops arithmetic on the natural numbers.  It
defines addition, multiplication, subtraction, division, and remainder,
proving the theorem $a \bmod b + (a/b)\times b = a$.  Division and
remainder are defined by repeated subtraction, which requires well-founded
rather than primitive recursion.

The file \ttindexbold{ZF/univ.ML} defines a ``universe'' ${\tt univ}(A)$,
for constructing datatypes such as trees.  This set contains $A$ and the
natural numbers.  Vitally, it is closed under finite products: ${\tt
  univ}(A)\times{\tt univ}(A)\subseteq{\tt univ}(A)$.  This file also
defines set theory's cumulative hierarchy, traditionally written $V@\alpha$
where $\alpha$ is any ordinal.

The file \ttindexbold{ZF/quniv.ML} defines a ``universe'' ${\tt quniv}(A)$,
for constructing co-datatypes such as streams.  It is analogous to ${\tt
  univ}(A)$ but is closed under the non-standard product and sum.

Figure~\ref{zf-fin} presents the finite set operator; ${\tt Fin}(A)$ is the
set of all finite sets over~$A$.  The definition employs Isabelle's
inductive definition package, which proves the introduction rules
automatically.  The induction rule shown is stronger than the one proved by
the package.  See file \ttindexbold{ZF/fin.ML}.

Figure~\ref{zf-list} presents the set of lists over~$A$, ${\tt list}(A)$.
The definition employs Isabelle's datatype package, which defines the
introduction and induction rules automatically, as well as the constructors
and case operator (\verb|list_case|).  See file \ttindexbold{ZF/list.ML}.
The file \ttindexbold{ZF/listfn.thy} proceeds to define structural
recursion and the usual list functions.

The constructions of the natural numbers and lists make use of a suite of
operators for handling recursive function definitions.  The developments are
summarized below:
\begin{description}
\item[\ttindexbold{ZF/trancl.ML}]
defines the transitive closure of a relation (as a least fixedpoint).

\item[\ttindexbold{ZF/wf.ML}]
proves the Well-Founded Recursion Theorem, using an elegant
approach of Tobias Nipkow.  This theorem permits general recursive
definitions within set theory.

\item[\ttindexbold{ZF/ord.ML}] defines the notions of transitive set and
  ordinal number.  It derives transfinite induction.  A key definition is
  {\bf less than}: $i<j$ if and only if $i$ and $j$ are both ordinals and
  $i\in j$.  As a special case, it includes less than on the natural
  numbers.

\item[\ttindexbold{ZF/epsilon.ML}]
derives $\epsilon$-induction and $\epsilon$-recursion, which are
generalizations of transfinite induction.  It also defines
\ttindexbold{rank}$(x)$, which is the least ordinal $\alpha$ such that $x$
is constructed at stage $\alpha$ of the cumulative hierarchy (thus $x\in
V@{\alpha+1}$).
\end{description}


\begin{figure}
\begin{eqnarray*}
  a\in \emptyset        & \bimp &  \bot\\
  a \in A \union B      & \bimp &  a\in A \disj a\in B\\
  a \in A \inter B      & \bimp &  a\in A \conj a\in B\\
  a \in A-B             & \bimp &  a\in A \conj \neg (a\in B)\\
  a \in {\tt cons}(b,B) & \bimp &  a=b \disj a\in B\\
  i \in {\tt succ}(j)   & \bimp &  i=j \disj i\in j\\
  \pair{a,b}\in {\tt Sigma}(A,B)
                        & \bimp &  a\in A \conj b\in B(a)\\
  a \in {\tt Collect}(A,P)      & \bimp &  a\in A \conj P(a)\\
  (\forall x \in A. \top)       & \bimp &  \top
\end{eqnarray*}
\caption{Rewrite rules for set theory} \label{ZF-simpdata}
\end{figure}


\section{Simplification rules}
{\ZF} does not merely inherit simplification from \FOL, but instantiates
the rewriting package new.  File \ttindexbold{ZF/simpdata.ML} contains the
details; here is a summary of the key differences:
\begin{itemize}
\item 
\ttindexbold{mk_rew_rules} is given as a function that can
strip bounded universal quantifiers from a formula.  For example, $\forall
x\in A.f(x)=g(x)$ yields the conditional rewrite rule $x\in A \Imp
f(x)=g(x)$.
\item
\ttindexbold{ZF_ss} contains congruence rules for all the operators of
{\ZF}, including the binding operators.  It contains all the conversion
rules, such as \ttindex{fst} and \ttindex{snd}, as well as the
rewrites shown in Figure~\ref{ZF-simpdata}.
\item
\ttindexbold{FOL_ss} is redeclared with the same {\FOL} rules as the
previous version, so that it may still be used.  
\end{itemize}


\section{The examples directory}
This directory contains further developments in {\ZF} set theory.  Here is
an overview; see the files themselves for more details.
\begin{description}
\item[\ttindexbold{ZF/ex/misc.ML}] contains miscellaneous examples such as
  Cantor's Theorem, the Schr\"oder-Bernstein Theorem.  and the
  ``Composition of homomorphisms'' challenge~\cite{boyer86}.

\item[\ttindexbold{ZF/ex/ramsey.ML}]
proves the finite exponent 2 version of Ramsey's Theorem, following Basin
and Kaufmann's presentation~\cite{basin91}.

\item[\ttindexbold{ZF/ex/equiv.ML}]
develops a ZF theory of equivalence classes, not using the Axiom of Choice.

\item[\ttindexbold{ZF/ex/integ.ML}]
develops a theory of the integers as equivalence classes of pairs of
natural numbers.

\item[\ttindexbold{ZF/ex/bin.ML}]
defines a datatype for two's complement binary integers.  File
\ttindexbold{ZF/ex/binfn.ML} then develops rewrite rules for binary
arithmetic.  For instance, $1359\times {-}2468 = {-}3354012$ takes under
14 seconds.

\item[\ttindexbold{ZF/ex/bt.ML}]
defines the recursive data structure ${\tt bt}(A)$, labelled binary trees.

\item[\ttindexbold{ZF/ex/term.ML}] 
  and \ttindexbold{ZF/ex/termfn.ML} define a recursive data structure for
  terms and term lists.  These are simply finite branching trees.

\item[\ttindexbold{ZF/ex/tf.ML}]
  and \ttindexbold{ZF/ex/tf_fn.ML} define primitives for solving mutually
  recursive equations over sets.  It constructs sets of trees and forests
  as an example, including induction and recursion rules that handle the
  mutual recursion.

\item[\ttindexbold{ZF/ex/prop.ML}]
  and \ttindexbold{ZF/ex/proplog.ML} proves soundness and completeness of
  propositional logic.  This illustrates datatype definitions, inductive
  definitions, structural induction and rule induction.

\item[\ttindexbold{ZF/ex/listn.ML}]
presents the inductive definition of the lists of $n$ elements~\cite{paulin92}.

\item[\ttindexbold{ZF/ex/acc.ML}]
presents the inductive definition of the accessible part of a
relation~\cite{paulin92}. 

\item[\ttindexbold{ZF/ex/comb.ML}]
  presents the datatype definition of combinators.  File
  \ttindexbold{ZF/ex/contract0.ML} defines contraction, while file
  \ttindexbold{ZF/ex/parcontract.ML} defines parallel contraction and
  proves the Church-Rosser Theorem.  This case study follows Camilleri and
  Melham~\cite{camilleri92}. 

\item[\ttindexbold{ZF/ex/llist.ML}]
  and \ttindexbold{ZF/ex/llist_eq.ML} develop lazy lists in ZF and a notion
  of co-induction for proving equations between them.
\end{description}


\section{A proof about powersets}
To demonstrate high-level reasoning about subsets, let us prove the
equation ${{\tt Pow}(A)\cap {\tt Pow}(B)}= {\tt Pow}(A\cap B)$.  Compared
with first-order logic, set theory involves a maze of rules, and theorems
have many different proofs.  Attempting other proofs of the theorem might
be instructive.  This proof exploits the lattice properties of
intersection.  It also uses the monotonicity of the powerset operation,
from {\tt ZF/mono.ML}:
\begin{ttbox}
\idx{Pow_mono}      A<=B ==> Pow(A) <= Pow(B)
\end{ttbox}
We enter the goal and make the first step, which breaks the equation into
two inclusions by extensionality:\index{equalityI}
\begin{ttbox}
goal ZF.thy "Pow(A Int B) = Pow(A) Int Pow(B)";
{\out Level 0}
{\out Pow(A Int B) = Pow(A) Int Pow(B)}
{\out  1. Pow(A Int B) = Pow(A) Int Pow(B)}
by (resolve_tac [equalityI] 1);
{\out Level 1}
{\out Pow(A Int B) = Pow(A) Int Pow(B)}
{\out  1. Pow(A Int B) <= Pow(A) Int Pow(B)}
{\out  2. Pow(A) Int Pow(B) <= Pow(A Int B)}
\end{ttbox}
Both inclusions could be tackled straightforwardly using {\tt subsetI}.
A shorter proof results from noting that intersection forms the greatest
lower bound:\index{*Int_greatest}
\begin{ttbox}
by (resolve_tac [Int_greatest] 1);
{\out Level 2}
{\out Pow(A Int B) = Pow(A) Int Pow(B)}
{\out  1. Pow(A Int B) <= Pow(A)}
{\out  2. Pow(A Int B) <= Pow(B)}
{\out  3. Pow(A) Int Pow(B) <= Pow(A Int B)}
\end{ttbox}
Subgoal~1 follows by applying the monotonicity of {\tt Pow} to $A\inter
B\subseteq A$; subgoal~2 follows similarly:
\index{*Int_lower1}\index{*Int_lower2}
\begin{ttbox}
by (resolve_tac [Int_lower1 RS Pow_mono] 1);
{\out Level 3}
{\out Pow(A Int B) = Pow(A) Int Pow(B)}
{\out  1. Pow(A Int B) <= Pow(B)}
{\out  2. Pow(A) Int Pow(B) <= Pow(A Int B)}
by (resolve_tac [Int_lower2 RS Pow_mono] 1);
{\out Level 4}
{\out Pow(A Int B) = Pow(A) Int Pow(B)}
{\out  1. Pow(A) Int Pow(B) <= Pow(A Int B)}
\end{ttbox}
We are left with the opposite inclusion, which we tackle in the
straightforward way:\index{*subsetI}
\begin{ttbox}
by (resolve_tac [subsetI] 1);
{\out Level 5}
{\out Pow(A Int B) = Pow(A) Int Pow(B)}
{\out  1. !!x. x : Pow(A) Int Pow(B) ==> x : Pow(A Int B)}
\end{ttbox}
The subgoal is to show $x\in {\tt Pow}(A\cap B)$ assuming $x\in{\tt
Pow}(A)\cap {\tt Pow}(B)$.  Eliminating this assumption produces two
subgoals, since intersection is like conjunction.\index{*IntE}
\begin{ttbox}
by (eresolve_tac [IntE] 1);
{\out Level 6}
{\out Pow(A Int B) = Pow(A) Int Pow(B)}
{\out  1. !!x. [| x : Pow(A); x : Pow(B) |] ==> x : Pow(A Int B)}
\end{ttbox}
The next step replaces the {\tt Pow} by the subset
relation~($\subseteq$).\index{*PowI}
\begin{ttbox}
by (resolve_tac [PowI] 1);
{\out Level 7}
{\out Pow(A Int B) = Pow(A) Int Pow(B)}
{\out  1. !!x. [| x : Pow(A); x : Pow(B) |] ==> x <= A Int B}
\end{ttbox}
We perform the same replacement in the assumptions:\index{*PowD}
\begin{ttbox}
by (REPEAT (dresolve_tac [PowD] 1));
{\out Level 8}
{\out Pow(A Int B) = Pow(A) Int Pow(B)}
{\out  1. !!x. [| x <= A; x <= B |] ==> x <= A Int B}
\end{ttbox}
Here, $x$ is a lower bound of $A$ and~$B$, but $A\inter B$ is the greatest
lower bound:\index{*Int_greatest}
\begin{ttbox}
by (resolve_tac [Int_greatest] 1);
{\out Level 9}
{\out Pow(A Int B) = Pow(A) Int Pow(B)}
{\out  1. !!x. [| x <= A; x <= B |] ==> x <= A}
{\out  2. !!x. [| x <= A; x <= B |] ==> x <= B}
by (REPEAT (assume_tac 1));
{\out Level 10}
{\out Pow(A Int B) = Pow(A) Int Pow(B)}
{\out No subgoals!}
\end{ttbox}
We could have performed this proof in one step by applying
\ttindex{fast_tac} with the classical rule set \ttindex{ZF_cs}.  But we
must add \ttindex{equalityI} as an introduction rule, since extensionality
is not used by default:
\begin{ttbox}
choplev 0;
{\out Level 0}
{\out Pow(A Int B) = Pow(A) Int Pow(B)}
{\out  1. Pow(A Int B) = Pow(A) Int Pow(B)}
by (fast_tac (ZF_cs addIs [equalityI]) 1);
{\out Level 1}
{\out Pow(A Int B) = Pow(A) Int Pow(B)}
{\out No subgoals!}
\end{ttbox}


\section{Monotonicity of the union operator}
For another example, we prove that general union is monotonic:
${C\subseteq D}$ implies $\bigcup(C)\subseteq \bigcup(D)$.  To begin, we
tackle the inclusion using \ttindex{subsetI}:
\begin{ttbox}
val [prem] = goal ZF.thy "C<=D ==> Union(C) <= Union(D)";
{\out Level 0}
{\out Union(C) <= Union(D)}
{\out  1. Union(C) <= Union(D)}
{\out val prem = "C <= D  [C <= D]" : thm}
\ttbreak
by (resolve_tac [subsetI] 1);
{\out Level 1}
{\out Union(C) <= Union(D)}
{\out  1. !!x. x : Union(C) ==> x : Union(D)}
\end{ttbox}
Big union is like an existential quantifier --- the occurrence in the
assumptions must be eliminated early, since it creates parameters.
\index{*UnionE}
\begin{ttbox}
by (eresolve_tac [UnionE] 1);
{\out Level 2}
{\out Union(C) <= Union(D)}
{\out  1. !!x B. [| x : B; B : C |] ==> x : Union(D)}
\end{ttbox}
Now we may apply \ttindex{UnionI}, which creates an unknown involving the
parameters.  To show $x\in \bigcup(D)$ it suffices to show that $x$ belongs
to some element, say~$\Var{B2}(x,B)$, of~$D$.
\begin{ttbox}
by (resolve_tac [UnionI] 1);
{\out Level 3}
{\out Union(C) <= Union(D)}
{\out  1. !!x B. [| x : B; B : C |] ==> ?B2(x,B) : D}
{\out  2. !!x B. [| x : B; B : C |] ==> x : ?B2(x,B)}
\end{ttbox}
Combining \ttindex{subsetD} with the premise $C\subseteq D$ yields 
$\Var{a}\in C \Imp \Var{a}\in D$, which reduces subgoal~1:
\begin{ttbox}
by (resolve_tac [prem RS subsetD] 1);
{\out Level 4}
{\out Union(C) <= Union(D)}
{\out  1. !!x B. [| x : B; B : C |] ==> ?B2(x,B) : C}
{\out  2. !!x B. [| x : B; B : C |] ==> x : ?B2(x,B)}
\end{ttbox}
The rest is routine.  Note how~$\Var{B2}(x,B)$ is instantiated.
\begin{ttbox}
by (assume_tac 1);
{\out Level 5}
{\out Union(C) <= Union(D)}
{\out  1. !!x B. [| x : B; B : C |] ==> x : B}
by (assume_tac 1);
{\out Level 6}
{\out Union(C) <= Union(D)}
{\out No subgoals!}
\end{ttbox}
Again, \ttindex{fast_tac} with \ttindex{ZF_cs} can do this proof in one
step, provided we somehow supply it with~{\tt prem}.  We can either add
this premise to the assumptions using \ttindex{cut_facts_tac}, or add
\hbox{\tt prem RS subsetD} to \ttindex{ZF_cs} as an introduction rule.

The file \ttindex{ZF/equalities.ML} has many similar proofs.
Reasoning about general intersection can be difficult because of its anomalous
behavior on the empty set.  However, \ttindex{fast_tac} copes well with
these.  Here is a typical example:
\begin{ttbox}
a:C ==> (INT x:C. A(x) Int B(x)) = (INT x:C.A(x)) Int (INT x:C.B(x))
\end{ttbox}
In traditional notation this is
\[ a\in C \,\Imp\, \bigcap@{x\in C} \Bigl(A(x) \inter B(x)\Bigr) =        
       \Bigl(\bigcap@{x\in C} A(x)\Bigr)  \inter  
       \Bigl(\bigcap@{x\in C} B(x)\Bigr)  \]

\section{Low-level reasoning about functions}
The derived rules {\tt lamI}, {\tt lamE}, {\tt lam_type}, {\tt beta}
and {\tt eta} support reasoning about functions in a
$\lambda$-calculus style.  This is generally easier than regarding
functions as sets of ordered pairs.  But sometimes we must look at the
underlying representation, as in the following proof
of~\ttindex{fun_disjoint_apply1}.  This states that if $f$ and~$g$ are
functions with disjoint domains~$A$ and~$C$, and if $a\in A$, then
$(f\union g)`a = f`a$:
\begin{ttbox}
val prems = goal ZF.thy
    "[| a:A;  f: A->B;  g: C->D;  A Int C = 0 |] ==>  \ttback
\ttback    (f Un g)`a = f`a";
{\out Level 0}
{\out (f Un g) ` a = f ` a}
{\out  1. (f Un g) ` a = f ` a}
\ttbreak
{\out val prems = ["a : A  [a : A]",}
{\out              "f : A -> B  [f : A -> B]",}
{\out              "g : C -> D  [g : C -> D]",}
{\out              "A Int C = 0  [A Int C = 0]"] : thm list}
\end{ttbox}
Using \ttindex{apply_equality}, we reduce the equality to reasoning about
ordered pairs.
\begin{ttbox}
by (resolve_tac [apply_equality] 1);
{\out Level 1}
{\out (f Un g) ` a = f ` a}
{\out  1. <a,f ` a> : f Un g}
{\out  2. f Un g : (PROD x:?A. ?B(x))}
\end{ttbox}
We must show that the pair belongs to~$f$ or~$g$; by~\ttindex{UnI1} we
choose~$f$:
\begin{ttbox}
by (resolve_tac [UnI1] 1);
{\out Level 2}
{\out (f Un g) ` a = f ` a}
{\out  1. <a,f ` a> : f}
{\out  2. f Un g : (PROD x:?A. ?B(x))}
\end{ttbox}
To show $\pair{a,f`a}\in f$ we use \ttindex{apply_Pair}, which is
essentially the converse of \ttindex{apply_equality}:
\begin{ttbox}
by (resolve_tac [apply_Pair] 1);
{\out Level 3}
{\out (f Un g) ` a = f ` a}
{\out  1. f : (PROD x:?A2. ?B2(x))}
{\out  2. a : ?A2}
{\out  3. f Un g : (PROD x:?A. ?B(x))}
\end{ttbox}
Using the premises $f\in A\to B$ and $a\in A$, we solve the two subgoals
from \ttindex{apply_Pair}.  Recall that a $\Pi$-set is merely a generalized
function space, and observe that~{\tt?A2} is instantiated to~{\tt A}.
\begin{ttbox}
by (resolve_tac prems 1);
{\out Level 4}
{\out (f Un g) ` a = f ` a}
{\out  1. a : A}
{\out  2. f Un g : (PROD x:?A. ?B(x))}
by (resolve_tac prems 1);
{\out Level 5}
{\out (f Un g) ` a = f ` a}
{\out  1. f Un g : (PROD x:?A. ?B(x))}
\end{ttbox}
To construct functions of the form $f\union g$, we apply
\ttindex{fun_disjoint_Un}:
\begin{ttbox}
by (resolve_tac [fun_disjoint_Un] 1);
{\out Level 6}
{\out (f Un g) ` a = f ` a}
{\out  1. f : ?A3 -> ?B3}
{\out  2. g : ?C3 -> ?D3}
{\out  3. ?A3 Int ?C3 = 0}
\end{ttbox}
The remaining subgoals are instances of the premises.  Again, observe how
unknowns are instantiated:
\begin{ttbox}
by (resolve_tac prems 1);
{\out Level 7}
{\out (f Un g) ` a = f ` a}
{\out  1. g : ?C3 -> ?D3}
{\out  2. A Int ?C3 = 0}
by (resolve_tac prems 1);
{\out Level 8}
{\out (f Un g) ` a = f ` a}
{\out  1. A Int C = 0}
by (resolve_tac prems 1);
{\out Level 9}
{\out (f Un g) ` a = f ` a}
{\out No subgoals!}
\end{ttbox}
See the files \ttindex{ZF/func.ML} and \ttindex{ZF/wf.ML} for more
examples of reasoning about functions.
