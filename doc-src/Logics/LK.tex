%% $Id$
\chapter{First-Order Sequent Calculus}
The directory~\ttindexbold{LK} implements classical first-order logic through
Gentzen's sequent calculus (see Gallier~\cite{gallier86} or
Takeuti~\cite{takeuti87}).  Resembling the method of semantic tableaux, the
calculus is well suited for backwards proof.  Assertions have the form
\(\Gamma\turn \Delta\), where \(\Gamma\) and \(\Delta\) are lists of
formulae.  Associative unification, simulated by higher-order unification,
handles lists.

The logic is many-sorted, using Isabelle's type classes.  The
class of first-order terms is called {\it term}.  No types of individuals
are provided, but extensions can define types such as $nat::term$ and type
constructors such as $list::(term)term$.  Below, the type variable $\alpha$
ranges over class {\it term\/}; the equality symbol and quantifiers are
polymorphic (many-sorted).  The type of formulae is~{\it o}, which belongs
to class {\it logic}.  

No generic packages are instantiated, since Isabelle does not provide
packages for sequent calculi at present.  \LK{} implements a classical
logic theorem prover that is as powerful as the generic classical module,
except that it does not perform equality reasoning.


\section{Unification for lists}
Higher-order unification includes associative unification as a special
case, by an encoding that involves function composition
\cite[page~37]{huet78}.  To represent lists, let $C$ be a new constant.
The empty list is $\lambda x.x$, while $[t@1,t@2,\ldots,t@n]$ is
represented by
\[ \lambda x.C(t@1,C(t@2,\ldots,C(t@n,x))).  \]
The unifiers of this with $\lambda x.\Var{f}(\Var{g}(x))$ give all the ways
of expressing $[t@1,t@2,\ldots,t@n]$ as the concatenation of two lists.

Unlike orthodox associative unification, this technique can represent certain
infinite sets of unifiers by flex-flex equations.   But note that the term
$\lambda x.C(t,\Var{a})$ does not represent any list.  Flex-flex equations
containing such garbage terms may accumulate during a proof.

This technique lets Isabelle formalize sequent calculus rules,
where the comma is the associative operator:
\[ \infer[\conj\hbox{-left}]
         {\Gamma,P\conj Q,\Delta \turn \Theta}
         {\Gamma,P,Q,\Delta \turn \Theta}  \] 
Multiple unifiers occur whenever this is resolved against a goal containing
more than one conjunction on the left.  Explicit formalization of sequents
can be tiresome, but gives precise control over contraction and weakening,
needed to handle relevant and linear logics.

\LK{} exploits this representation of lists.  As an alternative, the
sequent calculus can be formalized using an ordinary representation of
lists, with a logic program for removing a formula from a list.  Amy Felty
has applied this technique using the language $\lambda$Prolog~\cite{felty91a}.


\begin{figure} 
\begin{center}
\begin{tabular}{rrr} 
  \it name      &\it meta-type          & \it description       \\ 
  \idx{Trueprop}& $o\To prop$           & coercion to $prop$    \\
  \idx{Seqof}   & $[o,sobj]\To sobj$    & singleton sequence    \\
  \idx{Not}     & $o\To o$              & negation ($\neg$)     \\
  \idx{True}    & $o$                   & tautology ($\top$)    \\
  \idx{False}   & $o$                   & absurdity ($\bot$)
\end{tabular}
\end{center}
\subcaption{Constants}

\begin{center}
\begin{tabular}{llrrr} 
  \it symbol &\it name     &\it meta-type & \it precedence & \it description \\
  \idx{ALL}  & \idx{All}  & $(\alpha\To o)\To o$ & 10 & 
        universal quantifier ($\forall$) \\
  \idx{EX}   & \idx{Ex}   & $(\alpha\To o)\To o$ & 10 & 
        existential quantifier ($\exists$) \\
  \idx{THE} & \idx{The}  & $(\alpha\To o)\To \alpha$ & 10 & 
        definite description ($\iota$)
\end{tabular}
\end{center}
\subcaption{Binders} 

\begin{center}
\indexbold{*"=}
\indexbold{&@{\tt\&}}
\indexbold{*"|}
\indexbold{*"-"-">}
\indexbold{*"<"-">}
\begin{tabular}{rrrr} 
    \it symbol  & \it meta-type & \it precedence & \it description \\ 
    \tt = &     $[\alpha,\alpha]\To o$   & Left 50 & equality ($=$) \\
    \tt \& &    $[o,o]\To o$ & Right 35 & conjunction ($\conj$) \\
    \tt | &     $[o,o]\To o$ & Right 30 & disjunction ($\disj$) \\
    \tt --> &   $[o,o]\To o$ & Right 25 & implication ($\imp$) \\
    \tt <-> &   $[o,o]\To o$ & Right 25 & biconditional ($\bimp$) 
\end{tabular}
\end{center}
\subcaption{Infixes}

\begin{center}
\begin{tabular}{rrr} 
  \it external          & \it internal  & \it description \\ 
  \tt $\Gamma$ |- $\Delta$  &  \tt Trueprop($\Gamma$, $\Delta$) &
        sequent $\Gamma\turn \Delta$ 
\end{tabular}
\end{center}
\subcaption{Translations} 
\caption{Syntax of {\tt LK}} \label{lk-syntax}
\end{figure}


\begin{figure} 
\dquotes
\[\begin{array}{rcl}
    prop & = & sequence " |- " sequence 
\\[2ex]
sequence & = & elem \quad (", " elem)^* \\
         & | & empty 
\\[2ex]
    elem & = & "\$ " id \\
         & | & "\$ " var \\
         & | & formula 
\\[2ex]
 formula & = & \hbox{expression of type~$o$} \\
         & | & term " = " term \\
         & | & "\ttilde\ " formula \\
         & | & formula " \& " formula \\
         & | & formula " | " formula \\
         & | & formula " --> " formula \\
         & | & formula " <-> " formula \\
         & | & "ALL~" id~id^* " . " formula \\
         & | & "EX~~" id~id^* " . " formula \\
         & | & "THE~" id~     " . " formula
  \end{array}
\]
\caption{Grammar of {\tt LK}} \label{lk-grammar}
\end{figure}


\begin{figure} 
\begin{ttbox}
\idx{basic}       $H, P, $G |- $E, P, $F
\idx{thinR}       $H |- $E, $F ==> $H |- $E, P, $F
\idx{thinL}       $H, $G |- $E ==> $H, P, $G |- $E
\idx{cut}         [| $H |- $E, P;  $H, P |- $E |] ==> $H |- $E
\subcaption{Structural rules}

\idx{refl}        $H |- $E, a=a, $F
\idx{sym}         $H |- $E, a=b, $F ==> $H |- $E, b=a, $F
\idx{trans}       [| $H|- $E, a=b, $F;  $H|- $E, b=c, $F |] ==> 
            $H|- $E, a=c, $F
\subcaption{Equality rules}

\idx{True_def}    True  == False-->False
\idx{iff_def}     P<->Q == (P-->Q) & (Q-->P)

\idx{conjR}   [| $H|- $E, P, $F;  $H|- $E, Q, $F |] ==> $H|- $E, P&Q, $F
\idx{conjL}   $H, P, Q, $G |- $E ==> $H, P & Q, $G |- $E

\idx{disjR}   $H |- $E, P, Q, $F ==> $H |- $E, P|Q, $F
\idx{disjL}   [| $H, P, $G |- $E;  $H, Q, $G |- $E |] ==> $H, P|Q, $G |- $E
            
\idx{impR}    $H, P |- $E, Q, $F ==> $H |- $E, P-->Q, $
\idx{impL}    [| $H,$G |- $E,P;  $H, Q, $G |- $E |] ==> $H, P-->Q, $G |- $E
            
\idx{notR}    $H, P |- $E, $F ==> $H |- $E, ~P, $F
\idx{notL}    $H, $G |- $E, P ==> $H, ~P, $G |- $E

\idx{FalseL}  $H, False, $G |- $E
\end{ttbox}
\subcaption{Propositional rules}
\caption{Rules of {\tt LK}}  \label{lk-rules}
\end{figure}

\begin{figure} 
\begin{ttbox}
\idx{allR}    (!!x.$H|- $E, P(x), $F) ==> $H|- $E, ALL x.P(x), $F
\idx{allL}    $H, P(x), $G, ALL x.P(x) |- $E ==> $H, ALL x.P(x), $G|- $E
            
\idx{exR}     $H|- $E, P(x), $F, EX x.P(x) ==> $H|- $E, EX x.P(x), $F
\idx{exL}     (!!x.$H, P(x), $G|- $E) ==> $H, EX x.P(x), $G|- $E

\idx{The}     [| $H |- $E, P(a), $F;  !!x.$H, P(x) |- $E, x=a, $F |] ==>
        $H |- $E, P(THE x.P(x)), $F
\end{ttbox}
\subcaption{Quantifier rules}

\caption{Rules of {\tt LK} (continued)}  \label{lk-rules2}
\end{figure}


\begin{figure} 
\begin{ttbox}
\idx{conR}        $H |- $E, P, $F, P ==> $H |- $E, P, $F
\idx{conL}        $H, P, $G, P |- $E ==> $H, P, $G |- $E

\idx{symL}        $H, $G, B = A |- $E ==> $H, A = B, $G |- $E

\idx{TrueR}       $H |- $E, True, $F

\idx{iffR}        [| $H, P |- $E, Q, $F;  $H, Q |- $E, P, $F |] ==> 
            $H |- $E, P<->Q, $F

\idx{iffL}        [| $H, $G |- $E, P, Q;  $H, Q, P, $G |- $E |] ==>
            $H, P<->Q, $G |- $E

\idx{allL_thin}   $H, P(x), $G |- $E ==> $H, ALL x.P(x), $G |- $E
\idx{exR_thin}    $H |- $E, P(x), $F ==> $H |- $E, EX x.P(x), $F
\end{ttbox}

\caption{Derived rules for {\tt LK}} \label{lk-derived}
\end{figure}


\section{Syntax and rules of inference}
Figure~\ref{lk-syntax} gives the syntax for {\tt LK}, which is complicated
by the representation of sequents.  Type $sobj\To sobj$ represents a list
of formulae.

The {\bf definite description} operator~$\iota x.P(x)$ stands for the
unique~$a$ satisfying~$P(a)$, if such exists.  Since all terms in \LK{}
denote something, a description is always meaningful, but we do not know
its value unless $P[x]$ defines it uniquely.  The Isabelle notation is
\hbox{\tt THE $x$.$P[x]$}.

Traditionally, \(\Gamma\) and \(\Delta\) are meta-variables for sequences.
In Isabelle's notation, the prefix~\verb|$| on a variable makes it range
over sequences.  In a sequent, anything not prefixed by \verb|$| is taken
as a formula.

The theory has the \ML\ identifier \ttindexbold{LK.thy}.
Figures~\ref{lk-rules} and~\ref{lk-rules2} present the rules.  The
connective $\bimp$ is defined using $\conj$ and $\imp$.  The axiom for
basic sequents is expressed in a form that provides automatic thinning:
redundant formulae are simply ignored.  The other rules are expressed in
the form most suitable for backward proof --- they do not require exchange
or contraction rules.  The contraction rules are actually derivable (via
cut) in this formulation.

Figure~\ref{lk-derived} presents derived rules, including rules for
$\bimp$.  The weakened quantifier rules discard each quantification after a
single use; in an automatic proof procedure, they guarantee termination,
but are incomplete.  Multiple use of a quantifier can be obtained by a
contraction rule, which in backward proof duplicates a formula.  The tactic
\ttindex{res_inst_tac} can instantiate the variable~{\tt?P} in these rules,
specifying the formula to duplicate.

See the files {\tt LK/lk.thy} and {\tt LK/lk.ML}
for complete listings of the rules and derived rules.


\section{Tactics for the cut rule}
According to the cut-elimination theorem, the cut rule can be eliminated
from proofs of sequents.  But the rule is still essential.  It can be used
to structure a proof into lemmas, avoiding repeated proofs of the same
formula.  More importantly, the cut rule can not be eliminated from
derivations of rules.  For example, there is a trivial cut-free proof of
the sequent \(P\conj Q\turn Q\conj P\).
Noting this, we might want to derive a rule for swapping the conjuncts
in a right-hand formula:
\[ \Gamma\turn \Delta, P\conj Q\over \Gamma\turn \Delta, Q\conj P \]
The cut rule must be used, for $P\conj Q$ is not a subformula of $Q\conj
P$.  Most cuts directly involve a premise of the rule being derived (a
meta-assumption).  In a few cases, the cut formula is not part of any
premise, but serves as a bridge between the premises and the conclusion.
In such proofs, the cut formula is specified by calling an appropriate
tactic.

\begin{ttbox} 
cutR_tac : string -> int -> tactic
cutL_tac : string -> int -> tactic
\end{ttbox}
These tactics refine a subgoal into two by applying the cut rule.  The cut
formula is given as a string, and replaces some other formula in the sequent.
\begin{description}
\item[\ttindexbold{cutR_tac} {\it P\/} {\it i}] 
reads an \LK{} formula~$P$, and applies the cut rule to subgoal~$i$.  It
then deletes some formula from the right side of subgoal~$i$, replacing
that formula by~$P$.

\item[\ttindexbold{cutL_tac} {\it P\/} {\it i}] 
reads an \LK{} formula~$P$, and applies the cut rule to subgoal~$i$.  It
then deletes some formula from the left side of the new subgoal $i+1$,
replacing that formula by~$P$.
\end{description}
All the structural rules --- cut, contraction, and thinning --- can be
applied to particular formulae using \ttindex{res_inst_tac}.


\section{Tactics for sequents}
\begin{ttbox} 
forms_of_seq       : term -> term list
could_res          : term * term -> bool
could_resolve_seq  : term * term -> bool
filseq_resolve_tac : thm list -> int -> int -> tactic
\end{ttbox}
Associative unification is not as efficient as it might be, in part because
the representation of lists defeats some of Isabelle's internal
optimizations.  The following operations implement faster rule application,
and may have other uses.
\begin{description}
\item[\ttindexbold{forms_of_seq} {\it t}] 
returns the list of all formulae in the sequent~$t$, removing sequence
variables.

\item[\ttindexbold{could_res} $(t,u)$] 
tests whether two formula lists could be resolved.  List $t$ is from a
premise (subgoal), while $u$ is from the conclusion of an object-rule.
Assuming that each formula in $u$ is surrounded by sequence variables, it
checks that each conclusion formula is unifiable (using {\tt could_unify})
with some subgoal formula.

\item[\ttindexbold{could_resolve_seq} $(t,u)$] 
  tests whether two sequents could be resolved.  Sequent $t$ is a premise
  (subgoal), while $u$ is the conclusion of an object-rule.  It simply
  calls {\tt could_res} twice to check that both the left and the right
  sides of the sequents are compatible.

\item[\ttindexbold{filseq_resolve_tac} {\it thms} {\it maxr} {\it i}] 
uses {\tt filter_thms could_resolve} to extract the {\it thms} that are
applicable to subgoal~$i$.  If more than {\it maxr\/} theorems are
applicable then the tactic fails.  Otherwise it calls {\tt resolve_tac}.
Thus, it is the sequent calculus analogue of \ttindex{filt_resolve_tac}.
\end{description}



\section{Packaging sequent rules}
For theorem proving, rules can be classified as {\bf safe} or {\bf unsafe}.
An unsafe rule may reduce a provable goal to an unprovable set of subgoals,
and should only be used as a last resort.  Typical examples are the
weakened quantifier rules {\tt allL_thin} and {\tt exR_thin}.

A {\bf pack} is a pair whose first component is a list of safe
rules, and whose second is a list of unsafe rules.  Packs can be extended
in an obvious way to allow reasoning with various collections of rules.
For clarity, \LK{} declares the datatype
\ttindexbold{pack}:
\begin{ttbox}
datatype pack = Pack of thm list * thm list;
\end{ttbox}
The contents of any pack can be inspected by pattern-matching.  Packs
support the following operations:
\begin{ttbox} 
empty_pack  : pack
prop_pack   : pack
LK_pack     : pack
LK_dup_pack : pack
add_safes   : pack * thm list -> pack               \hfill{\bf infix 4}
add_unsafes : pack * thm list -> pack               \hfill{\bf infix 4}
\end{ttbox}
\begin{description}
\item[\ttindexbold{empty_pack}] is the empty pack.

\item[\ttindexbold{prop_pack}] contains the propositional rules, namely
those for $\conj$, $\disj$, $\neg$, $\imp$ and~$\bimp$, along with the
rules \ttindex{basic} and \ttindex{refl}.  These are all safe.

\item[\ttindexbold{LK_pack}] 
extends {\tt prop_pack} with the safe rules \ttindex{allR}
and~\ttindex{exL} and the unsafe rules \ttindex{allL_thin} and
\ttindex{exR_thin}.  Search using this is incomplete since quantified
formulae are used at most once.

\item[\ttindexbold{LK_dup_pack}] 
extends {\tt prop_pack} with the safe rules \ttindex{allR}
and~\ttindex{exL} and the unsafe rules \ttindex{allL} and~\ttindex{exR}.
Search using this is complete, since quantified formulae may be reused, but
frequently fails to terminate.  It is generally unsuitable for depth-first
search.

\item[$pack$ \ttindexbold{add_safes} $rules$] 
adds some safe~$rules$ to the pack~$pack$.

\item[$pack$ \ttindexbold{add_unsafes} $rules$] 
adds some unsafe~$rules$ to the pack~$pack$.
\end{description}


\section{Proof procedures}
The \LK{} proof procedure is similar to the generic classical module
described in the {\em Reference Manual}.  In fact it is simpler, since it
works directly with sequents rather than simulating them.  There is no need
to distinguish introduction rules from elimination rules, and of course
there is no swap rule.  As always, Isabelle's classical proof procedures
are less powerful than resolution theorem provers.  But they are more
natural and flexible, working with an open-ended set of rules.

Backtracking over the choice of a safe rule accomplishes nothing: applying
them in any order leads to essentially the same result.  Backtracking may
be necessary over basic sequents when they perform unification.  Suppose
that~0, 1, 2,~3 are constants in the subgoals
\[  \begin{array}{c}
      P(0), P(1), P(2) \turn P(\Var{a})  \\
      P(0), P(2), P(3) \turn P(\Var{a})  \\
      P(1), P(3), P(2) \turn P(\Var{a})  
    \end{array}
\]
The only assignment that satisfies all three subgoals is $\Var{a}\mapsto 2$,
and this can only be discovered by search.  The tactics given below permit
backtracking only over axioms, such as {\tt basic} and {\tt refl}.


\subsection{Method A}
\begin{ttbox} 
reresolve_tac   : thm list -> int -> tactic
repeat_goal_tac : pack -> int -> tactic
pc_tac          : pack -> int -> tactic
\end{ttbox}
These tactics use a method developed by Philippe de Groote.  A subgoal is
refined and the resulting subgoals are attempted in reverse order.  For
some reason, this is much faster than attempting the subgoals in order.
The method is inherently depth-first.

At present, these tactics only work for rules that have no more than two
premises.  They {\bf fail} if they can do nothing.
\begin{description}
\item[\ttindexbold{reresolve_tac} $thms$ $i$] 
repeatedly applies the $thms$ to subgoal $i$ and the resulting subgoals.

\item[\ttindexbold{repeat_goal_tac} $pack$ $i$] 
applies the safe rules in the pack to a goal and the resulting subgoals.
If no safe rule is applicable then it applies an unsafe rule and continues.

\item[\ttindexbold{pc_tac} $pack$ $i$] 
applies {\tt repeat_goal_tac} using depth-first search to solve subgoal~$i$.
\end{description}


\subsection{Method B}
\begin{ttbox} 
safe_goal_tac : pack -> int -> tactic
step_tac      : pack -> int -> tactic
fast_tac      : pack -> int -> tactic
best_tac      : pack -> int -> tactic
\end{ttbox}
These tactics are precisely analogous to those of the generic classical
module.  They use `Method~A' only on safe rules.  They {\bf fail} if they
can do nothing.
\begin{description}
\item[\ttindexbold{safe_goal_tac} $pack$ $i$] 
applies the safe rules in the pack to a goal and the resulting subgoals.
It ignores the unsafe rules.  

\item[\ttindexbold{step_tac} $pack$ $i$] 
either applies safe rules (using {\tt safe_goal_tac}) or applies one unsafe
rule.

\item[\ttindexbold{fast_tac} $pack$ $i$] 
applies {\tt step_tac} using depth-first search to solve subgoal~$i$.
Despite the names, {\tt pc_tac} is frequently faster.

\item[\ttindexbold{best_tac} $pack$ $i$] 
applies {\tt step_tac} using best-first search to solve subgoal~$i$.  It is
particularly useful for quantifier duplication (using \ttindex{LK_dup_pack}).
\end{description}



\section{A simple example of classical reasoning} 
The theorem $\turn\ex{y}\all{x}P(y)\imp P(x)$ is a standard example of the
classical treatment of the existential quantifier.  Classical reasoning
is easy using~{\LK}, as you can see by comparing this proof with the one
given in~\S\ref{fol-cla-example}.  From a logical point of view, the
proofs are essentially the same; the key step here is to use \ttindex{exR}
rather than the weaker~\ttindex{exR_thin}.
\begin{ttbox}
goal LK.thy "|- EX y. ALL x. P(y)-->P(x)";
{\out Level 0}
{\out  |- EX y. ALL x. P(y) --> P(x)}
{\out  1.  |- EX y. ALL x. P(y) --> P(x)}
by (resolve_tac [exR] 1);
{\out Level 1}
{\out  |- EX y. ALL x. P(y) --> P(x)}
{\out  1.  |- ALL x. P(?x) --> P(x), EX x. ALL xa. P(x) --> P(xa)}
\end{ttbox}
There are now two formulae on the right side.  Keeping the existential one
in reserve, we break down the universal one.
\begin{ttbox}
by (resolve_tac [allR] 1);
{\out Level 2}
{\out  |- EX y. ALL x. P(y) --> P(x)}
{\out  1. !!x.  |- P(?x) --> P(x), EX x. ALL xa. P(x) --> P(xa)}
by (resolve_tac [impR] 1);
{\out Level 3}
{\out  |- EX y. ALL x. P(y) --> P(x)}
{\out  1. !!x. P(?x) |- P(x), EX x. ALL xa. P(x) --> P(xa)}
\end{ttbox}
Because {\LK} is a sequent calculus, the formula~$P(\Var{x})$ does not
become an assumption;  instead, it moves to the left side.  The
resulting subgoal cannot be instantiated to a basic sequent: the bound
variable~$x$ is not unifiable with the unknown~$\Var{x}$.
\begin{ttbox}
by (resolve_tac [basic] 1);
{\out by: tactic failed}
\end{ttbox}
We reuse the existential formula using~\ttindex{exR_thin}, which discards
it; we will not need it a third time.  We again break down the resulting
formula.
\begin{ttbox}
by (resolve_tac [exR_thin] 1);
{\out Level 4}
{\out  |- EX y. ALL x. P(y) --> P(x)}
{\out  1. !!x. P(?x) |- P(x), ALL xa. P(?x7(x)) --> P(xa)}
by (resolve_tac [allR] 1);
{\out Level 5}
{\out  |- EX y. ALL x. P(y) --> P(x)}
{\out  1. !!x xa. P(?x) |- P(x), P(?x7(x)) --> P(xa)}
by (resolve_tac [impR] 1);
{\out Level 6}
{\out  |- EX y. ALL x. P(y) --> P(x)}
{\out  1. !!x xa. P(?x), P(?x7(x)) |- P(x), P(xa)}
\end{ttbox}
Subgoal~1 seems to offer lots of possibilities.  Actually the only useful
step is instantiating~$\Var{x@7}$ to $\lambda x.x$,
transforming~$\Var{x@7}(x)$ into~$x$.
\begin{ttbox}
by (resolve_tac [basic] 1);
{\out Level 7}
{\out  |- EX y. ALL x. P(y) --> P(x)}
{\out No subgoals!}
\end{ttbox}
This theorem can be proved automatically.  Because it involves quantifier
duplication, we employ best-first search:
\begin{ttbox}
goal LK.thy "|- EX y. ALL x. P(y)-->P(x)";
{\out Level 0}
{\out  |- EX y. ALL x. P(y) --> P(x)}
{\out  1.  |- EX y. ALL x. P(y) --> P(x)}
by (best_tac LK_dup_pack 1);
{\out Level 1}
{\out  |- EX y. ALL x. P(y) --> P(x)}
{\out No subgoals!}
\end{ttbox}



\section{A more complex proof}
Many of Pelletier's test problems for theorem provers \cite{pelletier86}
can be solved automatically.  Problem~39 concerns set theory, asserting
that there is no Russell set --- a set consisting of those sets that are
not members of themselves:
\[  \turn \neg (\exists x. \forall y. y\in x \bimp y\not\in y) \]
This does not require special properties of membership; we may
generalize $x\in y$ to an arbitrary predicate~$F(x,y)$.  The theorem has a
short manual proof.  See the directory {\tt LK/ex} for many more
examples.

We set the main goal and move the negated formula to the left.
\begin{ttbox}
goal LK.thy "|- ~ (EX x. ALL y. F(y,x) <-> ~F(y,y))";
{\out Level 0}
{\out  |- ~ (EX x. ALL y. F(y,x) <-> ~ F(y,y))}
{\out  1.  |- ~ (EX x. ALL y. F(y,x) <-> ~ F(y,y))}
by (resolve_tac [notR] 1);
{\out Level 1}
{\out  |- ~ (EX x. ALL y. F(y,x) <-> ~ F(y,y))}
{\out  1. EX x. ALL y. F(y,x) <-> ~ F(y,y) |-}
by (resolve_tac [exL] 1);
\end{ttbox}
The right side is empty; we strip both quantifiers from the formula on the
left.
\begin{ttbox}
{\out Level 2}
{\out  |- ~ (EX x. ALL y. F(y,x) <-> ~ F(y,y))}
{\out  1. !!x. ALL y. F(y,x) <-> ~ F(y,y) |-}
by (resolve_tac [allL_thin] 1);
{\out Level 3}
{\out  |- ~ (EX x. ALL y. F(y,x) <-> ~ F(y,y))}
{\out  1. !!x. F(?x2(x),x) <-> ~ F(?x2(x),?x2(x)) |-}
\end{ttbox}
The rule \ttindex{iffL} says, if $P\bimp Q$ then $P$ and~$Q$ are either
both true or both false.  It yields two subgoals.
\begin{ttbox}
by (resolve_tac [iffL] 1);
{\out Level 4}
{\out  |- ~ (EX x. ALL y. F(y,x) <-> ~ F(y,y))}
{\out  1. !!x.  |- F(?x2(x),x), ~ F(?x2(x),?x2(x))}
{\out  2. !!x. ~ F(?x2(x),?x2(x)), F(?x2(x),x) |-}
\end{ttbox}
We must instantiate~$\Var{x@2}$, the shared unknown, to satisfy both
subgoals.  Beginning with subgoal~2, we move a negated formula to the left
and create a basic sequent.
\begin{ttbox}
by (resolve_tac [notL] 2);
{\out Level 5}
{\out  |- ~ (EX x. ALL y. F(y,x) <-> ~ F(y,y))}
{\out  1. !!x.  |- F(?x2(x),x), ~ F(?x2(x),?x2(x))}
{\out  2. !!x. F(?x2(x),x) |- F(?x2(x),?x2(x))}
by (resolve_tac [basic] 2);
{\out Level 6}
{\out  |- ~ (EX x. ALL y. F(y,x) <-> ~ F(y,y))}
{\out  1. !!x.  |- F(x,x), ~ F(x,x)}
\end{ttbox}
Thanks to the instantiation of~$\Var{x@2}$, subgoal~1 is obviously true.
\begin{ttbox}
by (resolve_tac [notR] 1);
{\out Level 7}
{\out  |- ~ (EX x. ALL y. F(y,x) <-> ~ F(y,y))}
{\out  1. !!x. F(x,x) |- F(x,x)}
by (resolve_tac [basic] 1);
{\out Level 8}
{\out  |- ~ (EX x. ALL y. F(y,x) <-> ~ F(y,y))}
{\out No subgoals!}
\end{ttbox}
