Isabelle is a generic system for
implementing logical formalisms, and Isabelle/HOL is the specialization
of Isabelle for HOL, which abbreviates Higher-Order Logic. We introduce
HOL step by step following the equation
\[ \mbox{HOL} = \mbox{Functional Programming} + \mbox{Logic}. \]
We assume that the reader is familiar with the basic concepts of functional
programming and is used to logical and set theoretic notation.

\autoref{sec:FP} introduces HOL as a functional programming language and
explains how to write simple inductive proofs of mostly equational properties
of recursive functions.
\sem
\autoref{sec:CaseStudyExp} contains a
little case study: arithmetic and boolean expressions, their evaluation,
optimization and compilation.
\endsem
\autoref{ch:Logic} introduces the rest of HOL: the
language of formulas beyond equality, automatic proof tools, single
step proofs, and inductive definitions, an essential specification construct.
\autoref{ch:Isar} introduces Isar, Isabelle's language for writing structured
proofs.

%Further material (slides, demos etc) can be found online at
%\url{http://www.in.tum.de/~nipkow}.

% Relics:
% We aim to minimise the amount of background knowledge of logic we expect
% from the reader
% We have focussed on the core material
% in the intersection of computation and logic.

This introduction to the core of Isabelle is intentionally concrete and
example-based: we concentrate on examples that illustrate the typical cases;
we do not explain the general case if it can be inferred from the examples.
For a comprehensive treatment of all things Isabelle we recommend the
\emph{Isabelle/Isar Reference Manual}~\cite{IsarRef}, which comes with the
Isabelle distribution.
The tutorial by Nipkow, Paulson and Wenzel~\cite{LNCS2283} (in its updated version that comes with the Isabelle distribution) is still recommended for the wealth of examples and material, but its proof style is outdated. In particular it fails to cover the structured proof language Isar.

This introduction has grown out of many years of teaching Isabelle courses.
It tries to cover the essentials (from a functional programming point of
view) as quickly and compactly as possible. There is also an accompanying
set of \LaTeX-based slides available from the author on request.

\paragraph{Acknowledgements}
I wish to thank the following people for their comments on this text:
Florian Haftmann, Ren\'{e} Thiemann and Christian Sternagel.