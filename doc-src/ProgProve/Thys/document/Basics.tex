%
\begin{isabellebody}%
\def\isabellecontext{Basics}%
%
\isadelimtheory
%
\endisadelimtheory
%
\isatagtheory
%
\endisatagtheory
{\isafoldtheory}%
%
\isadelimtheory
%
\endisadelimtheory
%
\begin{isamarkuptext}%
This chapter introduces HOL as a functional programming language and shows
how to prove properties of functional programs by induction.

\section{Basics}

\subsection{Types, Terms and Formulae}
\label{sec:TypesTermsForms}

HOL is a typed logic whose type system resembles that of functional
programming languages. Thus there are
\begin{description}
\item[base types,] 
in particular \isa{bool}, the type of truth values,
\isa{nat}, the type of natural numbers ($\mathbb{N}$), and \isa{int},
the type of mathematical integers ($\mathbb{Z}$).
\item[type constructors,]
 in particular \isa{list}, the type of
lists, and \isa{set}, the type of sets. Type constructors are written
postfix, e.g.\ \isa{nat\ list} is the type of lists whose elements are
natural numbers.
\item[function types,]
denoted by \isa{{\isaliteral{5C3C52696768746172726F773E}{\isasymRightarrow}}}.
\item[type variables,]
  denoted by \isa{{\isaliteral{27}{\isacharprime}}a}, \isa{{\isaliteral{27}{\isacharprime}}b} etc., just like in ML\@.
\end{description}

\concept{Terms} are formed as in functional programming by
applying functions to arguments. If \isa{f} is a function of type
\isa{{\isaliteral{5C3C7461753E}{\isasymtau}}\isaliteral{5C3C5E697375623E}{}\isactrlisub {\isadigit{1}}\ {\isaliteral{5C3C52696768746172726F773E}{\isasymRightarrow}}\ {\isaliteral{5C3C7461753E}{\isasymtau}}\isaliteral{5C3C5E697375623E}{}\isactrlisub {\isadigit{2}}} and \isa{t} is a term of type
\isa{{\isaliteral{5C3C7461753E}{\isasymtau}}\isaliteral{5C3C5E697375623E}{}\isactrlisub {\isadigit{1}}} then \isa{f\ t} is a term of type \isa{{\isaliteral{5C3C7461753E}{\isasymtau}}\isaliteral{5C3C5E697375623E}{}\isactrlisub {\isadigit{2}}}. We write \isa{t\ {\isaliteral{3A}{\isacharcolon}}{\isaliteral{3A}{\isacharcolon}}\ {\isaliteral{5C3C7461753E}{\isasymtau}}} to mean that term \isa{t} has type \isa{{\isaliteral{5C3C7461753E}{\isasymtau}}}.

\begin{warn}
There are many predefined infix symbols like \isa{{\isaliteral{2B}{\isacharplus}}} and \isa{{\isaliteral{5C3C6C653E}{\isasymle}}}.
The name of the corresponding binary function is \isa{op\ {\isaliteral{2B}{\isacharplus}}},
not just \isa{{\isaliteral{2B}{\isacharplus}}}. That is, \isa{x\ {\isaliteral{2B}{\isacharplus}}\ y} is syntactic sugar for
\noquotes{\isa{{\isaliteral{22}{\isachardoublequote}}op\ {\isaliteral{2B}{\isacharplus}}\ x\ y{\isaliteral{22}{\isachardoublequote}}}}.
\end{warn}

HOL also supports some basic constructs from functional programming:
\begin{quote}
\isa{{\isaliteral{28}{\isacharparenleft}}if\ b\ then\ t\isaliteral{5C3C5E697375623E}{}\isactrlisub {\isadigit{1}}\ else\ t\isaliteral{5C3C5E697375623E}{}\isactrlisub {\isadigit{2}}{\isaliteral{29}{\isacharparenright}}}\\
\isa{{\isaliteral{28}{\isacharparenleft}}let\ x\ {\isaliteral{3D}{\isacharequal}}\ t\ in\ u{\isaliteral{29}{\isacharparenright}}}\\
\isa{{\isaliteral{28}{\isacharparenleft}}case\ t\ of\ pat\isaliteral{5C3C5E697375623E}{}\isactrlisub {\isadigit{1}}\ {\isaliteral{5C3C52696768746172726F773E}{\isasymRightarrow}}\ t\isaliteral{5C3C5E697375623E}{}\isactrlisub {\isadigit{1}}\ {\isaliteral{7C}{\isacharbar}}\ {\isaliteral{5C3C646F74733E}{\isasymdots}}\ {\isaliteral{7C}{\isacharbar}}\ pat\isaliteral{5C3C5E697375623E}{}\isactrlisub n\ {\isaliteral{5C3C52696768746172726F773E}{\isasymRightarrow}}\ t\isaliteral{5C3C5E697375623E}{}\isactrlisub n{\isaliteral{29}{\isacharparenright}}}
\end{quote}
\begin{warn}
The above three constructs must always be enclosed in parentheses
if they occur inside other constructs.
\end{warn}
Terms may also contain \isa{{\isaliteral{5C3C6C616D6264613E}{\isasymlambda}}}-abstractions. For example,
\isa{{\isaliteral{5C3C6C616D6264613E}{\isasymlambda}}x{\isaliteral{2E}{\isachardot}}\ x} is the identity function.

\concept{Formulae} are terms of type \isa{bool}.
There are the basic constants \isa{True} and \isa{False} and
the usual logical connectives (in decreasing order of precedence):
\isa{{\isaliteral{5C3C6E6F743E}{\isasymnot}}}, \isa{{\isaliteral{5C3C616E643E}{\isasymand}}}, \isa{{\isaliteral{5C3C6F723E}{\isasymor}}}, \isa{{\isaliteral{5C3C6C6F6E6772696768746172726F773E}{\isasymlongrightarrow}}}.

\concept{Equality} is available in the form of the infix function \isa{{\isaliteral{3D}{\isacharequal}}}
of type \isa{{\isaliteral{27}{\isacharprime}}a\ {\isaliteral{5C3C52696768746172726F773E}{\isasymRightarrow}}\ {\isaliteral{27}{\isacharprime}}a\ {\isaliteral{5C3C52696768746172726F773E}{\isasymRightarrow}}\ bool}. It also works for formulas, where
it means ``if and only if''.

\concept{Quantifiers} are written \isa{{\isaliteral{5C3C666F72616C6C3E}{\isasymforall}}x{\isaliteral{2E}{\isachardot}}\ P} and \isa{{\isaliteral{5C3C6578697374733E}{\isasymexists}}x{\isaliteral{2E}{\isachardot}}\ P}.

Isabelle automatically computes the type of each variable in a term. This is
called \concept{type inference}.  Despite type inference, it is sometimes
necessary to attach explicit \concept{type constraints} (or \concept{type
annotations}) to a variable or term.  The syntax is \isa{t\ {\isaliteral{3A}{\isacharcolon}}{\isaliteral{3A}{\isacharcolon}}\ {\isaliteral{5C3C7461753E}{\isasymtau}}} as in
\mbox{\noquotes{\isa{{\isaliteral{22}{\isachardoublequote}}m\ {\isaliteral{3C}{\isacharless}}\ {\isaliteral{28}{\isacharparenleft}}n{\isaliteral{3A}{\isacharcolon}}{\isaliteral{3A}{\isacharcolon}}nat{\isaliteral{29}{\isacharparenright}}{\isaliteral{22}{\isachardoublequote}}}}}. Type constraints may be
needed to
disambiguate terms involving overloaded functions such as \isa{{\isaliteral{2B}{\isacharplus}}}, \isa{{\isaliteral{2A}{\isacharasterisk}}} and \isa{{\isaliteral{5C3C6C653E}{\isasymle}}}.

Finally there are the universal quantifier \isa{{\isaliteral{5C3C416E643E}{\isasymAnd}}} and the implication
\isa{{\isaliteral{5C3C4C6F6E6772696768746172726F773E}{\isasymLongrightarrow}}}. They are part of the Isabelle framework, not the logic
HOL. Logically, they agree with their HOL counterparts \isa{{\isaliteral{5C3C666F72616C6C3E}{\isasymforall}}} and
\isa{{\isaliteral{5C3C6C6F6E6772696768746172726F773E}{\isasymlongrightarrow}}}, but operationally they behave differently. This will become
clearer as we go along.
\begin{warn}
Right-arrows of all kinds always associate to the right. In particular,
the formula
\isa{A\isaliteral{5C3C5E697375623E}{}\isactrlisub {\isadigit{1}}\ {\isaliteral{5C3C4C6F6E6772696768746172726F773E}{\isasymLongrightarrow}}\ A\isaliteral{5C3C5E697375623E}{}\isactrlisub {\isadigit{2}}\ {\isaliteral{5C3C4C6F6E6772696768746172726F773E}{\isasymLongrightarrow}}\ A\isaliteral{5C3C5E697375623E}{}\isactrlisub {\isadigit{3}}} means \isa{A\isaliteral{5C3C5E697375623E}{}\isactrlisub {\isadigit{1}}\ {\isaliteral{5C3C4C6F6E6772696768746172726F773E}{\isasymLongrightarrow}}\ {\isaliteral{28}{\isacharparenleft}}A\isaliteral{5C3C5E697375623E}{}\isactrlisub {\isadigit{2}}\ {\isaliteral{5C3C4C6F6E6772696768746172726F773E}{\isasymLongrightarrow}}\ A\isaliteral{5C3C5E697375623E}{}\isactrlisub {\isadigit{3}}{\isaliteral{29}{\isacharparenright}}}.
The (Isabelle specific) notation \mbox{\isa{{\isaliteral{5C3C6C6272616B6B3E}{\isasymlbrakk}}\ A\isaliteral{5C3C5E697375623E}{}\isactrlisub {\isadigit{1}}{\isaliteral{3B}{\isacharsemicolon}}\ {\isaliteral{5C3C646F74733E}{\isasymdots}}{\isaliteral{3B}{\isacharsemicolon}}\ A\isaliteral{5C3C5E697375623E}{}\isactrlisub n\ {\isaliteral{5C3C726272616B6B3E}{\isasymrbrakk}}\ {\isaliteral{5C3C4C6F6E6772696768746172726F773E}{\isasymLongrightarrow}}\ A}}
is short for the iterated implication \mbox{\isa{A\isaliteral{5C3C5E697375623E}{}\isactrlisub {\isadigit{1}}\ {\isaliteral{5C3C4C6F6E6772696768746172726F773E}{\isasymLongrightarrow}}\ {\isaliteral{5C3C646F74733E}{\isasymdots}}\ {\isaliteral{5C3C4C6F6E6772696768746172726F773E}{\isasymLongrightarrow}}\ A\isaliteral{5C3C5E697375623E}{}\isactrlisub n\ {\isaliteral{5C3C4C6F6E6772696768746172726F773E}{\isasymLongrightarrow}}\ A}}.
Sometimes we also employ inference rule notation:
\inferrule{\mbox{\isa{A\isaliteral{5C3C5E697375623E}{}\isactrlisub {\isadigit{1}}}}\\ \mbox{\isa{{\isaliteral{5C3C646F74733E}{\isasymdots}}}}\\ \mbox{\isa{A\isaliteral{5C3C5E697375623E}{}\isactrlisub n}}}
{\mbox{\isa{A}}}
\end{warn}


\subsection{Theories}
\label{sec:Basic:Theories}

Roughly speaking, a \concept{theory} is a named collection of types,
functions, and theorems, much like a module in a programming language.
All the Isabelle text that you ever type needs to go into a theory.
The general format of a theory \isa{T} is
\begin{quote}
\isacom{theory} \isa{T}\\
\isacom{imports} \isa{T\isaliteral{5C3C5E697375623E}{}\isactrlisub {\isadigit{1}}\ {\isaliteral{5C3C646F74733E}{\isasymdots}}\ T\isaliteral{5C3C5E697375623E}{}\isactrlisub n}\\
\isacom{begin}\\
\emph{definitions, theorems and proofs}\\
\isacom{end}
\end{quote}
where \isa{T\isaliteral{5C3C5E697375623E}{}\isactrlisub {\isadigit{1}}\ {\isaliteral{5C3C646F74733E}{\isasymdots}}\ T\isaliteral{5C3C5E697375623E}{}\isactrlisub n} are the names of existing
theories that \isa{T} is based on. The \isa{T\isaliteral{5C3C5E697375623E}{}\isactrlisub i} are the
direct \concept{parent theories} of \isa{T}.
Everything defined in the parent theories (and their parents, recursively) is
automatically visible. Each theory \isa{T} must
reside in a \concept{theory file} named \isa{T{\isaliteral{2E}{\isachardot}}thy}.

\begin{warn}
HOL contains a theory \isa{Main}, the union of all the basic
predefined theories like arithmetic, lists, sets, etc.
Unless you know what you are doing, always include \isa{Main}
as a direct or indirect parent of all your theories.
\end{warn}

In addition to the theories that come with the Isabelle/HOL distribution
(see \url{http://isabelle.in.tum.de/library/HOL/})
there is also the \emph{Archive of Formal Proofs}
at  \url{http://afp.sourceforge.net}, a growing collection of Isabelle theories
that everybody can contribute to.

\subsection{Quotation Marks}

The textual definition of a theory follows a fixed syntax with keywords like
\isacommand{begin} and \isacommand{datatype}.  Embedded in this syntax are
the types and formulae of HOL.  To distinguish the two levels, everything
HOL-specific (terms and types) must be enclosed in quotation marks:
\texttt{"}\dots\texttt{"}. To lessen this burden, quotation marks around a
single identifier can be dropped.  When Isabelle prints a syntax error
message, it refers to the HOL syntax as the \concept{inner syntax} and the
enclosing theory language as the \concept{outer syntax}.
\begin{warn}
For reasons of readability, we almost never show the quotation marks in this
book. Consult the accompanying theory files to see where they need to go.
\end{warn}%
\end{isamarkuptext}%
\isamarkuptrue%
%
\isadelimtheory
%
\endisadelimtheory
%
\isatagtheory
%
\endisatagtheory
{\isafoldtheory}%
%
\isadelimtheory
%
\endisadelimtheory
\end{isabellebody}%
%%% Local Variables:
%%% mode: latex
%%% TeX-master: "root"
%%% End:
