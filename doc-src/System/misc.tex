
% $Id$

\chapter{Miscellaneous tools} \label{ch:tools}

Subsequently we describe various Isabelle related utilities --- in
alphabetical order.


\section{Viewing documentation --- \texttt{isatool doc}} \label{sec:tool-doc}

The \tooldx{doc} utility displays online documentation:
\begin{ttbox}
Usage: isatool doc [DOC]

  View Isabelle documentation DOC, or show list of available documents.
\end{ttbox}
If called without arguments, it lists all available documents. Each line
starts with an identifier, followed by a short description. Any of these
identifiers may be specified as the first argument in order to have the
corresponding document displayed.

\medskip The \texttt{ISABELLE_DOCS} setting specifies the list of directories
(separated by colons) to be scanned for documentations.  The program for
viewing \texttt{dvi} files is determined by the \texttt{DVI_VIEWER} setting.


\section{Tuning proof scripts --- \texttt{isatool expandshort}}

The \tooldx{expandshort} utility tunes {\ML} proof scripts to enhance
readability:
\begin{ttbox}
Usage: expandshort [FILES|DIRS...]

  Recursively find .ML files, expand shorthand goal commands.  Also
  contracts uses of resolve_tac, dresolve_tac, eresolve_tac,
  forward_tac, rewrite_goals_tac on 1-element lists; furthermore
  expands tabs, which are forbidden in SML string constants.

  Renames old versions of files by appending "~~".
\end{ttbox}
In the files or directories supplied as arguments, all occurrences of the
shorthand commands \texttt{br}, \texttt{be} etc.\ in proof scripts are
replaced with the corresponding full commands.  The old versions of the files
are renamed to have the suffix``~\verb'~~'''.


\section{Getting logic images --- \texttt{isatool findlogics}}

The \tooldx{findlogics} utility traverses all directories specified in
\texttt{ISABELLE_PATH}, looking for Isabelle logic images. Its usage is:
\begin{ttbox}
Usage: isatool findlogics

  Collect heap file names from ISABELLE_PATH.
\end{ttbox}
The base names of all files found on the path are printed --- sorted and with
duplicates removed. Also note that \texttt{ISABELLE_PATH} implicitly depends
upon \texttt{ML_SYSTEM} and \texttt{ML_PLATFORM}. Thus switching to another
{\ML} compiler may change the set of logic images available.


\section{Inspecting the settings environment -- \texttt{isatool getenv}}
\label{sec:tool-getenv}

The Isabelle settings environment --- as provided by the site-default and
user-specific settings files --- can be inspected with the \tooldx{getenv}
utility:
\begin{ttbox}
Usage: isatool getenv [OPTIONS] [VARNAMES ...]

  Options are:
    -a           display complete environment
    -b           print values only (doesn't work for -a)

  Get value of VARNAMES from the Isabelle settings.
\end{ttbox}

With the \texttt{-a} option, one may inspect the full process environment that
Isabelle related programs are run in. This usually contains much more
variables than are actually Isabelle settings.  Normally, output is a list of
lines of the form \mbox{$name$\texttt{=}$value$}. The \texttt{-b} option
causes only the values to be printed.


\subsection*{Examples}

Get the {\ML} system identifier and the location where the compiler binaries
are supposed to reside as follows:
\begin{ttbox}
isatool getenv ML_SYSTEM ML_HOME
{\out ML_SYSTEM=smlnj-110}
{\out ML_HOME=/usr/local/smlnj-110/bin}
\end{ttbox}

The next one peeks at the search path that \texttt{isabelle} uses to locate
logic images:
\begin{ttbox}
isatool getenv -b ISABELLE_PATH
{\out /home/me/isabelle/heaps/smlnj-110:/usr/local/isabelle/heaps/smlnj-110}
\end{ttbox}
Here we have used the \texttt{-b} option to suppress the
\texttt{ISABELLE_PATH=} prefix.  The value above is what became of the
following assignment in the default settings file:
\begin{ttbox}
ISABELLE_PATH=\$ISABELLE_HOME_USER/heaps:\$ISABELLE_HOME/heaps
\end{ttbox}
Note how the \texttt{ML_SYSTEM} value got appended automatically to each path
component. This is a special feature of \texttt{ISABELLE_PATH} (and also of
\texttt{ISABELLE_OUTPUT}).


\section{Installing standalone Isabelle executables -- \texttt{isatool install}}
\label{sec:tool-install}

By default, the Isabelle binaries (\texttt{isabelle}, \texttt{isatool} etc.)
are just run from their location within the distribution directory, probably
indirectly by the shell through its \texttt{PATH}.  Other schemes of
installation are supported by the \tooldx{install} utility:
\begin{ttbox}
Usage: install [OPTIONS]

  Options are:
    -d DISTDIR   use DISTDIR as Isabelle distribution
                 (default ISABELLE_HOME)
    -k           install KDE application icon on Desktop
    -p DIR       install standalone binaries in DIR

  Install Isabelle executables with absolute references to the current
  distribution directory.
\end{ttbox}

The \texttt{-d} option overrides the current Isabelle distribution directory
as determined by \texttt{ISABELLE_HOME}.

The \texttt{-p} option installs executable wrapper scripts for
\texttt{isabelle}, \texttt{isatool}, \texttt{Isabelle}, containing proper
absolute references to the Isabelle distribution directory.  A typical
\texttt{DIR} specification would be some directory expected to be in the
shell's \texttt{PATH}, such as \texttt{/usr/local/bin}.  It is important to
note that a plain manual copy of the original Isabelle executables just would
not work!

The \texttt{-k} option creates an Isabelle application object for the popular
\textsl{K~Desktop Environment} (KDE)\index{KDE}.  The icon will appear
directly on Desktop.


\section{Creating instances of the Isabelle logo -- \texttt{isatool
    logo}}

The \tooldx{logo} utility creates any instance of the generic Isabelle logo as
an Encapsuled Postscript file (EPS):
\begin{ttbox}
Usage: logo [OPTIONS] NAME

  Create instance NAME of the Isabelle logo (as EPS).

  Options are:
    -o OUTFILE   set output file (default determined from NAME)
    -q           quiet mode
\end{ttbox}
You are encouraged to use this to create a derived logo for your Isabelle
project.  For example, \texttt{isatool logo HOOL} creates
\texttt{isabelle_hool.eps}.


\section{Isabelle's version of make --- \texttt{isatool make}}

The Isabelle \tooldx{make} utility is a very simple wrapper for
ordinary Unix \texttt{make}:
\begin{ttbox}
Usage: isatool make [ARGS ...]

  Compile the logic in current directory using IsaMakefile.
  ARGS are directly passed to the system make program.
\end{ttbox}
Note that the Isabelle settings environment is also active. Thus one
may refer to its values within the \ttindex{IsaMakefile}, e.g.\ 
\texttt{\$(ISABELLE_OUTPUT)}. Furthermore, programs started from the
make file also inherit this environment.  Typically,
\texttt{IsaMakefile}s defer the real work to the \texttt{usedir}
utility, see \S\ref{sec:tool-usedir}.

\medskip The basic \texttt{IsaMakefile} convention is that the default
target builds the actual logic, including its parents if appropriate.
The \texttt{images} target is intended to build all local logic
images, while the \texttt{test} target shall build all related
examples.  The \texttt{all} target shall do \texttt{images} and
\texttt{test}.


\subsection*{Examples}

Refer to the \texttt{IsaMakefile}s of the Isabelle distribution's
object-logics as a model for your own developements.  For example, see
\texttt{src/FOL/IsaMakefile}.


\section{Make all logics -- \texttt{isatool makeall}}

The \tooldx{makeall} utility applies Isabelle make to all logic
directories of the distribution:
\begin{ttbox}
Usage: makeall [ARGS ...]

  Apply isatool make to all logics (passing ARGS).
\end{ttbox}
The arguments \texttt{ARGS} are just passed verbatim to each
\texttt{make} invocation.

%%% Local Variables: 
%%% mode: latex
%%% TeX-master: "system"
%%% End: 
