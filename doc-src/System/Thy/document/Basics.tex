%
\begin{isabellebody}%
\def\isabellecontext{Basics}%
%
\isadelimtheory
\isanewline
\isanewline
%
\endisadelimtheory
%
\isatagtheory
\isacommand{theory}\isamarkupfalse%
\ Basics\isanewline
\isakeyword{imports}\ Pure\isanewline
\isakeyword{begin}%
\endisatagtheory
{\isafoldtheory}%
%
\isadelimtheory
%
\endisadelimtheory
%
\isamarkupchapter{The Isabelle system environment%
}
\isamarkuptrue%
%
\begin{isamarkuptext}%
This manual describes Isabelle together with related tools and user
  interfaces as seen from an outside (system oriented) view.  See also
  the \emph{Isabelle/Isar Reference Manual}~\cite{isabelle-isar-ref}
  and the \emph{Isabelle Reference Manual}~\cite{isabelle-ref} for the
  actual Isabelle commands and related functions.

  \medskip The Isabelle system environment emerges from a few general
  concepts.

  \begin{enumerate}

  \item The \emph{Isabelle settings} mechanism provides environment
  variables to all Isabelle programs (including tools and user
  interfaces).

  \item The \emph{raw Isabelle process} (\indexref{}{executable}{isabelle}\hyperlink{executable.isabelle}{\mbox{\isa{\isatt{isabelle}}}} or
  \indexref{}{executable}{isabelle-process}\hyperlink{executable.isabelle-process}{\mbox{\isa{\isatt{isabelle{\isacharminus}process}}}}) runs logic sessions either
  interactively or in batch mode.  In particular, this view abstracts
  over the invocation of the actual ML system to be used.  Regular
  users rarely need to care about the low-level process.

  \item The \emph{Isabelle tools wrapper} (\indexref{}{executable}{isatool}\hyperlink{executable.isatool}{\mbox{\isa{\isatt{isatool}}}})
  provides a generic startup environment Isabelle related utilities,
  user interfaces etc.  Such tools automatically benefit from the
  settings mechanism.

  \item The \emph{Isabelle interface wrapper} (\indexref{}{executable}{Isabelle}\hyperlink{executable.Isabelle}{\mbox{\isa{\isatt{Isabelle}}}} or \indexref{}{executable}{isabelle-interface}\hyperlink{executable.isabelle-interface}{\mbox{\isa{\isatt{isabelle{\isacharminus}interface}}}}) provides some
  abstraction over the actual user interface to be used.  The de-facto
  standard interface for Isabelle is Proof~General
  \cite{proofgeneral}.

  \end{enumerate}%
\end{isamarkuptext}%
\isamarkuptrue%
%
\isamarkupsection{Isabelle settings \label{sec:settings}%
}
\isamarkuptrue%
%
\begin{isamarkuptext}%
The Isabelle system heavily depends on the \emph{settings
  mechanism}\indexbold{settings}.  Essentially, this is a statically
  scoped collection of environment variables, such as \hyperlink{setting.ISABELLE-HOME}{\mbox{\isa{\isatt{ISABELLE{\isacharunderscore}HOME}}}}, \hyperlink{setting.ML-SYSTEM}{\mbox{\isa{\isatt{ML{\isacharunderscore}SYSTEM}}}}, \hyperlink{setting.ML-HOME}{\mbox{\isa{\isatt{ML{\isacharunderscore}HOME}}}}.  These
  variables are \emph{not} intended to be set directly from the shell,
  though.  Isabelle employs a somewhat more sophisticated scheme of
  \emph{settings files} --- one for site-wide defaults, another for
  additional user-specific modifications.  With all configuration
  variables in at most two places, this scheme is more maintainable
  and user-friendly than global shell environment variables.

  In particular, we avoid the typical situation where prospective
  users of a software package are told to put several things into
  their shell startup scripts, before being able to actually run the
  program. Isabelle requires none such administrative chores of its
  end-users --- the executables can be invoked straight away.
  Occasionally, users would still want to put the \hyperlink{file.$ISABELLE-HOME/bin}{\mbox{\isa{\isatt{{\isachardollar}ISABELLE{\isacharunderscore}HOME{\isacharslash}bin}}}} directory into their shell's search path, but
  this is not required.%
\end{isamarkuptext}%
\isamarkuptrue%
%
\isamarkupsubsection{Building the environment%
}
\isamarkuptrue%
%
\begin{isamarkuptext}%
Whenever any of the Isabelle executables is run, their settings
  environment is put together as follows.

  \begin{enumerate}

  \item The special variable \indexdef{}{setting}{ISABELLE\_HOME}\hypertarget{setting.ISABELLE-HOME}{\hyperlink{setting.ISABELLE-HOME}{\mbox{\isa{\isatt{ISABELLE{\isacharunderscore}HOME}}}}} is
  determined automatically from the location of the binary that has
  been run.
  
  You should not try to set \hyperlink{setting.ISABELLE-HOME}{\mbox{\isa{\isatt{ISABELLE{\isacharunderscore}HOME}}}} manually. Also
  note that the Isabelle executables either have to be run from their
  original location in the distribution directory, or via the
  executable objects created by the \hyperlink{tool.install}{\mbox{\isa{\isatt{install}}}} utility.  Symbolic
  links are admissible, but a plain copy of the \hyperlink{file.$ISABELLE-HOME/bin}{\mbox{\isa{\isatt{{\isachardollar}ISABELLE{\isacharunderscore}HOME{\isacharslash}bin}}}} files will not work!

  \item The file \hyperlink{file.$ISABELLE-HOME/etc/settings}{\mbox{\isa{\isatt{{\isachardollar}ISABELLE{\isacharunderscore}HOME{\isacharslash}etc{\isacharslash}settings}}}} ist run as a
  \indexref{}{executable}{bash}\hyperlink{executable.bash}{\mbox{\isa{\isatt{bash}}}} shell script with the auto-export option for
  variables enabled.
  
  This file holds a rather long list of shell variable assigments,
  thus providing the site-wide default settings.  The Isabelle
  distribution already contains a global settings file with sensible
  defaults for most variables.  When installing the system, only a few
  of these may have to be adapted (probably \hyperlink{setting.ML-SYSTEM}{\mbox{\isa{\isatt{ML{\isacharunderscore}SYSTEM}}}}
  etc.).
  
  \item The file \hyperlink{file.$ISABELLE-HOME-USER/etc/settings}{\mbox{\isa{\isatt{{\isachardollar}ISABELLE{\isacharunderscore}HOME{\isacharunderscore}USER{\isacharslash}etc{\isacharslash}settings}}}} (if it
  exists) is run in the same way as the site default settings. Note
  that the variable \hyperlink{setting.ISABELLE-HOME-USER}{\mbox{\isa{\isatt{ISABELLE{\isacharunderscore}HOME{\isacharunderscore}USER}}}} has already been set
  before --- usually to \verb|~/isabelle|.
  
  Thus individual users may override the site-wide defaults.  See also
  file \hyperlink{file.$ISABELLE-HOME/etc/user-settings.sample}{\mbox{\isa{\isatt{{\isachardollar}ISABELLE{\isacharunderscore}HOME{\isacharslash}etc{\isacharslash}user{\isacharminus}settings{\isachardot}sample}}}} in the
  distribution.  Typically, a user settings file would contain only a
  few lines, just the assigments that are really changed.  One should
  definitely \emph{not} start with a full copy the basic \hyperlink{file.$ISABELLE-HOME/etc/settings}{\mbox{\isa{\isatt{{\isachardollar}ISABELLE{\isacharunderscore}HOME{\isacharslash}etc{\isacharslash}settings}}}}. This could cause very annoying
  maintainance problems later, when the Isabelle installation is
  updated or changed otherwise.
  
  \end{enumerate}

  Since settings files are regular GNU \indexdef{}{executable}{bash}\hypertarget{executable.bash}{\hyperlink{executable.bash}{\mbox{\isa{\isatt{bash}}}}} scripts,
  one may use complex shell commands, such as \verb|if| or
  \verb|case| statements to set variables depending on the
  system architecture or other environment variables.  Such advanced
  features should be added only with great care, though. In
  particular, external environment references should be kept at a
  minimum.

  \medskip A few variables are somewhat special:

  \begin{itemize}

  \item \indexdef{}{setting}{ISABELLE}\hypertarget{setting.ISABELLE}{\hyperlink{setting.ISABELLE}{\mbox{\isa{\isatt{ISABELLE}}}}} and \indexdef{}{setting}{ISATOOL}\hypertarget{setting.ISATOOL}{\hyperlink{setting.ISATOOL}{\mbox{\isa{\isatt{ISATOOL}}}}} are set
  automatically to the absolute path names of the \hyperlink{executable.isabelle-process}{\mbox{\isa{\isatt{isabelle{\isacharminus}process}}}} and \hyperlink{executable.isatool}{\mbox{\isa{\isatt{isatool}}}} executables,
  respectively.
  
  \item \indexref{}{setting}{ISABELLE\_OUTPUT}\hyperlink{setting.ISABELLE-OUTPUT}{\mbox{\isa{\isatt{ISABELLE{\isacharunderscore}OUTPUT}}}} will have the identifiers of
  the Isabelle distribution (cf.\ \hyperlink{setting.ISABELLE-IDENTIFIER}{\mbox{\isa{\isatt{ISABELLE{\isacharunderscore}IDENTIFIER}}}}) and
  the ML system (cf.\ \hyperlink{setting.ML-IDENTIFIER}{\mbox{\isa{\isatt{ML{\isacharunderscore}IDENTIFIER}}}}) appended automatically
  to its value.

  \end{itemize}

  \medskip Note that the settings environment may be inspected with
  the Isabelle tool \hyperlink{tool.getenv}{\mbox{\isa{\isatt{getenv}}}}.  This might help to figure out the
  effect of complex settings scripts.%
\end{isamarkuptext}%
\isamarkuptrue%
%
\isamarkupsubsection{Common variables%
}
\isamarkuptrue%
%
\begin{isamarkuptext}%
This is a reference of common Isabelle settings variables. Note that
  the list is somewhat open-ended. Third-party utilities or interfaces
  may add their own selection. Variables that are special in some
  sense are marked with \isa{{\isachardoublequote}\isactrlsup {\isacharasterisk}{\isachardoublequote}}.

  \begin{description}

  \item[\indexdef{}{setting}{ISABELLE\_HOME}\hypertarget{setting.ISABELLE-HOME}{\hyperlink{setting.ISABELLE-HOME}{\mbox{\isa{\isatt{ISABELLE{\isacharunderscore}HOME}}}}}\isa{{\isachardoublequote}\isactrlsup {\isacharasterisk}{\isachardoublequote}}] is the
  location of the top-level Isabelle distribution directory. This is
  automatically determined from the Isabelle executable that has been
  invoked.  Do not attempt to set \hyperlink{setting.ISABELLE-HOME}{\mbox{\isa{\isatt{ISABELLE{\isacharunderscore}HOME}}}} yourself
  from the shell!
  
  \item[\indexdef{}{setting}{ISABELLE\_HOME\_USER}\hypertarget{setting.ISABELLE-HOME-USER}{\hyperlink{setting.ISABELLE-HOME-USER}{\mbox{\isa{\isatt{ISABELLE{\isacharunderscore}HOME{\isacharunderscore}USER}}}}}] is the user-specific
  counterpart of \hyperlink{setting.ISABELLE-HOME}{\mbox{\isa{\isatt{ISABELLE{\isacharunderscore}HOME}}}}. The default value is
  \verb|~/isabelle|, under rare circumstances this may be
  changed in the global setting file.  Typically, the \hyperlink{setting.ISABELLE-HOME-USER}{\mbox{\isa{\isatt{ISABELLE{\isacharunderscore}HOME{\isacharunderscore}USER}}}} directory mimics \hyperlink{setting.ISABELLE-HOME}{\mbox{\isa{\isatt{ISABELLE{\isacharunderscore}HOME}}}} to
  some extend. In particular, site-wide defaults may be overridden by
  a private \hyperlink{file.$ISABELLE-HOME-USER/etc/settings}{\mbox{\isa{\isatt{{\isachardollar}ISABELLE{\isacharunderscore}HOME{\isacharunderscore}USER{\isacharslash}etc{\isacharslash}settings}}}}.
  
  \item[\indexdef{}{setting}{ISABELLE}\hypertarget{setting.ISABELLE}{\hyperlink{setting.ISABELLE}{\mbox{\isa{\isatt{ISABELLE}}}}}\isa{{\isachardoublequote}\isactrlsup {\isacharasterisk}{\isachardoublequote}}, \hyperlink{setting.ISATOOL}{\mbox{\isa{\isatt{ISATOOL}}}}\isa{{\isachardoublequote}\isactrlsup {\isacharasterisk}{\isachardoublequote}}] are automatically set to the full path
  names of the \hyperlink{executable.isabelle-process}{\mbox{\isa{\isatt{isabelle{\isacharminus}process}}}} and \hyperlink{executable.isatool}{\mbox{\isa{\isatt{isatool}}}} executables, respectively.  Thus other tools and scripts
  need not assume that the \hyperlink{file.$ISABELLE-HOME/bin}{\mbox{\isa{\isatt{{\isachardollar}ISABELLE{\isacharunderscore}HOME{\isacharslash}bin}}}} directory is
  on the current search path of the shell.
  
  \item[\indexdef{}{setting}{ISABELLE\_IDENTIFIER}\hypertarget{setting.ISABELLE-IDENTIFIER}{\hyperlink{setting.ISABELLE-IDENTIFIER}{\mbox{\isa{\isatt{ISABELLE{\isacharunderscore}IDENTIFIER}}}}}\isa{{\isachardoublequote}\isactrlsup {\isacharasterisk}{\isachardoublequote}}] refers
  to the name of this Isabelle distribution, e.g.\ ``\verb|Isabelle2008|''.

  \item[\indexdef{}{setting}{ML\_SYSTEM}\hypertarget{setting.ML-SYSTEM}{\hyperlink{setting.ML-SYSTEM}{\mbox{\isa{\isatt{ML{\isacharunderscore}SYSTEM}}}}}, \indexdef{}{setting}{ML\_HOME}\hypertarget{setting.ML-HOME}{\hyperlink{setting.ML-HOME}{\mbox{\isa{\isatt{ML{\isacharunderscore}HOME}}}}},
  \indexdef{}{setting}{ML\_OPTIONS}\hypertarget{setting.ML-OPTIONS}{\hyperlink{setting.ML-OPTIONS}{\mbox{\isa{\isatt{ML{\isacharunderscore}OPTIONS}}}}}, \indexdef{}{setting}{ML\_PLATFORM}\hypertarget{setting.ML-PLATFORM}{\hyperlink{setting.ML-PLATFORM}{\mbox{\isa{\isatt{ML{\isacharunderscore}PLATFORM}}}}}, \indexdef{}{setting}{ML\_IDENTIFIER}\hypertarget{setting.ML-IDENTIFIER}{\hyperlink{setting.ML-IDENTIFIER}{\mbox{\isa{\isatt{ML{\isacharunderscore}IDENTIFIER}}}}}\isa{{\isachardoublequote}\isactrlsup {\isacharasterisk}{\isachardoublequote}}] specify the underlying ML system
  to be used for Isabelle.  There is only a fixed set of admissable
  \hyperlink{setting.ML-SYSTEM}{\mbox{\isa{\isatt{ML{\isacharunderscore}SYSTEM}}}} names (see the \hyperlink{file.$ISABELLE-HOME/etc/settings}{\mbox{\isa{\isatt{{\isachardollar}ISABELLE{\isacharunderscore}HOME{\isacharslash}etc{\isacharslash}settings}}}} file of the distribution).
  
  The actual compiler binary will be run from the directory \hyperlink{setting.ML-HOME}{\mbox{\isa{\isatt{ML{\isacharunderscore}HOME}}}}, with \hyperlink{setting.ML-OPTIONS}{\mbox{\isa{\isatt{ML{\isacharunderscore}OPTIONS}}}} as first arguments on the
  command line.  The optional \hyperlink{setting.ML-PLATFORM}{\mbox{\isa{\isatt{ML{\isacharunderscore}PLATFORM}}}} may specify the
  binary format of ML heap images, which is useful for cross-platform
  installations.  The value of \hyperlink{setting.ML-IDENTIFIER}{\mbox{\isa{\isatt{ML{\isacharunderscore}IDENTIFIER}}}} is
  automatically obtained by composing the values of \hyperlink{setting.ML-SYSTEM}{\mbox{\isa{\isatt{ML{\isacharunderscore}SYSTEM}}}}, \hyperlink{setting.ML-PLATFORM}{\mbox{\isa{\isatt{ML{\isacharunderscore}PLATFORM}}}} and the Isabelle version values.
  
  \item[\indexdef{}{setting}{ISABELLE\_PATH}\hypertarget{setting.ISABELLE-PATH}{\hyperlink{setting.ISABELLE-PATH}{\mbox{\isa{\isatt{ISABELLE{\isacharunderscore}PATH}}}}}] is a list of directories
  (separated by colons) where Isabelle logic images may reside.  When
  looking up heaps files, the value of \hyperlink{setting.ML-IDENTIFIER}{\mbox{\isa{\isatt{ML{\isacharunderscore}IDENTIFIER}}}} is
  appended to each component internally.
  
  \item[\indexdef{}{setting}{ISABELLE\_OUTPUT}\hypertarget{setting.ISABELLE-OUTPUT}{\hyperlink{setting.ISABELLE-OUTPUT}{\mbox{\isa{\isatt{ISABELLE{\isacharunderscore}OUTPUT}}}}}\isa{{\isachardoublequote}\isactrlsup {\isacharasterisk}{\isachardoublequote}}] is a
  directory where output heap files should be stored by default. The
  ML system and Isabelle version identifier is appended here, too.
  
  \item[\indexdef{}{setting}{ISABELLE\_BROWSER\_INFO}\hypertarget{setting.ISABELLE-BROWSER-INFO}{\hyperlink{setting.ISABELLE-BROWSER-INFO}{\mbox{\isa{\isatt{ISABELLE{\isacharunderscore}BROWSER{\isacharunderscore}INFO}}}}}] is the directory where
  theory browser information (HTML text, graph data, and printable
  documents) is stored (see also \secref{sec:info}).  The default
  value is \verb|$ISABELLE_HOME_USER/browser_info|.
  
  \item[\indexdef{}{setting}{ISABELLE\_LOGIC}\hypertarget{setting.ISABELLE-LOGIC}{\hyperlink{setting.ISABELLE-LOGIC}{\mbox{\isa{\isatt{ISABELLE{\isacharunderscore}LOGIC}}}}}] specifies the default logic to
  load if none is given explicitely by the user.  The default value is
  \verb|HOL|.
  
  \item[\indexdef{}{setting}{ISABELLE\_LINE\_EDITOR}\hypertarget{setting.ISABELLE-LINE-EDITOR}{\hyperlink{setting.ISABELLE-LINE-EDITOR}{\mbox{\isa{\isatt{ISABELLE{\isacharunderscore}LINE{\isacharunderscore}EDITOR}}}}}] specifies the default
  line editor for the \indexref{}{tool}{tty}\hyperlink{tool.tty}{\mbox{\isa{\isatt{tty}}}} interface.

  \item[\indexdef{}{setting}{ISABELLE\_USEDIR\_OPTIONS}\hypertarget{setting.ISABELLE-USEDIR-OPTIONS}{\hyperlink{setting.ISABELLE-USEDIR-OPTIONS}{\mbox{\isa{\isatt{ISABELLE{\isacharunderscore}USEDIR{\isacharunderscore}OPTIONS}}}}}] is implicitly prefixed
  to the command line of any \indexref{}{tool}{usedir}\hyperlink{tool.usedir}{\mbox{\isa{\isatt{usedir}}}} invocation. This
  typically contains compilation options for object-logics --- \hyperlink{tool.usedir}{\mbox{\isa{\isatt{usedir}}}} is the basic utility for managing logic sessions (cf.\ the
  \verb|IsaMakefile|s in the distribution).

  \item[\indexdef{}{setting}{ISABELLE\_FILE\_IDENT}\hypertarget{setting.ISABELLE-FILE-IDENT}{\hyperlink{setting.ISABELLE-FILE-IDENT}{\mbox{\isa{\isatt{ISABELLE{\isacharunderscore}FILE{\isacharunderscore}IDENT}}}}}] specifies a shell command
  for producing a source file identification, based on the actual
  content instead of the full physical path and date stamp (which is
  the default). A typical identification would produce a ``digest'' of
  the text, using a cryptographic hash function like SHA-1, for
  example.
  
  \item[\indexdef{}{setting}{ISABELLE\_LATEX}\hypertarget{setting.ISABELLE-LATEX}{\hyperlink{setting.ISABELLE-LATEX}{\mbox{\isa{\isatt{ISABELLE{\isacharunderscore}LATEX}}}}}, \indexdef{}{setting}{ISABELLE\_PDFLATEX}\hypertarget{setting.ISABELLE-PDFLATEX}{\hyperlink{setting.ISABELLE-PDFLATEX}{\mbox{\isa{\isatt{ISABELLE{\isacharunderscore}PDFLATEX}}}}}, \indexdef{}{setting}{ISABELLE\_BIBTEX}\hypertarget{setting.ISABELLE-BIBTEX}{\hyperlink{setting.ISABELLE-BIBTEX}{\mbox{\isa{\isatt{ISABELLE{\isacharunderscore}BIBTEX}}}}}, \indexdef{}{setting}{ISABELLE\_DVIPS}\hypertarget{setting.ISABELLE-DVIPS}{\hyperlink{setting.ISABELLE-DVIPS}{\mbox{\isa{\isatt{ISABELLE{\isacharunderscore}DVIPS}}}}}] refer to {\LaTeX} related tools for Isabelle
  document preparation (see also \secref{sec:tool-latex}).
  
  \item[\indexdef{}{setting}{ISABELLE\_TOOLS}\hypertarget{setting.ISABELLE-TOOLS}{\hyperlink{setting.ISABELLE-TOOLS}{\mbox{\isa{\isatt{ISABELLE{\isacharunderscore}TOOLS}}}}}] is a colon separated list of
  directories that are scanned by \hyperlink{executable.isatool}{\mbox{\isa{\isatt{isatool}}}} for external
  utility programs (see also \secref{sec:isatool}).
  
  \item[\indexdef{}{setting}{ISABELLE\_DOCS}\hypertarget{setting.ISABELLE-DOCS}{\hyperlink{setting.ISABELLE-DOCS}{\mbox{\isa{\isatt{ISABELLE{\isacharunderscore}DOCS}}}}}] is a colon separated list of
  directories with documentation files.
  
  \item[\indexdef{}{setting}{ISABELLE\_DOC\_FORMAT}\hypertarget{setting.ISABELLE-DOC-FORMAT}{\hyperlink{setting.ISABELLE-DOC-FORMAT}{\mbox{\isa{\isatt{ISABELLE{\isacharunderscore}DOC{\isacharunderscore}FORMAT}}}}}] specifies the preferred
  document format, typically \verb|dvi| or \verb|pdf|.
  
  \item[\indexdef{}{setting}{DVI\_VIEWER}\hypertarget{setting.DVI-VIEWER}{\hyperlink{setting.DVI-VIEWER}{\mbox{\isa{\isatt{DVI{\isacharunderscore}VIEWER}}}}}] specifies the command to be used
  for displaying \verb|dvi| files.
  
  \item[\indexdef{}{setting}{PDF\_VIEWER}\hypertarget{setting.PDF-VIEWER}{\hyperlink{setting.PDF-VIEWER}{\mbox{\isa{\isatt{PDF{\isacharunderscore}VIEWER}}}}}] specifies the command to be used
  for displaying \verb|pdf| files.
  
  \item[\indexdef{}{setting}{PRINT\_COMMAND}\hypertarget{setting.PRINT-COMMAND}{\hyperlink{setting.PRINT-COMMAND}{\mbox{\isa{\isatt{PRINT{\isacharunderscore}COMMAND}}}}}] specifies the standard printer
  spool command, which is expected to accept \verb|ps| files.
  
  \item[\indexdef{}{setting}{ISABELLE\_TMP\_PREFIX}\hypertarget{setting.ISABELLE-TMP-PREFIX}{\hyperlink{setting.ISABELLE-TMP-PREFIX}{\mbox{\isa{\isatt{ISABELLE{\isacharunderscore}TMP{\isacharunderscore}PREFIX}}}}}\isa{{\isachardoublequote}\isactrlsup {\isacharasterisk}{\isachardoublequote}}] is the
  prefix from which any running \hyperlink{executable.isabelle-process}{\mbox{\isa{\isatt{isabelle{\isacharminus}process}}}}
  derives an individual directory for temporary files.  The default is
  somewhere in \verb|/tmp|.
  
  \item[\indexdef{}{setting}{ISABELLE\_INTERFACE}\hypertarget{setting.ISABELLE-INTERFACE}{\hyperlink{setting.ISABELLE-INTERFACE}{\mbox{\isa{\isatt{ISABELLE{\isacharunderscore}INTERFACE}}}}}] is an identifier that
  specifies the actual user interface that the capital \hyperlink{executable.Isabelle}{\mbox{\isa{\isatt{Isabelle}}}} or \hyperlink{executable.isabelle-interface}{\mbox{\isa{\isatt{isabelle{\isacharminus}interface}}}} should invoke.  See
  \secref{sec:interface} for more details.

  \end{description}%
\end{isamarkuptext}%
\isamarkuptrue%
%
\isamarkupsection{The raw Isabelle process%
}
\isamarkuptrue%
%
\begin{isamarkuptext}%
The \indexdef{}{executable}{isabelle}\hypertarget{executable.isabelle}{\hyperlink{executable.isabelle}{\mbox{\isa{\isatt{isabelle}}}}} (or \indexdef{}{executable}{isabelle-process}\hypertarget{executable.isabelle-process}{\hyperlink{executable.isabelle-process}{\mbox{\isa{\isatt{isabelle{\isacharminus}process}}}}}) executable runs bare-bones Isabelle logic
  sessions --- either interactively or in batch mode.  It provides an
  abstraction over the underlying ML system, and over the actual heap
  file locations.  Its usage is:

\begin{ttbox}
Usage: isabelle-process [OPTIONS] [INPUT] [OUTPUT]

  Options are:
    -C           tell ML system to copy output image
    -I           startup Isar interaction mode
    -P           startup Proof General interaction mode
    -S           secure mode -- disallow critical operations
    -W OUTPUT    startup process wrapper, with messages going to OUTPUT stream
    -X           startup PGIP interaction mode
    -c           tell ML system to compress output image
    -e MLTEXT    pass MLTEXT to the ML session
    -f           pass 'Session.finish();' to the ML session
    -m MODE      add print mode for output
    -q           non-interactive session
    -r           open heap file read-only
    -u           pass 'use"ROOT.ML";' to the ML session
    -w           reset write permissions on OUTPUT

  INPUT (default "\$ISABELLE_LOGIC") and OUTPUT specify in/out heaps.
  These are either names to be searched in the Isabelle path, or
  actual file names (containing at least one /).
  If INPUT is "RAW_ML_SYSTEM", just start the bare bones ML system.
\end{ttbox}

  Input files without path specifications are looked up in the
  \hyperlink{setting.ISABELLE-PATH}{\mbox{\isa{\isatt{ISABELLE{\isacharunderscore}PATH}}}} setting, which may consist of multiple
  components separated by colons --- these are tried in the given
  order with the value of \hyperlink{setting.ML-IDENTIFIER}{\mbox{\isa{\isatt{ML{\isacharunderscore}IDENTIFIER}}}} appended
  internally.  In a similar way, base names are relative to the
  directory specified by \hyperlink{setting.ISABELLE-OUTPUT}{\mbox{\isa{\isatt{ISABELLE{\isacharunderscore}OUTPUT}}}}.  In any case,
  actual file locations may also be given by including at least one
  slash (\verb|/|) in the name (hint: use \verb|./| to
  refer to the current directory).%
\end{isamarkuptext}%
\isamarkuptrue%
%
\isamarkupsubsubsection{Options%
}
\isamarkuptrue%
%
\begin{isamarkuptext}%
If the input heap file does not have write permission bits set, or
  the \verb|-r| option is given explicitely, then the session
  started will be read-only.  That is, the ML world cannot be
  committed back into the image file.  Otherwise, a writable session
  enables commits into either the input file, or into another output
  heap file (if that is given as the second argument on the command
  line).

  The read-write state of sessions is determined at startup only, it
  cannot be changed intermediately. Also note that heap images may
  require considerable amounts of disk space (approximately
  50--200~MB). Users are responsible for themselves to dispose their
  heap files when they are no longer needed.

  \medskip The \verb|-w| option makes the output heap file
  read-only after terminating.  Thus subsequent invocations cause the
  logic image to be read-only automatically.

  \medskip The \verb|-c| option tells the underlying ML system
  to compress the output heap (fully transparently).  On Poly/ML for
  example, the image is garbage collected and all stored values are
  maximally shared, resulting in up to \isa{{\isachardoublequote}{\isadigit{5}}{\isadigit{0}}{\isacharpercent}{\isachardoublequote}} less disk space
  consumption.

  \medskip The \verb|-C| option tells the ML system to produce
  a completely self-contained output image, probably including a copy
  of the ML runtime system itself.

  \medskip Using the \verb|-e| option, arbitrary ML code may be
  passed to the Isabelle session from the command line. Multiple
  \verb|-e|'s are evaluated in the given order. Strange things
  may happen when errorneous ML code is provided. Also make sure that
  the ML commands are terminated properly by semicolon.

  \medskip The \verb|-u| option is a shortcut for \verb|-e| passing ``\verb|use "ROOT.ML";|'' to the ML session.
  The \verb|-f| option passes ``\verb|Session.finish();|'', which is intended mainly for administrative
  purposes.

  \medskip The \verb|-m| option adds identifiers of print modes
  to be made active for this session. Typically, this is used by some
  user interface, e.g.\ to enable output of proper mathematical
  symbols.

  \medskip Isabelle normally enters an interactive top-level loop
  (after processing the \verb|-e| texts). The \verb|-q|
  option inhibits interaction, thus providing a pure batch mode
  facility.

  \medskip The \verb|-I| option makes Isabelle enter Isar
  interaction mode on startup, instead of the primitive ML top-level.
  The \verb|-P| option configures the top-level loop for
  interaction with the Proof General user interface, and the
  \verb|-X| option enables XML-based PGIP communication.  The
  \verb|-W| option makes Isabelle enter a special process
  wrapper for interaction via an external program; the protocol is a
  stripped-down version of Proof General the interaction mode, see
  also \hyperlink{file.~~/src/Pure/Tools/isabelle-process.ML}{\mbox{\isa{\isatt{{\isachartilde}{\isachartilde}{\isacharslash}src{\isacharslash}Pure{\isacharslash}Tools{\isacharslash}isabelle{\isacharunderscore}process{\isachardot}ML}}}} and \hyperlink{file.~~/src/Pure/Tools/isabelle-process.scala}{\mbox{\isa{\isatt{{\isachartilde}{\isachartilde}{\isacharslash}src{\isacharslash}Pure{\isacharslash}Tools{\isacharslash}isabelle{\isacharunderscore}process{\isachardot}scala}}}}.

  \medskip The \verb|-S| option makes the Isabelle process more
  secure by disabling some critical operations, notably runtime
  compilation and evaluation of ML source code.%
\end{isamarkuptext}%
\isamarkuptrue%
%
\isamarkupsubsubsection{Examples%
}
\isamarkuptrue%
%
\begin{isamarkuptext}%
Run an interactive session of the default object-logic (as specified
  by the \hyperlink{setting.ISABELLE-LOGIC}{\mbox{\isa{\isatt{ISABELLE{\isacharunderscore}LOGIC}}}} setting) like this:
\begin{ttbox}
isabelle-process
\end{ttbox}

  Usually \hyperlink{setting.ISABELLE-LOGIC}{\mbox{\isa{\isatt{ISABELLE{\isacharunderscore}LOGIC}}}} refers to one of the standard
  logic images, which are read-only by default.  A writable session
  --- based on \verb|FOL|, but output to \verb|Foo| (in the
  directory specified by the \hyperlink{setting.ISABELLE-OUTPUT}{\mbox{\isa{\isatt{ISABELLE{\isacharunderscore}OUTPUT}}}} setting) ---
  may be invoked as follows:
\begin{ttbox}
isabelle-process FOL Foo
\end{ttbox}
  Ending this session normally (e.g.\ by typing control-D) dumps the
  whole ML system state into \verb|Foo|. Be prepared for several
  tens of megabytes.

  The \verb|Foo| session may be continued later (still in
  writable state) by:
\begin{ttbox}
isabelle-process Foo
\end{ttbox}
  A read-only \verb|Foo| session may be started by:
\begin{ttbox}
isabelle-process -r Foo
\end{ttbox}

  \medskip Note that manual session management like this does
  \emph{not} provide proper setup for theory presentation.  This would
  require the \hyperlink{tool.usedir}{\mbox{\isa{\isatt{usedir}}}} utility.

  \bigskip The next example demonstrates batch execution of Isabelle.
  We retrieve the \verb|FOL| theory value from the theory loader
  within ML:
\begin{ttbox}
isabelle-process -e 'theory "FOL";' -q -r FOL
\end{ttbox}
  Note that the output text will be interspersed with additional junk
  messages by the ML runtime environment.  The \verb|-W| option
  allows to communicate with the Isabelle process via an external
  program in a more robust fashion.%
\end{isamarkuptext}%
\isamarkuptrue%
%
\isamarkupsection{The Isabelle tools wrapper \label{sec:isatool}%
}
\isamarkuptrue%
%
\begin{isamarkuptext}%
All Isabelle related tools and interfaces are called via a common
  wrapper --- \hyperlink{executable.isatool}{\mbox{\isa{\isatt{isatool}}}}:

\begin{ttbox}
Usage: isatool TOOL [ARGS ...]

  Start Isabelle utility program TOOL with ARGS. Pass "-?" to TOOL
  for more specific help.

  Available tools are:

    browser - Isabelle graph browser
    \dots
\end{ttbox}

  In principle, Isabelle tools are ordinary executable scripts that
  are run within the Isabelle settings environment, see
  \secref{sec:settings}.  The set of available tools is collected by
  \hyperlink{executable.isatool}{\mbox{\isa{\isatt{isatool}}}} from the directories listed in the \hyperlink{setting.ISABELLE-TOOLS}{\mbox{\isa{\isatt{ISABELLE{\isacharunderscore}TOOLS}}}} setting.  Do not try to call the scripts directly
  from the shell.  Neither should you add the tool directories to your
  shell's search path!%
\end{isamarkuptext}%
\isamarkuptrue%
%
\isamarkupsubsubsection{Examples%
}
\isamarkuptrue%
%
\begin{isamarkuptext}%
Show the list of available documentation of the current Isabelle
  installation like this:

\begin{ttbox}
  isatool doc
\end{ttbox}

  View a certain document as follows:
\begin{ttbox}
  isatool doc isar-ref
\end{ttbox}

  Create an Isabelle session derived from HOL (see also
  \secref{sec:tool-mkdir} and \secref{sec:tool-make}):
\begin{ttbox}
  isatool mkdir HOL Test && isatool make
\end{ttbox}
  Note that \verb|isatool mkdir| is usually only invoked once;
  existing sessions (including document output etc.) are then updated
  by \verb|isatool make| alone.%
\end{isamarkuptext}%
\isamarkuptrue%
%
\isamarkupsection{The Isabelle interface wrapper \label{sec:interface}%
}
\isamarkuptrue%
%
\begin{isamarkuptext}%
Isabelle is a generic theorem prover, even w.r.t.\ its user
  interface.  The \indexdef{}{executable}{Isabelle}\hypertarget{executable.Isabelle}{\hyperlink{executable.Isabelle}{\mbox{\isa{\isatt{Isabelle}}}}} (or \indexdef{}{executable}{isabelle-interface}\hypertarget{executable.isabelle-interface}{\hyperlink{executable.isabelle-interface}{\mbox{\isa{\isatt{isabelle{\isacharminus}interface}}}}}) executable provides a uniform way for
  end-users to invoke a certain interface; which one to start is
  determined by the \indexref{}{setting}{ISABELLE\_INTERFACE}\hyperlink{setting.ISABELLE-INTERFACE}{\mbox{\isa{\isatt{ISABELLE{\isacharunderscore}INTERFACE}}}} setting
  variable, which should give a full path specification to the actual
  executable.

  Presently, the most prominent Isabelle interface is Proof
  General~\cite{proofgeneral}\index{user interface!Proof General}.
  The Proof General distribution includes an interface wrapper script
  for the regular Isar toplevel, see \verb|ProofGeneral/isar/interface|.  The canonical settings for
  Isabelle/Isar are as follows:

\begin{ttbox}
ISABELLE_INTERFACE=\$ISABELLE_HOME/contrib/ProofGeneral/isar/interface
PROOFGENERAL_OPTIONS=""
\end{ttbox}

  Thus \hyperlink{executable.Isabelle}{\mbox{\isa{\isatt{Isabelle}}}} would automatically invoke Emacs with
  proper setup of the Proof General Lisp packages.  There are some
  options available, such as \verb|-l| for passing the logic
  image to be used by default, or \verb|-m| to tune the
  standard print mode.  The \verb|-I| option allows to switch
  between the Isar and ML view, independently of the interface script
  being used.
  
  \medskip Note that the world may be also seen the other way round:
  Emacs may be started first (with proper setup of Proof General
  mode), and \hyperlink{executable.isabelle-process}{\mbox{\isa{\isatt{isabelle{\isacharminus}process}}}} run from within.  This
  requires further Emacs Lisp configuration, see the Proof General
  documentation \cite{proofgeneral} for more information.%
\end{isamarkuptext}%
\isamarkuptrue%
%
\isadelimtheory
%
\endisadelimtheory
%
\isatagtheory
\isacommand{end}\isamarkupfalse%
%
\endisatagtheory
{\isafoldtheory}%
%
\isadelimtheory
%
\endisadelimtheory
\end{isabellebody}%
%%% Local Variables:
%%% mode: latex
%%% TeX-master: "root"
%%% End:
