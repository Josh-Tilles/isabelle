
%$Id$

\chapter{Fonts and character encodings}

Using the print mode mechanism of Isabelle, variant forms of output become
very easy. As the canonical application of this feature, Pure and major
object-logics (FOL, ZF, HOL, HOLCF) support optional input and output of
proper mathematical symbols as built-in option.

Symbolic output is enabled by activating the ``\ttindex{symbols}'' print mode.
User interfaces (e.g.\ \texttt{isa-xterm}, see \S\ref{sec:interface}) usually
do this already by default.

\medskip Displaying non-standard characters requires special screen fonts. The
\texttt{installfonts} utility takes care of this (see
\S\ref{sec:tool-installfonts}). Furthermore, some {\ML} systems disallow
non-\textsc{ascii} characters in literal string constants.  This problem is
avoided by appropriate input filtering (see \S\ref{sec:tool-symbolinput}).

These things usually happen behind the scenes.  Users normally do not have to
read the explanations below, unless something really fails to work.


\section{Telling X11 about the Isabelle fonts --- \texttt{isatool installfonts}}
\label{sec:tool-installfonts}

The \tooldx{installfonts} utility ensures that your currently running X11
display server (as determined by the \texttt{DISPLAY} environment variable)
knows about the Isabelle fonts. Its usage is:
\begin{ttbox}
Usage: isatool installfonts

  Install the isabelle fonts into your X11 server.
  (May be safely called repeatedly.)
\end{ttbox}
Note that this need not be called manually under normal circumstances --- user
interfaces depending on the Isabelle fonts usually invoke
\texttt{installfonts} automatically.

\medskip As simple as this might appear to be, it is not! X11 fonts are a
surprisingly complicated matter. Depending on your network structure, fonts
might have to be installed differently. This has to be specified via the
\settdx{ISABELLE_INSTALLFONTS} variable in your local settings.

\medskip In the simplest situation, X11 is used only locally, i.e.\ the client
program (Isabelle) and the display server are run on the same machine. In that
case, most X11 display servers should be happy by being told about the
Isabelle fonts directory as follows:
\begin{ttbox}
ISABELLE_INSTALLFONTS="xset fp+ $ISABELLE_HOME/lib/fonts; xset fp rehash"%$
\end{ttbox}
The same also works for remote X11 sessions in a largely homogeneous network,
where any X11 display machine also mounts the Isabelle distribution under the
\emph{same} name as the client side.

\medskip Above method fails, though, if the display machine does have the font
files at the same location as the client. In case your server is a full
workstation with its own file system, you could in principle just copy the
fonts there and do an appropriate \texttt{xset~fp+} manually before running
the Isabelle interface. This is very awkward, of course. It is even impossible
for proper X11 terminals that do not have their own file system.

A much better solution is to have a \emph{font server} offer the Isabelle
fonts to any X11 display on the network.  There are already suitable servers
running at Munich and Cambridge. So in case you have a permanent Internet
connection to either site, you may just attach yourself as follows:
\begin{ttbox}
ISABELLE_INSTALLFONTS="xset fp+ tcp/isafonts.informatik.tu-muenchen.de:7200"
\end{ttbox}
or
\begin{ttbox}
ISABELLE_INSTALLFONTS="xset fp+ tcp/font-serv.cl.cam.ac.uk:7100"
\end{ttbox}

\medskip In the unfortunate case that neither local fonts work, nor accessing
our world-wide font service is practical, it might be best to start your own
in-house font service. This is in principle quite easy to setup. The program
is called \texttt{xfs} (sometimes just \texttt{fs)}, see the \texttt{man}
pages of your system. There is an example fontserver configuration available
in the \texttt{lib/fontserver} directory of the Isabelle distribution.


\section{Check for non-ASCII characters --- \texttt{isatool nonascii}}
\label{sec:tool-nonascii}

The \tooldx{nonascii} utility checks files for non-\textsc{ascii} characters:
\begin{ttbox}
Usage: nonascii [FILES|DIRS...]

Recursively find .thy/.ML files and check for non-\textsc{ascii}
characters.
\end{ttbox}
Under normal circumstances, non-\textsc{ascii} characters do not appear in
theories and proof scripts.


\section{Filtering non-ASCII characters --- \texttt{isatool symbolinput}}
\label{sec:tool-symbolinput}

Processing non-\textsc{ascii} text is notoriously difficult.  In particular,
some {\ML} systems reject character codes outside the range 32--127 as part of
literal string constants. In order to circumvent such restrictions, Isabelle
employs a general notation where glyphs are referred by some symbolic name
instead of their actual encoding: the general form is
\verb|\<|$charname$\verb|>|.

The \tooldx{symbolinput} utility converts an input stream encoded according to
the standard Isabelle font layout into pure \textsc{ascii} text. There is no
usage --- \texttt{symbolinput} just works from standard input to standard
output, without any options available.

\medskip For example, the non-\textsc{ascii} input string \texttt{"A $\land$ B
  $\lor$ C"} will be replaced by \verb|"A \\<and> B \\<or> C"|.  Note that the
\verb|\| are escaped, accommodating concrete {\ML} string syntax.

\medskip In many cases, it might be wise not to rely on symbolic characters
and avoid non-\textsc{ascii} text in files altogether (see also
\S\ref{sec:tool-nonascii}). Then symbolic syntax would be really optional,
with always suitable \textsc{ascii} representations available.  In syntax
definitions, symbols appear only in mixfix annotations (using the
\verb|\<|$charname$\verb|>| form), while proof texts are left in plain
\textsc{ascii}.  Thus users with \textsc{ascii}-only facilities will still be
able to read your files.

%%% Local Variables: 
%%% mode: latex
%%% TeX-master: "system"
%%% End: 
