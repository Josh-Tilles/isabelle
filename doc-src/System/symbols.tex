
% $Id$

\chapter{Standard Isabelle symbols}\label{app:symbols}

Isabelle supports an infinite number of non-ASCII symbols, which are
represented in source text as \verb,\<,$name$\verb,>, (where $name$ may be any
identifier).  It is left to front-end tools how to present these symbols to
the user.  The collection of predefined standard symbols given below is
available by default for Isabelle document output, due to appropriate
definitions of \verb,\isasym,$name$ for each \verb,\<,$name$\verb,>, in the
\verb,isabellesym.sty, file.  Most of these symbols are displayed properly in
Proof~General if used with the X-Symbol package.

Moreover, any single symbol (or ASCII character) may be prefixed by
\verb,\<^sup>, for superscript and \verb,\<^sub>, for subscript, such as
\verb,A\<^sup>\<star>, for \isa{A\isactrlsup{\isasymstar}}; the alternative
versions \verb,\<^isub>, and \verb,\<^isup>, are considered as quasi letters
and may be used within identifiers.  Sub- and superscripts that span a region
of text are marked up with \verb,\<^bsub>,\dots\verb,\<^esub>, and
\verb,\<^bsup>,\dots\verb,\<^esup>,, respectively.  Furthermore, all ASCII
characters and most other symbols may be printed in bold by prefixing
\verb,\<^bold>,, such as \verb,\<^bold>\<alpha>, for
\isa{\isactrlbold{\isasymalpha}}.  Note that \verb,\<^bold>, may \emph{not} be
combined with sub- or superscripts for single symbols.

Further details of Isabelle document preparation are covered in
chapter~\ref{ch:present}.

\begin{center}
  \begin{isabellebody}
    \input{syms}  
  \end{isabellebody}
\end{center}

%%% Local Variables: 
%%% mode: latex
%%% TeX-master: "system"
%%% End: 
