\begin{isabelle}%
%
\begin{isamarkuptext}%
Here is a simple example, the Fibonacci function:%
\end{isamarkuptext}%
\isacommand{consts}\ fib\ {\isacharcolon}{\isacharcolon}\ {\isachardoublequote}nat\ {\isasymRightarrow}\ nat{\isachardoublequote}\isanewline
\isacommand{recdef}\ fib\ {\isachardoublequote}measure{\isacharparenleft}{\isasymlambda}n{\isachardot}\ n{\isacharparenright}{\isachardoublequote}\isanewline
\ \ {\isachardoublequote}fib\ \isadigit{0}\ {\isacharequal}\ \isadigit{0}{\isachardoublequote}\isanewline
\ \ {\isachardoublequote}fib\ \isadigit{1}\ {\isacharequal}\ \isadigit{1}{\isachardoublequote}\isanewline
\ \ {\isachardoublequote}fib\ {\isacharparenleft}Suc{\isacharparenleft}Suc\ x{\isacharparenright}{\isacharparenright}\ {\isacharequal}\ fib\ x\ {\isacharplus}\ fib\ {\isacharparenleft}Suc\ x{\isacharparenright}{\isachardoublequote}%
\begin{isamarkuptext}%
\noindent
The definition of \isa{fib} is accompanied by a \bfindex{measure function}
\isa{{\isasymlambda}\mbox{n}{\isachardot}\ \mbox{n}} which maps the argument of \isa{fib} to a
natural number. The requirement is that in each equation the measure of the
argument on the left-hand side is strictly greater than the measure of the
argument of each recursive call. In the case of \isa{fib} this is
obviously true because the measure function is the identity and
\isa{Suc\ {\isacharparenleft}Suc\ \mbox{x}{\isacharparenright}} is strictly greater than both \isa{x} and
\isa{Suc\ \mbox{x}}.

Slightly more interesting is the insertion of a fixed element
between any two elements of a list:%
\end{isamarkuptext}%
\isacommand{consts}\ sep\ {\isacharcolon}{\isacharcolon}\ {\isachardoublequote}{\isacharprime}a\ {\isacharasterisk}\ {\isacharprime}a\ list\ {\isasymRightarrow}\ {\isacharprime}a\ list{\isachardoublequote}\isanewline
\isacommand{recdef}\ sep\ {\isachardoublequote}measure\ {\isacharparenleft}{\isasymlambda}{\isacharparenleft}a{\isacharcomma}xs{\isacharparenright}{\isachardot}\ length\ xs{\isacharparenright}{\isachardoublequote}\isanewline
\ \ {\isachardoublequote}sep{\isacharparenleft}a{\isacharcomma}\ {\isacharbrackleft}{\isacharbrackright}{\isacharparenright}\ \ \ \ \ {\isacharequal}\ {\isacharbrackleft}{\isacharbrackright}{\isachardoublequote}\isanewline
\ \ {\isachardoublequote}sep{\isacharparenleft}a{\isacharcomma}\ {\isacharbrackleft}x{\isacharbrackright}{\isacharparenright}\ \ \ \ {\isacharequal}\ {\isacharbrackleft}x{\isacharbrackright}{\isachardoublequote}\isanewline
\ \ {\isachardoublequote}sep{\isacharparenleft}a{\isacharcomma}\ x{\isacharhash}y{\isacharhash}zs{\isacharparenright}\ {\isacharequal}\ x\ {\isacharhash}\ a\ {\isacharhash}\ sep{\isacharparenleft}a{\isacharcomma}y{\isacharhash}zs{\isacharparenright}{\isachardoublequote}%
\begin{isamarkuptext}%
\noindent
This time the measure is the length of the list, which decreases with the
recursive call; the first component of the argument tuple is irrelevant.

Pattern matching need not be exhaustive:%
\end{isamarkuptext}%
\isacommand{consts}\ last\ {\isacharcolon}{\isacharcolon}\ {\isachardoublequote}{\isacharprime}a\ list\ {\isasymRightarrow}\ {\isacharprime}a{\isachardoublequote}\isanewline
\isacommand{recdef}\ last\ {\isachardoublequote}measure\ {\isacharparenleft}{\isasymlambda}xs{\isachardot}\ length\ xs{\isacharparenright}{\isachardoublequote}\isanewline
\ \ {\isachardoublequote}last\ {\isacharbrackleft}x{\isacharbrackright}\ \ \ \ \ \ {\isacharequal}\ x{\isachardoublequote}\isanewline
\ \ {\isachardoublequote}last\ {\isacharparenleft}x{\isacharhash}y{\isacharhash}zs{\isacharparenright}\ {\isacharequal}\ last\ {\isacharparenleft}y{\isacharhash}zs{\isacharparenright}{\isachardoublequote}%
\begin{isamarkuptext}%
Overlapping patterns are disambiguated by taking the order of equations into
account, just as in functional programming:%
\end{isamarkuptext}%
\isacommand{consts}\ sep\isadigit{1}\ {\isacharcolon}{\isacharcolon}\ {\isachardoublequote}{\isacharprime}a\ {\isacharasterisk}\ {\isacharprime}a\ list\ {\isasymRightarrow}\ {\isacharprime}a\ list{\isachardoublequote}\isanewline
\isacommand{recdef}\ sep\isadigit{1}\ {\isachardoublequote}measure\ {\isacharparenleft}{\isasymlambda}{\isacharparenleft}a{\isacharcomma}xs{\isacharparenright}{\isachardot}\ length\ xs{\isacharparenright}{\isachardoublequote}\isanewline
\ \ {\isachardoublequote}sep\isadigit{1}{\isacharparenleft}a{\isacharcomma}\ x{\isacharhash}y{\isacharhash}zs{\isacharparenright}\ {\isacharequal}\ x\ {\isacharhash}\ a\ {\isacharhash}\ sep\isadigit{1}{\isacharparenleft}a{\isacharcomma}y{\isacharhash}zs{\isacharparenright}{\isachardoublequote}\isanewline
\ \ {\isachardoublequote}sep\isadigit{1}{\isacharparenleft}a{\isacharcomma}\ xs{\isacharparenright}\ \ \ \ \ {\isacharequal}\ xs{\isachardoublequote}%
\begin{isamarkuptext}%
\noindent
This defines exactly the same function as \isa{sep} above, i.e.\
\isa{sep1 = sep}.

\begin{warn}
  \isacommand{recdef} only takes the first argument of a (curried)
  recursive function into account. This means both the termination measure
  and pattern matching can only use that first argument. In general, you will
  therefore have to combine several arguments into a tuple. In case only one
  argument is relevant for termination, you can also rearrange the order of
  arguments as in the following definition:
\end{warn}%
\end{isamarkuptext}%
\isacommand{consts}\ sep\isadigit{2}\ {\isacharcolon}{\isacharcolon}\ {\isachardoublequote}{\isacharprime}a\ list\ {\isasymRightarrow}\ {\isacharprime}a\ {\isasymRightarrow}\ {\isacharprime}a\ list{\isachardoublequote}\isanewline
\isacommand{recdef}\ sep\isadigit{2}\ {\isachardoublequote}measure\ length{\isachardoublequote}\isanewline
\ \ {\isachardoublequote}sep\isadigit{2}\ {\isacharparenleft}x{\isacharhash}y{\isacharhash}zs{\isacharparenright}\ {\isacharequal}\ {\isacharparenleft}{\isasymlambda}a{\isachardot}\ x\ {\isacharhash}\ a\ {\isacharhash}\ sep\isadigit{2}\ zs\ a{\isacharparenright}{\isachardoublequote}\isanewline
\ \ {\isachardoublequote}sep\isadigit{2}\ xs\ \ \ \ \ \ \ {\isacharequal}\ {\isacharparenleft}{\isasymlambda}a{\isachardot}\ xs{\isacharparenright}{\isachardoublequote}%
\begin{isamarkuptext}%
Because of its pattern-matching syntax, \isacommand{recdef} is also useful
for the definition of non-recursive functions:%
\end{isamarkuptext}%
\isacommand{consts}\ swap\isadigit{1}\isadigit{2}\ {\isacharcolon}{\isacharcolon}\ {\isachardoublequote}{\isacharprime}a\ list\ {\isasymRightarrow}\ {\isacharprime}a\ list{\isachardoublequote}\isanewline
\isacommand{recdef}\ swap\isadigit{1}\isadigit{2}\ {\isachardoublequote}{\isacharbraceleft}{\isacharbraceright}{\isachardoublequote}\isanewline
\ \ {\isachardoublequote}swap\isadigit{1}\isadigit{2}\ {\isacharparenleft}x{\isacharhash}y{\isacharhash}zs{\isacharparenright}\ {\isacharequal}\ y{\isacharhash}x{\isacharhash}zs{\isachardoublequote}\isanewline
\ \ {\isachardoublequote}swap\isadigit{1}\isadigit{2}\ zs\ \ \ \ \ \ \ {\isacharequal}\ zs{\isachardoublequote}%
\begin{isamarkuptext}%
\noindent
For non-recursive functions the termination measure degenerates to the empty
set \isa{\{\}}.%
\end{isamarkuptext}%
\end{isabelle}%
%%% Local Variables:
%%% mode: latex
%%% TeX-master: "root"
%%% End:
