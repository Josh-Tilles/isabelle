\begin{isabelle}%
%
\begin{isamarkuptext}%
In \S\ref{sec:nested-datatype} we defined the datatype of terms%
\end{isamarkuptext}%
\isacommand{datatype}\ {\isacharparenleft}{\isacharprime}a{\isacharcomma}{\isacharprime}b{\isacharparenright}{\isachardoublequote}term{\isachardoublequote}\ {\isacharequal}\ Var\ {\isacharprime}a\ {\isacharbar}\ App\ {\isacharprime}b\ {\isachardoublequote}{\isacharparenleft}{\isacharprime}a{\isacharcomma}{\isacharprime}b{\isacharparenright}term\ list{\isachardoublequote}%
\begin{isamarkuptext}%
\noindent
and closed with the observation that the associated schema for the definition
of primitive recursive functions leads to overly verbose definitions. Moreover,
if you have worked exercise~\ref{ex:trev-trev} you will have noticed that
you needed to reprove many lemmas reminiscent of similar lemmas about
\isa{rev}. We will now show you how \isacommand{recdef} can simplify
definitions and proofs about nested recursive datatypes. As an example we
chose exercise~\ref{ex:trev-trev}:

FIXME: declare trev now!%
\end{isamarkuptext}%
\end{isabelle}%
%%% Local Variables:
%%% mode: latex
%%% TeX-master: "root"
%%% End:
