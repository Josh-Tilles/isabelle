%
\begin{isabellebody}%
\def\isabellecontext{termination}%
%
\isadelimtheory
%
\endisadelimtheory
%
\isatagtheory
%
\endisatagtheory
{\isafoldtheory}%
%
\isadelimtheory
%
\endisadelimtheory
%
\begin{isamarkuptext}%
When a function~$f$ is defined via \isacommand{recdef}, Isabelle tries to prove
its termination with the help of the user-supplied measure.  Each of the examples
above is simple enough that Isabelle can automatically prove that the
argument's measure decreases in each recursive call. As a result,
$f$\isa{{\isachardot}simps} will contain the defining equations (or variants derived
from them) as theorems. For example, look (via \isacommand{thm}) at
\isa{sep{\isachardot}simps} and \isa{sep{\isadigit{1}}{\isachardot}simps} to see that they define
the same function. What is more, those equations are automatically declared as
simplification rules.

Isabelle may fail to prove the termination condition for some
recursive call.  Let us try to define Quicksort:%
\end{isamarkuptext}%
\isamarkuptrue%
\isacommand{consts}\isamarkupfalse%
\ qs\ {\isacharcolon}{\isacharcolon}\ {\isachardoublequoteopen}nat\ list\ {\isasymRightarrow}\ nat\ list{\isachardoublequoteclose}\isanewline
\isacommand{recdef}\isamarkupfalse%
\ qs\ {\isachardoublequoteopen}measure\ length{\isachardoublequoteclose}\isanewline
\ {\isachardoublequoteopen}qs\ {\isacharbrackleft}{\isacharbrackright}\ {\isacharequal}\ {\isacharbrackleft}{\isacharbrackright}{\isachardoublequoteclose}\isanewline
\ {\isachardoublequoteopen}qs{\isacharparenleft}x{\isacharhash}xs{\isacharparenright}\ {\isacharequal}\ qs{\isacharparenleft}filter\ {\isacharparenleft}{\isasymlambda}y{\isachardot}\ y{\isasymle}x{\isacharparenright}\ xs{\isacharparenright}\ {\isacharat}\ {\isacharbrackleft}x{\isacharbrackright}\ {\isacharat}\ qs{\isacharparenleft}filter\ {\isacharparenleft}{\isasymlambda}y{\isachardot}\ x{\isacharless}y{\isacharparenright}\ xs{\isacharparenright}{\isachardoublequoteclose}%
\begin{isamarkuptext}%
\noindent where \isa{filter} is predefined and \isa{filter\ P\ xs}
is the list of elements of \isa{xs} satisfying \isa{P}.
This definition of \isa{qs} fails, and Isabelle prints an error message
showing you what it was unable to prove:
\begin{isabelle}%
\ \ \ \ \ length\ {\isacharparenleft}filter\ {\isachardot}{\isachardot}{\isachardot}\ xs{\isacharparenright}\ {\isacharless}\ Suc\ {\isacharparenleft}length\ xs{\isacharparenright}%
\end{isabelle}
We can either prove this as a separate lemma, or try to figure out which
existing lemmas may help. We opt for the second alternative. The theory of
lists contains the simplification rule \isa{length\ {\isacharparenleft}filter\ P\ xs{\isacharparenright}\ {\isasymle}\ length\ xs},
which is what we need, provided we turn \mbox{\isa{{\isacharless}\ Suc}}
into
\isa{{\isasymle}} so that the rule applies. Lemma
\isa{less{\isacharunderscore}Suc{\isacharunderscore}eq{\isacharunderscore}le} does just that: \isa{{\isacharparenleft}m\ {\isacharless}\ Suc\ n{\isacharparenright}\ {\isacharequal}\ {\isacharparenleft}m\ {\isasymle}\ n{\isacharparenright}}.

Now we retry the above definition but supply the lemma(s) just found (or
proved). Because \isacommand{recdef}'s termination prover involves
simplification, we include in our second attempt a hint: the
\attrdx{recdef_simp} attribute says to use \isa{less{\isacharunderscore}Suc{\isacharunderscore}eq{\isacharunderscore}le} as a
simplification rule.\cmmdx{hints}%
\end{isamarkuptext}%
\isamarkuptrue%
\isacommand{recdef}\isamarkupfalse%
\ qs\ {\isachardoublequoteopen}measure\ length{\isachardoublequoteclose}\isanewline
\ {\isachardoublequoteopen}qs\ {\isacharbrackleft}{\isacharbrackright}\ {\isacharequal}\ {\isacharbrackleft}{\isacharbrackright}{\isachardoublequoteclose}\isanewline
\ {\isachardoublequoteopen}qs{\isacharparenleft}x{\isacharhash}xs{\isacharparenright}\ {\isacharequal}\ qs{\isacharparenleft}filter\ {\isacharparenleft}{\isasymlambda}y{\isachardot}\ y{\isasymle}x{\isacharparenright}\ xs{\isacharparenright}\ {\isacharat}\ {\isacharbrackleft}x{\isacharbrackright}\ {\isacharat}\ qs{\isacharparenleft}filter\ {\isacharparenleft}{\isasymlambda}y{\isachardot}\ x{\isacharless}y{\isacharparenright}\ xs{\isacharparenright}{\isachardoublequoteclose}\isanewline
{\isacharparenleft}\isakeyword{hints}\ recdef{\isacharunderscore}simp{\isacharcolon}\ less{\isacharunderscore}Suc{\isacharunderscore}eq{\isacharunderscore}le{\isacharparenright}%
\begin{isamarkuptext}%
\noindent
This time everything works fine. Now \isa{qs{\isachardot}simps} contains precisely
the stated recursion equations for \isa{qs} and they have become
simplification rules.
Thus we can automatically prove results such as this one:%
\end{isamarkuptext}%
\isamarkuptrue%
\isacommand{theorem}\isamarkupfalse%
\ {\isachardoublequoteopen}qs{\isacharbrackleft}{\isadigit{2}}{\isacharcomma}{\isadigit{3}}{\isacharcomma}{\isadigit{0}}{\isacharbrackright}\ {\isacharequal}\ qs{\isacharbrackleft}{\isadigit{3}}{\isacharcomma}{\isadigit{0}}{\isacharcomma}{\isadigit{2}}{\isacharbrackright}{\isachardoublequoteclose}\isanewline
%
\isadelimproof
%
\endisadelimproof
%
\isatagproof
\isacommand{apply}\isamarkupfalse%
{\isacharparenleft}simp{\isacharparenright}\isanewline
\isacommand{done}\isamarkupfalse%
%
\endisatagproof
{\isafoldproof}%
%
\isadelimproof
%
\endisadelimproof
%
\begin{isamarkuptext}%
\noindent
More exciting theorems require induction, which is discussed below.

If the termination proof requires a lemma that is of general use, you can
turn it permanently into a simplification rule, in which case the above
\isacommand{hint} is not necessary. But in the case of
\isa{less{\isacharunderscore}Suc{\isacharunderscore}eq{\isacharunderscore}le} this would be of dubious value.%
\end{isamarkuptext}%
\isamarkuptrue%
%
\isadelimtheory
%
\endisadelimtheory
%
\isatagtheory
%
\endisatagtheory
{\isafoldtheory}%
%
\isadelimtheory
%
\endisadelimtheory
\end{isabellebody}%
%%% Local Variables:
%%% mode: latex
%%% TeX-master: "root"
%%% End:
