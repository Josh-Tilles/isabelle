%
\begin{isabellebody}%
\def\isabellecontext{Nested1}%
%
\begin{isamarkuptext}%
\noindent
Although the definition of \isa{trev} is quite natural, we will have
overcome a minor difficulty in convincing Isabelle of is termination.
It is precisely this difficulty that is the \textit{raison d'\^etre} of
this subsection.

Defining \isa{trev} by \isacommand{recdef} rather than \isacommand{primrec}
simplifies matters because we are now free to use the recursion equation
suggested at the end of \S\ref{sec:nested-datatype}:%
\end{isamarkuptext}%
\isacommand{recdef}\ trev\ {\isachardoublequote}measure\ size{\isachardoublequote}\isanewline
\ {\isachardoublequote}trev\ {\isacharparenleft}Var\ x{\isacharparenright}\ \ \ \ {\isacharequal}\ Var\ x{\isachardoublequote}\isanewline
\ {\isachardoublequote}trev\ {\isacharparenleft}App\ f\ ts{\isacharparenright}\ {\isacharequal}\ App\ f\ {\isacharparenleft}rev{\isacharparenleft}map\ trev\ ts{\isacharparenright}{\isacharparenright}{\isachardoublequote}%
\begin{isamarkuptext}%
\noindent
Remember that function \isa{size} is defined for each \isacommand{datatype}.
However, the definition does not succeed. Isabelle complains about an
unproved termination condition
\begin{isabelle}%
\ \ \ \ \ t\ {\isasymin}\ set\ ts\ {\isasymlongrightarrow}\ size\ t\ {\isacharless}\ Suc\ {\isacharparenleft}term{\isacharunderscore}list{\isacharunderscore}size\ ts{\isacharparenright}%
\end{isabelle}
where \isa{set} returns the set of elements of a list
and \isa{term{\isacharunderscore}list{\isacharunderscore}size\ {\isacharcolon}{\isacharcolon}\ term\ list\ {\isasymRightarrow}\ nat} is an auxiliary
function automatically defined by Isabelle
(when \isa{term} was defined).  First we have to understand why the
recursive call of \isa{trev} underneath \isa{map} leads to the above
condition. The reason is that \isacommand{recdef} ``knows'' that \isa{map}
will apply \isa{trev} only to elements of \isa{ts}. Thus the above
condition expresses that the size of the argument \isa{t\ {\isasymin}\ set\ ts} of any
recursive call of \isa{trev} is strictly less than \isa{size\ {\isacharparenleft}App\ f\ ts{\isacharparenright}\ {\isacharequal}\ Suc\ {\isacharparenleft}term{\isacharunderscore}list{\isacharunderscore}size\ ts{\isacharparenright}}.  We will now prove the termination condition and
continue with our definition.  Below we return to the question of how
\isacommand{recdef} ``knows'' about \isa{map}.%
\end{isamarkuptext}%
\end{isabellebody}%
%%% Local Variables:
%%% mode: latex
%%% TeX-master: "root"
%%% End:
