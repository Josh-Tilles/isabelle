%
\begin{isabellebody}%
\def\isabellecontext{Nested{\isadigit{2}}}%
%
\isadelimtheory
%
\endisadelimtheory
%
\isatagtheory
%
\endisatagtheory
{\isafoldtheory}%
%
\isadelimtheory
\isanewline
%
\endisadelimtheory
\isamarkupfalse%
\isacommand{lemma}\ {\isacharbrackleft}simp{\isacharbrackright}{\isacharcolon}\ {\isachardoublequote}t\ {\isasymin}\ set\ ts\ {\isasymlongrightarrow}\ size\ t\ {\isacharless}\ Suc{\isacharparenleft}term{\isacharunderscore}list{\isacharunderscore}size\ ts{\isacharparenright}{\isachardoublequote}\isanewline
%
\isadelimproof
%
\endisadelimproof
%
\isatagproof
\isamarkupfalse%
\isacommand{by}{\isacharparenleft}induct{\isacharunderscore}tac\ ts{\isacharcomma}\ auto{\isacharparenright}%
\endisatagproof
{\isafoldproof}%
%
\isadelimproof
%
\endisadelimproof
\isamarkuptrue%
%
\begin{isamarkuptext}%
\noindent
By making this theorem a simplification rule, \isacommand{recdef}
applies it automatically and the definition of \isa{trev}
succeeds now. As a reward for our effort, we can now prove the desired
lemma directly.  We no longer need the verbose
induction schema for type \isa{term} and can use the simpler one arising from
\isa{trev}:%
\end{isamarkuptext}%
\isamarkupfalse%
\isacommand{lemma}\ {\isachardoublequote}trev{\isacharparenleft}trev\ t{\isacharparenright}\ {\isacharequal}\ t{\isachardoublequote}\isanewline
%
\isadelimproof
%
\endisadelimproof
%
\isatagproof
\isamarkupfalse%
\isacommand{apply}{\isacharparenleft}induct{\isacharunderscore}tac\ t\ rule{\isacharcolon}\ trev{\isachardot}induct{\isacharparenright}\isamarkuptrue%
%
\begin{isamarkuptxt}%
\begin{isabelle}%
\ {\isadigit{1}}{\isachardot}\ {\isasymAnd}x{\isachardot}\ trev\ {\isacharparenleft}trev\ {\isacharparenleft}Var\ x{\isacharparenright}{\isacharparenright}\ {\isacharequal}\ Var\ x\isanewline
\ {\isadigit{2}}{\isachardot}\ {\isasymAnd}f\ ts{\isachardot}\isanewline
\isaindent{\ {\isadigit{2}}{\isachardot}\ \ \ \ }{\isasymforall}x{\isachardot}\ x\ {\isasymin}\ set\ ts\ {\isasymlongrightarrow}\ trev\ {\isacharparenleft}trev\ x{\isacharparenright}\ {\isacharequal}\ x\ {\isasymLongrightarrow}\isanewline
\isaindent{\ {\isadigit{2}}{\isachardot}\ \ \ \ }trev\ {\isacharparenleft}trev\ {\isacharparenleft}App\ f\ ts{\isacharparenright}{\isacharparenright}\ {\isacharequal}\ App\ f\ ts%
\end{isabelle}
Both the base case and the induction step fall to simplification:%
\end{isamarkuptxt}%
\isamarkupfalse%
\isacommand{by}{\isacharparenleft}simp{\isacharunderscore}all\ add{\isacharcolon}\ rev{\isacharunderscore}map\ sym{\isacharbrackleft}OF\ map{\isacharunderscore}compose{\isacharbrackright}\ cong{\isacharcolon}\ map{\isacharunderscore}cong{\isacharparenright}%
\endisatagproof
{\isafoldproof}%
%
\isadelimproof
%
\endisadelimproof
\isamarkuptrue%
%
\begin{isamarkuptext}%
\noindent
If the proof of the induction step mystifies you, we recommend that you go through
the chain of simplification steps in detail; you will probably need the help of
\isa{trace{\isacharunderscore}simp}. Theorem \isa{map{\isacharunderscore}cong} is discussed below.
%\begin{quote}
%{term[display]"trev(trev(App f ts))"}\\
%{term[display]"App f (rev(map trev (rev(map trev ts))))"}\\
%{term[display]"App f (map trev (rev(rev(map trev ts))))"}\\
%{term[display]"App f (map trev (map trev ts))"}\\
%{term[display]"App f (map (trev o trev) ts)"}\\
%{term[display]"App f (map (%x. x) ts)"}\\
%{term[display]"App f ts"}
%\end{quote}

The definition of \isa{trev} above is superior to the one in
\S\ref{sec:nested-datatype} because it uses \isa{rev}
and lets us use existing facts such as \hbox{\isa{rev\ {\isacharparenleft}rev\ xs{\isacharparenright}\ {\isacharequal}\ xs}}.
Thus this proof is a good example of an important principle:
\begin{quote}
\emph{Chose your definitions carefully\\
because they determine the complexity of your proofs.}
\end{quote}

Let us now return to the question of how \isacommand{recdef} can come up with
sensible termination conditions in the presence of higher-order functions
like \isa{map}. For a start, if nothing were known about \isa{map}, then
\isa{map\ trev\ ts} might apply \isa{trev} to arbitrary terms, and thus
\isacommand{recdef} would try to prove the unprovable \isa{size\ t\ {\isacharless}\ Suc\ {\isacharparenleft}term{\isacharunderscore}list{\isacharunderscore}size\ ts{\isacharparenright}}, without any assumption about \isa{t}.  Therefore
\isacommand{recdef} has been supplied with the congruence theorem
\isa{map{\isacharunderscore}cong}:
\begin{isabelle}%
\ \ \ \ \ {\isasymlbrakk}xs\ {\isacharequal}\ ys{\isacharsemicolon}\ {\isasymAnd}x{\isachardot}\ x\ {\isasymin}\ set\ ys\ {\isasymLongrightarrow}\ f\ x\ {\isacharequal}\ g\ x{\isasymrbrakk}\isanewline
\isaindent{\ \ \ \ \ }{\isasymLongrightarrow}\ map\ f\ xs\ {\isacharequal}\ map\ g\ ys%
\end{isabelle}
Its second premise expresses that in \isa{map\ f\ xs},
function \isa{f} is only applied to elements of list \isa{xs}.  Congruence
rules for other higher-order functions on lists are similar.  If you get
into a situation where you need to supply \isacommand{recdef} with new
congruence rules, you can append a hint after the end of
the recursion equations:\cmmdx{hints}%
\end{isamarkuptext}%
{\isacharparenleft}\isakeyword{hints}\ recdef{\isacharunderscore}cong{\isacharcolon}\ map{\isacharunderscore}cong{\isacharparenright}\isamarkuptrue%
%
\begin{isamarkuptext}%
\noindent
Or you can declare them globally
by giving them the \attrdx{recdef_cong} attribute:%
\end{isamarkuptext}%
\isamarkupfalse%
\isacommand{declare}\ map{\isacharunderscore}cong{\isacharbrackleft}recdef{\isacharunderscore}cong{\isacharbrackright}\isamarkuptrue%
%
\begin{isamarkuptext}%
The \isa{cong} and \isa{recdef{\isacharunderscore}cong} attributes are
intentionally kept apart because they control different activities, namely
simplification and making recursive definitions.
%The simplifier's congruence rules cannot be used by recdef.
%For example the weak congruence rules for if and case would prevent
%recdef from generating sensible termination conditions.%
\end{isamarkuptext}%
%
\isadelimtheory
%
\endisadelimtheory
%
\isatagtheory
%
\endisatagtheory
{\isafoldtheory}%
%
\isadelimtheory
%
\endisadelimtheory
\end{isabellebody}%
%%% Local Variables:
%%% mode: latex
%%% TeX-master: "root"
%%% End:
