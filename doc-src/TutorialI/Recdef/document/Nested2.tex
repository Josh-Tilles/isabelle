%
\begin{isabellebody}%
%
\begin{isamarkuptext}%
\noindent
The termintion condition is easily proved by induction:%
\end{isamarkuptext}%
\isacommand{lemma}\ {\isacharbrackleft}simp{\isacharbrackright}{\isacharcolon}\ {\isachardoublequote}t\ {\isasymin}\ set\ ts\ {\isasymlongrightarrow}\ size\ t\ {\isacharless}\ Suc{\isacharparenleft}term{\isacharunderscore}size\ ts{\isacharparenright}{\isachardoublequote}\isanewline
\isacommand{by}{\isacharparenleft}induct{\isacharunderscore}tac\ ts{\isacharcomma}\ auto{\isacharparenright}%
\begin{isamarkuptext}%
\noindent
By making this theorem a simplification rule, \isacommand{recdef}
applies it automatically and the above definition of \isa{trev}
succeeds now. As a reward for our effort, we can now prove the desired
lemma directly. The key is the fact that we no longer need the verbose
induction schema for type \isa{term} but the simpler one arising from
\isa{trev}:%
\end{isamarkuptext}%
\isacommand{lemmas}\ {\isacharbrackleft}cong{\isacharbrackright}\ {\isacharequal}\ map{\isacharunderscore}cong\isanewline
\isacommand{lemma}\ {\isachardoublequote}trev{\isacharparenleft}trev\ t{\isacharparenright}\ {\isacharequal}\ t{\isachardoublequote}\isanewline
\isacommand{apply}{\isacharparenleft}induct{\isacharunderscore}tac\ t\ rule{\isacharcolon}trev{\isachardot}induct{\isacharparenright}%
\begin{isamarkuptxt}%
\noindent
This leaves us with a trivial base case \isa{trev\ {\isacharparenleft}trev\ {\isacharparenleft}Var\ \mbox{x}{\isacharparenright}{\isacharparenright}\ {\isacharequal}\ Var\ \mbox{x}} and the step case
\begin{quote}

\begin{isabelle}%
{\isasymforall}\mbox{t}{\isachardot}\ \mbox{t}\ {\isasymin}\ set\ \mbox{ts}\ {\isasymlongrightarrow}\ trev\ {\isacharparenleft}trev\ \mbox{t}{\isacharparenright}\ {\isacharequal}\ \mbox{t}\ {\isasymLongrightarrow}\isanewline
trev\ {\isacharparenleft}trev\ {\isacharparenleft}App\ \mbox{f}\ \mbox{ts}{\isacharparenright}{\isacharparenright}\ {\isacharequal}\ App\ \mbox{f}\ \mbox{ts}
\end{isabelle}%

\end{quote}
both of which are solved by simplification:%
\end{isamarkuptxt}%
\isacommand{by}{\isacharparenleft}simp{\isacharunderscore}all\ add{\isacharcolon}rev{\isacharunderscore}map\ sym{\isacharbrackleft}OF\ map{\isacharunderscore}compose{\isacharbrackright}{\isacharparenright}%
\begin{isamarkuptext}%
\noindent
If the proof of the induction step mystifies you, we recommend to go through
the chain of simplification steps in detail, probably with the help of
\isa{trace\_simp}.
%\begin{quote}
%{term[display]"trev(trev(App f ts))"}\\
%{term[display]"App f (rev(map trev (rev(map trev ts))))"}\\
%{term[display]"App f (map trev (rev(rev(map trev ts))))"}\\
%{term[display]"App f (map trev (map trev ts))"}\\
%{term[display]"App f (map (trev o trev) ts)"}\\
%{term[display]"App f (map (%x. x) ts)"}\\
%{term[display]"App f ts"}
%\end{quote}

The above definition of \isa{trev} is superior to the one in \S\ref{sec:nested-datatype}
because it brings \isa{rev} into play, about which already know a lot, in particular
\isa{rev\ {\isacharparenleft}rev\ \mbox{xs}{\isacharparenright}\ {\isacharequal}\ \mbox{xs}}.
Thus this proof is a good example of an important principle:
\begin{quote}
\emph{Chose your definitions carefully\\
because they determine the complexity of your proofs.}
\end{quote}

Let us now return to the question of how \isacommand{recdef} can come up with
sensible termination conditions in the presence of higher-order functions
like \isa{map}. For a start, if nothing were known about \isa{map},
\isa{map\ trev\ \mbox{ts}} might apply \isa{trev} to arbitrary terms, and thus
\isacommand{recdef} would try to prove the unprovable \isa{size\ \mbox{t}\ {\isacharless}\ Suc\ {\isacharparenleft}term{\isacharunderscore}size\ \mbox{ts}{\isacharparenright}}, without any assumption about \isa{t}.  Therefore
\isacommand{recdef} has been supplied with the congruence theorem
\isa{map\_cong}:
\begin{quote}

\begin{isabelle}%
{\isasymlbrakk}\mbox{xs}\ {\isacharequal}\ \mbox{ys}{\isacharsemicolon}\ {\isasymAnd}\mbox{x}{\isachardot}\ \mbox{x}\ {\isasymin}\ set\ \mbox{ys}\ {\isasymLongrightarrow}\ \mbox{f}\ \mbox{x}\ {\isacharequal}\ \mbox{g}\ \mbox{x}{\isasymrbrakk}\isanewline
{\isasymLongrightarrow}\ map\ \mbox{f}\ \mbox{xs}\ {\isacharequal}\ map\ \mbox{g}\ \mbox{ys}
\end{isabelle}%

\end{quote}
Its second premise expresses (indirectly) that the second argument of
\isa{map} is only applied to elements of its third argument. Congruence
rules for other higher-order functions on lists would look very similar but
have not been proved yet because they were never needed. If you get into a
situation where you need to supply \isacommand{recdef} with new congruence
rules, you can either append the line
\begin{ttbox}
congs <congruence rules>
\end{ttbox}
to the specific occurrence of \isacommand{recdef} or declare them globally:
\begin{ttbox}
lemmas [????????] = <congruence rules>
\end{ttbox}

Note that \isacommand{recdef} feeds on exactly the same \emph{kind} of
congruence rules as the simplifier (\S\ref{sec:simp-cong}) but that
declaring a congruence rule for the simplifier does not make it
available to \isacommand{recdef}, and vice versa. This is intentional.%
\end{isamarkuptext}%
\end{isabellebody}%
%%% Local Variables:
%%% mode: latex
%%% TeX-master: "root"
%%% End:
