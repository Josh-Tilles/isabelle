%
\begin{isabellebody}%
\def\isabellecontext{Numbers}%
\isanewline
\isacommand{theory}\ Numbers\ {\isacharequal}\ Main{\isacharcolon}\isanewline
\isanewline
\isacommand{ML}\ {\isachardoublequote}Pretty{\isachardot}setmargin\ {\isadigit{6}}{\isadigit{4}}{\isachardoublequote}%
\begin{isamarkuptext}%
numeric literals; default simprules; can re-orient%
\end{isamarkuptext}%
\isacommand{lemma}\ {\isachardoublequote}{\isacharhash}{\isadigit{2}}\ {\isacharasterisk}\ m\ {\isacharequal}\ m\ {\isacharplus}\ m{\isachardoublequote}\isanewline
\isacommand{oops}%
\begin{isamarkuptext}%
proof\ {\isacharparenleft}prove{\isacharparenright}{\isacharcolon}\ step\ {\isadigit{0}}\isanewline
\isanewline
goal\ {\isacharparenleft}lemma{\isacharparenright}{\isacharcolon}\isanewline
{\isacharparenleft}{\isacharhash}{\isadigit{2}}{\isasymColon}{\isacharprime}a{\isacharparenright}\ {\isacharasterisk}\ m\ {\isacharequal}\ m\ {\isacharplus}\ m\isanewline
\ {\isadigit{1}}{\isachardot}\ {\isacharparenleft}{\isacharhash}{\isadigit{2}}{\isasymColon}{\isacharprime}a{\isacharparenright}\ {\isacharasterisk}\ m\ {\isacharequal}\ m\ {\isacharplus}\ m


\begin{isabelle}%
\ \ \ \ \ {\isacharhash}{\isadigit{0}}\ {\isacharequal}\ {\isadigit{0}}%
\end{isabelle}
\rulename{numeral_0_eq_0}

\begin{isabelle}%
\ \ \ \ \ {\isacharhash}{\isadigit{1}}\ {\isacharequal}\ {\isadigit{1}}%
\end{isabelle}
\rulename{numeral_1_eq_1}

\begin{isabelle}%
\ \ \ \ \ {\isacharhash}{\isadigit{2}}\ {\isacharplus}\ n\ {\isacharequal}\ Suc\ {\isacharparenleft}Suc\ n{\isacharparenright}%
\end{isabelle}
\rulename{add_2_eq_Suc}

\begin{isabelle}%
\ \ \ \ \ n\ {\isacharplus}\ {\isacharhash}{\isadigit{2}}\ {\isacharequal}\ Suc\ {\isacharparenleft}Suc\ n{\isacharparenright}%
\end{isabelle}
\rulename{add_2_eq_Suc'}

\begin{isabelle}%
\ \ \ \ \ m\ {\isacharplus}\ n\ {\isacharplus}\ k\ {\isacharequal}\ m\ {\isacharplus}\ {\isacharparenleft}n\ {\isacharplus}\ k{\isacharparenright}%
\end{isabelle}
\rulename{add_assoc}

\begin{isabelle}%
\ \ \ \ \ m\ {\isacharplus}\ n\ {\isacharequal}\ n\ {\isacharplus}\ m%
\end{isabelle}
\rulename{add_commute}

\begin{isabelle}%
\ \ \ \ \ x\ {\isacharplus}\ {\isacharparenleft}y\ {\isacharplus}\ z{\isacharparenright}\ {\isacharequal}\ y\ {\isacharplus}\ {\isacharparenleft}x\ {\isacharplus}\ z{\isacharparenright}%
\end{isabelle}
\rulename{add_left_commute}

these form add_ac; similarly there is mult_ac%
\end{isamarkuptext}%
\isacommand{lemma}\ {\isachardoublequote}Suc{\isacharparenleft}i\ {\isacharplus}\ j{\isacharasterisk}l{\isacharasterisk}k\ {\isacharplus}\ m{\isacharasterisk}n{\isacharparenright}\ {\isacharequal}\ f\ {\isacharparenleft}n{\isacharasterisk}m\ {\isacharplus}\ i\ {\isacharplus}\ k{\isacharasterisk}j{\isacharasterisk}l{\isacharparenright}{\isachardoublequote}\isanewline
\isacommand{apply}\ {\isacharparenleft}simp\ add{\isacharcolon}\ add{\isacharunderscore}ac\ mult{\isacharunderscore}ac{\isacharparenright}\isanewline
\isacommand{oops}%
\begin{isamarkuptext}%
proof\ {\isacharparenleft}prove{\isacharparenright}{\isacharcolon}\ step\ {\isadigit{0}}\isanewline
\isanewline
goal\ {\isacharparenleft}lemma{\isacharparenright}{\isacharcolon}\isanewline
Suc\ {\isacharparenleft}i\ {\isacharplus}\ j\ {\isacharasterisk}\ l\ {\isacharasterisk}\ k\ {\isacharplus}\ m\ {\isacharasterisk}\ n{\isacharparenright}\ {\isacharequal}\ f\ {\isacharparenleft}n\ {\isacharasterisk}\ m\ {\isacharplus}\ i\ {\isacharplus}\ k\ {\isacharasterisk}\ j\ {\isacharasterisk}\ l{\isacharparenright}\isanewline
\ {\isadigit{1}}{\isachardot}\ Suc\ {\isacharparenleft}i\ {\isacharplus}\ j\ {\isacharasterisk}\ l\ {\isacharasterisk}\ k\ {\isacharplus}\ m\ {\isacharasterisk}\ n{\isacharparenright}\ {\isacharequal}\ f\ {\isacharparenleft}n\ {\isacharasterisk}\ m\ {\isacharplus}\ i\ {\isacharplus}\ k\ {\isacharasterisk}\ j\ {\isacharasterisk}\ l{\isacharparenright}

proof\ {\isacharparenleft}prove{\isacharparenright}{\isacharcolon}\ step\ {\isadigit{1}}\isanewline
\isanewline
goal\ {\isacharparenleft}lemma{\isacharparenright}{\isacharcolon}\isanewline
Suc\ {\isacharparenleft}i\ {\isacharplus}\ j\ {\isacharasterisk}\ l\ {\isacharasterisk}\ k\ {\isacharplus}\ m\ {\isacharasterisk}\ n{\isacharparenright}\ {\isacharequal}\ f\ {\isacharparenleft}n\ {\isacharasterisk}\ m\ {\isacharplus}\ i\ {\isacharplus}\ k\ {\isacharasterisk}\ j\ {\isacharasterisk}\ l{\isacharparenright}\isanewline
\ {\isadigit{1}}{\isachardot}\ Suc\ {\isacharparenleft}i\ {\isacharplus}\ {\isacharparenleft}m\ {\isacharasterisk}\ n\ {\isacharplus}\ j\ {\isacharasterisk}\ {\isacharparenleft}k\ {\isacharasterisk}\ l{\isacharparenright}{\isacharparenright}{\isacharparenright}\ {\isacharequal}\isanewline
\ \ \ \ f\ {\isacharparenleft}i\ {\isacharplus}\ {\isacharparenleft}m\ {\isacharasterisk}\ n\ {\isacharplus}\ j\ {\isacharasterisk}\ {\isacharparenleft}k\ {\isacharasterisk}\ l{\isacharparenright}{\isacharparenright}{\isacharparenright}%
\end{isamarkuptext}%
%
\begin{isamarkuptext}%
\begin{isabelle}%
\ \ \ \ \ {\isasymlbrakk}i\ {\isasymle}\ j{\isacharsemicolon}\ k\ {\isasymle}\ l{\isasymrbrakk}\ {\isasymLongrightarrow}\ i\ {\isacharasterisk}\ k\ {\isasymle}\ j\ {\isacharasterisk}\ l%
\end{isabelle}
\rulename{mult_le_mono}

\begin{isabelle}%
\ \ \ \ \ {\isasymlbrakk}i\ {\isacharless}\ j{\isacharsemicolon}\ {\isadigit{0}}\ {\isacharless}\ k{\isasymrbrakk}\ {\isasymLongrightarrow}\ i\ {\isacharasterisk}\ k\ {\isacharless}\ j\ {\isacharasterisk}\ k%
\end{isabelle}
\rulename{mult_less_mono1}

\begin{isabelle}%
\ \ \ \ \ m\ {\isasymle}\ n\ {\isasymLongrightarrow}\ m\ div\ k\ {\isasymle}\ n\ div\ k%
\end{isabelle}
\rulename{div_le_mono}

\begin{isabelle}%
\ \ \ \ \ {\isacharparenleft}m\ {\isacharplus}\ n{\isacharparenright}\ {\isacharasterisk}\ k\ {\isacharequal}\ m\ {\isacharasterisk}\ k\ {\isacharplus}\ n\ {\isacharasterisk}\ k%
\end{isabelle}
\rulename{add_mult_distrib}

\begin{isabelle}%
\ \ \ \ \ {\isacharparenleft}m\ {\isacharminus}\ n{\isacharparenright}\ {\isacharasterisk}\ k\ {\isacharequal}\ m\ {\isacharasterisk}\ k\ {\isacharminus}\ n\ {\isacharasterisk}\ k%
\end{isabelle}
\rulename{diff_mult_distrib}

\begin{isabelle}%
\ \ \ \ \ m\ mod\ n\ {\isacharasterisk}\ k\ {\isacharequal}\ m\ {\isacharasterisk}\ k\ mod\ {\isacharparenleft}n\ {\isacharasterisk}\ k{\isacharparenright}%
\end{isabelle}
\rulename{mod_mult_distrib}

\begin{isabelle}%
\ \ \ \ \ P\ {\isacharparenleft}a\ {\isacharminus}\ b{\isacharparenright}\ {\isacharequal}\ {\isacharparenleft}{\isacharparenleft}a\ {\isacharless}\ b\ {\isasymlongrightarrow}\ P\ {\isadigit{0}}{\isacharparenright}\ {\isasymand}\ {\isacharparenleft}{\isasymforall}d{\isachardot}\ a\ {\isacharequal}\ b\ {\isacharplus}\ d\ {\isasymlongrightarrow}\ P\ d{\isacharparenright}{\isacharparenright}%
\end{isabelle}
\rulename{nat_diff_split}%
\end{isamarkuptext}%
\isacommand{lemma}\ {\isachardoublequote}{\isacharparenleft}n{\isacharminus}{\isadigit{1}}{\isacharparenright}{\isacharasterisk}{\isacharparenleft}n{\isacharplus}{\isadigit{1}}{\isacharparenright}\ {\isacharequal}\ n{\isacharasterisk}n\ {\isacharminus}\ {\isadigit{1}}{\isachardoublequote}\isanewline
\isacommand{apply}\ {\isacharparenleft}simp\ split{\isacharcolon}\ nat{\isacharunderscore}diff{\isacharunderscore}split{\isacharparenright}\isanewline
\isacommand{done}%
\begin{isamarkuptext}%
\begin{isabelle}%
\ \ \ \ \ m\ mod\ n\ {\isacharequal}\ {\isacharparenleft}if\ m\ {\isacharless}\ n\ then\ m\ else\ {\isacharparenleft}m\ {\isacharminus}\ n{\isacharparenright}\ mod\ n{\isacharparenright}%
\end{isabelle}
\rulename{mod_if}

\begin{isabelle}%
\ \ \ \ \ m\ div\ n\ {\isacharasterisk}\ n\ {\isacharplus}\ m\ mod\ n\ {\isacharequal}\ m%
\end{isabelle}
\rulename{mod_div_equality}


\begin{isabelle}%
\ \ \ \ \ a\ {\isacharasterisk}\ b\ div\ c\ {\isacharequal}\ a\ {\isacharasterisk}\ {\isacharparenleft}b\ div\ c{\isacharparenright}\ {\isacharplus}\ a\ {\isacharasterisk}\ {\isacharparenleft}b\ mod\ c{\isacharparenright}\ div\ c%
\end{isabelle}
\rulename{div_mult1_eq}

\begin{isabelle}%
\ \ \ \ \ a\ {\isacharasterisk}\ b\ mod\ c\ {\isacharequal}\ a\ {\isacharasterisk}\ {\isacharparenleft}b\ mod\ c{\isacharparenright}\ mod\ c%
\end{isabelle}
\rulename{mod_mult1_eq}

\begin{isabelle}%
\ \ \ \ \ a\ div\ {\isacharparenleft}b\ {\isacharasterisk}\ c{\isacharparenright}\ {\isacharequal}\ a\ div\ b\ div\ c%
\end{isabelle}
\rulename{div_mult2_eq}

\begin{isabelle}%
\ \ \ \ \ a\ mod\ {\isacharparenleft}b\ {\isacharasterisk}\ c{\isacharparenright}\ {\isacharequal}\ b\ {\isacharasterisk}\ {\isacharparenleft}a\ div\ b\ mod\ c{\isacharparenright}\ {\isacharplus}\ a\ mod\ b%
\end{isabelle}
\rulename{mod_mult2_eq}

\begin{isabelle}%
\ \ \ \ \ {\isadigit{0}}\ {\isacharless}\ c\ {\isasymLongrightarrow}\ c\ {\isacharasterisk}\ a\ div\ {\isacharparenleft}c\ {\isacharasterisk}\ b{\isacharparenright}\ {\isacharequal}\ a\ div\ b%
\end{isabelle}
\rulename{div_mult_mult1}

\begin{isabelle}%
\ \ \ \ \ a\ div\ {\isadigit{0}}\ {\isacharequal}\ {\isadigit{0}}%
\end{isabelle}
\rulename{DIVISION_BY_ZERO_DIV}

\begin{isabelle}%
\ \ \ \ \ a\ mod\ {\isadigit{0}}\ {\isacharequal}\ a%
\end{isabelle}
\rulename{DIVISION_BY_ZERO_MOD}

\begin{isabelle}%
\ \ \ \ \ {\isasymlbrakk}m\ dvd\ n{\isacharsemicolon}\ n\ dvd\ m{\isasymrbrakk}\ {\isasymLongrightarrow}\ m\ {\isacharequal}\ n%
\end{isabelle}
\rulename{dvd_anti_sym}

\begin{isabelle}%
\ \ \ \ \ {\isasymlbrakk}k\ dvd\ m{\isacharsemicolon}\ k\ dvd\ n{\isasymrbrakk}\ {\isasymLongrightarrow}\ k\ dvd\ {\isacharparenleft}m\ {\isacharplus}\ n{\isacharparenright}%
\end{isabelle}
\rulename{dvd_add}

For the integers, I'd list a few theorems that somehow involve negative 
numbers.  

Division, remainder of negatives


\begin{isabelle}%
\ \ \ \ \ {\isacharhash}{\isadigit{0}}\ {\isacharless}\ b\ {\isasymLongrightarrow}\ {\isacharhash}{\isadigit{0}}\ {\isasymle}\ a\ mod\ b%
\end{isabelle}
\rulename{pos_mod_sign}

\begin{isabelle}%
\ \ \ \ \ {\isacharhash}{\isadigit{0}}\ {\isacharless}\ b\ {\isasymLongrightarrow}\ a\ mod\ b\ {\isacharless}\ b%
\end{isabelle}
\rulename{pos_mod_bound}

\begin{isabelle}%
\ \ \ \ \ b\ {\isacharless}\ {\isacharhash}{\isadigit{0}}\ {\isasymLongrightarrow}\ a\ mod\ b\ {\isasymle}\ {\isacharhash}{\isadigit{0}}%
\end{isabelle}
\rulename{neg_mod_sign}

\begin{isabelle}%
\ \ \ \ \ b\ {\isacharless}\ {\isacharhash}{\isadigit{0}}\ {\isasymLongrightarrow}\ b\ {\isacharless}\ a\ mod\ b%
\end{isabelle}
\rulename{neg_mod_bound}

\begin{isabelle}%
\ \ \ \ \ {\isacharparenleft}a\ {\isacharplus}\ b{\isacharparenright}\ div\ c\ {\isacharequal}\isanewline
\ \ \ \ \ a\ div\ c\ {\isacharplus}\ b\ div\ c\ {\isacharplus}\ {\isacharparenleft}a\ mod\ c\ {\isacharplus}\ b\ mod\ c{\isacharparenright}\ div\ c%
\end{isabelle}
\rulename{zdiv_zadd1_eq}

\begin{isabelle}%
\ \ \ \ \ {\isacharparenleft}a\ {\isacharplus}\ b{\isacharparenright}\ mod\ c\ {\isacharequal}\ {\isacharparenleft}a\ mod\ c\ {\isacharplus}\ b\ mod\ c{\isacharparenright}\ mod\ c%
\end{isabelle}
\rulename{zmod_zadd1_eq}

\begin{isabelle}%
\ \ \ \ \ a\ {\isacharasterisk}\ b\ div\ c\ {\isacharequal}\ a\ {\isacharasterisk}\ {\isacharparenleft}b\ div\ c{\isacharparenright}\ {\isacharplus}\ a\ {\isacharasterisk}\ {\isacharparenleft}b\ mod\ c{\isacharparenright}\ div\ c%
\end{isabelle}
\rulename{zdiv_zmult1_eq}

\begin{isabelle}%
\ \ \ \ \ a\ {\isacharasterisk}\ b\ mod\ c\ {\isacharequal}\ a\ {\isacharasterisk}\ {\isacharparenleft}b\ mod\ c{\isacharparenright}\ mod\ c%
\end{isabelle}
\rulename{zmod_zmult1_eq}

\begin{isabelle}%
\ \ \ \ \ {\isacharhash}{\isadigit{0}}\ {\isacharless}\ c\ {\isasymLongrightarrow}\ a\ div\ {\isacharparenleft}b\ {\isacharasterisk}\ c{\isacharparenright}\ {\isacharequal}\ a\ div\ b\ div\ c%
\end{isabelle}
\rulename{zdiv_zmult2_eq}

\begin{isabelle}%
\ \ \ \ \ {\isacharhash}{\isadigit{0}}\ {\isacharless}\ c\ {\isasymLongrightarrow}\ a\ mod\ {\isacharparenleft}b\ {\isacharasterisk}\ c{\isacharparenright}\ {\isacharequal}\ b\ {\isacharasterisk}\ {\isacharparenleft}a\ div\ b\ mod\ c{\isacharparenright}\ {\isacharplus}\ a\ mod\ b%
\end{isabelle}
\rulename{zmod_zmult2_eq}

\begin{isabelle}%
\ \ \ \ \ {\isasymbar}x\ {\isacharasterisk}\ y{\isasymbar}\ {\isacharequal}\ {\isasymbar}x{\isasymbar}\ {\isacharasterisk}\ {\isasymbar}y{\isasymbar}%
\end{isabelle}
\rulename{abs_mult}%
\end{isamarkuptext}%
\isacommand{end}\isanewline
\end{isabellebody}%
%%% Local Variables:
%%% mode: latex
%%% TeX-master: "root"
%%% End:
