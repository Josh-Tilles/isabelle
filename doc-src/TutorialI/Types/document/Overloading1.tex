%
\begin{isabellebody}%
\def\isabellecontext{Overloading{\isadigit{1}}}%
\isamarkupfalse%
%
\isadelimtheory
%
\endisadelimtheory
%
\isatagtheory
%
\endisatagtheory
{\isafoldtheory}%
%
\isadelimtheory
%
\endisadelimtheory
%
\isamarkupsubsubsection{Controlled Overloading with Type Classes%
}
\isamarkuptrue%
%
\begin{isamarkuptext}%
We now start with the theory of ordering relations, which we shall phrase
in terms of the two binary symbols \isa{{\isacharless}{\isacharless}} and \isa{{\isacharless}{\isacharless}{\isacharequal}}
to avoid clashes with \isa{{\isacharless}} and \isa{{\isacharless}{\isacharequal}} in theory \isa{Main}. To restrict the application of \isa{{\isacharless}{\isacharless}} and \isa{{\isacharless}{\isacharless}{\isacharequal}} we
introduce the class \isa{ordrel}:%
\end{isamarkuptext}%
\isamarkuptrue%
\isacommand{axclass}\isamarkupfalse%
\ ordrel\ {\isacharless}\ type%
\begin{isamarkuptext}%
\noindent
This introduces a new class \isa{ordrel} and makes it a subclass of
the predefined class \isa{type}, which
is the class of all HOL types.
This is a degenerate form of axiomatic type class without any axioms.
Its sole purpose is to restrict the use of overloaded constants to meaningful
instances:%
\end{isamarkuptext}%
\isamarkuptrue%
\isacommand{consts}\isamarkupfalse%
\ less\ {\isacharcolon}{\isacharcolon}\ {\isachardoublequoteopen}{\isacharparenleft}{\isacharprime}a{\isacharcolon}{\isacharcolon}ordrel{\isacharparenright}\ {\isasymRightarrow}\ {\isacharprime}a\ {\isasymRightarrow}\ bool{\isachardoublequoteclose}\ \ \ \ \ {\isacharparenleft}\isakeyword{infixl}\ {\isachardoublequoteopen}{\isacharless}{\isacharless}{\isachardoublequoteclose}\ \ {\isadigit{5}}{\isadigit{0}}{\isacharparenright}\isanewline
\ \ \ \ \ \ \ le\ \ \ {\isacharcolon}{\isacharcolon}\ {\isachardoublequoteopen}{\isacharparenleft}{\isacharprime}a{\isacharcolon}{\isacharcolon}ordrel{\isacharparenright}\ {\isasymRightarrow}\ {\isacharprime}a\ {\isasymRightarrow}\ bool{\isachardoublequoteclose}\ \ \ \ \ {\isacharparenleft}\isakeyword{infixl}\ {\isachardoublequoteopen}{\isacharless}{\isacharless}{\isacharequal}{\isachardoublequoteclose}\ {\isadigit{5}}{\isadigit{0}}{\isacharparenright}%
\begin{isamarkuptext}%
\noindent
Note that only one occurrence of a type variable in a type needs to be
constrained with a class; the constraint is propagated to the other
occurrences automatically.

So far there are no types of class \isa{ordrel}. To breathe life
into \isa{ordrel} we need to declare a type to be an \bfindex{instance} of
\isa{ordrel}:%
\end{isamarkuptext}%
\isamarkuptrue%
\isacommand{instance}\isamarkupfalse%
\ bool\ {\isacharcolon}{\isacharcolon}\ ordrel%
\isadelimproof
%
\endisadelimproof
%
\isatagproof
%
\begin{isamarkuptxt}%
\noindent
Command \isacommand{instance} actually starts a proof, namely that
\isa{bool} satisfies all axioms of \isa{ordrel}.
There are none, but we still need to finish that proof, which we do
by invoking the \methdx{intro_classes} method:%
\end{isamarkuptxt}%
\isamarkuptrue%
\isacommand{by}\isamarkupfalse%
\ intro{\isacharunderscore}classes%
\endisatagproof
{\isafoldproof}%
%
\isadelimproof
%
\endisadelimproof
%
\begin{isamarkuptext}%
\noindent
More interesting \isacommand{instance} proofs will arise below
in the context of proper axiomatic type classes.

Although terms like \isa{False\ {\isacharless}{\isacharless}{\isacharequal}\ P} are now legal, we still need to say
what the relation symbols actually mean at type \isa{bool}:%
\end{isamarkuptext}%
\isamarkuptrue%
\isacommand{defs}\isamarkupfalse%
\ {\isacharparenleft}\isakeyword{overloaded}{\isacharparenright}\isanewline
le{\isacharunderscore}bool{\isacharunderscore}def{\isacharcolon}\ \ {\isachardoublequoteopen}P\ {\isacharless}{\isacharless}{\isacharequal}\ Q\ {\isasymequiv}\ P\ {\isasymlongrightarrow}\ Q{\isachardoublequoteclose}\isanewline
less{\isacharunderscore}bool{\isacharunderscore}def{\isacharcolon}\ {\isachardoublequoteopen}P\ {\isacharless}{\isacharless}\ Q\ {\isasymequiv}\ {\isasymnot}P\ {\isasymand}\ Q{\isachardoublequoteclose}%
\begin{isamarkuptext}%
\noindent
Now \isa{False\ {\isacharless}{\isacharless}{\isacharequal}\ P} is provable:%
\end{isamarkuptext}%
\isamarkuptrue%
\isacommand{lemma}\isamarkupfalse%
\ {\isachardoublequoteopen}False\ {\isacharless}{\isacharless}{\isacharequal}\ P{\isachardoublequoteclose}\isanewline
%
\isadelimproof
%
\endisadelimproof
%
\isatagproof
\isacommand{by}\isamarkupfalse%
{\isacharparenleft}simp\ add{\isacharcolon}\ le{\isacharunderscore}bool{\isacharunderscore}def{\isacharparenright}%
\endisatagproof
{\isafoldproof}%
%
\isadelimproof
%
\endisadelimproof
%
\begin{isamarkuptext}%
\noindent
At this point, \isa{{\isacharbrackleft}{\isacharbrackright}\ {\isacharless}{\isacharless}{\isacharequal}\ {\isacharbrackleft}{\isacharbrackright}} is not even well-typed.
To make it well-typed,
we need to make lists a type of class \isa{ordrel}:%
\end{isamarkuptext}%
\isamarkuptrue%
%
\isadelimtheory
%
\endisadelimtheory
%
\isatagtheory
%
\endisatagtheory
{\isafoldtheory}%
%
\isadelimtheory
%
\endisadelimtheory
\end{isabellebody}%
%%% Local Variables:
%%% mode: latex
%%% TeX-master: "root"
%%% End:
