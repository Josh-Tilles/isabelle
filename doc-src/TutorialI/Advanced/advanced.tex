\chapter{Advanced Simplification and Induction}

Although we have already learned a lot about simplification and
induction, there are some advanced proof techniques that we have not covered
yet and which are worth learning. The sections of this chapter are
independent of each other and can be read in any order.

%
\begin{isabellebody}%
\def\isabellecontext{simp}%
%
\isamarkupsection{Simplification}
%
\begin{isamarkuptext}%
\label{sec:simplification-II}\index{simplification|(}
This section discusses some additional nifty features not covered so far and
gives a short introduction to the simplification process itself. The latter
is helpful to understand why a particular rule does or does not apply in some
situation.%
\end{isamarkuptext}%
%
\isamarkupsubsection{Advanced features}
%
\isamarkupsubsubsection{Congruence rules}
%
\begin{isamarkuptext}%
\label{sec:simp-cong}
It is hardwired into the simplifier that while simplifying the conclusion $Q$
of $P \isasymImp Q$ it is legal to make uses of the assumptions $P$. This
kind of contextual information can also be made available for other
operators. For example, \isa{xs\ {\isacharequal}\ {\isacharbrackleft}{\isacharbrackright}\ {\isasymlongrightarrow}\ xs\ {\isacharat}\ xs\ {\isacharequal}\ xs} simplifies to \isa{True} because we may use \isa{xs\ {\isacharequal}\ {\isacharbrackleft}{\isacharbrackright}} when simplifying \isa{xs\ {\isacharat}\ xs\ {\isacharequal}\ xs}. The generation of contextual information during simplification is
controlled by so-called \bfindex{congruence rules}. This is the one for
\isa{{\isasymlongrightarrow}}:
\begin{isabelle}%
\ \ \ \ \ {\isasymlbrakk}P\ {\isacharequal}\ P{\isacharprime}{\isacharsemicolon}\ P{\isacharprime}\ {\isasymLongrightarrow}\ Q\ {\isacharequal}\ Q{\isacharprime}{\isasymrbrakk}\ {\isasymLongrightarrow}\ {\isacharparenleft}P\ {\isasymlongrightarrow}\ Q{\isacharparenright}\ {\isacharequal}\ {\isacharparenleft}P{\isacharprime}\ {\isasymlongrightarrow}\ Q{\isacharprime}{\isacharparenright}%
\end{isabelle}
It should be read as follows:
In order to simplify \isa{P\ {\isasymlongrightarrow}\ Q} to \isa{P{\isacharprime}\ {\isasymlongrightarrow}\ Q{\isacharprime}},
simplify \isa{P} to \isa{P{\isacharprime}}
and assume \isa{P{\isacharprime}} when simplifying \isa{Q} to \isa{Q{\isacharprime}}.

Here are some more examples.  The congruence rules for bounded
quantifiers supply contextual information about the bound variable:
\begin{isabelle}%
\ \ \ \ \ {\isasymlbrakk}A\ {\isacharequal}\ B{\isacharsemicolon}\ {\isasymAnd}x{\isachardot}\ x\ {\isasymin}\ B\ {\isasymLongrightarrow}\ P\ x\ {\isacharequal}\ Q\ x{\isasymrbrakk}\isanewline
\ \ \ \ \ {\isasymLongrightarrow}\ {\isacharparenleft}{\isasymforall}x{\isasymin}A{\isachardot}\ P\ x{\isacharparenright}\ {\isacharequal}\ {\isacharparenleft}{\isasymforall}x{\isasymin}B{\isachardot}\ Q\ x{\isacharparenright}%
\end{isabelle}
The congruence rule for conditional expressions supply contextual
information for simplifying the arms:
\begin{isabelle}%
\ \ \ \ \ {\isasymlbrakk}b\ {\isacharequal}\ c{\isacharsemicolon}\ c\ {\isasymLongrightarrow}\ x\ {\isacharequal}\ u{\isacharsemicolon}\ {\isasymnot}\ c\ {\isasymLongrightarrow}\ y\ {\isacharequal}\ v{\isasymrbrakk}\isanewline
\ \ \ \ \ {\isasymLongrightarrow}\ {\isacharparenleft}if\ b\ then\ x\ else\ y{\isacharparenright}\ {\isacharequal}\ {\isacharparenleft}if\ c\ then\ u\ else\ v{\isacharparenright}%
\end{isabelle}
A congruence rule can also \emph{prevent} simplification of some arguments.
Here is an alternative congruence rule for conditional expressions:
\begin{isabelle}%
\ \ \ \ \ b\ {\isacharequal}\ c\ {\isasymLongrightarrow}\ {\isacharparenleft}if\ b\ then\ x\ else\ y{\isacharparenright}\ {\isacharequal}\ {\isacharparenleft}if\ c\ then\ x\ else\ y{\isacharparenright}%
\end{isabelle}
Only the first argument is simplified; the others remain unchanged.
This makes simplification much faster and is faithful to the evaluation
strategy in programming languages, which is why this is the default
congruence rule for \isa{if}. Analogous rules control the evaluaton of
\isa{case} expressions.

You can delare your own congruence rules with the attribute \isa{cong},
either globally, in the usual manner,
\begin{quote}
\isacommand{declare} \textit{theorem-name} \isa{{\isacharbrackleft}cong{\isacharbrackright}}
\end{quote}
or locally in a \isa{simp} call by adding the modifier
\begin{quote}
\isa{cong{\isacharcolon}} \textit{list of theorem names}
\end{quote}
The effect is reversed by \isa{cong\ del} instead of \isa{cong}.

\begin{warn}
The congruence rule \isa{conj{\isacharunderscore}cong}
\begin{isabelle}%
\ \ \ \ \ {\isasymlbrakk}P\ {\isacharequal}\ P{\isacharprime}{\isacharsemicolon}\ P{\isacharprime}\ {\isasymLongrightarrow}\ Q\ {\isacharequal}\ Q{\isacharprime}{\isasymrbrakk}\ {\isasymLongrightarrow}\ {\isacharparenleft}P\ {\isasymand}\ Q{\isacharparenright}\ {\isacharequal}\ {\isacharparenleft}P{\isacharprime}\ {\isasymand}\ Q{\isacharprime}{\isacharparenright}%
\end{isabelle}
is occasionally useful but not a default rule; you have to use it explicitly.
\end{warn}%
\end{isamarkuptext}%
%
\isamarkupsubsubsection{Permutative rewrite rules}
%
\begin{isamarkuptext}%
\index{rewrite rule!permutative|bold}
\index{rewriting!ordered|bold}
\index{ordered rewriting|bold}
\index{simplification!ordered|bold}
An equation is a \bfindex{permutative rewrite rule} if the left-hand
side and right-hand side are the same up to renaming of variables.  The most
common permutative rule is commutativity: \isa{x\ {\isacharplus}\ y\ {\isacharequal}\ y\ {\isacharplus}\ x}.  Other examples
include \isa{x\ {\isacharminus}\ y\ {\isacharminus}\ z\ {\isacharequal}\ x\ {\isacharminus}\ z\ {\isacharminus}\ y} in arithmetic and \isa{insert\ x\ {\isacharparenleft}insert\ y\ A{\isacharparenright}\ {\isacharequal}\ insert\ y\ {\isacharparenleft}insert\ x\ A{\isacharparenright}} for sets. Such rules are problematic because
once they apply, they can be used forever. The simplifier is aware of this
danger and treats permutative rules by means of a special strategy, called
\bfindex{ordered rewriting}: a permutative rewrite
rule is only applied if the term becomes ``smaller'' (w.r.t.\ some fixed
lexicographic ordering on terms). For example, commutativity rewrites
\isa{b\ {\isacharplus}\ a} to \isa{a\ {\isacharplus}\ b}, but then stops because \isa{a\ {\isacharplus}\ b} is strictly
smaller than \isa{b\ {\isacharplus}\ a}.  Permutative rewrite rules can be turned into
simplification rules in the usual manner via the \isa{simp} attribute; the
simplifier recognizes their special status automatically.

Permutative rewrite rules are most effective in the case of
associative-commutative functions.  (Associativity by itself is not
permutative.)  When dealing with an AC-function~$f$, keep the
following points in mind:
\begin{itemize}\index{associative-commutative function}
  
\item The associative law must always be oriented from left to right,
  namely $f(f(x,y),z) = f(x,f(y,z))$.  The opposite orientation, if
  used with commutativity, can lead to nontermination.

\item To complete your set of rewrite rules, you must add not just
  associativity~(A) and commutativity~(C) but also a derived rule, {\bf
    left-com\-mut\-ativ\-ity} (LC): $f(x,f(y,z)) = f(y,f(x,z))$.
\end{itemize}
Ordered rewriting with the combination of A, C, and LC sorts a term
lexicographically:
\[\def\maps#1{~\stackrel{#1}{\leadsto}~}
 f(f(b,c),a) \maps{A} f(b,f(c,a)) \maps{C} f(b,f(a,c)) \maps{LC} f(a,f(b,c)) \]

Note that ordered rewriting for \isa{{\isacharplus}} and \isa{{\isacharasterisk}} on numbers is rarely
necessary because the builtin arithmetic capabilities often take care of
this.%
\end{isamarkuptext}%
%
\isamarkupsubsection{How it works}
%
\begin{isamarkuptext}%
\label{sec:SimpHow}
Roughly speaking, the simplifier proceeds bottom-up (subterms are simplified
first) and a conditional equation is only applied if its condition could be
proved (again by simplification). Below we explain some special features of the rewriting process.%
\end{isamarkuptext}%
%
\isamarkupsubsubsection{Higher-order patterns}
%
\begin{isamarkuptext}%
\index{simplification rule|(}
So far we have pretended the simplifier can deal with arbitrary
rewrite rules. This is not quite true.  Due to efficiency (and
potentially also computability) reasons, the simplifier expects the
left-hand side of each rule to be a so-called \emph{higher-order
pattern}~\cite{nipkow-patterns}\indexbold{higher-order
pattern}\indexbold{pattern, higher-order}. This restricts where
unknowns may occur.  Higher-order patterns are terms in $\beta$-normal
form (this will always be the case unless you have done something
strange) where each occurrence of an unknown is of the form
$\Var{f}~x@1~\dots~x@n$, where the $x@i$ are distinct bound
variables. Thus all ``standard'' rewrite rules, where all unknowns are
of base type, for example \isa{{\isacharquery}m\ {\isacharplus}\ {\isacharquery}n\ {\isacharplus}\ {\isacharquery}k\ {\isacharequal}\ {\isacharquery}m\ {\isacharplus}\ {\isacharparenleft}{\isacharquery}n\ {\isacharplus}\ {\isacharquery}k{\isacharparenright}}, are OK: if an unknown is
of base type, it cannot have any arguments. Additionally, the rule
\isa{{\isacharparenleft}{\isasymforall}x{\isachardot}\ {\isacharquery}P\ x\ {\isasymand}\ {\isacharquery}Q\ x{\isacharparenright}\ {\isacharequal}\ {\isacharparenleft}{\isacharparenleft}{\isasymforall}x{\isachardot}\ {\isacharquery}P\ x{\isacharparenright}\ {\isasymand}\ {\isacharparenleft}{\isasymforall}x{\isachardot}\ {\isacharquery}Q\ x{\isacharparenright}{\isacharparenright}} is also OK, in
both directions: all arguments of the unknowns \isa{{\isacharquery}P} and
\isa{{\isacharquery}Q} are distinct bound variables.

If the left-hand side is not a higher-order pattern, not all is lost
and the simplifier will still try to apply the rule, but only if it
matches ``directly'', i.e.\ without much $\lambda$-calculus hocus
pocus. For example, \isa{{\isacharquery}f\ {\isacharquery}x\ {\isasymin}\ range\ {\isacharquery}f\ {\isacharequal}\ True} rewrites
\isa{g\ a\ {\isasymin}\ range\ g} to \isa{True}, but will fail to match
\isa{g{\isacharparenleft}h\ b{\isacharparenright}\ {\isasymin}\ range{\isacharparenleft}{\isasymlambda}x{\isachardot}\ g{\isacharparenleft}h\ x{\isacharparenright}{\isacharparenright}}.  However, you can
replace the offending subterms (in our case \isa{{\isacharquery}f\ {\isacharquery}x}, which
is not a pattern) by adding new variables and conditions: \isa{{\isacharquery}y\ {\isacharequal}\ {\isacharquery}f\ {\isacharquery}x\ {\isasymLongrightarrow}\ {\isacharquery}y\ {\isasymin}\ range\ {\isacharquery}f\ {\isacharequal}\ True} is fine
as a conditional rewrite rule since conditions can be arbitrary
terms. However, this trick is not a panacea because the newly
introduced conditions may be hard to prove, which has to take place
before the rule can actually be applied.
  
There is basically no restriction on the form of the right-hand
sides.  They may not contain extraneous term or type variables, though.%
\end{isamarkuptext}%
%
\isamarkupsubsubsection{The preprocessor}
%
\begin{isamarkuptext}%
When some theorem is declared a simplification rule, it need not be a
conditional equation already.  The simplifier will turn it into a set of
conditional equations automatically.  For example, given \isa{f\ x\ {\isacharequal}\ g\ x\ {\isasymand}\ h\ x\ {\isacharequal}\ k\ x} the simplifier will turn this into the two separate
simplifiction rules \isa{f\ x\ {\isacharequal}\ g\ x} and \isa{h\ x\ {\isacharequal}\ k\ x}. In
general, the input theorem is converted as follows:
\begin{eqnarray}
\neg P &\mapsto& P = False \nonumber\\
P \longrightarrow Q &\mapsto& P \Longrightarrow Q \nonumber\\
P \land Q &\mapsto& P,\ Q \nonumber\\
\forall x.~P~x &\mapsto& P~\Var{x}\nonumber\\
\forall x \in A.\ P~x &\mapsto& \Var{x} \in A \Longrightarrow P~\Var{x} \nonumber\\
\isa{if}\ P\ \isa{then}\ Q\ \isa{else}\ R &\mapsto&
 P \Longrightarrow Q,\ \neg P \Longrightarrow R \nonumber
\end{eqnarray}
Once this conversion process is finished, all remaining non-equations
$P$ are turned into trivial equations $P = True$.
For example, the formula \isa{{\isacharparenleft}p\ {\isasymlongrightarrow}\ q\ {\isasymand}\ r{\isacharparenright}\ {\isasymand}\ s} is converted into the three rules
\begin{center}
\isa{p\ {\isasymLongrightarrow}\ q\ {\isacharequal}\ True},\quad  \isa{p\ {\isasymLongrightarrow}\ r\ {\isacharequal}\ True},\quad  \isa{s\ {\isacharequal}\ True}.
\end{center}
\index{simplification rule|)}
\index{simplification|)}%
\end{isamarkuptext}%
\end{isabellebody}%
%%% Local Variables:
%%% mode: latex
%%% TeX-master: "root"
%%% End:


\section{Advanced Induction Techniques}
\label{sec:advanced-ind}
\index{induction|(}
%
\begin{isabellebody}%
\def\isabellecontext{AdvancedInd}%
%
\begin{isamarkuptext}%
\noindent
Now that we have learned about rules and logic, we take another look at the
finer points of induction. The two questions we answer are: what to do if the
proposition to be proved is not directly amenable to induction, and how to
utilize and even derive new induction schemas.%
\end{isamarkuptext}%
%
\isamarkupsubsection{Massaging the proposition\label{sec:ind-var-in-prems}}
%
\begin{isamarkuptext}%
\noindent
So far we have assumed that the theorem we want to prove is already in a form
that is amenable to induction, but this is not always the case:%
\end{isamarkuptext}%
\isacommand{lemma}\ {\isachardoublequote}xs\ {\isasymnoteq}\ {\isacharbrackleft}{\isacharbrackright}\ {\isasymLongrightarrow}\ hd{\isacharparenleft}rev\ xs{\isacharparenright}\ {\isacharequal}\ last\ xs{\isachardoublequote}\isanewline
\isacommand{apply}{\isacharparenleft}induct{\isacharunderscore}tac\ xs{\isacharparenright}%
\begin{isamarkuptxt}%
\noindent
(where \isa{hd} and \isa{last} return the first and last element of a
non-empty list)
produces the warning
\begin{quote}\tt
Induction variable occurs also among premises!
\end{quote}
and leads to the base case
\begin{isabelle}
\ 1.\ xs\ {\isasymnoteq}\ []\ {\isasymLongrightarrow}\ hd\ (rev\ [])\ =\ last\ []
\end{isabelle}
which, after simplification, becomes
\begin{isabelle}
\ 1.\ xs\ {\isasymnoteq}\ []\ {\isasymLongrightarrow}\ hd\ []\ =\ last\ []
\end{isabelle}
We cannot prove this equality because we do not know what \isa{hd} and
\isa{last} return when applied to \isa{{\isacharbrackleft}{\isacharbrackright}}.

The point is that we have violated the above warning. Because the induction
formula is only the conclusion, the occurrence of \isa{xs} in the premises is
not modified by induction. Thus the case that should have been trivial
becomes unprovable. Fortunately, the solution is easy:
\begin{quote}
\emph{Pull all occurrences of the induction variable into the conclusion
using \isa{{\isasymlongrightarrow}}.}
\end{quote}
This means we should prove%
\end{isamarkuptxt}%
\isacommand{lemma}\ hd{\isacharunderscore}rev{\isacharcolon}\ {\isachardoublequote}xs\ {\isasymnoteq}\ {\isacharbrackleft}{\isacharbrackright}\ {\isasymlongrightarrow}\ hd{\isacharparenleft}rev\ xs{\isacharparenright}\ {\isacharequal}\ last\ xs{\isachardoublequote}%
\begin{isamarkuptext}%
\noindent
This time, induction leaves us with the following base case
\begin{isabelle}
\ 1.\ []\ {\isasymnoteq}\ []\ {\isasymlongrightarrow}\ hd\ (rev\ [])\ =\ last\ []
\end{isabelle}
which is trivial, and \isa{auto} finishes the whole proof.

If \isa{hd{\isacharunderscore}rev} is meant to be a simplification rule, you are
done. But if you really need the \isa{{\isasymLongrightarrow}}-version of
\isa{hd{\isacharunderscore}rev}, for example because you want to apply it as an
introduction rule, you need to derive it separately, by combining it with
modus ponens:%
\end{isamarkuptext}%
\isacommand{lemmas}\ hd{\isacharunderscore}revI\ {\isacharequal}\ hd{\isacharunderscore}rev{\isacharbrackleft}THEN\ mp{\isacharbrackright}%
\begin{isamarkuptext}%
\noindent
which yields the lemma we originally set out to prove.

In case there are multiple premises $A@1$, \dots, $A@n$ containing the
induction variable, you should turn the conclusion $C$ into
\[ A@1 \longrightarrow \cdots A@n \longrightarrow C \]
(see the remark?? in \S\ref{??}).
Additionally, you may also have to universally quantify some other variables,
which can yield a fairly complex conclusion.
Here is a simple example (which is proved by \isa{blast}):%
\end{isamarkuptext}%
\isacommand{lemma}\ simple{\isacharcolon}\ {\isachardoublequote}{\isasymforall}y{\isachardot}\ A\ y\ {\isasymlongrightarrow}\ B\ y\ {\isasymlongrightarrow}\ B\ y\ {\isacharampersand}\ A\ y{\isachardoublequote}%
\begin{isamarkuptext}%
\noindent
You can get the desired lemma by explicit
application of modus ponens and \isa{spec}:%
\end{isamarkuptext}%
\isacommand{lemmas}\ myrule\ {\isacharequal}\ simple{\isacharbrackleft}THEN\ spec{\isacharcomma}\ THEN\ mp{\isacharcomma}\ THEN\ mp{\isacharbrackright}%
\begin{isamarkuptext}%
\noindent
or the wholesale stripping of \isa{{\isasymforall}} and
\isa{{\isasymlongrightarrow}} in the conclusion via \isa{rule{\isacharunderscore}format}%
\end{isamarkuptext}%
\isacommand{lemmas}\ myrule\ {\isacharequal}\ simple{\isacharbrackleft}rule{\isacharunderscore}format{\isacharbrackright}%
\begin{isamarkuptext}%
\noindent
yielding \isa{{\isasymlbrakk}A\ y{\isacharsemicolon}\ B\ y{\isasymrbrakk}\ {\isasymLongrightarrow}\ B\ y\ {\isasymand}\ A\ y}.
You can go one step further and include these derivations already in the
statement of your original lemma, thus avoiding the intermediate step:%
\end{isamarkuptext}%
\isacommand{lemma}\ myrule{\isacharbrackleft}rule{\isacharunderscore}format{\isacharbrackright}{\isacharcolon}\ \ {\isachardoublequote}{\isasymforall}y{\isachardot}\ A\ y\ {\isasymlongrightarrow}\ B\ y\ {\isasymlongrightarrow}\ B\ y\ {\isacharampersand}\ A\ y{\isachardoublequote}%
\begin{isamarkuptext}%
\bigskip

A second reason why your proposition may not be amenable to induction is that
you want to induct on a whole term, rather than an individual variable. In
general, when inducting on some term $t$ you must rephrase the conclusion as
\[ \forall y@1 \dots y@n.~ x = t \longrightarrow C \] where $y@1 \dots y@n$
are the free variables in $t$ and $x$ is new, and perform induction on $x$
afterwards. An example appears below.%
\end{isamarkuptext}%
%
\isamarkupsubsection{Beyond structural and recursion induction}
%
\begin{isamarkuptext}%
So far, inductive proofs where by structural induction for
primitive recursive functions and recursion induction for total recursive
functions. But sometimes structural induction is awkward and there is no
recursive function in sight either that could furnish a more appropriate
induction schema. In such cases some existing standard induction schema can
be helpful. We show how to apply such induction schemas by an example.

Structural induction on \isa{nat} is
usually known as ``mathematical induction''. There is also ``complete
induction'', where you must prove $P(n)$ under the assumption that $P(m)$
holds for all $m<n$. In Isabelle, this is the theorem \isa{nat{\isacharunderscore}less{\isacharunderscore}induct}:
\begin{isabelle}%
\ \ \ \ \ {\isacharparenleft}{\isasymAnd}n{\isachardot}\ {\isasymforall}m{\isachardot}\ m\ {\isacharless}\ n\ {\isasymlongrightarrow}\ P\ m\ {\isasymLongrightarrow}\ P\ n{\isacharparenright}\ {\isasymLongrightarrow}\ P\ n%
\end{isabelle}
Here is an example of its application.%
\end{isamarkuptext}%
\isacommand{consts}\ f\ {\isacharcolon}{\isacharcolon}\ {\isachardoublequote}nat\ {\isacharequal}{\isachargreater}\ nat{\isachardoublequote}\isanewline
\isacommand{axioms}\ f{\isacharunderscore}ax{\isacharcolon}\ {\isachardoublequote}f{\isacharparenleft}f{\isacharparenleft}n{\isacharparenright}{\isacharparenright}\ {\isacharless}\ f{\isacharparenleft}Suc{\isacharparenleft}n{\isacharparenright}{\isacharparenright}{\isachardoublequote}%
\begin{isamarkuptext}%
\noindent
From the above axiom\footnote{In general, the use of axioms is strongly
discouraged, because of the danger of inconsistencies. The above axiom does
not introduce an inconsistency because, for example, the identity function
satisfies it.}
for \isa{f} it follows that \isa{n\ {\isasymle}\ f\ n}, which can
be proved by induction on \isa{f\ n}. Following the recipy outlined
above, we have to phrase the proposition as follows to allow induction:%
\end{isamarkuptext}%
\isacommand{lemma}\ f{\isacharunderscore}incr{\isacharunderscore}lem{\isacharcolon}\ {\isachardoublequote}{\isasymforall}i{\isachardot}\ k\ {\isacharequal}\ f\ i\ {\isasymlongrightarrow}\ i\ {\isasymle}\ f\ i{\isachardoublequote}%
\begin{isamarkuptxt}%
\noindent
To perform induction on \isa{k} using \isa{nat{\isacharunderscore}less{\isacharunderscore}induct}, we use the same
general induction method as for recursion induction (see
\S\ref{sec:recdef-induction}):%
\end{isamarkuptxt}%
\isacommand{apply}{\isacharparenleft}induct{\isacharunderscore}tac\ k\ rule{\isacharcolon}\ nat{\isacharunderscore}less{\isacharunderscore}induct{\isacharparenright}%
\begin{isamarkuptxt}%
\noindent
which leaves us with the following proof state:
\begin{isabelle}
\ 1.\ {\isasymAnd}\mbox{n}.\ {\isasymforall}\mbox{m}.\ \mbox{m}\ <\ \mbox{n}\ {\isasymlongrightarrow}\ ({\isasymforall}\mbox{i}.\ \mbox{m}\ =\ f\ \mbox{i}\ {\isasymlongrightarrow}\ \mbox{i}\ {\isasymle}\ f\ \mbox{i})\isanewline
\ \ \ \ \ \ \ {\isasymLongrightarrow}\ {\isasymforall}\mbox{i}.\ \mbox{n}\ =\ f\ \mbox{i}\ {\isasymlongrightarrow}\ \mbox{i}\ {\isasymle}\ f\ \mbox{i}
\end{isabelle}
After stripping the \isa{{\isasymforall}i}, the proof continues with a case
distinction on \isa{i}. The case \isa{i\ {\isacharequal}\ \isadigit{0}} is trivial and we focus on
the other case:
\begin{isabelle}
\ 1.\ {\isasymAnd}\mbox{n}\ \mbox{i}\ \mbox{nat}.\isanewline
\ \ \ \ \ \ \ {\isasymlbrakk}{\isasymforall}\mbox{m}.\ \mbox{m}\ <\ \mbox{n}\ {\isasymlongrightarrow}\ ({\isasymforall}\mbox{i}.\ \mbox{m}\ =\ f\ \mbox{i}\ {\isasymlongrightarrow}\ \mbox{i}\ {\isasymle}\ f\ \mbox{i});\ \mbox{i}\ =\ Suc\ \mbox{nat}{\isasymrbrakk}\isanewline
\ \ \ \ \ \ \ {\isasymLongrightarrow}\ \mbox{n}\ =\ f\ \mbox{i}\ {\isasymlongrightarrow}\ \mbox{i}\ {\isasymle}\ f\ \mbox{i}
\end{isabelle}%
\end{isamarkuptxt}%
\isacommand{by}{\isacharparenleft}blast\ intro{\isacharbang}{\isacharcolon}\ f{\isacharunderscore}ax\ Suc{\isacharunderscore}leI\ intro{\isacharcolon}\ le{\isacharunderscore}less{\isacharunderscore}trans{\isacharparenright}%
\begin{isamarkuptext}%
\noindent
It is not surprising if you find the last step puzzling.
The proof goes like this (writing \isa{j} instead of \isa{nat}).
Since \isa{i\ {\isacharequal}\ Suc\ j} it suffices to show
\isa{j\ {\isacharless}\ f\ {\isacharparenleft}Suc\ j{\isacharparenright}} (by \isa{Suc{\isacharunderscore}leI}: \isa{m\ {\isacharless}\ n\ {\isasymLongrightarrow}\ Suc\ m\ {\isasymle}\ n}). This is
proved as follows. From \isa{f{\isacharunderscore}ax} we have \isa{f\ {\isacharparenleft}f\ j{\isacharparenright}\ {\isacharless}\ f\ {\isacharparenleft}Suc\ j{\isacharparenright}}
(1) which implies \isa{f\ j\ {\isasymle}\ f\ {\isacharparenleft}f\ j{\isacharparenright}} (by the induction hypothesis).
Using (1) once more we obtain \isa{f\ j\ {\isacharless}\ f\ {\isacharparenleft}Suc\ j{\isacharparenright}} (2) by transitivity
(\isa{le{\isacharunderscore}less{\isacharunderscore}trans}: \isa{{\isasymlbrakk}i\ {\isasymle}\ j{\isacharsemicolon}\ j\ {\isacharless}\ k{\isasymrbrakk}\ {\isasymLongrightarrow}\ i\ {\isacharless}\ k}).
Using the induction hypothesis once more we obtain \isa{j\ {\isasymle}\ f\ j}
which, together with (2) yields \isa{j\ {\isacharless}\ f\ {\isacharparenleft}Suc\ j{\isacharparenright}} (again by
\isa{le{\isacharunderscore}less{\isacharunderscore}trans}).

This last step shows both the power and the danger of automatic proofs: they
will usually not tell you how the proof goes, because it can be very hard to
translate the internal proof into a human-readable format. Therefore
\S\ref{sec:part2?} introduces a language for writing readable yet concise
proofs.

We can now derive the desired \isa{i\ {\isasymle}\ f\ i} from \isa{f{\isacharunderscore}incr}:%
\end{isamarkuptext}%
\isacommand{lemmas}\ f{\isacharunderscore}incr\ {\isacharequal}\ f{\isacharunderscore}incr{\isacharunderscore}lem{\isacharbrackleft}rule{\isacharunderscore}format{\isacharcomma}\ OF\ refl{\isacharbrackright}%
\begin{isamarkuptext}%
\noindent
The final \isa{refl} gets rid of the premise \isa{{\isacharquery}k\ {\isacharequal}\ f\ {\isacharquery}i}. Again,
we could have included this derivation in the original statement of the lemma:%
\end{isamarkuptext}%
\isacommand{lemma}\ f{\isacharunderscore}incr{\isacharbrackleft}rule{\isacharunderscore}format{\isacharcomma}\ OF\ refl{\isacharbrackright}{\isacharcolon}\ {\isachardoublequote}{\isasymforall}i{\isachardot}\ k\ {\isacharequal}\ f\ i\ {\isasymlongrightarrow}\ i\ {\isasymle}\ f\ i{\isachardoublequote}%
\begin{isamarkuptext}%
\begin{exercise}
From the above axiom and lemma for \isa{f} show that \isa{f} is the
identity.
\end{exercise}

In general, \isa{induct{\isacharunderscore}tac} can be applied with any rule $r$
whose conclusion is of the form ${?}P~?x@1 \dots ?x@n$, in which case the
format is
\begin{quote}
\isacommand{apply}\isa{{\isacharparenleft}induct{\isacharunderscore}tac} $y@1 \dots y@n$ \isa{rule{\isacharcolon}} $r$\isa{{\isacharparenright}}
\end{quote}\index{*induct_tac}%
where $y@1, \dots, y@n$ are variables in the first subgoal.
In fact, \isa{induct{\isacharunderscore}tac} even allows the conclusion of
$r$ to be an (iterated) conjunction of formulae of the above form, in
which case the application is
\begin{quote}
\isacommand{apply}\isa{{\isacharparenleft}induct{\isacharunderscore}tac} $y@1 \dots y@n$ \isa{and} \dots\ \isa{and} $z@1 \dots z@m$ \isa{rule{\isacharcolon}} $r$\isa{{\isacharparenright}}
\end{quote}%
\end{isamarkuptext}%
%
\isamarkupsubsection{Derivation of new induction schemas}
%
\begin{isamarkuptext}%
\label{sec:derive-ind}
Induction schemas are ordinary theorems and you can derive new ones
whenever you wish.  This section shows you how to, using the example
of \isa{nat{\isacharunderscore}less{\isacharunderscore}induct}. Assume we only have structural induction
available for \isa{nat} and want to derive complete induction. This
requires us to generalize the statement first:%
\end{isamarkuptext}%
\isacommand{lemma}\ induct{\isacharunderscore}lem{\isacharcolon}\ {\isachardoublequote}{\isacharparenleft}{\isasymAnd}n{\isacharcolon}{\isacharcolon}nat{\isachardot}\ {\isasymforall}m{\isacharless}n{\isachardot}\ P\ m\ {\isasymLongrightarrow}\ P\ n{\isacharparenright}\ {\isasymLongrightarrow}\ {\isasymforall}m{\isacharless}n{\isachardot}\ P\ m{\isachardoublequote}\isanewline
\isacommand{apply}{\isacharparenleft}induct{\isacharunderscore}tac\ n{\isacharparenright}%
\begin{isamarkuptxt}%
\noindent
The base case is trivially true. For the induction step (\isa{m\ {\isacharless}\ Suc\ n}) we distinguish two cases: case \isa{m\ {\isacharless}\ n} is true by induction
hypothesis and case \isa{m\ {\isacharequal}\ n} follows from the assumption, again using
the induction hypothesis:%
\end{isamarkuptxt}%
\isacommand{apply}{\isacharparenleft}blast{\isacharparenright}\isanewline
\isacommand{by}{\isacharparenleft}blast\ elim{\isacharcolon}less{\isacharunderscore}SucE{\isacharparenright}%
\begin{isamarkuptext}%
\noindent
The elimination rule \isa{less{\isacharunderscore}SucE} expresses the case distinction:
\begin{isabelle}%
\ \ \ \ \ {\isasymlbrakk}m\ {\isacharless}\ Suc\ n{\isacharsemicolon}\ m\ {\isacharless}\ n\ {\isasymLongrightarrow}\ P{\isacharsemicolon}\ m\ {\isacharequal}\ n\ {\isasymLongrightarrow}\ P{\isasymrbrakk}\ {\isasymLongrightarrow}\ P%
\end{isabelle}

Now it is straightforward to derive the original version of
\isa{nat{\isacharunderscore}less{\isacharunderscore}induct} by manipulting the conclusion of the above lemma:
instantiate \isa{n} by \isa{Suc\ n} and \isa{m} by \isa{n} and
remove the trivial condition \isa{n\ {\isacharless}\ Sc\ n}. Fortunately, this
happens automatically when we add the lemma as a new premise to the
desired goal:%
\end{isamarkuptext}%
\isacommand{theorem}\ nat{\isacharunderscore}less{\isacharunderscore}induct{\isacharcolon}\ {\isachardoublequote}{\isacharparenleft}{\isasymAnd}n{\isacharcolon}{\isacharcolon}nat{\isachardot}\ {\isasymforall}m{\isacharless}n{\isachardot}\ P\ m\ {\isasymLongrightarrow}\ P\ n{\isacharparenright}\ {\isasymLongrightarrow}\ P\ n{\isachardoublequote}\isanewline
\isacommand{by}{\isacharparenleft}insert\ induct{\isacharunderscore}lem{\isacharcomma}\ blast{\isacharparenright}%
\begin{isamarkuptext}%
Finally we should mention that HOL already provides the mother of all
inductions, \emph{wellfounded induction} (\isa{wf{\isacharunderscore}induct}):
\begin{isabelle}%
\ \ \ \ \ {\isasymlbrakk}wf\ r{\isacharsemicolon}\ {\isasymAnd}x{\isachardot}\ {\isasymforall}y{\isachardot}\ {\isacharparenleft}y{\isacharcomma}\ x{\isacharparenright}\ {\isasymin}\ r\ {\isasymlongrightarrow}\ P\ y\ {\isasymLongrightarrow}\ P\ x{\isasymrbrakk}\ {\isasymLongrightarrow}\ P\ a%
\end{isabelle}
where \isa{wf\ r} means that the relation \isa{r} is wellfounded.
For example, theorem \isa{nat{\isacharunderscore}less{\isacharunderscore}induct} can be viewed (and
derived) as a special case of \isa{wf{\isacharunderscore}induct} where 
\isa{r} is \isa{{\isacharless}} on \isa{nat}. For details see the library.%
\end{isamarkuptext}%
\end{isabellebody}%
%%% Local Variables:
%%% mode: latex
%%% TeX-master: "root"
%%% End:

%
\begin{isabellebody}%
\def\isabellecontext{CTLind}%
%
\isamarkupsubsection{CTL revisited%
}
%
\begin{isamarkuptext}%
\label{sec:CTL-revisited}
The purpose of this section is twofold: we want to demonstrate
some of the induction principles and heuristics discussed above and we want to
show how inductive definitions can simplify proofs.
In \S\ref{sec:CTL} we gave a fairly involved proof of the correctness of a
model checker for CTL\@. In particular the proof of the
\isa{infinity{\isacharunderscore}lemma} on the way to \isa{AF{\isacharunderscore}lemma{\isadigit{2}}} is not as
simple as one might intuitively expect, due to the \isa{SOME} operator
involved. Below we give a simpler proof of \isa{AF{\isacharunderscore}lemma{\isadigit{2}}}
based on an auxiliary inductive definition.

Let us call a (finite or infinite) path \emph{\isa{A}-avoiding} if it does
not touch any node in the set \isa{A}. Then \isa{AF{\isacharunderscore}lemma{\isadigit{2}}} says
that if no infinite path from some state \isa{s} is \isa{A}-avoiding,
then \isa{s\ {\isasymin}\ lfp\ {\isacharparenleft}af\ A{\isacharparenright}}. We prove this by inductively defining the set
\isa{Avoid\ s\ A} of states reachable from \isa{s} by a finite \isa{A}-avoiding path:
% Second proof of opposite direction, directly by well-founded induction
% on the initial segment of M that avoids A.%
\end{isamarkuptext}%
\isacommand{consts}\ Avoid\ {\isacharcolon}{\isacharcolon}\ {\isachardoublequote}state\ {\isasymRightarrow}\ state\ set\ {\isasymRightarrow}\ state\ set{\isachardoublequote}\isanewline
\isacommand{inductive}\ {\isachardoublequote}Avoid\ s\ A{\isachardoublequote}\isanewline
\isakeyword{intros}\ {\isachardoublequote}s\ {\isasymin}\ Avoid\ s\ A{\isachardoublequote}\isanewline
\ \ \ \ \ \ \ {\isachardoublequote}{\isasymlbrakk}\ t\ {\isasymin}\ Avoid\ s\ A{\isacharsemicolon}\ t\ {\isasymnotin}\ A{\isacharsemicolon}\ {\isacharparenleft}t{\isacharcomma}u{\isacharparenright}\ {\isasymin}\ M\ {\isasymrbrakk}\ {\isasymLongrightarrow}\ u\ {\isasymin}\ Avoid\ s\ A{\isachardoublequote}%
\begin{isamarkuptext}%
It is easy to see that for any infinite \isa{A}-avoiding path \isa{f}
with \isa{f\ {\isadigit{0}}\ {\isasymin}\ Avoid\ s\ A} there is an infinite \isa{A}-avoiding path
starting with \isa{s} because (by definition of \isa{Avoid}) there is a
finite \isa{A}-avoiding path from \isa{s} to \isa{f\ {\isadigit{0}}}.
The proof is by induction on \isa{f\ {\isadigit{0}}\ {\isasymin}\ Avoid\ s\ A}. However,
this requires the following
reformulation, as explained in \S\ref{sec:ind-var-in-prems} above;
the \isa{rule{\isacharunderscore}format} directive undoes the reformulation after the proof.%
\end{isamarkuptext}%
\isacommand{lemma}\ ex{\isacharunderscore}infinite{\isacharunderscore}path{\isacharbrackleft}rule{\isacharunderscore}format{\isacharbrackright}{\isacharcolon}\isanewline
\ \ {\isachardoublequote}t\ {\isasymin}\ Avoid\ s\ A\ \ {\isasymLongrightarrow}\isanewline
\ \ \ {\isasymforall}f{\isasymin}Paths\ t{\isachardot}\ {\isacharparenleft}{\isasymforall}i{\isachardot}\ f\ i\ {\isasymnotin}\ A{\isacharparenright}\ {\isasymlongrightarrow}\ {\isacharparenleft}{\isasymexists}p{\isasymin}Paths\ s{\isachardot}\ {\isasymforall}i{\isachardot}\ p\ i\ {\isasymnotin}\ A{\isacharparenright}{\isachardoublequote}\isanewline
\isacommand{apply}{\isacharparenleft}erule\ Avoid{\isachardot}induct{\isacharparenright}\isanewline
\ \isacommand{apply}{\isacharparenleft}blast{\isacharparenright}\isanewline
\isacommand{apply}{\isacharparenleft}clarify{\isacharparenright}\isanewline
\isacommand{apply}{\isacharparenleft}drule{\isacharunderscore}tac\ x\ {\isacharequal}\ {\isachardoublequote}{\isasymlambda}i{\isachardot}\ case\ i\ of\ {\isadigit{0}}\ {\isasymRightarrow}\ t\ {\isacharbar}\ Suc\ i\ {\isasymRightarrow}\ f\ i{\isachardoublequote}\ \isakeyword{in}\ bspec{\isacharparenright}\isanewline
\isacommand{apply}{\isacharparenleft}simp{\isacharunderscore}all\ add{\isacharcolon}Paths{\isacharunderscore}def\ split{\isacharcolon}nat{\isachardot}split{\isacharparenright}\isanewline
\isacommand{done}%
\begin{isamarkuptext}%
\noindent
The base case (\isa{t\ {\isacharequal}\ s}) is trivial (\isa{blast}).
In the induction step, we have an infinite \isa{A}-avoiding path \isa{f}
starting from \isa{u}, a successor of \isa{t}. Now we simply instantiate
the \isa{{\isasymforall}f{\isasymin}Paths\ t} in the induction hypothesis by the path starting with
\isa{t} and continuing with \isa{f}. That is what the above $\lambda$-term
expresses. That fact that this is a path starting with \isa{t} and that
the instantiated induction hypothesis implies the conclusion is shown by
simplification.

Now we come to the key lemma. It says that if \isa{t} can be reached by a
finite \isa{A}-avoiding path from \isa{s}, then \isa{t\ {\isasymin}\ lfp\ {\isacharparenleft}af\ A{\isacharparenright}},
provided there is no infinite \isa{A}-avoiding path starting from \isa{s}.%
\end{isamarkuptext}%
\isacommand{lemma}\ Avoid{\isacharunderscore}in{\isacharunderscore}lfp{\isacharbrackleft}rule{\isacharunderscore}format{\isacharparenleft}no{\isacharunderscore}asm{\isacharparenright}{\isacharbrackright}{\isacharcolon}\isanewline
\ \ {\isachardoublequote}{\isasymforall}p{\isasymin}Paths\ s{\isachardot}\ {\isasymexists}i{\isachardot}\ p\ i\ {\isasymin}\ A\ {\isasymLongrightarrow}\ t\ {\isasymin}\ Avoid\ s\ A\ {\isasymlongrightarrow}\ t\ {\isasymin}\ lfp{\isacharparenleft}af\ A{\isacharparenright}{\isachardoublequote}%
\begin{isamarkuptxt}%
\noindent
The trick is not to induct on \isa{t\ {\isasymin}\ Avoid\ s\ A}, as already the base
case would be a problem, but to proceed by well-founded induction \isa{t}. Hence \isa{t\ {\isasymin}\ Avoid\ s\ A} needs to be brought into the conclusion as
well, which the directive \isa{rule{\isacharunderscore}format} undoes at the end (see below).
But induction with respect to which well-founded relation? The restriction
of \isa{M} to \isa{Avoid\ s\ A}:
\begin{isabelle}%
\ \ \ \ \ {\isacharbraceleft}{\isacharparenleft}y{\isacharcomma}\ x{\isacharparenright}{\isachardot}\ {\isacharparenleft}x{\isacharcomma}\ y{\isacharparenright}\ {\isasymin}\ M\ {\isasymand}\ x\ {\isasymin}\ Avoid\ s\ A\ {\isasymand}\ y\ {\isasymin}\ Avoid\ s\ A\ {\isasymand}\ x\ {\isasymnotin}\ A{\isacharbraceright}%
\end{isabelle}
As we shall see in a moment, the absence of infinite \isa{A}-avoiding paths
starting from \isa{s} implies well-foundedness of this relation. For the
moment we assume this and proceed with the induction:%
\end{isamarkuptxt}%
\isacommand{apply}{\isacharparenleft}subgoal{\isacharunderscore}tac\isanewline
\ \ {\isachardoublequote}wf{\isacharbraceleft}{\isacharparenleft}y{\isacharcomma}x{\isacharparenright}{\isachardot}\ {\isacharparenleft}x{\isacharcomma}y{\isacharparenright}{\isasymin}M\ {\isasymand}\ x\ {\isasymin}\ Avoid\ s\ A\ {\isasymand}\ y\ {\isasymin}\ Avoid\ s\ A\ {\isasymand}\ x\ {\isasymnotin}\ A{\isacharbraceright}{\isachardoublequote}{\isacharparenright}\isanewline
\ \isacommand{apply}{\isacharparenleft}erule{\isacharunderscore}tac\ a\ {\isacharequal}\ t\ \isakeyword{in}\ wf{\isacharunderscore}induct{\isacharparenright}\isanewline
\ \isacommand{apply}{\isacharparenleft}clarsimp{\isacharparenright}%
\begin{isamarkuptxt}%
\noindent
Now can assume additionally (induction hypothesis) that if \isa{t\ {\isasymnotin}\ A}
then all successors of \isa{t} that are in \isa{Avoid\ s\ A} are in
\isa{lfp\ {\isacharparenleft}af\ A{\isacharparenright}}. To prove the actual goal we unfold \isa{lfp} once. Now
we have to prove that \isa{t} is in \isa{A} or all successors of \isa{t} are in \isa{lfp\ {\isacharparenleft}af\ A{\isacharparenright}}. If \isa{t} is not in \isa{A}, the second
\isa{Avoid}-rule implies that all successors of \isa{t} are in
\isa{Avoid\ s\ A} (because we also assume \isa{t\ {\isasymin}\ Avoid\ s\ A}), and
hence, by the induction hypothesis, all successors of \isa{t} are indeed in
\isa{lfp\ {\isacharparenleft}af\ A{\isacharparenright}}. Mechanically:%
\end{isamarkuptxt}%
\ \isacommand{apply}{\isacharparenleft}rule\ ssubst\ {\isacharbrackleft}OF\ lfp{\isacharunderscore}unfold{\isacharbrackleft}OF\ mono{\isacharunderscore}af{\isacharbrackright}{\isacharbrackright}{\isacharparenright}\isanewline
\ \isacommand{apply}{\isacharparenleft}simp\ only{\isacharcolon}\ af{\isacharunderscore}def{\isacharparenright}\isanewline
\ \isacommand{apply}{\isacharparenleft}blast\ intro{\isacharcolon}Avoid{\isachardot}intros{\isacharparenright}%
\begin{isamarkuptxt}%
Having proved the main goal we return to the proof obligation that the above
relation is indeed well-founded. This is proved by contraposition: we assume
the relation is not well-founded. Thus there exists an infinite \isa{A}-avoiding path all in \isa{Avoid\ s\ A}, by theorem
\isa{wf{\isacharunderscore}iff{\isacharunderscore}no{\isacharunderscore}infinite{\isacharunderscore}down{\isacharunderscore}chain}:
\begin{isabelle}%
\ \ \ \ \ wf\ r\ {\isacharequal}\ {\isacharparenleft}{\isasymnot}\ {\isacharparenleft}{\isasymexists}f{\isachardot}\ {\isasymforall}i{\isachardot}\ {\isacharparenleft}f\ {\isacharparenleft}Suc\ i{\isacharparenright}{\isacharcomma}\ f\ i{\isacharparenright}\ {\isasymin}\ r{\isacharparenright}{\isacharparenright}%
\end{isabelle}
From lemma \isa{ex{\isacharunderscore}infinite{\isacharunderscore}path} the existence of an infinite
\isa{A}-avoiding path starting in \isa{s} follows, just as required for
the contraposition.%
\end{isamarkuptxt}%
\isacommand{apply}{\isacharparenleft}erule\ contrapos{\isacharunderscore}pp{\isacharparenright}\isanewline
\isacommand{apply}{\isacharparenleft}simp\ add{\isacharcolon}wf{\isacharunderscore}iff{\isacharunderscore}no{\isacharunderscore}infinite{\isacharunderscore}down{\isacharunderscore}chain{\isacharparenright}\isanewline
\isacommand{apply}{\isacharparenleft}erule\ exE{\isacharparenright}\isanewline
\isacommand{apply}{\isacharparenleft}rule\ ex{\isacharunderscore}infinite{\isacharunderscore}path{\isacharparenright}\isanewline
\isacommand{apply}{\isacharparenleft}auto\ simp\ add{\isacharcolon}Paths{\isacharunderscore}def{\isacharparenright}\isanewline
\isacommand{done}%
\begin{isamarkuptext}%
The \isa{{\isacharparenleft}no{\isacharunderscore}asm{\isacharparenright}} modifier of the \isa{rule{\isacharunderscore}format} directive means
that the assumption is left unchanged---otherwise the \isa{{\isasymforall}p} is turned
into a \isa{{\isasymAnd}p}, which would complicate matters below. As it is,
\isa{Avoid{\isacharunderscore}in{\isacharunderscore}lfp} is now
\begin{isabelle}%
\ \ \ \ \ {\isasymlbrakk}{\isasymforall}p{\isasymin}Paths\ s{\isachardot}\ {\isasymexists}i{\isachardot}\ p\ i\ {\isasymin}\ A{\isacharsemicolon}\ t\ {\isasymin}\ Avoid\ s\ A{\isasymrbrakk}\ {\isasymLongrightarrow}\ t\ {\isasymin}\ lfp\ {\isacharparenleft}af\ A{\isacharparenright}%
\end{isabelle}
The main theorem is simply the corollary where \isa{t\ {\isacharequal}\ s},
in which case the assumption \isa{t\ {\isasymin}\ Avoid\ s\ A} is trivially true
by the first \isa{Avoid}-rule). Isabelle confirms this:%
\end{isamarkuptext}%
\isacommand{theorem}\ AF{\isacharunderscore}lemma{\isadigit{2}}{\isacharcolon}\isanewline
\ \ {\isachardoublequote}{\isacharbraceleft}s{\isachardot}\ {\isasymforall}p\ {\isasymin}\ Paths\ s{\isachardot}\ {\isasymexists}\ i{\isachardot}\ p\ i\ {\isasymin}\ A{\isacharbraceright}\ {\isasymsubseteq}\ lfp{\isacharparenleft}af\ A{\isacharparenright}{\isachardoublequote}\isanewline
\isacommand{by}{\isacharparenleft}auto\ elim{\isacharcolon}Avoid{\isacharunderscore}in{\isacharunderscore}lfp\ intro{\isacharcolon}Avoid{\isachardot}intros{\isacharparenright}\isanewline
\isanewline
\end{isabellebody}%
%%% Local Variables:
%%% mode: latex
%%% TeX-master: "root"
%%% End:

\index{induction|)}

%\section{Advanced Forms of Recursion}
%\index{recdef@\isacommand {recdef} (command)|(}

%This section introduces advanced forms of
%\isacommand{recdef}: how to establish termination by means other than measure
%functions, how to define recursive functions over nested recursive datatypes
%and how to deal with partial functions.
%
%If, after reading this section, you feel that the definition of recursive
%functions is overly complicated by the requirement of
%totality, you should ponder the alternatives.  In a logic of partial functions,
%recursive definitions are always accepted.  But there are many
%such logics, and no clear winner has emerged. And in all of these logics you
%are (more or less frequently) required to reason about the definedness of
%terms explicitly. Thus one shifts definedness arguments from definition time to
%proof time. In HOL you may have to work hard to define a function, but proofs
%can then proceed unencumbered by worries about undefinedness.

%\subsection{Beyond Measure}
%\label{sec:beyond-measure}
%%
\begin{isabellebody}%
\def\isabellecontext{WFrec}%
%
\begin{isamarkuptext}%
\noindent
So far, all recursive definitions where shown to terminate via measure
functions. Sometimes this can be quite inconvenient or even
impossible. Fortunately, \isacommand{recdef} supports much more
general definitions. For example, termination of Ackermann's function
can be shown by means of the lexicographic product \isa{{\isacharless}{\isacharasterisk}lex{\isacharasterisk}{\isachargreater}}:%
\end{isamarkuptext}%
\isacommand{consts}\ ack\ {\isacharcolon}{\isacharcolon}\ {\isachardoublequote}nat{\isasymtimes}nat\ {\isasymRightarrow}\ nat{\isachardoublequote}\isanewline
\isacommand{recdef}\ ack\ {\isachardoublequote}measure{\isacharparenleft}{\isasymlambda}m{\isachardot}\ m{\isacharparenright}\ {\isacharless}{\isacharasterisk}lex{\isacharasterisk}{\isachargreater}\ measure{\isacharparenleft}{\isasymlambda}n{\isachardot}\ n{\isacharparenright}{\isachardoublequote}\isanewline
\ \ {\isachardoublequote}ack{\isacharparenleft}{\isadigit{0}}{\isacharcomma}n{\isacharparenright}\ \ \ \ \ \ \ \ \ {\isacharequal}\ Suc\ n{\isachardoublequote}\isanewline
\ \ {\isachardoublequote}ack{\isacharparenleft}Suc\ m{\isacharcomma}{\isadigit{0}}{\isacharparenright}\ \ \ \ \ {\isacharequal}\ ack{\isacharparenleft}m{\isacharcomma}\ {\isadigit{1}}{\isacharparenright}{\isachardoublequote}\isanewline
\ \ {\isachardoublequote}ack{\isacharparenleft}Suc\ m{\isacharcomma}Suc\ n{\isacharparenright}\ {\isacharequal}\ ack{\isacharparenleft}m{\isacharcomma}ack{\isacharparenleft}Suc\ m{\isacharcomma}n{\isacharparenright}{\isacharparenright}{\isachardoublequote}%
\begin{isamarkuptext}%
\noindent
The lexicographic product decreases if either its first component
decreases (as in the second equation and in the outer call in the
third equation) or its first component stays the same and the second
component decreases (as in the inner call in the third equation).

In general, \isacommand{recdef} supports termination proofs based on
arbitrary well-founded relations as introduced in \S\ref{sec:Well-founded}.
This is called \textbf{well-founded
recursion}\indexbold{recursion!well-founded}\index{well-founded
recursion|see{recursion, well-founded}}. Clearly, a function definition is
total iff the set of all pairs $(r,l)$, where $l$ is the argument on the
left-hand side of an equation and $r$ the argument of some recursive call on
the corresponding right-hand side, induces a well-founded relation.  For a
systematic account of termination proofs via well-founded relations see, for
example, \cite{Baader-Nipkow}.

Each \isacommand{recdef} definition should be accompanied (after the name of
the function) by a well-founded relation on the argument type of the
function.  The HOL library formalizes some of the most important
constructions of well-founded relations (see \S\ref{sec:Well-founded}). For
example, \isa{measure\ f} is always well-founded, and the lexicographic
product of two well-founded relations is again well-founded, which we relied
on when defining Ackermann's function above.
Of course the lexicographic product can also be interated:%
\end{isamarkuptext}%
\isacommand{consts}\ contrived\ {\isacharcolon}{\isacharcolon}\ {\isachardoublequote}nat\ {\isasymtimes}\ nat\ {\isasymtimes}\ nat\ {\isasymRightarrow}\ nat{\isachardoublequote}\isanewline
\isacommand{recdef}\ contrived\isanewline
\ \ {\isachardoublequote}measure{\isacharparenleft}{\isasymlambda}i{\isachardot}\ i{\isacharparenright}\ {\isacharless}{\isacharasterisk}lex{\isacharasterisk}{\isachargreater}\ measure{\isacharparenleft}{\isasymlambda}j{\isachardot}\ j{\isacharparenright}\ {\isacharless}{\isacharasterisk}lex{\isacharasterisk}{\isachargreater}\ measure{\isacharparenleft}{\isasymlambda}k{\isachardot}\ k{\isacharparenright}{\isachardoublequote}\isanewline
{\isachardoublequote}contrived{\isacharparenleft}i{\isacharcomma}j{\isacharcomma}Suc\ k{\isacharparenright}\ {\isacharequal}\ contrived{\isacharparenleft}i{\isacharcomma}j{\isacharcomma}k{\isacharparenright}{\isachardoublequote}\isanewline
{\isachardoublequote}contrived{\isacharparenleft}i{\isacharcomma}Suc\ j{\isacharcomma}{\isadigit{0}}{\isacharparenright}\ {\isacharequal}\ contrived{\isacharparenleft}i{\isacharcomma}j{\isacharcomma}j{\isacharparenright}{\isachardoublequote}\isanewline
{\isachardoublequote}contrived{\isacharparenleft}Suc\ i{\isacharcomma}{\isadigit{0}}{\isacharcomma}{\isadigit{0}}{\isacharparenright}\ {\isacharequal}\ contrived{\isacharparenleft}i{\isacharcomma}i{\isacharcomma}i{\isacharparenright}{\isachardoublequote}\isanewline
{\isachardoublequote}contrived{\isacharparenleft}{\isadigit{0}}{\isacharcomma}{\isadigit{0}}{\isacharcomma}{\isadigit{0}}{\isacharparenright}\ \ \ \ \ {\isacharequal}\ {\isadigit{0}}{\isachardoublequote}%
\begin{isamarkuptext}%
Lexicographic products of measure functions already go a long
way. Furthermore you may embed some type in an
existing well-founded relation via the inverse image construction \isa{inv{\isacharunderscore}image}. All these constructions are known to \isacommand{recdef}. Thus you
will never have to prove well-foundedness of any relation composed
solely of these building blocks. But of course the proof of
termination of your function definition, i.e.\ that the arguments
decrease with every recursive call, may still require you to provide
additional lemmas.

It is also possible to use your own well-founded relations with \isacommand{recdef}.
Here is a simplistic example:%
\end{isamarkuptext}%
\isacommand{consts}\ f\ {\isacharcolon}{\isacharcolon}\ {\isachardoublequote}nat\ {\isasymRightarrow}\ nat{\isachardoublequote}\isanewline
\isacommand{recdef}\ f\ {\isachardoublequote}id{\isacharparenleft}less{\isacharunderscore}than{\isacharparenright}{\isachardoublequote}\isanewline
{\isachardoublequote}f\ {\isadigit{0}}\ {\isacharequal}\ {\isadigit{0}}{\isachardoublequote}\isanewline
{\isachardoublequote}f\ {\isacharparenleft}Suc\ n{\isacharparenright}\ {\isacharequal}\ f\ n{\isachardoublequote}%
\begin{isamarkuptext}%
\noindent
Since \isacommand{recdef} is not prepared for \isa{id}, the identity
function, this leads to the complaint that it could not prove
\isa{wf\ {\isacharparenleft}id\ less{\isacharunderscore}than{\isacharparenright}}.
We should first have proved that \isa{id} preserves well-foundedness%
\end{isamarkuptext}%
\isacommand{lemma}\ wf{\isacharunderscore}id{\isacharcolon}\ {\isachardoublequote}wf\ r\ {\isasymLongrightarrow}\ wf{\isacharparenleft}id\ r{\isacharparenright}{\isachardoublequote}\isanewline
\isacommand{by}\ simp%
\begin{isamarkuptext}%
\noindent
and should have appended the following hint to our above definition:%
\end{isamarkuptext}%
{\isacharparenleft}\isakeyword{hints}\ recdef{\isacharunderscore}wf\ add{\isacharcolon}\ wf{\isacharunderscore}id{\isacharparenright}\end{isabellebody}%
%%% Local Variables:
%%% mode: latex
%%% TeX-master: "root"
%%% End:

%
%\subsection{Recursion Over Nested Datatypes}
%\label{sec:nested-recdef}
%%
\begin{isabellebody}%
\def\isabellecontext{Nested{\isadigit{0}}}%
%
\begin{isamarkuptext}%
In \S\ref{sec:nested-datatype} we defined the datatype of terms%
\end{isamarkuptext}%
\isacommand{datatype}\ {\isacharparenleft}{\isacharprime}a{\isacharcomma}{\isacharprime}b{\isacharparenright}{\isachardoublequote}term{\isachardoublequote}\ {\isacharequal}\ Var\ {\isacharprime}a\ {\isacharbar}\ App\ {\isacharprime}b\ {\isachardoublequote}{\isacharparenleft}{\isacharprime}a{\isacharcomma}{\isacharprime}b{\isacharparenright}term\ list{\isachardoublequote}%
\begin{isamarkuptext}%
\noindent
and closed with the observation that the associated schema for the definition
of primitive recursive functions leads to overly verbose definitions. Moreover,
if you have worked exercise~\ref{ex:trev-trev} you will have noticed that
you needed to reprove many lemmas reminiscent of similar lemmas about
\isa{rev}. We will now show you how \isacommand{recdef} can simplify
definitions and proofs about nested recursive datatypes. As an example we
choose exercise~\ref{ex:trev-trev}:%
\end{isamarkuptext}%
\isacommand{consts}\ trev\ \ {\isacharcolon}{\isacharcolon}\ {\isachardoublequote}{\isacharparenleft}{\isacharprime}a{\isacharcomma}{\isacharprime}b{\isacharparenright}term\ {\isasymRightarrow}\ {\isacharparenleft}{\isacharprime}a{\isacharcomma}{\isacharprime}b{\isacharparenright}term{\isachardoublequote}\end{isabellebody}%
%%% Local Variables:
%%% mode: latex
%%% TeX-master: "root"
%%% End:

%%
\begin{isabellebody}%
\def\isabellecontext{Nested1}%
\isacommand{consts}\ trev\ \ {\isacharcolon}{\isacharcolon}\ {\isachardoublequote}{\isacharparenleft}{\isacharprime}a{\isacharcomma}{\isacharprime}b{\isacharparenright}term\ {\isasymRightarrow}\ {\isacharparenleft}{\isacharprime}a{\isacharcomma}{\isacharprime}b{\isacharparenright}term{\isachardoublequote}%
\begin{isamarkuptext}%
\noindent
Although the definition of \isa{trev} is quite natural, we will have
overcome a minor difficulty in convincing Isabelle of is termination.
It is precisely this difficulty that is the \textit{raison d'\^etre} of
this subsection.

Defining \isa{trev} by \isacommand{recdef} rather than \isacommand{primrec}
simplifies matters because we are now free to use the recursion equation
suggested at the end of \S\ref{sec:nested-datatype}:%
\end{isamarkuptext}%
\isacommand{recdef}\ trev\ {\isachardoublequote}measure\ size{\isachardoublequote}\isanewline
\ {\isachardoublequote}trev\ {\isacharparenleft}Var\ x{\isacharparenright}\ \ \ \ {\isacharequal}\ Var\ x{\isachardoublequote}\isanewline
\ {\isachardoublequote}trev\ {\isacharparenleft}App\ f\ ts{\isacharparenright}\ {\isacharequal}\ App\ f\ {\isacharparenleft}rev{\isacharparenleft}map\ trev\ ts{\isacharparenright}{\isacharparenright}{\isachardoublequote}%
\begin{isamarkuptext}%
\noindent
Remember that function \isa{size} is defined for each \isacommand{datatype}.
However, the definition does not succeed. Isabelle complains about an
unproved termination condition
\begin{isabelle}%
\ \ \ \ \ t\ {\isasymin}\ set\ ts\ {\isasymlongrightarrow}\ size\ t\ {\isacharless}\ Suc\ {\isacharparenleft}term{\isacharunderscore}list{\isacharunderscore}size\ ts{\isacharparenright}%
\end{isabelle}
where \isa{set} returns the set of elements of a list (no special
knowledge of sets is required in the following) and \isa{term{\isacharunderscore}list{\isacharunderscore}size\ {\isacharcolon}{\isacharcolon}\ term\ list\ {\isasymRightarrow}\ nat} is an auxiliary function automatically defined by Isabelle
(when \isa{term} was defined).  First we have to understand why the
recursive call of \isa{trev} underneath \isa{map} leads to the above
condition. The reason is that \isacommand{recdef} ``knows'' that \isa{map}
will apply \isa{trev} only to elements of \isa{ts}. Thus the above
condition expresses that the size of the argument \isa{t\ {\isasymin}\ set\ ts} of any
recursive call of \isa{trev} is strictly less than \isa{size\ {\isacharparenleft}App\ f\ ts{\isacharparenright}\ {\isacharequal}\ Suc\ {\isacharparenleft}term{\isacharunderscore}list{\isacharunderscore}size\ ts{\isacharparenright}}.  We will now prove the termination condition and
continue with our definition.  Below we return to the question of how
\isacommand{recdef} ``knows'' about \isa{map}.%
\end{isamarkuptext}%
\end{isabellebody}%
%%% Local Variables:
%%% mode: latex
%%% TeX-master: "root"
%%% End:

%%
\begin{isabellebody}%
%
\begin{isamarkuptext}%
\noindent
The termintion condition is easily proved by induction:%
\end{isamarkuptext}%
\isacommand{lemma}\ {\isacharbrackleft}simp{\isacharbrackright}{\isacharcolon}\ {\isachardoublequote}t\ {\isasymin}\ set\ ts\ {\isasymlongrightarrow}\ size\ t\ {\isacharless}\ Suc{\isacharparenleft}term{\isacharunderscore}size\ ts{\isacharparenright}{\isachardoublequote}\isanewline
\isacommand{by}{\isacharparenleft}induct{\isacharunderscore}tac\ ts{\isacharcomma}\ auto{\isacharparenright}%
\begin{isamarkuptext}%
\noindent
By making this theorem a simplification rule, \isacommand{recdef}
applies it automatically and the above definition of \isa{trev}
succeeds now. As a reward for our effort, we can now prove the desired
lemma directly. The key is the fact that we no longer need the verbose
induction schema for type \isa{term} but the simpler one arising from
\isa{trev}:%
\end{isamarkuptext}%
\isacommand{lemmas}\ {\isacharbrackleft}cong{\isacharbrackright}\ {\isacharequal}\ map{\isacharunderscore}cong\isanewline
\isacommand{lemma}\ {\isachardoublequote}trev{\isacharparenleft}trev\ t{\isacharparenright}\ {\isacharequal}\ t{\isachardoublequote}\isanewline
\isacommand{apply}{\isacharparenleft}induct{\isacharunderscore}tac\ t\ rule{\isacharcolon}trev{\isachardot}induct{\isacharparenright}%
\begin{isamarkuptxt}%
\noindent
This leaves us with a trivial base case \isa{trev\ {\isacharparenleft}trev\ {\isacharparenleft}Var\ \mbox{x}{\isacharparenright}{\isacharparenright}\ {\isacharequal}\ Var\ \mbox{x}} and the step case
\begin{quote}

\begin{isabelle}%
{\isasymforall}\mbox{t}{\isachardot}\ \mbox{t}\ {\isasymin}\ set\ \mbox{ts}\ {\isasymlongrightarrow}\ trev\ {\isacharparenleft}trev\ \mbox{t}{\isacharparenright}\ {\isacharequal}\ \mbox{t}\ {\isasymLongrightarrow}\isanewline
trev\ {\isacharparenleft}trev\ {\isacharparenleft}App\ \mbox{f}\ \mbox{ts}{\isacharparenright}{\isacharparenright}\ {\isacharequal}\ App\ \mbox{f}\ \mbox{ts}
\end{isabelle}%

\end{quote}
both of which are solved by simplification:%
\end{isamarkuptxt}%
\isacommand{by}{\isacharparenleft}simp{\isacharunderscore}all\ del{\isacharcolon}map{\isacharunderscore}compose\ add{\isacharcolon}sym{\isacharbrackleft}OF\ map{\isacharunderscore}compose{\isacharbrackright}\ rev{\isacharunderscore}map{\isacharparenright}%
\begin{isamarkuptext}%
\noindent
If this surprises you, see Datatype/Nested2......

The above definition of \isa{trev} is superior to the one in \S\ref{sec:nested-datatype}
because it brings \isa{rev} into play, about which already know a lot, in particular
\isa{rev\ {\isacharparenleft}rev\ \mbox{xs}{\isacharparenright}\ {\isacharequal}\ \mbox{xs}}.
Thus this proof is a good example of an important principle:
\begin{quote}
\emph{Chose your definitions carefully\\
because they determine the complexity of your proofs.}
\end{quote}

Let us now return to the question of how \isacommand{recdef} can come up with sensible termination
conditions in the presence of higher-order functions like \isa{map}. For a start, if nothing
were known about \isa{map}, \isa{map\ trev\ \mbox{ts}} might apply \isa{trev} to arbitrary terms,
and thus \isacommand{recdef} would try to prove the unprovable
\isa{size\ \mbox{t}\ {\isacharless}\ Suc\ {\isacharparenleft}term{\isacharunderscore}size\ \mbox{ts}{\isacharparenright}}, without any assumption about \isa{t}.
Therefore \isacommand{recdef} has been supplied with the congruence theorem \isa{map\_cong}: 
\begin{quote}

\begin{isabelle}%
{\isasymlbrakk}\mbox{xs}\ {\isacharequal}\ \mbox{ys}{\isacharsemicolon}\ {\isasymAnd}\mbox{x}{\isachardot}\ \mbox{x}\ {\isasymin}\ set\ \mbox{ys}\ {\isasymLongrightarrow}\ \mbox{f}\ \mbox{x}\ {\isacharequal}\ \mbox{g}\ \mbox{x}{\isasymrbrakk}\isanewline
{\isasymLongrightarrow}\ map\ \mbox{f}\ \mbox{xs}\ {\isacharequal}\ map\ \mbox{g}\ \mbox{ys}
\end{isabelle}%

\end{quote}
Its second premise expresses (indirectly) that the second argument of \isa{map} is only applied
to elements of its third argument. Congruence rules for other higher-order functions on lists would
look very similar but have not been proved yet because they were never needed.
If you get into a situation where you need to supply \isacommand{recdef} with new congruence
rules, you can either append the line
\begin{ttbox}
congs <congruence rules>
\end{ttbox}
to the specific occurrence of \isacommand{recdef} or declare them globally:
\begin{ttbox}
lemmas [????????] = <congruence rules>
\end{ttbox}

Note that \isacommand{recdef} feeds on exactly the same \emph{kind} of
congruence rules as the simplifier (\S\ref{sec:simp-cong}) but that
declaring a congruence rule for the simplifier does not make it
available to \isacommand{recdef}, and vice versa. This is intentional.%
\end{isamarkuptext}%
\end{isabellebody}%
%%% Local Variables:
%%% mode: latex
%%% TeX-master: "root"
%%% End:

%
%\subsection{Partial Functions}
%\index{functions!partial}
%%
\begin{isabellebody}%
\def\isabellecontext{Partial}%
%
\begin{isamarkuptext}%
\noindent
Throughout the tutorial we have emphasized the fact that all functions
in HOL are total. Hence we cannot hope to define truly partial
functions. The best we can do are functions that are
\emph{underdefined}\index{underdefined function}:
for certain arguments we only know that a result
exists, but we don't know what it is. When defining functions that are
normally considered partial, underdefinedness turns out to be a very
reasonable alternative.

We have already seen an instance of underdefinedness by means of
non-exhaustive pattern matching: the definition of \isa{last} in
\S\ref{sec:recdef-examples}. The same is allowed for \isacommand{primrec}%
\end{isamarkuptext}%
\isacommand{consts}\ hd\ {\isacharcolon}{\isacharcolon}\ {\isachardoublequote}{\isacharprime}a\ list\ {\isasymRightarrow}\ {\isacharprime}a{\isachardoublequote}\isanewline
\isacommand{primrec}\ {\isachardoublequote}hd\ {\isacharparenleft}x{\isacharhash}xs{\isacharparenright}\ {\isacharequal}\ x{\isachardoublequote}%
\begin{isamarkuptext}%
\noindent
although it generates a warning.

Even ordinary definitions allow underdefinedness, this time by means of
preconditions:%
\end{isamarkuptext}%
\isacommand{constdefs}\ minus\ {\isacharcolon}{\isacharcolon}\ {\isachardoublequote}nat\ {\isasymRightarrow}\ nat\ {\isasymRightarrow}\ nat{\isachardoublequote}\isanewline
{\isachardoublequote}n\ {\isasymle}\ m\ {\isasymLongrightarrow}\ minus\ m\ n\ {\isasymequiv}\ m\ {\isacharminus}\ n{\isachardoublequote}%
\begin{isamarkuptext}%
The rest of this section is devoted to the question of how to define
partial recursive functions by other means that non-exhaustive pattern
matching.%
\end{isamarkuptext}%
%
\isamarkupsubsubsection{Guarded recursion%
}
%
\begin{isamarkuptext}%
Neither \isacommand{primrec} nor \isacommand{recdef} allow to
prefix an equation with a condition in the way ordinary definitions do
(see \isa{minus} above). Instead we have to move the condition over
to the right-hand side of the equation. Given a partial function $f$
that should satisfy the recursion equation $f(x) = t$ over its domain
$dom(f)$, we turn this into the \isacommand{recdef}
\begin{isabelle}%
\ \ \ \ \ f\ x\ {\isacharequal}\ {\isacharparenleft}if\ x\ {\isasymin}\ dom\ f\ then\ t\ else\ arbitrary{\isacharparenright}%
\end{isabelle}
where \isa{arbitrary} is a predeclared constant of type \isa{{\isacharprime}a}
which has no definition. Thus we know nothing about its value,
which is ideal for specifying underdefined functions on top of it.

As a simple example we define division on \isa{nat}:%
\end{isamarkuptext}%
\isacommand{consts}\ divi\ {\isacharcolon}{\isacharcolon}\ {\isachardoublequote}nat\ {\isasymtimes}\ nat\ {\isasymRightarrow}\ nat{\isachardoublequote}\isanewline
\isacommand{recdef}\ divi\ {\isachardoublequote}measure{\isacharparenleft}{\isasymlambda}{\isacharparenleft}m{\isacharcomma}n{\isacharparenright}{\isachardot}\ m{\isacharparenright}{\isachardoublequote}\isanewline
\ \ {\isachardoublequote}divi{\isacharparenleft}m{\isacharcomma}n{\isacharparenright}\ {\isacharequal}\ {\isacharparenleft}if\ n\ {\isacharequal}\ {\isadigit{0}}\ then\ arbitrary\ else\isanewline
\ \ \ \ \ \ \ \ \ \ \ \ \ \ \ \ if\ m\ {\isacharless}\ n\ then\ {\isadigit{0}}\ else\ divi{\isacharparenleft}m{\isacharminus}n{\isacharcomma}n{\isacharparenright}{\isacharplus}{\isadigit{1}}{\isacharparenright}{\isachardoublequote}%
\begin{isamarkuptext}%
\noindent Of course we could also have defined
\isa{divi\ {\isacharparenleft}m{\isacharcomma}\ {\isadigit{0}}{\isacharparenright}} to be some specific number, for example 0. The
latter option is chosen for the predefined \isa{div} function, which
simplifies proofs at the expense of moving further away from the
standard mathematical divison function.

As a more substantial example we consider the problem of searching a graph.
For simplicity our graph is given by a function (\isa{f}) of
type \isa{{\isacharprime}a\ {\isasymRightarrow}\ {\isacharprime}a} which
maps each node to its successor, and the task is to find the end of a chain,
i.e.\ a node pointing to itself. Here is a first attempt:
\begin{isabelle}%
\ \ \ \ \ find\ {\isacharparenleft}f{\isacharcomma}\ x{\isacharparenright}\ {\isacharequal}\ {\isacharparenleft}if\ f\ x\ {\isacharequal}\ x\ then\ x\ else\ find\ {\isacharparenleft}f{\isacharcomma}\ f\ x{\isacharparenright}{\isacharparenright}%
\end{isabelle}
This may be viewed as a fixed point finder or as one half of the well known
\emph{Union-Find} algorithm.
The snag is that it may not terminate if \isa{f} has nontrivial cycles.
Phrased differently, the relation%
\end{isamarkuptext}%
\isacommand{constdefs}\ step{\isadigit{1}}\ {\isacharcolon}{\isacharcolon}\ {\isachardoublequote}{\isacharparenleft}{\isacharprime}a\ {\isasymRightarrow}\ {\isacharprime}a{\isacharparenright}\ {\isasymRightarrow}\ {\isacharparenleft}{\isacharprime}a\ {\isasymtimes}\ {\isacharprime}a{\isacharparenright}set{\isachardoublequote}\isanewline
\ \ {\isachardoublequote}step{\isadigit{1}}\ f\ {\isasymequiv}\ {\isacharbraceleft}{\isacharparenleft}y{\isacharcomma}x{\isacharparenright}{\isachardot}\ y\ {\isacharequal}\ f\ x\ {\isasymand}\ y\ {\isasymnoteq}\ x{\isacharbraceright}{\isachardoublequote}%
\begin{isamarkuptext}%
\noindent
must be well-founded. Thus we make the following definition:%
\end{isamarkuptext}%
\isacommand{consts}\ find\ {\isacharcolon}{\isacharcolon}\ {\isachardoublequote}{\isacharparenleft}{\isacharprime}a\ {\isasymRightarrow}\ {\isacharprime}a{\isacharparenright}\ {\isasymtimes}\ {\isacharprime}a\ {\isasymRightarrow}\ {\isacharprime}a{\isachardoublequote}\isanewline
\isacommand{recdef}\ find\ {\isachardoublequote}same{\isacharunderscore}fst\ {\isacharparenleft}{\isasymlambda}f{\isachardot}\ wf{\isacharparenleft}step{\isadigit{1}}\ f{\isacharparenright}{\isacharparenright}\ step{\isadigit{1}}{\isachardoublequote}\isanewline
\ \ {\isachardoublequote}find{\isacharparenleft}f{\isacharcomma}x{\isacharparenright}\ {\isacharequal}\ {\isacharparenleft}if\ wf{\isacharparenleft}step{\isadigit{1}}\ f{\isacharparenright}\isanewline
\ \ \ \ \ \ \ \ \ \ \ \ \ \ \ \ then\ if\ f\ x\ {\isacharequal}\ x\ then\ x\ else\ find{\isacharparenleft}f{\isacharcomma}\ f\ x{\isacharparenright}\isanewline
\ \ \ \ \ \ \ \ \ \ \ \ \ \ \ \ else\ arbitrary{\isacharparenright}{\isachardoublequote}\isanewline
{\isacharparenleft}\isakeyword{hints}\ recdef{\isacharunderscore}simp{\isacharcolon}same{\isacharunderscore}fst{\isacharunderscore}def\ step{\isadigit{1}}{\isacharunderscore}def{\isacharparenright}%
\begin{isamarkuptext}%
\noindent
The recursion equation itself should be clear enough: it is our aborted
first attempt augmented with a check that there are no non-trivial loops.

What complicates the termination proof is that the argument of
\isa{find} is a pair. To express the required well-founded relation
we employ the predefined combinator \isa{same{\isacharunderscore}fst} of type
\begin{isabelle}%
\ \ \ \ \ {\isacharparenleft}{\isacharprime}a\ {\isasymRightarrow}\ bool{\isacharparenright}\ {\isasymRightarrow}\ {\isacharparenleft}{\isacharprime}a\ {\isasymRightarrow}\ {\isacharparenleft}{\isacharprime}b{\isasymtimes}{\isacharprime}b{\isacharparenright}set{\isacharparenright}\ {\isasymRightarrow}\ {\isacharparenleft}{\isacharparenleft}{\isacharprime}a{\isasymtimes}{\isacharprime}b{\isacharparenright}\ {\isasymtimes}\ {\isacharparenleft}{\isacharprime}a{\isasymtimes}{\isacharprime}b{\isacharparenright}{\isacharparenright}set%
\end{isabelle}
defined as
\begin{isabelle}%
\ \ \ \ \ same{\isacharunderscore}fst\ P\ R\ {\isasymequiv}\ {\isacharbraceleft}{\isacharparenleft}{\isacharparenleft}x{\isacharprime}{\isacharcomma}\ y{\isacharprime}{\isacharparenright}{\isacharcomma}\ x{\isacharcomma}\ y{\isacharparenright}{\isachardot}\ x{\isacharprime}\ {\isacharequal}\ x\ {\isasymand}\ P\ x\ {\isasymand}\ {\isacharparenleft}y{\isacharprime}{\isacharcomma}\ y{\isacharparenright}\ {\isasymin}\ R\ x{\isacharbraceright}%
\end{isabelle}
This combinator is designed for recursive functions on pairs where the first
component of the argument is passed unchanged to all recursive
calls. Given a constraint on the first component and a relation on the second
component, \isa{same{\isacharunderscore}fst} builds the required relation on pairs.
The theorem \begin{isabelle}%
\ \ \ \ \ {\isacharparenleft}{\isasymAnd}x{\isachardot}\ P\ x\ {\isasymLongrightarrow}\ wf\ {\isacharparenleft}R\ x{\isacharparenright}{\isacharparenright}\ {\isasymLongrightarrow}\ wf\ {\isacharparenleft}same{\isacharunderscore}fst\ P\ R{\isacharparenright}%
\end{isabelle}
is known to the well-foundedness prover of \isacommand{recdef}.
Thus well-foundedness of the given relation is immediate.
Furthermore, each recursive call descends along the given relation:
the first argument stays unchanged and the second one descends along
\isa{step{\isadigit{1}}\ f}. The proof merely requires unfolding of some definitions.

Normally you will then derive the following conditional variant of and from
the recursion equation%
\end{isamarkuptext}%
\isacommand{lemma}\ {\isacharbrackleft}simp{\isacharbrackright}{\isacharcolon}\isanewline
\ \ {\isachardoublequote}wf{\isacharparenleft}step{\isadigit{1}}\ f{\isacharparenright}\ {\isasymLongrightarrow}\ find{\isacharparenleft}f{\isacharcomma}x{\isacharparenright}\ {\isacharequal}\ {\isacharparenleft}if\ f\ x\ {\isacharequal}\ x\ then\ x\ else\ find{\isacharparenleft}f{\isacharcomma}\ f\ x{\isacharparenright}{\isacharparenright}{\isachardoublequote}\isanewline
\isacommand{by}\ simp%
\begin{isamarkuptext}%
\noindent and then disable the original recursion equation:%
\end{isamarkuptext}%
\isacommand{declare}\ find{\isachardot}simps{\isacharbrackleft}simp\ del{\isacharbrackright}%
\begin{isamarkuptext}%
We can reason about such underdefined functions just like about any other
recursive function. Here is a simple example of recursion induction:%
\end{isamarkuptext}%
\isacommand{lemma}\ {\isachardoublequote}wf{\isacharparenleft}step{\isadigit{1}}\ f{\isacharparenright}\ {\isasymlongrightarrow}\ f{\isacharparenleft}find{\isacharparenleft}f{\isacharcomma}x{\isacharparenright}{\isacharparenright}\ {\isacharequal}\ find{\isacharparenleft}f{\isacharcomma}x{\isacharparenright}{\isachardoublequote}\isanewline
\isacommand{apply}{\isacharparenleft}induct{\isacharunderscore}tac\ f\ x\ rule{\isacharcolon}find{\isachardot}induct{\isacharparenright}\isanewline
\isacommand{apply}\ simp\isanewline
\isacommand{done}%
\isamarkupsubsubsection{The {\tt\slshape while} combinator%
}
%
\begin{isamarkuptext}%
If the recursive function happens to be tail recursive, its
definition becomes a triviality if based on the predefined \isaindexbold{while}
combinator.  The latter lives in the library theory
\isa{While_Combinator}, which is not part of \isa{Main} but needs to
be included explicitly among the ancestor theories.

Constant \isa{while} is of type \isa{{\isacharparenleft}{\isacharprime}a\ {\isasymRightarrow}\ bool{\isacharparenright}\ {\isasymRightarrow}\ {\isacharparenleft}{\isacharprime}a\ {\isasymRightarrow}\ {\isacharprime}a{\isacharparenright}\ {\isasymRightarrow}\ {\isacharprime}a}
and satisfies the recursion equation \begin{isabelle}%
\ \ \ \ \ while\ b\ c\ s\ {\isacharequal}\ {\isacharparenleft}if\ b\ s\ then\ while\ b\ c\ {\isacharparenleft}c\ s{\isacharparenright}\ else\ s{\isacharparenright}%
\end{isabelle}
That is, \isa{while\ b\ c\ s} is equivalent to the imperative program
\begin{verbatim}
     x := s; while b(x) do x := c(x); return x
\end{verbatim}
In general, \isa{s} will be a tuple (better still: a record). As an example
consider the tail recursive variant of function \isa{find} above:%
\end{isamarkuptext}%
\isacommand{constdefs}\ find{\isadigit{2}}\ {\isacharcolon}{\isacharcolon}\ {\isachardoublequote}{\isacharparenleft}{\isacharprime}a\ {\isasymRightarrow}\ {\isacharprime}a{\isacharparenright}\ {\isasymRightarrow}\ {\isacharprime}a\ {\isasymRightarrow}\ {\isacharprime}a{\isachardoublequote}\isanewline
\ \ {\isachardoublequote}find{\isadigit{2}}\ f\ x\ {\isasymequiv}\isanewline
\ \ \ fst{\isacharparenleft}while\ {\isacharparenleft}{\isasymlambda}{\isacharparenleft}x{\isacharcomma}x{\isacharprime}{\isacharparenright}{\isachardot}\ x{\isacharprime}\ {\isasymnoteq}\ x{\isacharparenright}\ {\isacharparenleft}{\isasymlambda}{\isacharparenleft}x{\isacharcomma}x{\isacharprime}{\isacharparenright}{\isachardot}\ {\isacharparenleft}x{\isacharprime}{\isacharcomma}f\ x{\isacharprime}{\isacharparenright}{\isacharparenright}\ {\isacharparenleft}x{\isacharcomma}f\ x{\isacharparenright}{\isacharparenright}{\isachardoublequote}%
\begin{isamarkuptext}%
\noindent
The loop operates on two ``local variables'' \isa{x} and \isa{x{\isacharprime}}
containing the ``current'' and the ``next'' value of function \isa{f}.
They are initalized with the global \isa{x} and \isa{f\ x}. At the
end \isa{fst} selects the local \isa{x}.

This looks like we can define at least tail recursive functions
without bothering about termination after all. But there is no free
lunch: when proving properties of functions defined by \isa{while},
termination rears its ugly head again. Here is
\isa{while{\isacharunderscore}rule}, the well known proof rule for total
correctness of loops expressed with \isa{while}:
\begin{isabelle}%
\ \ \ \ \ {\isasymlbrakk}P\ s{\isacharsemicolon}\ {\isasymAnd}s{\isachardot}\ {\isasymlbrakk}P\ s{\isacharsemicolon}\ b\ s{\isasymrbrakk}\ {\isasymLongrightarrow}\ P\ {\isacharparenleft}c\ s{\isacharparenright}{\isacharsemicolon}\isanewline
\ \ \ \ \ \ \ \ {\isasymAnd}s{\isachardot}\ {\isasymlbrakk}P\ s{\isacharsemicolon}\ {\isasymnot}\ b\ s{\isasymrbrakk}\ {\isasymLongrightarrow}\ Q\ s{\isacharsemicolon}\ wf\ r{\isacharsemicolon}\isanewline
\ \ \ \ \ \ \ \ {\isasymAnd}s{\isachardot}\ {\isasymlbrakk}P\ s{\isacharsemicolon}\ b\ s{\isasymrbrakk}\ {\isasymLongrightarrow}\ {\isacharparenleft}c\ s{\isacharcomma}\ s{\isacharparenright}\ {\isasymin}\ r{\isasymrbrakk}\isanewline
\ \ \ \ \ {\isasymLongrightarrow}\ Q\ {\isacharparenleft}while\ b\ c\ s{\isacharparenright}%
\end{isabelle} \isa{P} needs to be
true of the initial state \isa{s} and invariant under \isa{c}
(premises 1 and 2).The post-condition \isa{Q} must become true when
leaving the loop (premise 3). And each loop iteration must descend
along a well-founded relation \isa{r} (premises 4 and 5).

Let us now prove that \isa{find{\isadigit{2}}} does indeed find a fixed point. Instead
of induction we apply the above while rule, suitably instantiated.
Only the final premise of \isa{while{\isacharunderscore}rule} is left unproved
by \isa{auto} but falls to \isa{simp}:%
\end{isamarkuptext}%
\isacommand{lemma}\ lem{\isacharcolon}\ {\isachardoublequote}{\isasymlbrakk}\ wf{\isacharparenleft}step{\isadigit{1}}\ f{\isacharparenright}{\isacharsemicolon}\ x{\isacharprime}\ {\isacharequal}\ f\ x\ {\isasymrbrakk}\ {\isasymLongrightarrow}\ {\isasymexists}y\ y{\isacharprime}{\isachardot}\isanewline
\ \ \ while\ {\isacharparenleft}{\isasymlambda}{\isacharparenleft}x{\isacharcomma}x{\isacharprime}{\isacharparenright}{\isachardot}\ x{\isacharprime}\ {\isasymnoteq}\ x{\isacharparenright}\ {\isacharparenleft}{\isasymlambda}{\isacharparenleft}x{\isacharcomma}x{\isacharprime}{\isacharparenright}{\isachardot}\ {\isacharparenleft}x{\isacharprime}{\isacharcomma}f\ x{\isacharprime}{\isacharparenright}{\isacharparenright}\ {\isacharparenleft}x{\isacharcomma}x{\isacharprime}{\isacharparenright}\ {\isacharequal}\ {\isacharparenleft}y{\isacharcomma}y{\isacharprime}{\isacharparenright}\ {\isasymand}\isanewline
\ \ \ y{\isacharprime}\ {\isacharequal}\ y\ {\isasymand}\ f\ y\ {\isacharequal}\ y{\isachardoublequote}\isanewline
\isacommand{apply}{\isacharparenleft}rule{\isacharunderscore}tac\ P\ {\isacharequal}\ {\isachardoublequote}{\isasymlambda}{\isacharparenleft}x{\isacharcomma}x{\isacharprime}{\isacharparenright}{\isachardot}\ x{\isacharprime}\ {\isacharequal}\ f\ x{\isachardoublequote}\ \isakeyword{and}\isanewline
\ \ \ \ \ \ \ \ \ \ \ \ \ \ \ r\ {\isacharequal}\ {\isachardoublequote}inv{\isacharunderscore}image\ {\isacharparenleft}step{\isadigit{1}}\ f{\isacharparenright}\ fst{\isachardoublequote}\ \isakeyword{in}\ while{\isacharunderscore}rule{\isacharparenright}\isanewline
\isacommand{apply}\ auto\isanewline
\isacommand{apply}{\isacharparenleft}simp\ add{\isacharcolon}inv{\isacharunderscore}image{\isacharunderscore}def\ step{\isadigit{1}}{\isacharunderscore}def{\isacharparenright}\isanewline
\isacommand{done}%
\begin{isamarkuptext}%
The theorem itself is a simple consequence of this lemma:%
\end{isamarkuptext}%
\isacommand{theorem}\ {\isachardoublequote}wf{\isacharparenleft}step{\isadigit{1}}\ f{\isacharparenright}\ {\isasymLongrightarrow}\ f{\isacharparenleft}find{\isadigit{2}}\ f\ x{\isacharparenright}\ {\isacharequal}\ find{\isadigit{2}}\ f\ x{\isachardoublequote}\isanewline
\isacommand{apply}{\isacharparenleft}drule{\isacharunderscore}tac\ x\ {\isacharequal}\ x\ \isakeyword{in}\ lem{\isacharparenright}\isanewline
\isacommand{apply}{\isacharparenleft}auto\ simp\ add{\isacharcolon}find{\isadigit{2}}{\isacharunderscore}def{\isacharparenright}\isanewline
\isacommand{done}%
\begin{isamarkuptext}%
Let us conclude this section on partial functions by a
discussion of the merits of the \isa{while} combinator. We have
already seen that the advantage (if it is one) of not having to
provide a termintion argument when defining a function via \isa{while} merely puts off the evil hour. On top of that, tail recursive
functions tend to be more complicated to reason about. So why use
\isa{while} at all? The only reason is executability: the recursion
equation for \isa{while} is a directly executable functional
program. This is in stark contrast to guarded recursion as introduced
above which requires an explicit test \isa{x\ {\isasymin}\ dom\ f} in the
function body.  Unless \isa{dom} is trivial, this leads to a
definition which is either not at all executable or prohibitively
expensive. Thus, if you are aiming for an efficiently executable definition
of a partial function, you are likely to need \isa{while}.%
\end{isamarkuptext}%
\end{isabellebody}%
%%% Local Variables:
%%% mode: latex
%%% TeX-master: "root"
%%% End:

%
%\index{recdef@\isacommand {recdef} (command)|)}
