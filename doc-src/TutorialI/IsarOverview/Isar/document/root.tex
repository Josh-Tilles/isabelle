\documentclass[11pt,a4paper]{article}
\usepackage{isabelle,isabellesym,pdfsetup}

%for best-style documents ...
\urlstyle{rm}
%\isabellestyle{it}

\newtheorem{Exercise}{Exercise}[section]
\newenvironment{exercise}{\begin{Exercise}\rm}{\end{Exercise}}

\begin{document}

\title{A Compact Introduction to Structured Proofs in Isar/HOL}
\author{Tobias Nipkow\\Institut f{\"u}r Informatik, TU M{\"u}nchen\\
 {\small\url{http://www.in.tum.de/~nipkow/}}}
\date{}
\maketitle

\begin{abstract}
  Isar is an extension of the theorem prover Isabelle with a language
  for writing human-readable structured proofs. This paper is an
  introduction to the basic constructs of this language. It is aimed
  at potential users of Isar but also discusses the design rationals
  behind the language and its constructs.
\end{abstract}

\section{Introduction}

Isar is an extension of Isabelle with structured proofs in a stylized
language of mathematics. These proofs are readable for both a human
and a machine.  This document is a very compact introduction to
structured proofs in Isar/HOL, an extension of
Isabelle/HOL~\cite{LNCS2283}. We intentionally do not present the full
language but concentrate on the essentials. Neither do we give a
formal semantics of Isar. Instead we introduce Isar by example. Again
this is intentional: we believe that the language ``speaks for
itself'' in the same way that traditional mathamtical proofs do, which
are also introduced by example rather than by teaching students logic
first. A detailed exposition of Isar can be found in Markus Wenzel's
PhD thesis~\cite{Wenzel-PhD} and the Isar reference
manual~\cite{Isar-Ref-Man}.

\subsection{Background}

Interactive theorem proving has been dominated by a model of proof
that goes back to the LCF system~\cite{LCF}: a proof is a more or less
structured sequence of commands that manipulate an implicit proof
state. Thus the proof proof text is only suitable for the machine; for
a human, the proof only comes alive when he can see the state changes
caused by the stepwise execution of the commands. Such a proof is like
an uncommented assembly language program. We call them
\emph{tactic-style} proofs because LCF proof-commands were called
tactics.

A radically different approach was taken by the Mizar
system~\cite{Mizar} where proofs are written a stylized language akin
to that used in ordinary mathematics texts. The most important
argument in favour of a mathematics-like proof language is
communication: as soon as not just the theorem but also the proof
becomes an object of interest, it should be readable as it is.  From a
system development point of view there is a second important argument
against tactic-style proofs: they are much harder to maintain when the
system is modified. The reason is as follows. Since it is usually
quite unclear what exactly is proved at some point in the middle of a
command sequence, updating a failed proof often requires executing the
proof up to the point of failure using the old version of the system.
In contrast, mathematics-like proofs contain enough information
to tell you what is proved at any point.

For these reasons the Isabelle system, originally firmly in the
LCF-tradition, was extended with a language for writing structured
proofs in a mathematics-like style. As the name already indicates,
Isar was certainly inspired by Mizar. However, there are very many
differences. For a start, Isar is generic: only a few of the language
constructs described below are specific to HOL; many are generic and
thus available for any logic defined in Isabelle, e.g.\ ZF.
Furthermore, we have Isabelle's powerful automatic proof procedures
(e.g.\ simplification and a tableau-style prover) at our disposal.
A closer comparison of Isar and Mizar can be found
elsewhere~\cite{Isar-Mizar}.

Finally, a word of warning for potential writers of structured Isar
proofs.  It has always been easier to write obscure rather than
readable texts. Similarly, tactic-based proofs are often (though by no
means always!) easier to write than readable ones: structure does not
emerge automatically but needs to be understood and imposed. If the
precise structure of the proof is not clear from the beginning, it can
be useful to start in a tactic-based style for exploratory purposes
until one has found a proof which can then be converted into a
structured text in a second step. Top down conversion is possible
because Isar allows tactic-based proofs as components of structured
ones.

\subsection{An overview of the Isar syntax}

We begin by looking at a simplified grammar for Isar proofs
where parentheses are used for grouping and $^?$ indicates an optional item:
\begin{center}
\begin{tabular}{lrl}
\emph{proof} & ::= & \isakeyword{proof} \emph{method}$^?$
                     \emph{statement}*
                     \isakeyword{qed} \\
                 &$\mid$& \isakeyword{by} \emph{method}\\[1ex]
\emph{statement} &::= & \isakeyword{fix} \emph{variables} \\
             &$\mid$& \isakeyword{assume} \emph{propositions} \\
             &$\mid$& (\isakeyword{from} \emph{facts})$^?$ 
                    (\isakeyword{show} $\mid$ \isakeyword{have})
                      \emph{propositions} \emph{proof} \\[1ex]
\emph{proposition} &::=& (\emph{label}{\bf:})$^?$ \emph{string} \\[1ex]
\emph{fact} &::=& \emph{label}
\end{tabular}
\end{center}
A proof can be either compound (\isakeyword{proof} --
\isakeyword{qed}) or atomic (\isakeyword{by}). A \emph{method} is a
proof method offered by the underlying theorem prover, for example
\isa{rule} or \isa{simp} in Isabelle.  Thus this grammar is
generic both w.r.t.\ the logic and the theorem prover.

This is a typical proof skeleton:
\begin{center}
\begin{tabular}{@{}ll}
\isakeyword{proof}\\
\hspace*{3ex}\isakeyword{assume} \isa{"}\emph{the-assm}\isa{"}\\
\hspace*{3ex}\isakeyword{have} \isa{"}\dots\isa{"} & --- intermediate result\\
\hspace*{3ex}\vdots\\
\hspace*{3ex}\isakeyword{have} \isa{"}\dots\isa{"} & --- intermediate result\\
\hspace*{3ex}\isakeyword{show} \isa{"}\emph{the-concl}\isa{"}\\
\isakeyword{qed}
\end{tabular}
\end{center}
It proves \emph{the-assm}~$\Longrightarrow$~\emph{the-concl}. Text starting with
``---'' is a comment. The intermediate \isakeyword{have}s are only
there to bridge the gap between the assumption and the conclusion and
do not contribute to the theorem being proved. In contrast,
\isakeyword{show} establishes the conclusion of the theorem.

As a final bit of syntax note that \emph{propositions} (following
\isakeyword{assume} etc) may but need not be separated by
\isakeyword{and}, whose purpose is to structure them into possibly
named blocks. For example in
\begin{center}
\isakeyword{assume} \isa{A:} $A_1$ $A_2$ \isakeyword{and} \isa{B:} $A_3$
\isakeyword{and} $A_4$
\end{center}
label \isa{A} refers to the list of propositions $A_1$ $A_2$ and
label \isa{B} to $A_3$.

\subsection{Overview of the paper}

The rest of the paper is divided into two parts.
Section~\ref{sec:Logic} introduces proofs in pure logic based on
natural deduction. Section~\ref{sec:Induct} is dedicated to induction,
the key reasoning principle for computer science applications.

There are at least two further areas where Isar provides specific
support, but which we do not document here: reasoning by chains of
(in)equations is described elsewhere~\cite{BauerW-TPHOL}, whereas
reasoning about axiomatically defined structures by means of so called
``locales'' \cite{BallarinPW-TPHOL} is only described in a very early
form and has evolved much since then.

Do not be mislead by the simplicity of the formulae being proved,
especially in the beginning. Isar has been used very successfully in
large applications, for example the formalization of Java, in
particular the verification of the bytecode verifier~\cite{KleinN-TCS}.

%
\begin{isabellebody}%
\def\isabellecontext{Logic}%
%
\isadelimtheory
%
\endisadelimtheory
%
\isatagtheory
\isacommand{theory}\isamarkupfalse%
\ Logic\isanewline
\isakeyword{imports}\ Base\isanewline
\isakeyword{begin}%
\endisatagtheory
{\isafoldtheory}%
%
\isadelimtheory
%
\endisadelimtheory
%
\isamarkupchapter{Primitive logic \label{ch:logic}%
}
\isamarkuptrue%
%
\begin{isamarkuptext}%
The logical foundations of Isabelle/Isar are that of the Pure logic,
  which has been introduced as a Natural Deduction framework in
  \cite{paulson700}.  This is essentially the same logic as ``\isa{{\isasymlambda}HOL}'' in the more abstract setting of Pure Type Systems (PTS)
  \cite{Barendregt-Geuvers:2001}, although there are some key
  differences in the specific treatment of simple types in
  Isabelle/Pure.

  Following type-theoretic parlance, the Pure logic consists of three
  levels of \isa{{\isasymlambda}}-calculus with corresponding arrows, \isa{{\isasymRightarrow}} for syntactic function space (terms depending on terms), \isa{{\isasymAnd}} for universal quantification (proofs depending on terms), and
  \isa{{\isasymLongrightarrow}} for implication (proofs depending on proofs).

  Derivations are relative to a logical theory, which declares type
  constructors, constants, and axioms.  Theory declarations support
  schematic polymorphism, which is strictly speaking outside the
  logic.\footnote{This is the deeper logical reason, why the theory
  context \isa{{\isasymTheta}} is separate from the proof context \isa{{\isasymGamma}}
  of the core calculus.}%
\end{isamarkuptext}%
\isamarkuptrue%
%
\isamarkupsection{Types \label{sec:types}%
}
\isamarkuptrue%
%
\begin{isamarkuptext}%
The language of types is an uninterpreted order-sorted first-order
  algebra; types are qualified by ordered type classes.

  \medskip A \emph{type class} is an abstract syntactic entity
  declared in the theory context.  The \emph{subclass relation} \isa{c\isactrlisub {\isadigit{1}}\ {\isasymsubseteq}\ c\isactrlisub {\isadigit{2}}} is specified by stating an acyclic
  generating relation; the transitive closure is maintained
  internally.  The resulting relation is an ordering: reflexive,
  transitive, and antisymmetric.

  A \emph{sort} is a list of type classes written as \isa{s\ {\isacharequal}\ {\isacharbraceleft}c\isactrlisub {\isadigit{1}}{\isacharcomma}\ {\isasymdots}{\isacharcomma}\ c\isactrlisub m{\isacharbraceright}}, which represents symbolic
  intersection.  Notationally, the curly braces are omitted for
  singleton intersections, i.e.\ any class \isa{c} may be read as
  a sort \isa{{\isacharbraceleft}c{\isacharbraceright}}.  The ordering on type classes is extended to
  sorts according to the meaning of intersections: \isa{{\isacharbraceleft}c\isactrlisub {\isadigit{1}}{\isacharcomma}\ {\isasymdots}\ c\isactrlisub m{\isacharbraceright}\ {\isasymsubseteq}\ {\isacharbraceleft}d\isactrlisub {\isadigit{1}}{\isacharcomma}\ {\isasymdots}{\isacharcomma}\ d\isactrlisub n{\isacharbraceright}} iff
  \isa{{\isasymforall}j{\isachardot}\ {\isasymexists}i{\isachardot}\ c\isactrlisub i\ {\isasymsubseteq}\ d\isactrlisub j}.  The empty intersection
  \isa{{\isacharbraceleft}{\isacharbraceright}} refers to the universal sort, which is the largest
  element wrt.\ the sort order.  The intersections of all (finitely
  many) classes declared in the current theory are the minimal
  elements wrt.\ the sort order.

  \medskip A \emph{fixed type variable} is a pair of a basic name
  (starting with a \isa{{\isacharprime}} character) and a sort constraint, e.g.\
  \isa{{\isacharparenleft}{\isacharprime}a{\isacharcomma}\ s{\isacharparenright}} which is usually printed as \isa{{\isasymalpha}\isactrlisub s}.
  A \emph{schematic type variable} is a pair of an indexname and a
  sort constraint, e.g.\ \isa{{\isacharparenleft}{\isacharparenleft}{\isacharprime}a{\isacharcomma}\ {\isadigit{0}}{\isacharparenright}{\isacharcomma}\ s{\isacharparenright}} which is usually
  printed as \isa{{\isacharquery}{\isasymalpha}\isactrlisub s}.

  Note that \emph{all} syntactic components contribute to the identity
  of type variables, including the sort constraint.  The core logic
  handles type variables with the same name but different sorts as
  different, although some outer layers of the system make it hard to
  produce anything like this.

  A \emph{type constructor} \isa{{\isasymkappa}} is a \isa{k}-ary operator
  on types declared in the theory.  Type constructor application is
  written postfix as \isa{{\isacharparenleft}{\isasymalpha}\isactrlisub {\isadigit{1}}{\isacharcomma}\ {\isasymdots}{\isacharcomma}\ {\isasymalpha}\isactrlisub k{\isacharparenright}{\isasymkappa}}.  For
  \isa{k\ {\isacharequal}\ {\isadigit{0}}} the argument tuple is omitted, e.g.\ \isa{prop}
  instead of \isa{{\isacharparenleft}{\isacharparenright}prop}.  For \isa{k\ {\isacharequal}\ {\isadigit{1}}} the parentheses
  are omitted, e.g.\ \isa{{\isasymalpha}\ list} instead of \isa{{\isacharparenleft}{\isasymalpha}{\isacharparenright}list}.
  Further notation is provided for specific constructors, notably the
  right-associative infix \isa{{\isasymalpha}\ {\isasymRightarrow}\ {\isasymbeta}} instead of \isa{{\isacharparenleft}{\isasymalpha}{\isacharcomma}\ {\isasymbeta}{\isacharparenright}fun}.
  
  A \emph{type} is defined inductively over type variables and type
  constructors as follows: \isa{{\isasymtau}\ {\isacharequal}\ {\isasymalpha}\isactrlisub s\ {\isacharbar}\ {\isacharquery}{\isasymalpha}\isactrlisub s\ {\isacharbar}\ {\isacharparenleft}{\isasymtau}\isactrlsub {\isadigit{1}}{\isacharcomma}\ {\isasymdots}{\isacharcomma}\ {\isasymtau}\isactrlsub k{\isacharparenright}{\isasymkappa}}.

  A \emph{type abbreviation} is a syntactic definition \isa{{\isacharparenleft}\isactrlvec {\isasymalpha}{\isacharparenright}{\isasymkappa}\ {\isacharequal}\ {\isasymtau}} of an arbitrary type expression \isa{{\isasymtau}} over
  variables \isa{\isactrlvec {\isasymalpha}}.  Type abbreviations appear as type
  constructors in the syntax, but are expanded before entering the
  logical core.

  A \emph{type arity} declares the image behavior of a type
  constructor wrt.\ the algebra of sorts: \isa{{\isasymkappa}\ {\isacharcolon}{\isacharcolon}\ {\isacharparenleft}s\isactrlisub {\isadigit{1}}{\isacharcomma}\ {\isasymdots}{\isacharcomma}\ s\isactrlisub k{\isacharparenright}s} means that \isa{{\isacharparenleft}{\isasymtau}\isactrlisub {\isadigit{1}}{\isacharcomma}\ {\isasymdots}{\isacharcomma}\ {\isasymtau}\isactrlisub k{\isacharparenright}{\isasymkappa}} is
  of sort \isa{s} if every argument type \isa{{\isasymtau}\isactrlisub i} is
  of sort \isa{s\isactrlisub i}.  Arity declarations are implicitly
  completed, i.e.\ \isa{{\isasymkappa}\ {\isacharcolon}{\isacharcolon}\ {\isacharparenleft}\isactrlvec s{\isacharparenright}c} entails \isa{{\isasymkappa}\ {\isacharcolon}{\isacharcolon}\ {\isacharparenleft}\isactrlvec s{\isacharparenright}c{\isacharprime}} for any \isa{c{\isacharprime}\ {\isasymsupseteq}\ c}.

  \medskip The sort algebra is always maintained as \emph{coregular},
  which means that type arities are consistent with the subclass
  relation: for any type constructor \isa{{\isasymkappa}}, and classes \isa{c\isactrlisub {\isadigit{1}}\ {\isasymsubseteq}\ c\isactrlisub {\isadigit{2}}}, and arities \isa{{\isasymkappa}\ {\isacharcolon}{\isacharcolon}\ {\isacharparenleft}\isactrlvec s\isactrlisub {\isadigit{1}}{\isacharparenright}c\isactrlisub {\isadigit{1}}} and \isa{{\isasymkappa}\ {\isacharcolon}{\isacharcolon}\ {\isacharparenleft}\isactrlvec s\isactrlisub {\isadigit{2}}{\isacharparenright}c\isactrlisub {\isadigit{2}}} holds \isa{\isactrlvec s\isactrlisub {\isadigit{1}}\ {\isasymsubseteq}\ \isactrlvec s\isactrlisub {\isadigit{2}}} component-wise.

  The key property of a coregular order-sorted algebra is that sort
  constraints can be solved in a most general fashion: for each type
  constructor \isa{{\isasymkappa}} and sort \isa{s} there is a most general
  vector of argument sorts \isa{{\isacharparenleft}s\isactrlisub {\isadigit{1}}{\isacharcomma}\ {\isasymdots}{\isacharcomma}\ s\isactrlisub k{\isacharparenright}} such
  that a type scheme \isa{{\isacharparenleft}{\isasymalpha}\isactrlbsub s\isactrlisub {\isadigit{1}}\isactrlesub {\isacharcomma}\ {\isasymdots}{\isacharcomma}\ {\isasymalpha}\isactrlbsub s\isactrlisub k\isactrlesub {\isacharparenright}{\isasymkappa}} is of sort \isa{s}.
  Consequently, type unification has most general solutions (modulo
  equivalence of sorts), so type-inference produces primary types as
  expected \cite{nipkow-prehofer}.%
\end{isamarkuptext}%
\isamarkuptrue%
%
\isadelimmlref
%
\endisadelimmlref
%
\isatagmlref
%
\begin{isamarkuptext}%
\begin{mldecls}
  \indexdef{}{ML type}{class}\verb|type class| \\
  \indexdef{}{ML type}{sort}\verb|type sort| \\
  \indexdef{}{ML type}{arity}\verb|type arity| \\
  \indexdef{}{ML type}{typ}\verb|type typ| \\
  \indexdef{}{ML}{map\_atyps}\verb|map_atyps: (typ -> typ) -> typ -> typ| \\
  \indexdef{}{ML}{fold\_atyps}\verb|fold_atyps: (typ -> 'a -> 'a) -> typ -> 'a -> 'a| \\
  \end{mldecls}
  \begin{mldecls}
  \indexdef{}{ML}{Sign.subsort}\verb|Sign.subsort: theory -> sort * sort -> bool| \\
  \indexdef{}{ML}{Sign.of\_sort}\verb|Sign.of_sort: theory -> typ * sort -> bool| \\
  \indexdef{}{ML}{Sign.add\_types}\verb|Sign.add_types: (binding * int * mixfix) list -> theory -> theory| \\
  \indexdef{}{ML}{Sign.add\_tyabbrs\_i}\verb|Sign.add_tyabbrs_i: |\isasep\isanewline%
\verb|  (binding * string list * typ * mixfix) list -> theory -> theory| \\
  \indexdef{}{ML}{Sign.primitive\_class}\verb|Sign.primitive_class: binding * class list -> theory -> theory| \\
  \indexdef{}{ML}{Sign.primitive\_classrel}\verb|Sign.primitive_classrel: class * class -> theory -> theory| \\
  \indexdef{}{ML}{Sign.primitive\_arity}\verb|Sign.primitive_arity: arity -> theory -> theory| \\
  \end{mldecls}

  \begin{description}

  \item \verb|class| represents type classes; this is an alias for
  \verb|string|.

  \item \verb|sort| represents sorts; this is an alias for
  \verb|class list|.

  \item \verb|arity| represents type arities; this is an alias for
  triples of the form \isa{{\isacharparenleft}{\isasymkappa}{\isacharcomma}\ \isactrlvec s{\isacharcomma}\ s{\isacharparenright}} for \isa{{\isasymkappa}\ {\isacharcolon}{\isacharcolon}\ {\isacharparenleft}\isactrlvec s{\isacharparenright}s} described above.

  \item \verb|typ| represents types; this is a datatype with
  constructors \verb|TFree|, \verb|TVar|, \verb|Type|.

  \item \verb|map_atyps|~\isa{f\ {\isasymtau}} applies the mapping \isa{f}
  to all atomic types (\verb|TFree|, \verb|TVar|) occurring in \isa{{\isasymtau}}.

  \item \verb|fold_atyps|~\isa{f\ {\isasymtau}} iterates the operation \isa{f} over all occurrences of atomic types (\verb|TFree|, \verb|TVar|)
  in \isa{{\isasymtau}}; the type structure is traversed from left to right.

  \item \verb|Sign.subsort|~\isa{thy\ {\isacharparenleft}s\isactrlisub {\isadigit{1}}{\isacharcomma}\ s\isactrlisub {\isadigit{2}}{\isacharparenright}}
  tests the subsort relation \isa{s\isactrlisub {\isadigit{1}}\ {\isasymsubseteq}\ s\isactrlisub {\isadigit{2}}}.

  \item \verb|Sign.of_sort|~\isa{thy\ {\isacharparenleft}{\isasymtau}{\isacharcomma}\ s{\isacharparenright}} tests whether type
  \isa{{\isasymtau}} is of sort \isa{s}.

  \item \verb|Sign.add_types|~\isa{{\isacharbrackleft}{\isacharparenleft}{\isasymkappa}{\isacharcomma}\ k{\isacharcomma}\ mx{\isacharparenright}{\isacharcomma}\ {\isasymdots}{\isacharbrackright}} declares a new
  type constructors \isa{{\isasymkappa}} with \isa{k} arguments and
  optional mixfix syntax.

  \item \verb|Sign.add_tyabbrs_i|~\isa{{\isacharbrackleft}{\isacharparenleft}{\isasymkappa}{\isacharcomma}\ \isactrlvec {\isasymalpha}{\isacharcomma}\ {\isasymtau}{\isacharcomma}\ mx{\isacharparenright}{\isacharcomma}\ {\isasymdots}{\isacharbrackright}}
  defines a new type abbreviation \isa{{\isacharparenleft}\isactrlvec {\isasymalpha}{\isacharparenright}{\isasymkappa}\ {\isacharequal}\ {\isasymtau}} with
  optional mixfix syntax.

  \item \verb|Sign.primitive_class|~\isa{{\isacharparenleft}c{\isacharcomma}\ {\isacharbrackleft}c\isactrlisub {\isadigit{1}}{\isacharcomma}\ {\isasymdots}{\isacharcomma}\ c\isactrlisub n{\isacharbrackright}{\isacharparenright}} declares a new class \isa{c}, together with class
  relations \isa{c\ {\isasymsubseteq}\ c\isactrlisub i}, for \isa{i\ {\isacharequal}\ {\isadigit{1}}{\isacharcomma}\ {\isasymdots}{\isacharcomma}\ n}.

  \item \verb|Sign.primitive_classrel|~\isa{{\isacharparenleft}c\isactrlisub {\isadigit{1}}{\isacharcomma}\ c\isactrlisub {\isadigit{2}}{\isacharparenright}} declares the class relation \isa{c\isactrlisub {\isadigit{1}}\ {\isasymsubseteq}\ c\isactrlisub {\isadigit{2}}}.

  \item \verb|Sign.primitive_arity|~\isa{{\isacharparenleft}{\isasymkappa}{\isacharcomma}\ \isactrlvec s{\isacharcomma}\ s{\isacharparenright}} declares
  the arity \isa{{\isasymkappa}\ {\isacharcolon}{\isacharcolon}\ {\isacharparenleft}\isactrlvec s{\isacharparenright}s}.

  \end{description}%
\end{isamarkuptext}%
\isamarkuptrue%
%
\endisatagmlref
{\isafoldmlref}%
%
\isadelimmlref
%
\endisadelimmlref
%
\isamarkupsection{Terms \label{sec:terms}%
}
\isamarkuptrue%
%
\begin{isamarkuptext}%
The language of terms is that of simply-typed \isa{{\isasymlambda}}-calculus
  with de-Bruijn indices for bound variables (cf.\ \cite{debruijn72}
  or \cite{paulson-ml2}), with the types being determined by the
  corresponding binders.  In contrast, free variables and constants
  are have an explicit name and type in each occurrence.

  \medskip A \emph{bound variable} is a natural number \isa{b},
  which accounts for the number of intermediate binders between the
  variable occurrence in the body and its binding position.  For
  example, the de-Bruijn term \isa{{\isasymlambda}\isactrlbsub nat\isactrlesub {\isachardot}\ {\isasymlambda}\isactrlbsub nat\isactrlesub {\isachardot}\ {\isadigit{1}}\ {\isacharplus}\ {\isadigit{0}}} would
  correspond to \isa{{\isasymlambda}x\isactrlbsub nat\isactrlesub {\isachardot}\ {\isasymlambda}y\isactrlbsub nat\isactrlesub {\isachardot}\ x\ {\isacharplus}\ y} in a named
  representation.  Note that a bound variable may be represented by
  different de-Bruijn indices at different occurrences, depending on
  the nesting of abstractions.

  A \emph{loose variable} is a bound variable that is outside the
  scope of local binders.  The types (and names) for loose variables
  can be managed as a separate context, that is maintained as a stack
  of hypothetical binders.  The core logic operates on closed terms,
  without any loose variables.

  A \emph{fixed variable} is a pair of a basic name and a type, e.g.\
  \isa{{\isacharparenleft}x{\isacharcomma}\ {\isasymtau}{\isacharparenright}} which is usually printed \isa{x\isactrlisub {\isasymtau}}.  A
  \emph{schematic variable} is a pair of an indexname and a type,
  e.g.\ \isa{{\isacharparenleft}{\isacharparenleft}x{\isacharcomma}\ {\isadigit{0}}{\isacharparenright}{\isacharcomma}\ {\isasymtau}{\isacharparenright}} which is usually printed as \isa{{\isacharquery}x\isactrlisub {\isasymtau}}.

  \medskip A \emph{constant} is a pair of a basic name and a type,
  e.g.\ \isa{{\isacharparenleft}c{\isacharcomma}\ {\isasymtau}{\isacharparenright}} which is usually printed as \isa{c\isactrlisub {\isasymtau}}.  Constants are declared in the context as polymorphic
  families \isa{c\ {\isacharcolon}{\isacharcolon}\ {\isasymsigma}}, meaning that all substitution instances
  \isa{c\isactrlisub {\isasymtau}} for \isa{{\isasymtau}\ {\isacharequal}\ {\isasymsigma}{\isasymvartheta}} are valid.

  The vector of \emph{type arguments} of constant \isa{c\isactrlisub {\isasymtau}}
  wrt.\ the declaration \isa{c\ {\isacharcolon}{\isacharcolon}\ {\isasymsigma}} is defined as the codomain of
  the matcher \isa{{\isasymvartheta}\ {\isacharequal}\ {\isacharbraceleft}{\isacharquery}{\isasymalpha}\isactrlisub {\isadigit{1}}\ {\isasymmapsto}\ {\isasymtau}\isactrlisub {\isadigit{1}}{\isacharcomma}\ {\isasymdots}{\isacharcomma}\ {\isacharquery}{\isasymalpha}\isactrlisub n\ {\isasymmapsto}\ {\isasymtau}\isactrlisub n{\isacharbraceright}} presented in canonical order \isa{{\isacharparenleft}{\isasymtau}\isactrlisub {\isadigit{1}}{\isacharcomma}\ {\isasymdots}{\isacharcomma}\ {\isasymtau}\isactrlisub n{\isacharparenright}}.  Within a given theory context,
  there is a one-to-one correspondence between any constant \isa{c\isactrlisub {\isasymtau}} and the application \isa{c{\isacharparenleft}{\isasymtau}\isactrlisub {\isadigit{1}}{\isacharcomma}\ {\isasymdots}{\isacharcomma}\ {\isasymtau}\isactrlisub n{\isacharparenright}} of its type arguments.  For example, with \isa{plus\ {\isacharcolon}{\isacharcolon}\ {\isasymalpha}\ {\isasymRightarrow}\ {\isasymalpha}\ {\isasymRightarrow}\ {\isasymalpha}}, the instance \isa{plus\isactrlbsub nat\ {\isasymRightarrow}\ nat\ {\isasymRightarrow}\ nat\isactrlesub } corresponds to \isa{plus{\isacharparenleft}nat{\isacharparenright}}.

  Constant declarations \isa{c\ {\isacharcolon}{\isacharcolon}\ {\isasymsigma}} may contain sort constraints
  for type variables in \isa{{\isasymsigma}}.  These are observed by
  type-inference as expected, but \emph{ignored} by the core logic.
  This means the primitive logic is able to reason with instances of
  polymorphic constants that the user-level type-checker would reject
  due to violation of type class restrictions.

  \medskip An \emph{atomic} term is either a variable or constant.  A
  \emph{term} is defined inductively over atomic terms, with
  abstraction and application as follows: \isa{t\ {\isacharequal}\ b\ {\isacharbar}\ x\isactrlisub {\isasymtau}\ {\isacharbar}\ {\isacharquery}x\isactrlisub {\isasymtau}\ {\isacharbar}\ c\isactrlisub {\isasymtau}\ {\isacharbar}\ {\isasymlambda}\isactrlisub {\isasymtau}{\isachardot}\ t\ {\isacharbar}\ t\isactrlisub {\isadigit{1}}\ t\isactrlisub {\isadigit{2}}}.
  Parsing and printing takes care of converting between an external
  representation with named bound variables.  Subsequently, we shall
  use the latter notation instead of internal de-Bruijn
  representation.

  The inductive relation \isa{t\ {\isacharcolon}{\isacharcolon}\ {\isasymtau}} assigns a (unique) type to a
  term according to the structure of atomic terms, abstractions, and
  applicatins:
  \[
  \infer{\isa{a\isactrlisub {\isasymtau}\ {\isacharcolon}{\isacharcolon}\ {\isasymtau}}}{}
  \qquad
  \infer{\isa{{\isacharparenleft}{\isasymlambda}x\isactrlsub {\isasymtau}{\isachardot}\ t{\isacharparenright}\ {\isacharcolon}{\isacharcolon}\ {\isasymtau}\ {\isasymRightarrow}\ {\isasymsigma}}}{\isa{t\ {\isacharcolon}{\isacharcolon}\ {\isasymsigma}}}
  \qquad
  \infer{\isa{t\ u\ {\isacharcolon}{\isacharcolon}\ {\isasymsigma}}}{\isa{t\ {\isacharcolon}{\isacharcolon}\ {\isasymtau}\ {\isasymRightarrow}\ {\isasymsigma}} & \isa{u\ {\isacharcolon}{\isacharcolon}\ {\isasymtau}}}
  \]
  A \emph{well-typed term} is a term that can be typed according to these rules.

  Typing information can be omitted: type-inference is able to
  reconstruct the most general type of a raw term, while assigning
  most general types to all of its variables and constants.
  Type-inference depends on a context of type constraints for fixed
  variables, and declarations for polymorphic constants.

  The identity of atomic terms consists both of the name and the type
  component.  This means that different variables \isa{x\isactrlbsub {\isasymtau}\isactrlisub {\isadigit{1}}\isactrlesub } and \isa{x\isactrlbsub {\isasymtau}\isactrlisub {\isadigit{2}}\isactrlesub } may become the same after type
  instantiation.  Some outer layers of the system make it hard to
  produce variables of the same name, but different types.  In
  contrast, mixed instances of polymorphic constants occur frequently.

  \medskip The \emph{hidden polymorphism} of a term \isa{t\ {\isacharcolon}{\isacharcolon}\ {\isasymsigma}}
  is the set of type variables occurring in \isa{t}, but not in
  \isa{{\isasymsigma}}.  This means that the term implicitly depends on type
  arguments that are not accounted in the result type, i.e.\ there are
  different type instances \isa{t{\isasymvartheta}\ {\isacharcolon}{\isacharcolon}\ {\isasymsigma}} and \isa{t{\isasymvartheta}{\isacharprime}\ {\isacharcolon}{\isacharcolon}\ {\isasymsigma}} with the same type.  This slightly
  pathological situation notoriously demands additional care.

  \medskip A \emph{term abbreviation} is a syntactic definition \isa{c\isactrlisub {\isasymsigma}\ {\isasymequiv}\ t} of a closed term \isa{t} of type \isa{{\isasymsigma}},
  without any hidden polymorphism.  A term abbreviation looks like a
  constant in the syntax, but is expanded before entering the logical
  core.  Abbreviations are usually reverted when printing terms, using
  \isa{t\ {\isasymrightarrow}\ c\isactrlisub {\isasymsigma}} as rules for higher-order rewriting.

  \medskip Canonical operations on \isa{{\isasymlambda}}-terms include \isa{{\isasymalpha}{\isasymbeta}{\isasymeta}}-conversion: \isa{{\isasymalpha}}-conversion refers to capture-free
  renaming of bound variables; \isa{{\isasymbeta}}-conversion contracts an
  abstraction applied to an argument term, substituting the argument
  in the body: \isa{{\isacharparenleft}{\isasymlambda}x{\isachardot}\ b{\isacharparenright}a} becomes \isa{b{\isacharbrackleft}a{\isacharslash}x{\isacharbrackright}}; \isa{{\isasymeta}}-conversion contracts vacuous application-abstraction: \isa{{\isasymlambda}x{\isachardot}\ f\ x} becomes \isa{f}, provided that the bound variable
  does not occur in \isa{f}.

  Terms are normally treated modulo \isa{{\isasymalpha}}-conversion, which is
  implicit in the de-Bruijn representation.  Names for bound variables
  in abstractions are maintained separately as (meaningless) comments,
  mostly for parsing and printing.  Full \isa{{\isasymalpha}{\isasymbeta}{\isasymeta}}-conversion is
  commonplace in various standard operations (\secref{sec:obj-rules})
  that are based on higher-order unification and matching.%
\end{isamarkuptext}%
\isamarkuptrue%
%
\isadelimmlref
%
\endisadelimmlref
%
\isatagmlref
%
\begin{isamarkuptext}%
\begin{mldecls}
  \indexdef{}{ML type}{term}\verb|type term| \\
  \indexdef{}{ML}{op aconv}\verb|op aconv: term * term -> bool| \\
  \indexdef{}{ML}{map\_types}\verb|map_types: (typ -> typ) -> term -> term| \\
  \indexdef{}{ML}{fold\_types}\verb|fold_types: (typ -> 'a -> 'a) -> term -> 'a -> 'a| \\
  \indexdef{}{ML}{map\_aterms}\verb|map_aterms: (term -> term) -> term -> term| \\
  \indexdef{}{ML}{fold\_aterms}\verb|fold_aterms: (term -> 'a -> 'a) -> term -> 'a -> 'a| \\
  \end{mldecls}
  \begin{mldecls}
  \indexdef{}{ML}{fastype\_of}\verb|fastype_of: term -> typ| \\
  \indexdef{}{ML}{lambda}\verb|lambda: term -> term -> term| \\
  \indexdef{}{ML}{betapply}\verb|betapply: term * term -> term| \\
  \indexdef{}{ML}{Sign.declare\_const}\verb|Sign.declare_const: (binding * typ) * mixfix ->|\isasep\isanewline%
\verb|  theory -> term * theory| \\
  \indexdef{}{ML}{Sign.add\_abbrev}\verb|Sign.add_abbrev: string -> binding * term ->|\isasep\isanewline%
\verb|  theory -> (term * term) * theory| \\
  \indexdef{}{ML}{Sign.const\_typargs}\verb|Sign.const_typargs: theory -> string * typ -> typ list| \\
  \indexdef{}{ML}{Sign.const\_instance}\verb|Sign.const_instance: theory -> string * typ list -> typ| \\
  \end{mldecls}

  \begin{description}

  \item \verb|term| represents de-Bruijn terms, with comments in
  abstractions, and explicitly named free variables and constants;
  this is a datatype with constructors \verb|Bound|, \verb|Free|, \verb|Var|, \verb|Const|, \verb|Abs|, \verb|op $|.

  \item \isa{t}~\verb|aconv|~\isa{u} checks \isa{{\isasymalpha}}-equivalence of two terms.  This is the basic equality relation
  on type \verb|term|; raw datatype equality should only be used
  for operations related to parsing or printing!

  \item \verb|map_types|~\isa{f\ t} applies the mapping \isa{f} to all types occurring in \isa{t}.

  \item \verb|fold_types|~\isa{f\ t} iterates the operation \isa{f} over all occurrences of types in \isa{t}; the term
  structure is traversed from left to right.

  \item \verb|map_aterms|~\isa{f\ t} applies the mapping \isa{f}
  to all atomic terms (\verb|Bound|, \verb|Free|, \verb|Var|, \verb|Const|) occurring in \isa{t}.

  \item \verb|fold_aterms|~\isa{f\ t} iterates the operation \isa{f} over all occurrences of atomic terms (\verb|Bound|, \verb|Free|,
  \verb|Var|, \verb|Const|) in \isa{t}; the term structure is
  traversed from left to right.

  \item \verb|fastype_of|~\isa{t} determines the type of a
  well-typed term.  This operation is relatively slow, despite the
  omission of any sanity checks.

  \item \verb|lambda|~\isa{a\ b} produces an abstraction \isa{{\isasymlambda}a{\isachardot}\ b}, where occurrences of the atomic term \isa{a} in the
  body \isa{b} are replaced by bound variables.

  \item \verb|betapply|~\isa{{\isacharparenleft}t{\isacharcomma}\ u{\isacharparenright}} produces an application \isa{t\ u}, with topmost \isa{{\isasymbeta}}-conversion if \isa{t} is an
  abstraction.

  \item \verb|Sign.declare_const|~\isa{{\isacharparenleft}{\isacharparenleft}c{\isacharcomma}\ {\isasymsigma}{\isacharparenright}{\isacharcomma}\ mx{\isacharparenright}}
  declares a new constant \isa{c\ {\isacharcolon}{\isacharcolon}\ {\isasymsigma}} with optional mixfix
  syntax.

  \item \verb|Sign.add_abbrev|~\isa{print{\isacharunderscore}mode\ {\isacharparenleft}c{\isacharcomma}\ t{\isacharparenright}}
  introduces a new term abbreviation \isa{c\ {\isasymequiv}\ t}.

  \item \verb|Sign.const_typargs|~\isa{thy\ {\isacharparenleft}c{\isacharcomma}\ {\isasymtau}{\isacharparenright}} and \verb|Sign.const_instance|~\isa{thy\ {\isacharparenleft}c{\isacharcomma}\ {\isacharbrackleft}{\isasymtau}\isactrlisub {\isadigit{1}}{\isacharcomma}\ {\isasymdots}{\isacharcomma}\ {\isasymtau}\isactrlisub n{\isacharbrackright}{\isacharparenright}}
  convert between two representations of polymorphic constants: full
  type instance vs.\ compact type arguments form.

  \end{description}%
\end{isamarkuptext}%
\isamarkuptrue%
%
\endisatagmlref
{\isafoldmlref}%
%
\isadelimmlref
%
\endisadelimmlref
%
\isamarkupsection{Theorems \label{sec:thms}%
}
\isamarkuptrue%
%
\begin{isamarkuptext}%
A \emph{proposition} is a well-typed term of type \isa{prop}, a
  \emph{theorem} is a proven proposition (depending on a context of
  hypotheses and the background theory).  Primitive inferences include
  plain Natural Deduction rules for the primary connectives \isa{{\isasymAnd}} and \isa{{\isasymLongrightarrow}} of the framework.  There is also a builtin
  notion of equality/equivalence \isa{{\isasymequiv}}.%
\end{isamarkuptext}%
\isamarkuptrue%
%
\isamarkupsubsection{Primitive connectives and rules \label{sec:prim-rules}%
}
\isamarkuptrue%
%
\begin{isamarkuptext}%
The theory \isa{Pure} contains constant declarations for the
  primitive connectives \isa{{\isasymAnd}}, \isa{{\isasymLongrightarrow}}, and \isa{{\isasymequiv}} of
  the logical framework, see \figref{fig:pure-connectives}.  The
  derivability judgment \isa{A\isactrlisub {\isadigit{1}}{\isacharcomma}\ {\isasymdots}{\isacharcomma}\ A\isactrlisub n\ {\isasymturnstile}\ B} is
  defined inductively by the primitive inferences given in
  \figref{fig:prim-rules}, with the global restriction that the
  hypotheses must \emph{not} contain any schematic variables.  The
  builtin equality is conceptually axiomatized as shown in
  \figref{fig:pure-equality}, although the implementation works
  directly with derived inferences.

  \begin{figure}[htb]
  \begin{center}
  \begin{tabular}{ll}
  \isa{all\ {\isacharcolon}{\isacharcolon}\ {\isacharparenleft}{\isasymalpha}\ {\isasymRightarrow}\ prop{\isacharparenright}\ {\isasymRightarrow}\ prop} & universal quantification (binder \isa{{\isasymAnd}}) \\
  \isa{{\isasymLongrightarrow}\ {\isacharcolon}{\isacharcolon}\ prop\ {\isasymRightarrow}\ prop\ {\isasymRightarrow}\ prop} & implication (right associative infix) \\
  \isa{{\isasymequiv}\ {\isacharcolon}{\isacharcolon}\ {\isasymalpha}\ {\isasymRightarrow}\ {\isasymalpha}\ {\isasymRightarrow}\ prop} & equality relation (infix) \\
  \end{tabular}
  \caption{Primitive connectives of Pure}\label{fig:pure-connectives}
  \end{center}
  \end{figure}

  \begin{figure}[htb]
  \begin{center}
  \[
  \infer[\isa{{\isacharparenleft}axiom{\isacharparenright}}]{\isa{{\isasymturnstile}\ A}}{\isa{A\ {\isasymin}\ {\isasymTheta}}}
  \qquad
  \infer[\isa{{\isacharparenleft}assume{\isacharparenright}}]{\isa{A\ {\isasymturnstile}\ A}}{}
  \]
  \[
  \infer[\isa{{\isacharparenleft}{\isasymAnd}{\isacharunderscore}intro{\isacharparenright}}]{\isa{{\isasymGamma}\ {\isasymturnstile}\ {\isasymAnd}x{\isachardot}\ b{\isacharbrackleft}x{\isacharbrackright}}}{\isa{{\isasymGamma}\ {\isasymturnstile}\ b{\isacharbrackleft}x{\isacharbrackright}} & \isa{x\ {\isasymnotin}\ {\isasymGamma}}}
  \qquad
  \infer[\isa{{\isacharparenleft}{\isasymAnd}{\isacharunderscore}elim{\isacharparenright}}]{\isa{{\isasymGamma}\ {\isasymturnstile}\ b{\isacharbrackleft}a{\isacharbrackright}}}{\isa{{\isasymGamma}\ {\isasymturnstile}\ {\isasymAnd}x{\isachardot}\ b{\isacharbrackleft}x{\isacharbrackright}}}
  \]
  \[
  \infer[\isa{{\isacharparenleft}{\isasymLongrightarrow}{\isacharunderscore}intro{\isacharparenright}}]{\isa{{\isasymGamma}\ {\isacharminus}\ A\ {\isasymturnstile}\ A\ {\isasymLongrightarrow}\ B}}{\isa{{\isasymGamma}\ {\isasymturnstile}\ B}}
  \qquad
  \infer[\isa{{\isacharparenleft}{\isasymLongrightarrow}{\isacharunderscore}elim{\isacharparenright}}]{\isa{{\isasymGamma}\isactrlsub {\isadigit{1}}\ {\isasymunion}\ {\isasymGamma}\isactrlsub {\isadigit{2}}\ {\isasymturnstile}\ B}}{\isa{{\isasymGamma}\isactrlsub {\isadigit{1}}\ {\isasymturnstile}\ A\ {\isasymLongrightarrow}\ B} & \isa{{\isasymGamma}\isactrlsub {\isadigit{2}}\ {\isasymturnstile}\ A}}
  \]
  \caption{Primitive inferences of Pure}\label{fig:prim-rules}
  \end{center}
  \end{figure}

  \begin{figure}[htb]
  \begin{center}
  \begin{tabular}{ll}
  \isa{{\isasymturnstile}\ {\isacharparenleft}{\isasymlambda}x{\isachardot}\ b{\isacharbrackleft}x{\isacharbrackright}{\isacharparenright}\ a\ {\isasymequiv}\ b{\isacharbrackleft}a{\isacharbrackright}} & \isa{{\isasymbeta}}-conversion \\
  \isa{{\isasymturnstile}\ x\ {\isasymequiv}\ x} & reflexivity \\
  \isa{{\isasymturnstile}\ x\ {\isasymequiv}\ y\ {\isasymLongrightarrow}\ P\ x\ {\isasymLongrightarrow}\ P\ y} & substitution \\
  \isa{{\isasymturnstile}\ {\isacharparenleft}{\isasymAnd}x{\isachardot}\ f\ x\ {\isasymequiv}\ g\ x{\isacharparenright}\ {\isasymLongrightarrow}\ f\ {\isasymequiv}\ g} & extensionality \\
  \isa{{\isasymturnstile}\ {\isacharparenleft}A\ {\isasymLongrightarrow}\ B{\isacharparenright}\ {\isasymLongrightarrow}\ {\isacharparenleft}B\ {\isasymLongrightarrow}\ A{\isacharparenright}\ {\isasymLongrightarrow}\ A\ {\isasymequiv}\ B} & logical equivalence \\
  \end{tabular}
  \caption{Conceptual axiomatization of Pure equality}\label{fig:pure-equality}
  \end{center}
  \end{figure}

  The introduction and elimination rules for \isa{{\isasymAnd}} and \isa{{\isasymLongrightarrow}} are analogous to formation of dependently typed \isa{{\isasymlambda}}-terms representing the underlying proof objects.  Proof terms
  are irrelevant in the Pure logic, though; they cannot occur within
  propositions.  The system provides a runtime option to record
  explicit proof terms for primitive inferences.  Thus all three
  levels of \isa{{\isasymlambda}}-calculus become explicit: \isa{{\isasymRightarrow}} for
  terms, and \isa{{\isasymAnd}{\isacharslash}{\isasymLongrightarrow}} for proofs (cf.\
  \cite{Berghofer-Nipkow:2000:TPHOL}).

  Observe that locally fixed parameters (as in \isa{{\isasymAnd}{\isacharunderscore}intro}) need
  not be recorded in the hypotheses, because the simple syntactic
  types of Pure are always inhabitable.  ``Assumptions'' \isa{x\ {\isacharcolon}{\isacharcolon}\ {\isasymtau}} for type-membership are only present as long as some \isa{x\isactrlisub {\isasymtau}} occurs in the statement body.\footnote{This is the key
  difference to ``\isa{{\isasymlambda}HOL}'' in the PTS framework
  \cite{Barendregt-Geuvers:2001}, where hypotheses \isa{x\ {\isacharcolon}\ A} are
  treated uniformly for propositions and types.}

  \medskip The axiomatization of a theory is implicitly closed by
  forming all instances of type and term variables: \isa{{\isasymturnstile}\ A{\isasymvartheta}} holds for any substitution instance of an axiom
  \isa{{\isasymturnstile}\ A}.  By pushing substitutions through derivations
  inductively, we also get admissible \isa{generalize} and \isa{instance} rules as shown in \figref{fig:subst-rules}.

  \begin{figure}[htb]
  \begin{center}
  \[
  \infer{\isa{{\isasymGamma}\ {\isasymturnstile}\ B{\isacharbrackleft}{\isacharquery}{\isasymalpha}{\isacharbrackright}}}{\isa{{\isasymGamma}\ {\isasymturnstile}\ B{\isacharbrackleft}{\isasymalpha}{\isacharbrackright}} & \isa{{\isasymalpha}\ {\isasymnotin}\ {\isasymGamma}}}
  \quad
  \infer[\quad\isa{{\isacharparenleft}generalize{\isacharparenright}}]{\isa{{\isasymGamma}\ {\isasymturnstile}\ B{\isacharbrackleft}{\isacharquery}x{\isacharbrackright}}}{\isa{{\isasymGamma}\ {\isasymturnstile}\ B{\isacharbrackleft}x{\isacharbrackright}} & \isa{x\ {\isasymnotin}\ {\isasymGamma}}}
  \]
  \[
  \infer{\isa{{\isasymGamma}\ {\isasymturnstile}\ B{\isacharbrackleft}{\isasymtau}{\isacharbrackright}}}{\isa{{\isasymGamma}\ {\isasymturnstile}\ B{\isacharbrackleft}{\isacharquery}{\isasymalpha}{\isacharbrackright}}}
  \quad
  \infer[\quad\isa{{\isacharparenleft}instantiate{\isacharparenright}}]{\isa{{\isasymGamma}\ {\isasymturnstile}\ B{\isacharbrackleft}t{\isacharbrackright}}}{\isa{{\isasymGamma}\ {\isasymturnstile}\ B{\isacharbrackleft}{\isacharquery}x{\isacharbrackright}}}
  \]
  \caption{Admissible substitution rules}\label{fig:subst-rules}
  \end{center}
  \end{figure}

  Note that \isa{instantiate} does not require an explicit
  side-condition, because \isa{{\isasymGamma}} may never contain schematic
  variables.

  In principle, variables could be substituted in hypotheses as well,
  but this would disrupt the monotonicity of reasoning: deriving
  \isa{{\isasymGamma}{\isasymvartheta}\ {\isasymturnstile}\ B{\isasymvartheta}} from \isa{{\isasymGamma}\ {\isasymturnstile}\ B} is
  correct, but \isa{{\isasymGamma}{\isasymvartheta}\ {\isasymsupseteq}\ {\isasymGamma}} does not necessarily hold:
  the result belongs to a different proof context.

  \medskip An \emph{oracle} is a function that produces axioms on the
  fly.  Logically, this is an instance of the \isa{axiom} rule
  (\figref{fig:prim-rules}), but there is an operational difference.
  The system always records oracle invocations within derivations of
  theorems by a unique tag.

  Axiomatizations should be limited to the bare minimum, typically as
  part of the initial logical basis of an object-logic formalization.
  Later on, theories are usually developed in a strictly definitional
  fashion, by stating only certain equalities over new constants.

  A \emph{simple definition} consists of a constant declaration \isa{c\ {\isacharcolon}{\isacharcolon}\ {\isasymsigma}} together with an axiom \isa{{\isasymturnstile}\ c\ {\isasymequiv}\ t}, where \isa{t\ {\isacharcolon}{\isacharcolon}\ {\isasymsigma}} is a closed term without any hidden polymorphism.  The RHS
  may depend on further defined constants, but not \isa{c} itself.
  Definitions of functions may be presented as \isa{c\ \isactrlvec x\ {\isasymequiv}\ t} instead of the puristic \isa{c\ {\isasymequiv}\ {\isasymlambda}\isactrlvec x{\isachardot}\ t}.

  An \emph{overloaded definition} consists of a collection of axioms
  for the same constant, with zero or one equations \isa{c{\isacharparenleft}{\isacharparenleft}\isactrlvec {\isasymalpha}{\isacharparenright}{\isasymkappa}{\isacharparenright}\ {\isasymequiv}\ t} for each type constructor \isa{{\isasymkappa}} (for
  distinct variables \isa{\isactrlvec {\isasymalpha}}).  The RHS may mention
  previously defined constants as above, or arbitrary constants \isa{d{\isacharparenleft}{\isasymalpha}\isactrlisub i{\isacharparenright}} for some \isa{{\isasymalpha}\isactrlisub i} projected from \isa{\isactrlvec {\isasymalpha}}.  Thus overloaded definitions essentially work by
  primitive recursion over the syntactic structure of a single type
  argument.%
\end{isamarkuptext}%
\isamarkuptrue%
%
\isadelimmlref
%
\endisadelimmlref
%
\isatagmlref
%
\begin{isamarkuptext}%
\begin{mldecls}
  \indexdef{}{ML type}{ctyp}\verb|type ctyp| \\
  \indexdef{}{ML type}{cterm}\verb|type cterm| \\
  \indexdef{}{ML}{Thm.ctyp\_of}\verb|Thm.ctyp_of: theory -> typ -> ctyp| \\
  \indexdef{}{ML}{Thm.cterm\_of}\verb|Thm.cterm_of: theory -> term -> cterm| \\
  \end{mldecls}
  \begin{mldecls}
  \indexdef{}{ML type}{thm}\verb|type thm| \\
  \indexdef{}{ML}{proofs}\verb|proofs: int Unsynchronized.ref| \\
  \indexdef{}{ML}{Thm.assume}\verb|Thm.assume: cterm -> thm| \\
  \indexdef{}{ML}{Thm.forall\_intr}\verb|Thm.forall_intr: cterm -> thm -> thm| \\
  \indexdef{}{ML}{Thm.forall\_elim}\verb|Thm.forall_elim: cterm -> thm -> thm| \\
  \indexdef{}{ML}{Thm.implies\_intr}\verb|Thm.implies_intr: cterm -> thm -> thm| \\
  \indexdef{}{ML}{Thm.implies\_elim}\verb|Thm.implies_elim: thm -> thm -> thm| \\
  \indexdef{}{ML}{Thm.generalize}\verb|Thm.generalize: string list * string list -> int -> thm -> thm| \\
  \indexdef{}{ML}{Thm.instantiate}\verb|Thm.instantiate: (ctyp * ctyp) list * (cterm * cterm) list -> thm -> thm| \\
  \indexdef{}{ML}{Thm.axiom}\verb|Thm.axiom: theory -> string -> thm| \\
  \indexdef{}{ML}{Thm.add\_oracle}\verb|Thm.add_oracle: binding * ('a -> cterm) -> theory|\isasep\isanewline%
\verb|  -> (string * ('a -> thm)) * theory| \\
  \end{mldecls}
  \begin{mldecls}
  \indexdef{}{ML}{Theory.add\_axioms\_i}\verb|Theory.add_axioms_i: (binding * term) list -> theory -> theory| \\
  \indexdef{}{ML}{Theory.add\_deps}\verb|Theory.add_deps: string -> string * typ -> (string * typ) list -> theory -> theory| \\
  \indexdef{}{ML}{Theory.add\_defs\_i}\verb|Theory.add_defs_i: bool -> bool -> (binding * term) list -> theory -> theory| \\
  \end{mldecls}

  \begin{description}

  \item \verb|ctyp| and \verb|cterm| represent certified types
  and terms, respectively.  These are abstract datatypes that
  guarantee that its values have passed the full well-formedness (and
  well-typedness) checks, relative to the declarations of type
  constructors, constants etc. in the theory.

  \item \verb|Thm.ctyp_of|~\isa{thy\ {\isasymtau}} and \verb|Thm.cterm_of|~\isa{thy\ t} explicitly checks types and terms,
  respectively.  This also involves some basic normalizations, such
  expansion of type and term abbreviations from the theory context.

  Re-certification is relatively slow and should be avoided in tight
  reasoning loops.  There are separate operations to decompose
  certified entities (including actual theorems).

  \item \verb|thm| represents proven propositions.  This is an
  abstract datatype that guarantees that its values have been
  constructed by basic principles of the \verb|Thm| module.
  Every \verb|thm| value contains a sliding back-reference to the
  enclosing theory, cf.\ \secref{sec:context-theory}.

  \item \verb|proofs| determines the detail of proof recording within
  \verb|thm| values: \verb|0| records only the names of oracles,
  \verb|1| records oracle names and propositions, \verb|2| additionally
  records full proof terms.  Officially named theorems that contribute
  to a result are always recorded.

  \item \verb|Thm.assume|, \verb|Thm.forall_intr|, \verb|Thm.forall_elim|, \verb|Thm.implies_intr|, and \verb|Thm.implies_elim|
  correspond to the primitive inferences of \figref{fig:prim-rules}.

  \item \verb|Thm.generalize|~\isa{{\isacharparenleft}\isactrlvec {\isasymalpha}{\isacharcomma}\ \isactrlvec x{\isacharparenright}}
  corresponds to the \isa{generalize} rules of
  \figref{fig:subst-rules}.  Here collections of type and term
  variables are generalized simultaneously, specified by the given
  basic names.

  \item \verb|Thm.instantiate|~\isa{{\isacharparenleft}\isactrlvec {\isasymalpha}\isactrlisub s{\isacharcomma}\ \isactrlvec x\isactrlisub {\isasymtau}{\isacharparenright}} corresponds to the \isa{instantiate} rules
  of \figref{fig:subst-rules}.  Type variables are substituted before
  term variables.  Note that the types in \isa{\isactrlvec x\isactrlisub {\isasymtau}}
  refer to the instantiated versions.

  \item \verb|Thm.axiom|~\isa{thy\ name} retrieves a named
  axiom, cf.\ \isa{axiom} in \figref{fig:prim-rules}.

  \item \verb|Thm.add_oracle|~\isa{{\isacharparenleft}binding{\isacharcomma}\ oracle{\isacharparenright}} produces a named
  oracle rule, essentially generating arbitrary axioms on the fly,
  cf.\ \isa{axiom} in \figref{fig:prim-rules}.

  \item \verb|Theory.add_axioms_i|~\isa{{\isacharbrackleft}{\isacharparenleft}name{\isacharcomma}\ A{\isacharparenright}{\isacharcomma}\ {\isasymdots}{\isacharbrackright}} declares
  arbitrary propositions as axioms.

  \item \verb|Theory.add_deps|~\isa{name\ c\isactrlisub {\isasymtau}\ \isactrlvec d\isactrlisub {\isasymsigma}} declares dependencies of a named specification
  for constant \isa{c\isactrlisub {\isasymtau}}, relative to existing
  specifications for constants \isa{\isactrlvec d\isactrlisub {\isasymsigma}}.

  \item \verb|Theory.add_defs_i|~\isa{unchecked\ overloaded\ {\isacharbrackleft}{\isacharparenleft}name{\isacharcomma}\ c\ \isactrlvec x\ {\isasymequiv}\ t{\isacharparenright}{\isacharcomma}\ {\isasymdots}{\isacharbrackright}} states a definitional axiom for an existing
  constant \isa{c}.  Dependencies are recorded (cf.\ \verb|Theory.add_deps|), unless the \isa{unchecked} option is set.

  \end{description}%
\end{isamarkuptext}%
\isamarkuptrue%
%
\endisatagmlref
{\isafoldmlref}%
%
\isadelimmlref
%
\endisadelimmlref
%
\isamarkupsubsection{Auxiliary definitions%
}
\isamarkuptrue%
%
\begin{isamarkuptext}%
Theory \isa{Pure} provides a few auxiliary definitions, see
  \figref{fig:pure-aux}.  These special constants are normally not
  exposed to the user, but appear in internal encodings.

  \begin{figure}[htb]
  \begin{center}
  \begin{tabular}{ll}
  \isa{conjunction\ {\isacharcolon}{\isacharcolon}\ prop\ {\isasymRightarrow}\ prop\ {\isasymRightarrow}\ prop} & (infix \isa{{\isacharampersand}}) \\
  \isa{{\isasymturnstile}\ A\ {\isacharampersand}\ B\ {\isasymequiv}\ {\isacharparenleft}{\isasymAnd}C{\isachardot}\ {\isacharparenleft}A\ {\isasymLongrightarrow}\ B\ {\isasymLongrightarrow}\ C{\isacharparenright}\ {\isasymLongrightarrow}\ C{\isacharparenright}} \\[1ex]
  \isa{prop\ {\isacharcolon}{\isacharcolon}\ prop\ {\isasymRightarrow}\ prop} & (prefix \isa{{\isacharhash}}, suppressed) \\
  \isa{{\isacharhash}A\ {\isasymequiv}\ A} \\[1ex]
  \isa{term\ {\isacharcolon}{\isacharcolon}\ {\isasymalpha}\ {\isasymRightarrow}\ prop} & (prefix \isa{TERM}) \\
  \isa{term\ x\ {\isasymequiv}\ {\isacharparenleft}{\isasymAnd}A{\isachardot}\ A\ {\isasymLongrightarrow}\ A{\isacharparenright}} \\[1ex]
  \isa{TYPE\ {\isacharcolon}{\isacharcolon}\ {\isasymalpha}\ itself} & (prefix \isa{TYPE}) \\
  \isa{{\isacharparenleft}unspecified{\isacharparenright}} \\
  \end{tabular}
  \caption{Definitions of auxiliary connectives}\label{fig:pure-aux}
  \end{center}
  \end{figure}

  Derived conjunction rules include introduction \isa{A\ {\isasymLongrightarrow}\ B\ {\isasymLongrightarrow}\ A\ {\isacharampersand}\ B}, and destructions \isa{A\ {\isacharampersand}\ B\ {\isasymLongrightarrow}\ A} and \isa{A\ {\isacharampersand}\ B\ {\isasymLongrightarrow}\ B}.
  Conjunction allows to treat simultaneous assumptions and conclusions
  uniformly.  For example, multiple claims are intermediately
  represented as explicit conjunction, but this is refined into
  separate sub-goals before the user continues the proof; the final
  result is projected into a list of theorems (cf.\
  \secref{sec:tactical-goals}).

  The \isa{prop} marker (\isa{{\isacharhash}}) makes arbitrarily complex
  propositions appear as atomic, without changing the meaning: \isa{{\isasymGamma}\ {\isasymturnstile}\ A} and \isa{{\isasymGamma}\ {\isasymturnstile}\ {\isacharhash}A} are interchangeable.  See
  \secref{sec:tactical-goals} for specific operations.

  The \isa{term} marker turns any well-typed term into a derivable
  proposition: \isa{{\isasymturnstile}\ TERM\ t} holds unconditionally.  Although
  this is logically vacuous, it allows to treat terms and proofs
  uniformly, similar to a type-theoretic framework.

  The \isa{TYPE} constructor is the canonical representative of
  the unspecified type \isa{{\isasymalpha}\ itself}; it essentially injects the
  language of types into that of terms.  There is specific notation
  \isa{TYPE{\isacharparenleft}{\isasymtau}{\isacharparenright}} for \isa{TYPE\isactrlbsub {\isasymtau}\ itself\isactrlesub }.
  Although being devoid of any particular meaning, the \isa{TYPE{\isacharparenleft}{\isasymtau}{\isacharparenright}} accounts for the type \isa{{\isasymtau}} within the term
  language.  In particular, \isa{TYPE{\isacharparenleft}{\isasymalpha}{\isacharparenright}} may be used as formal
  argument in primitive definitions, in order to circumvent hidden
  polymorphism (cf.\ \secref{sec:terms}).  For example, \isa{c\ TYPE{\isacharparenleft}{\isasymalpha}{\isacharparenright}\ {\isasymequiv}\ A{\isacharbrackleft}{\isasymalpha}{\isacharbrackright}} defines \isa{c\ {\isacharcolon}{\isacharcolon}\ {\isasymalpha}\ itself\ {\isasymRightarrow}\ prop} in terms of
  a proposition \isa{A} that depends on an additional type
  argument, which is essentially a predicate on types.%
\end{isamarkuptext}%
\isamarkuptrue%
%
\isadelimmlref
%
\endisadelimmlref
%
\isatagmlref
%
\begin{isamarkuptext}%
\begin{mldecls}
  \indexdef{}{ML}{Conjunction.intr}\verb|Conjunction.intr: thm -> thm -> thm| \\
  \indexdef{}{ML}{Conjunction.elim}\verb|Conjunction.elim: thm -> thm * thm| \\
  \indexdef{}{ML}{Drule.mk\_term}\verb|Drule.mk_term: cterm -> thm| \\
  \indexdef{}{ML}{Drule.dest\_term}\verb|Drule.dest_term: thm -> cterm| \\
  \indexdef{}{ML}{Logic.mk\_type}\verb|Logic.mk_type: typ -> term| \\
  \indexdef{}{ML}{Logic.dest\_type}\verb|Logic.dest_type: term -> typ| \\
  \end{mldecls}

  \begin{description}

  \item \verb|Conjunction.intr| derives \isa{A\ {\isacharampersand}\ B} from \isa{A} and \isa{B}.

  \item \verb|Conjunction.elim| derives \isa{A} and \isa{B}
  from \isa{A\ {\isacharampersand}\ B}.

  \item \verb|Drule.mk_term| derives \isa{TERM\ t}.

  \item \verb|Drule.dest_term| recovers term \isa{t} from \isa{TERM\ t}.

  \item \verb|Logic.mk_type|~\isa{{\isasymtau}} produces the term \isa{TYPE{\isacharparenleft}{\isasymtau}{\isacharparenright}}.

  \item \verb|Logic.dest_type|~\isa{TYPE{\isacharparenleft}{\isasymtau}{\isacharparenright}} recovers the type
  \isa{{\isasymtau}}.

  \end{description}%
\end{isamarkuptext}%
\isamarkuptrue%
%
\endisatagmlref
{\isafoldmlref}%
%
\isadelimmlref
%
\endisadelimmlref
%
\isamarkupsection{Object-level rules \label{sec:obj-rules}%
}
\isamarkuptrue%
%
\begin{isamarkuptext}%
The primitive inferences covered so far mostly serve foundational
  purposes.  User-level reasoning usually works via object-level rules
  that are represented as theorems of Pure.  Composition of rules
  involves \emph{backchaining}, \emph{higher-order unification} modulo
  \isa{{\isasymalpha}{\isasymbeta}{\isasymeta}}-conversion of \isa{{\isasymlambda}}-terms, and so-called
  \emph{lifting} of rules into a context of \isa{{\isasymAnd}} and \isa{{\isasymLongrightarrow}} connectives.  Thus the full power of higher-order Natural
  Deduction in Isabelle/Pure becomes readily available.%
\end{isamarkuptext}%
\isamarkuptrue%
%
\isamarkupsubsection{Hereditary Harrop Formulae%
}
\isamarkuptrue%
%
\begin{isamarkuptext}%
The idea of object-level rules is to model Natural Deduction
  inferences in the style of Gentzen \cite{Gentzen:1935}, but we allow
  arbitrary nesting similar to \cite{extensions91}.  The most basic
  rule format is that of a \emph{Horn Clause}:
  \[
  \infer{\isa{A}}{\isa{A\isactrlsub {\isadigit{1}}} & \isa{{\isasymdots}} & \isa{A\isactrlsub n}}
  \]
  where \isa{A{\isacharcomma}\ A\isactrlsub {\isadigit{1}}{\isacharcomma}\ {\isasymdots}{\isacharcomma}\ A\isactrlsub n} are atomic propositions
  of the framework, usually of the form \isa{Trueprop\ B}, where
  \isa{B} is a (compound) object-level statement.  This
  object-level inference corresponds to an iterated implication in
  Pure like this:
  \[
  \isa{A\isactrlsub {\isadigit{1}}\ {\isasymLongrightarrow}\ {\isasymdots}\ A\isactrlsub n\ {\isasymLongrightarrow}\ A}
  \]
  As an example consider conjunction introduction: \isa{A\ {\isasymLongrightarrow}\ B\ {\isasymLongrightarrow}\ A\ {\isasymand}\ B}.  Any parameters occurring in such rule statements are
  conceptionally treated as arbitrary:
  \[
  \isa{{\isasymAnd}x\isactrlsub {\isadigit{1}}\ {\isasymdots}\ x\isactrlsub m{\isachardot}\ A\isactrlsub {\isadigit{1}}\ x\isactrlsub {\isadigit{1}}\ {\isasymdots}\ x\isactrlsub m\ {\isasymLongrightarrow}\ {\isasymdots}\ A\isactrlsub n\ x\isactrlsub {\isadigit{1}}\ {\isasymdots}\ x\isactrlsub m\ {\isasymLongrightarrow}\ A\ x\isactrlsub {\isadigit{1}}\ {\isasymdots}\ x\isactrlsub m}
  \]

  Nesting of rules means that the positions of \isa{A\isactrlsub i} may
  again hold compound rules, not just atomic propositions.
  Propositions of this format are called \emph{Hereditary Harrop
  Formulae} in the literature \cite{Miller:1991}.  Here we give an
  inductive characterization as follows:

  \medskip
  \begin{tabular}{ll}
  \isa{\isactrlbold x} & set of variables \\
  \isa{\isactrlbold A} & set of atomic propositions \\
  \isa{\isactrlbold H\ \ {\isacharequal}\ \ {\isasymAnd}\isactrlbold x\isactrlsup {\isacharasterisk}{\isachardot}\ \isactrlbold H\isactrlsup {\isacharasterisk}\ {\isasymLongrightarrow}\ \isactrlbold A} & set of Hereditary Harrop Formulas \\
  \end{tabular}
  \medskip

  \noindent Thus we essentially impose nesting levels on propositions
  formed from \isa{{\isasymAnd}} and \isa{{\isasymLongrightarrow}}.  At each level there is a
  prefix of parameters and compound premises, concluding an atomic
  proposition.  Typical examples are \isa{{\isasymlongrightarrow}}-introduction \isa{{\isacharparenleft}A\ {\isasymLongrightarrow}\ B{\isacharparenright}\ {\isasymLongrightarrow}\ A\ {\isasymlongrightarrow}\ B} or mathematical induction \isa{P\ {\isadigit{0}}\ {\isasymLongrightarrow}\ {\isacharparenleft}{\isasymAnd}n{\isachardot}\ P\ n\ {\isasymLongrightarrow}\ P\ {\isacharparenleft}Suc\ n{\isacharparenright}{\isacharparenright}\ {\isasymLongrightarrow}\ P\ n}.  Even deeper nesting occurs in well-founded
  induction \isa{{\isacharparenleft}{\isasymAnd}x{\isachardot}\ {\isacharparenleft}{\isasymAnd}y{\isachardot}\ y\ {\isasymprec}\ x\ {\isasymLongrightarrow}\ P\ y{\isacharparenright}\ {\isasymLongrightarrow}\ P\ x{\isacharparenright}\ {\isasymLongrightarrow}\ P\ x}, but this
  already marks the limit of rule complexity seen in practice.

  \medskip Regular user-level inferences in Isabelle/Pure always
  maintain the following canonical form of results:

  \begin{itemize}

  \item Normalization by \isa{{\isacharparenleft}A\ {\isasymLongrightarrow}\ {\isacharparenleft}{\isasymAnd}x{\isachardot}\ B\ x{\isacharparenright}{\isacharparenright}\ {\isasymequiv}\ {\isacharparenleft}{\isasymAnd}x{\isachardot}\ A\ {\isasymLongrightarrow}\ B\ x{\isacharparenright}},
  which is a theorem of Pure, means that quantifiers are pushed in
  front of implication at each level of nesting.  The normal form is a
  Hereditary Harrop Formula.

  \item The outermost prefix of parameters is represented via
  schematic variables: instead of \isa{{\isasymAnd}\isactrlvec x{\isachardot}\ \isactrlvec H\ \isactrlvec x\ {\isasymLongrightarrow}\ A\ \isactrlvec x} we have \isa{\isactrlvec H\ {\isacharquery}\isactrlvec x\ {\isasymLongrightarrow}\ A\ {\isacharquery}\isactrlvec x}.
  Note that this representation looses information about the order of
  parameters, and vacuous quantifiers vanish automatically.

  \end{itemize}%
\end{isamarkuptext}%
\isamarkuptrue%
%
\isadelimmlref
%
\endisadelimmlref
%
\isatagmlref
%
\begin{isamarkuptext}%
\begin{mldecls}
  \indexdef{}{ML}{Simplifier.norm\_hhf}\verb|Simplifier.norm_hhf: thm -> thm| \\
  \end{mldecls}

  \begin{description}

  \item \verb|Simplifier.norm_hhf|~\isa{thm} normalizes the given
  theorem according to the canonical form specified above.  This is
  occasionally helpful to repair some low-level tools that do not
  handle Hereditary Harrop Formulae properly.

  \end{description}%
\end{isamarkuptext}%
\isamarkuptrue%
%
\endisatagmlref
{\isafoldmlref}%
%
\isadelimmlref
%
\endisadelimmlref
%
\isamarkupsubsection{Rule composition%
}
\isamarkuptrue%
%
\begin{isamarkuptext}%
The rule calculus of Isabelle/Pure provides two main inferences:
  \hyperlink{inference.resolution}{\mbox{\isa{resolution}}} (i.e.\ back-chaining of rules) and
  \hyperlink{inference.assumption}{\mbox{\isa{assumption}}} (i.e.\ closing a branch), both modulo
  higher-order unification.  There are also combined variants, notably
  \hyperlink{inference.elim-resolution}{\mbox{\isa{elim{\isacharunderscore}resolution}}} and \hyperlink{inference.dest-resolution}{\mbox{\isa{dest{\isacharunderscore}resolution}}}.

  To understand the all-important \hyperlink{inference.resolution}{\mbox{\isa{resolution}}} principle,
  we first consider raw \indexdef{}{inference}{composition}\hypertarget{inference.composition}{\hyperlink{inference.composition}{\mbox{\isa{composition}}}} (modulo
  higher-order unification with substitution \isa{{\isasymvartheta}}):
  \[
  \infer[(\indexdef{}{inference}{composition}\hypertarget{inference.composition}{\hyperlink{inference.composition}{\mbox{\isa{composition}}}})]{\isa{\isactrlvec A{\isasymvartheta}\ {\isasymLongrightarrow}\ C{\isasymvartheta}}}
  {\isa{\isactrlvec A\ {\isasymLongrightarrow}\ B} & \isa{B{\isacharprime}\ {\isasymLongrightarrow}\ C} & \isa{B{\isasymvartheta}\ {\isacharequal}\ B{\isacharprime}{\isasymvartheta}}}
  \]
  Here the conclusion of the first rule is unified with the premise of
  the second; the resulting rule instance inherits the premises of the
  first and conclusion of the second.  Note that \isa{C} can again
  consist of iterated implications.  We can also permute the premises
  of the second rule back-and-forth in order to compose with \isa{B{\isacharprime}} in any position (subsequently we shall always refer to
  position 1 w.l.o.g.).

  In \hyperlink{inference.composition}{\mbox{\isa{composition}}} the internal structure of the common
  part \isa{B} and \isa{B{\isacharprime}} is not taken into account.  For
  proper \hyperlink{inference.resolution}{\mbox{\isa{resolution}}} we require \isa{B} to be atomic,
  and explicitly observe the structure \isa{{\isasymAnd}\isactrlvec x{\isachardot}\ \isactrlvec H\ \isactrlvec x\ {\isasymLongrightarrow}\ B{\isacharprime}\ \isactrlvec x} of the premise of the second rule.  The
  idea is to adapt the first rule by ``lifting'' it into this context,
  by means of iterated application of the following inferences:
  \[
  \infer[(\indexdef{}{inference}{imp\_lift}\hypertarget{inference.imp-lift}{\hyperlink{inference.imp-lift}{\mbox{\isa{imp{\isacharunderscore}lift}}}})]{\isa{{\isacharparenleft}\isactrlvec H\ {\isasymLongrightarrow}\ \isactrlvec A{\isacharparenright}\ {\isasymLongrightarrow}\ {\isacharparenleft}\isactrlvec H\ {\isasymLongrightarrow}\ B{\isacharparenright}}}{\isa{\isactrlvec A\ {\isasymLongrightarrow}\ B}}
  \]
  \[
  \infer[(\indexdef{}{inference}{all\_lift}\hypertarget{inference.all-lift}{\hyperlink{inference.all-lift}{\mbox{\isa{all{\isacharunderscore}lift}}}})]{\isa{{\isacharparenleft}{\isasymAnd}\isactrlvec x{\isachardot}\ \isactrlvec A\ {\isacharparenleft}{\isacharquery}\isactrlvec a\ \isactrlvec x{\isacharparenright}{\isacharparenright}\ {\isasymLongrightarrow}\ {\isacharparenleft}{\isasymAnd}\isactrlvec x{\isachardot}\ B\ {\isacharparenleft}{\isacharquery}\isactrlvec a\ \isactrlvec x{\isacharparenright}{\isacharparenright}}}{\isa{\isactrlvec A\ {\isacharquery}\isactrlvec a\ {\isasymLongrightarrow}\ B\ {\isacharquery}\isactrlvec a}}
  \]
  By combining raw composition with lifting, we get full \hyperlink{inference.resolution}{\mbox{\isa{resolution}}} as follows:
  \[
  \infer[(\indexdef{}{inference}{resolution}\hypertarget{inference.resolution}{\hyperlink{inference.resolution}{\mbox{\isa{resolution}}}})]
  {\isa{{\isacharparenleft}{\isasymAnd}\isactrlvec x{\isachardot}\ \isactrlvec H\ \isactrlvec x\ {\isasymLongrightarrow}\ \isactrlvec A\ {\isacharparenleft}{\isacharquery}\isactrlvec a\ \isactrlvec x{\isacharparenright}{\isacharparenright}{\isasymvartheta}\ {\isasymLongrightarrow}\ C{\isasymvartheta}}}
  {\begin{tabular}{l}
    \isa{\isactrlvec A\ {\isacharquery}\isactrlvec a\ {\isasymLongrightarrow}\ B\ {\isacharquery}\isactrlvec a} \\
    \isa{{\isacharparenleft}{\isasymAnd}\isactrlvec x{\isachardot}\ \isactrlvec H\ \isactrlvec x\ {\isasymLongrightarrow}\ B{\isacharprime}\ \isactrlvec x{\isacharparenright}\ {\isasymLongrightarrow}\ C} \\
    \isa{{\isacharparenleft}{\isasymlambda}\isactrlvec x{\isachardot}\ B\ {\isacharparenleft}{\isacharquery}\isactrlvec a\ \isactrlvec x{\isacharparenright}{\isacharparenright}{\isasymvartheta}\ {\isacharequal}\ B{\isacharprime}{\isasymvartheta}} \\
   \end{tabular}}
  \]

  Continued resolution of rules allows to back-chain a problem towards
  more and sub-problems.  Branches are closed either by resolving with
  a rule of 0 premises, or by producing a ``short-circuit'' within a
  solved situation (again modulo unification):
  \[
  \infer[(\indexdef{}{inference}{assumption}\hypertarget{inference.assumption}{\hyperlink{inference.assumption}{\mbox{\isa{assumption}}}})]{\isa{C{\isasymvartheta}}}
  {\isa{{\isacharparenleft}{\isasymAnd}\isactrlvec x{\isachardot}\ \isactrlvec H\ \isactrlvec x\ {\isasymLongrightarrow}\ A\ \isactrlvec x{\isacharparenright}\ {\isasymLongrightarrow}\ C} & \isa{A{\isasymvartheta}\ {\isacharequal}\ H\isactrlsub i{\isasymvartheta}}~~\text{(for some~\isa{i})}}
  \]

  FIXME \indexdef{}{inference}{elim\_resolution}\hypertarget{inference.elim-resolution}{\hyperlink{inference.elim-resolution}{\mbox{\isa{elim{\isacharunderscore}resolution}}}}, \indexdef{}{inference}{dest\_resolution}\hypertarget{inference.dest-resolution}{\hyperlink{inference.dest-resolution}{\mbox{\isa{dest{\isacharunderscore}resolution}}}}%
\end{isamarkuptext}%
\isamarkuptrue%
%
\isadelimmlref
%
\endisadelimmlref
%
\isatagmlref
%
\begin{isamarkuptext}%
\begin{mldecls}
  \indexdef{}{ML}{op RS}\verb|op RS: thm * thm -> thm| \\
  \indexdef{}{ML}{op OF}\verb|op OF: thm * thm list -> thm| \\
  \end{mldecls}

  \begin{description}

  \item \isa{rule\isactrlsub {\isadigit{1}}\ RS\ rule\isactrlsub {\isadigit{2}}} resolves \isa{rule\isactrlsub {\isadigit{1}}} with \isa{rule\isactrlsub {\isadigit{2}}} according to the
  \hyperlink{inference.resolution}{\mbox{\isa{resolution}}} principle explained above.  Note that the
  corresponding attribute in the Isar language is called \hyperlink{attribute.THEN}{\mbox{\isa{THEN}}}.

  \item \isa{rule\ OF\ rules} resolves a list of rules with the
  first rule, addressing its premises \isa{{\isadigit{1}}{\isacharcomma}\ {\isasymdots}{\isacharcomma}\ length\ rules}
  (operating from last to first).  This means the newly emerging
  premises are all concatenated, without interfering.  Also note that
  compared to \isa{RS}, the rule argument order is swapped: \isa{rule\isactrlsub {\isadigit{1}}\ RS\ rule\isactrlsub {\isadigit{2}}\ {\isacharequal}\ rule\isactrlsub {\isadigit{2}}\ OF\ {\isacharbrackleft}rule\isactrlsub {\isadigit{1}}{\isacharbrackright}}.

  \end{description}%
\end{isamarkuptext}%
\isamarkuptrue%
%
\endisatagmlref
{\isafoldmlref}%
%
\isadelimmlref
%
\endisadelimmlref
%
\isadelimtheory
%
\endisadelimtheory
%
\isatagtheory
\isacommand{end}\isamarkupfalse%
%
\endisatagtheory
{\isafoldtheory}%
%
\isadelimtheory
%
\endisadelimtheory
\isanewline
\end{isabellebody}%
%%% Local Variables:
%%% mode: latex
%%% TeX-master: "root"
%%% End:

%
\begin{isabellebody}%
\def\isabellecontext{Induction}%
%
\isadelimtheory
%
\endisadelimtheory
%
\isatagtheory
\isamarkupfalse%
%
\endisatagtheory
{\isafoldtheory}%
%
\isadelimtheory
%
\endisadelimtheory
%
\isamarkupsection{Case distinction and induction \label{sec:Induct}%
}
\isamarkuptrue%
%
\begin{isamarkuptext}%
Computer science applications abound with inductively defined
structures, which is why we treat them in more detail. HOL already
comes with a datatype of lists with the two constructors \isa{Nil}
and \isa{Cons}. \isa{Nil} is written \isa{{\isacharbrackleft}{\isacharbrackright}} and \isa{Cons\ x\ xs} is written \isa{x\ {\isacharhash}\ xs}.%
\end{isamarkuptext}%
\isamarkuptrue%
%
\isamarkupsubsection{Case distinction\label{sec:CaseDistinction}%
}
\isamarkuptrue%
%
\begin{isamarkuptext}%
We have already met the \isa{cases} method for performing
binary case splits. Here is another example:%
\end{isamarkuptext}%
\isamarkuptrue%
\isacommand{lemma}\isamarkupfalse%
\ {\isachardoublequoteopen}{\isasymnot}\ A\ {\isasymor}\ A{\isachardoublequoteclose}\isanewline
%
\isadelimproof
%
\endisadelimproof
%
\isatagproof
\isacommand{proof}\isamarkupfalse%
\ cases\isanewline
\ \ \isacommand{assume}\isamarkupfalse%
\ {\isachardoublequoteopen}A{\isachardoublequoteclose}\ \isacommand{thus}\isamarkupfalse%
\ {\isacharquery}thesis\ \isacommand{{\isachardot}{\isachardot}}\isamarkupfalse%
\isanewline
\isacommand{next}\isamarkupfalse%
\isanewline
\ \ \isacommand{assume}\isamarkupfalse%
\ {\isachardoublequoteopen}{\isasymnot}\ A{\isachardoublequoteclose}\ \isacommand{thus}\isamarkupfalse%
\ {\isacharquery}thesis\ \isacommand{{\isachardot}{\isachardot}}\isamarkupfalse%
\isanewline
\isacommand{qed}\isamarkupfalse%
%
\endisatagproof
{\isafoldproof}%
%
\isadelimproof
%
\endisadelimproof
%
\begin{isamarkuptext}%
\noindent The two cases must come in this order because \isa{cases} merely abbreviates \isa{{\isacharparenleft}rule\ case{\isacharunderscore}split{\isacharunderscore}thm{\isacharparenright}} where
\isa{case{\isacharunderscore}split{\isacharunderscore}thm} is \isa{{\isasymlbrakk}{\isacharquery}P\ {\isasymLongrightarrow}\ {\isacharquery}Q{\isacharsemicolon}\ {\isasymnot}\ {\isacharquery}P\ {\isasymLongrightarrow}\ {\isacharquery}Q{\isasymrbrakk}\ {\isasymLongrightarrow}\ {\isacharquery}Q}. If we reverse
the order of the two cases in the proof, the first case would prove
\isa{{\isasymnot}\ A\ {\isasymLongrightarrow}\ {\isasymnot}\ A\ {\isasymor}\ A} which would solve the first premise of
\isa{case{\isacharunderscore}split{\isacharunderscore}thm}, instantiating \isa{{\isacharquery}P} with \isa{{\isasymnot}\ A}, thus making the second premise \isa{{\isasymnot}\ {\isasymnot}\ A\ {\isasymLongrightarrow}\ {\isasymnot}\ A\ {\isasymor}\ A}.
Therefore the order of subgoals is not always completely arbitrary.

The above proof is appropriate if \isa{A} is textually small.
However, if \isa{A} is large, we do not want to repeat it. This can
be avoided by the following idiom%
\end{isamarkuptext}%
\isamarkuptrue%
\isacommand{lemma}\isamarkupfalse%
\ {\isachardoublequoteopen}{\isasymnot}\ A\ {\isasymor}\ A{\isachardoublequoteclose}\isanewline
%
\isadelimproof
%
\endisadelimproof
%
\isatagproof
\isacommand{proof}\isamarkupfalse%
\ {\isacharparenleft}cases\ {\isachardoublequoteopen}A{\isachardoublequoteclose}{\isacharparenright}\isanewline
\ \ \isacommand{case}\isamarkupfalse%
\ True\ \isacommand{thus}\isamarkupfalse%
\ {\isacharquery}thesis\ \isacommand{{\isachardot}{\isachardot}}\isamarkupfalse%
\isanewline
\isacommand{next}\isamarkupfalse%
\isanewline
\ \ \isacommand{case}\isamarkupfalse%
\ False\ \isacommand{thus}\isamarkupfalse%
\ {\isacharquery}thesis\ \isacommand{{\isachardot}{\isachardot}}\isamarkupfalse%
\isanewline
\isacommand{qed}\isamarkupfalse%
%
\endisatagproof
{\isafoldproof}%
%
\isadelimproof
%
\endisadelimproof
%
\begin{isamarkuptext}%
\noindent which is like the previous proof but instantiates
\isa{{\isacharquery}P} right away with \isa{A}. Thus we could prove the two
cases in any order. The phrase `\isakeyword{case}~\isa{True}'
abbreviates `\isakeyword{assume}~\isa{True{\isacharcolon}\ A}' and analogously for
\isa{False} and \isa{{\isasymnot}\ A}.

The same game can be played with other datatypes, for example lists,
where \isa{tl} is the tail of a list, and \isa{length} returns a
natural number (remember: $0-1=0$):%
\end{isamarkuptext}%
\isamarkuptrue%
\isamarkupfalse%
\isacommand{lemma}\isamarkupfalse%
\ {\isachardoublequoteopen}length{\isacharparenleft}tl\ xs{\isacharparenright}\ {\isacharequal}\ length\ xs\ {\isacharminus}\ {\isadigit{1}}{\isachardoublequoteclose}\isanewline
%
\isadelimproof
%
\endisadelimproof
%
\isatagproof
\isacommand{proof}\isamarkupfalse%
\ {\isacharparenleft}cases\ xs{\isacharparenright}\isanewline
\ \ \isacommand{case}\isamarkupfalse%
\ Nil\ \isacommand{thus}\isamarkupfalse%
\ {\isacharquery}thesis\ \isacommand{by}\isamarkupfalse%
\ simp\isanewline
\isacommand{next}\isamarkupfalse%
\isanewline
\ \ \isacommand{case}\isamarkupfalse%
\ Cons\ \isacommand{thus}\isamarkupfalse%
\ {\isacharquery}thesis\ \isacommand{by}\isamarkupfalse%
\ simp\isanewline
\isacommand{qed}\isamarkupfalse%
%
\endisatagproof
{\isafoldproof}%
%
\isadelimproof
%
\endisadelimproof
%
\begin{isamarkuptext}%
\noindent Here `\isakeyword{case}~\isa{Nil}' abbreviates
`\isakeyword{assume}~\isa{Nil{\isacharcolon}}~\isa{xs\ {\isacharequal}\ {\isacharbrackleft}{\isacharbrackright}}' and
`\isakeyword{case}~\isa{Cons}'
abbreviates `\isakeyword{fix}~\isa{{\isacharquery}\ {\isacharquery}{\isacharquery}}
\isakeyword{assume}~\isa{Cons{\isacharcolon}}~\isa{xs\ {\isacharequal}\ {\isacharquery}\ {\isacharhash}\ {\isacharquery}{\isacharquery}}'
where \isa{{\isacharquery}} and \isa{{\isacharquery}{\isacharquery}}
stand for variable names that have been chosen by the system.
Therefore we cannot refer to them.
Luckily, this proof is simple enough we do not need to refer to them.
However, sometimes one may have to. Hence Isar offers a simple scheme for
naming those variables: replace the anonymous \isa{Cons} by
\isa{{\isacharparenleft}Cons\ y\ ys{\isacharparenright}}, which abbreviates `\isakeyword{fix}~\isa{y\ ys}
\isakeyword{assume}~\isa{Cons{\isacharcolon}}~\isa{xs\ {\isacharequal}\ y\ {\isacharhash}\ ys}'.
In each \isakeyword{case} the assumption can be
referred to inside the proof by the name of the constructor. In
Section~\ref{sec:full-Ind} below we will come across an example
of this.%
\end{isamarkuptext}%
\isamarkuptrue%
%
\isamarkupsubsection{Structural induction%
}
\isamarkuptrue%
%
\begin{isamarkuptext}%
We start with an inductive proof where both cases are proved automatically:%
\end{isamarkuptext}%
\isamarkuptrue%
\isacommand{lemma}\isamarkupfalse%
\ {\isachardoublequoteopen}{\isadigit{2}}\ {\isacharasterisk}\ {\isacharparenleft}{\isasymSum}i{\isacharcolon}{\isacharcolon}nat{\isasymle}n{\isachardot}\ i{\isacharparenright}\ {\isacharequal}\ n{\isacharasterisk}{\isacharparenleft}n{\isacharplus}{\isadigit{1}}{\isacharparenright}{\isachardoublequoteclose}\isanewline
%
\isadelimproof
%
\endisadelimproof
%
\isatagproof
\isacommand{by}\isamarkupfalse%
\ {\isacharparenleft}induct\ n{\isacharcomma}\ simp{\isacharunderscore}all{\isacharparenright}%
\endisatagproof
{\isafoldproof}%
%
\isadelimproof
%
\endisadelimproof
%
\begin{isamarkuptext}%
\noindent The constraint \isa{{\isacharcolon}{\isacharcolon}nat} is needed because all of
the operations involved are overloaded.

If we want to expose more of the structure of the
proof, we can use pattern matching to avoid having to repeat the goal
statement:%
\end{isamarkuptext}%
\isamarkuptrue%
\isacommand{lemma}\isamarkupfalse%
\ {\isachardoublequoteopen}{\isadigit{2}}\ {\isacharasterisk}\ {\isacharparenleft}{\isasymSum}i{\isacharcolon}{\isacharcolon}nat{\isasymle}n{\isachardot}\ i{\isacharparenright}\ {\isacharequal}\ n{\isacharasterisk}{\isacharparenleft}n{\isacharplus}{\isadigit{1}}{\isacharparenright}{\isachardoublequoteclose}\ {\isacharparenleft}\isakeyword{is}\ {\isachardoublequoteopen}{\isacharquery}P\ n{\isachardoublequoteclose}{\isacharparenright}\isanewline
%
\isadelimproof
%
\endisadelimproof
%
\isatagproof
\isacommand{proof}\isamarkupfalse%
\ {\isacharparenleft}induct\ n{\isacharparenright}\isanewline
\ \ \isacommand{show}\isamarkupfalse%
\ {\isachardoublequoteopen}{\isacharquery}P\ {\isadigit{0}}{\isachardoublequoteclose}\ \isacommand{by}\isamarkupfalse%
\ simp\isanewline
\isacommand{next}\isamarkupfalse%
\isanewline
\ \ \isacommand{fix}\isamarkupfalse%
\ n\ \isacommand{assume}\isamarkupfalse%
\ {\isachardoublequoteopen}{\isacharquery}P\ n{\isachardoublequoteclose}\isanewline
\ \ \isacommand{thus}\isamarkupfalse%
\ {\isachardoublequoteopen}{\isacharquery}P{\isacharparenleft}Suc\ n{\isacharparenright}{\isachardoublequoteclose}\ \isacommand{by}\isamarkupfalse%
\ simp\isanewline
\isacommand{qed}\isamarkupfalse%
%
\endisatagproof
{\isafoldproof}%
%
\isadelimproof
%
\endisadelimproof
%
\begin{isamarkuptext}%
\noindent We could refine this further to show more of the equational
proof. Instead we explore the same avenue as for case distinctions:
introducing context via the \isakeyword{case} command:%
\end{isamarkuptext}%
\isamarkuptrue%
\isacommand{lemma}\isamarkupfalse%
\ {\isachardoublequoteopen}{\isadigit{2}}\ {\isacharasterisk}\ {\isacharparenleft}{\isasymSum}i{\isacharcolon}{\isacharcolon}nat\ {\isasymle}\ n{\isachardot}\ i{\isacharparenright}\ {\isacharequal}\ n{\isacharasterisk}{\isacharparenleft}n{\isacharplus}{\isadigit{1}}{\isacharparenright}{\isachardoublequoteclose}\isanewline
%
\isadelimproof
%
\endisadelimproof
%
\isatagproof
\isacommand{proof}\isamarkupfalse%
\ {\isacharparenleft}induct\ n{\isacharparenright}\isanewline
\ \ \isacommand{case}\isamarkupfalse%
\ {\isadigit{0}}\ \isacommand{show}\isamarkupfalse%
\ {\isacharquery}case\ \isacommand{by}\isamarkupfalse%
\ simp\isanewline
\isacommand{next}\isamarkupfalse%
\isanewline
\ \ \isacommand{case}\isamarkupfalse%
\ Suc\ \isacommand{thus}\isamarkupfalse%
\ {\isacharquery}case\ \isacommand{by}\isamarkupfalse%
\ simp\isanewline
\isacommand{qed}\isamarkupfalse%
%
\endisatagproof
{\isafoldproof}%
%
\isadelimproof
%
\endisadelimproof
%
\begin{isamarkuptext}%
\noindent The implicitly defined \isa{{\isacharquery}case} refers to the
corresponding case to be proved, i.e.\ \isa{{\isacharquery}P\ {\isadigit{0}}} in the first case and
\isa{{\isacharquery}P{\isacharparenleft}Suc\ n{\isacharparenright}} in the second case. Context \isakeyword{case}~\isa{{\isadigit{0}}} is
empty whereas \isakeyword{case}~\isa{Suc} assumes \isa{{\isacharquery}P\ n}. Again we
have the same problem as with case distinctions: we cannot refer to an anonymous \isa{n}
in the induction step because it has not been introduced via \isakeyword{fix}
(in contrast to the previous proof). The solution is the one outlined for
\isa{Cons} above: replace \isa{Suc} by \isa{{\isacharparenleft}Suc\ i{\isacharparenright}}:%
\end{isamarkuptext}%
\isamarkuptrue%
\isacommand{lemma}\isamarkupfalse%
\ \isakeyword{fixes}\ n{\isacharcolon}{\isacharcolon}nat\ \isakeyword{shows}\ {\isachardoublequoteopen}n\ {\isacharless}\ n{\isacharasterisk}n\ {\isacharplus}\ {\isadigit{1}}{\isachardoublequoteclose}\isanewline
%
\isadelimproof
%
\endisadelimproof
%
\isatagproof
\isacommand{proof}\isamarkupfalse%
\ {\isacharparenleft}induct\ n{\isacharparenright}\isanewline
\ \ \isacommand{case}\isamarkupfalse%
\ {\isadigit{0}}\ \isacommand{show}\isamarkupfalse%
\ {\isacharquery}case\ \isacommand{by}\isamarkupfalse%
\ simp\isanewline
\isacommand{next}\isamarkupfalse%
\isanewline
\ \ \isacommand{case}\isamarkupfalse%
\ {\isacharparenleft}Suc\ i{\isacharparenright}\ \isacommand{thus}\isamarkupfalse%
\ {\isachardoublequoteopen}Suc\ i\ {\isacharless}\ Suc\ i\ {\isacharasterisk}\ Suc\ i\ {\isacharplus}\ {\isadigit{1}}{\isachardoublequoteclose}\ \isacommand{by}\isamarkupfalse%
\ simp\isanewline
\isacommand{qed}\isamarkupfalse%
%
\endisatagproof
{\isafoldproof}%
%
\isadelimproof
%
\endisadelimproof
%
\begin{isamarkuptext}%
\noindent Of course we could again have written
\isakeyword{thus}~\isa{{\isacharquery}case} instead of giving the term explicitly
but we wanted to use \isa{i} somewhere.%
\end{isamarkuptext}%
\isamarkuptrue%
%
\isamarkupsubsection{Induction formulae involving \isa{{\isasymAnd}} or \isa{{\isasymLongrightarrow}}\label{sec:full-Ind}%
}
\isamarkuptrue%
%
\begin{isamarkuptext}%
Let us now consider the situation where the goal to be proved contains
\isa{{\isasymAnd}} or \isa{{\isasymLongrightarrow}}, say \isa{{\isasymAnd}x{\isachardot}\ P\ x\ {\isasymLongrightarrow}\ Q\ x} --- motivation and a
real example follow shortly.  This means that in each case of the induction,
\isa{{\isacharquery}case} would be of the form \isa{{\isasymAnd}x{\isachardot}\ P{\isacharprime}\ x\ {\isasymLongrightarrow}\ Q{\isacharprime}\ x}.  Thus the
first proof steps will be the canonical ones, fixing \isa{x} and assuming
\isa{P{\isacharprime}\ x}. To avoid this tedium, induction performs these steps
automatically: for example in case \isa{{\isacharparenleft}Suc\ n{\isacharparenright}}, \isa{{\isacharquery}case} is only
\isa{Q{\isacharprime}\ x} whereas the assumptions (named \isa{Suc}!) contain both the
usual induction hypothesis \emph{and} \isa{P{\isacharprime}\ x}.
It should be clear how this generalises to more complex formulae.

As an example we will now prove complete induction via
structural induction.%
\end{isamarkuptext}%
\isamarkuptrue%
\isacommand{lemma}\isamarkupfalse%
\ \isakeyword{assumes}\ A{\isacharcolon}\ {\isachardoublequoteopen}{\isacharparenleft}{\isasymAnd}n{\isachardot}\ {\isacharparenleft}{\isasymAnd}m{\isachardot}\ m\ {\isacharless}\ n\ {\isasymLongrightarrow}\ P\ m{\isacharparenright}\ {\isasymLongrightarrow}\ P\ n{\isacharparenright}{\isachardoublequoteclose}\isanewline
\ \ \isakeyword{shows}\ {\isachardoublequoteopen}P{\isacharparenleft}n{\isacharcolon}{\isacharcolon}nat{\isacharparenright}{\isachardoublequoteclose}\isanewline
%
\isadelimproof
%
\endisadelimproof
%
\isatagproof
\isacommand{proof}\isamarkupfalse%
\ {\isacharparenleft}rule\ A{\isacharparenright}\isanewline
\ \ \isacommand{show}\isamarkupfalse%
\ {\isachardoublequoteopen}{\isasymAnd}m{\isachardot}\ m\ {\isacharless}\ n\ {\isasymLongrightarrow}\ P\ m{\isachardoublequoteclose}\isanewline
\ \ \isacommand{proof}\isamarkupfalse%
\ {\isacharparenleft}induct\ n{\isacharparenright}\isanewline
\ \ \ \ \isacommand{case}\isamarkupfalse%
\ {\isadigit{0}}\ \isacommand{thus}\isamarkupfalse%
\ {\isacharquery}case\ \isacommand{by}\isamarkupfalse%
\ simp\isanewline
\ \ \isacommand{next}\isamarkupfalse%
\isanewline
\ \ \ \ \isacommand{case}\isamarkupfalse%
\ {\isacharparenleft}Suc\ n{\isacharparenright}\ \ \ %
\isamarkupcmt{\isakeyword{fix} \isa{m} \isakeyword{assume} \isa{Suc}: \isa{{\isachardoublequote}{\isacharquery}m\ {\isacharless}\ n\ {\isasymLongrightarrow}\ P\ {\isacharquery}m{\isachardoublequote}} \isa{{\isachardoublequote}m\ {\isacharless}\ Suc\ n{\isachardoublequote}}%
}
\isanewline
\ \ \ \ \isacommand{show}\isamarkupfalse%
\ {\isacharquery}case\ \ \ \ %
\isamarkupcmt{\isa{P\ m}%
}
\isanewline
\ \ \ \ \isacommand{proof}\isamarkupfalse%
\ cases\isanewline
\ \ \ \ \ \ \isacommand{assume}\isamarkupfalse%
\ eq{\isacharcolon}\ {\isachardoublequoteopen}m\ {\isacharequal}\ n{\isachardoublequoteclose}\isanewline
\ \ \ \ \ \ \isacommand{from}\isamarkupfalse%
\ Suc\ \isakeyword{and}\ A\ \isacommand{have}\isamarkupfalse%
\ {\isachardoublequoteopen}P\ n{\isachardoublequoteclose}\ \isacommand{by}\isamarkupfalse%
\ blast\isanewline
\ \ \ \ \ \ \isacommand{with}\isamarkupfalse%
\ eq\ \isacommand{show}\isamarkupfalse%
\ {\isachardoublequoteopen}P\ m{\isachardoublequoteclose}\ \isacommand{by}\isamarkupfalse%
\ simp\isanewline
\ \ \ \ \isacommand{next}\isamarkupfalse%
\isanewline
\ \ \ \ \ \ \isacommand{assume}\isamarkupfalse%
\ {\isachardoublequoteopen}m\ {\isasymnoteq}\ n{\isachardoublequoteclose}\isanewline
\ \ \ \ \ \ \isacommand{with}\isamarkupfalse%
\ Suc\ \isacommand{have}\isamarkupfalse%
\ {\isachardoublequoteopen}m\ {\isacharless}\ n{\isachardoublequoteclose}\ \isacommand{by}\isamarkupfalse%
\ arith\isanewline
\ \ \ \ \ \ \isacommand{thus}\isamarkupfalse%
\ {\isachardoublequoteopen}P\ m{\isachardoublequoteclose}\ \isacommand{by}\isamarkupfalse%
{\isacharparenleft}rule\ Suc{\isacharparenright}\isanewline
\ \ \ \ \isacommand{qed}\isamarkupfalse%
\isanewline
\ \ \isacommand{qed}\isamarkupfalse%
\isanewline
\isacommand{qed}\isamarkupfalse%
%
\endisatagproof
{\isafoldproof}%
%
\isadelimproof
%
\endisadelimproof
%
\begin{isamarkuptext}%
\noindent Given the explanations above and the comments in the
proof text (only necessary for novices), the proof should be quite
readable.

The statement of the lemma is interesting because it deviates from the style in
the Tutorial~\cite{LNCS2283}, which suggests to introduce \isa{{\isasymforall}} or
\isa{{\isasymlongrightarrow}} into a theorem to strengthen it for induction. In Isar
proofs we can use \isa{{\isasymAnd}} and \isa{{\isasymLongrightarrow}} instead. This simplifies the
proof and means we do not have to convert between the two kinds of
connectives.

Note that in a nested induction over the same data type, the inner
case labels hide the outer ones of the same name. If you want to refer
to the outer ones inside, you need to name them on the outside, e.g.\
\isakeyword{note}~\isa{outer{\isacharunderscore}IH\ {\isacharequal}\ Suc}.%
\end{isamarkuptext}%
\isamarkuptrue%
%
\isamarkupsubsection{Rule induction%
}
\isamarkuptrue%
%
\begin{isamarkuptext}%
HOL also supports inductively defined sets. See \cite{LNCS2283}
for details. As an example we define our own version of the reflexive
transitive closure of a relation --- HOL provides a predefined one as well.%
\end{isamarkuptext}%
\isamarkuptrue%
\isacommand{consts}\isamarkupfalse%
\ rtc\ {\isacharcolon}{\isacharcolon}\ {\isachardoublequoteopen}{\isacharparenleft}{\isacharprime}a\ {\isasymtimes}\ {\isacharprime}a{\isacharparenright}set\ {\isasymRightarrow}\ {\isacharparenleft}{\isacharprime}a\ {\isasymtimes}\ {\isacharprime}a{\isacharparenright}set{\isachardoublequoteclose}\ \ \ {\isacharparenleft}{\isachardoublequoteopen}{\isacharunderscore}{\isacharasterisk}{\isachardoublequoteclose}\ {\isacharbrackleft}{\isadigit{1}}{\isadigit{0}}{\isadigit{0}}{\isadigit{0}}{\isacharbrackright}\ {\isadigit{9}}{\isadigit{9}}{\isadigit{9}}{\isacharparenright}\isanewline
\isacommand{inductive}\isamarkupfalse%
\ {\isachardoublequoteopen}r{\isacharasterisk}{\isachardoublequoteclose}\isanewline
\isakeyword{intros}\isanewline
refl{\isacharcolon}\ \ {\isachardoublequoteopen}{\isacharparenleft}x{\isacharcomma}x{\isacharparenright}\ {\isasymin}\ r{\isacharasterisk}{\isachardoublequoteclose}\isanewline
step{\isacharcolon}\ \ {\isachardoublequoteopen}{\isasymlbrakk}\ {\isacharparenleft}x{\isacharcomma}y{\isacharparenright}\ {\isasymin}\ r{\isacharsemicolon}\ {\isacharparenleft}y{\isacharcomma}z{\isacharparenright}\ {\isasymin}\ r{\isacharasterisk}\ {\isasymrbrakk}\ {\isasymLongrightarrow}\ {\isacharparenleft}x{\isacharcomma}z{\isacharparenright}\ {\isasymin}\ r{\isacharasterisk}{\isachardoublequoteclose}%
\begin{isamarkuptext}%
\noindent
First the constant is declared as a function on binary
relations (with concrete syntax \isa{r{\isacharasterisk}} instead of \isa{rtc\ r}), then the defining clauses are given. We will now prove that
\isa{r{\isacharasterisk}} is indeed transitive:%
\end{isamarkuptext}%
\isamarkuptrue%
\isacommand{lemma}\isamarkupfalse%
\ \isakeyword{assumes}\ A{\isacharcolon}\ {\isachardoublequoteopen}{\isacharparenleft}x{\isacharcomma}y{\isacharparenright}\ {\isasymin}\ r{\isacharasterisk}{\isachardoublequoteclose}\ \isakeyword{shows}\ {\isachardoublequoteopen}{\isacharparenleft}y{\isacharcomma}z{\isacharparenright}\ {\isasymin}\ r{\isacharasterisk}\ {\isasymLongrightarrow}\ {\isacharparenleft}x{\isacharcomma}z{\isacharparenright}\ {\isasymin}\ r{\isacharasterisk}{\isachardoublequoteclose}\isanewline
%
\isadelimproof
%
\endisadelimproof
%
\isatagproof
\isacommand{using}\isamarkupfalse%
\ A\isanewline
\isacommand{proof}\isamarkupfalse%
\ induct\isanewline
\ \ \isacommand{case}\isamarkupfalse%
\ refl\ \isacommand{thus}\isamarkupfalse%
\ {\isacharquery}case\ \isacommand{{\isachardot}}\isamarkupfalse%
\isanewline
\isacommand{next}\isamarkupfalse%
\isanewline
\ \ \isacommand{case}\isamarkupfalse%
\ step\ \isacommand{thus}\isamarkupfalse%
\ {\isacharquery}case\ \isacommand{by}\isamarkupfalse%
{\isacharparenleft}blast\ intro{\isacharcolon}\ rtc{\isachardot}step{\isacharparenright}\isanewline
\isacommand{qed}\isamarkupfalse%
%
\endisatagproof
{\isafoldproof}%
%
\isadelimproof
%
\endisadelimproof
%
\begin{isamarkuptext}%
\noindent Rule induction is triggered by a fact $(x_1,\dots,x_n)
\in R$ piped into the proof, here \isakeyword{using}~\isa{A}. The
proof itself follows the inductive definition very
closely: there is one case for each rule, and it has the same name as
the rule, analogous to structural induction.

However, this proof is rather terse. Here is a more readable version:%
\end{isamarkuptext}%
\isamarkuptrue%
\isacommand{lemma}\isamarkupfalse%
\ \isakeyword{assumes}\ A{\isacharcolon}\ {\isachardoublequoteopen}{\isacharparenleft}x{\isacharcomma}y{\isacharparenright}\ {\isasymin}\ r{\isacharasterisk}{\isachardoublequoteclose}\ \isakeyword{and}\ B{\isacharcolon}\ {\isachardoublequoteopen}{\isacharparenleft}y{\isacharcomma}z{\isacharparenright}\ {\isasymin}\ r{\isacharasterisk}{\isachardoublequoteclose}\isanewline
\ \ \isakeyword{shows}\ {\isachardoublequoteopen}{\isacharparenleft}x{\isacharcomma}z{\isacharparenright}\ {\isasymin}\ r{\isacharasterisk}{\isachardoublequoteclose}\isanewline
%
\isadelimproof
%
\endisadelimproof
%
\isatagproof
\isacommand{proof}\isamarkupfalse%
\ {\isacharminus}\isanewline
\ \ \isacommand{from}\isamarkupfalse%
\ A\ B\ \isacommand{show}\isamarkupfalse%
\ {\isacharquery}thesis\isanewline
\ \ \isacommand{proof}\isamarkupfalse%
\ induct\isanewline
\ \ \ \ \isacommand{fix}\isamarkupfalse%
\ x\ \isacommand{assume}\isamarkupfalse%
\ {\isachardoublequoteopen}{\isacharparenleft}x{\isacharcomma}z{\isacharparenright}\ {\isasymin}\ r{\isacharasterisk}{\isachardoublequoteclose}\ \ %
\isamarkupcmt{\isa{B}[\isa{y} := \isa{x}]%
}
\isanewline
\ \ \ \ \isacommand{thus}\isamarkupfalse%
\ {\isachardoublequoteopen}{\isacharparenleft}x{\isacharcomma}z{\isacharparenright}\ {\isasymin}\ r{\isacharasterisk}{\isachardoublequoteclose}\ \isacommand{{\isachardot}}\isamarkupfalse%
\isanewline
\ \ \isacommand{next}\isamarkupfalse%
\isanewline
\ \ \ \ \isacommand{fix}\isamarkupfalse%
\ x{\isacharprime}\ x\ y\isanewline
\ \ \ \ \isacommand{assume}\isamarkupfalse%
\ {\isadigit{1}}{\isacharcolon}\ {\isachardoublequoteopen}{\isacharparenleft}x{\isacharprime}{\isacharcomma}x{\isacharparenright}\ {\isasymin}\ r{\isachardoublequoteclose}\ \isakeyword{and}\isanewline
\ \ \ \ \ \ \ \ \ \ \ IH{\isacharcolon}\ {\isachardoublequoteopen}{\isacharparenleft}y{\isacharcomma}z{\isacharparenright}\ {\isasymin}\ r{\isacharasterisk}\ {\isasymLongrightarrow}\ {\isacharparenleft}x{\isacharcomma}z{\isacharparenright}\ {\isasymin}\ r{\isacharasterisk}{\isachardoublequoteclose}\ \isakeyword{and}\isanewline
\ \ \ \ \ \ \ \ \ \ \ B{\isacharcolon}\ \ {\isachardoublequoteopen}{\isacharparenleft}y{\isacharcomma}z{\isacharparenright}\ {\isasymin}\ r{\isacharasterisk}{\isachardoublequoteclose}\isanewline
\ \ \ \ \isacommand{from}\isamarkupfalse%
\ {\isadigit{1}}\ IH{\isacharbrackleft}OF\ B{\isacharbrackright}\ \isacommand{show}\isamarkupfalse%
\ {\isachardoublequoteopen}{\isacharparenleft}x{\isacharprime}{\isacharcomma}z{\isacharparenright}\ {\isasymin}\ r{\isacharasterisk}{\isachardoublequoteclose}\ \isacommand{by}\isamarkupfalse%
{\isacharparenleft}rule\ rtc{\isachardot}step{\isacharparenright}\isanewline
\ \ \isacommand{qed}\isamarkupfalse%
\isanewline
\isacommand{qed}\isamarkupfalse%
%
\endisatagproof
{\isafoldproof}%
%
\isadelimproof
%
\endisadelimproof
%
\begin{isamarkuptext}%
\noindent We start the proof with \isakeyword{from}~\isa{A\ B}. Only \isa{A} is ``consumed'' by the induction step.
Since \isa{B} is left over we don't just prove \isa{{\isacharquery}thesis} but \isa{B\ {\isasymLongrightarrow}\ {\isacharquery}thesis}, just as in the previous proof. The
base case is trivial. In the assumptions for the induction step we can
see very clearly how things fit together and permit ourselves the
obvious forward step \isa{IH{\isacharbrackleft}OF\ B{\isacharbrackright}}.

The notation `\isakeyword{case}~\isa{(}\emph{constructor} \emph{vars}\isa{)}'
is also supported for inductive definitions. The \emph{constructor} is (the
name of) the rule and the \emph{vars} fix the free variables in the
rule; the order of the \emph{vars} must correspond to the
\emph{alphabetical order} of the variables as they appear in the rule.
For example, we could start the above detailed proof of the induction
with \isakeyword{case}~\isa{(step x' x y)}. However, we can then only
refer to the assumptions named \isa{step} collectively and not
individually, as the above proof requires.%
\end{isamarkuptext}%
\isamarkuptrue%
%
\isamarkupsubsection{More induction%
}
\isamarkuptrue%
%
\begin{isamarkuptext}%
We close the section by demonstrating how arbitrary induction
rules are applied. As a simple example we have chosen recursion
induction, i.e.\ induction based on a recursive function
definition. However, most of what we show works for induction in
general.

The example is an unusual definition of rotation:%
\end{isamarkuptext}%
\isamarkuptrue%
\isacommand{consts}\isamarkupfalse%
\ rot\ {\isacharcolon}{\isacharcolon}\ {\isachardoublequoteopen}{\isacharprime}a\ list\ {\isasymRightarrow}\ {\isacharprime}a\ list{\isachardoublequoteclose}\isanewline
\isacommand{recdef}\isamarkupfalse%
\ rot\ {\isachardoublequoteopen}measure\ length{\isachardoublequoteclose}\ \ %
\isamarkupcmt{for the internal termination proof%
}
\isanewline
{\isachardoublequoteopen}rot\ {\isacharbrackleft}{\isacharbrackright}\ {\isacharequal}\ {\isacharbrackleft}{\isacharbrackright}{\isachardoublequoteclose}\isanewline
{\isachardoublequoteopen}rot\ {\isacharbrackleft}x{\isacharbrackright}\ {\isacharequal}\ {\isacharbrackleft}x{\isacharbrackright}{\isachardoublequoteclose}\isanewline
{\isachardoublequoteopen}rot\ {\isacharparenleft}x{\isacharhash}y{\isacharhash}zs{\isacharparenright}\ {\isacharequal}\ y\ {\isacharhash}\ rot{\isacharparenleft}x{\isacharhash}zs{\isacharparenright}{\isachardoublequoteclose}%
\begin{isamarkuptext}%
\noindent This yields, among other things, the induction rule
\isa{rot{\isachardot}induct}: \begin{isabelle}%
{\isasymlbrakk}P\ {\isacharbrackleft}{\isacharbrackright}{\isacharsemicolon}\ {\isasymAnd}x{\isachardot}\ P\ {\isacharbrackleft}x{\isacharbrackright}{\isacharsemicolon}\ {\isasymAnd}x\ y\ zs{\isachardot}\ P\ {\isacharparenleft}x\ {\isacharhash}\ zs{\isacharparenright}\ {\isasymLongrightarrow}\ P\ {\isacharparenleft}x\ {\isacharhash}\ y\ {\isacharhash}\ zs{\isacharparenright}{\isasymrbrakk}\ {\isasymLongrightarrow}\ P\ x%
\end{isabelle}
In the following proof we rely on a default naming scheme for cases: they are
called 1, 2, etc, unless they have been named explicitly. The latter happens
only with datatypes and inductively defined sets, but not with recursive
functions.%
\end{isamarkuptext}%
\isamarkuptrue%
\isacommand{lemma}\isamarkupfalse%
\ {\isachardoublequoteopen}xs\ {\isasymnoteq}\ {\isacharbrackleft}{\isacharbrackright}\ {\isasymLongrightarrow}\ rot\ xs\ {\isacharequal}\ tl\ xs\ {\isacharat}\ {\isacharbrackleft}hd\ xs{\isacharbrackright}{\isachardoublequoteclose}\isanewline
%
\isadelimproof
%
\endisadelimproof
%
\isatagproof
\isacommand{proof}\isamarkupfalse%
\ {\isacharparenleft}induct\ xs\ rule{\isacharcolon}\ rot{\isachardot}induct{\isacharparenright}\isanewline
\ \ \isacommand{case}\isamarkupfalse%
\ {\isadigit{1}}\ \isacommand{thus}\isamarkupfalse%
\ {\isacharquery}case\ \isacommand{by}\isamarkupfalse%
\ simp\isanewline
\isacommand{next}\isamarkupfalse%
\isanewline
\ \ \isacommand{case}\isamarkupfalse%
\ {\isadigit{2}}\ \isacommand{show}\isamarkupfalse%
\ {\isacharquery}case\ \isacommand{by}\isamarkupfalse%
\ simp\isanewline
\isacommand{next}\isamarkupfalse%
\isanewline
\ \ \isacommand{case}\isamarkupfalse%
\ {\isacharparenleft}{\isadigit{3}}\ a\ b\ cs{\isacharparenright}\isanewline
\ \ \isacommand{have}\isamarkupfalse%
\ {\isachardoublequoteopen}rot\ {\isacharparenleft}a\ {\isacharhash}\ b\ {\isacharhash}\ cs{\isacharparenright}\ {\isacharequal}\ b\ {\isacharhash}\ rot{\isacharparenleft}a\ {\isacharhash}\ cs{\isacharparenright}{\isachardoublequoteclose}\ \isacommand{by}\isamarkupfalse%
\ simp\isanewline
\ \ \isacommand{also}\isamarkupfalse%
\ \isacommand{have}\isamarkupfalse%
\ {\isachardoublequoteopen}{\isasymdots}\ {\isacharequal}\ b\ {\isacharhash}\ tl{\isacharparenleft}a\ {\isacharhash}\ cs{\isacharparenright}\ {\isacharat}\ {\isacharbrackleft}hd{\isacharparenleft}a\ {\isacharhash}\ cs{\isacharparenright}{\isacharbrackright}{\isachardoublequoteclose}\ \isacommand{by}\isamarkupfalse%
{\isacharparenleft}simp\ add{\isacharcolon}{\isadigit{3}}{\isacharparenright}\isanewline
\ \ \isacommand{also}\isamarkupfalse%
\ \isacommand{have}\isamarkupfalse%
\ {\isachardoublequoteopen}{\isasymdots}\ {\isacharequal}\ tl\ {\isacharparenleft}a\ {\isacharhash}\ b\ {\isacharhash}\ cs{\isacharparenright}\ {\isacharat}\ {\isacharbrackleft}hd\ {\isacharparenleft}a\ {\isacharhash}\ b\ {\isacharhash}\ cs{\isacharparenright}{\isacharbrackright}{\isachardoublequoteclose}\ \isacommand{by}\isamarkupfalse%
\ simp\isanewline
\ \ \isacommand{finally}\isamarkupfalse%
\ \isacommand{show}\isamarkupfalse%
\ {\isacharquery}case\ \isacommand{{\isachardot}}\isamarkupfalse%
\isanewline
\isacommand{qed}\isamarkupfalse%
%
\endisatagproof
{\isafoldproof}%
%
\isadelimproof
%
\endisadelimproof
%
\begin{isamarkuptext}%
\noindent
The third case is only shown in gory detail (see \cite{BauerW-TPHOLs01}
for how to reason with chains of equations) to demonstrate that the
`\isakeyword{case}~\isa{(}\emph{constructor} \emph{vars}\isa{)}' notation also
works for arbitrary induction theorems with numbered cases. The order
of the \emph{vars} corresponds to the order of the
\isa{{\isasymAnd}}-quantified variables in each case of the induction
theorem. For induction theorems produced by \isakeyword{recdef} it is
the order in which the variables appear on the left-hand side of the
equation.

The proof is so simple that it can be condensed to%
\end{isamarkuptext}%
\isamarkuptrue%
\isamarkupfalse%
%
\isadelimproof
%
\endisadelimproof
%
\isatagproof
\isacommand{by}\isamarkupfalse%
\ {\isacharparenleft}induct\ xs\ rule{\isacharcolon}\ rot{\isachardot}induct{\isacharcomma}\ simp{\isacharunderscore}all{\isacharparenright}\isanewline
%
\endisatagproof
{\isafoldproof}%
%
\isadelimproof
%
\endisadelimproof
%
\isadelimtheory
%
\endisadelimtheory
%
\isatagtheory
\isamarkupfalse%
%
\endisatagtheory
{\isafoldtheory}%
%
\isadelimtheory
%
\endisadelimtheory
\end{isabellebody}%
%%% Local Variables:
%%% mode: latex
%%% TeX-master: "root"
%%% End:


%%
\begin{isabellebody}%
\def\isabellecontext{Isar}%
%
\isadelimtheory
%
\endisadelimtheory
%
\isatagtheory
%
\endisatagtheory
{\isafoldtheory}%
%
\isadelimtheory
%
\endisadelimtheory
%
\isadelimML
%
\endisadelimML
%
\isatagML
%
\endisatagML
{\isafoldML}%
%
\isadelimML
%
\endisadelimML
%
\begin{isamarkuptext}%
Apply-scripts are unreadable and hard to maintain. The language of choice
for larger proofs is \concept{Isar}. The two key features of Isar are:
\begin{itemize}
\item It is structured, not linear.
\item It is readable without running it because
you need to state what you are proving at any given point.
\end{itemize}
Whereas apply-scripts are like assembly language programs, Isar proofs
are like structured programs with comments. A typical Isar proof looks like this:%
\end{isamarkuptext}%
\isamarkuptrue%
%
\begin{isamarkuptext}%
\begin{tabular}{@ {}l}
\isacom{proof}\\
\quad\isacom{assume} \isa{{\isaliteral{22}{\isachardoublequote}}}$\mathit{formula}_0$\isa{{\isaliteral{22}{\isachardoublequote}}}\\
\quad\isacom{have} \isa{{\isaliteral{22}{\isachardoublequote}}}$\mathit{formula}_1$\isa{{\isaliteral{22}{\isachardoublequote}}} \quad\isacom{by} \isa{simp}\\
\quad\vdots\\
\quad\isacom{have} \isa{{\isaliteral{22}{\isachardoublequote}}}$\mathit{formula}_n$\isa{{\isaliteral{22}{\isachardoublequote}}} \quad\isacom{by} \isa{blast}\\
\quad\isacom{show} \isa{{\isaliteral{22}{\isachardoublequote}}}$\mathit{formula}_{n+1}$\isa{{\isaliteral{22}{\isachardoublequote}}} \quad\isacom{by} \isa{{\isaliteral{5C3C646F74733E}{\isasymdots}}}\\
\isacom{qed}
\end{tabular}%
\end{isamarkuptext}%
\isamarkuptrue%
%
\begin{isamarkuptext}%
It proves $\mathit{formula}_0 \Longrightarrow \mathit{formula}_{n+1}$
(provided each proof step succeeds).
The intermediate \isacom{have} statements are merely stepping stones
on the way towards the \isacom{show} statement that proves the actual
goal. In more detail, this is the Isar core syntax:
\medskip

\begin{tabular}{@ {}lcl@ {}}
\textit{proof} &=& \isacom{by} \textit{method}\\
      &$\mid$& \isacom{proof} [\textit{method}] \ \textit{step}$^*$ \ \isacom{qed}
\end{tabular}
\medskip

\begin{tabular}{@ {}lcl@ {}}
\textit{step} &=& \isacom{fix} \textit{variables} \\
      &$\mid$& \isacom{assume} \textit{proposition} \\
      &$\mid$& [\isacom{from} \textit{fact}$^+$] (\isacom{have} $\mid$ \isacom{show}) \ \textit{proposition} \ \textit{proof}
\end{tabular}
\medskip

\begin{tabular}{@ {}lcl@ {}}
\textit{proposition} &=& [\textit{name}:] \isa{{\isaliteral{22}{\isachardoublequote}}}\textit{formula}\isa{{\isaliteral{22}{\isachardoublequote}}}
\end{tabular}
\medskip

\begin{tabular}{@ {}lcl@ {}}
\textit{fact} &=& \textit{name} \ $\mid$ \ \dots
\end{tabular}
\medskip

\noindent A proof can either be an atomic \isacom{by} with a single proof
method which must finish off the statement being proved, for example \isa{auto}.  Or it can be a \isacom{proof}--\isacom{qed} block of multiple
steps. Such a block can optionally begin with a proof method that indicates
how to start off the proof, e.g.\ \mbox{\isa{{\isaliteral{28}{\isacharparenleft}}induction\ xs{\isaliteral{29}{\isacharparenright}}}}.

A step either assumes a proposition or states a proposition
together with its proof. The optional \isacom{from} clause
indicates which facts are to be used in the proof.
Intermediate propositions are stated with \isacom{have}, the overall goal
with \isacom{show}. A step can also introduce new local variables with
\isacom{fix}. Logically, \isacom{fix} introduces \isa{{\isaliteral{5C3C416E643E}{\isasymAnd}}}-quantified
variables, \isacom{assume} introduces the assumption of an implication
(\isa{{\isaliteral{5C3C4C6F6E6772696768746172726F773E}{\isasymLongrightarrow}}}) and \isacom{have}/\isacom{show} the conclusion.

Propositions are optionally named formulas. These names can be referred to in
later \isacom{from} clauses. In the simplest case, a fact is such a name.
But facts can also be composed with \isa{OF} and \isa{of} as shown in
\S\ref{sec:forward-proof}---hence the \dots\ in the above grammar.  Note
that assumptions, intermediate \isacom{have} statements and global lemmas all
have the same status and are thus collectively referred to as
\concept{facts}.

Fact names can stand for whole lists of facts. For example, if \isa{f} is
defined by command \isacom{fun}, \isa{f{\isaliteral{2E}{\isachardot}}simps} refers to the whole list of
recursion equations defining \isa{f}. Individual facts can be selected by
writing \isa{f{\isaliteral{2E}{\isachardot}}simps{\isaliteral{28}{\isacharparenleft}}{\isadigit{2}}{\isaliteral{29}{\isacharparenright}}}, whole sublists by \isa{f{\isaliteral{2E}{\isachardot}}simps{\isaliteral{28}{\isacharparenleft}}{\isadigit{2}}{\isaliteral{2D}{\isacharminus}}{\isadigit{4}}{\isaliteral{29}{\isacharparenright}}}.


\section{Isar by example}

We show a number of proofs of Cantor's theorem that a function from a set to
its powerset cannot be surjective, illustrating various features of Isar. The
constant \isa{surj} is predefined.%
\end{isamarkuptext}%
\isamarkuptrue%
\isacommand{lemma}\isamarkupfalse%
\ {\isaliteral{22}{\isachardoublequoteopen}}{\isaliteral{5C3C6E6F743E}{\isasymnot}}\ surj{\isaliteral{28}{\isacharparenleft}}f\ {\isaliteral{3A}{\isacharcolon}}{\isaliteral{3A}{\isacharcolon}}\ {\isaliteral{27}{\isacharprime}}a\ {\isaliteral{5C3C52696768746172726F773E}{\isasymRightarrow}}\ {\isaliteral{27}{\isacharprime}}a\ set{\isaliteral{29}{\isacharparenright}}{\isaliteral{22}{\isachardoublequoteclose}}\isanewline
%
\isadelimproof
%
\endisadelimproof
%
\isatagproof
\isacommand{proof}\isamarkupfalse%
\isanewline
\ \ \isacommand{assume}\isamarkupfalse%
\ {\isadigit{0}}{\isaliteral{3A}{\isacharcolon}}\ {\isaliteral{22}{\isachardoublequoteopen}}surj\ f{\isaliteral{22}{\isachardoublequoteclose}}\isanewline
\ \ \isacommand{from}\isamarkupfalse%
\ {\isadigit{0}}\ \isacommand{have}\isamarkupfalse%
\ {\isadigit{1}}{\isaliteral{3A}{\isacharcolon}}\ {\isaliteral{22}{\isachardoublequoteopen}}{\isaliteral{5C3C666F72616C6C3E}{\isasymforall}}A{\isaliteral{2E}{\isachardot}}\ {\isaliteral{5C3C6578697374733E}{\isasymexists}}a{\isaliteral{2E}{\isachardot}}\ A\ {\isaliteral{3D}{\isacharequal}}\ f\ a{\isaliteral{22}{\isachardoublequoteclose}}\ \isacommand{by}\isamarkupfalse%
{\isaliteral{28}{\isacharparenleft}}simp\ add{\isaliteral{3A}{\isacharcolon}}\ surj{\isaliteral{5F}{\isacharunderscore}}def{\isaliteral{29}{\isacharparenright}}\isanewline
\ \ \isacommand{from}\isamarkupfalse%
\ {\isadigit{1}}\ \isacommand{have}\isamarkupfalse%
\ {\isadigit{2}}{\isaliteral{3A}{\isacharcolon}}\ {\isaliteral{22}{\isachardoublequoteopen}}{\isaliteral{5C3C6578697374733E}{\isasymexists}}a{\isaliteral{2E}{\isachardot}}\ {\isaliteral{7B}{\isacharbraceleft}}x{\isaliteral{2E}{\isachardot}}\ x\ {\isaliteral{5C3C6E6F74696E3E}{\isasymnotin}}\ f\ x{\isaliteral{7D}{\isacharbraceright}}\ {\isaliteral{3D}{\isacharequal}}\ f\ a{\isaliteral{22}{\isachardoublequoteclose}}\ \isacommand{by}\isamarkupfalse%
\ blast\isanewline
\ \ \isacommand{from}\isamarkupfalse%
\ {\isadigit{2}}\ \isacommand{show}\isamarkupfalse%
\ {\isaliteral{22}{\isachardoublequoteopen}}False{\isaliteral{22}{\isachardoublequoteclose}}\ \isacommand{by}\isamarkupfalse%
\ blast\isanewline
\isacommand{qed}\isamarkupfalse%
%
\endisatagproof
{\isafoldproof}%
%
\isadelimproof
%
\endisadelimproof
%
\begin{isamarkuptext}%
The \isacom{proof} command lacks an explicit method how to perform
the proof. In such cases Isabelle tries to use some standard introduction
rule, in the above case for \isa{{\isaliteral{5C3C6E6F743E}{\isasymnot}}}:
\[
\inferrule{
\mbox{\isa{P\ {\isaliteral{5C3C4C6F6E6772696768746172726F773E}{\isasymLongrightarrow}}\ False}}}
{\mbox{\isa{{\isaliteral{5C3C6E6F743E}{\isasymnot}}\ P}}}
\]
In order to prove \isa{{\isaliteral{5C3C6E6F743E}{\isasymnot}}\ P}, assume \isa{P} and show \isa{False}.
Thus we may assume \isa{surj\ f}. The proof shows that names of propositions
may be (single!) digits---meaningful names are hard to invent and are often
not necessary. Both \isacom{have} steps are obvious. The second one introduces
the diagonal set \isa{{\isaliteral{7B}{\isacharbraceleft}}x{\isaliteral{2E}{\isachardot}}\ x\ {\isaliteral{5C3C6E6F74696E3E}{\isasymnotin}}\ f\ x{\isaliteral{7D}{\isacharbraceright}}}, the key idea in the proof.
If you wonder why \isa{{\isadigit{2}}} directly implies \isa{False}: from \isa{{\isadigit{2}}}
it follows that \isa{{\isaliteral{28}{\isacharparenleft}}a\ {\isaliteral{5C3C6E6F74696E3E}{\isasymnotin}}\ f\ a{\isaliteral{29}{\isacharparenright}}\ {\isaliteral{3D}{\isacharequal}}\ {\isaliteral{28}{\isacharparenleft}}a\ {\isaliteral{5C3C696E3E}{\isasymin}}\ f\ a{\isaliteral{29}{\isacharparenright}}}.

\subsection{\isa{this}, \isa{then}, \isa{hence} and \isa{thus}}

Labels should be avoided. They interrupt the flow of the reader who has to
scan the context for the point where the label was introduced. Ideally, the
proof is a linear flow, where the output of one step becomes the input of the
next step, piping the previously proved fact into the next proof, just like
in a UNIX pipe. In such cases the predefined name \isa{this} can be used
to refer to the proposition proved in the previous step. This allows us to
eliminate all labels from our proof (we suppress the \isacom{lemma} statement):%
\end{isamarkuptext}%
\isamarkuptrue%
%
\isadelimproof
%
\endisadelimproof
%
\isatagproof
\isacommand{proof}\isamarkupfalse%
\isanewline
\ \ \isacommand{assume}\isamarkupfalse%
\ {\isaliteral{22}{\isachardoublequoteopen}}surj\ f{\isaliteral{22}{\isachardoublequoteclose}}\isanewline
\ \ \isacommand{from}\isamarkupfalse%
\ this\ \isacommand{have}\isamarkupfalse%
\ {\isaliteral{22}{\isachardoublequoteopen}}{\isaliteral{5C3C6578697374733E}{\isasymexists}}a{\isaliteral{2E}{\isachardot}}\ {\isaliteral{7B}{\isacharbraceleft}}x{\isaliteral{2E}{\isachardot}}\ x\ {\isaliteral{5C3C6E6F74696E3E}{\isasymnotin}}\ f\ x{\isaliteral{7D}{\isacharbraceright}}\ {\isaliteral{3D}{\isacharequal}}\ f\ a{\isaliteral{22}{\isachardoublequoteclose}}\ \isacommand{by}\isamarkupfalse%
{\isaliteral{28}{\isacharparenleft}}auto\ simp{\isaliteral{3A}{\isacharcolon}}\ surj{\isaliteral{5F}{\isacharunderscore}}def{\isaliteral{29}{\isacharparenright}}\isanewline
\ \ \isacommand{from}\isamarkupfalse%
\ this\ \isacommand{show}\isamarkupfalse%
\ {\isaliteral{22}{\isachardoublequoteopen}}False{\isaliteral{22}{\isachardoublequoteclose}}\ \isacommand{by}\isamarkupfalse%
\ blast\isanewline
\isacommand{qed}\isamarkupfalse%
%
\endisatagproof
{\isafoldproof}%
%
\isadelimproof
%
\endisadelimproof
%
\begin{isamarkuptext}%
We have also taken the opportunity to compress the two \isacom{have}
steps into one.

To compact the text further, Isar has a few convenient abbreviations:
\medskip

\begin{tabular}{rcl}
\isacom{then} &=& \isacom{from} \isa{this}\\
\isacom{thus} &=& \isacom{then} \isacom{show}\\
\isacom{hence} &=& \isacom{then} \isacom{have}
\end{tabular}
\medskip

\noindent
With the help of these abbreviations the proof becomes%
\end{isamarkuptext}%
\isamarkuptrue%
%
\isadelimproof
%
\endisadelimproof
%
\isatagproof
\isacommand{proof}\isamarkupfalse%
\isanewline
\ \ \isacommand{assume}\isamarkupfalse%
\ {\isaliteral{22}{\isachardoublequoteopen}}surj\ f{\isaliteral{22}{\isachardoublequoteclose}}\isanewline
\ \ \isacommand{hence}\isamarkupfalse%
\ {\isaliteral{22}{\isachardoublequoteopen}}{\isaliteral{5C3C6578697374733E}{\isasymexists}}a{\isaliteral{2E}{\isachardot}}\ {\isaliteral{7B}{\isacharbraceleft}}x{\isaliteral{2E}{\isachardot}}\ x\ {\isaliteral{5C3C6E6F74696E3E}{\isasymnotin}}\ f\ x{\isaliteral{7D}{\isacharbraceright}}\ {\isaliteral{3D}{\isacharequal}}\ f\ a{\isaliteral{22}{\isachardoublequoteclose}}\ \isacommand{by}\isamarkupfalse%
{\isaliteral{28}{\isacharparenleft}}auto\ simp{\isaliteral{3A}{\isacharcolon}}\ surj{\isaliteral{5F}{\isacharunderscore}}def{\isaliteral{29}{\isacharparenright}}\isanewline
\ \ \isacommand{thus}\isamarkupfalse%
\ {\isaliteral{22}{\isachardoublequoteopen}}False{\isaliteral{22}{\isachardoublequoteclose}}\ \isacommand{by}\isamarkupfalse%
\ blast\isanewline
\isacommand{qed}\isamarkupfalse%
%
\endisatagproof
{\isafoldproof}%
%
\isadelimproof
%
\endisadelimproof
%
\begin{isamarkuptext}%
There are two further linguistic variations:
\medskip

\begin{tabular}{rcl}
(\isacom{have}$\mid$\isacom{show}) \ \textit{prop} \ \isacom{using} \ \textit{facts}
&=&
\isacom{from} \ \textit{facts} \ (\isacom{have}$\mid$\isacom{show}) \ \textit{prop}\\
\isacom{with} \ \textit{facts} &=& \isacom{from} \ \textit{facts} \isa{this}
\end{tabular}
\medskip

\noindent The \isacom{using} idiom de-emphasises the used facts by moving them
behind the proposition.

\subsection{Structured lemma statements: \isacom{fixes}, \isacom{assumes}, \isacom{shows}}

Lemmas can also be stated in a more structured fashion. To demonstrate this
feature with Cantor's theorem, we rephrase \isa{{\isaliteral{5C3C6E6F743E}{\isasymnot}}\ surj\ f}
a little:%
\end{isamarkuptext}%
\isamarkuptrue%
\isacommand{lemma}\isamarkupfalse%
\isanewline
\ \ \isakeyword{fixes}\ f\ {\isaliteral{3A}{\isacharcolon}}{\isaliteral{3A}{\isacharcolon}}\ {\isaliteral{22}{\isachardoublequoteopen}}{\isaliteral{27}{\isacharprime}}a\ {\isaliteral{5C3C52696768746172726F773E}{\isasymRightarrow}}\ {\isaliteral{27}{\isacharprime}}a\ set{\isaliteral{22}{\isachardoublequoteclose}}\isanewline
\ \ \isakeyword{assumes}\ s{\isaliteral{3A}{\isacharcolon}}\ {\isaliteral{22}{\isachardoublequoteopen}}surj\ f{\isaliteral{22}{\isachardoublequoteclose}}\isanewline
\ \ \isakeyword{shows}\ {\isaliteral{22}{\isachardoublequoteopen}}False{\isaliteral{22}{\isachardoublequoteclose}}%
\isadelimproof
%
\endisadelimproof
%
\isatagproof
%
\begin{isamarkuptxt}%
The optional \isacom{fixes} part allows you to state the types of
variables up front rather than by decorating one of their occurrences in the
formula with a type constraint. The key advantage of the structured format is
the \isacom{assumes} part that allows you to name each assumption; multiple
assumptions can be separated by \isacom{and}. The
\isacom{shows} part gives the goal. The actual theorem that will come out of
the proof is \isa{surj\ f\ {\isaliteral{5C3C4C6F6E6772696768746172726F773E}{\isasymLongrightarrow}}\ False}, but during the proof the assumption
\isa{surj\ f} is available under the name \isa{s} like any other fact.%
\end{isamarkuptxt}%
\isamarkuptrue%
\isacommand{proof}\isamarkupfalse%
\ {\isaliteral{2D}{\isacharminus}}\isanewline
\ \ \isacommand{have}\isamarkupfalse%
\ {\isaliteral{22}{\isachardoublequoteopen}}{\isaliteral{5C3C6578697374733E}{\isasymexists}}\ a{\isaliteral{2E}{\isachardot}}\ {\isaliteral{7B}{\isacharbraceleft}}x{\isaliteral{2E}{\isachardot}}\ x\ {\isaliteral{5C3C6E6F74696E3E}{\isasymnotin}}\ f\ x{\isaliteral{7D}{\isacharbraceright}}\ {\isaliteral{3D}{\isacharequal}}\ f\ a{\isaliteral{22}{\isachardoublequoteclose}}\ \isacommand{using}\isamarkupfalse%
\ s\isanewline
\ \ \ \ \isacommand{by}\isamarkupfalse%
{\isaliteral{28}{\isacharparenleft}}auto\ simp{\isaliteral{3A}{\isacharcolon}}\ surj{\isaliteral{5F}{\isacharunderscore}}def{\isaliteral{29}{\isacharparenright}}\isanewline
\ \ \isacommand{thus}\isamarkupfalse%
\ {\isaliteral{22}{\isachardoublequoteopen}}False{\isaliteral{22}{\isachardoublequoteclose}}\ \isacommand{by}\isamarkupfalse%
\ blast\isanewline
\isacommand{qed}\isamarkupfalse%
%
\endisatagproof
{\isafoldproof}%
%
\isadelimproof
%
\endisadelimproof
%
\begin{isamarkuptext}%
In the \isacom{have} step the assumption \isa{surj\ f} is now
referenced by its name \isa{s}. The duplication of \isa{surj\ f} in the
above proofs (once in the statement of the lemma, once in its proof) has been
eliminated.

\begin{warn}
Note the dash after the \isacom{proof}
command.  It is the null method that does nothing to the goal. Leaving it out
would ask Isabelle to try some suitable introduction rule on the goal \isa{False}---but there is no suitable introduction rule and \isacom{proof}
would fail.
\end{warn}

Stating a lemma with \isacom{assumes}-\isacom{shows} implicitly introduces the
name \isa{assms} that stands for the list of all assumptions. You can refer
to individual assumptions by \isa{assms{\isaliteral{28}{\isacharparenleft}}{\isadigit{1}}{\isaliteral{29}{\isacharparenright}}}, \isa{assms{\isaliteral{28}{\isacharparenleft}}{\isadigit{2}}{\isaliteral{29}{\isacharparenright}}} etc,
thus obviating the need to name them individually.

\section{Proof patterns}

We show a number of important basic proof patterns. Many of them arise from
the rules of natural deduction that are applied by \isacom{proof} by
default. The patterns are phrased in terms of \isacom{show} but work for
\isacom{have} and \isacom{lemma}, too.

We start with two forms of \concept{case distinction}:
starting from a formula \isa{P} we have the two cases \isa{P} and
\isa{{\isaliteral{5C3C6E6F743E}{\isasymnot}}\ P}, and starting from a fact \isa{P\ {\isaliteral{5C3C6F723E}{\isasymor}}\ Q}
we have the two cases \isa{P} and \isa{Q}:%
\end{isamarkuptext}%
\isamarkuptrue%
%
\begin{tabular}{@ {}ll@ {}}
\begin{minipage}[t]{.4\textwidth}
\isa{%
%
\isadelimproof
%
\endisadelimproof
%
\isatagproof
\isacommand{show}\isamarkupfalse%
\ {\isaliteral{22}{\isachardoublequoteopen}}R{\isaliteral{22}{\isachardoublequoteclose}}\isanewline
\isacommand{proof}\isamarkupfalse%
\ cases\isanewline
\ \ \isacommand{assume}\isamarkupfalse%
\ {\isaliteral{22}{\isachardoublequoteopen}}P{\isaliteral{22}{\isachardoublequoteclose}}%
\\\mbox{}\quad$\vdots$\\\mbox{}\hspace{-1.4ex}
\ \ \isacommand{show}\isamarkupfalse%
\ {\isaliteral{22}{\isachardoublequoteopen}}R{\isaliteral{22}{\isachardoublequoteclose}}\ %
\ $\dots$\\
\isacommand{next}\isamarkupfalse%
\isanewline
\ \ \isacommand{assume}\isamarkupfalse%
\ {\isaliteral{22}{\isachardoublequoteopen}}{\isaliteral{5C3C6E6F743E}{\isasymnot}}\ P{\isaliteral{22}{\isachardoublequoteclose}}%
\\\mbox{}\quad$\vdots$\\\mbox{}\hspace{-1.4ex}
\ \ \isacommand{show}\isamarkupfalse%
\ {\isaliteral{22}{\isachardoublequoteopen}}R{\isaliteral{22}{\isachardoublequoteclose}}\ %
\ $\dots$\\
\isacommand{qed}\isamarkupfalse%
%
\endisatagproof
{\isafoldproof}%
%
\isadelimproof
%
\endisadelimproof
%
}
\end{minipage}
&
\begin{minipage}[t]{.4\textwidth}
\isa{%
%
\isadelimproof
%
\endisadelimproof
%
\isatagproof
\isacommand{have}\isamarkupfalse%
\ {\isaliteral{22}{\isachardoublequoteopen}}P\ {\isaliteral{5C3C6F723E}{\isasymor}}\ Q{\isaliteral{22}{\isachardoublequoteclose}}\ %
\ $\dots$\\
\isacommand{then}\isamarkupfalse%
\ \isacommand{show}\isamarkupfalse%
\ {\isaliteral{22}{\isachardoublequoteopen}}R{\isaliteral{22}{\isachardoublequoteclose}}\isanewline
\isacommand{proof}\isamarkupfalse%
\isanewline
\ \ \isacommand{assume}\isamarkupfalse%
\ {\isaliteral{22}{\isachardoublequoteopen}}P{\isaliteral{22}{\isachardoublequoteclose}}%
\\\mbox{}\quad$\vdots$\\\mbox{}\hspace{-1.4ex}
\ \ \isacommand{show}\isamarkupfalse%
\ {\isaliteral{22}{\isachardoublequoteopen}}R{\isaliteral{22}{\isachardoublequoteclose}}\ %
\ $\dots$\\
\isacommand{next}\isamarkupfalse%
\isanewline
\ \ \isacommand{assume}\isamarkupfalse%
\ {\isaliteral{22}{\isachardoublequoteopen}}Q{\isaliteral{22}{\isachardoublequoteclose}}%
\\\mbox{}\quad$\vdots$\\\mbox{}\hspace{-1.4ex}
\ \ \isacommand{show}\isamarkupfalse%
\ {\isaliteral{22}{\isachardoublequoteopen}}R{\isaliteral{22}{\isachardoublequoteclose}}\ %
\ $\dots$\\
\isacommand{qed}\isamarkupfalse%
%
\endisatagproof
{\isafoldproof}%
%
\isadelimproof
%
\endisadelimproof
%
}
\end{minipage}
\end{tabular}
\medskip
\begin{isamarkuptext}%
How to prove a logical equivalence:
\end{isamarkuptext}%
\isa{%
%
\isadelimproof
%
\endisadelimproof
%
\isatagproof
\isacommand{show}\isamarkupfalse%
\ {\isaliteral{22}{\isachardoublequoteopen}}P\ {\isaliteral{5C3C6C6F6E676C65667472696768746172726F773E}{\isasymlongleftrightarrow}}\ Q{\isaliteral{22}{\isachardoublequoteclose}}\isanewline
\isacommand{proof}\isamarkupfalse%
\isanewline
\ \ \isacommand{assume}\isamarkupfalse%
\ {\isaliteral{22}{\isachardoublequoteopen}}P{\isaliteral{22}{\isachardoublequoteclose}}%
\\\mbox{}\quad$\vdots$\\\mbox{}\hspace{-1.4ex}
\ \ \isacommand{show}\isamarkupfalse%
\ {\isaliteral{22}{\isachardoublequoteopen}}Q{\isaliteral{22}{\isachardoublequoteclose}}\ %
\ $\dots$\\
\isacommand{next}\isamarkupfalse%
\isanewline
\ \ \isacommand{assume}\isamarkupfalse%
\ {\isaliteral{22}{\isachardoublequoteopen}}Q{\isaliteral{22}{\isachardoublequoteclose}}%
\\\mbox{}\quad$\vdots$\\\mbox{}\hspace{-1.4ex}
\ \ \isacommand{show}\isamarkupfalse%
\ {\isaliteral{22}{\isachardoublequoteopen}}P{\isaliteral{22}{\isachardoublequoteclose}}\ %
\ $\dots$\\
\isacommand{qed}\isamarkupfalse%
%
\endisatagproof
{\isafoldproof}%
%
\isadelimproof
%
\endisadelimproof
%
}
\medskip
\begin{isamarkuptext}%
Proofs by contradiction:
\end{isamarkuptext}%
\begin{tabular}{@ {}ll@ {}}
\begin{minipage}[t]{.4\textwidth}
\isa{%
%
\isadelimproof
%
\endisadelimproof
%
\isatagproof
\isacommand{show}\isamarkupfalse%
\ {\isaliteral{22}{\isachardoublequoteopen}}{\isaliteral{5C3C6E6F743E}{\isasymnot}}\ P{\isaliteral{22}{\isachardoublequoteclose}}\isanewline
\isacommand{proof}\isamarkupfalse%
\isanewline
\ \ \isacommand{assume}\isamarkupfalse%
\ {\isaliteral{22}{\isachardoublequoteopen}}P{\isaliteral{22}{\isachardoublequoteclose}}%
\\\mbox{}\quad$\vdots$\\\mbox{}\hspace{-1.4ex}
\ \ \isacommand{show}\isamarkupfalse%
\ {\isaliteral{22}{\isachardoublequoteopen}}False{\isaliteral{22}{\isachardoublequoteclose}}\ %
\ $\dots$\\
\isacommand{qed}\isamarkupfalse%
%
\endisatagproof
{\isafoldproof}%
%
\isadelimproof
%
\endisadelimproof
%
}
\end{minipage}
&
\begin{minipage}[t]{.4\textwidth}
\isa{%
%
\isadelimproof
%
\endisadelimproof
%
\isatagproof
\isacommand{show}\isamarkupfalse%
\ {\isaliteral{22}{\isachardoublequoteopen}}P{\isaliteral{22}{\isachardoublequoteclose}}\isanewline
\isacommand{proof}\isamarkupfalse%
\ {\isaliteral{28}{\isacharparenleft}}rule\ ccontr{\isaliteral{29}{\isacharparenright}}\isanewline
\ \ \isacommand{assume}\isamarkupfalse%
\ {\isaliteral{22}{\isachardoublequoteopen}}{\isaliteral{5C3C6E6F743E}{\isasymnot}}P{\isaliteral{22}{\isachardoublequoteclose}}%
\\\mbox{}\quad$\vdots$\\\mbox{}\hspace{-1.4ex}
\ \ \isacommand{show}\isamarkupfalse%
\ {\isaliteral{22}{\isachardoublequoteopen}}False{\isaliteral{22}{\isachardoublequoteclose}}\ %
\ $\dots$\\
\isacommand{qed}\isamarkupfalse%
%
\endisatagproof
{\isafoldproof}%
%
\isadelimproof
%
\endisadelimproof
%
}
\end{minipage}
\end{tabular}
\medskip
\begin{isamarkuptext}%
The name \isa{ccontr} stands for ``classical contradiction''.

How to prove quantified formulas:
\end{isamarkuptext}%
\begin{tabular}{@ {}ll@ {}}
\begin{minipage}[t]{.4\textwidth}
\isa{%
%
\isadelimproof
%
\endisadelimproof
%
\isatagproof
\isacommand{show}\isamarkupfalse%
\ {\isaliteral{22}{\isachardoublequoteopen}}{\isaliteral{5C3C666F72616C6C3E}{\isasymforall}}x{\isaliteral{2E}{\isachardot}}\ P{\isaliteral{28}{\isacharparenleft}}x{\isaliteral{29}{\isacharparenright}}{\isaliteral{22}{\isachardoublequoteclose}}\isanewline
\isacommand{proof}\isamarkupfalse%
\isanewline
\ \ \isacommand{fix}\isamarkupfalse%
\ x%
\\\mbox{}\quad$\vdots$\\\mbox{}\hspace{-1.4ex}
\ \ \isacommand{show}\isamarkupfalse%
\ {\isaliteral{22}{\isachardoublequoteopen}}P{\isaliteral{28}{\isacharparenleft}}x{\isaliteral{29}{\isacharparenright}}{\isaliteral{22}{\isachardoublequoteclose}}\ %
\ $\dots$\\
\isacommand{qed}\isamarkupfalse%
%
\endisatagproof
{\isafoldproof}%
%
\isadelimproof
%
\endisadelimproof
%
}
\end{minipage}
&
\begin{minipage}[t]{.4\textwidth}
\isa{%
%
\isadelimproof
%
\endisadelimproof
%
\isatagproof
\isacommand{show}\isamarkupfalse%
\ {\isaliteral{22}{\isachardoublequoteopen}}{\isaliteral{5C3C6578697374733E}{\isasymexists}}x{\isaliteral{2E}{\isachardot}}\ P{\isaliteral{28}{\isacharparenleft}}x{\isaliteral{29}{\isacharparenright}}{\isaliteral{22}{\isachardoublequoteclose}}\isanewline
\isacommand{proof}\isamarkupfalse%
%
\\\mbox{}\quad$\vdots$\\\mbox{}\hspace{-1.4ex}
\ \ \isacommand{show}\isamarkupfalse%
\ {\isaliteral{22}{\isachardoublequoteopen}}P{\isaliteral{28}{\isacharparenleft}}witness{\isaliteral{29}{\isacharparenright}}{\isaliteral{22}{\isachardoublequoteclose}}\ %
\ $\dots$\\
\isacommand{qed}\isamarkupfalse%
%
\endisatagproof
{\isafoldproof}%
%
\isadelimproof
%
\endisadelimproof
%
}
\end{minipage}
\end{tabular}
\medskip
\begin{isamarkuptext}%
In the proof of \noquotes{\isa{{\isaliteral{22}{\isachardoublequote}}{\isaliteral{5C3C666F72616C6C3E}{\isasymforall}}x{\isaliteral{2E}{\isachardot}}\ P{\isaliteral{28}{\isacharparenleft}}x{\isaliteral{29}{\isacharparenright}}{\isaliteral{22}{\isachardoublequote}}}},
the step \isacom{fix}~\isa{x} introduces a locally fixed variable \isa{x}
into the subproof, the proverbial ``arbitrary but fixed value''.
Instead of \isa{x} we could have chosen any name in the subproof.
In the proof of \noquotes{\isa{{\isaliteral{22}{\isachardoublequote}}{\isaliteral{5C3C6578697374733E}{\isasymexists}}x{\isaliteral{2E}{\isachardot}}\ P{\isaliteral{28}{\isacharparenleft}}x{\isaliteral{29}{\isacharparenright}}{\isaliteral{22}{\isachardoublequote}}}},
\isa{witness} is some arbitrary
term for which we can prove that it satisfies \isa{P}.

How to reason forward from \noquotes{\isa{{\isaliteral{22}{\isachardoublequote}}{\isaliteral{5C3C6578697374733E}{\isasymexists}}x{\isaliteral{2E}{\isachardot}}\ P{\isaliteral{28}{\isacharparenleft}}x{\isaliteral{29}{\isacharparenright}}{\isaliteral{22}{\isachardoublequote}}}}:
\end{isamarkuptext}%
%
\isadelimproof
%
\endisadelimproof
%
\isatagproof
\isacommand{have}\isamarkupfalse%
\ {\isaliteral{22}{\isachardoublequoteopen}}{\isaliteral{5C3C6578697374733E}{\isasymexists}}x{\isaliteral{2E}{\isachardot}}\ P{\isaliteral{28}{\isacharparenleft}}x{\isaliteral{29}{\isacharparenright}}{\isaliteral{22}{\isachardoublequoteclose}}\ %
\ $\dots$\\
\isacommand{then}\isamarkupfalse%
\ \isacommand{obtain}\isamarkupfalse%
\ x\ \isakeyword{where}\ p{\isaliteral{3A}{\isacharcolon}}\ {\isaliteral{22}{\isachardoublequoteopen}}P{\isaliteral{28}{\isacharparenleft}}x{\isaliteral{29}{\isacharparenright}}{\isaliteral{22}{\isachardoublequoteclose}}\ \isacommand{by}\isamarkupfalse%
\ blast%
\endisatagproof
{\isafoldproof}%
%
\isadelimproof
%
\endisadelimproof
%
\begin{isamarkuptext}%
After the \isacom{obtain} step, \isa{x} (we could have chosen any name)
is a fixed local
variable, and \isa{p} is the name of the fact
\noquotes{\isa{{\isaliteral{22}{\isachardoublequote}}P{\isaliteral{28}{\isacharparenleft}}x{\isaliteral{29}{\isacharparenright}}{\isaliteral{22}{\isachardoublequote}}}}.
This pattern works for one or more \isa{x}.
As an example of the \isacom{obtain} command, here is the proof of
Cantor's theorem in more detail:%
\end{isamarkuptext}%
\isamarkuptrue%
\isacommand{lemma}\isamarkupfalse%
\ {\isaliteral{22}{\isachardoublequoteopen}}{\isaliteral{5C3C6E6F743E}{\isasymnot}}\ surj{\isaliteral{28}{\isacharparenleft}}f\ {\isaliteral{3A}{\isacharcolon}}{\isaliteral{3A}{\isacharcolon}}\ {\isaliteral{27}{\isacharprime}}a\ {\isaliteral{5C3C52696768746172726F773E}{\isasymRightarrow}}\ {\isaliteral{27}{\isacharprime}}a\ set{\isaliteral{29}{\isacharparenright}}{\isaliteral{22}{\isachardoublequoteclose}}\isanewline
%
\isadelimproof
%
\endisadelimproof
%
\isatagproof
\isacommand{proof}\isamarkupfalse%
\isanewline
\ \ \isacommand{assume}\isamarkupfalse%
\ {\isaliteral{22}{\isachardoublequoteopen}}surj\ f{\isaliteral{22}{\isachardoublequoteclose}}\isanewline
\ \ \isacommand{hence}\isamarkupfalse%
\ \ {\isaliteral{22}{\isachardoublequoteopen}}{\isaliteral{5C3C6578697374733E}{\isasymexists}}a{\isaliteral{2E}{\isachardot}}\ {\isaliteral{7B}{\isacharbraceleft}}x{\isaliteral{2E}{\isachardot}}\ x\ {\isaliteral{5C3C6E6F74696E3E}{\isasymnotin}}\ f\ x{\isaliteral{7D}{\isacharbraceright}}\ {\isaliteral{3D}{\isacharequal}}\ f\ a{\isaliteral{22}{\isachardoublequoteclose}}\ \isacommand{by}\isamarkupfalse%
{\isaliteral{28}{\isacharparenleft}}auto\ simp{\isaliteral{3A}{\isacharcolon}}\ surj{\isaliteral{5F}{\isacharunderscore}}def{\isaliteral{29}{\isacharparenright}}\isanewline
\ \ \isacommand{then}\isamarkupfalse%
\ \isacommand{obtain}\isamarkupfalse%
\ a\ \isakeyword{where}\ \ {\isaliteral{22}{\isachardoublequoteopen}}{\isaliteral{7B}{\isacharbraceleft}}x{\isaliteral{2E}{\isachardot}}\ x\ {\isaliteral{5C3C6E6F74696E3E}{\isasymnotin}}\ f\ x{\isaliteral{7D}{\isacharbraceright}}\ {\isaliteral{3D}{\isacharequal}}\ f\ a{\isaliteral{22}{\isachardoublequoteclose}}\ \ \isacommand{by}\isamarkupfalse%
\ blast\isanewline
\ \ \isacommand{hence}\isamarkupfalse%
\ \ {\isaliteral{22}{\isachardoublequoteopen}}a\ {\isaliteral{5C3C6E6F74696E3E}{\isasymnotin}}\ f\ a\ {\isaliteral{5C3C6C6F6E676C65667472696768746172726F773E}{\isasymlongleftrightarrow}}\ a\ {\isaliteral{5C3C696E3E}{\isasymin}}\ f\ a{\isaliteral{22}{\isachardoublequoteclose}}\ \ \isacommand{by}\isamarkupfalse%
\ blast\isanewline
\ \ \isacommand{thus}\isamarkupfalse%
\ {\isaliteral{22}{\isachardoublequoteopen}}False{\isaliteral{22}{\isachardoublequoteclose}}\ \isacommand{by}\isamarkupfalse%
\ blast\isanewline
\isacommand{qed}\isamarkupfalse%
%
\endisatagproof
{\isafoldproof}%
%
\isadelimproof
%
\endisadelimproof
%
\begin{isamarkuptext}%

Finally, how to prove set equality and subset relationship:
\end{isamarkuptext}%
\begin{tabular}{@ {}ll@ {}}
\begin{minipage}[t]{.4\textwidth}
\isa{%
%
\isadelimproof
%
\endisadelimproof
%
\isatagproof
\isacommand{show}\isamarkupfalse%
\ {\isaliteral{22}{\isachardoublequoteopen}}A\ {\isaliteral{3D}{\isacharequal}}\ B{\isaliteral{22}{\isachardoublequoteclose}}\isanewline
\isacommand{proof}\isamarkupfalse%
\isanewline
\ \ \isacommand{show}\isamarkupfalse%
\ {\isaliteral{22}{\isachardoublequoteopen}}A\ {\isaliteral{5C3C73756273657465713E}{\isasymsubseteq}}\ B{\isaliteral{22}{\isachardoublequoteclose}}\ %
\ $\dots$\\
\isacommand{next}\isamarkupfalse%
\isanewline
\ \ \isacommand{show}\isamarkupfalse%
\ {\isaliteral{22}{\isachardoublequoteopen}}B\ {\isaliteral{5C3C73756273657465713E}{\isasymsubseteq}}\ A{\isaliteral{22}{\isachardoublequoteclose}}\ %
\ $\dots$\\
\isacommand{qed}\isamarkupfalse%
%
\endisatagproof
{\isafoldproof}%
%
\isadelimproof
%
\endisadelimproof
%
}
\end{minipage}
&
\begin{minipage}[t]{.4\textwidth}
\isa{%
%
\isadelimproof
%
\endisadelimproof
%
\isatagproof
\isacommand{show}\isamarkupfalse%
\ {\isaliteral{22}{\isachardoublequoteopen}}A\ {\isaliteral{5C3C73756273657465713E}{\isasymsubseteq}}\ B{\isaliteral{22}{\isachardoublequoteclose}}\isanewline
\isacommand{proof}\isamarkupfalse%
\isanewline
\ \ \isacommand{fix}\isamarkupfalse%
\ x\isanewline
\ \ \isacommand{assume}\isamarkupfalse%
\ {\isaliteral{22}{\isachardoublequoteopen}}x\ {\isaliteral{5C3C696E3E}{\isasymin}}\ A{\isaliteral{22}{\isachardoublequoteclose}}%
\\\mbox{}\quad$\vdots$\\\mbox{}\hspace{-1.4ex}
\ \ \isacommand{show}\isamarkupfalse%
\ {\isaliteral{22}{\isachardoublequoteopen}}x\ {\isaliteral{5C3C696E3E}{\isasymin}}\ B{\isaliteral{22}{\isachardoublequoteclose}}\ %
\ $\dots$\\
\isacommand{qed}\isamarkupfalse%
%
\endisatagproof
{\isafoldproof}%
%
\isadelimproof
%
\endisadelimproof
%
}
\end{minipage}
\end{tabular}
\begin{isamarkuptext}%
\section{Streamlining proofs}

\subsection{Pattern matching and quotations}

In the proof patterns shown above, formulas are often duplicated.
This can make the text harder to read, write and maintain. Pattern matching
is an abbreviation mechanism to avoid such duplication. Writing
\begin{quote}
\isacom{show} \ \textit{formula} \isa{{\isaliteral{28}{\isacharparenleft}}}\isacom{is} \textit{pattern}\isa{{\isaliteral{29}{\isacharparenright}}}
\end{quote}
matches the pattern against the formula, thus instantiating the unknowns in
the pattern for later use. As an example, consider the proof pattern for
\isa{{\isaliteral{5C3C6C6F6E676C65667472696768746172726F773E}{\isasymlongleftrightarrow}}}:
\end{isamarkuptext}%
%
\isadelimproof
%
\endisadelimproof
%
\isatagproof
\isacommand{show}\isamarkupfalse%
\ {\isaliteral{22}{\isachardoublequoteopen}}formula\isaliteral{5C3C5E697375623E}{}\isactrlisub {\isadigit{1}}\ {\isaliteral{5C3C6C6F6E676C65667472696768746172726F773E}{\isasymlongleftrightarrow}}\ formula\isaliteral{5C3C5E697375623E}{}\isactrlisub {\isadigit{2}}{\isaliteral{22}{\isachardoublequoteclose}}\ {\isaliteral{28}{\isacharparenleft}}\isakeyword{is}\ {\isaliteral{22}{\isachardoublequoteopen}}{\isaliteral{3F}{\isacharquery}}L\ {\isaliteral{5C3C6C6F6E676C65667472696768746172726F773E}{\isasymlongleftrightarrow}}\ {\isaliteral{3F}{\isacharquery}}R{\isaliteral{22}{\isachardoublequoteclose}}{\isaliteral{29}{\isacharparenright}}\isanewline
\isacommand{proof}\isamarkupfalse%
\isanewline
\ \ \isacommand{assume}\isamarkupfalse%
\ {\isaliteral{22}{\isachardoublequoteopen}}{\isaliteral{3F}{\isacharquery}}L{\isaliteral{22}{\isachardoublequoteclose}}%
\\\mbox{}\quad$\vdots$\\\mbox{}\hspace{-1.4ex}
\ \ \isacommand{show}\isamarkupfalse%
\ {\isaliteral{22}{\isachardoublequoteopen}}{\isaliteral{3F}{\isacharquery}}R{\isaliteral{22}{\isachardoublequoteclose}}\ %
\ $\dots$\\
\isacommand{next}\isamarkupfalse%
\isanewline
\ \ \isacommand{assume}\isamarkupfalse%
\ {\isaliteral{22}{\isachardoublequoteopen}}{\isaliteral{3F}{\isacharquery}}R{\isaliteral{22}{\isachardoublequoteclose}}%
\\\mbox{}\quad$\vdots$\\\mbox{}\hspace{-1.4ex}
\ \ \isacommand{show}\isamarkupfalse%
\ {\isaliteral{22}{\isachardoublequoteopen}}{\isaliteral{3F}{\isacharquery}}L{\isaliteral{22}{\isachardoublequoteclose}}\ %
\ $\dots$\\
\isacommand{qed}\isamarkupfalse%
%
\endisatagproof
{\isafoldproof}%
%
\isadelimproof
%
\endisadelimproof
%
\begin{isamarkuptext}%
Instead of duplicating \isa{formula\isaliteral{5C3C5E697375623E}{}\isactrlisub i} in the text, we introduce
the two abbreviations \isa{{\isaliteral{3F}{\isacharquery}}L} and \isa{{\isaliteral{3F}{\isacharquery}}R} by pattern matching.
Pattern matching works wherever a formula is stated, in particular
with \isacom{have} and \isacom{lemma}.

The unknown \isa{{\isaliteral{3F}{\isacharquery}}thesis} is implicitly matched against any goal stated by
\isacom{lemma} or \isacom{show}. Here is a typical example:%
\end{isamarkuptext}%
\isamarkuptrue%
\isacommand{lemma}\isamarkupfalse%
\ {\isaliteral{22}{\isachardoublequoteopen}}formula{\isaliteral{22}{\isachardoublequoteclose}}\isanewline
%
\isadelimproof
%
\endisadelimproof
%
\isatagproof
\isacommand{proof}\isamarkupfalse%
\ {\isaliteral{2D}{\isacharminus}}%
\\\mbox{}\quad$\vdots$\\\mbox{}\hspace{-1.4ex}
\ \ \isacommand{show}\isamarkupfalse%
\ {\isaliteral{3F}{\isacharquery}}thesis\ %
\ $\dots$\\
\isacommand{qed}\isamarkupfalse%
%
\endisatagproof
{\isafoldproof}%
%
\isadelimproof
%
\endisadelimproof
%
\begin{isamarkuptext}%
Unknowns can also be instantiated with \isacom{let} commands
\begin{quote}
\isacom{let} \isa{{\isaliteral{3F}{\isacharquery}}t} = \isa{{\isaliteral{22}{\isachardoublequote}}}\textit{some-big-term}\isa{{\isaliteral{22}{\isachardoublequote}}}
\end{quote}
Later proof steps can refer to \isa{{\isaliteral{3F}{\isacharquery}}t}:
\begin{quote}
\isacom{have} \isa{{\isaliteral{22}{\isachardoublequote}}}\dots \isa{{\isaliteral{3F}{\isacharquery}}t} \dots\isa{{\isaliteral{22}{\isachardoublequote}}}
\end{quote}
\begin{warn}
Names of facts are introduced with \isa{name{\isaliteral{3A}{\isacharcolon}}} and refer to proved
theorems. Unknowns \isa{{\isaliteral{3F}{\isacharquery}}X} refer to terms or formulas.
\end{warn}

Although abbreviations shorten the text, the reader needs to remember what
they stand for. Similarly for names of facts. Names like \isa{{\isadigit{1}}}, \isa{{\isadigit{2}}}
and \isa{{\isadigit{3}}} are not helpful and should only be used in short proofs. For
longer proofs, descriptive names are better. But look at this example:
\begin{quote}
\isacom{have} \ \isa{x{\isaliteral{5F}{\isacharunderscore}}gr{\isaliteral{5F}{\isacharunderscore}}{\isadigit{0}}{\isaliteral{3A}{\isacharcolon}}\ {\isaliteral{22}{\isachardoublequote}}x\ {\isaliteral{3E}{\isachargreater}}\ {\isadigit{0}}{\isaliteral{22}{\isachardoublequote}}}\\
$\vdots$\\
\isacom{from} \isa{x{\isaliteral{5F}{\isacharunderscore}}gr{\isaliteral{5F}{\isacharunderscore}}{\isadigit{0}}} \dots
\end{quote}
The name is longer than the fact it stands for! Short facts do not need names,
one can refer to them easily by quoting them:
\begin{quote}
\isacom{have} \ \isa{{\isaliteral{22}{\isachardoublequote}}x\ {\isaliteral{3E}{\isachargreater}}\ {\isadigit{0}}{\isaliteral{22}{\isachardoublequote}}}\\
$\vdots$\\
\isacom{from} \isa{{\isaliteral{60}{\isacharbackquote}}x{\isaliteral{3E}{\isachargreater}}{\isadigit{0}}{\isaliteral{60}{\isacharbackquote}}} \dots
\end{quote}
Note that the quotes around \isa{x{\isaliteral{3E}{\isachargreater}}{\isadigit{0}}} are \concept{back quotes}.
They refer to the fact not by name but by value.

\subsection{\isacom{moreover}}

Sometimes one needs a number of facts to enable some deduction. Of course
one can name these facts individually, as shown on the right,
but one can also combine them with \isacom{moreover}, as shown on the left:%
\end{isamarkuptext}%
\isamarkuptrue%
%
\begin{tabular}{@ {}ll@ {}}
\begin{minipage}[t]{.4\textwidth}
\isa{%
%
\isadelimproof
%
\endisadelimproof
%
\isatagproof
\isacommand{have}\isamarkupfalse%
\ {\isaliteral{22}{\isachardoublequoteopen}}P\isaliteral{5C3C5E697375623E}{}\isactrlisub {\isadigit{1}}{\isaliteral{22}{\isachardoublequoteclose}}\ %
\ $\dots$\\
\isacommand{moreover}\isamarkupfalse%
\ \isacommand{have}\isamarkupfalse%
\ {\isaliteral{22}{\isachardoublequoteopen}}P\isaliteral{5C3C5E697375623E}{}\isactrlisub {\isadigit{2}}{\isaliteral{22}{\isachardoublequoteclose}}\ %
\ $\dots$\\
\isacommand{moreover}\isamarkupfalse%
%
\\$\vdots$\\\hspace{-1.4ex}
\isacommand{moreover}\isamarkupfalse%
\ \isacommand{have}\isamarkupfalse%
\ {\isaliteral{22}{\isachardoublequoteopen}}P\isaliteral{5C3C5E697375623E}{}\isactrlisub n{\isaliteral{22}{\isachardoublequoteclose}}\ %
\ $\dots$\\
\isacommand{ultimately}\isamarkupfalse%
\ \isacommand{have}\isamarkupfalse%
\ {\isaliteral{22}{\isachardoublequoteopen}}P{\isaliteral{22}{\isachardoublequoteclose}}\ \ %
\ $\dots$\\
%
\endisatagproof
{\isafoldproof}%
%
\isadelimproof
%
\endisadelimproof
%
}
\end{minipage}
&
\qquad
\begin{minipage}[t]{.4\textwidth}
\isa{%
%
\isadelimproof
%
\endisadelimproof
%
\isatagproof
\isacommand{have}\isamarkupfalse%
\ lab\isaliteral{5C3C5E697375623E}{}\isactrlisub {\isadigit{1}}{\isaliteral{3A}{\isacharcolon}}\ {\isaliteral{22}{\isachardoublequoteopen}}P\isaliteral{5C3C5E697375623E}{}\isactrlisub {\isadigit{1}}{\isaliteral{22}{\isachardoublequoteclose}}\ %
\ $\dots$\\
\isacommand{have}\isamarkupfalse%
\ lab\isaliteral{5C3C5E697375623E}{}\isactrlisub {\isadigit{2}}{\isaliteral{3A}{\isacharcolon}}\ {\isaliteral{22}{\isachardoublequoteopen}}P\isaliteral{5C3C5E697375623E}{}\isactrlisub {\isadigit{2}}{\isaliteral{22}{\isachardoublequoteclose}}\ %
\ $\dots$
%
\\$\vdots$\\\hspace{-1.4ex}
\isacommand{have}\isamarkupfalse%
\ lab\isaliteral{5C3C5E697375623E}{}\isactrlisub n{\isaliteral{3A}{\isacharcolon}}\ {\isaliteral{22}{\isachardoublequoteopen}}P\isaliteral{5C3C5E697375623E}{}\isactrlisub n{\isaliteral{22}{\isachardoublequoteclose}}\ %
\ $\dots$\\
\isacommand{from}\isamarkupfalse%
\ lab\isaliteral{5C3C5E697375623E}{}\isactrlisub {\isadigit{1}}\ lab\isaliteral{5C3C5E697375623E}{}\isactrlisub {\isadigit{2}}%
\ $\dots$\\
\isacommand{have}\isamarkupfalse%
\ {\isaliteral{22}{\isachardoublequoteopen}}P{\isaliteral{22}{\isachardoublequoteclose}}\ \ %
\ $\dots$\\
%
\endisatagproof
{\isafoldproof}%
%
\isadelimproof
%
\endisadelimproof
%
}
\end{minipage}
\end{tabular}
\begin{isamarkuptext}%
The \isacom{moreover} version is no shorter but expresses the structure more
clearly and avoids new names.

\subsection{Raw proof blocks}

Sometimes one would like to prove some lemma locally within a proof.
A lemma that shares the current context of assumptions but that
has its own assumptions and is generalised over its locally fixed
variables at the end. This is what a \concept{raw proof block} does:
\begin{quote}
\isa{{\isaliteral{7B}{\isacharbraceleft}}} \isacom{fix} \isa{x\isaliteral{5C3C5E697375623E}{}\isactrlisub {\isadigit{1}}\ {\isaliteral{5C3C646F74733E}{\isasymdots}}\ x\isaliteral{5C3C5E697375623E}{}\isactrlisub n}\\
\mbox{}\ \ \ \isacom{assume} \isa{A\isaliteral{5C3C5E697375623E}{}\isactrlisub {\isadigit{1}}\ {\isaliteral{5C3C646F74733E}{\isasymdots}}\ A\isaliteral{5C3C5E697375623E}{}\isactrlisub m}\\
\mbox{}\ \ \ $\vdots$\\
\mbox{}\ \ \ \isacom{have} \isa{B}\\
\isa{{\isaliteral{7D}{\isacharbraceright}}}
\end{quote}
proves \isa{{\isaliteral{5C3C6C6272616B6B3E}{\isasymlbrakk}}\ A\isaliteral{5C3C5E697375623E}{}\isactrlisub {\isadigit{1}}{\isaliteral{3B}{\isacharsemicolon}}\ {\isaliteral{5C3C646F74733E}{\isasymdots}}\ {\isaliteral{3B}{\isacharsemicolon}}\ A\isaliteral{5C3C5E697375623E}{}\isactrlisub m\ {\isaliteral{5C3C726272616B6B3E}{\isasymrbrakk}}\ {\isaliteral{5C3C4C6F6E6772696768746172726F773E}{\isasymLongrightarrow}}\ B}
where all \isa{x\isaliteral{5C3C5E697375623E}{}\isactrlisub i} have been replaced by unknowns \isa{{\isaliteral{3F}{\isacharquery}}x\isaliteral{5C3C5E697375623E}{}\isactrlisub i}.
\begin{warn}
The conclusion of a raw proof block is \emph{not} indicated by \isacom{show}
but is simply the final \isacom{have}.
\end{warn}

As an example we prove a simple fact about divisibility on integers.
The definition of \isa{dvd} is \isa{{\isaliteral{28}{\isacharparenleft}}b\ dvd\ a{\isaliteral{29}{\isacharparenright}}\ {\isaliteral{3D}{\isacharequal}}\ {\isaliteral{28}{\isacharparenleft}}{\isaliteral{5C3C6578697374733E}{\isasymexists}}k{\isaliteral{2E}{\isachardot}}\ a\ {\isaliteral{3D}{\isacharequal}}\ b\ {\isaliteral{2A}{\isacharasterisk}}\ k{\isaliteral{29}{\isacharparenright}}}.
\end{isamarkuptext}%
\isacommand{lemma}\isamarkupfalse%
\ \isakeyword{fixes}\ a\ b\ {\isaliteral{3A}{\isacharcolon}}{\isaliteral{3A}{\isacharcolon}}\ int\ \isakeyword{assumes}\ {\isaliteral{22}{\isachardoublequoteopen}}b\ dvd\ {\isaliteral{28}{\isacharparenleft}}a{\isaliteral{2B}{\isacharplus}}b{\isaliteral{29}{\isacharparenright}}{\isaliteral{22}{\isachardoublequoteclose}}\ \isakeyword{shows}\ {\isaliteral{22}{\isachardoublequoteopen}}b\ dvd\ a{\isaliteral{22}{\isachardoublequoteclose}}\isanewline
%
\isadelimproof
%
\endisadelimproof
%
\isatagproof
\isacommand{proof}\isamarkupfalse%
\ {\isaliteral{2D}{\isacharminus}}\isanewline
\ \ \isacommand{{\isaliteral{7B}{\isacharbraceleft}}}\isamarkupfalse%
\ \isacommand{fix}\isamarkupfalse%
\ k\ \isacommand{assume}\isamarkupfalse%
\ k{\isaliteral{3A}{\isacharcolon}}\ {\isaliteral{22}{\isachardoublequoteopen}}a{\isaliteral{2B}{\isacharplus}}b\ {\isaliteral{3D}{\isacharequal}}\ b{\isaliteral{2A}{\isacharasterisk}}k{\isaliteral{22}{\isachardoublequoteclose}}\isanewline
\ \ \ \ \isacommand{have}\isamarkupfalse%
\ {\isaliteral{22}{\isachardoublequoteopen}}{\isaliteral{5C3C6578697374733E}{\isasymexists}}k{\isaliteral{27}{\isacharprime}}{\isaliteral{2E}{\isachardot}}\ a\ {\isaliteral{3D}{\isacharequal}}\ b{\isaliteral{2A}{\isacharasterisk}}k{\isaliteral{27}{\isacharprime}}{\isaliteral{22}{\isachardoublequoteclose}}\isanewline
\ \ \ \ \isacommand{proof}\isamarkupfalse%
\isanewline
\ \ \ \ \ \ \isacommand{show}\isamarkupfalse%
\ {\isaliteral{22}{\isachardoublequoteopen}}a\ {\isaliteral{3D}{\isacharequal}}\ b{\isaliteral{2A}{\isacharasterisk}}{\isaliteral{28}{\isacharparenleft}}k\ {\isaliteral{2D}{\isacharminus}}\ {\isadigit{1}}{\isaliteral{29}{\isacharparenright}}{\isaliteral{22}{\isachardoublequoteclose}}\ \isacommand{using}\isamarkupfalse%
\ k\ \isacommand{by}\isamarkupfalse%
{\isaliteral{28}{\isacharparenleft}}simp\ add{\isaliteral{3A}{\isacharcolon}}\ algebra{\isaliteral{5F}{\isacharunderscore}}simps{\isaliteral{29}{\isacharparenright}}\isanewline
\ \ \ \ \isacommand{qed}\isamarkupfalse%
\ \isacommand{{\isaliteral{7D}{\isacharbraceright}}}\isamarkupfalse%
\isanewline
\ \ \isacommand{then}\isamarkupfalse%
\ \isacommand{show}\isamarkupfalse%
\ {\isaliteral{3F}{\isacharquery}}thesis\ \isacommand{using}\isamarkupfalse%
\ assms\ \isacommand{by}\isamarkupfalse%
{\isaliteral{28}{\isacharparenleft}}auto\ simp\ add{\isaliteral{3A}{\isacharcolon}}\ dvd{\isaliteral{5F}{\isacharunderscore}}def{\isaliteral{29}{\isacharparenright}}\isanewline
\isacommand{qed}\isamarkupfalse%
%
\endisatagproof
{\isafoldproof}%
%
\isadelimproof
%
\endisadelimproof
%
\begin{isamarkuptext}%
Note that the result of a raw proof block has no name. In this example
it was directly piped (via \isacom{then}) into the final proof, but it can
also be named for later reference: you simply follow the block directly by a
\isacom{note} command:
\begin{quote}
\isacom{note} \ \isa{name\ {\isaliteral{3D}{\isacharequal}}\ this}
\end{quote}
This introduces a new name \isa{name} that refers to \isa{this},
the fact just proved, in this case the preceding block. In general,
\isacom{note} introduces a new name for one or more facts.

\section{Case distinction and induction}

\subsection{Datatype case distinction}

We have seen case distinction on formulas. Now we want to distinguish
which form some term takes: is it \isa{{\isadigit{0}}} or of the form \isa{Suc\ n},
is it \isa{{\isaliteral{5B}{\isacharbrackleft}}{\isaliteral{5D}{\isacharbrackright}}} or of the form \isa{x\ {\isaliteral{23}{\isacharhash}}\ xs}, etc. Here is a typical example
proof by case distinction on the form of \isa{xs}:%
\end{isamarkuptext}%
\isamarkuptrue%
\isacommand{lemma}\isamarkupfalse%
\ {\isaliteral{22}{\isachardoublequoteopen}}length{\isaliteral{28}{\isacharparenleft}}tl\ xs{\isaliteral{29}{\isacharparenright}}\ {\isaliteral{3D}{\isacharequal}}\ length\ xs\ {\isaliteral{2D}{\isacharminus}}\ {\isadigit{1}}{\isaliteral{22}{\isachardoublequoteclose}}\isanewline
%
\isadelimproof
%
\endisadelimproof
%
\isatagproof
\isacommand{proof}\isamarkupfalse%
\ {\isaliteral{28}{\isacharparenleft}}cases\ xs{\isaliteral{29}{\isacharparenright}}\isanewline
\ \ \isacommand{assume}\isamarkupfalse%
\ {\isaliteral{22}{\isachardoublequoteopen}}xs\ {\isaliteral{3D}{\isacharequal}}\ {\isaliteral{5B}{\isacharbrackleft}}{\isaliteral{5D}{\isacharbrackright}}{\isaliteral{22}{\isachardoublequoteclose}}\isanewline
\ \ \isacommand{thus}\isamarkupfalse%
\ {\isaliteral{3F}{\isacharquery}}thesis\ \isacommand{by}\isamarkupfalse%
\ simp\isanewline
\isacommand{next}\isamarkupfalse%
\isanewline
\ \ \isacommand{fix}\isamarkupfalse%
\ y\ ys\ \isacommand{assume}\isamarkupfalse%
\ {\isaliteral{22}{\isachardoublequoteopen}}xs\ {\isaliteral{3D}{\isacharequal}}\ y{\isaliteral{23}{\isacharhash}}ys{\isaliteral{22}{\isachardoublequoteclose}}\isanewline
\ \ \isacommand{thus}\isamarkupfalse%
\ {\isaliteral{3F}{\isacharquery}}thesis\ \isacommand{by}\isamarkupfalse%
\ simp\isanewline
\isacommand{qed}\isamarkupfalse%
%
\endisatagproof
{\isafoldproof}%
%
\isadelimproof
%
\endisadelimproof
%
\begin{isamarkuptext}%
Function \isa{tl} (''tail'') is defined by \isa{tl\ {\isaliteral{5B}{\isacharbrackleft}}{\isaliteral{5D}{\isacharbrackright}}\ {\isaliteral{3D}{\isacharequal}}\ {\isaliteral{5B}{\isacharbrackleft}}{\isaliteral{5D}{\isacharbrackright}}} and
\isa{tl\ {\isaliteral{28}{\isacharparenleft}}x\ {\isaliteral{23}{\isacharhash}}\ xs{\isaliteral{29}{\isacharparenright}}\ {\isaliteral{3D}{\isacharequal}}\ xs}. Note that the result type of \isa{length} is \isa{nat}
and \isa{{\isadigit{0}}\ {\isaliteral{2D}{\isacharminus}}\ {\isadigit{1}}\ {\isaliteral{3D}{\isacharequal}}\ {\isadigit{0}}}.

This proof pattern works for any term \isa{t} whose type is a datatype.
The goal has to be proved for each constructor \isa{C}:
\begin{quote}
\isacom{fix} \ \isa{x\isaliteral{5C3C5E697375623E}{}\isactrlisub {\isadigit{1}}\ {\isaliteral{5C3C646F74733E}{\isasymdots}}\ x\isaliteral{5C3C5E697375623E}{}\isactrlisub n} \isacom{assume} \isa{{\isaliteral{22}{\isachardoublequote}}t\ {\isaliteral{3D}{\isacharequal}}\ C\ x\isaliteral{5C3C5E697375623E}{}\isactrlisub {\isadigit{1}}\ {\isaliteral{5C3C646F74733E}{\isasymdots}}\ x\isaliteral{5C3C5E697375623E}{}\isactrlisub n{\isaliteral{22}{\isachardoublequote}}}
\end{quote}
Each case can be written in a more compact form by means of the \isacom{case}
command:
\begin{quote}
\isacom{case} \isa{{\isaliteral{28}{\isacharparenleft}}C\ x\isaliteral{5C3C5E697375623E}{}\isactrlisub {\isadigit{1}}\ {\isaliteral{5C3C646F74733E}{\isasymdots}}\ x\isaliteral{5C3C5E697375623E}{}\isactrlisub n{\isaliteral{29}{\isacharparenright}}}
\end{quote}
This is equivalent to the explicit \isacom{fix}-\isacom{assume} line
but also gives the assumption \isa{{\isaliteral{22}{\isachardoublequote}}t\ {\isaliteral{3D}{\isacharequal}}\ C\ x\isaliteral{5C3C5E697375623E}{}\isactrlisub {\isadigit{1}}\ {\isaliteral{5C3C646F74733E}{\isasymdots}}\ x\isaliteral{5C3C5E697375623E}{}\isactrlisub n{\isaliteral{22}{\isachardoublequote}}} a name: \isa{C},
like the constructor.
Here is the \isacom{case} version of the proof above:%
\end{isamarkuptext}%
\isamarkuptrue%
%
\isadelimproof
%
\endisadelimproof
%
\isatagproof
\isacommand{proof}\isamarkupfalse%
\ {\isaliteral{28}{\isacharparenleft}}cases\ xs{\isaliteral{29}{\isacharparenright}}\isanewline
\ \ \isacommand{case}\isamarkupfalse%
\ Nil\isanewline
\ \ \isacommand{thus}\isamarkupfalse%
\ {\isaliteral{3F}{\isacharquery}}thesis\ \isacommand{by}\isamarkupfalse%
\ simp\isanewline
\isacommand{next}\isamarkupfalse%
\isanewline
\ \ \isacommand{case}\isamarkupfalse%
\ {\isaliteral{28}{\isacharparenleft}}Cons\ y\ ys{\isaliteral{29}{\isacharparenright}}\isanewline
\ \ \isacommand{thus}\isamarkupfalse%
\ {\isaliteral{3F}{\isacharquery}}thesis\ \isacommand{by}\isamarkupfalse%
\ simp\isanewline
\isacommand{qed}\isamarkupfalse%
%
\endisatagproof
{\isafoldproof}%
%
\isadelimproof
%
\endisadelimproof
%
\begin{isamarkuptext}%
Remember that \isa{Nil} and \isa{Cons} are the alphanumeric names
for \isa{{\isaliteral{5B}{\isacharbrackleft}}{\isaliteral{5D}{\isacharbrackright}}} and \isa{{\isaliteral{23}{\isacharhash}}}. The names of the assumptions
are not used because they are directly piped (via \isacom{thus})
into the proof of the claim.

\subsection{Structural induction}

We illustrate structural induction with an example based on natural numbers:
the sum (\isa{{\isaliteral{5C3C53756D3E}{\isasymSum}}}) of the first \isa{n} natural numbers
(\isa{{\isaliteral{7B}{\isacharbraceleft}}{\isadigit{0}}{\isaliteral{2E}{\isachardot}}{\isaliteral{2E}{\isachardot}}n{\isaliteral{3A}{\isacharcolon}}{\isaliteral{3A}{\isacharcolon}}nat{\isaliteral{7D}{\isacharbraceright}}}) is equal to \mbox{\isa{n\ {\isaliteral{2A}{\isacharasterisk}}\ {\isaliteral{28}{\isacharparenleft}}n\ {\isaliteral{2B}{\isacharplus}}\ {\isadigit{1}}{\isaliteral{29}{\isacharparenright}}\ div\ {\isadigit{2}}}}.
Never mind the details, just focus on the pattern:%
\end{isamarkuptext}%
\isamarkuptrue%
\isacommand{lemma}\isamarkupfalse%
\ {\isaliteral{22}{\isachardoublequoteopen}}{\isaliteral{5C3C53756D3E}{\isasymSum}}{\isaliteral{7B}{\isacharbraceleft}}{\isadigit{0}}{\isaliteral{2E}{\isachardot}}{\isaliteral{2E}{\isachardot}}n{\isaliteral{3A}{\isacharcolon}}{\isaliteral{3A}{\isacharcolon}}nat{\isaliteral{7D}{\isacharbraceright}}\ {\isaliteral{3D}{\isacharequal}}\ n{\isaliteral{2A}{\isacharasterisk}}{\isaliteral{28}{\isacharparenleft}}n{\isaliteral{2B}{\isacharplus}}{\isadigit{1}}{\isaliteral{29}{\isacharparenright}}\ div\ {\isadigit{2}}{\isaliteral{22}{\isachardoublequoteclose}}\ {\isaliteral{28}{\isacharparenleft}}\isakeyword{is}\ {\isaliteral{22}{\isachardoublequoteopen}}{\isaliteral{3F}{\isacharquery}}P\ n{\isaliteral{22}{\isachardoublequoteclose}}{\isaliteral{29}{\isacharparenright}}\isanewline
%
\isadelimproof
%
\endisadelimproof
%
\isatagproof
\isacommand{proof}\isamarkupfalse%
\ {\isaliteral{28}{\isacharparenleft}}induction\ n{\isaliteral{29}{\isacharparenright}}\isanewline
\ \ \isacommand{show}\isamarkupfalse%
\ {\isaliteral{22}{\isachardoublequoteopen}}{\isaliteral{5C3C53756D3E}{\isasymSum}}{\isaliteral{7B}{\isacharbraceleft}}{\isadigit{0}}{\isaliteral{2E}{\isachardot}}{\isaliteral{2E}{\isachardot}}{\isadigit{0}}{\isaliteral{3A}{\isacharcolon}}{\isaliteral{3A}{\isacharcolon}}nat{\isaliteral{7D}{\isacharbraceright}}\ {\isaliteral{3D}{\isacharequal}}\ {\isadigit{0}}{\isaliteral{2A}{\isacharasterisk}}{\isaliteral{28}{\isacharparenleft}}{\isadigit{0}}{\isaliteral{2B}{\isacharplus}}{\isadigit{1}}{\isaliteral{29}{\isacharparenright}}\ div\ {\isadigit{2}}{\isaliteral{22}{\isachardoublequoteclose}}\ \isacommand{by}\isamarkupfalse%
\ simp\isanewline
\isacommand{next}\isamarkupfalse%
\isanewline
\ \ \isacommand{fix}\isamarkupfalse%
\ n\ \isacommand{assume}\isamarkupfalse%
\ {\isaliteral{22}{\isachardoublequoteopen}}{\isaliteral{5C3C53756D3E}{\isasymSum}}{\isaliteral{7B}{\isacharbraceleft}}{\isadigit{0}}{\isaliteral{2E}{\isachardot}}{\isaliteral{2E}{\isachardot}}n{\isaliteral{3A}{\isacharcolon}}{\isaliteral{3A}{\isacharcolon}}nat{\isaliteral{7D}{\isacharbraceright}}\ {\isaliteral{3D}{\isacharequal}}\ n{\isaliteral{2A}{\isacharasterisk}}{\isaliteral{28}{\isacharparenleft}}n{\isaliteral{2B}{\isacharplus}}{\isadigit{1}}{\isaliteral{29}{\isacharparenright}}\ div\ {\isadigit{2}}{\isaliteral{22}{\isachardoublequoteclose}}\isanewline
\ \ \isacommand{thus}\isamarkupfalse%
\ {\isaliteral{22}{\isachardoublequoteopen}}{\isaliteral{5C3C53756D3E}{\isasymSum}}{\isaliteral{7B}{\isacharbraceleft}}{\isadigit{0}}{\isaliteral{2E}{\isachardot}}{\isaliteral{2E}{\isachardot}}Suc\ n{\isaliteral{3A}{\isacharcolon}}{\isaliteral{3A}{\isacharcolon}}nat{\isaliteral{7D}{\isacharbraceright}}\ {\isaliteral{3D}{\isacharequal}}\ Suc\ n{\isaliteral{2A}{\isacharasterisk}}{\isaliteral{28}{\isacharparenleft}}Suc\ n{\isaliteral{2B}{\isacharplus}}{\isadigit{1}}{\isaliteral{29}{\isacharparenright}}\ div\ {\isadigit{2}}{\isaliteral{22}{\isachardoublequoteclose}}\ \isacommand{by}\isamarkupfalse%
\ simp\isanewline
\isacommand{qed}\isamarkupfalse%
%
\endisatagproof
{\isafoldproof}%
%
\isadelimproof
%
\endisadelimproof
%
\begin{isamarkuptext}%
Except for the rewrite steps, everything is explicitly given. This
makes the proof easily readable, but the duplication means it is tedious to
write and maintain. Here is how pattern
matching can completely avoid any duplication:%
\end{isamarkuptext}%
\isamarkuptrue%
\isacommand{lemma}\isamarkupfalse%
\ {\isaliteral{22}{\isachardoublequoteopen}}{\isaliteral{5C3C53756D3E}{\isasymSum}}{\isaliteral{7B}{\isacharbraceleft}}{\isadigit{0}}{\isaliteral{2E}{\isachardot}}{\isaliteral{2E}{\isachardot}}n{\isaliteral{3A}{\isacharcolon}}{\isaliteral{3A}{\isacharcolon}}nat{\isaliteral{7D}{\isacharbraceright}}\ {\isaliteral{3D}{\isacharequal}}\ n{\isaliteral{2A}{\isacharasterisk}}{\isaliteral{28}{\isacharparenleft}}n{\isaliteral{2B}{\isacharplus}}{\isadigit{1}}{\isaliteral{29}{\isacharparenright}}\ div\ {\isadigit{2}}{\isaliteral{22}{\isachardoublequoteclose}}\ {\isaliteral{28}{\isacharparenleft}}\isakeyword{is}\ {\isaliteral{22}{\isachardoublequoteopen}}{\isaliteral{3F}{\isacharquery}}P\ n{\isaliteral{22}{\isachardoublequoteclose}}{\isaliteral{29}{\isacharparenright}}\isanewline
%
\isadelimproof
%
\endisadelimproof
%
\isatagproof
\isacommand{proof}\isamarkupfalse%
\ {\isaliteral{28}{\isacharparenleft}}induction\ n{\isaliteral{29}{\isacharparenright}}\isanewline
\ \ \isacommand{show}\isamarkupfalse%
\ {\isaliteral{22}{\isachardoublequoteopen}}{\isaliteral{3F}{\isacharquery}}P\ {\isadigit{0}}{\isaliteral{22}{\isachardoublequoteclose}}\ \isacommand{by}\isamarkupfalse%
\ simp\isanewline
\isacommand{next}\isamarkupfalse%
\isanewline
\ \ \isacommand{fix}\isamarkupfalse%
\ n\ \isacommand{assume}\isamarkupfalse%
\ {\isaliteral{22}{\isachardoublequoteopen}}{\isaliteral{3F}{\isacharquery}}P\ n{\isaliteral{22}{\isachardoublequoteclose}}\isanewline
\ \ \isacommand{thus}\isamarkupfalse%
\ {\isaliteral{22}{\isachardoublequoteopen}}{\isaliteral{3F}{\isacharquery}}P{\isaliteral{28}{\isacharparenleft}}Suc\ n{\isaliteral{29}{\isacharparenright}}{\isaliteral{22}{\isachardoublequoteclose}}\ \isacommand{by}\isamarkupfalse%
\ simp\isanewline
\isacommand{qed}\isamarkupfalse%
%
\endisatagproof
{\isafoldproof}%
%
\isadelimproof
%
\endisadelimproof
%
\begin{isamarkuptext}%
The first line introduces an abbreviation \isa{{\isaliteral{3F}{\isacharquery}}P\ n} for the goal.
Pattern matching \isa{{\isaliteral{3F}{\isacharquery}}P\ n} with the goal instantiates \isa{{\isaliteral{3F}{\isacharquery}}P} to the
function \isa{{\isaliteral{5C3C6C616D6264613E}{\isasymlambda}}n{\isaliteral{2E}{\isachardot}}\ {\isaliteral{5C3C53756D3E}{\isasymSum}}{\isaliteral{7B}{\isacharbraceleft}}{\isadigit{0}}{\isaliteral{2E}{\isachardot}}{\isaliteral{2E}{\isachardot}}n{\isaliteral{7D}{\isacharbraceright}}\ {\isaliteral{3D}{\isacharequal}}\ n\ {\isaliteral{2A}{\isacharasterisk}}\ {\isaliteral{28}{\isacharparenleft}}n\ {\isaliteral{2B}{\isacharplus}}\ {\isadigit{1}}{\isaliteral{29}{\isacharparenright}}\ div\ {\isadigit{2}}}.  Now the proposition to
be proved in the base case can be written as \isa{{\isaliteral{3F}{\isacharquery}}P\ {\isadigit{0}}}, the induction
hypothesis as \isa{{\isaliteral{3F}{\isacharquery}}P\ n}, and the conclusion of the induction step as
\isa{{\isaliteral{3F}{\isacharquery}}P{\isaliteral{28}{\isacharparenleft}}Suc\ n{\isaliteral{29}{\isacharparenright}}}.

Induction also provides the \isacom{case} idiom that abbreviates
the \isacom{fix}-\isacom{assume} step. The above proof becomes%
\end{isamarkuptext}%
\isamarkuptrue%
%
\isadelimproof
%
\endisadelimproof
%
\isatagproof
\isacommand{proof}\isamarkupfalse%
\ {\isaliteral{28}{\isacharparenleft}}induction\ n{\isaliteral{29}{\isacharparenright}}\isanewline
\ \ \isacommand{case}\isamarkupfalse%
\ {\isadigit{0}}\isanewline
\ \ \isacommand{show}\isamarkupfalse%
\ {\isaliteral{3F}{\isacharquery}}case\ \isacommand{by}\isamarkupfalse%
\ simp\isanewline
\isacommand{next}\isamarkupfalse%
\isanewline
\ \ \isacommand{case}\isamarkupfalse%
\ {\isaliteral{28}{\isacharparenleft}}Suc\ n{\isaliteral{29}{\isacharparenright}}\isanewline
\ \ \isacommand{thus}\isamarkupfalse%
\ {\isaliteral{3F}{\isacharquery}}case\ \isacommand{by}\isamarkupfalse%
\ simp\isanewline
\isacommand{qed}\isamarkupfalse%
%
\endisatagproof
{\isafoldproof}%
%
\isadelimproof
%
\endisadelimproof
%
\begin{isamarkuptext}%
The unknown \isa{{\isaliteral{3F}{\isacharquery}}case} is set in each case to the required
claim, i.e.\ \isa{{\isaliteral{3F}{\isacharquery}}P\ {\isadigit{0}}} and \mbox{\isa{{\isaliteral{3F}{\isacharquery}}P{\isaliteral{28}{\isacharparenleft}}Suc\ n{\isaliteral{29}{\isacharparenright}}}} in the above proof,
without requiring the user to define a \isa{{\isaliteral{3F}{\isacharquery}}P}. The general
pattern for induction over \isa{nat} is shown on the left-hand side:%
\end{isamarkuptext}%
\isamarkuptrue%
%
\begin{tabular}{@ {}ll@ {}}
\begin{minipage}[t]{.4\textwidth}
\isa{%
%
\isadelimproof
%
\endisadelimproof
%
\isatagproof
\isacommand{show}\isamarkupfalse%
\ {\isaliteral{22}{\isachardoublequoteopen}}P{\isaliteral{28}{\isacharparenleft}}n{\isaliteral{29}{\isacharparenright}}{\isaliteral{22}{\isachardoublequoteclose}}\isanewline
\isacommand{proof}\isamarkupfalse%
\ {\isaliteral{28}{\isacharparenleft}}induction\ n{\isaliteral{29}{\isacharparenright}}\isanewline
\ \ \isacommand{case}\isamarkupfalse%
\ {\isadigit{0}}%
\\\mbox{}\ \ $\vdots$\\\mbox{}\hspace{-1ex}
\ \ \isacommand{show}\isamarkupfalse%
\ {\isaliteral{3F}{\isacharquery}}case\ %
\ $\dots$\\
\isacommand{next}\isamarkupfalse%
\isanewline
\ \ \isacommand{case}\isamarkupfalse%
\ {\isaliteral{28}{\isacharparenleft}}Suc\ n{\isaliteral{29}{\isacharparenright}}%
\\\mbox{}\ \ $\vdots$\\\mbox{}\hspace{-1ex}
\ \ \isacommand{show}\isamarkupfalse%
\ {\isaliteral{3F}{\isacharquery}}case\ %
\ $\dots$\\
\isacommand{qed}\isamarkupfalse%
%
\endisatagproof
{\isafoldproof}%
%
\isadelimproof
%
\endisadelimproof
%
}
\end{minipage}
&
\begin{minipage}[t]{.4\textwidth}
~\\
~\\
\isacom{let} \isa{{\isaliteral{3F}{\isacharquery}}case\ {\isaliteral{3D}{\isacharequal}}\ {\isaliteral{22}{\isachardoublequote}}P{\isaliteral{28}{\isacharparenleft}}{\isadigit{0}}{\isaliteral{29}{\isacharparenright}}{\isaliteral{22}{\isachardoublequote}}}\\
~\\
~\\
~\\[1ex]
\isacom{fix} \isa{n} \isacom{assume} \isa{Suc{\isaliteral{3A}{\isacharcolon}}\ {\isaliteral{22}{\isachardoublequote}}P{\isaliteral{28}{\isacharparenleft}}n{\isaliteral{29}{\isacharparenright}}{\isaliteral{22}{\isachardoublequote}}}\\
\isacom{let} \isa{{\isaliteral{3F}{\isacharquery}}case\ {\isaliteral{3D}{\isacharequal}}\ {\isaliteral{22}{\isachardoublequote}}P{\isaliteral{28}{\isacharparenleft}}Suc\ n{\isaliteral{29}{\isacharparenright}}{\isaliteral{22}{\isachardoublequote}}}\\
\end{minipage}
\end{tabular}
\medskip
%
\begin{isamarkuptext}%
On the right side you can see what the \isacom{case} command
on the left stands for.

In case the goal is an implication, induction does one more thing: the
proposition to be proved in each case is not the whole implication but only
its conclusion; the premises of the implication are immediately made
assumptions of that case. That is, if in the above proof we replace
\isacom{show}~\isa{P{\isaliteral{28}{\isacharparenleft}}n{\isaliteral{29}{\isacharparenright}}} by
\mbox{\isacom{show}~\isa{A{\isaliteral{28}{\isacharparenleft}}n{\isaliteral{29}{\isacharparenright}}\ {\isaliteral{5C3C4C6F6E6772696768746172726F773E}{\isasymLongrightarrow}}\ P{\isaliteral{28}{\isacharparenleft}}n{\isaliteral{29}{\isacharparenright}}}}
then \isacom{case}~\isa{{\isadigit{0}}} stands for
\begin{quote}
\isacom{assume} \ \isa{{\isadigit{0}}{\isaliteral{3A}{\isacharcolon}}\ {\isaliteral{22}{\isachardoublequote}}A{\isaliteral{28}{\isacharparenleft}}{\isadigit{0}}{\isaliteral{29}{\isacharparenright}}{\isaliteral{22}{\isachardoublequote}}}\\
\isacom{let} \isa{{\isaliteral{3F}{\isacharquery}}case\ {\isaliteral{3D}{\isacharequal}}\ {\isaliteral{22}{\isachardoublequote}}P{\isaliteral{28}{\isacharparenleft}}{\isadigit{0}}{\isaliteral{29}{\isacharparenright}}{\isaliteral{22}{\isachardoublequote}}}
\end{quote}
and \isacom{case}~\isa{{\isaliteral{28}{\isacharparenleft}}Suc\ n{\isaliteral{29}{\isacharparenright}}} stands for
\begin{quote}
\isacom{fix} \isa{n}\\
\isacom{assume} \isa{Suc{\isaliteral{3A}{\isacharcolon}}}
  \begin{tabular}[t]{l}\isa{{\isaliteral{22}{\isachardoublequote}}A{\isaliteral{28}{\isacharparenleft}}n{\isaliteral{29}{\isacharparenright}}\ {\isaliteral{5C3C4C6F6E6772696768746172726F773E}{\isasymLongrightarrow}}\ P{\isaliteral{28}{\isacharparenleft}}n{\isaliteral{29}{\isacharparenright}}{\isaliteral{22}{\isachardoublequote}}}\\\isa{{\isaliteral{22}{\isachardoublequote}}A{\isaliteral{28}{\isacharparenleft}}Suc\ n{\isaliteral{29}{\isacharparenright}}{\isaliteral{22}{\isachardoublequote}}}\end{tabular}\\
\isacom{let} \isa{{\isaliteral{3F}{\isacharquery}}case\ {\isaliteral{3D}{\isacharequal}}\ {\isaliteral{22}{\isachardoublequote}}P{\isaliteral{28}{\isacharparenleft}}Suc\ n{\isaliteral{29}{\isacharparenright}}{\isaliteral{22}{\isachardoublequote}}}
\end{quote}
The list of assumptions \isa{Suc} is actually subdivided
into \isa{Suc{\isaliteral{2E}{\isachardot}}IH}, the induction hypotheses (here \isa{A{\isaliteral{28}{\isacharparenleft}}n{\isaliteral{29}{\isacharparenright}}\ {\isaliteral{5C3C4C6F6E6772696768746172726F773E}{\isasymLongrightarrow}}\ P{\isaliteral{28}{\isacharparenleft}}n{\isaliteral{29}{\isacharparenright}}})
and \isa{Suc{\isaliteral{2E}{\isachardot}}prems}, the premises of the goal being proved
(here \isa{A{\isaliteral{28}{\isacharparenleft}}Suc\ n{\isaliteral{29}{\isacharparenright}}}).

Induction works for any datatype.
Proving a goal \isa{{\isaliteral{5C3C6C6272616B6B3E}{\isasymlbrakk}}\ A\isaliteral{5C3C5E697375623E}{}\isactrlisub {\isadigit{1}}{\isaliteral{28}{\isacharparenleft}}x{\isaliteral{29}{\isacharparenright}}{\isaliteral{3B}{\isacharsemicolon}}\ {\isaliteral{5C3C646F74733E}{\isasymdots}}{\isaliteral{3B}{\isacharsemicolon}}\ A\isaliteral{5C3C5E697375623E}{}\isactrlisub k{\isaliteral{28}{\isacharparenleft}}x{\isaliteral{29}{\isacharparenright}}\ {\isaliteral{5C3C726272616B6B3E}{\isasymrbrakk}}\ {\isaliteral{5C3C4C6F6E6772696768746172726F773E}{\isasymLongrightarrow}}\ P{\isaliteral{28}{\isacharparenleft}}x{\isaliteral{29}{\isacharparenright}}}
by induction on \isa{x} generates a proof obligation for each constructor
\isa{C} of the datatype. The command \isa{case\ {\isaliteral{28}{\isacharparenleft}}C\ x\isaliteral{5C3C5E697375623E}{}\isactrlisub {\isadigit{1}}\ {\isaliteral{5C3C646F74733E}{\isasymdots}}\ x\isaliteral{5C3C5E697375623E}{}\isactrlisub n{\isaliteral{29}{\isacharparenright}}}
performs the following steps:
\begin{enumerate}
\item \isacom{fix} \isa{x\isaliteral{5C3C5E697375623E}{}\isactrlisub {\isadigit{1}}\ {\isaliteral{5C3C646F74733E}{\isasymdots}}\ x\isaliteral{5C3C5E697375623E}{}\isactrlisub n}
\item \isacom{assume} the induction hypotheses (calling them \isa{C{\isaliteral{2E}{\isachardot}}IH})
 and the premises \mbox{\isa{A\isaliteral{5C3C5E697375623E}{}\isactrlisub i{\isaliteral{28}{\isacharparenleft}}C\ x\isaliteral{5C3C5E697375623E}{}\isactrlisub {\isadigit{1}}\ {\isaliteral{5C3C646F74733E}{\isasymdots}}\ x\isaliteral{5C3C5E697375623E}{}\isactrlisub n{\isaliteral{29}{\isacharparenright}}}} (calling them \isa{C{\isaliteral{2E}{\isachardot}}prems})
 and calling the whole list \isa{C}
\item \isacom{let} \isa{{\isaliteral{3F}{\isacharquery}}case\ {\isaliteral{3D}{\isacharequal}}\ {\isaliteral{22}{\isachardoublequote}}P{\isaliteral{28}{\isacharparenleft}}C\ x\isaliteral{5C3C5E697375623E}{}\isactrlisub {\isadigit{1}}\ {\isaliteral{5C3C646F74733E}{\isasymdots}}\ x\isaliteral{5C3C5E697375623E}{}\isactrlisub n{\isaliteral{29}{\isacharparenright}}{\isaliteral{22}{\isachardoublequote}}}
\end{enumerate}

\subsection{Rule induction}

Recall the inductive and recursive definitions of even numbers in
\autoref{sec:inductive-defs}:%
\end{isamarkuptext}%
\isamarkuptrue%
\isacommand{inductive}\isamarkupfalse%
\ ev\ {\isaliteral{3A}{\isacharcolon}}{\isaliteral{3A}{\isacharcolon}}\ {\isaliteral{22}{\isachardoublequoteopen}}nat\ {\isaliteral{5C3C52696768746172726F773E}{\isasymRightarrow}}\ bool{\isaliteral{22}{\isachardoublequoteclose}}\ \isakeyword{where}\isanewline
ev{\isadigit{0}}{\isaliteral{3A}{\isacharcolon}}\ {\isaliteral{22}{\isachardoublequoteopen}}ev\ {\isadigit{0}}{\isaliteral{22}{\isachardoublequoteclose}}\ {\isaliteral{7C}{\isacharbar}}\isanewline
evSS{\isaliteral{3A}{\isacharcolon}}\ {\isaliteral{22}{\isachardoublequoteopen}}ev\ n\ {\isaliteral{5C3C4C6F6E6772696768746172726F773E}{\isasymLongrightarrow}}\ ev{\isaliteral{28}{\isacharparenleft}}Suc{\isaliteral{28}{\isacharparenleft}}Suc\ n{\isaliteral{29}{\isacharparenright}}{\isaliteral{29}{\isacharparenright}}{\isaliteral{22}{\isachardoublequoteclose}}\isanewline
\isanewline
\isacommand{fun}\isamarkupfalse%
\ even\ {\isaliteral{3A}{\isacharcolon}}{\isaliteral{3A}{\isacharcolon}}\ {\isaliteral{22}{\isachardoublequoteopen}}nat\ {\isaliteral{5C3C52696768746172726F773E}{\isasymRightarrow}}\ bool{\isaliteral{22}{\isachardoublequoteclose}}\ \isakeyword{where}\isanewline
{\isaliteral{22}{\isachardoublequoteopen}}even\ {\isadigit{0}}\ {\isaliteral{3D}{\isacharequal}}\ True{\isaliteral{22}{\isachardoublequoteclose}}\ {\isaliteral{7C}{\isacharbar}}\isanewline
{\isaliteral{22}{\isachardoublequoteopen}}even\ {\isaliteral{28}{\isacharparenleft}}Suc\ {\isadigit{0}}{\isaliteral{29}{\isacharparenright}}\ {\isaliteral{3D}{\isacharequal}}\ False{\isaliteral{22}{\isachardoublequoteclose}}\ {\isaliteral{7C}{\isacharbar}}\isanewline
{\isaliteral{22}{\isachardoublequoteopen}}even\ {\isaliteral{28}{\isacharparenleft}}Suc{\isaliteral{28}{\isacharparenleft}}Suc\ n{\isaliteral{29}{\isacharparenright}}{\isaliteral{29}{\isacharparenright}}\ {\isaliteral{3D}{\isacharequal}}\ even\ n{\isaliteral{22}{\isachardoublequoteclose}}%
\begin{isamarkuptext}%
We recast the proof of \isa{ev\ n\ {\isaliteral{5C3C4C6F6E6772696768746172726F773E}{\isasymLongrightarrow}}\ even\ n} in Isar. The
left column shows the actual proof text, the right column shows
the implicit effect of the two \isacom{case} commands:%
\end{isamarkuptext}%
\isamarkuptrue%
%
\begin{tabular}{@ {}l@ {\qquad}l@ {}}
\begin{minipage}[t]{.5\textwidth}
\isa{%
\isacommand{lemma}\isamarkupfalse%
\ {\isaliteral{22}{\isachardoublequoteopen}}ev\ n\ {\isaliteral{5C3C4C6F6E6772696768746172726F773E}{\isasymLongrightarrow}}\ even\ n{\isaliteral{22}{\isachardoublequoteclose}}\isanewline
%
\isadelimproof
%
\endisadelimproof
%
\isatagproof
\isacommand{proof}\isamarkupfalse%
{\isaliteral{28}{\isacharparenleft}}induction\ rule{\isaliteral{3A}{\isacharcolon}}\ ev{\isaliteral{2E}{\isachardot}}induct{\isaliteral{29}{\isacharparenright}}\isanewline
\ \ \isacommand{case}\isamarkupfalse%
\ ev{\isadigit{0}}\isanewline
\ \ \isacommand{show}\isamarkupfalse%
\ {\isaliteral{3F}{\isacharquery}}case\ \isacommand{by}\isamarkupfalse%
\ simp\isanewline
\isacommand{next}\isamarkupfalse%
\isanewline
\ \ \isacommand{case}\isamarkupfalse%
\ evSS\isanewline
\isanewline
\isanewline
\isanewline
\ \ \isacommand{thus}\isamarkupfalse%
\ {\isaliteral{3F}{\isacharquery}}case\ \isacommand{by}\isamarkupfalse%
\ simp\isanewline
\isacommand{qed}\isamarkupfalse%
%
\endisatagproof
{\isafoldproof}%
%
\isadelimproof
%
\endisadelimproof
%
}
\end{minipage}
&
\begin{minipage}[t]{.5\textwidth}
~\\
~\\
\isacom{let} \isa{{\isaliteral{3F}{\isacharquery}}case\ {\isaliteral{3D}{\isacharequal}}\ {\isaliteral{22}{\isachardoublequote}}even\ {\isadigit{0}}{\isaliteral{22}{\isachardoublequote}}}\\
~\\
~\\
\isacom{fix} \isa{n}\\
\isacom{assume} \isa{evSS{\isaliteral{3A}{\isacharcolon}}}
  \begin{tabular}[t]{l} \isa{{\isaliteral{22}{\isachardoublequote}}ev\ n{\isaliteral{22}{\isachardoublequote}}}\\\isa{{\isaliteral{22}{\isachardoublequote}}even\ n{\isaliteral{22}{\isachardoublequote}}}\end{tabular}\\
\isacom{let} \isa{{\isaliteral{3F}{\isacharquery}}case\ {\isaliteral{3D}{\isacharequal}}\ {\isaliteral{22}{\isachardoublequote}}even{\isaliteral{28}{\isacharparenleft}}Suc{\isaliteral{28}{\isacharparenleft}}Suc\ n{\isaliteral{29}{\isacharparenright}}{\isaliteral{29}{\isacharparenright}}{\isaliteral{22}{\isachardoublequote}}}\\
\end{minipage}
\end{tabular}
\medskip
%
\begin{isamarkuptext}%
The proof resembles structural induction, but the induction rule is given
explicitly and the names of the cases are the names of the rules in the
inductive definition.
Let us examine the two assumptions named \isa{evSS}:
\isa{ev\ n} is the premise of rule \isa{evSS}, which we may assume
because we are in the case where that rule was used; \isa{even\ n}
is the induction hypothesis.
\begin{warn}
Because each \isacom{case} command introduces a list of assumptions
named like the case name, which is the name of a rule of the inductive
definition, those rules now need to be accessed with a qualified name, here
\isa{ev{\isaliteral{2E}{\isachardot}}ev{\isadigit{0}}} and \isa{ev{\isaliteral{2E}{\isachardot}}evSS}
\end{warn}

In the case \isa{evSS} of the proof above we have pretended that the
system fixes a variable \isa{n}.  But unless the user provides the name
\isa{n}, the system will just invent its own name that cannot be referred
to.  In the above proof, we do not need to refer to it, hence we do not give
it a specific name. In case one needs to refer to it one writes
\begin{quote}
\isacom{case} \isa{{\isaliteral{28}{\isacharparenleft}}evSS\ m{\isaliteral{29}{\isacharparenright}}}
\end{quote}
just like \isacom{case}~\isa{{\isaliteral{28}{\isacharparenleft}}Suc\ n{\isaliteral{29}{\isacharparenright}}} in earlier structural inductions.
The name \isa{m} is an arbitrary choice. As a result,
case \isa{evSS} is derived from a renamed version of
rule \isa{evSS}: \isa{ev\ m\ {\isaliteral{5C3C4C6F6E6772696768746172726F773E}{\isasymLongrightarrow}}\ ev{\isaliteral{28}{\isacharparenleft}}Suc{\isaliteral{28}{\isacharparenleft}}Suc\ m{\isaliteral{29}{\isacharparenright}}{\isaliteral{29}{\isacharparenright}}}.
Here is an example with a (contrived) intermediate step that refers to \isa{m}:%
\end{isamarkuptext}%
\isamarkuptrue%
\isacommand{lemma}\isamarkupfalse%
\ {\isaliteral{22}{\isachardoublequoteopen}}ev\ n\ {\isaliteral{5C3C4C6F6E6772696768746172726F773E}{\isasymLongrightarrow}}\ even\ n{\isaliteral{22}{\isachardoublequoteclose}}\isanewline
%
\isadelimproof
%
\endisadelimproof
%
\isatagproof
\isacommand{proof}\isamarkupfalse%
{\isaliteral{28}{\isacharparenleft}}induction\ rule{\isaliteral{3A}{\isacharcolon}}\ ev{\isaliteral{2E}{\isachardot}}induct{\isaliteral{29}{\isacharparenright}}\isanewline
\ \ \isacommand{case}\isamarkupfalse%
\ ev{\isadigit{0}}\ \isacommand{show}\isamarkupfalse%
\ {\isaliteral{3F}{\isacharquery}}case\ \isacommand{by}\isamarkupfalse%
\ simp\isanewline
\isacommand{next}\isamarkupfalse%
\isanewline
\ \ \isacommand{case}\isamarkupfalse%
\ {\isaliteral{28}{\isacharparenleft}}evSS\ m{\isaliteral{29}{\isacharparenright}}\isanewline
\ \ \isacommand{have}\isamarkupfalse%
\ {\isaliteral{22}{\isachardoublequoteopen}}even{\isaliteral{28}{\isacharparenleft}}Suc{\isaliteral{28}{\isacharparenleft}}Suc\ m{\isaliteral{29}{\isacharparenright}}{\isaliteral{29}{\isacharparenright}}\ {\isaliteral{3D}{\isacharequal}}\ even\ m{\isaliteral{22}{\isachardoublequoteclose}}\ \isacommand{by}\isamarkupfalse%
\ simp\isanewline
\ \ \isacommand{thus}\isamarkupfalse%
\ {\isaliteral{3F}{\isacharquery}}case\ \isacommand{using}\isamarkupfalse%
\ {\isaliteral{60}{\isacharbackquoteopen}}even\ m{\isaliteral{60}{\isacharbackquoteclose}}\ \isacommand{by}\isamarkupfalse%
\ blast\isanewline
\isacommand{qed}\isamarkupfalse%
%
\endisatagproof
{\isafoldproof}%
%
\isadelimproof
%
\endisadelimproof
%
\begin{isamarkuptext}%
\indent
In general, let \isa{I} be a (for simplicity unary) inductively defined
predicate and let the rules in the definition of \isa{I}
be called \isa{rule\isaliteral{5C3C5E697375623E}{}\isactrlisub {\isadigit{1}}}, \dots, \isa{rule\isaliteral{5C3C5E697375623E}{}\isactrlisub n}. A proof by rule
induction follows this pattern:%
\end{isamarkuptext}%
\isamarkuptrue%
%
\isadelimproof
%
\endisadelimproof
%
\isatagproof
\isacommand{show}\isamarkupfalse%
\ {\isaliteral{22}{\isachardoublequoteopen}}I\ x\ {\isaliteral{5C3C4C6F6E6772696768746172726F773E}{\isasymLongrightarrow}}\ P\ x{\isaliteral{22}{\isachardoublequoteclose}}\isanewline
\isacommand{proof}\isamarkupfalse%
{\isaliteral{28}{\isacharparenleft}}induction\ rule{\isaliteral{3A}{\isacharcolon}}\ I{\isaliteral{2E}{\isachardot}}induct{\isaliteral{29}{\isacharparenright}}\isanewline
\ \ \isacommand{case}\isamarkupfalse%
\ rule\isaliteral{5C3C5E697375623E}{}\isactrlisub {\isadigit{1}}%
\\[-.4ex]\mbox{}\ \ $\vdots$\\[-.4ex]\mbox{}\hspace{-1ex}
\ \ \isacommand{show}\isamarkupfalse%
\ {\isaliteral{3F}{\isacharquery}}case\ %
\ $\dots$\\
\isacommand{next}\isamarkupfalse%
%
\\[-.4ex]$\vdots$\\[-.4ex]\mbox{}\hspace{-1ex}
\isacommand{next}\isamarkupfalse%
\isanewline
\ \ \isacommand{case}\isamarkupfalse%
\ rule\isaliteral{5C3C5E697375623E}{}\isactrlisub n%
\\[-.4ex]\mbox{}\ \ $\vdots$\\[-.4ex]\mbox{}\hspace{-1ex}
\ \ \isacommand{show}\isamarkupfalse%
\ {\isaliteral{3F}{\isacharquery}}case\ %
\ $\dots$\\
\isacommand{qed}\isamarkupfalse%
%
\endisatagproof
{\isafoldproof}%
%
\isadelimproof
%
\endisadelimproof
%
\begin{isamarkuptext}%
One can provide explicit variable names by writing
\isacom{case}~\isa{{\isaliteral{28}{\isacharparenleft}}rule\isaliteral{5C3C5E697375623E}{}\isactrlisub i\ x\isaliteral{5C3C5E697375623E}{}\isactrlisub {\isadigit{1}}\ {\isaliteral{5C3C646F74733E}{\isasymdots}}\ x\isaliteral{5C3C5E697375623E}{}\isactrlisub k{\isaliteral{29}{\isacharparenright}}}, thus renaming the first \isa{k}
free variables in rule \isa{i} to \isa{x\isaliteral{5C3C5E697375623E}{}\isactrlisub {\isadigit{1}}\ {\isaliteral{5C3C646F74733E}{\isasymdots}}\ x\isaliteral{5C3C5E697375623E}{}\isactrlisub k},
going through rule \isa{i} from left to right.

\subsection{Assumption naming}

In any induction, \isacom{case}~\isa{name} sets up a list of assumptions
also called \isa{name}, which is subdivided into three parts:
\begin{description}
\item[\isa{name{\isaliteral{2E}{\isachardot}}IH}] contains the induction hypotheses.
\item[\isa{name{\isaliteral{2E}{\isachardot}}hyps}] contains all the other hypotheses of this case in the
induction rule. For rule inductions these are the hypotheses of rule
\isa{name}, for structural inductions these are empty.
\item[\isa{name{\isaliteral{2E}{\isachardot}}prems}] contains the (suitably instantiated) premises
of the statement being proved, i.e. the \isa{A\isaliteral{5C3C5E697375623E}{}\isactrlisub i} when
proving \isa{{\isaliteral{5C3C6C6272616B6B3E}{\isasymlbrakk}}\ A\isaliteral{5C3C5E697375623E}{}\isactrlisub {\isadigit{1}}{\isaliteral{3B}{\isacharsemicolon}}\ {\isaliteral{5C3C646F74733E}{\isasymdots}}{\isaliteral{3B}{\isacharsemicolon}}\ A\isaliteral{5C3C5E697375623E}{}\isactrlisub n\ {\isaliteral{5C3C726272616B6B3E}{\isasymrbrakk}}\ {\isaliteral{5C3C4C6F6E6772696768746172726F773E}{\isasymLongrightarrow}}\ A}.
\end{description}
\begin{warn}
Proof method \isa{induct} differs from \isa{induction}
only in this naming policy: \isa{induct} does not distinguish
\isa{IH} from \isa{hyps} but subsumes \isa{IH} under \isa{hyps}.
\end{warn}

More complicated inductive proofs than the ones we have seen so far
often need to refer to specific assumptions---just \isa{name} or even
\isa{name{\isaliteral{2E}{\isachardot}}prems} and \isa{name{\isaliteral{2E}{\isachardot}}IH} can be too unspecific.
This is where the indexing of fact lists comes in handy, e.g.\
\isa{name{\isaliteral{2E}{\isachardot}}IH{\isaliteral{28}{\isacharparenleft}}{\isadigit{2}}{\isaliteral{29}{\isacharparenright}}} or \isa{name{\isaliteral{2E}{\isachardot}}prems{\isaliteral{28}{\isacharparenleft}}{\isadigit{1}}{\isaliteral{2D}{\isacharminus}}{\isadigit{2}}{\isaliteral{29}{\isacharparenright}}}.

\subsection{Rule inversion}

Rule inversion is case distinction on which rule could have been used to
derive some fact. The name \concept{rule inversion} emphasizes that we are
reasoning backwards: by which rules could some given fact have been proved?
For the inductive definition of \isa{ev}, rule inversion can be summarized
like this:
\begin{isabelle}%
ev\ n\ {\isaliteral{5C3C4C6F6E6772696768746172726F773E}{\isasymLongrightarrow}}\ n\ {\isaliteral{3D}{\isacharequal}}\ {\isadigit{0}}\ {\isaliteral{5C3C6F723E}{\isasymor}}\ {\isaliteral{28}{\isacharparenleft}}{\isaliteral{5C3C6578697374733E}{\isasymexists}}k{\isaliteral{2E}{\isachardot}}\ n\ {\isaliteral{3D}{\isacharequal}}\ Suc\ {\isaliteral{28}{\isacharparenleft}}Suc\ k{\isaliteral{29}{\isacharparenright}}\ {\isaliteral{5C3C616E643E}{\isasymand}}\ ev\ k{\isaliteral{29}{\isacharparenright}}%
\end{isabelle}
The realisation in Isabelle is a case distinction.
A simple example is the proof that \isa{ev\ n\ {\isaliteral{5C3C4C6F6E6772696768746172726F773E}{\isasymLongrightarrow}}\ ev\ {\isaliteral{28}{\isacharparenleft}}n\ {\isaliteral{2D}{\isacharminus}}\ {\isadigit{2}}{\isaliteral{29}{\isacharparenright}}}. We
already went through the details informally in \autoref{sec:Logic:even}. This
is the Isar proof:%
\end{isamarkuptext}%
\isamarkuptrue%
%
\isadelimproof
%
\endisadelimproof
%
\isatagproof
\ \ \isacommand{assume}\isamarkupfalse%
\ {\isaliteral{22}{\isachardoublequoteopen}}ev\ n{\isaliteral{22}{\isachardoublequoteclose}}\isanewline
\ \ \isacommand{from}\isamarkupfalse%
\ this\ \isacommand{have}\isamarkupfalse%
\ {\isaliteral{22}{\isachardoublequoteopen}}ev{\isaliteral{28}{\isacharparenleft}}n\ {\isaliteral{2D}{\isacharminus}}\ {\isadigit{2}}{\isaliteral{29}{\isacharparenright}}{\isaliteral{22}{\isachardoublequoteclose}}\isanewline
\ \ \isacommand{proof}\isamarkupfalse%
\ cases\isanewline
\ \ \ \ \isacommand{case}\isamarkupfalse%
\ ev{\isadigit{0}}\ \isacommand{thus}\isamarkupfalse%
\ {\isaliteral{22}{\isachardoublequoteopen}}ev{\isaliteral{28}{\isacharparenleft}}n\ {\isaliteral{2D}{\isacharminus}}\ {\isadigit{2}}{\isaliteral{29}{\isacharparenright}}{\isaliteral{22}{\isachardoublequoteclose}}\ \isacommand{by}\isamarkupfalse%
\ {\isaliteral{28}{\isacharparenleft}}simp\ add{\isaliteral{3A}{\isacharcolon}}\ ev{\isaliteral{2E}{\isachardot}}ev{\isadigit{0}}{\isaliteral{29}{\isacharparenright}}\isanewline
\ \ \isacommand{next}\isamarkupfalse%
\isanewline
\ \ \ \ \isacommand{case}\isamarkupfalse%
\ {\isaliteral{28}{\isacharparenleft}}evSS\ k{\isaliteral{29}{\isacharparenright}}\ \isacommand{thus}\isamarkupfalse%
\ {\isaliteral{22}{\isachardoublequoteopen}}ev{\isaliteral{28}{\isacharparenleft}}n\ {\isaliteral{2D}{\isacharminus}}\ {\isadigit{2}}{\isaliteral{29}{\isacharparenright}}{\isaliteral{22}{\isachardoublequoteclose}}\ \isacommand{by}\isamarkupfalse%
\ {\isaliteral{28}{\isacharparenleft}}simp\ add{\isaliteral{3A}{\isacharcolon}}\ ev{\isaliteral{2E}{\isachardot}}evSS{\isaliteral{29}{\isacharparenright}}\isanewline
\ \ \isacommand{qed}\isamarkupfalse%
%
\endisatagproof
{\isafoldproof}%
%
\isadelimproof
%
\endisadelimproof
%
\begin{isamarkuptext}%
The key point here is that a case distinction over some inductively
defined predicate is triggered by piping the given fact
(here: \isacom{from}~\isa{this}) into a proof by \isa{cases}.
Let us examine the assumptions available in each case. In case \isa{ev{\isadigit{0}}}
we have \isa{n\ {\isaliteral{3D}{\isacharequal}}\ {\isadigit{0}}} and in case \isa{evSS} we have \isa{n\ {\isaliteral{3D}{\isacharequal}}\ Suc\ {\isaliteral{28}{\isacharparenleft}}Suc\ k{\isaliteral{29}{\isacharparenright}}}
and \isa{ev\ k}. In each case the assumptions are available under the name
of the case; there is no fine grained naming schema like for induction.

Sometimes some rules could not have been used to derive the given fact
because constructors clash. As an extreme example consider
rule inversion applied to \isa{ev\ {\isaliteral{28}{\isacharparenleft}}Suc\ {\isadigit{0}}{\isaliteral{29}{\isacharparenright}}}: neither rule \isa{ev{\isadigit{0}}} nor
rule \isa{evSS} can yield \isa{ev\ {\isaliteral{28}{\isacharparenleft}}Suc\ {\isadigit{0}}{\isaliteral{29}{\isacharparenright}}} because \isa{Suc\ {\isadigit{0}}} unifies
neither with \isa{{\isadigit{0}}} nor with \isa{Suc\ {\isaliteral{28}{\isacharparenleft}}Suc\ n{\isaliteral{29}{\isacharparenright}}}. Impossible cases do not
have to be proved. Hence we can prove anything from \isa{ev\ {\isaliteral{28}{\isacharparenleft}}Suc\ {\isadigit{0}}{\isaliteral{29}{\isacharparenright}}}:%
\end{isamarkuptext}%
\isamarkuptrue%
%
\isadelimproof
%
\endisadelimproof
%
\isatagproof
\ \ \isacommand{assume}\isamarkupfalse%
\ {\isaliteral{22}{\isachardoublequoteopen}}ev{\isaliteral{28}{\isacharparenleft}}Suc\ {\isadigit{0}}{\isaliteral{29}{\isacharparenright}}{\isaliteral{22}{\isachardoublequoteclose}}\ \isacommand{then}\isamarkupfalse%
\ \isacommand{have}\isamarkupfalse%
\ P\ \isacommand{by}\isamarkupfalse%
\ cases%
\endisatagproof
{\isafoldproof}%
%
\isadelimproof
%
\endisadelimproof
%
\begin{isamarkuptext}%
That is, \isa{ev\ {\isaliteral{28}{\isacharparenleft}}Suc\ {\isadigit{0}}{\isaliteral{29}{\isacharparenright}}} is simply not provable:%
\end{isamarkuptext}%
\isamarkuptrue%
\isacommand{lemma}\isamarkupfalse%
\ {\isaliteral{22}{\isachardoublequoteopen}}{\isaliteral{5C3C6E6F743E}{\isasymnot}}\ ev{\isaliteral{28}{\isacharparenleft}}Suc\ {\isadigit{0}}{\isaliteral{29}{\isacharparenright}}{\isaliteral{22}{\isachardoublequoteclose}}\isanewline
%
\isadelimproof
%
\endisadelimproof
%
\isatagproof
\isacommand{proof}\isamarkupfalse%
\isanewline
\ \ \isacommand{assume}\isamarkupfalse%
\ {\isaliteral{22}{\isachardoublequoteopen}}ev{\isaliteral{28}{\isacharparenleft}}Suc\ {\isadigit{0}}{\isaliteral{29}{\isacharparenright}}{\isaliteral{22}{\isachardoublequoteclose}}\ \isacommand{then}\isamarkupfalse%
\ \isacommand{show}\isamarkupfalse%
\ False\ \isacommand{by}\isamarkupfalse%
\ cases\isanewline
\isacommand{qed}\isamarkupfalse%
%
\endisatagproof
{\isafoldproof}%
%
\isadelimproof
%
\endisadelimproof
%
\begin{isamarkuptext}%
Normally not all cases will be impossible. As a simple exercise,
prove that \mbox{\isa{{\isaliteral{5C3C6E6F743E}{\isasymnot}}\ ev\ {\isaliteral{28}{\isacharparenleft}}Suc\ {\isaliteral{28}{\isacharparenleft}}Suc\ {\isaliteral{28}{\isacharparenleft}}Suc\ {\isadigit{0}}{\isaliteral{29}{\isacharparenright}}{\isaliteral{29}{\isacharparenright}}{\isaliteral{29}{\isacharparenright}}}.}%
\end{isamarkuptext}%
\isamarkuptrue%
%
\isadelimtheory
%
\endisadelimtheory
%
\isatagtheory
%
\endisatagtheory
{\isafoldtheory}%
%
\isadelimtheory
%
\endisadelimtheory
\end{isabellebody}%
%%% Local Variables:
%%% mode: latex
%%% TeX-master: "root"
%%% End:


%%% Local Variables:
%%% mode: latex
%%% TeX-master: "root"
%%% End:

{\small
\paragraph{Acknowledgment}
I am deeply indebted to Markus Wenzel for conceiving Isar. Clemens Ballarin,
Gertrud Bauer, Stefan Berghofer, Gerwin Klein, Norbert Schirmer and
Markus Wenzel commented on and improved this document.
}

\begingroup
\bibliographystyle{plain} \small\raggedright\frenchspacing
\bibliography{root}
\endgroup

\end{document}
