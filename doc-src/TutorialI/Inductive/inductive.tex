\chapter{Inductively Defined Sets} \label{chap:inductive}
\index{inductive definitions|(}

This chapter is dedicated to the most important definition principle after
recursive functions and datatypes: inductively defined sets.

We start with a simple example: the set of even numbers.  A slightly more
complicated example, the reflexive transitive closure, is the subject of
{\S}\ref{sec:rtc}. In particular, some standard induction heuristics are
discussed. Advanced forms of inductive definitions are discussed in
{\S}\ref{sec:adv-ind-def}. To demonstrate the versatility of inductive
definitions, the chapter closes with a case study from the realm of
context-free grammars. The first two sections are required reading for anybody
interested in mathematical modelling.

\begin{warn}
Predicates can also be defined inductively.
See {\S}\ref{sec:ind-predicates}.
\end{warn}

% ID:         $Id$
\section{The set of even numbers}

The set of even numbers can be inductively defined as the least set
containing 0 and closed under the operation ${\cdots}+2$.  Obviously,
\emph{even} can also be expressed using the divides relation (\isa{dvd}). 
We shall prove below that the two formulations coincide.  On the way we
shall examine the primary means of reasoning about inductively defined
sets: rule induction.

\subsection{Making an inductive definition}

Using \isacommand{consts}, we declare the constant \isa{even} to be a set
of natural numbers. The \isacommand{inductive} declaration gives it the
desired properties.
\begin{isabelle}
\isacommand{consts}\ even\ ::\ "nat\ set"\isanewline
\isacommand{inductive}\ even\isanewline
\isakeyword{intros}\isanewline
zero[intro!]:\ "0\ \isasymin \ even"\isanewline
step[intro!]:\ "n\ \isasymin \ even\ \isasymLongrightarrow \ (Suc\ (Suc\
n))\ \isasymin \ even"
\end{isabelle}

An inductive definition consists of introduction rules.  The first one
above states that 0 is even; the second states that if $n$ is even, then so
is
$n+2$.  Given this declaration, Isabelle generates a fixed point definition
for \isa{even} and proves theorems about it.  These theorems include the
introduction rules specified in the declaration, an elimination rule for case
analysis and an induction rule.  We can refer to these theorems by
automatically-generated names.  Here are two examples:
%
\begin{isabelle}
n\ \isasymin \ even\ \isasymLongrightarrow \ Suc\ (Suc\ n)\ \isasymin \
even%
\rulename{even.step}
\par\medskip
\isasymlbrakk xa\ \isasymin \ even;\isanewline
\ P\ 0;\isanewline
\ \isasymAnd n.\ \isasymlbrakk n\ \isasymin \ even;\ P\ n\isasymrbrakk \
\isasymLongrightarrow \ P\ (Suc\ (Suc\ n))\isasymrbrakk\isanewline
\ \isasymLongrightarrow \ P\ xa%
\rulename{even.induct}
\end{isabelle}

The introduction rules can be given attributes.  Here both rules are
specified as \isa{intro!}, directing the classical reasoner to 
apply them aggressively. Obviously, regarding 0 as even is safe.  The
\isa{step} rule is also safe because $n+2$ is even if and only if $n$ is
even.  We prove this equivalence later.

\subsection{Using introduction rules}

Our first lemma states that numbers of the form $2\times k$ are even.
Introduction rules are used to show that specific values belong to the
inductive set.  Such proofs typically involve 
induction, perhaps over some other inductive set.
\begin{isabelle}
\isacommand{lemma}\ two_times_even[intro!]:\ "\#2*k\ \isasymin \ even"
\isanewline
\isacommand{apply}\ (induct\ "k")\isanewline
\ \isacommand{apply}\ auto\isanewline
\isacommand{done}
\end{isabelle}
%
The first step is induction on the natural number \isa{k}, which leaves
two subgoals:
\begin{isabelle}
\ 1.\ \#2\ *\ 0\ \isasymin \ even\isanewline
\ 2.\ \isasymAnd n.\ \#2\ *\ n\ \isasymin \ even\ \isasymLongrightarrow \ \#2\ *\ Suc\ n\ \isasymin \ even
\end{isabelle}
%
Here \isa{auto} simplifies both subgoals so that they match the introduction
rules, which are then applied automatically.

Our ultimate goal is to prove the equivalence between the traditional
definition of \isa{even} (using the divides relation) and our inductive
definition.  One direction of this equivalence is immediate by the lemma
just proved, whose \isa{intro!} attribute ensures it will be used.
\begin{isabelle}
\isacommand{lemma}\ dvd_imp_even:\ "\#2\ dvd\ n\ \isasymLongrightarrow \ n\ \isasymin \ even"\isanewline
\isacommand{apply}\ (auto\ simp\ add:\ dvd_def)\isanewline
\isacommand{done}
\end{isabelle}

\subsection{Rule induction}

The other direction of this equivalence is proved by induction over the set
\isa{even}.   This type of inductive argument is called \textbf{rule induction}. 
It is the usual way of proving a property of the elements of an inductively
defined set.
\begin{isabelle}
\isacommand{lemma}\ even_imp_dvd:\ "n\ \isasymin \ even\ \isasymLongrightarrow \ \#2\ dvd\ n"\isanewline
\isacommand{apply}\ (erule\ even.induct)\isanewline
\ \isacommand{apply}\ simp\isanewline
\isacommand{apply}\ (simp\ add:\ dvd_def)\isanewline
\isacommand{apply}\ clarify\isanewline
\isacommand{apply}\ (rule_tac\ x\ =\ "Suc\ k"\ \isakeyword{in}\ exI)\isanewline
\isacommand{apply}\ simp\isanewline
\isacommand{done}
\end{isabelle}
%
We begin by applying induction.  Note that \isa{even.induct} has the form
of an elimination rule, so we use the method \isa{erule}.  We get two
subgoals:
\begin{isabelle}
\ 1.\ \#2\ dvd\ 0\isanewline
\ 2.\ \isasymAnd n.\ \isasymlbrakk n\ \isasymin \ even;\ \#2\ dvd\ n\isasymrbrakk \ \isasymLongrightarrow \ \#2\ dvd\ Suc\ (Suc\ n)
\end{isabelle}
%
The first subgoal is trivial (by \isa{simp}).  For the second
subgoal, we unfold the definition of \isa{dvd}:
\begin{isabelle}
\ 1.\ \isasymAnd n.\ \isasymlbrakk n\ \isasymin \ even;\ \isasymexists k.\
n\ =\ \#2\ *\ k\isasymrbrakk \ \isasymLongrightarrow \ \isasymexists k.\
Suc\ (Suc\ n)\ =\ \#2\ *\ k
\end{isabelle}
%
Then we use
\isa{clarify} to eliminate the existential quantifier from the assumption
and replace \isa{n} by \isa{\#2\ *\ k}.
\begin{isabelle}
\ 1.\ \isasymAnd n\ k.\ \#2\ *\ k\ \isasymin \ even\ \isasymLongrightarrow \ \isasymexists ka.\ Suc\ (Suc\ (\#2\ *\ k))\ =\ \#2\ *\ ka%
\end{isabelle}
%
The \isa{rule_tac\ldots exI} tells Isabelle that the desired
\isa{ka} is
\isa{Suc\ k}.  With this hint, the subgoal becomes trivial, and \isa{simp}
concludes the proof.

\medskip
Combining the previous two results yields our objective, the
equivalence relating \isa{even} and \isa{dvd}. 
%
%we don't want [iff]: discuss?
\begin{isabelle}
\isacommand{theorem}\ even_iff_dvd:\ "(n\ \isasymin \ even)\ =\ (\#2\ dvd\ n)"\isanewline
\isacommand{apply}\ (blast\ intro:\ dvd_imp_even\ even_imp_dvd)\isanewline
\isacommand{done}
\end{isabelle}

\subsection{Generalization and rule induction}

Everybody knows that when before applying induction we often must generalize
the induction formula.  This step is just as important with rule induction,
and the required generalizations can be complicated.  In this 
example, the obvious statement of the result is not inductive:
%
\begin{isabelle}
\isacommand{lemma}\ "Suc\ (Suc\ n)\ \isasymin \ even\
\isasymLongrightarrow \ n\ \isasymin \ even"\isanewline
\isacommand{apply}\ (erule\ even.induct)\isanewline
\isacommand{oops}
\end{isabelle}
%
Rule induction finds no occurrences of \isa{Suc\ (Suc\ n)} in the
conclusion, which it therefore leaves unchanged.  (Look at
\isa{even.induct} to see why this happens.)  We get these subgoals:
\begin{isabelle}
\ 1.\ n\ \isasymin \ even\isanewline
\ 2.\ \isasymAnd na.\ \isasymlbrakk na\ \isasymin \ even;\ n\ \isasymin \ even\isasymrbrakk \ \isasymLongrightarrow \ n\ \isasymin \ even%
\end{isabelle}
The first one is hopeless.  In general, rule inductions involving
non-trivial terms will probably go wrong.  The solution is easy provided
we have the necessary inverses, here subtraction:
\begin{isabelle}
\isacommand{lemma}\ even_imp_even_minus_2:\ "n\ \isasymin \ even\ \isasymLongrightarrow \ n-\#2\ \isasymin \ even"\isanewline
\isacommand{apply}\ (erule\ even.induct)\isanewline
\ \isacommand{apply}\ auto\isanewline
\isacommand{done}
\end{isabelle}
%
This lemma is trivially inductive.  Here are the subgoals:
\begin{isabelle}
\ 1.\ 0\ -\ \#2\ \isasymin \ even\isanewline
\ 2.\ \isasymAnd n.\ \isasymlbrakk n\ \isasymin \ even;\ n\ -\ \#2\ \isasymin \ even\isasymrbrakk \ \isasymLongrightarrow \ Suc\ (Suc\ n)\ -\ \#2\ \isasymin \ even%
\end{isabelle}
The first is trivial because \isa{0\ -\ \#2} simplifies to \isa{0}, which is
even.  The second is trivial too: \isa{Suc\ (Suc\ n)\ -\ \#2} simplifies to
\isa{n}, matching the assumption.

\medskip
Using our lemma, we can easily prove the result we originally wanted:
\begin{isabelle}
\isacommand{lemma}\ Suc_Suc_even_imp_even:\ "Suc\ (Suc\ n)\ \isasymin \ even\ \isasymLongrightarrow \ n\ \isasymin \ even"\isanewline
\isacommand{apply}\ (drule\ even_imp_even_minus_2)\isanewline
\isacommand{apply}\ simp\isanewline
\isacommand{done}
\end{isabelle}

We have just proved the converse of the introduction rule \isa{even.step}. 
This suggests proving the following equivalence.  We give it the \isa{iff}
attribute because of its obvious value for simplification.
\begin{isabelle}
\isacommand{lemma}\ [iff]:\ "((Suc\ (Suc\ n))\ \isasymin \ even)\ =\ (n\
\isasymin \ even)"\isanewline
\isacommand{apply}\ (blast\ dest:\ Suc_Suc_even_imp_even)\isanewline
\isacommand{done}\isanewline
\end{isabelle}

%
\begin{isabellebody}%
\def\isabellecontext{Mutual}%
\isamarkupfalse%
%
\isamarkupsubsection{Mutually Inductive Definitions%
}
\isamarkuptrue%
%
\begin{isamarkuptext}%
Just as there are datatypes defined by mutual recursion, there are sets defined
by mutual induction. As a trivial example we consider the even and odd
natural numbers:%
\end{isamarkuptext}%
\isamarkuptrue%
\isacommand{consts}\ even\ {\isacharcolon}{\isacharcolon}\ {\isachardoublequote}nat\ set{\isachardoublequote}\isanewline
\ \ \ \ \ \ \ odd\ \ {\isacharcolon}{\isacharcolon}\ {\isachardoublequote}nat\ set{\isachardoublequote}\isanewline
\isanewline
\isamarkupfalse%
\isacommand{inductive}\ even\ odd\isanewline
\isakeyword{intros}\isanewline
zero{\isacharcolon}\ \ {\isachardoublequote}{\isadigit{0}}\ {\isasymin}\ even{\isachardoublequote}\isanewline
evenI{\isacharcolon}\ {\isachardoublequote}n\ {\isasymin}\ odd\ {\isasymLongrightarrow}\ Suc\ n\ {\isasymin}\ even{\isachardoublequote}\isanewline
oddI{\isacharcolon}\ \ {\isachardoublequote}n\ {\isasymin}\ even\ {\isasymLongrightarrow}\ Suc\ n\ {\isasymin}\ odd{\isachardoublequote}\isamarkupfalse%
%
\begin{isamarkuptext}%
\noindent
The mutually inductive definition of multiple sets is no different from
that of a single set, except for induction: just as for mutually recursive
datatypes, induction needs to involve all the simultaneously defined sets. In
the above case, the induction rule is called \isa{even{\isacharunderscore}odd{\isachardot}induct}
(simply concatenate the names of the sets involved) and has the conclusion
\begin{isabelle}%
\ \ \ \ \ {\isacharparenleft}{\isacharquery}x\ {\isasymin}\ even\ {\isasymlongrightarrow}\ {\isacharquery}P\ {\isacharquery}x{\isacharparenright}\ {\isasymand}\ {\isacharparenleft}{\isacharquery}y\ {\isasymin}\ odd\ {\isasymlongrightarrow}\ {\isacharquery}Q\ {\isacharquery}y{\isacharparenright}%
\end{isabelle}

If we want to prove that all even numbers are divisible by two, we have to
generalize the statement as follows:%
\end{isamarkuptext}%
\isamarkuptrue%
\isacommand{lemma}\ {\isachardoublequote}{\isacharparenleft}m\ {\isasymin}\ even\ {\isasymlongrightarrow}\ {\isadigit{2}}\ dvd\ m{\isacharparenright}\ {\isasymand}\ {\isacharparenleft}n\ {\isasymin}\ odd\ {\isasymlongrightarrow}\ {\isadigit{2}}\ dvd\ {\isacharparenleft}Suc\ n{\isacharparenright}{\isacharparenright}{\isachardoublequote}\isamarkupfalse%
%
\begin{isamarkuptxt}%
\noindent
The proof is by rule induction. Because of the form of the induction theorem,
it is applied by \isa{rule} rather than \isa{erule} as for ordinary
inductive definitions:%
\end{isamarkuptxt}%
\isamarkuptrue%
\isacommand{apply}{\isacharparenleft}rule\ even{\isacharunderscore}odd{\isachardot}induct{\isacharparenright}\isamarkupfalse%
%
\begin{isamarkuptxt}%
\begin{isabelle}%
\ {\isadigit{1}}{\isachardot}\ {\isadigit{2}}\ dvd\ {\isadigit{0}}\isanewline
\ {\isadigit{2}}{\isachardot}\ {\isasymAnd}n{\isachardot}\ {\isasymlbrakk}n\ {\isasymin}\ odd{\isacharsemicolon}\ {\isadigit{2}}\ dvd\ Suc\ n{\isasymrbrakk}\ {\isasymLongrightarrow}\ {\isadigit{2}}\ dvd\ Suc\ n\isanewline
\ {\isadigit{3}}{\isachardot}\ {\isasymAnd}n{\isachardot}\ {\isasymlbrakk}n\ {\isasymin}\ even{\isacharsemicolon}\ {\isadigit{2}}\ dvd\ n{\isasymrbrakk}\ {\isasymLongrightarrow}\ {\isadigit{2}}\ dvd\ Suc\ {\isacharparenleft}Suc\ n{\isacharparenright}%
\end{isabelle}
The first two subgoals are proved by simplification and the final one can be
proved in the same manner as in \S\ref{sec:rule-induction}
where the same subgoal was encountered before.
We do not show the proof script.%
\end{isamarkuptxt}%
\isamarkuptrue%
\isamarkupfalse%
\isamarkupfalse%
\isamarkupfalse%
\isamarkupfalse%
\isamarkupfalse%
\isamarkupfalse%
\isanewline
\isamarkupfalse%
\isamarkupfalse%
\end{isabellebody}%
%%% Local Variables:
%%% mode: latex
%%% TeX-master: "root"
%%% End:

%
\begin{isabellebody}%
\def\isabellecontext{Star}%
%
\isadelimtheory
%
\endisadelimtheory
%
\isatagtheory
%
\endisatagtheory
{\isafoldtheory}%
%
\isadelimtheory
%
\endisadelimtheory
%
\isamarkupsection{The Reflexive Transitive Closure%
}
\isamarkuptrue%
%
\begin{isamarkuptext}%
\label{sec:rtc}
\index{reflexive transitive closure!defining inductively|(}%
An inductive definition may accept parameters, so it can express 
functions that yield sets.
Relations too can be defined inductively, since they are just sets of pairs.
A perfect example is the function that maps a relation to its
reflexive transitive closure.  This concept was already
introduced in \S\ref{sec:Relations}, where the operator \isa{\isaliteral{5C3C5E7375703E}{}\isactrlsup {\isaliteral{2A}{\isacharasterisk}}} was
defined as a least fixed point because inductive definitions were not yet
available. But now they are:%
\end{isamarkuptext}%
\isamarkuptrue%
\isacommand{inductive{\isaliteral{5F}{\isacharunderscore}}set}\isamarkupfalse%
\isanewline
\ \ rtc\ {\isaliteral{3A}{\isacharcolon}}{\isaliteral{3A}{\isacharcolon}}\ {\isaliteral{22}{\isachardoublequoteopen}}{\isaliteral{28}{\isacharparenleft}}{\isaliteral{27}{\isacharprime}}a\ {\isaliteral{5C3C74696D65733E}{\isasymtimes}}\ {\isaliteral{27}{\isacharprime}}a{\isaliteral{29}{\isacharparenright}}set\ {\isaliteral{5C3C52696768746172726F773E}{\isasymRightarrow}}\ {\isaliteral{28}{\isacharparenleft}}{\isaliteral{27}{\isacharprime}}a\ {\isaliteral{5C3C74696D65733E}{\isasymtimes}}\ {\isaliteral{27}{\isacharprime}}a{\isaliteral{29}{\isacharparenright}}set{\isaliteral{22}{\isachardoublequoteclose}}\ \ \ {\isaliteral{28}{\isacharparenleft}}{\isaliteral{22}{\isachardoublequoteopen}}{\isaliteral{5F}{\isacharunderscore}}{\isaliteral{2A}{\isacharasterisk}}{\isaliteral{22}{\isachardoublequoteclose}}\ {\isaliteral{5B}{\isacharbrackleft}}{\isadigit{1}}{\isadigit{0}}{\isadigit{0}}{\isadigit{0}}{\isaliteral{5D}{\isacharbrackright}}\ {\isadigit{9}}{\isadigit{9}}{\isadigit{9}}{\isaliteral{29}{\isacharparenright}}\isanewline
\ \ \isakeyword{for}\ r\ {\isaliteral{3A}{\isacharcolon}}{\isaliteral{3A}{\isacharcolon}}\ {\isaliteral{22}{\isachardoublequoteopen}}{\isaliteral{28}{\isacharparenleft}}{\isaliteral{27}{\isacharprime}}a\ {\isaliteral{5C3C74696D65733E}{\isasymtimes}}\ {\isaliteral{27}{\isacharprime}}a{\isaliteral{29}{\isacharparenright}}set{\isaliteral{22}{\isachardoublequoteclose}}\isanewline
\isakeyword{where}\isanewline
\ \ rtc{\isaliteral{5F}{\isacharunderscore}}refl{\isaliteral{5B}{\isacharbrackleft}}iff{\isaliteral{5D}{\isacharbrackright}}{\isaliteral{3A}{\isacharcolon}}\ \ {\isaliteral{22}{\isachardoublequoteopen}}{\isaliteral{28}{\isacharparenleft}}x{\isaliteral{2C}{\isacharcomma}}x{\isaliteral{29}{\isacharparenright}}\ {\isaliteral{5C3C696E3E}{\isasymin}}\ r{\isaliteral{2A}{\isacharasterisk}}{\isaliteral{22}{\isachardoublequoteclose}}\isanewline
{\isaliteral{7C}{\isacharbar}}\ rtc{\isaliteral{5F}{\isacharunderscore}}step{\isaliteral{3A}{\isacharcolon}}\ \ \ \ \ \ \ {\isaliteral{22}{\isachardoublequoteopen}}{\isaliteral{5C3C6C6272616B6B3E}{\isasymlbrakk}}\ {\isaliteral{28}{\isacharparenleft}}x{\isaliteral{2C}{\isacharcomma}}y{\isaliteral{29}{\isacharparenright}}\ {\isaliteral{5C3C696E3E}{\isasymin}}\ r{\isaliteral{3B}{\isacharsemicolon}}\ {\isaliteral{28}{\isacharparenleft}}y{\isaliteral{2C}{\isacharcomma}}z{\isaliteral{29}{\isacharparenright}}\ {\isaliteral{5C3C696E3E}{\isasymin}}\ r{\isaliteral{2A}{\isacharasterisk}}\ {\isaliteral{5C3C726272616B6B3E}{\isasymrbrakk}}\ {\isaliteral{5C3C4C6F6E6772696768746172726F773E}{\isasymLongrightarrow}}\ {\isaliteral{28}{\isacharparenleft}}x{\isaliteral{2C}{\isacharcomma}}z{\isaliteral{29}{\isacharparenright}}\ {\isaliteral{5C3C696E3E}{\isasymin}}\ r{\isaliteral{2A}{\isacharasterisk}}{\isaliteral{22}{\isachardoublequoteclose}}%
\begin{isamarkuptext}%
\noindent
The function \isa{rtc} is annotated with concrete syntax: instead of
\isa{rtc\ r} we can write \isa{r{\isaliteral{2A}{\isacharasterisk}}}. The actual definition
consists of two rules. Reflexivity is obvious and is immediately given the
\isa{iff} attribute to increase automation. The
second rule, \isa{rtc{\isaliteral{5F}{\isacharunderscore}}step}, says that we can always add one more
\isa{r}-step to the left. Although we could make \isa{rtc{\isaliteral{5F}{\isacharunderscore}}step} an
introduction rule, this is dangerous: the recursion in the second premise
slows down and may even kill the automatic tactics.

The above definition of the concept of reflexive transitive closure may
be sufficiently intuitive but it is certainly not the only possible one:
for a start, it does not even mention transitivity.
The rest of this section is devoted to proving that it is equivalent to
the standard definition. We start with a simple lemma:%
\end{isamarkuptext}%
\isamarkuptrue%
\isacommand{lemma}\isamarkupfalse%
\ {\isaliteral{5B}{\isacharbrackleft}}intro{\isaliteral{5D}{\isacharbrackright}}{\isaliteral{3A}{\isacharcolon}}\ {\isaliteral{22}{\isachardoublequoteopen}}{\isaliteral{28}{\isacharparenleft}}x{\isaliteral{2C}{\isacharcomma}}y{\isaliteral{29}{\isacharparenright}}\ {\isaliteral{5C3C696E3E}{\isasymin}}\ r\ {\isaliteral{5C3C4C6F6E6772696768746172726F773E}{\isasymLongrightarrow}}\ {\isaliteral{28}{\isacharparenleft}}x{\isaliteral{2C}{\isacharcomma}}y{\isaliteral{29}{\isacharparenright}}\ {\isaliteral{5C3C696E3E}{\isasymin}}\ r{\isaliteral{2A}{\isacharasterisk}}{\isaliteral{22}{\isachardoublequoteclose}}\isanewline
%
\isadelimproof
%
\endisadelimproof
%
\isatagproof
\isacommand{by}\isamarkupfalse%
{\isaliteral{28}{\isacharparenleft}}blast\ intro{\isaliteral{3A}{\isacharcolon}}\ rtc{\isaliteral{5F}{\isacharunderscore}}step{\isaliteral{29}{\isacharparenright}}%
\endisatagproof
{\isafoldproof}%
%
\isadelimproof
%
\endisadelimproof
%
\begin{isamarkuptext}%
\noindent
Although the lemma itself is an unremarkable consequence of the basic rules,
it has the advantage that it can be declared an introduction rule without the
danger of killing the automatic tactics because \isa{r{\isaliteral{2A}{\isacharasterisk}}} occurs only in
the conclusion and not in the premise. Thus some proofs that would otherwise
need \isa{rtc{\isaliteral{5F}{\isacharunderscore}}step} can now be found automatically. The proof also
shows that \isa{blast} is able to handle \isa{rtc{\isaliteral{5F}{\isacharunderscore}}step}. But
some of the other automatic tactics are more sensitive, and even \isa{blast} can be lead astray in the presence of large numbers of rules.

To prove transitivity, we need rule induction, i.e.\ theorem
\isa{rtc{\isaliteral{2E}{\isachardot}}induct}:
\begin{isabelle}%
\ \ \ \ \ {\isaliteral{5C3C6C6272616B6B3E}{\isasymlbrakk}}{\isaliteral{28}{\isacharparenleft}}{\isaliteral{3F}{\isacharquery}}x{\isadigit{1}}{\isaliteral{2E}{\isachardot}}{\isadigit{0}}{\isaliteral{2C}{\isacharcomma}}\ {\isaliteral{3F}{\isacharquery}}x{\isadigit{2}}{\isaliteral{2E}{\isachardot}}{\isadigit{0}}{\isaliteral{29}{\isacharparenright}}\ {\isaliteral{5C3C696E3E}{\isasymin}}\ {\isaliteral{3F}{\isacharquery}}r{\isaliteral{2A}{\isacharasterisk}}{\isaliteral{3B}{\isacharsemicolon}}\ {\isaliteral{5C3C416E643E}{\isasymAnd}}x{\isaliteral{2E}{\isachardot}}\ {\isaliteral{3F}{\isacharquery}}P\ x\ x{\isaliteral{3B}{\isacharsemicolon}}\isanewline
\isaindent{\ \ \ \ \ \ }{\isaliteral{5C3C416E643E}{\isasymAnd}}x\ y\ z{\isaliteral{2E}{\isachardot}}\ {\isaliteral{5C3C6C6272616B6B3E}{\isasymlbrakk}}{\isaliteral{28}{\isacharparenleft}}x{\isaliteral{2C}{\isacharcomma}}\ y{\isaliteral{29}{\isacharparenright}}\ {\isaliteral{5C3C696E3E}{\isasymin}}\ {\isaliteral{3F}{\isacharquery}}r{\isaliteral{3B}{\isacharsemicolon}}\ {\isaliteral{28}{\isacharparenleft}}y{\isaliteral{2C}{\isacharcomma}}\ z{\isaliteral{29}{\isacharparenright}}\ {\isaliteral{5C3C696E3E}{\isasymin}}\ {\isaliteral{3F}{\isacharquery}}r{\isaliteral{2A}{\isacharasterisk}}{\isaliteral{3B}{\isacharsemicolon}}\ {\isaliteral{3F}{\isacharquery}}P\ y\ z{\isaliteral{5C3C726272616B6B3E}{\isasymrbrakk}}\ {\isaliteral{5C3C4C6F6E6772696768746172726F773E}{\isasymLongrightarrow}}\ {\isaliteral{3F}{\isacharquery}}P\ x\ z{\isaliteral{5C3C726272616B6B3E}{\isasymrbrakk}}\isanewline
\isaindent{\ \ \ \ \ }{\isaliteral{5C3C4C6F6E6772696768746172726F773E}{\isasymLongrightarrow}}\ {\isaliteral{3F}{\isacharquery}}P\ {\isaliteral{3F}{\isacharquery}}x{\isadigit{1}}{\isaliteral{2E}{\isachardot}}{\isadigit{0}}\ {\isaliteral{3F}{\isacharquery}}x{\isadigit{2}}{\isaliteral{2E}{\isachardot}}{\isadigit{0}}%
\end{isabelle}
It says that \isa{{\isaliteral{3F}{\isacharquery}}P} holds for an arbitrary pair \isa{{\isaliteral{28}{\isacharparenleft}}{\isaliteral{3F}{\isacharquery}}x{\isadigit{1}}{\isaliteral{2E}{\isachardot}}{\isadigit{0}}{\isaliteral{2C}{\isacharcomma}}\ {\isaliteral{3F}{\isacharquery}}x{\isadigit{2}}{\isaliteral{2E}{\isachardot}}{\isadigit{0}}{\isaliteral{29}{\isacharparenright}}\ {\isaliteral{5C3C696E3E}{\isasymin}}\ {\isaliteral{3F}{\isacharquery}}r{\isaliteral{2A}{\isacharasterisk}}}
if \isa{{\isaliteral{3F}{\isacharquery}}P} is preserved by all rules of the inductive definition,
i.e.\ if \isa{{\isaliteral{3F}{\isacharquery}}P} holds for the conclusion provided it holds for the
premises. In general, rule induction for an $n$-ary inductive relation $R$
expects a premise of the form $(x@1,\dots,x@n) \in R$.

Now we turn to the inductive proof of transitivity:%
\end{isamarkuptext}%
\isamarkuptrue%
\isacommand{lemma}\isamarkupfalse%
\ rtc{\isaliteral{5F}{\isacharunderscore}}trans{\isaliteral{3A}{\isacharcolon}}\ {\isaliteral{22}{\isachardoublequoteopen}}{\isaliteral{5C3C6C6272616B6B3E}{\isasymlbrakk}}\ {\isaliteral{28}{\isacharparenleft}}x{\isaliteral{2C}{\isacharcomma}}y{\isaliteral{29}{\isacharparenright}}\ {\isaliteral{5C3C696E3E}{\isasymin}}\ r{\isaliteral{2A}{\isacharasterisk}}{\isaliteral{3B}{\isacharsemicolon}}\ {\isaliteral{28}{\isacharparenleft}}y{\isaliteral{2C}{\isacharcomma}}z{\isaliteral{29}{\isacharparenright}}\ {\isaliteral{5C3C696E3E}{\isasymin}}\ r{\isaliteral{2A}{\isacharasterisk}}\ {\isaliteral{5C3C726272616B6B3E}{\isasymrbrakk}}\ {\isaliteral{5C3C4C6F6E6772696768746172726F773E}{\isasymLongrightarrow}}\ {\isaliteral{28}{\isacharparenleft}}x{\isaliteral{2C}{\isacharcomma}}z{\isaliteral{29}{\isacharparenright}}\ {\isaliteral{5C3C696E3E}{\isasymin}}\ r{\isaliteral{2A}{\isacharasterisk}}{\isaliteral{22}{\isachardoublequoteclose}}\isanewline
%
\isadelimproof
%
\endisadelimproof
%
\isatagproof
\isacommand{apply}\isamarkupfalse%
{\isaliteral{28}{\isacharparenleft}}erule\ rtc{\isaliteral{2E}{\isachardot}}induct{\isaliteral{29}{\isacharparenright}}%
\begin{isamarkuptxt}%
\noindent
Unfortunately, even the base case is a problem:
\begin{isabelle}%
\ {\isadigit{1}}{\isaliteral{2E}{\isachardot}}\ {\isaliteral{5C3C416E643E}{\isasymAnd}}x{\isaliteral{2E}{\isachardot}}\ {\isaliteral{28}{\isacharparenleft}}y{\isaliteral{2C}{\isacharcomma}}\ z{\isaliteral{29}{\isacharparenright}}\ {\isaliteral{5C3C696E3E}{\isasymin}}\ r{\isaliteral{2A}{\isacharasterisk}}\ {\isaliteral{5C3C4C6F6E6772696768746172726F773E}{\isasymLongrightarrow}}\ {\isaliteral{28}{\isacharparenleft}}x{\isaliteral{2C}{\isacharcomma}}\ z{\isaliteral{29}{\isacharparenright}}\ {\isaliteral{5C3C696E3E}{\isasymin}}\ r{\isaliteral{2A}{\isacharasterisk}}%
\end{isabelle}
We have to abandon this proof attempt.
To understand what is going on, let us look again at \isa{rtc{\isaliteral{2E}{\isachardot}}induct}.
In the above application of \isa{erule}, the first premise of
\isa{rtc{\isaliteral{2E}{\isachardot}}induct} is unified with the first suitable assumption, which
is \isa{{\isaliteral{28}{\isacharparenleft}}x{\isaliteral{2C}{\isacharcomma}}\ y{\isaliteral{29}{\isacharparenright}}\ {\isaliteral{5C3C696E3E}{\isasymin}}\ r{\isaliteral{2A}{\isacharasterisk}}} rather than \isa{{\isaliteral{28}{\isacharparenleft}}y{\isaliteral{2C}{\isacharcomma}}\ z{\isaliteral{29}{\isacharparenright}}\ {\isaliteral{5C3C696E3E}{\isasymin}}\ r{\isaliteral{2A}{\isacharasterisk}}}. Although that
is what we want, it is merely due to the order in which the assumptions occur
in the subgoal, which it is not good practice to rely on. As a result,
\isa{{\isaliteral{3F}{\isacharquery}}xb} becomes \isa{x}, \isa{{\isaliteral{3F}{\isacharquery}}xa} becomes
\isa{y} and \isa{{\isaliteral{3F}{\isacharquery}}P} becomes \isa{{\isaliteral{5C3C6C616D6264613E}{\isasymlambda}}u\ v{\isaliteral{2E}{\isachardot}}\ {\isaliteral{28}{\isacharparenleft}}u{\isaliteral{2C}{\isacharcomma}}\ z{\isaliteral{29}{\isacharparenright}}\ {\isaliteral{5C3C696E3E}{\isasymin}}\ r{\isaliteral{2A}{\isacharasterisk}}}, thus
yielding the above subgoal. So what went wrong?

When looking at the instantiation of \isa{{\isaliteral{3F}{\isacharquery}}P} we see that it does not
depend on its second parameter at all. The reason is that in our original
goal, of the pair \isa{{\isaliteral{28}{\isacharparenleft}}x{\isaliteral{2C}{\isacharcomma}}\ y{\isaliteral{29}{\isacharparenright}}} only \isa{x} appears also in the
conclusion, but not \isa{y}. Thus our induction statement is too
general. Fortunately, it can easily be specialized:
transfer the additional premise \isa{{\isaliteral{28}{\isacharparenleft}}y{\isaliteral{2C}{\isacharcomma}}\ z{\isaliteral{29}{\isacharparenright}}\ {\isaliteral{5C3C696E3E}{\isasymin}}\ r{\isaliteral{2A}{\isacharasterisk}}} into the conclusion:%
\end{isamarkuptxt}%
\isamarkuptrue%
%
\endisatagproof
{\isafoldproof}%
%
\isadelimproof
%
\endisadelimproof
\isacommand{lemma}\isamarkupfalse%
\ rtc{\isaliteral{5F}{\isacharunderscore}}trans{\isaliteral{5B}{\isacharbrackleft}}rule{\isaliteral{5F}{\isacharunderscore}}format{\isaliteral{5D}{\isacharbrackright}}{\isaliteral{3A}{\isacharcolon}}\isanewline
\ \ {\isaliteral{22}{\isachardoublequoteopen}}{\isaliteral{28}{\isacharparenleft}}x{\isaliteral{2C}{\isacharcomma}}y{\isaliteral{29}{\isacharparenright}}\ {\isaliteral{5C3C696E3E}{\isasymin}}\ r{\isaliteral{2A}{\isacharasterisk}}\ {\isaliteral{5C3C4C6F6E6772696768746172726F773E}{\isasymLongrightarrow}}\ {\isaliteral{28}{\isacharparenleft}}y{\isaliteral{2C}{\isacharcomma}}z{\isaliteral{29}{\isacharparenright}}\ {\isaliteral{5C3C696E3E}{\isasymin}}\ r{\isaliteral{2A}{\isacharasterisk}}\ {\isaliteral{5C3C6C6F6E6772696768746172726F773E}{\isasymlongrightarrow}}\ {\isaliteral{28}{\isacharparenleft}}x{\isaliteral{2C}{\isacharcomma}}z{\isaliteral{29}{\isacharparenright}}\ {\isaliteral{5C3C696E3E}{\isasymin}}\ r{\isaliteral{2A}{\isacharasterisk}}{\isaliteral{22}{\isachardoublequoteclose}}%
\isadelimproof
%
\endisadelimproof
%
\isatagproof
%
\begin{isamarkuptxt}%
\noindent
This is not an obscure trick but a generally applicable heuristic:
\begin{quote}\em
When proving a statement by rule induction on $(x@1,\dots,x@n) \in R$,
pull all other premises containing any of the $x@i$ into the conclusion
using $\longrightarrow$.
\end{quote}
A similar heuristic for other kinds of inductions is formulated in
\S\ref{sec:ind-var-in-prems}. The \isa{rule{\isaliteral{5F}{\isacharunderscore}}format} directive turns
\isa{{\isaliteral{5C3C6C6F6E6772696768746172726F773E}{\isasymlongrightarrow}}} back into \isa{{\isaliteral{5C3C4C6F6E6772696768746172726F773E}{\isasymLongrightarrow}}}: in the end we obtain the original
statement of our lemma.%
\end{isamarkuptxt}%
\isamarkuptrue%
\isacommand{apply}\isamarkupfalse%
{\isaliteral{28}{\isacharparenleft}}erule\ rtc{\isaliteral{2E}{\isachardot}}induct{\isaliteral{29}{\isacharparenright}}%
\begin{isamarkuptxt}%
\noindent
Now induction produces two subgoals which are both proved automatically:
\begin{isabelle}%
\ {\isadigit{1}}{\isaliteral{2E}{\isachardot}}\ {\isaliteral{5C3C416E643E}{\isasymAnd}}x{\isaliteral{2E}{\isachardot}}\ {\isaliteral{28}{\isacharparenleft}}x{\isaliteral{2C}{\isacharcomma}}\ z{\isaliteral{29}{\isacharparenright}}\ {\isaliteral{5C3C696E3E}{\isasymin}}\ r{\isaliteral{2A}{\isacharasterisk}}\ {\isaliteral{5C3C6C6F6E6772696768746172726F773E}{\isasymlongrightarrow}}\ {\isaliteral{28}{\isacharparenleft}}x{\isaliteral{2C}{\isacharcomma}}\ z{\isaliteral{29}{\isacharparenright}}\ {\isaliteral{5C3C696E3E}{\isasymin}}\ r{\isaliteral{2A}{\isacharasterisk}}\isanewline
\ {\isadigit{2}}{\isaliteral{2E}{\isachardot}}\ {\isaliteral{5C3C416E643E}{\isasymAnd}}x\ y\ za{\isaliteral{2E}{\isachardot}}\isanewline
\isaindent{\ {\isadigit{2}}{\isaliteral{2E}{\isachardot}}\ \ \ \ }{\isaliteral{5C3C6C6272616B6B3E}{\isasymlbrakk}}{\isaliteral{28}{\isacharparenleft}}x{\isaliteral{2C}{\isacharcomma}}\ y{\isaliteral{29}{\isacharparenright}}\ {\isaliteral{5C3C696E3E}{\isasymin}}\ r{\isaliteral{3B}{\isacharsemicolon}}\ {\isaliteral{28}{\isacharparenleft}}y{\isaliteral{2C}{\isacharcomma}}\ za{\isaliteral{29}{\isacharparenright}}\ {\isaliteral{5C3C696E3E}{\isasymin}}\ r{\isaliteral{2A}{\isacharasterisk}}{\isaliteral{3B}{\isacharsemicolon}}\ {\isaliteral{28}{\isacharparenleft}}za{\isaliteral{2C}{\isacharcomma}}\ z{\isaliteral{29}{\isacharparenright}}\ {\isaliteral{5C3C696E3E}{\isasymin}}\ r{\isaliteral{2A}{\isacharasterisk}}\ {\isaliteral{5C3C6C6F6E6772696768746172726F773E}{\isasymlongrightarrow}}\ {\isaliteral{28}{\isacharparenleft}}y{\isaliteral{2C}{\isacharcomma}}\ z{\isaliteral{29}{\isacharparenright}}\ {\isaliteral{5C3C696E3E}{\isasymin}}\ r{\isaliteral{2A}{\isacharasterisk}}{\isaliteral{5C3C726272616B6B3E}{\isasymrbrakk}}\isanewline
\isaindent{\ {\isadigit{2}}{\isaliteral{2E}{\isachardot}}\ \ \ \ }{\isaliteral{5C3C4C6F6E6772696768746172726F773E}{\isasymLongrightarrow}}\ {\isaliteral{28}{\isacharparenleft}}za{\isaliteral{2C}{\isacharcomma}}\ z{\isaliteral{29}{\isacharparenright}}\ {\isaliteral{5C3C696E3E}{\isasymin}}\ r{\isaliteral{2A}{\isacharasterisk}}\ {\isaliteral{5C3C6C6F6E6772696768746172726F773E}{\isasymlongrightarrow}}\ {\isaliteral{28}{\isacharparenleft}}x{\isaliteral{2C}{\isacharcomma}}\ z{\isaliteral{29}{\isacharparenright}}\ {\isaliteral{5C3C696E3E}{\isasymin}}\ r{\isaliteral{2A}{\isacharasterisk}}%
\end{isabelle}%
\end{isamarkuptxt}%
\isamarkuptrue%
\ \isacommand{apply}\isamarkupfalse%
{\isaliteral{28}{\isacharparenleft}}blast{\isaliteral{29}{\isacharparenright}}\isanewline
\isacommand{apply}\isamarkupfalse%
{\isaliteral{28}{\isacharparenleft}}blast\ intro{\isaliteral{3A}{\isacharcolon}}\ rtc{\isaliteral{5F}{\isacharunderscore}}step{\isaliteral{29}{\isacharparenright}}\isanewline
\isacommand{done}\isamarkupfalse%
%
\endisatagproof
{\isafoldproof}%
%
\isadelimproof
%
\endisadelimproof
%
\begin{isamarkuptext}%
Let us now prove that \isa{r{\isaliteral{2A}{\isacharasterisk}}} is really the reflexive transitive closure
of \isa{r}, i.e.\ the least reflexive and transitive
relation containing \isa{r}. The latter is easily formalized%
\end{isamarkuptext}%
\isamarkuptrue%
\isacommand{inductive{\isaliteral{5F}{\isacharunderscore}}set}\isamarkupfalse%
\isanewline
\ \ rtc{\isadigit{2}}\ {\isaliteral{3A}{\isacharcolon}}{\isaliteral{3A}{\isacharcolon}}\ {\isaliteral{22}{\isachardoublequoteopen}}{\isaliteral{28}{\isacharparenleft}}{\isaliteral{27}{\isacharprime}}a\ {\isaliteral{5C3C74696D65733E}{\isasymtimes}}\ {\isaliteral{27}{\isacharprime}}a{\isaliteral{29}{\isacharparenright}}set\ {\isaliteral{5C3C52696768746172726F773E}{\isasymRightarrow}}\ {\isaliteral{28}{\isacharparenleft}}{\isaliteral{27}{\isacharprime}}a\ {\isaliteral{5C3C74696D65733E}{\isasymtimes}}\ {\isaliteral{27}{\isacharprime}}a{\isaliteral{29}{\isacharparenright}}set{\isaliteral{22}{\isachardoublequoteclose}}\isanewline
\ \ \isakeyword{for}\ r\ {\isaliteral{3A}{\isacharcolon}}{\isaliteral{3A}{\isacharcolon}}\ {\isaliteral{22}{\isachardoublequoteopen}}{\isaliteral{28}{\isacharparenleft}}{\isaliteral{27}{\isacharprime}}a\ {\isaliteral{5C3C74696D65733E}{\isasymtimes}}\ {\isaliteral{27}{\isacharprime}}a{\isaliteral{29}{\isacharparenright}}set{\isaliteral{22}{\isachardoublequoteclose}}\isanewline
\isakeyword{where}\isanewline
\ \ {\isaliteral{22}{\isachardoublequoteopen}}{\isaliteral{28}{\isacharparenleft}}x{\isaliteral{2C}{\isacharcomma}}y{\isaliteral{29}{\isacharparenright}}\ {\isaliteral{5C3C696E3E}{\isasymin}}\ r\ {\isaliteral{5C3C4C6F6E6772696768746172726F773E}{\isasymLongrightarrow}}\ {\isaliteral{28}{\isacharparenleft}}x{\isaliteral{2C}{\isacharcomma}}y{\isaliteral{29}{\isacharparenright}}\ {\isaliteral{5C3C696E3E}{\isasymin}}\ rtc{\isadigit{2}}\ r{\isaliteral{22}{\isachardoublequoteclose}}\isanewline
{\isaliteral{7C}{\isacharbar}}\ {\isaliteral{22}{\isachardoublequoteopen}}{\isaliteral{28}{\isacharparenleft}}x{\isaliteral{2C}{\isacharcomma}}x{\isaliteral{29}{\isacharparenright}}\ {\isaliteral{5C3C696E3E}{\isasymin}}\ rtc{\isadigit{2}}\ r{\isaliteral{22}{\isachardoublequoteclose}}\isanewline
{\isaliteral{7C}{\isacharbar}}\ {\isaliteral{22}{\isachardoublequoteopen}}{\isaliteral{5C3C6C6272616B6B3E}{\isasymlbrakk}}\ {\isaliteral{28}{\isacharparenleft}}x{\isaliteral{2C}{\isacharcomma}}y{\isaliteral{29}{\isacharparenright}}\ {\isaliteral{5C3C696E3E}{\isasymin}}\ rtc{\isadigit{2}}\ r{\isaliteral{3B}{\isacharsemicolon}}\ {\isaliteral{28}{\isacharparenleft}}y{\isaliteral{2C}{\isacharcomma}}z{\isaliteral{29}{\isacharparenright}}\ {\isaliteral{5C3C696E3E}{\isasymin}}\ rtc{\isadigit{2}}\ r\ {\isaliteral{5C3C726272616B6B3E}{\isasymrbrakk}}\ {\isaliteral{5C3C4C6F6E6772696768746172726F773E}{\isasymLongrightarrow}}\ {\isaliteral{28}{\isacharparenleft}}x{\isaliteral{2C}{\isacharcomma}}z{\isaliteral{29}{\isacharparenright}}\ {\isaliteral{5C3C696E3E}{\isasymin}}\ rtc{\isadigit{2}}\ r{\isaliteral{22}{\isachardoublequoteclose}}%
\begin{isamarkuptext}%
\noindent
and the equivalence of the two definitions is easily shown by the obvious rule
inductions:%
\end{isamarkuptext}%
\isamarkuptrue%
\isacommand{lemma}\isamarkupfalse%
\ {\isaliteral{22}{\isachardoublequoteopen}}{\isaliteral{28}{\isacharparenleft}}x{\isaliteral{2C}{\isacharcomma}}y{\isaliteral{29}{\isacharparenright}}\ {\isaliteral{5C3C696E3E}{\isasymin}}\ rtc{\isadigit{2}}\ r\ {\isaliteral{5C3C4C6F6E6772696768746172726F773E}{\isasymLongrightarrow}}\ {\isaliteral{28}{\isacharparenleft}}x{\isaliteral{2C}{\isacharcomma}}y{\isaliteral{29}{\isacharparenright}}\ {\isaliteral{5C3C696E3E}{\isasymin}}\ r{\isaliteral{2A}{\isacharasterisk}}{\isaliteral{22}{\isachardoublequoteclose}}\isanewline
%
\isadelimproof
%
\endisadelimproof
%
\isatagproof
\isacommand{apply}\isamarkupfalse%
{\isaliteral{28}{\isacharparenleft}}erule\ rtc{\isadigit{2}}{\isaliteral{2E}{\isachardot}}induct{\isaliteral{29}{\isacharparenright}}\isanewline
\ \ \isacommand{apply}\isamarkupfalse%
{\isaliteral{28}{\isacharparenleft}}blast{\isaliteral{29}{\isacharparenright}}\isanewline
\ \isacommand{apply}\isamarkupfalse%
{\isaliteral{28}{\isacharparenleft}}blast{\isaliteral{29}{\isacharparenright}}\isanewline
\isacommand{apply}\isamarkupfalse%
{\isaliteral{28}{\isacharparenleft}}blast\ intro{\isaliteral{3A}{\isacharcolon}}\ rtc{\isaliteral{5F}{\isacharunderscore}}trans{\isaliteral{29}{\isacharparenright}}\isanewline
\isacommand{done}\isamarkupfalse%
%
\endisatagproof
{\isafoldproof}%
%
\isadelimproof
\isanewline
%
\endisadelimproof
\isanewline
\isacommand{lemma}\isamarkupfalse%
\ {\isaliteral{22}{\isachardoublequoteopen}}{\isaliteral{28}{\isacharparenleft}}x{\isaliteral{2C}{\isacharcomma}}y{\isaliteral{29}{\isacharparenright}}\ {\isaliteral{5C3C696E3E}{\isasymin}}\ r{\isaliteral{2A}{\isacharasterisk}}\ {\isaliteral{5C3C4C6F6E6772696768746172726F773E}{\isasymLongrightarrow}}\ {\isaliteral{28}{\isacharparenleft}}x{\isaliteral{2C}{\isacharcomma}}y{\isaliteral{29}{\isacharparenright}}\ {\isaliteral{5C3C696E3E}{\isasymin}}\ rtc{\isadigit{2}}\ r{\isaliteral{22}{\isachardoublequoteclose}}\isanewline
%
\isadelimproof
%
\endisadelimproof
%
\isatagproof
\isacommand{apply}\isamarkupfalse%
{\isaliteral{28}{\isacharparenleft}}erule\ rtc{\isaliteral{2E}{\isachardot}}induct{\isaliteral{29}{\isacharparenright}}\isanewline
\ \isacommand{apply}\isamarkupfalse%
{\isaliteral{28}{\isacharparenleft}}blast\ intro{\isaliteral{3A}{\isacharcolon}}\ rtc{\isadigit{2}}{\isaliteral{2E}{\isachardot}}intros{\isaliteral{29}{\isacharparenright}}\isanewline
\isacommand{apply}\isamarkupfalse%
{\isaliteral{28}{\isacharparenleft}}blast\ intro{\isaliteral{3A}{\isacharcolon}}\ rtc{\isadigit{2}}{\isaliteral{2E}{\isachardot}}intros{\isaliteral{29}{\isacharparenright}}\isanewline
\isacommand{done}\isamarkupfalse%
%
\endisatagproof
{\isafoldproof}%
%
\isadelimproof
%
\endisadelimproof
%
\begin{isamarkuptext}%
So why did we start with the first definition? Because it is simpler. It
contains only two rules, and the single step rule is simpler than
transitivity.  As a consequence, \isa{rtc{\isaliteral{2E}{\isachardot}}induct} is simpler than
\isa{rtc{\isadigit{2}}{\isaliteral{2E}{\isachardot}}induct}. Since inductive proofs are hard enough
anyway, we should always pick the simplest induction schema available.
Hence \isa{rtc} is the definition of choice.
\index{reflexive transitive closure!defining inductively|)}

\begin{exercise}\label{ex:converse-rtc-step}
Show that the converse of \isa{rtc{\isaliteral{5F}{\isacharunderscore}}step} also holds:
\begin{isabelle}%
\ \ \ \ \ {\isaliteral{5C3C6C6272616B6B3E}{\isasymlbrakk}}{\isaliteral{28}{\isacharparenleft}}x{\isaliteral{2C}{\isacharcomma}}\ y{\isaliteral{29}{\isacharparenright}}\ {\isaliteral{5C3C696E3E}{\isasymin}}\ r{\isaliteral{2A}{\isacharasterisk}}{\isaliteral{3B}{\isacharsemicolon}}\ {\isaliteral{28}{\isacharparenleft}}y{\isaliteral{2C}{\isacharcomma}}\ z{\isaliteral{29}{\isacharparenright}}\ {\isaliteral{5C3C696E3E}{\isasymin}}\ r{\isaliteral{5C3C726272616B6B3E}{\isasymrbrakk}}\ {\isaliteral{5C3C4C6F6E6772696768746172726F773E}{\isasymLongrightarrow}}\ {\isaliteral{28}{\isacharparenleft}}x{\isaliteral{2C}{\isacharcomma}}\ z{\isaliteral{29}{\isacharparenright}}\ {\isaliteral{5C3C696E3E}{\isasymin}}\ r{\isaliteral{2A}{\isacharasterisk}}%
\end{isabelle}
\end{exercise}
\begin{exercise}
Repeat the development of this section, but starting with a definition of
\isa{rtc} where \isa{rtc{\isaliteral{5F}{\isacharunderscore}}step} is replaced by its converse as shown
in exercise~\ref{ex:converse-rtc-step}.
\end{exercise}%
\end{isamarkuptext}%
\isamarkuptrue%
%
\isadelimproof
%
\endisadelimproof
%
\isatagproof
%
\endisatagproof
{\isafoldproof}%
%
\isadelimproof
%
\endisadelimproof
%
\isadelimtheory
%
\endisadelimtheory
%
\isatagtheory
%
\endisatagtheory
{\isafoldtheory}%
%
\isadelimtheory
%
\endisadelimtheory
\end{isabellebody}%
%%% Local Variables:
%%% mode: latex
%%% TeX-master: "root"
%%% End:


\section{Advanced Inductive Definitions}
\label{sec:adv-ind-def}
%
\begin{isabelle}
\def\isabellecontext{Advanced}
\isanewline
\isacommand{theory}\ Advanced\ =\ Even:\isanewline
\isanewline
\isanewline
\isacommand{datatype}\ 'f\ gterm\ =\ Apply\ 'f\ "'f\ gterm\ list"\isanewline
\isanewline
\isacommand{datatype}\ integer_op\ =\ Number\ int\ |\ UnaryMinus\ |\ Plus\isanewline
\isanewline
\isacommand{consts}\ gterms\ ::\ "'f\ set\ \isasymRightarrow \ 'f\ gterm\ set"\isanewline
\isacommand{inductive}\ "gterms\ F"\isanewline
\isakeyword{intros}\isanewline
step[intro!]:\ "\isasymlbrakk \isasymforall t\ \isasymin \ set\ args.\ t\ \isasymin \ gterms\ F;\ \ f\ \isasymin \ F\isasymrbrakk \isanewline
\ \ \ \ \ \ \ \ \ \ \ \ \ \ \ \isasymLongrightarrow \ (Apply\ f\ args)\ \isasymin \ gterms\ F"\isanewline
\isanewline
\isacommand{lemma}\ gterms_mono:\ "F\isasymsubseteq G\ \isasymLongrightarrow \ gterms\ F\ \isasymsubseteq \ gterms\ G"\isanewline
\isacommand{apply}\ clarify\isanewline
\isacommand{apply}\ (erule\ gterms.induct)
\begin{isamarkuptxt}
\begin{isabelle}
\ 1.\ \isasymAnd x\ args\ f.\isanewline
\isaindent{\ 1.\ \ \ \ }\isasymlbrakk F\ \isasymsubseteq \ G;\ \isasymforall t\isasymin set\ args.\ t\ \isasymin \ gterms\ F\ \isasymand \ t\ \isasymin \ gterms\ G;\ f\ \isasymin \ F\isasymrbrakk \isanewline
\isaindent{\ 1.\ \ \ \ }\isasymLongrightarrow \ Apply\ f\ args\ \isasymin \ gterms\ G%
\end{isabelle}
\end{isamarkuptxt}
\isacommand{apply}\ blast\isanewline
\isacommand{done}
\begin{isamarkuptext}
\begin{isabelle}
\ \ \ \ \ \isasymlbrakk a\ \isasymin \ even;\ a\ =\ 0\ \isasymLongrightarrow \ P;\ \isasymAnd n.\ \isasymlbrakk a\ =\ Suc\ (Suc\ n);\ n\ \isasymin \ even\isasymrbrakk \ \isasymLongrightarrow \ P\isasymrbrakk \ \isasymLongrightarrow \ P%
\rulename{even.cases}
\end{isabelle}

Just as a demo I include
the two forms that Markus has made available. First the one for binding the
result to a name%
\end{isamarkuptext}
\isacommand{inductive_cases}\ Suc_Suc_cases\ [elim!]:\isanewline
\ \ "Suc(Suc\ n)\ \isasymin \ even"\isanewline
\isanewline
\isacommand{thm}\ Suc_Suc_cases%
\begin{isamarkuptext}
\begin{isabelle}
\ \ \ \ \ \isasymlbrakk Suc\ (Suc\ n)\ \isasymin \ even;\ n\ \isasymin \ even\ \isasymLongrightarrow \ P\isasymrbrakk \ \isasymLongrightarrow \ P%
\rulename{Suc_Suc_cases}
\end{isabelle}

and now the one for local usage:%
\end{isamarkuptext}
\isacommand{lemma}\ "Suc(Suc\ n)\ \isasymin \ even\ \isasymLongrightarrow \ P\ n"\isanewline
\isacommand{apply}\ (ind_cases\ "Suc(Suc\ n)\ \isasymin \ even")\isanewline
\isacommand{oops}\isanewline
\isanewline
\isacommand{inductive_cases}\ gterm_Apply_elim\ [elim!]:\ "Apply\ f\ args\ \isasymin \ gterms\ F"
\begin{isamarkuptext}
this is what we get:

\begin{isabelle}
\ \ \ \ \ \isasymlbrakk Apply\ f\ args\ \isasymin \ gterms\ F;\ \isasymlbrakk \isasymforall t\isasymin set\ args.\ t\ \isasymin \ gterms\ F;\ f\ \isasymin \ F\isasymrbrakk \ \isasymLongrightarrow \ P\isasymrbrakk \ \isasymLongrightarrow \ P%
\rulename{gterm_Apply_elim}
\end{isabelle}
\end{isamarkuptext}
\isacommand{lemma}\ gterms_IntI\ [rule_format,\ intro!]:\isanewline
\ \ \ \ \ "t\ \isasymin \ gterms\ F\ \isasymLongrightarrow \ t\ \isasymin \ gterms\ G\ \isasymlongrightarrow \ t\ \isasymin \ gterms\ (F\isasyminter G)"\isanewline
\isacommand{apply}\ (erule\ gterms.induct)
\begin{isamarkuptxt}
\begin{isabelle}
\ 1.\ \isasymAnd args\ f.\isanewline
\isaindent{\ 1.\ \ \ \ }\isasymlbrakk \isasymforall t\isasymin set\ args.\isanewline
\isaindent{\ 1.\ \ \ \ \isasymlbrakk \ \ \ }t\ \isasymin \ gterms\ F\ \isasymand \ (t\ \isasymin \ gterms\ G\ \isasymlongrightarrow \ t\ \isasymin \ gterms\ (F\ \isasyminter \ G));\isanewline
\isaindent{\ 1.\ \ \ \ \ \ \ }f\ \isasymin \ F\isasymrbrakk \isanewline
\isaindent{\ 1.\ \ \ \ }\isasymLongrightarrow \ Apply\ f\ args\ \isasymin \ gterms\ G\ \isasymlongrightarrow \isanewline
\isaindent{\ 1.\ \ \ \ \isasymLongrightarrow \ }Apply\ f\ args\ \isasymin \ gterms\ (F\ \isasyminter \ G)
\end{isabelle}
\end{isamarkuptxt}
\isacommand{apply}\ blast\isanewline
\isacommand{done}
\begin{isamarkuptext}
\begin{isabelle}
\ \ \ \ \ mono\ f\ \isasymLongrightarrow \ f\ (A\ \isasyminter \ B)\ \isasymsubseteq \ f\ A\ \isasyminter \ f\ B%
\rulename{mono_Int}
\end{isabelle}
\end{isamarkuptext}
\isacommand{lemma}\ gterms_Int_eq\ [simp]:\isanewline
\ \ \ \ \ "gterms\ (F\isasyminter G)\ =\ gterms\ F\ \isasyminter \ gterms\ G"\isanewline
\isacommand{by}\ (blast\ intro!:\ mono_Int\ monoI\ gterms_mono)\isanewline
\isanewline
\isanewline
\isacommand{consts}\ integer_arity\ ::\ "integer_op\ \isasymRightarrow \ nat"\isanewline
\isacommand{primrec}\isanewline
"integer_arity\ (Number\ n)\ \ \ \ \ \ \ \ =\ \#0"\isanewline
"integer_arity\ UnaryMinus\ \ \ \ \ \ \ \ =\ \#1"\isanewline
"integer_arity\ Plus\ \ \ \ \ \ \ \ \ \ \ \ \ \ =\ \#2"\isanewline
\isanewline
\isacommand{consts}\ well_formed_gterm\ ::\ "('f\ \isasymRightarrow \ nat)\ \isasymRightarrow \ 'f\ gterm\ set"\isanewline
\isacommand{inductive}\ "well_formed_gterm\ arity"\isanewline
\isakeyword{intros}\isanewline
step[intro!]:\ "\isasymlbrakk \isasymforall t\ \isasymin \ set\ args.\ t\ \isasymin \ well_formed_gterm\ arity;\ \ \isanewline
\ \ \ \ \ \ \ \ \ \ \ \ \ \ \ \ length\ args\ =\ arity\ f\isasymrbrakk \isanewline
\ \ \ \ \ \ \ \ \ \ \ \ \ \ \ \isasymLongrightarrow \ (Apply\ f\ args)\ \isasymin \ well_formed_gterm\ arity"\isanewline
\isanewline
\isanewline
\isacommand{consts}\ well_formed_gterm'\ ::\ "('f\ \isasymRightarrow \ nat)\ \isasymRightarrow \ 'f\ gterm\ set"\isanewline
\isacommand{inductive}\ "well_formed_gterm'\ arity"\isanewline
\isakeyword{intros}\isanewline
step[intro!]:\ "\isasymlbrakk args\ \isasymin \ lists\ (well_formed_gterm'\ arity);\ \ \isanewline
\ \ \ \ \ \ \ \ \ \ \ \ \ \ \ \ length\ args\ =\ arity\ f\isasymrbrakk \isanewline
\ \ \ \ \ \ \ \ \ \ \ \ \ \ \ \isasymLongrightarrow \ (Apply\ f\ args)\ \isasymin \ well_formed_gterm'\ arity"\isanewline
\isakeyword{monos}\ lists_mono\isanewline
\isanewline
\isacommand{lemma}\ "well_formed_gterm\ arity\ \isasymsubseteq \ well_formed_gterm'\ arity"\isanewline
\isacommand{apply}\ clarify%
\begin{isamarkuptxt}
The situation after clarify
\begin{isabelle}
\ 1.\ \isasymAnd x.\ x\ \isasymin \ well_formed_gterm\ arity\ \isasymLongrightarrow \isanewline
\isaindent{\ 1.\ \isasymAnd x.\ }x\ \isasymin \ well_formed_gterm'\ arity%
\end{isabelle}
\end{isamarkuptxt}
\isacommand{apply}\ (erule\ well_formed_gterm.induct)
\begin{isamarkuptxt}
note the induction hypothesis!
\begin{isabelle}
\ 1.\ \isasymAnd x\ args\ f.\isanewline
\isaindent{\ 1.\ \ \ \ }\isasymlbrakk \isasymforall t\isasymin set\ args.\isanewline
\isaindent{\ 1.\ \ \ \ \isasymlbrakk \ \ \ }t\ \isasymin \ well_formed_gterm\ arity\ \isasymand \isanewline
\isaindent{\ 1.\ \ \ \ \isasymlbrakk \ \ \ }t\ \isasymin \ well_formed_gterm'\ arity;\isanewline
\isaindent{\ 1.\ \ \ \ \ \ \ }length\ args\ =\ arity\ f\isasymrbrakk \isanewline
\isaindent{\ 1.\ \ \ \ }\isasymLongrightarrow \ Apply\ f\ args\ \isasymin \ well_formed_gterm'\ arity%
\end{isabelle}
\end{isamarkuptxt}
\isacommand{apply}\ auto\isanewline
\isacommand{done}\isanewline
\isanewline
\isanewline
\isanewline
\isacommand{lemma}\ "well_formed_gterm'\ arity\ \isasymsubseteq \ well_formed_gterm\ arity"\isanewline
\isacommand{apply}\ clarify%
\begin{isamarkuptxt}
The situation after clarify
\begin{isabelle}
\ 1.\ \isasymAnd x.\ x\ \isasymin \ well_formed_gterm'\ arity\ \isasymLongrightarrow \isanewline
\isaindent{\ 1.\ \isasymAnd x.\ }x\ \isasymin \ well_formed_gterm\ arity%
\end{isabelle}
\end{isamarkuptxt}
\isacommand{apply}\ (erule\ well_formed_gterm'.induct)
\begin{isamarkuptxt}
note the induction hypothesis!
\begin{isabelle}
\ 1.\ \isasymAnd x\ args\ f.\isanewline
\isaindent{\ 1.\ \ \ \ }\isasymlbrakk args\isanewline
\isaindent{\ 1.\ \ \ \ \isasymlbrakk }\isasymin \ lists\isanewline
\isaindent{\ 1.\ \ \ \ \isasymlbrakk \isasymin \ \ }(well_formed_gterm'\ arity\ \isasyminter \isanewline
\isaindent{\ 1.\ \ \ \ \isasymlbrakk \isasymin \ \ (}\isacharbraceleft x.\ x\ \isasymin \ well_formed_gterm\ arity\isacharbraceright );\isanewline
\isaindent{\ 1.\ \ \ \ \ \ \ }length\ args\ =\ arity\ f\isasymrbrakk \isanewline
\isaindent{\ 1.\ \ \ \ }\isasymLongrightarrow \ Apply\ f\ args\ \isasymin \ well_formed_gterm\ arity%
\end{isabelle}
\end{isamarkuptxt}
\isacommand{apply}\ auto\isanewline
\isacommand{done}
\begin{isamarkuptext}
\begin{isabelle}
\ \ \ \ \ lists\ (A\ \isasyminter \ B)\ =\ lists\ A\ \isasyminter \ lists\ B%
\end{isabelle}
\end{isamarkuptext}
%
\begin{isamarkuptext}
the rest isn't used: too complicated.  OK for an exercise though.%
\end{isamarkuptext}
\isacommand{consts}\ integer_signature\ ::\ "(integer_op\ *\ (unit\ list\ *\ unit))\ set"\isanewline
\isacommand{inductive}\ "integer_signature"\isanewline
\isakeyword{intros}\isanewline
Number:\ \ \ \ \ "(Number\ n,\ \ \ ([],\ ()))\ \isasymin \ integer_signature"\isanewline
UnaryMinus:\ "(UnaryMinus,\ ([()],\ ()))\ \isasymin \ integer_signature"\isanewline
Plus:\ \ \ \ \ \ \ "(Plus,\ \ \ \ \ \ \ ([(),()],\ ()))\ \isasymin \ integer_signature"\isanewline
\isanewline
\isanewline
\isacommand{consts}\ well_typed_gterm\ ::\ "('f\ \isasymRightarrow \ 't\ list\ *\ 't)\ \isasymRightarrow \ ('f\ gterm\ *\ 't)set"\isanewline
\isacommand{inductive}\ "well_typed_gterm\ sig"\isanewline
\isakeyword{intros}\isanewline
step[intro!]:\ \isanewline
\ \ \ \ "\isasymlbrakk \isasymforall pair\ \isasymin \ set\ args.\ pair\ \isasymin \ well_typed_gterm\ sig;\ \isanewline
\ \ \ \ \ \ sig\ f\ =\ (map\ snd\ args,\ rtype)\isasymrbrakk \isanewline
\ \ \ \ \ \isasymLongrightarrow \ (Apply\ f\ (map\ fst\ args),\ rtype)\ \isanewline
\ \ \ \ \ \ \ \ \ \isasymin \ well_typed_gterm\ sig"\isanewline
\isanewline
\isacommand{consts}\ well_typed_gterm'\ ::\ "('f\ \isasymRightarrow \ 't\ list\ *\ 't)\ \isasymRightarrow \ ('f\ gterm\ *\ 't)set"\isanewline
\isacommand{inductive}\ "well_typed_gterm'\ sig"\isanewline
\isakeyword{intros}\isanewline
step[intro!]:\ \isanewline
\ \ \ \ "\isasymlbrakk args\ \isasymin \ lists(well_typed_gterm'\ sig);\ \isanewline
\ \ \ \ \ \ sig\ f\ =\ (map\ snd\ args,\ rtype)\isasymrbrakk \isanewline
\ \ \ \ \ \isasymLongrightarrow \ (Apply\ f\ (map\ fst\ args),\ rtype)\ \isanewline
\ \ \ \ \ \ \ \ \ \isasymin \ well_typed_gterm'\ sig"\isanewline
\isakeyword{monos}\ lists_mono\isanewline
\isanewline
\isanewline
\isacommand{lemma}\ "well_typed_gterm\ sig\ \isasymsubseteq \ well_typed_gterm'\ sig"\isanewline
\isacommand{apply}\ clarify\isanewline
\isacommand{apply}\ (erule\ well_typed_gterm.induct)\isanewline
\isacommand{apply}\ auto\isanewline
\isacommand{done}\isanewline
\isanewline
\isacommand{lemma}\ "well_typed_gterm'\ sig\ \isasymsubseteq \ well_typed_gterm\ sig"\isanewline
\isacommand{apply}\ clarify\isanewline
\isacommand{apply}\ (erule\ well_typed_gterm'.induct)\isanewline
\isacommand{apply}\ auto\isanewline
\isacommand{done}\isanewline
\isanewline
\isanewline
\isacommand{end}\isanewline
\isanewline
\end{isabelle}
%%% Local Variables:
%%% mode: latex
%%% TeX-master: "root"
%%% End:


%
\begin{isabellebody}%
\def\isabellecontext{AB}%
\isamarkupfalse%
%
\isadelimtheory
%
\endisadelimtheory
%
\isatagtheory
%
\endisatagtheory
{\isafoldtheory}%
%
\isadelimtheory
%
\endisadelimtheory
%
\isamarkupsection{Case Study: A Context Free Grammar%
}
\isamarkuptrue%
%
\begin{isamarkuptext}%
\label{sec:CFG}
\index{grammars!defining inductively|(}%
Grammars are nothing but shorthands for inductive definitions of nonterminals
which represent sets of strings. For example, the production
$A \to B c$ is short for
\[ w \in B \Longrightarrow wc \in A \]
This section demonstrates this idea with an example
due to Hopcroft and Ullman, a grammar for generating all words with an
equal number of $a$'s and~$b$'s:
\begin{eqnarray}
S &\to& \epsilon \mid b A \mid a B \nonumber\\
A &\to& a S \mid b A A \nonumber\\
B &\to& b S \mid a B B \nonumber
\end{eqnarray}
At the end we say a few words about the relationship between
the original proof \cite[p.\ts81]{HopcroftUllman} and our formal version.

We start by fixing the alphabet, which consists only of \isa{a}'s
and~\isa{b}'s:%
\end{isamarkuptext}%
\isamarkuptrue%
\isacommand{datatype}\isamarkupfalse%
\ alfa\ {\isacharequal}\ a\ {\isacharbar}\ b%
\begin{isamarkuptext}%
\noindent
For convenience we include the following easy lemmas as simplification rules:%
\end{isamarkuptext}%
\isamarkuptrue%
\isacommand{lemma}\isamarkupfalse%
\ {\isacharbrackleft}simp{\isacharbrackright}{\isacharcolon}\ {\isachardoublequoteopen}{\isacharparenleft}x\ {\isasymnoteq}\ a{\isacharparenright}\ {\isacharequal}\ {\isacharparenleft}x\ {\isacharequal}\ b{\isacharparenright}\ {\isasymand}\ {\isacharparenleft}x\ {\isasymnoteq}\ b{\isacharparenright}\ {\isacharequal}\ {\isacharparenleft}x\ {\isacharequal}\ a{\isacharparenright}{\isachardoublequoteclose}\isanewline
%
\isadelimproof
%
\endisadelimproof
%
\isatagproof
\isacommand{by}\isamarkupfalse%
\ {\isacharparenleft}case{\isacharunderscore}tac\ x{\isacharcomma}\ auto{\isacharparenright}%
\endisatagproof
{\isafoldproof}%
%
\isadelimproof
%
\endisadelimproof
%
\begin{isamarkuptext}%
\noindent
Words over this alphabet are of type \isa{alfa\ list}, and
the three nonterminals are declared as sets of such words:%
\end{isamarkuptext}%
\isamarkuptrue%
\isacommand{consts}\isamarkupfalse%
\ S\ {\isacharcolon}{\isacharcolon}\ {\isachardoublequoteopen}alfa\ list\ set{\isachardoublequoteclose}\isanewline
\ \ \ \ \ \ \ A\ {\isacharcolon}{\isacharcolon}\ {\isachardoublequoteopen}alfa\ list\ set{\isachardoublequoteclose}\isanewline
\ \ \ \ \ \ \ B\ {\isacharcolon}{\isacharcolon}\ {\isachardoublequoteopen}alfa\ list\ set{\isachardoublequoteclose}%
\begin{isamarkuptext}%
\noindent
The productions above are recast as a \emph{mutual} inductive
definition\index{inductive definition!simultaneous}
of \isa{S}, \isa{A} and~\isa{B}:%
\end{isamarkuptext}%
\isamarkuptrue%
\isacommand{inductive}\isamarkupfalse%
\ S\ A\ B\isanewline
\isakeyword{intros}\isanewline
\ \ {\isachardoublequoteopen}{\isacharbrackleft}{\isacharbrackright}\ {\isasymin}\ S{\isachardoublequoteclose}\isanewline
\ \ {\isachardoublequoteopen}w\ {\isasymin}\ A\ {\isasymLongrightarrow}\ b{\isacharhash}w\ {\isasymin}\ S{\isachardoublequoteclose}\isanewline
\ \ {\isachardoublequoteopen}w\ {\isasymin}\ B\ {\isasymLongrightarrow}\ a{\isacharhash}w\ {\isasymin}\ S{\isachardoublequoteclose}\isanewline
\isanewline
\ \ {\isachardoublequoteopen}w\ {\isasymin}\ S\ \ \ \ \ \ \ \ {\isasymLongrightarrow}\ a{\isacharhash}w\ \ \ {\isasymin}\ A{\isachardoublequoteclose}\isanewline
\ \ {\isachardoublequoteopen}{\isasymlbrakk}\ v{\isasymin}A{\isacharsemicolon}\ w{\isasymin}A\ {\isasymrbrakk}\ {\isasymLongrightarrow}\ b{\isacharhash}v{\isacharat}w\ {\isasymin}\ A{\isachardoublequoteclose}\isanewline
\isanewline
\ \ {\isachardoublequoteopen}w\ {\isasymin}\ S\ \ \ \ \ \ \ \ \ \ \ \ {\isasymLongrightarrow}\ b{\isacharhash}w\ \ \ {\isasymin}\ B{\isachardoublequoteclose}\isanewline
\ \ {\isachardoublequoteopen}{\isasymlbrakk}\ v\ {\isasymin}\ B{\isacharsemicolon}\ w\ {\isasymin}\ B\ {\isasymrbrakk}\ {\isasymLongrightarrow}\ a{\isacharhash}v{\isacharat}w\ {\isasymin}\ B{\isachardoublequoteclose}%
\begin{isamarkuptext}%
\noindent
First we show that all words in \isa{S} contain the same number of \isa{a}'s and \isa{b}'s. Since the definition of \isa{S} is by mutual
induction, so is the proof: we show at the same time that all words in
\isa{A} contain one more \isa{a} than \isa{b} and all words in \isa{B} contains one more \isa{b} than \isa{a}.%
\end{isamarkuptext}%
\isamarkuptrue%
\isacommand{lemma}\isamarkupfalse%
\ correctness{\isacharcolon}\isanewline
\ \ {\isachardoublequoteopen}{\isacharparenleft}w\ {\isasymin}\ S\ {\isasymlongrightarrow}\ size{\isacharbrackleft}x{\isasymin}w{\isachardot}\ x{\isacharequal}a{\isacharbrackright}\ {\isacharequal}\ size{\isacharbrackleft}x{\isasymin}w{\isachardot}\ x{\isacharequal}b{\isacharbrackright}{\isacharparenright}\ \ \ \ \ {\isasymand}\isanewline
\ \ \ {\isacharparenleft}w\ {\isasymin}\ A\ {\isasymlongrightarrow}\ size{\isacharbrackleft}x{\isasymin}w{\isachardot}\ x{\isacharequal}a{\isacharbrackright}\ {\isacharequal}\ size{\isacharbrackleft}x{\isasymin}w{\isachardot}\ x{\isacharequal}b{\isacharbrackright}\ {\isacharplus}\ {\isadigit{1}}{\isacharparenright}\ {\isasymand}\isanewline
\ \ \ {\isacharparenleft}w\ {\isasymin}\ B\ {\isasymlongrightarrow}\ size{\isacharbrackleft}x{\isasymin}w{\isachardot}\ x{\isacharequal}b{\isacharbrackright}\ {\isacharequal}\ size{\isacharbrackleft}x{\isasymin}w{\isachardot}\ x{\isacharequal}a{\isacharbrackright}\ {\isacharplus}\ {\isadigit{1}}{\isacharparenright}{\isachardoublequoteclose}%
\isadelimproof
%
\endisadelimproof
%
\isatagproof
%
\begin{isamarkuptxt}%
\noindent
These propositions are expressed with the help of the predefined \isa{filter} function on lists, which has the convenient syntax \isa{{\isacharbrackleft}x{\isasymin}xs{\isachardot}\ P\ x{\isacharbrackright}}, the list of all elements \isa{x} in \isa{xs} such that \isa{P\ x}
holds. Remember that on lists \isa{size} and \isa{length} are synonymous.

The proof itself is by rule induction and afterwards automatic:%
\end{isamarkuptxt}%
\isamarkuptrue%
\isacommand{by}\isamarkupfalse%
\ {\isacharparenleft}rule\ S{\isacharunderscore}A{\isacharunderscore}B{\isachardot}induct{\isacharcomma}\ auto{\isacharparenright}%
\endisatagproof
{\isafoldproof}%
%
\isadelimproof
%
\endisadelimproof
%
\begin{isamarkuptext}%
\noindent
This may seem surprising at first, and is indeed an indication of the power
of inductive definitions. But it is also quite straightforward. For example,
consider the production $A \to b A A$: if $v,w \in A$ and the elements of $A$
contain one more $a$ than~$b$'s, then $bvw$ must again contain one more $a$
than~$b$'s.

As usual, the correctness of syntactic descriptions is easy, but completeness
is hard: does \isa{S} contain \emph{all} words with an equal number of
\isa{a}'s and \isa{b}'s? It turns out that this proof requires the
following lemma: every string with two more \isa{a}'s than \isa{b}'s can be cut somewhere such that each half has one more \isa{a} than
\isa{b}. This is best seen by imagining counting the difference between the
number of \isa{a}'s and \isa{b}'s starting at the left end of the
word. We start with 0 and end (at the right end) with 2. Since each move to the
right increases or decreases the difference by 1, we must have passed through
1 on our way from 0 to 2. Formally, we appeal to the following discrete
intermediate value theorem \isa{nat{\isadigit{0}}{\isacharunderscore}intermed{\isacharunderscore}int{\isacharunderscore}val}
\begin{isabelle}%
\ \ \ \ \ {\isasymlbrakk}{\isasymforall}i{\isacharless}n{\isachardot}\ {\isasymbar}f\ {\isacharparenleft}i\ {\isacharplus}\ {\isadigit{1}}{\isacharparenright}\ {\isacharminus}\ f\ i{\isasymbar}\ {\isasymle}\ {\isadigit{1}}{\isacharsemicolon}\ f\ {\isadigit{0}}\ {\isasymle}\ k{\isacharsemicolon}\ k\ {\isasymle}\ f\ n{\isasymrbrakk}\isanewline
\isaindent{\ \ \ \ \ }{\isasymLongrightarrow}\ {\isasymexists}i{\isasymle}n{\isachardot}\ f\ i\ {\isacharequal}\ k%
\end{isabelle}
where \isa{f} is of type \isa{nat\ {\isasymRightarrow}\ int}, \isa{int} are the integers,
\isa{{\isasymbar}{\isachardot}{\isasymbar}} is the absolute value function\footnote{See
Table~\ref{tab:ascii} in the Appendix for the correct \textsc{ascii}
syntax.}, and \isa{{\isadigit{1}}} is the integer 1 (see \S\ref{sec:numbers}).

First we show that our specific function, the difference between the
numbers of \isa{a}'s and \isa{b}'s, does indeed only change by 1 in every
move to the right. At this point we also start generalizing from \isa{a}'s
and \isa{b}'s to an arbitrary property \isa{P}. Otherwise we would have
to prove the desired lemma twice, once as stated above and once with the
roles of \isa{a}'s and \isa{b}'s interchanged.%
\end{isamarkuptext}%
\isamarkuptrue%
\isacommand{lemma}\isamarkupfalse%
\ step{\isadigit{1}}{\isacharcolon}\ {\isachardoublequoteopen}{\isasymforall}i\ {\isacharless}\ size\ w{\isachardot}\isanewline
\ \ {\isasymbar}{\isacharparenleft}int{\isacharparenleft}size{\isacharbrackleft}x{\isasymin}take\ {\isacharparenleft}i{\isacharplus}{\isadigit{1}}{\isacharparenright}\ w{\isachardot}\ P\ x{\isacharbrackright}{\isacharparenright}{\isacharminus}int{\isacharparenleft}size{\isacharbrackleft}x{\isasymin}take\ {\isacharparenleft}i{\isacharplus}{\isadigit{1}}{\isacharparenright}\ w{\isachardot}\ {\isasymnot}P\ x{\isacharbrackright}{\isacharparenright}{\isacharparenright}\isanewline
\ \ \ {\isacharminus}\ {\isacharparenleft}int{\isacharparenleft}size{\isacharbrackleft}x{\isasymin}take\ i\ w{\isachardot}\ P\ x{\isacharbrackright}{\isacharparenright}{\isacharminus}int{\isacharparenleft}size{\isacharbrackleft}x{\isasymin}take\ i\ w{\isachardot}\ {\isasymnot}P\ x{\isacharbrackright}{\isacharparenright}{\isacharparenright}{\isasymbar}\ {\isasymle}\ {\isadigit{1}}{\isachardoublequoteclose}%
\isadelimproof
%
\endisadelimproof
%
\isatagproof
%
\begin{isamarkuptxt}%
\noindent
The lemma is a bit hard to read because of the coercion function
\isa{int\ {\isacharcolon}{\isacharcolon}\ nat\ {\isasymRightarrow}\ int}. It is required because \isa{size} returns
a natural number, but subtraction on type~\isa{nat} will do the wrong thing.
Function \isa{take} is predefined and \isa{take\ i\ xs} is the prefix of
length \isa{i} of \isa{xs}; below we also need \isa{drop\ i\ xs}, which
is what remains after that prefix has been dropped from \isa{xs}.

The proof is by induction on \isa{w}, with a trivial base case, and a not
so trivial induction step. Since it is essentially just arithmetic, we do not
discuss it.%
\end{isamarkuptxt}%
\isamarkuptrue%
\isacommand{apply}\isamarkupfalse%
{\isacharparenleft}induct{\isacharunderscore}tac\ w{\isacharparenright}\isanewline
\isacommand{apply}\isamarkupfalse%
{\isacharparenleft}auto\ simp\ add{\isacharcolon}\ abs{\isacharunderscore}if\ take{\isacharunderscore}Cons\ split{\isacharcolon}\ nat{\isachardot}split{\isacharparenright}\isanewline
\isacommand{done}\isamarkupfalse%
%
\endisatagproof
{\isafoldproof}%
%
\isadelimproof
%
\endisadelimproof
%
\begin{isamarkuptext}%
Finally we come to the above-mentioned lemma about cutting in half a word with two more elements of one sort than of the other sort:%
\end{isamarkuptext}%
\isamarkuptrue%
\isacommand{lemma}\isamarkupfalse%
\ part{\isadigit{1}}{\isacharcolon}\isanewline
\ {\isachardoublequoteopen}size{\isacharbrackleft}x{\isasymin}w{\isachardot}\ P\ x{\isacharbrackright}\ {\isacharequal}\ size{\isacharbrackleft}x{\isasymin}w{\isachardot}\ {\isasymnot}P\ x{\isacharbrackright}{\isacharplus}{\isadigit{2}}\ {\isasymLongrightarrow}\isanewline
\ \ {\isasymexists}i{\isasymle}size\ w{\isachardot}\ size{\isacharbrackleft}x{\isasymin}take\ i\ w{\isachardot}\ P\ x{\isacharbrackright}\ {\isacharequal}\ size{\isacharbrackleft}x{\isasymin}take\ i\ w{\isachardot}\ {\isasymnot}P\ x{\isacharbrackright}{\isacharplus}{\isadigit{1}}{\isachardoublequoteclose}%
\isadelimproof
%
\endisadelimproof
%
\isatagproof
%
\begin{isamarkuptxt}%
\noindent
This is proved by \isa{force} with the help of the intermediate value theorem,
instantiated appropriately and with its first premise disposed of by lemma
\isa{step{\isadigit{1}}}:%
\end{isamarkuptxt}%
\isamarkuptrue%
\isacommand{apply}\isamarkupfalse%
{\isacharparenleft}insert\ nat{\isadigit{0}}{\isacharunderscore}intermed{\isacharunderscore}int{\isacharunderscore}val{\isacharbrackleft}OF\ step{\isadigit{1}}{\isacharcomma}\ of\ {\isachardoublequoteopen}P{\isachardoublequoteclose}\ {\isachardoublequoteopen}w{\isachardoublequoteclose}\ {\isachardoublequoteopen}{\isadigit{1}}{\isachardoublequoteclose}{\isacharbrackright}{\isacharparenright}\isanewline
\isacommand{by}\isamarkupfalse%
\ force%
\endisatagproof
{\isafoldproof}%
%
\isadelimproof
%
\endisadelimproof
%
\begin{isamarkuptext}%
\noindent

Lemma \isa{part{\isadigit{1}}} tells us only about the prefix \isa{take\ i\ w}.
An easy lemma deals with the suffix \isa{drop\ i\ w}:%
\end{isamarkuptext}%
\isamarkuptrue%
\isacommand{lemma}\isamarkupfalse%
\ part{\isadigit{2}}{\isacharcolon}\isanewline
\ \ {\isachardoublequoteopen}{\isasymlbrakk}size{\isacharbrackleft}x{\isasymin}take\ i\ w\ {\isacharat}\ drop\ i\ w{\isachardot}\ P\ x{\isacharbrackright}\ {\isacharequal}\isanewline
\ \ \ \ size{\isacharbrackleft}x{\isasymin}take\ i\ w\ {\isacharat}\ drop\ i\ w{\isachardot}\ {\isasymnot}P\ x{\isacharbrackright}{\isacharplus}{\isadigit{2}}{\isacharsemicolon}\isanewline
\ \ \ \ size{\isacharbrackleft}x{\isasymin}take\ i\ w{\isachardot}\ P\ x{\isacharbrackright}\ {\isacharequal}\ size{\isacharbrackleft}x{\isasymin}take\ i\ w{\isachardot}\ {\isasymnot}P\ x{\isacharbrackright}{\isacharplus}{\isadigit{1}}{\isasymrbrakk}\isanewline
\ \ \ {\isasymLongrightarrow}\ size{\isacharbrackleft}x{\isasymin}drop\ i\ w{\isachardot}\ P\ x{\isacharbrackright}\ {\isacharequal}\ size{\isacharbrackleft}x{\isasymin}drop\ i\ w{\isachardot}\ {\isasymnot}P\ x{\isacharbrackright}{\isacharplus}{\isadigit{1}}{\isachardoublequoteclose}\isanewline
%
\isadelimproof
%
\endisadelimproof
%
\isatagproof
\isacommand{by}\isamarkupfalse%
{\isacharparenleft}simp\ del{\isacharcolon}\ append{\isacharunderscore}take{\isacharunderscore}drop{\isacharunderscore}id{\isacharparenright}%
\endisatagproof
{\isafoldproof}%
%
\isadelimproof
%
\endisadelimproof
%
\begin{isamarkuptext}%
\noindent
In the proof we have disabled the normally useful lemma
\begin{isabelle}
\isa{take\ n\ xs\ {\isacharat}\ drop\ n\ xs\ {\isacharequal}\ xs}
\rulename{append_take_drop_id}
\end{isabelle}
to allow the simplifier to apply the following lemma instead:
\begin{isabelle}%
\ \ \ \ \ {\isacharbrackleft}x{\isasymin}xs{\isacharat}ys{\isachardot}\ P\ x{\isacharbrackright}\ {\isacharequal}\ {\isacharbrackleft}x{\isasymin}xs{\isachardot}\ P\ x{\isacharbrackright}\ {\isacharat}\ {\isacharbrackleft}x{\isasymin}ys{\isachardot}\ P\ x{\isacharbrackright}%
\end{isabelle}

To dispose of trivial cases automatically, the rules of the inductive
definition are declared simplification rules:%
\end{isamarkuptext}%
\isamarkuptrue%
\isacommand{declare}\isamarkupfalse%
\ S{\isacharunderscore}A{\isacharunderscore}B{\isachardot}intros{\isacharbrackleft}simp{\isacharbrackright}%
\begin{isamarkuptext}%
\noindent
This could have been done earlier but was not necessary so far.

The completeness theorem tells us that if a word has the same number of
\isa{a}'s and \isa{b}'s, then it is in \isa{S}, and similarly 
for \isa{A} and \isa{B}:%
\end{isamarkuptext}%
\isamarkuptrue%
\isacommand{theorem}\isamarkupfalse%
\ completeness{\isacharcolon}\isanewline
\ \ {\isachardoublequoteopen}{\isacharparenleft}size{\isacharbrackleft}x{\isasymin}w{\isachardot}\ x{\isacharequal}a{\isacharbrackright}\ {\isacharequal}\ size{\isacharbrackleft}x{\isasymin}w{\isachardot}\ x{\isacharequal}b{\isacharbrackright}\ \ \ \ \ {\isasymlongrightarrow}\ w\ {\isasymin}\ S{\isacharparenright}\ {\isasymand}\isanewline
\ \ \ {\isacharparenleft}size{\isacharbrackleft}x{\isasymin}w{\isachardot}\ x{\isacharequal}a{\isacharbrackright}\ {\isacharequal}\ size{\isacharbrackleft}x{\isasymin}w{\isachardot}\ x{\isacharequal}b{\isacharbrackright}\ {\isacharplus}\ {\isadigit{1}}\ {\isasymlongrightarrow}\ w\ {\isasymin}\ A{\isacharparenright}\ {\isasymand}\isanewline
\ \ \ {\isacharparenleft}size{\isacharbrackleft}x{\isasymin}w{\isachardot}\ x{\isacharequal}b{\isacharbrackright}\ {\isacharequal}\ size{\isacharbrackleft}x{\isasymin}w{\isachardot}\ x{\isacharequal}a{\isacharbrackright}\ {\isacharplus}\ {\isadigit{1}}\ {\isasymlongrightarrow}\ w\ {\isasymin}\ B{\isacharparenright}{\isachardoublequoteclose}%
\isadelimproof
%
\endisadelimproof
%
\isatagproof
%
\begin{isamarkuptxt}%
\noindent
The proof is by induction on \isa{w}. Structural induction would fail here
because, as we can see from the grammar, we need to make bigger steps than
merely appending a single letter at the front. Hence we induct on the length
of \isa{w}, using the induction rule \isa{length{\isacharunderscore}induct}:%
\end{isamarkuptxt}%
\isamarkuptrue%
\isacommand{apply}\isamarkupfalse%
{\isacharparenleft}induct{\isacharunderscore}tac\ w\ rule{\isacharcolon}\ length{\isacharunderscore}induct{\isacharparenright}%
\begin{isamarkuptxt}%
\noindent
The \isa{rule} parameter tells \isa{induct{\isacharunderscore}tac} explicitly which induction
rule to use. For details see \S\ref{sec:complete-ind} below.
In this case the result is that we may assume the lemma already
holds for all words shorter than \isa{w}.

The proof continues with a case distinction on \isa{w},
on whether \isa{w} is empty or not.%
\end{isamarkuptxt}%
\isamarkuptrue%
\isacommand{apply}\isamarkupfalse%
{\isacharparenleft}case{\isacharunderscore}tac\ w{\isacharparenright}\isanewline
\ \isacommand{apply}\isamarkupfalse%
{\isacharparenleft}simp{\isacharunderscore}all{\isacharparenright}%
\begin{isamarkuptxt}%
\noindent
Simplification disposes of the base case and leaves only a conjunction
of two step cases to be proved:
if \isa{w\ {\isacharequal}\ a\ {\isacharhash}\ v} and \begin{isabelle}%
\ \ \ \ \ length\ {\isacharbrackleft}x{\isasymin}v\ {\isachardot}\ x\ {\isacharequal}\ a{\isacharbrackright}\ {\isacharequal}\ length\ {\isacharbrackleft}x{\isasymin}v\ {\isachardot}\ x\ {\isacharequal}\ b{\isacharbrackright}\ {\isacharplus}\ {\isadigit{2}}%
\end{isabelle} then
\isa{b\ {\isacharhash}\ v\ {\isasymin}\ A}, and similarly for \isa{w\ {\isacharequal}\ b\ {\isacharhash}\ v}.
We only consider the first case in detail.

After breaking the conjunction up into two cases, we can apply
\isa{part{\isadigit{1}}} to the assumption that \isa{w} contains two more \isa{a}'s than \isa{b}'s.%
\end{isamarkuptxt}%
\isamarkuptrue%
\isacommand{apply}\isamarkupfalse%
{\isacharparenleft}rule\ conjI{\isacharparenright}\isanewline
\ \isacommand{apply}\isamarkupfalse%
{\isacharparenleft}clarify{\isacharparenright}\isanewline
\ \isacommand{apply}\isamarkupfalse%
{\isacharparenleft}frule\ part{\isadigit{1}}{\isacharbrackleft}of\ {\isachardoublequoteopen}{\isasymlambda}x{\isachardot}\ x{\isacharequal}a{\isachardoublequoteclose}{\isacharcomma}\ simplified{\isacharbrackright}{\isacharparenright}\isanewline
\ \isacommand{apply}\isamarkupfalse%
{\isacharparenleft}clarify{\isacharparenright}%
\begin{isamarkuptxt}%
\noindent
This yields an index \isa{i\ {\isasymle}\ length\ v} such that
\begin{isabelle}%
\ \ \ \ \ length\ {\isacharbrackleft}x{\isasymin}take\ i\ v\ {\isachardot}\ x\ {\isacharequal}\ a{\isacharbrackright}\ {\isacharequal}\ length\ {\isacharbrackleft}x{\isasymin}take\ i\ v\ {\isachardot}\ x\ {\isacharequal}\ b{\isacharbrackright}\ {\isacharplus}\ {\isadigit{1}}%
\end{isabelle}
With the help of \isa{part{\isadigit{2}}} it follows that
\begin{isabelle}%
\ \ \ \ \ length\ {\isacharbrackleft}x{\isasymin}drop\ i\ v\ {\isachardot}\ x\ {\isacharequal}\ a{\isacharbrackright}\ {\isacharequal}\ length\ {\isacharbrackleft}x{\isasymin}drop\ i\ v\ {\isachardot}\ x\ {\isacharequal}\ b{\isacharbrackright}\ {\isacharplus}\ {\isadigit{1}}%
\end{isabelle}%
\end{isamarkuptxt}%
\isamarkuptrue%
\ \isacommand{apply}\isamarkupfalse%
{\isacharparenleft}drule\ part{\isadigit{2}}{\isacharbrackleft}of\ {\isachardoublequoteopen}{\isasymlambda}x{\isachardot}\ x{\isacharequal}a{\isachardoublequoteclose}{\isacharcomma}\ simplified{\isacharbrackright}{\isacharparenright}\isanewline
\ \ \isacommand{apply}\isamarkupfalse%
{\isacharparenleft}assumption{\isacharparenright}%
\begin{isamarkuptxt}%
\noindent
Now it is time to decompose \isa{v} in the conclusion \isa{b\ {\isacharhash}\ v\ {\isasymin}\ A}
into \isa{take\ i\ v\ {\isacharat}\ drop\ i\ v},%
\end{isamarkuptxt}%
\isamarkuptrue%
\ \isacommand{apply}\isamarkupfalse%
{\isacharparenleft}rule{\isacharunderscore}tac\ n{\isadigit{1}}{\isacharequal}i\ \isakeyword{and}\ t{\isacharequal}v\ \isakeyword{in}\ subst{\isacharbrackleft}OF\ append{\isacharunderscore}take{\isacharunderscore}drop{\isacharunderscore}id{\isacharbrackright}{\isacharparenright}%
\begin{isamarkuptxt}%
\noindent
(the variables \isa{n{\isadigit{1}}} and \isa{t} are the result of composing the
theorems \isa{subst} and \isa{append{\isacharunderscore}take{\isacharunderscore}drop{\isacharunderscore}id})
after which the appropriate rule of the grammar reduces the goal
to the two subgoals \isa{take\ i\ v\ {\isasymin}\ A} and \isa{drop\ i\ v\ {\isasymin}\ A}:%
\end{isamarkuptxt}%
\isamarkuptrue%
\ \isacommand{apply}\isamarkupfalse%
{\isacharparenleft}rule\ S{\isacharunderscore}A{\isacharunderscore}B{\isachardot}intros{\isacharparenright}%
\begin{isamarkuptxt}%
Both subgoals follow from the induction hypothesis because both \isa{take\ i\ v} and \isa{drop\ i\ v} are shorter than \isa{w}:%
\end{isamarkuptxt}%
\isamarkuptrue%
\ \ \isacommand{apply}\isamarkupfalse%
{\isacharparenleft}force\ simp\ add{\isacharcolon}\ min{\isacharunderscore}less{\isacharunderscore}iff{\isacharunderscore}disj{\isacharparenright}\isanewline
\ \isacommand{apply}\isamarkupfalse%
{\isacharparenleft}force\ split\ add{\isacharcolon}\ nat{\isacharunderscore}diff{\isacharunderscore}split{\isacharparenright}%
\begin{isamarkuptxt}%
The case \isa{w\ {\isacharequal}\ b\ {\isacharhash}\ v} is proved analogously:%
\end{isamarkuptxt}%
\isamarkuptrue%
\isacommand{apply}\isamarkupfalse%
{\isacharparenleft}clarify{\isacharparenright}\isanewline
\isacommand{apply}\isamarkupfalse%
{\isacharparenleft}frule\ part{\isadigit{1}}{\isacharbrackleft}of\ {\isachardoublequoteopen}{\isasymlambda}x{\isachardot}\ x{\isacharequal}b{\isachardoublequoteclose}{\isacharcomma}\ simplified{\isacharbrackright}{\isacharparenright}\isanewline
\isacommand{apply}\isamarkupfalse%
{\isacharparenleft}clarify{\isacharparenright}\isanewline
\isacommand{apply}\isamarkupfalse%
{\isacharparenleft}drule\ part{\isadigit{2}}{\isacharbrackleft}of\ {\isachardoublequoteopen}{\isasymlambda}x{\isachardot}\ x{\isacharequal}b{\isachardoublequoteclose}{\isacharcomma}\ simplified{\isacharbrackright}{\isacharparenright}\isanewline
\ \isacommand{apply}\isamarkupfalse%
{\isacharparenleft}assumption{\isacharparenright}\isanewline
\isacommand{apply}\isamarkupfalse%
{\isacharparenleft}rule{\isacharunderscore}tac\ n{\isadigit{1}}{\isacharequal}i\ \isakeyword{and}\ t{\isacharequal}v\ \isakeyword{in}\ subst{\isacharbrackleft}OF\ append{\isacharunderscore}take{\isacharunderscore}drop{\isacharunderscore}id{\isacharbrackright}{\isacharparenright}\isanewline
\isacommand{apply}\isamarkupfalse%
{\isacharparenleft}rule\ S{\isacharunderscore}A{\isacharunderscore}B{\isachardot}intros{\isacharparenright}\isanewline
\ \isacommand{apply}\isamarkupfalse%
{\isacharparenleft}force\ simp\ add{\isacharcolon}\ min{\isacharunderscore}less{\isacharunderscore}iff{\isacharunderscore}disj{\isacharparenright}\isanewline
\isacommand{by}\isamarkupfalse%
{\isacharparenleft}force\ simp\ add{\isacharcolon}\ min{\isacharunderscore}less{\isacharunderscore}iff{\isacharunderscore}disj\ split\ add{\isacharcolon}\ nat{\isacharunderscore}diff{\isacharunderscore}split{\isacharparenright}%
\endisatagproof
{\isafoldproof}%
%
\isadelimproof
%
\endisadelimproof
%
\begin{isamarkuptext}%
We conclude this section with a comparison of our proof with 
Hopcroft\index{Hopcroft, J. E.} and Ullman's\index{Ullman, J. D.}
\cite[p.\ts81]{HopcroftUllman}.
For a start, the textbook
grammar, for no good reason, excludes the empty word, thus complicating
matters just a little bit: they have 8 instead of our 7 productions.

More importantly, the proof itself is different: rather than
separating the two directions, they perform one induction on the
length of a word. This deprives them of the beauty of rule induction,
and in the easy direction (correctness) their reasoning is more
detailed than our \isa{auto}. For the hard part (completeness), they
consider just one of the cases that our \isa{simp{\isacharunderscore}all} disposes of
automatically. Then they conclude the proof by saying about the
remaining cases: ``We do this in a manner similar to our method of
proof for part (1); this part is left to the reader''. But this is
precisely the part that requires the intermediate value theorem and
thus is not at all similar to the other cases (which are automatic in
Isabelle). The authors are at least cavalier about this point and may
even have overlooked the slight difficulty lurking in the omitted
cases.  Such errors are found in many pen-and-paper proofs when they
are scrutinized formally.%
\index{grammars!defining inductively|)}%
\end{isamarkuptext}%
\isamarkuptrue%
%
\isadelimtheory
%
\endisadelimtheory
%
\isatagtheory
%
\endisatagtheory
{\isafoldtheory}%
%
\isadelimtheory
%
\endisadelimtheory
\end{isabellebody}%
%%% Local Variables:
%%% mode: latex
%%% TeX-master: "root"
%%% End:


\index{inductive definitions|)}
