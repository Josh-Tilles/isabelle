%
\begin{isabellebody}%
\def\isabellecontext{Even}%
%
\isadelimtheory
%
\endisadelimtheory
%
\isatagtheory
%
\endisatagtheory
{\isafoldtheory}%
%
\isadelimtheory
%
\endisadelimtheory
%
\isamarkupsection{The Set of Even Numbers%
}
\isamarkuptrue%
%
\begin{isamarkuptext}%
\index{even numbers!defining inductively|(}%
The set of even numbers can be inductively defined as the least set
containing 0 and closed under the operation $+2$.  Obviously,
\emph{even} can also be expressed using the divides relation (\isa{dvd}). 
We shall prove below that the two formulations coincide.  On the way we
shall examine the primary means of reasoning about inductively defined
sets: rule induction.%
\end{isamarkuptext}%
\isamarkuptrue%
%
\isamarkupsubsection{Making an Inductive Definition%
}
\isamarkuptrue%
%
\begin{isamarkuptext}%
Using \commdx{inductive\protect\_set}, we declare the constant \isa{even} to be
a set of natural numbers with the desired properties.%
\end{isamarkuptext}%
\isamarkuptrue%
\isacommand{inductive{\isaliteral{5F}{\isacharunderscore}}set}\isamarkupfalse%
\ even\ {\isaliteral{3A}{\isacharcolon}}{\isaliteral{3A}{\isacharcolon}}\ {\isaliteral{22}{\isachardoublequoteopen}}nat\ set{\isaliteral{22}{\isachardoublequoteclose}}\ \isakeyword{where}\isanewline
zero{\isaliteral{5B}{\isacharbrackleft}}intro{\isaliteral{21}{\isacharbang}}{\isaliteral{5D}{\isacharbrackright}}{\isaliteral{3A}{\isacharcolon}}\ {\isaliteral{22}{\isachardoublequoteopen}}{\isadigit{0}}\ {\isaliteral{5C3C696E3E}{\isasymin}}\ even{\isaliteral{22}{\isachardoublequoteclose}}\ {\isaliteral{7C}{\isacharbar}}\isanewline
step{\isaliteral{5B}{\isacharbrackleft}}intro{\isaliteral{21}{\isacharbang}}{\isaliteral{5D}{\isacharbrackright}}{\isaliteral{3A}{\isacharcolon}}\ {\isaliteral{22}{\isachardoublequoteopen}}n\ {\isaliteral{5C3C696E3E}{\isasymin}}\ even\ {\isaliteral{5C3C4C6F6E6772696768746172726F773E}{\isasymLongrightarrow}}\ {\isaliteral{28}{\isacharparenleft}}Suc\ {\isaliteral{28}{\isacharparenleft}}Suc\ n{\isaliteral{29}{\isacharparenright}}{\isaliteral{29}{\isacharparenright}}\ {\isaliteral{5C3C696E3E}{\isasymin}}\ even{\isaliteral{22}{\isachardoublequoteclose}}%
\begin{isamarkuptext}%
An inductive definition consists of introduction rules.  The first one
above states that 0 is even; the second states that if $n$ is even, then so
is~$n+2$.  Given this declaration, Isabelle generates a fixed point
definition for \isa{even} and proves theorems about it,
thus following the definitional approach (see {\S}\ref{sec:definitional}).
These theorems
include the introduction rules specified in the declaration, an elimination
rule for case analysis and an induction rule.  We can refer to these
theorems by automatically-generated names.  Here are two examples:
\begin{isabelle}%
{\isadigit{0}}\ {\isaliteral{5C3C696E3E}{\isasymin}}\ even\rulename{even{\isaliteral{2E}{\isachardot}}zero}\par\smallskip%
n\ {\isaliteral{5C3C696E3E}{\isasymin}}\ even\ {\isaliteral{5C3C4C6F6E6772696768746172726F773E}{\isasymLongrightarrow}}\ Suc\ {\isaliteral{28}{\isacharparenleft}}Suc\ n{\isaliteral{29}{\isacharparenright}}\ {\isaliteral{5C3C696E3E}{\isasymin}}\ even\rulename{even{\isaliteral{2E}{\isachardot}}step}%
\end{isabelle}

The introduction rules can be given attributes.  Here
both rules are specified as \isa{intro!},%
\index{intro"!@\isa {intro"!} (attribute)}
directing the classical reasoner to 
apply them aggressively. Obviously, regarding 0 as even is safe.  The
\isa{step} rule is also safe because $n+2$ is even if and only if $n$ is
even.  We prove this equivalence later.%
\end{isamarkuptext}%
\isamarkuptrue%
%
\isamarkupsubsection{Using Introduction Rules%
}
\isamarkuptrue%
%
\begin{isamarkuptext}%
Our first lemma states that numbers of the form $2\times k$ are even.
Introduction rules are used to show that specific values belong to the
inductive set.  Such proofs typically involve 
induction, perhaps over some other inductive set.%
\end{isamarkuptext}%
\isamarkuptrue%
\isacommand{lemma}\isamarkupfalse%
\ two{\isaliteral{5F}{\isacharunderscore}}times{\isaliteral{5F}{\isacharunderscore}}even{\isaliteral{5B}{\isacharbrackleft}}intro{\isaliteral{21}{\isacharbang}}{\isaliteral{5D}{\isacharbrackright}}{\isaliteral{3A}{\isacharcolon}}\ {\isaliteral{22}{\isachardoublequoteopen}}{\isadigit{2}}{\isaliteral{2A}{\isacharasterisk}}k\ {\isaliteral{5C3C696E3E}{\isasymin}}\ even{\isaliteral{22}{\isachardoublequoteclose}}\isanewline
%
\isadelimproof
%
\endisadelimproof
%
\isatagproof
\isacommand{apply}\isamarkupfalse%
\ {\isaliteral{28}{\isacharparenleft}}induct{\isaliteral{5F}{\isacharunderscore}}tac\ k{\isaliteral{29}{\isacharparenright}}\isanewline
\ \isacommand{apply}\isamarkupfalse%
\ auto\isanewline
\isacommand{done}\isamarkupfalse%
%
\endisatagproof
{\isafoldproof}%
%
\isadelimproof
%
\endisadelimproof
%
\isadelimproof
%
\endisadelimproof
%
\isatagproof
%
\begin{isamarkuptxt}%
\noindent
The first step is induction on the natural number \isa{k}, which leaves
two subgoals:
\begin{isabelle}%
\ {\isadigit{1}}{\isaliteral{2E}{\isachardot}}\ {\isadigit{2}}\ {\isaliteral{2A}{\isacharasterisk}}\ {\isadigit{0}}\ {\isaliteral{5C3C696E3E}{\isasymin}}\ even\isanewline
\ {\isadigit{2}}{\isaliteral{2E}{\isachardot}}\ {\isaliteral{5C3C416E643E}{\isasymAnd}}n{\isaliteral{2E}{\isachardot}}\ {\isadigit{2}}\ {\isaliteral{2A}{\isacharasterisk}}\ n\ {\isaliteral{5C3C696E3E}{\isasymin}}\ even\ {\isaliteral{5C3C4C6F6E6772696768746172726F773E}{\isasymLongrightarrow}}\ {\isadigit{2}}\ {\isaliteral{2A}{\isacharasterisk}}\ Suc\ n\ {\isaliteral{5C3C696E3E}{\isasymin}}\ even%
\end{isabelle}
Here \isa{auto} simplifies both subgoals so that they match the introduction
rules, which are then applied automatically.

Our ultimate goal is to prove the equivalence between the traditional
definition of \isa{even} (using the divides relation) and our inductive
definition.  One direction of this equivalence is immediate by the lemma
just proved, whose \isa{intro{\isaliteral{21}{\isacharbang}}} attribute ensures it is applied automatically.%
\end{isamarkuptxt}%
\isamarkuptrue%
%
\endisatagproof
{\isafoldproof}%
%
\isadelimproof
%
\endisadelimproof
\isacommand{lemma}\isamarkupfalse%
\ dvd{\isaliteral{5F}{\isacharunderscore}}imp{\isaliteral{5F}{\isacharunderscore}}even{\isaliteral{3A}{\isacharcolon}}\ {\isaliteral{22}{\isachardoublequoteopen}}{\isadigit{2}}\ dvd\ n\ {\isaliteral{5C3C4C6F6E6772696768746172726F773E}{\isasymLongrightarrow}}\ n\ {\isaliteral{5C3C696E3E}{\isasymin}}\ even{\isaliteral{22}{\isachardoublequoteclose}}\isanewline
%
\isadelimproof
%
\endisadelimproof
%
\isatagproof
\isacommand{by}\isamarkupfalse%
\ {\isaliteral{28}{\isacharparenleft}}auto\ simp\ add{\isaliteral{3A}{\isacharcolon}}\ dvd{\isaliteral{5F}{\isacharunderscore}}def{\isaliteral{29}{\isacharparenright}}%
\endisatagproof
{\isafoldproof}%
%
\isadelimproof
%
\endisadelimproof
%
\isamarkupsubsection{Rule Induction \label{sec:rule-induction}%
}
\isamarkuptrue%
%
\begin{isamarkuptext}%
\index{rule induction|(}%
From the definition of the set
\isa{even}, Isabelle has
generated an induction rule:
\begin{isabelle}%
{\isaliteral{5C3C6C6272616B6B3E}{\isasymlbrakk}}x\ {\isaliteral{5C3C696E3E}{\isasymin}}\ even{\isaliteral{3B}{\isacharsemicolon}}\ P\ {\isadigit{0}}{\isaliteral{3B}{\isacharsemicolon}}\isanewline
\isaindent{\ }{\isaliteral{5C3C416E643E}{\isasymAnd}}n{\isaliteral{2E}{\isachardot}}\ {\isaliteral{5C3C6C6272616B6B3E}{\isasymlbrakk}}n\ {\isaliteral{5C3C696E3E}{\isasymin}}\ even{\isaliteral{3B}{\isacharsemicolon}}\ P\ n{\isaliteral{5C3C726272616B6B3E}{\isasymrbrakk}}\ {\isaliteral{5C3C4C6F6E6772696768746172726F773E}{\isasymLongrightarrow}}\ P\ {\isaliteral{28}{\isacharparenleft}}Suc\ {\isaliteral{28}{\isacharparenleft}}Suc\ n{\isaliteral{29}{\isacharparenright}}{\isaliteral{29}{\isacharparenright}}{\isaliteral{5C3C726272616B6B3E}{\isasymrbrakk}}\isanewline
{\isaliteral{5C3C4C6F6E6772696768746172726F773E}{\isasymLongrightarrow}}\ P\ x\rulename{even{\isaliteral{2E}{\isachardot}}induct}%
\end{isabelle}
A property \isa{P} holds for every even number provided it
holds for~\isa{{\isadigit{0}}} and is closed under the operation
\isa{Suc(Suc \(\cdot\))}.  Then \isa{P} is closed under the introduction
rules for \isa{even}, which is the least set closed under those rules. 
This type of inductive argument is called \textbf{rule induction}. 

Apart from the double application of \isa{Suc}, the induction rule above
resembles the familiar mathematical induction, which indeed is an instance
of rule induction; the natural numbers can be defined inductively to be
the least set containing \isa{{\isadigit{0}}} and closed under~\isa{Suc}.

Induction is the usual way of proving a property of the elements of an
inductively defined set.  Let us prove that all members of the set
\isa{even} are multiples of two.%
\end{isamarkuptext}%
\isamarkuptrue%
\isacommand{lemma}\isamarkupfalse%
\ even{\isaliteral{5F}{\isacharunderscore}}imp{\isaliteral{5F}{\isacharunderscore}}dvd{\isaliteral{3A}{\isacharcolon}}\ {\isaliteral{22}{\isachardoublequoteopen}}n\ {\isaliteral{5C3C696E3E}{\isasymin}}\ even\ {\isaliteral{5C3C4C6F6E6772696768746172726F773E}{\isasymLongrightarrow}}\ {\isadigit{2}}\ dvd\ n{\isaliteral{22}{\isachardoublequoteclose}}%
\isadelimproof
%
\endisadelimproof
%
\isatagproof
%
\begin{isamarkuptxt}%
We begin by applying induction.  Note that \isa{even{\isaliteral{2E}{\isachardot}}induct} has the form
of an elimination rule, so we use the method \isa{erule}.  We get two
subgoals:%
\end{isamarkuptxt}%
\isamarkuptrue%
\isacommand{apply}\isamarkupfalse%
\ {\isaliteral{28}{\isacharparenleft}}erule\ even{\isaliteral{2E}{\isachardot}}induct{\isaliteral{29}{\isacharparenright}}%
\begin{isamarkuptxt}%
\begin{isabelle}%
\ {\isadigit{1}}{\isaliteral{2E}{\isachardot}}\ {\isadigit{2}}\ dvd\ {\isadigit{0}}\isanewline
\ {\isadigit{2}}{\isaliteral{2E}{\isachardot}}\ {\isaliteral{5C3C416E643E}{\isasymAnd}}n{\isaliteral{2E}{\isachardot}}\ {\isaliteral{5C3C6C6272616B6B3E}{\isasymlbrakk}}n\ {\isaliteral{5C3C696E3E}{\isasymin}}\ even{\isaliteral{3B}{\isacharsemicolon}}\ {\isadigit{2}}\ dvd\ n{\isaliteral{5C3C726272616B6B3E}{\isasymrbrakk}}\ {\isaliteral{5C3C4C6F6E6772696768746172726F773E}{\isasymLongrightarrow}}\ {\isadigit{2}}\ dvd\ Suc\ {\isaliteral{28}{\isacharparenleft}}Suc\ n{\isaliteral{29}{\isacharparenright}}%
\end{isabelle}
We unfold the definition of \isa{dvd} in both subgoals, proving the first
one and simplifying the second:%
\end{isamarkuptxt}%
\isamarkuptrue%
\isacommand{apply}\isamarkupfalse%
\ {\isaliteral{28}{\isacharparenleft}}simp{\isaliteral{5F}{\isacharunderscore}}all\ add{\isaliteral{3A}{\isacharcolon}}\ dvd{\isaliteral{5F}{\isacharunderscore}}def{\isaliteral{29}{\isacharparenright}}%
\begin{isamarkuptxt}%
\begin{isabelle}%
\ {\isadigit{1}}{\isaliteral{2E}{\isachardot}}\ {\isaliteral{5C3C416E643E}{\isasymAnd}}n{\isaliteral{2E}{\isachardot}}\ {\isaliteral{5C3C6C6272616B6B3E}{\isasymlbrakk}}n\ {\isaliteral{5C3C696E3E}{\isasymin}}\ even{\isaliteral{3B}{\isacharsemicolon}}\ {\isaliteral{5C3C6578697374733E}{\isasymexists}}k{\isaliteral{2E}{\isachardot}}\ n\ {\isaliteral{3D}{\isacharequal}}\ {\isadigit{2}}\ {\isaliteral{2A}{\isacharasterisk}}\ k{\isaliteral{5C3C726272616B6B3E}{\isasymrbrakk}}\ {\isaliteral{5C3C4C6F6E6772696768746172726F773E}{\isasymLongrightarrow}}\ {\isaliteral{5C3C6578697374733E}{\isasymexists}}k{\isaliteral{2E}{\isachardot}}\ Suc\ {\isaliteral{28}{\isacharparenleft}}Suc\ n{\isaliteral{29}{\isacharparenright}}\ {\isaliteral{3D}{\isacharequal}}\ {\isadigit{2}}\ {\isaliteral{2A}{\isacharasterisk}}\ k%
\end{isabelle}
The next command eliminates the existential quantifier from the assumption
and replaces \isa{n} by \isa{{\isadigit{2}}\ {\isaliteral{2A}{\isacharasterisk}}\ k}.%
\end{isamarkuptxt}%
\isamarkuptrue%
\isacommand{apply}\isamarkupfalse%
\ clarify%
\begin{isamarkuptxt}%
\begin{isabelle}%
\ {\isadigit{1}}{\isaliteral{2E}{\isachardot}}\ {\isaliteral{5C3C416E643E}{\isasymAnd}}n\ k{\isaliteral{2E}{\isachardot}}\ {\isadigit{2}}\ {\isaliteral{2A}{\isacharasterisk}}\ k\ {\isaliteral{5C3C696E3E}{\isasymin}}\ even\ {\isaliteral{5C3C4C6F6E6772696768746172726F773E}{\isasymLongrightarrow}}\ {\isaliteral{5C3C6578697374733E}{\isasymexists}}ka{\isaliteral{2E}{\isachardot}}\ Suc\ {\isaliteral{28}{\isacharparenleft}}Suc\ {\isaliteral{28}{\isacharparenleft}}{\isadigit{2}}\ {\isaliteral{2A}{\isacharasterisk}}\ k{\isaliteral{29}{\isacharparenright}}{\isaliteral{29}{\isacharparenright}}\ {\isaliteral{3D}{\isacharequal}}\ {\isadigit{2}}\ {\isaliteral{2A}{\isacharasterisk}}\ ka%
\end{isabelle}
To conclude, we tell Isabelle that the desired value is
\isa{Suc\ k}.  With this hint, the subgoal falls to \isa{simp}.%
\end{isamarkuptxt}%
\isamarkuptrue%
\isacommand{apply}\isamarkupfalse%
\ {\isaliteral{28}{\isacharparenleft}}rule{\isaliteral{5F}{\isacharunderscore}}tac\ x\ {\isaliteral{3D}{\isacharequal}}\ {\isaliteral{22}{\isachardoublequoteopen}}Suc\ k{\isaliteral{22}{\isachardoublequoteclose}}\ \isakeyword{in}\ exI{\isaliteral{2C}{\isacharcomma}}\ simp{\isaliteral{29}{\isacharparenright}}%
\endisatagproof
{\isafoldproof}%
%
\isadelimproof
%
\endisadelimproof
%
\begin{isamarkuptext}%
Combining the previous two results yields our objective, the
equivalence relating \isa{even} and \isa{dvd}. 
%
%we don't want [iff]: discuss?%
\end{isamarkuptext}%
\isamarkuptrue%
\isacommand{theorem}\isamarkupfalse%
\ even{\isaliteral{5F}{\isacharunderscore}}iff{\isaliteral{5F}{\isacharunderscore}}dvd{\isaliteral{3A}{\isacharcolon}}\ {\isaliteral{22}{\isachardoublequoteopen}}{\isaliteral{28}{\isacharparenleft}}n\ {\isaliteral{5C3C696E3E}{\isasymin}}\ even{\isaliteral{29}{\isacharparenright}}\ {\isaliteral{3D}{\isacharequal}}\ {\isaliteral{28}{\isacharparenleft}}{\isadigit{2}}\ dvd\ n{\isaliteral{29}{\isacharparenright}}{\isaliteral{22}{\isachardoublequoteclose}}\isanewline
%
\isadelimproof
%
\endisadelimproof
%
\isatagproof
\isacommand{by}\isamarkupfalse%
\ {\isaliteral{28}{\isacharparenleft}}blast\ intro{\isaliteral{3A}{\isacharcolon}}\ dvd{\isaliteral{5F}{\isacharunderscore}}imp{\isaliteral{5F}{\isacharunderscore}}even\ even{\isaliteral{5F}{\isacharunderscore}}imp{\isaliteral{5F}{\isacharunderscore}}dvd{\isaliteral{29}{\isacharparenright}}%
\endisatagproof
{\isafoldproof}%
%
\isadelimproof
%
\endisadelimproof
%
\isamarkupsubsection{Generalization and Rule Induction \label{sec:gen-rule-induction}%
}
\isamarkuptrue%
%
\begin{isamarkuptext}%
\index{generalizing for induction}%
Before applying induction, we typically must generalize
the induction formula.  With rule induction, the required generalization
can be hard to find and sometimes requires a complete reformulation of the
problem.  In this  example, our first attempt uses the obvious statement of
the result.  It fails:%
\end{isamarkuptext}%
\isamarkuptrue%
\isacommand{lemma}\isamarkupfalse%
\ {\isaliteral{22}{\isachardoublequoteopen}}Suc\ {\isaliteral{28}{\isacharparenleft}}Suc\ n{\isaliteral{29}{\isacharparenright}}\ {\isaliteral{5C3C696E3E}{\isasymin}}\ even\ {\isaliteral{5C3C4C6F6E6772696768746172726F773E}{\isasymLongrightarrow}}\ n\ {\isaliteral{5C3C696E3E}{\isasymin}}\ even{\isaliteral{22}{\isachardoublequoteclose}}\isanewline
%
\isadelimproof
%
\endisadelimproof
%
\isatagproof
\isacommand{apply}\isamarkupfalse%
\ {\isaliteral{28}{\isacharparenleft}}erule\ even{\isaliteral{2E}{\isachardot}}induct{\isaliteral{29}{\isacharparenright}}\isanewline
\isacommand{oops}\isamarkupfalse%
%
\endisatagproof
{\isafoldproof}%
%
\isadelimproof
%
\endisadelimproof
%
\isadelimproof
%
\endisadelimproof
%
\isatagproof
%
\begin{isamarkuptxt}%
Rule induction finds no occurrences of \isa{Suc\ {\isaliteral{28}{\isacharparenleft}}Suc\ n{\isaliteral{29}{\isacharparenright}}} in the
conclusion, which it therefore leaves unchanged.  (Look at
\isa{even{\isaliteral{2E}{\isachardot}}induct} to see why this happens.)  We have these subgoals:
\begin{isabelle}%
\ {\isadigit{1}}{\isaliteral{2E}{\isachardot}}\ n\ {\isaliteral{5C3C696E3E}{\isasymin}}\ even\isanewline
\ {\isadigit{2}}{\isaliteral{2E}{\isachardot}}\ {\isaliteral{5C3C416E643E}{\isasymAnd}}na{\isaliteral{2E}{\isachardot}}\ {\isaliteral{5C3C6C6272616B6B3E}{\isasymlbrakk}}na\ {\isaliteral{5C3C696E3E}{\isasymin}}\ even{\isaliteral{3B}{\isacharsemicolon}}\ n\ {\isaliteral{5C3C696E3E}{\isasymin}}\ even{\isaliteral{5C3C726272616B6B3E}{\isasymrbrakk}}\ {\isaliteral{5C3C4C6F6E6772696768746172726F773E}{\isasymLongrightarrow}}\ n\ {\isaliteral{5C3C696E3E}{\isasymin}}\ even%
\end{isabelle}
The first one is hopeless.  Rule induction on
a non-variable term discards information, and usually fails.
How to deal with such situations
in general is described in {\S}\ref{sec:ind-var-in-prems} below.
In the current case the solution is easy because
we have the necessary inverse, subtraction:%
\end{isamarkuptxt}%
\isamarkuptrue%
%
\endisatagproof
{\isafoldproof}%
%
\isadelimproof
%
\endisadelimproof
\isacommand{lemma}\isamarkupfalse%
\ even{\isaliteral{5F}{\isacharunderscore}}imp{\isaliteral{5F}{\isacharunderscore}}even{\isaliteral{5F}{\isacharunderscore}}minus{\isaliteral{5F}{\isacharunderscore}}{\isadigit{2}}{\isaliteral{3A}{\isacharcolon}}\ {\isaliteral{22}{\isachardoublequoteopen}}n\ {\isaliteral{5C3C696E3E}{\isasymin}}\ even\ {\isaliteral{5C3C4C6F6E6772696768746172726F773E}{\isasymLongrightarrow}}\ n\ {\isaliteral{2D}{\isacharminus}}\ {\isadigit{2}}\ {\isaliteral{5C3C696E3E}{\isasymin}}\ even{\isaliteral{22}{\isachardoublequoteclose}}\isanewline
%
\isadelimproof
%
\endisadelimproof
%
\isatagproof
\isacommand{apply}\isamarkupfalse%
\ {\isaliteral{28}{\isacharparenleft}}erule\ even{\isaliteral{2E}{\isachardot}}induct{\isaliteral{29}{\isacharparenright}}\isanewline
\ \isacommand{apply}\isamarkupfalse%
\ auto\isanewline
\isacommand{done}\isamarkupfalse%
%
\endisatagproof
{\isafoldproof}%
%
\isadelimproof
%
\endisadelimproof
%
\isadelimproof
%
\endisadelimproof
%
\isatagproof
%
\begin{isamarkuptxt}%
This lemma is trivially inductive.  Here are the subgoals:
\begin{isabelle}%
\ {\isadigit{1}}{\isaliteral{2E}{\isachardot}}\ {\isadigit{0}}\ {\isaliteral{2D}{\isacharminus}}\ {\isadigit{2}}\ {\isaliteral{5C3C696E3E}{\isasymin}}\ even\isanewline
\ {\isadigit{2}}{\isaliteral{2E}{\isachardot}}\ {\isaliteral{5C3C416E643E}{\isasymAnd}}n{\isaliteral{2E}{\isachardot}}\ {\isaliteral{5C3C6C6272616B6B3E}{\isasymlbrakk}}n\ {\isaliteral{5C3C696E3E}{\isasymin}}\ even{\isaliteral{3B}{\isacharsemicolon}}\ n\ {\isaliteral{2D}{\isacharminus}}\ {\isadigit{2}}\ {\isaliteral{5C3C696E3E}{\isasymin}}\ even{\isaliteral{5C3C726272616B6B3E}{\isasymrbrakk}}\ {\isaliteral{5C3C4C6F6E6772696768746172726F773E}{\isasymLongrightarrow}}\ Suc\ {\isaliteral{28}{\isacharparenleft}}Suc\ n{\isaliteral{29}{\isacharparenright}}\ {\isaliteral{2D}{\isacharminus}}\ {\isadigit{2}}\ {\isaliteral{5C3C696E3E}{\isasymin}}\ even%
\end{isabelle}
The first is trivial because \isa{{\isadigit{0}}\ {\isaliteral{2D}{\isacharminus}}\ {\isadigit{2}}} simplifies to \isa{{\isadigit{0}}}, which is
even.  The second is trivial too: \isa{Suc\ {\isaliteral{28}{\isacharparenleft}}Suc\ n{\isaliteral{29}{\isacharparenright}}\ {\isaliteral{2D}{\isacharminus}}\ {\isadigit{2}}} simplifies to
\isa{n}, matching the assumption.%
\index{rule induction|)}  %the sequel isn't really about induction

\medskip
Using our lemma, we can easily prove the result we originally wanted:%
\end{isamarkuptxt}%
\isamarkuptrue%
%
\endisatagproof
{\isafoldproof}%
%
\isadelimproof
%
\endisadelimproof
\isacommand{lemma}\isamarkupfalse%
\ Suc{\isaliteral{5F}{\isacharunderscore}}Suc{\isaliteral{5F}{\isacharunderscore}}even{\isaliteral{5F}{\isacharunderscore}}imp{\isaliteral{5F}{\isacharunderscore}}even{\isaliteral{3A}{\isacharcolon}}\ {\isaliteral{22}{\isachardoublequoteopen}}Suc\ {\isaliteral{28}{\isacharparenleft}}Suc\ n{\isaliteral{29}{\isacharparenright}}\ {\isaliteral{5C3C696E3E}{\isasymin}}\ even\ {\isaliteral{5C3C4C6F6E6772696768746172726F773E}{\isasymLongrightarrow}}\ n\ {\isaliteral{5C3C696E3E}{\isasymin}}\ even{\isaliteral{22}{\isachardoublequoteclose}}\isanewline
%
\isadelimproof
%
\endisadelimproof
%
\isatagproof
\isacommand{by}\isamarkupfalse%
\ {\isaliteral{28}{\isacharparenleft}}drule\ even{\isaliteral{5F}{\isacharunderscore}}imp{\isaliteral{5F}{\isacharunderscore}}even{\isaliteral{5F}{\isacharunderscore}}minus{\isaliteral{5F}{\isacharunderscore}}{\isadigit{2}}{\isaliteral{2C}{\isacharcomma}}\ simp{\isaliteral{29}{\isacharparenright}}%
\endisatagproof
{\isafoldproof}%
%
\isadelimproof
%
\endisadelimproof
%
\begin{isamarkuptext}%
We have just proved the converse of the introduction rule \isa{even{\isaliteral{2E}{\isachardot}}step}.
This suggests proving the following equivalence.  We give it the
\attrdx{iff} attribute because of its obvious value for simplification.%
\end{isamarkuptext}%
\isamarkuptrue%
\isacommand{lemma}\isamarkupfalse%
\ {\isaliteral{5B}{\isacharbrackleft}}iff{\isaliteral{5D}{\isacharbrackright}}{\isaliteral{3A}{\isacharcolon}}\ {\isaliteral{22}{\isachardoublequoteopen}}{\isaliteral{28}{\isacharparenleft}}{\isaliteral{28}{\isacharparenleft}}Suc\ {\isaliteral{28}{\isacharparenleft}}Suc\ n{\isaliteral{29}{\isacharparenright}}{\isaliteral{29}{\isacharparenright}}\ {\isaliteral{5C3C696E3E}{\isasymin}}\ even{\isaliteral{29}{\isacharparenright}}\ {\isaliteral{3D}{\isacharequal}}\ {\isaliteral{28}{\isacharparenleft}}n\ {\isaliteral{5C3C696E3E}{\isasymin}}\ even{\isaliteral{29}{\isacharparenright}}{\isaliteral{22}{\isachardoublequoteclose}}\isanewline
%
\isadelimproof
%
\endisadelimproof
%
\isatagproof
\isacommand{by}\isamarkupfalse%
\ {\isaliteral{28}{\isacharparenleft}}blast\ dest{\isaliteral{3A}{\isacharcolon}}\ Suc{\isaliteral{5F}{\isacharunderscore}}Suc{\isaliteral{5F}{\isacharunderscore}}even{\isaliteral{5F}{\isacharunderscore}}imp{\isaliteral{5F}{\isacharunderscore}}even{\isaliteral{29}{\isacharparenright}}%
\endisatagproof
{\isafoldproof}%
%
\isadelimproof
%
\endisadelimproof
%
\isamarkupsubsection{Rule Inversion \label{sec:rule-inversion}%
}
\isamarkuptrue%
%
\begin{isamarkuptext}%
\index{rule inversion|(}%
Case analysis on an inductive definition is called \textbf{rule
inversion}.  It is frequently used in proofs about operational
semantics.  It can be highly effective when it is applied
automatically.  Let us look at how rule inversion is done in
Isabelle/HOL\@.

Recall that \isa{even} is the minimal set closed under these two rules:
\begin{isabelle}%
{\isadigit{0}}\ {\isaliteral{5C3C696E3E}{\isasymin}}\ even\isasep\isanewline%
n\ {\isaliteral{5C3C696E3E}{\isasymin}}\ even\ {\isaliteral{5C3C4C6F6E6772696768746172726F773E}{\isasymLongrightarrow}}\ Suc\ {\isaliteral{28}{\isacharparenleft}}Suc\ n{\isaliteral{29}{\isacharparenright}}\ {\isaliteral{5C3C696E3E}{\isasymin}}\ even%
\end{isabelle}
Minimality means that \isa{even} contains only the elements that these
rules force it to contain.  If we are told that \isa{a}
belongs to
\isa{even} then there are only two possibilities.  Either \isa{a} is \isa{{\isadigit{0}}}
or else \isa{a} has the form \isa{Suc\ {\isaliteral{28}{\isacharparenleft}}Suc\ n{\isaliteral{29}{\isacharparenright}}}, for some suitable \isa{n}
that belongs to
\isa{even}.  That is the gist of the \isa{cases} rule, which Isabelle proves
for us when it accepts an inductive definition:
\begin{isabelle}%
{\isaliteral{5C3C6C6272616B6B3E}{\isasymlbrakk}}a\ {\isaliteral{5C3C696E3E}{\isasymin}}\ even{\isaliteral{3B}{\isacharsemicolon}}\ a\ {\isaliteral{3D}{\isacharequal}}\ {\isadigit{0}}\ {\isaliteral{5C3C4C6F6E6772696768746172726F773E}{\isasymLongrightarrow}}\ P{\isaliteral{3B}{\isacharsemicolon}}\isanewline
\isaindent{\ }{\isaliteral{5C3C416E643E}{\isasymAnd}}n{\isaliteral{2E}{\isachardot}}\ {\isaliteral{5C3C6C6272616B6B3E}{\isasymlbrakk}}a\ {\isaliteral{3D}{\isacharequal}}\ Suc\ {\isaliteral{28}{\isacharparenleft}}Suc\ n{\isaliteral{29}{\isacharparenright}}{\isaliteral{3B}{\isacharsemicolon}}\ n\ {\isaliteral{5C3C696E3E}{\isasymin}}\ even{\isaliteral{5C3C726272616B6B3E}{\isasymrbrakk}}\ {\isaliteral{5C3C4C6F6E6772696768746172726F773E}{\isasymLongrightarrow}}\ P{\isaliteral{5C3C726272616B6B3E}{\isasymrbrakk}}\isanewline
{\isaliteral{5C3C4C6F6E6772696768746172726F773E}{\isasymLongrightarrow}}\ P\rulename{even{\isaliteral{2E}{\isachardot}}cases}%
\end{isabelle}
This general rule is less useful than instances of it for
specific patterns.  For example, if \isa{a} has the form
\isa{Suc\ {\isaliteral{28}{\isacharparenleft}}Suc\ n{\isaliteral{29}{\isacharparenright}}} then the first case becomes irrelevant, while the second
case tells us that \isa{n} belongs to \isa{even}.  Isabelle will generate
this instance for us:%
\end{isamarkuptext}%
\isamarkuptrue%
\isacommand{inductive{\isaliteral{5F}{\isacharunderscore}}cases}\isamarkupfalse%
\ Suc{\isaliteral{5F}{\isacharunderscore}}Suc{\isaliteral{5F}{\isacharunderscore}}cases\ {\isaliteral{5B}{\isacharbrackleft}}elim{\isaliteral{21}{\isacharbang}}{\isaliteral{5D}{\isacharbrackright}}{\isaliteral{3A}{\isacharcolon}}\ {\isaliteral{22}{\isachardoublequoteopen}}Suc{\isaliteral{28}{\isacharparenleft}}Suc\ n{\isaliteral{29}{\isacharparenright}}\ {\isaliteral{5C3C696E3E}{\isasymin}}\ even{\isaliteral{22}{\isachardoublequoteclose}}%
\begin{isamarkuptext}%
The \commdx{inductive\protect\_cases} command generates an instance of
the \isa{cases} rule for the supplied pattern and gives it the supplied name:
\begin{isabelle}%
{\isaliteral{5C3C6C6272616B6B3E}{\isasymlbrakk}}Suc\ {\isaliteral{28}{\isacharparenleft}}Suc\ n{\isaliteral{29}{\isacharparenright}}\ {\isaliteral{5C3C696E3E}{\isasymin}}\ even{\isaliteral{3B}{\isacharsemicolon}}\ n\ {\isaliteral{5C3C696E3E}{\isasymin}}\ even\ {\isaliteral{5C3C4C6F6E6772696768746172726F773E}{\isasymLongrightarrow}}\ P{\isaliteral{5C3C726272616B6B3E}{\isasymrbrakk}}\ {\isaliteral{5C3C4C6F6E6772696768746172726F773E}{\isasymLongrightarrow}}\ P\rulename{Suc{\isaliteral{5F}{\isacharunderscore}}Suc{\isaliteral{5F}{\isacharunderscore}}cases}%
\end{isabelle}
Applying this as an elimination rule yields one case where \isa{even{\isaliteral{2E}{\isachardot}}cases}
would yield two.  Rule inversion works well when the conclusions of the
introduction rules involve datatype constructors like \isa{Suc} and \isa{{\isaliteral{23}{\isacharhash}}}
(list ``cons''); freeness reasoning discards all but one or two cases.

In the \isacommand{inductive\_cases} command we supplied an
attribute, \isa{elim{\isaliteral{21}{\isacharbang}}},
\index{elim"!@\isa {elim"!} (attribute)}%
indicating that this elimination rule can be
applied aggressively.  The original
\isa{cases} rule would loop if used in that manner because the
pattern~\isa{a} matches everything.

The rule \isa{Suc{\isaliteral{5F}{\isacharunderscore}}Suc{\isaliteral{5F}{\isacharunderscore}}cases} is equivalent to the following implication:
\begin{isabelle}%
Suc\ {\isaliteral{28}{\isacharparenleft}}Suc\ n{\isaliteral{29}{\isacharparenright}}\ {\isaliteral{5C3C696E3E}{\isasymin}}\ even\ {\isaliteral{5C3C4C6F6E6772696768746172726F773E}{\isasymLongrightarrow}}\ n\ {\isaliteral{5C3C696E3E}{\isasymin}}\ even%
\end{isabelle}
Just above we devoted some effort to reaching precisely
this result.  Yet we could have obtained it by a one-line declaration,
dispensing with the lemma \isa{even{\isaliteral{5F}{\isacharunderscore}}imp{\isaliteral{5F}{\isacharunderscore}}even{\isaliteral{5F}{\isacharunderscore}}minus{\isaliteral{5F}{\isacharunderscore}}{\isadigit{2}}}. 
This example also justifies the terminology
\textbf{rule inversion}: the new rule inverts the introduction rule
\isa{even{\isaliteral{2E}{\isachardot}}step}.  In general, a rule can be inverted when the set of elements
it introduces is disjoint from those of the other introduction rules.

For one-off applications of rule inversion, use the \methdx{ind_cases} method. 
Here is an example:%
\end{isamarkuptext}%
\isamarkuptrue%
%
\isadelimproof
%
\endisadelimproof
%
\isatagproof
\isacommand{apply}\isamarkupfalse%
\ {\isaliteral{28}{\isacharparenleft}}ind{\isaliteral{5F}{\isacharunderscore}}cases\ {\isaliteral{22}{\isachardoublequoteopen}}Suc{\isaliteral{28}{\isacharparenleft}}Suc\ n{\isaliteral{29}{\isacharparenright}}\ {\isaliteral{5C3C696E3E}{\isasymin}}\ even{\isaliteral{22}{\isachardoublequoteclose}}{\isaliteral{29}{\isacharparenright}}%
\endisatagproof
{\isafoldproof}%
%
\isadelimproof
%
\endisadelimproof
%
\begin{isamarkuptext}%
The specified instance of the \isa{cases} rule is generated, then applied
as an elimination rule.

To summarize, every inductive definition produces a \isa{cases} rule.  The
\commdx{inductive\protect\_cases} command stores an instance of the
\isa{cases} rule for a given pattern.  Within a proof, the
\isa{ind{\isaliteral{5F}{\isacharunderscore}}cases} method applies an instance of the \isa{cases}
rule.

The even numbers example has shown how inductive definitions can be
used.  Later examples will show that they are actually worth using.%
\index{rule inversion|)}%
\index{even numbers!defining inductively|)}%
\end{isamarkuptext}%
\isamarkuptrue%
%
\isadelimtheory
%
\endisadelimtheory
%
\isatagtheory
%
\endisatagtheory
{\isafoldtheory}%
%
\isadelimtheory
%
\endisadelimtheory
\end{isabellebody}%
%%% Local Variables:
%%% mode: latex
%%% TeX-master: "root"
%%% End:
