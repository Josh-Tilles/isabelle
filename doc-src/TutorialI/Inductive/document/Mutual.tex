%
\begin{isabellebody}%
\def\isabellecontext{Mutual}%
\isamarkupfalse%
%
\isamarkupsubsection{Mutually Inductive Definitions%
}
\isamarkuptrue%
%
\begin{isamarkuptext}%
Just as there are datatypes defined by mutual recursion, there are sets defined
by mutual induction. As a trivial example we consider the even and odd
natural numbers:%
\end{isamarkuptext}%
\isamarkuptrue%
\isacommand{consts}\ even\ {\isacharcolon}{\isacharcolon}\ {\isachardoublequote}nat\ set{\isachardoublequote}\isanewline
\ \ \ \ \ \ \ odd\ \ {\isacharcolon}{\isacharcolon}\ {\isachardoublequote}nat\ set{\isachardoublequote}\isanewline
\isanewline
\isamarkupfalse%
\isacommand{inductive}\ even\ odd\isanewline
\isakeyword{intros}\isanewline
zero{\isacharcolon}\ \ {\isachardoublequote}{\isadigit{0}}\ {\isasymin}\ even{\isachardoublequote}\isanewline
evenI{\isacharcolon}\ {\isachardoublequote}n\ {\isasymin}\ odd\ {\isasymLongrightarrow}\ Suc\ n\ {\isasymin}\ even{\isachardoublequote}\isanewline
oddI{\isacharcolon}\ \ {\isachardoublequote}n\ {\isasymin}\ even\ {\isasymLongrightarrow}\ Suc\ n\ {\isasymin}\ odd{\isachardoublequote}\isamarkupfalse%
%
\begin{isamarkuptext}%
\noindent
The mutually inductive definition of multiple sets is no different from
that of a single set, except for induction: just as for mutually recursive
datatypes, induction needs to involve all the simultaneously defined sets. In
the above case, the induction rule is called \isa{even{\isacharunderscore}odd{\isachardot}induct}
(simply concatenate the names of the sets involved) and has the conclusion
\begin{isabelle}%
\ \ \ \ \ {\isacharparenleft}{\isacharquery}x\ {\isasymin}\ even\ {\isasymlongrightarrow}\ {\isacharquery}P\ {\isacharquery}x{\isacharparenright}\ {\isasymand}\ {\isacharparenleft}{\isacharquery}y\ {\isasymin}\ odd\ {\isasymlongrightarrow}\ {\isacharquery}Q\ {\isacharquery}y{\isacharparenright}%
\end{isabelle}

If we want to prove that all even numbers are divisible by two, we have to
generalize the statement as follows:%
\end{isamarkuptext}%
\isamarkuptrue%
\isacommand{lemma}\ {\isachardoublequote}{\isacharparenleft}m\ {\isasymin}\ even\ {\isasymlongrightarrow}\ {\isadigit{2}}\ dvd\ m{\isacharparenright}\ {\isasymand}\ {\isacharparenleft}n\ {\isasymin}\ odd\ {\isasymlongrightarrow}\ {\isadigit{2}}\ dvd\ {\isacharparenleft}Suc\ n{\isacharparenright}{\isacharparenright}{\isachardoublequote}\isamarkupfalse%
%
\begin{isamarkuptxt}%
\noindent
The proof is by rule induction. Because of the form of the induction theorem,
it is applied by \isa{rule} rather than \isa{erule} as for ordinary
inductive definitions:%
\end{isamarkuptxt}%
\isamarkuptrue%
\isacommand{apply}{\isacharparenleft}rule\ even{\isacharunderscore}odd{\isachardot}induct{\isacharparenright}\isamarkupfalse%
%
\begin{isamarkuptxt}%
\begin{isabelle}%
\ {\isadigit{1}}{\isachardot}\ {\isadigit{2}}\ dvd\ {\isadigit{0}}\isanewline
\ {\isadigit{2}}{\isachardot}\ {\isasymAnd}n{\isachardot}\ {\isasymlbrakk}n\ {\isasymin}\ odd{\isacharsemicolon}\ {\isadigit{2}}\ dvd\ Suc\ n{\isasymrbrakk}\ {\isasymLongrightarrow}\ {\isadigit{2}}\ dvd\ Suc\ n\isanewline
\ {\isadigit{3}}{\isachardot}\ {\isasymAnd}n{\isachardot}\ {\isasymlbrakk}n\ {\isasymin}\ Mutual{\isachardot}even{\isacharsemicolon}\ {\isadigit{2}}\ dvd\ n{\isasymrbrakk}\ {\isasymLongrightarrow}\ {\isadigit{2}}\ dvd\ Suc\ {\isacharparenleft}Suc\ n{\isacharparenright}%
\end{isabelle}
The first two subgoals are proved by simplification and the final one can be
proved in the same manner as in \S\ref{sec:rule-induction}
where the same subgoal was encountered before.
We do not show the proof script.%
\end{isamarkuptxt}%
\isamarkuptrue%
\isamarkupfalse%
\isamarkupfalse%
\isamarkupfalse%
\isamarkupfalse%
\isamarkupfalse%
\isamarkupfalse%
\isamarkupfalse%
\isamarkupfalse%
\end{isabellebody}%
%%% Local Variables:
%%% mode: latex
%%% TeX-master: "root"
%%% End:
