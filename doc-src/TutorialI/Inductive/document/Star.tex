%
\begin{isabellebody}%
\def\isabellecontext{Star}%
%
\isamarkupsection{The reflexive transitive closure}
%
\begin{isamarkuptext}%
A perfect example of an inductive definition is the reflexive transitive
closure of a relation. This concept was already introduced in
\S\ref{sec:rtrancl}, but it was not shown how it is defined. In fact,
the operator \isa{{\isacharcircum}{\isacharasterisk}} is not defined inductively but via a least
fixpoint because at that point in the theory hierarchy
inductive definitions are not yet available. But now they are:%
\end{isamarkuptext}%
\isacommand{consts}\ rtc\ {\isacharcolon}{\isacharcolon}\ {\isachardoublequote}{\isacharparenleft}{\isacharprime}a\ {\isasymtimes}\ {\isacharprime}a{\isacharparenright}set\ {\isasymRightarrow}\ {\isacharparenleft}{\isacharprime}a\ {\isasymtimes}\ {\isacharprime}a{\isacharparenright}set{\isachardoublequote}\ \ {\isacharparenleft}{\isachardoublequote}{\isacharunderscore}{\isacharasterisk}{\isachardoublequote}\ {\isacharbrackleft}{\isadigit{1}}{\isadigit{0}}{\isadigit{0}}{\isadigit{0}}{\isacharbrackright}\ {\isadigit{9}}{\isadigit{9}}{\isadigit{9}}{\isacharparenright}\isanewline
\isacommand{inductive}\ {\isachardoublequote}r{\isacharasterisk}{\isachardoublequote}\isanewline
\isakeyword{intros}\isanewline
rtc{\isacharunderscore}refl{\isacharbrackleft}intro{\isacharbang}{\isacharbrackright}{\isacharcolon}\ \ \ \ \ \ \ \ \ \ \ \ \ \ \ \ \ \ \ \ \ \ \ \ {\isachardoublequote}{\isacharparenleft}x{\isacharcomma}x{\isacharparenright}\ {\isasymin}\ r{\isacharasterisk}{\isachardoublequote}\isanewline
rtc{\isacharunderscore}step{\isacharcolon}\ {\isachardoublequote}{\isasymlbrakk}\ {\isacharparenleft}x{\isacharcomma}y{\isacharparenright}\ {\isasymin}\ r{\isacharasterisk}{\isacharsemicolon}\ {\isacharparenleft}y{\isacharcomma}z{\isacharparenright}\ {\isasymin}\ r\ {\isasymrbrakk}\ {\isasymLongrightarrow}\ {\isacharparenleft}x{\isacharcomma}z{\isacharparenright}\ {\isasymin}\ r{\isacharasterisk}{\isachardoublequote}\isanewline
\isanewline
\isacommand{lemma}\ {\isacharbrackleft}intro{\isacharbrackright}{\isacharcolon}\ {\isachardoublequote}{\isacharparenleft}x{\isacharcomma}y{\isacharparenright}\ {\isacharcolon}\ r\ {\isasymLongrightarrow}\ {\isacharparenleft}x{\isacharcomma}y{\isacharparenright}\ {\isasymin}\ r{\isacharasterisk}{\isachardoublequote}\isanewline
\isacommand{by}{\isacharparenleft}blast\ intro{\isacharcolon}\ rtc{\isacharunderscore}step{\isacharparenright}\isanewline
\isanewline
\isacommand{lemma}\ step{\isadigit{2}}{\isacharbrackleft}rule{\isacharunderscore}format{\isacharbrackright}{\isacharcolon}\isanewline
\ \ {\isachardoublequote}{\isacharparenleft}y{\isacharcomma}z{\isacharparenright}\ {\isasymin}\ r{\isacharasterisk}\ \ {\isasymLongrightarrow}\ {\isacharparenleft}x{\isacharcomma}y{\isacharparenright}\ {\isasymin}\ r\ {\isasymlongrightarrow}\ {\isacharparenleft}x{\isacharcomma}z{\isacharparenright}\ {\isasymin}\ r{\isacharasterisk}{\isachardoublequote}\isanewline
\isacommand{apply}{\isacharparenleft}erule\ rtc{\isachardot}induct{\isacharparenright}\isanewline
\ \isacommand{apply}{\isacharparenleft}blast{\isacharparenright}\isanewline
\isacommand{apply}{\isacharparenleft}blast\ intro{\isacharcolon}\ rtc{\isacharunderscore}step{\isacharparenright}\isanewline
\isacommand{done}\isanewline
\isanewline
\isacommand{lemma}\ rtc{\isacharunderscore}trans{\isacharbrackleft}rule{\isacharunderscore}format{\isacharbrackright}{\isacharcolon}\isanewline
\ \ {\isachardoublequote}{\isacharparenleft}x{\isacharcomma}y{\isacharparenright}\ {\isasymin}\ r{\isacharasterisk}\ \ {\isasymLongrightarrow}\ {\isasymforall}z{\isachardot}\ {\isacharparenleft}y{\isacharcomma}z{\isacharparenright}\ {\isasymin}\ r{\isacharasterisk}\ {\isasymlongrightarrow}\ {\isacharparenleft}x{\isacharcomma}z{\isacharparenright}\ {\isasymin}\ r{\isacharasterisk}{\isachardoublequote}\isanewline
\isacommand{apply}{\isacharparenleft}erule\ rtc{\isachardot}induct{\isacharparenright}\isanewline
\ \isacommand{apply}\ blast\isanewline
\isacommand{apply}{\isacharparenleft}blast\ intro{\isacharcolon}\ step{\isadigit{2}}{\isacharparenright}\isanewline
\isacommand{done}\isanewline
\isanewline
\isacommand{consts}\ rtc{\isadigit{2}}\ {\isacharcolon}{\isacharcolon}\ {\isachardoublequote}{\isacharparenleft}{\isacharprime}a\ {\isasymtimes}\ {\isacharprime}a{\isacharparenright}set\ {\isasymRightarrow}\ {\isacharparenleft}{\isacharprime}a\ {\isasymtimes}\ {\isacharprime}a{\isacharparenright}set{\isachardoublequote}\isanewline
\isacommand{inductive}\ {\isachardoublequote}rtc{\isadigit{2}}\ r{\isachardoublequote}\isanewline
\isakeyword{intros}\isanewline
{\isachardoublequote}{\isacharparenleft}x{\isacharcomma}y{\isacharparenright}\ {\isasymin}\ r\ {\isasymLongrightarrow}\ {\isacharparenleft}x{\isacharcomma}y{\isacharparenright}\ {\isasymin}\ rtc{\isadigit{2}}\ r{\isachardoublequote}\isanewline
{\isachardoublequote}{\isacharparenleft}x{\isacharcomma}x{\isacharparenright}\ {\isasymin}\ rtc{\isadigit{2}}\ r{\isachardoublequote}\isanewline
{\isachardoublequote}{\isasymlbrakk}\ {\isacharparenleft}x{\isacharcomma}y{\isacharparenright}\ {\isasymin}\ rtc{\isadigit{2}}\ r{\isacharsemicolon}\ {\isacharparenleft}y{\isacharcomma}z{\isacharparenright}\ {\isasymin}\ rtc{\isadigit{2}}\ r\ {\isasymrbrakk}\ {\isasymLongrightarrow}\ {\isacharparenleft}x{\isacharcomma}z{\isacharparenright}\ {\isasymin}\ rtc{\isadigit{2}}\ r{\isachardoublequote}%
\begin{isamarkuptext}%
The equivalence of the two definitions is easily shown by the obvious rule
inductions:%
\end{isamarkuptext}%
\isacommand{lemma}\ {\isachardoublequote}{\isacharparenleft}x{\isacharcomma}y{\isacharparenright}\ {\isasymin}\ rtc{\isadigit{2}}\ r\ {\isasymLongrightarrow}\ {\isacharparenleft}x{\isacharcomma}y{\isacharparenright}\ {\isasymin}\ r{\isacharasterisk}{\isachardoublequote}\isanewline
\isacommand{apply}{\isacharparenleft}erule\ rtc{\isadigit{2}}{\isachardot}induct{\isacharparenright}\isanewline
\ \ \isacommand{apply}{\isacharparenleft}blast{\isacharparenright}\isanewline
\ \isacommand{apply}{\isacharparenleft}blast{\isacharparenright}\isanewline
\isacommand{apply}{\isacharparenleft}blast\ intro{\isacharcolon}\ rtc{\isacharunderscore}trans{\isacharparenright}\isanewline
\isacommand{done}\isanewline
\isanewline
\isacommand{lemma}\ {\isachardoublequote}{\isacharparenleft}x{\isacharcomma}y{\isacharparenright}\ {\isasymin}\ r{\isacharasterisk}\ {\isasymLongrightarrow}\ {\isacharparenleft}x{\isacharcomma}y{\isacharparenright}\ {\isasymin}\ rtc{\isadigit{2}}\ r{\isachardoublequote}\isanewline
\isacommand{apply}{\isacharparenleft}erule\ rtc{\isachardot}induct{\isacharparenright}\isanewline
\ \isacommand{apply}{\isacharparenleft}blast\ intro{\isacharcolon}\ rtc{\isadigit{2}}{\isachardot}intros{\isacharparenright}\isanewline
\isacommand{apply}{\isacharparenleft}blast\ intro{\isacharcolon}\ rtc{\isadigit{2}}{\isachardot}intros{\isacharparenright}\isanewline
\isacommand{done}\isanewline
\end{isabellebody}%
%%% Local Variables:
%%% mode: latex
%%% TeX-master: "root"
%%% End:
