\begin{isabelle}%
%
\begin{isamarkuptext}%
So far all examples of rewrite rules were equations. The simplifier also
accepts \emph{conditional} equations, for example%
\end{isamarkuptext}%
\isacommand{lemma}\ hd{\isacharunderscore}Cons{\isacharunderscore}tl{\isacharbrackleft}simp{\isacharbrackright}{\isacharcolon}\ {\isachardoublequote}xs\ {\isasymnoteq}\ {\isacharbrackleft}{\isacharbrackright}\ \ {\isasymLongrightarrow}\ \ hd\ xs\ {\isacharhash}\ tl\ xs\ {\isacharequal}\ xs{\isachardoublequote}\isanewline
\isacommand{by}{\isacharparenleft}case{\isacharunderscore}tac\ xs{\isacharcomma}\ simp{\isacharcomma}\ simp{\isacharparenright}%
\begin{isamarkuptext}%
\noindent
Note the use of ``\ttindexboldpos{,}{$Isar}'' to string together a
sequence of methods. Assuming that the simplification rule
\isa{{\isacharparenleft}rev\ \mbox{xs}\ {\isacharequal}\ {\isacharbrackleft}{\isacharbrackright}{\isacharparenright}\ {\isacharequal}\ {\isacharparenleft}\mbox{xs}\ {\isacharequal}\ {\isacharbrackleft}{\isacharbrackright}{\isacharparenright}}
is present as well,%
\end{isamarkuptext}%
\isacommand{lemma}\ {\isachardoublequote}xs\ {\isasymnoteq}\ {\isacharbrackleft}{\isacharbrackright}\ {\isasymLongrightarrow}\ hd{\isacharparenleft}rev\ xs{\isacharparenright}\ {\isacharhash}\ tl{\isacharparenleft}rev\ xs{\isacharparenright}\ {\isacharequal}\ rev\ xs{\isachardoublequote}%
\begin{isamarkuptext}%
\noindent
is proved by plain simplification:
the conditional equation \isa{hd_Cons_tl} above
can simplify \isa{hd(rev~xs)~\#~tl(rev~xs)} to \isa{rev xs}
because the corresponding precondition \isa{rev xs \isasymnoteq\ []}
simplifies to \isa{xs \isasymnoteq\ []}, which is exactly the local
assumption of the subgoal.%
\end{isamarkuptext}%
\end{isabelle}%
%%% Local Variables:
%%% mode: latex
%%% TeX-master: "root"
%%% End:
