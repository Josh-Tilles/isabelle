%
\begin{isabellebody}%
\def\isabellecontext{case_exprs}%
%
\isamarkupsubsection{Case expressions}
%
\begin{isamarkuptext}%
\label{sec:case-expressions}
HOL also features \isaindexbold{case}-expressions for analyzing
elements of a datatype. For example,
\begin{isabelle}%
\ \ \ \ \ case\ xs\ of\ {\isacharbrackleft}{\isacharbrackright}\ {\isasymRightarrow}\ \isadigit{1}\ {\isacharbar}\ y\ {\isacharhash}\ ys\ {\isasymRightarrow}\ y%
\end{isabelle}
evaluates to \isa{\isadigit{1}} if \isa{xs} is \isa{{\isacharbrackleft}{\isacharbrackright}} and to \isa{y} if 
\isa{xs} is \isa{y\ {\isacharhash}\ ys}. (Since the result in both branches must be of
the same type, it follows that \isa{y} is of type \isa{nat} and hence
that \isa{xs} is of type \isa{nat\ list}.)

In general, if $e$ is a term of the datatype $t$ defined in
\S\ref{sec:general-datatype} above, the corresponding
\isa{case}-expression analyzing $e$ is
\[
\begin{array}{rrcl}
\isa{case}~e~\isa{of} & C@1~x@ {11}~\dots~x@ {1k@1} & \To & e@1 \\
                           \vdots \\
                           \mid & C@m~x@ {m1}~\dots~x@ {mk@m} & \To & e@m
\end{array}
\]

\begin{warn}
\emph{All} constructors must be present, their order is fixed, and nested
patterns are not supported.  Violating these restrictions results in strange
error messages.
\end{warn}
\noindent
Nested patterns can be simulated by nested \isa{case}-expressions: instead
of
\begin{isabelle}%
\ \ \ \ \ case\ xs\ of\ {\isacharbrackleft}{\isacharbrackright}\ {\isacharequal}{\isachargreater}\ \isadigit{1}\ {\isacharbar}\ {\isacharbrackleft}x{\isacharbrackright}\ {\isacharequal}{\isachargreater}\ x\ {\isacharbar}\ x\ {\isacharhash}\ {\isacharparenleft}y\ {\isacharhash}\ zs{\isacharparenright}\ {\isacharequal}{\isachargreater}\ y%
\end{isabelle}
write
\begin{isabelle}%
\ \ \ \ \ case\ xs\ of\ {\isacharbrackleft}{\isacharbrackright}\ {\isasymRightarrow}\ \isadigit{1}\isanewline
\ \ \ \ \ {\isacharbar}\ x\ {\isacharhash}\ ys\ {\isasymRightarrow}\ case\ ys\ of\ {\isacharbrackleft}{\isacharbrackright}\ {\isasymRightarrow}\ x\ {\isacharbar}\ y\ {\isacharhash}\ zs\ {\isasymRightarrow}\ y%
\end{isabelle}

Note that \isa{case}-expressions may need to be enclosed in parentheses to
indicate their scope%
\end{isamarkuptext}%
%
\isamarkupsubsection{Structural induction and case distinction}
%
\begin{isamarkuptext}%
\indexbold{structural induction}
\indexbold{induction!structural}
\indexbold{case distinction}
Almost all the basic laws about a datatype are applied automatically during
simplification. Only induction is invoked by hand via \isaindex{induct_tac},
which works for any datatype. In some cases, induction is overkill and a case
distinction over all constructors of the datatype suffices. This is performed
by \isaindexbold{case_tac}. A trivial example:%
\end{isamarkuptext}%
\isacommand{lemma}\ {\isachardoublequote}{\isacharparenleft}case\ xs\ of\ {\isacharbrackleft}{\isacharbrackright}\ {\isasymRightarrow}\ {\isacharbrackleft}{\isacharbrackright}\ {\isacharbar}\ y{\isacharhash}ys\ {\isasymRightarrow}\ xs{\isacharparenright}\ {\isacharequal}\ xs{\isachardoublequote}\isanewline
\isacommand{apply}{\isacharparenleft}case{\isacharunderscore}tac\ xs{\isacharparenright}%
\begin{isamarkuptxt}%
\noindent
results in the proof state
\begin{isabelle}
~1.~xs~=~[]~{\isasymLongrightarrow}~(case~xs~of~[]~{\isasymRightarrow}~[]~|~y~\#~ys~{\isasymRightarrow}~xs)~=~xs\isanewline
~2.~{\isasymAnd}a~list.~xs=a\#list~{\isasymLongrightarrow}~(case~xs~of~[]~{\isasymRightarrow}~[]~|~y\#ys~{\isasymRightarrow}~xs)~=~xs%
\end{isabelle}
which is solved automatically:%
\end{isamarkuptxt}%
\isacommand{apply}{\isacharparenleft}auto{\isacharparenright}%
\begin{isamarkuptext}%
Note that we do not need to give a lemma a name if we do not intend to refer
to it explicitly in the future.%
\end{isamarkuptext}%
\end{isabellebody}%
%%% Local Variables:
%%% mode: latex
%%% TeX-master: "root"
%%% End:
