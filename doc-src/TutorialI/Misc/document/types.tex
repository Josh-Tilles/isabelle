%
\begin{isabellebody}%
\def\isabellecontext{types}%
%
\isadelimtheory
%
\endisadelimtheory
%
\isatagtheory
%
\endisatagtheory
{\isafoldtheory}%
%
\isadelimtheory
%
\endisadelimtheory
\isacommand{types}\isamarkupfalse%
\ number\ \ \ \ \ \ \ {\isacharequal}\ nat\isanewline
\ \ \ \ \ \ gate\ \ \ \ \ \ \ \ \ {\isacharequal}\ {\isachardoublequoteopen}bool\ {\isasymRightarrow}\ bool\ {\isasymRightarrow}\ bool{\isachardoublequoteclose}\isanewline
\ \ \ \ \ \ {\isacharparenleft}{\isacharprime}a{\isacharcomma}{\isacharprime}b{\isacharparenright}alist\ {\isacharequal}\ {\isachardoublequoteopen}{\isacharparenleft}{\isacharprime}a\ {\isasymtimes}\ {\isacharprime}b{\isacharparenright}list{\isachardoublequoteclose}%
\begin{isamarkuptext}%
\noindent
Internally all synonyms are fully expanded.  As a consequence Isabelle's
output never contains synonyms.  Their main purpose is to improve the
readability of theories.  Synonyms can be used just like any other
type.%
\end{isamarkuptext}%
\isamarkuptrue%
%
\isamarkupsubsection{Constant Definitions%
}
\isamarkuptrue%
%
\begin{isamarkuptext}%
\label{sec:ConstDefinitions}\indexbold{definitions}%
Nonrecursive definitions can be made with the \commdx{definition}
command, for example \isa{nand} and \isa{xor} gates
(based on type \isa{gate} above):%
\end{isamarkuptext}%
\isamarkuptrue%
\isacommand{definition}\isamarkupfalse%
\ nand\ {\isacharcolon}{\isacharcolon}\ gate\ \isakeyword{where}\ {\isachardoublequoteopen}nand\ A\ B\ {\isasymequiv}\ {\isasymnot}{\isacharparenleft}A\ {\isasymand}\ B{\isacharparenright}{\isachardoublequoteclose}\isanewline
\isacommand{definition}\isamarkupfalse%
\ xor\ \ {\isacharcolon}{\isacharcolon}\ gate\ \isakeyword{where}\ {\isachardoublequoteopen}xor\ \ A\ B\ {\isasymequiv}\ A\ {\isasymand}\ {\isasymnot}B\ {\isasymor}\ {\isasymnot}A\ {\isasymand}\ B{\isachardoublequoteclose}%
\begin{isamarkuptext}%
\noindent%
The symbol \indexboldpos{\isasymequiv}{$IsaEq} is a special form of equality
that must be used in constant definitions.
Pattern-matching is not allowed: each definition must be of
the form $f\,x@1\,\dots\,x@n~\isasymequiv~t$.
Section~\ref{sec:Simp-with-Defs} explains how definitions are used
in proofs. The default name of each definition is $f$\isa{{\isacharunderscore}def}, where
$f$ is the name of the defined constant.%
\end{isamarkuptext}%
\isamarkuptrue%
%
\isadelimtheory
%
\endisadelimtheory
%
\isatagtheory
%
\endisatagtheory
{\isafoldtheory}%
%
\isadelimtheory
%
\endisadelimtheory
\end{isabellebody}%
%%% Local Variables:
%%% mode: latex
%%% TeX-master: "root"
%%% End:
