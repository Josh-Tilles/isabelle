%
\begin{isabellebody}%
\def\isabellecontext{types}%
%
\isadelimtheory
%
\endisadelimtheory
%
\isatagtheory
%
\endisatagtheory
{\isafoldtheory}%
%
\isadelimtheory
%
\endisadelimtheory
\isacommand{types}\isamarkupfalse%
\ number\ \ \ \ \ \ \ {\isaliteral{3D}{\isacharequal}}\ nat\isanewline
\ \ \ \ \ \ gate\ \ \ \ \ \ \ \ \ {\isaliteral{3D}{\isacharequal}}\ {\isaliteral{22}{\isachardoublequoteopen}}bool\ {\isaliteral{5C3C52696768746172726F773E}{\isasymRightarrow}}\ bool\ {\isaliteral{5C3C52696768746172726F773E}{\isasymRightarrow}}\ bool{\isaliteral{22}{\isachardoublequoteclose}}\isanewline
\ \ \ \ \ \ {\isaliteral{28}{\isacharparenleft}}{\isaliteral{27}{\isacharprime}}a{\isaliteral{2C}{\isacharcomma}}{\isaliteral{27}{\isacharprime}}b{\isaliteral{29}{\isacharparenright}}alist\ {\isaliteral{3D}{\isacharequal}}\ {\isaliteral{22}{\isachardoublequoteopen}}{\isaliteral{28}{\isacharparenleft}}{\isaliteral{27}{\isacharprime}}a\ {\isaliteral{5C3C74696D65733E}{\isasymtimes}}\ {\isaliteral{27}{\isacharprime}}b{\isaliteral{29}{\isacharparenright}}list{\isaliteral{22}{\isachardoublequoteclose}}%
\begin{isamarkuptext}%
\noindent
Internally all synonyms are fully expanded.  As a consequence Isabelle's
output never contains synonyms.  Their main purpose is to improve the
readability of theories.  Synonyms can be used just like any other
type.%
\end{isamarkuptext}%
\isamarkuptrue%
%
\isamarkupsubsection{Constant Definitions%
}
\isamarkuptrue%
%
\begin{isamarkuptext}%
\label{sec:ConstDefinitions}\indexbold{definitions}%
Nonrecursive definitions can be made with the \commdx{definition}
command, for example \isa{nand} and \isa{xor} gates
(based on type \isa{gate} above):%
\end{isamarkuptext}%
\isamarkuptrue%
\isacommand{definition}\isamarkupfalse%
\ nand\ {\isaliteral{3A}{\isacharcolon}}{\isaliteral{3A}{\isacharcolon}}\ gate\ \isakeyword{where}\ {\isaliteral{22}{\isachardoublequoteopen}}nand\ A\ B\ {\isaliteral{5C3C65717569763E}{\isasymequiv}}\ {\isaliteral{5C3C6E6F743E}{\isasymnot}}{\isaliteral{28}{\isacharparenleft}}A\ {\isaliteral{5C3C616E643E}{\isasymand}}\ B{\isaliteral{29}{\isacharparenright}}{\isaliteral{22}{\isachardoublequoteclose}}\isanewline
\isacommand{definition}\isamarkupfalse%
\ xor\ \ {\isaliteral{3A}{\isacharcolon}}{\isaliteral{3A}{\isacharcolon}}\ gate\ \isakeyword{where}\ {\isaliteral{22}{\isachardoublequoteopen}}xor\ \ A\ B\ {\isaliteral{5C3C65717569763E}{\isasymequiv}}\ A\ {\isaliteral{5C3C616E643E}{\isasymand}}\ {\isaliteral{5C3C6E6F743E}{\isasymnot}}B\ {\isaliteral{5C3C6F723E}{\isasymor}}\ {\isaliteral{5C3C6E6F743E}{\isasymnot}}A\ {\isaliteral{5C3C616E643E}{\isasymand}}\ B{\isaliteral{22}{\isachardoublequoteclose}}%
\begin{isamarkuptext}%
\noindent%
The symbol \indexboldpos{\isasymequiv}{$IsaEq} is a special form of equality
that must be used in constant definitions.
Pattern-matching is not allowed: each definition must be of
the form $f\,x@1\,\dots\,x@n~\isasymequiv~t$.
Section~\ref{sec:Simp-with-Defs} explains how definitions are used
in proofs. The default name of each definition is $f$\isa{{\isaliteral{5F}{\isacharunderscore}}def}, where
$f$ is the name of the defined constant.%
\end{isamarkuptext}%
\isamarkuptrue%
%
\isadelimtheory
%
\endisadelimtheory
%
\isatagtheory
%
\endisatagtheory
{\isafoldtheory}%
%
\isadelimtheory
%
\endisadelimtheory
\end{isabellebody}%
%%% Local Variables:
%%% mode: latex
%%% TeX-master: "root"
%%% End:
