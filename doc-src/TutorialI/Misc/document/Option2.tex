%
\begin{isabellebody}%
\def\isabellecontext{Option{\isadigit{2}}}%
%
\begin{isamarkuptext}%
\indexbold{*option}\indexbold{*None}\indexbold{*Some}
Our final datatype is very simple but still eminently useful:%
\end{isamarkuptext}%
\isacommand{datatype}\ {\isacharprime}a\ option\ {\isacharequal}\ None\ {\isacharbar}\ Some\ {\isacharprime}a%
\begin{isamarkuptext}%
\noindent
Frequently one needs to add a distinguished element to some existing type.
For example, type \isa{t\ option} can model the result of a computation that
may either terminate with an error (represented by \isa{None}) or return
some value \isa{v} (represented by \isa{Some\ v}).
Similarly, \isa{nat} extended with $\infty$ can be modeled by type
\isa{nat\ option}. In both cases one could define a new datatype with
customized constructors like \isa{Error} and \isa{Infinity},
but it is often simpler to use \isa{option}. For an application see
\S\ref{sec:Trie}.%
\end{isamarkuptext}%
\end{isabellebody}%
%%% Local Variables:
%%% mode: latex
%%% TeX-master: "root"
%%% End:
