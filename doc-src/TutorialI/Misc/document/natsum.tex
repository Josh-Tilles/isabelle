\begin{isabelle}%
%
\begin{isamarkuptext}%
\noindent
In particular, there are \isa{case}-expressions, for example%
\end{isamarkuptext}%
~{"}case~n~of~0~{\isasymRightarrow}~0~|~Suc~m~{\isasymRightarrow}~m{"}%
\begin{isamarkuptext}%
\noindent
primitive recursion, for example%
\end{isamarkuptext}%
\isacommand{consts}~sum~::~{"}nat~{\isasymRightarrow}~nat{"}\isanewline
\isacommand{primrec}~{"}sum~0~=~0{"}\isanewline
~~~~~~~~{"}sum~(Suc~n)~=~Suc~n~+~sum~n{"}%
\begin{isamarkuptext}%
\noindent
and induction, for example%
\end{isamarkuptext}%
\isacommand{lemma}~{"}sum~n~+~sum~n~=~n*(Suc~n){"}\isanewline
\isacommand{apply}(induct\_tac~n)\isanewline
\isacommand{apply}(auto)\isacommand{.}\isanewline
\end{isabelle}%
