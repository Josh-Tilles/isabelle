%
\begin{isabellebody}%
\def\isabellecontext{Fundata}%
\isamarkupfalse%
\isacommand{datatype}\ {\isacharparenleft}{\isacharprime}a{\isacharcomma}{\isacharprime}i{\isacharparenright}bigtree\ {\isacharequal}\ Tip\ {\isacharbar}\ Br\ {\isacharprime}a\ {\isachardoublequote}{\isacharprime}i\ {\isasymRightarrow}\ {\isacharparenleft}{\isacharprime}a{\isacharcomma}{\isacharprime}i{\isacharparenright}bigtree{\isachardoublequote}\isamarkupfalse%
%
\begin{isamarkuptext}%
\noindent
Parameter \isa{{\isacharprime}a} is the type of values stored in
the \isa{Br}anches of the tree, whereas \isa{{\isacharprime}i} is the index
type over which the tree branches. If \isa{{\isacharprime}i} is instantiated to
\isa{bool}, the result is a binary tree; if it is instantiated to
\isa{nat}, we have an infinitely branching tree because each node
has as many subtrees as there are natural numbers. How can we possibly
write down such a tree? Using functional notation! For example, the term
\begin{isabelle}%
\ \ \ \ \ Br\ {\isadigit{0}}\ {\isacharparenleft}{\isasymlambda}i{\isachardot}\ Br\ i\ {\isacharparenleft}{\isasymlambda}n{\isachardot}\ Tip{\isacharparenright}{\isacharparenright}%
\end{isabelle}
of type \isa{{\isacharparenleft}nat{\isacharcomma}\ nat{\isacharparenright}\ bigtree} is the tree whose
root is labeled with 0 and whose $i$th subtree is labeled with $i$ and
has merely \isa{Tip}s as further subtrees.

Function \isa{map{\isacharunderscore}bt} applies a function to all labels in a \isa{bigtree}:%
\end{isamarkuptext}%
\isamarkuptrue%
\isacommand{consts}\ map{\isacharunderscore}bt\ {\isacharcolon}{\isacharcolon}\ {\isachardoublequote}{\isacharparenleft}{\isacharprime}a\ {\isasymRightarrow}\ {\isacharprime}b{\isacharparenright}\ {\isasymRightarrow}\ {\isacharparenleft}{\isacharprime}a{\isacharcomma}{\isacharprime}i{\isacharparenright}bigtree\ {\isasymRightarrow}\ {\isacharparenleft}{\isacharprime}b{\isacharcomma}{\isacharprime}i{\isacharparenright}bigtree{\isachardoublequote}\isanewline
\isamarkupfalse%
\isacommand{primrec}\isanewline
{\isachardoublequote}map{\isacharunderscore}bt\ f\ Tip\ \ \ \ \ \ {\isacharequal}\ Tip{\isachardoublequote}\isanewline
{\isachardoublequote}map{\isacharunderscore}bt\ f\ {\isacharparenleft}Br\ a\ F{\isacharparenright}\ {\isacharequal}\ Br\ {\isacharparenleft}f\ a{\isacharparenright}\ {\isacharparenleft}{\isasymlambda}i{\isachardot}\ map{\isacharunderscore}bt\ f\ {\isacharparenleft}F\ i{\isacharparenright}{\isacharparenright}{\isachardoublequote}\isamarkupfalse%
%
\begin{isamarkuptext}%
\noindent This is a valid \isacommand{primrec} definition because the
recursive calls of \isa{map{\isacharunderscore}bt} involve only subtrees of
\isa{F}, which is itself a subterm of the left-hand side. Thus termination
is assured.  The seasoned functional programmer might try expressing
\isa{{\isasymlambda}i{\isachardot}\ map{\isacharunderscore}bt\ f\ {\isacharparenleft}F\ i{\isacharparenright}} as \isa{map{\isacharunderscore}bt\ f\ {\isasymcirc}\ F}, which Isabelle 
however will reject.  Applying \isa{map{\isacharunderscore}bt} to only one of its arguments
makes the termination proof less obvious.

The following lemma has a simple proof by induction:%
\end{isamarkuptext}%
\isamarkuptrue%
\isacommand{lemma}\ {\isachardoublequote}map{\isacharunderscore}bt\ {\isacharparenleft}g\ o\ f{\isacharparenright}\ T\ {\isacharequal}\ map{\isacharunderscore}bt\ g\ {\isacharparenleft}map{\isacharunderscore}bt\ f\ T{\isacharparenright}{\isachardoublequote}\isanewline
\isamarkupfalse%
\isacommand{apply}{\isacharparenleft}induct{\isacharunderscore}tac\ T{\isacharcomma}\ simp{\isacharunderscore}all{\isacharparenright}\isanewline
\isamarkupfalse%
\isacommand{done}\isamarkupfalse%
\isamarkupfalse%
\isamarkupfalse%
%
\begin{isamarkuptxt}%
\noindent
Because of the function type, the proof state after induction looks unusual.
Notice the quantified induction hypothesis:
\begin{isabelle}%
\ {\isadigit{1}}{\isachardot}\ map{\isacharunderscore}bt\ {\isacharparenleft}g\ {\isasymcirc}\ f{\isacharparenright}\ Tip\ {\isacharequal}\ map{\isacharunderscore}bt\ g\ {\isacharparenleft}map{\isacharunderscore}bt\ f\ Tip{\isacharparenright}\isanewline
\ {\isadigit{2}}{\isachardot}\ {\isasymAnd}a\ F{\isachardot}\ {\isacharparenleft}{\isasymAnd}x{\isachardot}\ map{\isacharunderscore}bt\ {\isacharparenleft}g\ {\isasymcirc}\ f{\isacharparenright}\ {\isacharparenleft}F\ x{\isacharparenright}\ {\isacharequal}\ map{\isacharunderscore}bt\ g\ {\isacharparenleft}map{\isacharunderscore}bt\ f\ {\isacharparenleft}F\ x{\isacharparenright}{\isacharparenright}{\isacharparenright}\ {\isasymLongrightarrow}\isanewline
\isaindent{\ {\isadigit{2}}{\isachardot}\ {\isasymAnd}a\ F{\isachardot}\ }map{\isacharunderscore}bt\ {\isacharparenleft}g\ {\isasymcirc}\ f{\isacharparenright}\ {\isacharparenleft}Br\ a\ F{\isacharparenright}\ {\isacharequal}\ map{\isacharunderscore}bt\ g\ {\isacharparenleft}map{\isacharunderscore}bt\ f\ {\isacharparenleft}Br\ a\ F{\isacharparenright}{\isacharparenright}%
\end{isabelle}%
\end{isamarkuptxt}%
\isamarkuptrue%
\isamarkupfalse%
\isamarkupfalse%
\end{isabellebody}%
%%% Local Variables:
%%% mode: latex
%%% TeX-master: "root"
%%% End:
