%
\begin{isabellebody}%
\def\isabellecontext{Fundata}%
\isacommand{datatype}\ {\isacharparenleft}{\isacharprime}a{\isacharcomma}{\isacharprime}i{\isacharparenright}bigtree\ {\isacharequal}\ Tip\ {\isacharbar}\ Branch\ {\isacharprime}a\ {\isachardoublequote}{\isacharprime}i\ {\isasymRightarrow}\ {\isacharparenleft}{\isacharprime}a{\isacharcomma}{\isacharprime}i{\isacharparenright}bigtree{\isachardoublequote}%
\begin{isamarkuptext}%
\noindent
Parameter \isa{{\isacharprime}a} is the type of values stored in
the \isa{Branch}es of the tree, whereas \isa{{\isacharprime}i} is the index
type over which the tree branches. If \isa{{\isacharprime}i} is instantiated to
\isa{bool}, the result is a binary tree; if it is instantiated to
\isa{nat}, we have an infinitely branching tree because each node
has as many subtrees as there are natural numbers. How can we possibly
write down such a tree? Using functional notation! For example, the term
\begin{isabelle}%
\ \ \ \ \ Branch\ \isadigit{0}\ {\isacharparenleft}{\isasymlambda}i{\isachardot}\ Branch\ i\ {\isacharparenleft}{\isasymlambda}n{\isachardot}\ Tip{\isacharparenright}{\isacharparenright}%
\end{isabelle}
of type \isa{{\isacharparenleft}nat{\isacharcomma}\ nat{\isacharparenright}\ bigtree} is the tree whose
root is labeled with 0 and whose $i$th subtree is labeled with $i$ and
has merely \isa{Tip}s as further subtrees.

Function \isa{map{\isacharunderscore}bt} applies a function to all labels in a \isa{bigtree}:%
\end{isamarkuptext}%
\isacommand{consts}\ map{\isacharunderscore}bt\ {\isacharcolon}{\isacharcolon}\ {\isachardoublequote}{\isacharparenleft}{\isacharprime}a\ {\isasymRightarrow}\ {\isacharprime}b{\isacharparenright}\ {\isasymRightarrow}\ {\isacharparenleft}{\isacharprime}a{\isacharcomma}{\isacharprime}i{\isacharparenright}bigtree\ {\isasymRightarrow}\ {\isacharparenleft}{\isacharprime}b{\isacharcomma}{\isacharprime}i{\isacharparenright}bigtree{\isachardoublequote}\isanewline
\isacommand{primrec}\isanewline
{\isachardoublequote}map{\isacharunderscore}bt\ f\ Tip\ \ \ \ \ \ \ \ \ \ {\isacharequal}\ Tip{\isachardoublequote}\isanewline
{\isachardoublequote}map{\isacharunderscore}bt\ f\ {\isacharparenleft}Branch\ a\ F{\isacharparenright}\ {\isacharequal}\ Branch\ {\isacharparenleft}f\ a{\isacharparenright}\ {\isacharparenleft}{\isasymlambda}i{\isachardot}\ map{\isacharunderscore}bt\ f\ {\isacharparenleft}F\ i{\isacharparenright}{\isacharparenright}{\isachardoublequote}%
\begin{isamarkuptext}%
\noindent This is a valid \isacommand{primrec} definition because the
recursive calls of \isa{map{\isacharunderscore}bt} involve only subtrees obtained from
\isa{F}, i.e.\ the left-hand side. Thus termination is assured.  The
seasoned functional programmer might have written \isa{map{\isacharunderscore}bt\ f\ {\isasymcirc}\ F}
instead of \isa{{\isasymlambda}i{\isachardot}\ map{\isacharunderscore}bt\ f\ {\isacharparenleft}F\ i{\isacharparenright}}, but the former is not accepted by
Isabelle because the termination proof is not as obvious since
\isa{map{\isacharunderscore}bt} is only partially applied.

The following lemma has a canonical proof%
\end{isamarkuptext}%
\isacommand{lemma}\ {\isachardoublequote}map{\isacharunderscore}bt\ {\isacharparenleft}g\ o\ f{\isacharparenright}\ T\ {\isacharequal}\ map{\isacharunderscore}bt\ g\ {\isacharparenleft}map{\isacharunderscore}bt\ f\ T{\isacharparenright}{\isachardoublequote}\isanewline
\isacommand{by}{\isacharparenleft}induct{\isacharunderscore}tac\ {\isachardoublequote}T{\isachardoublequote}{\isacharcomma}\ simp{\isacharunderscore}all{\isacharparenright}%
\begin{isamarkuptext}%
\noindent
but it is worth taking a look at the proof state after the induction step
to understand what the presence of the function type entails:
\begin{isabelle}
~1.~map\_bt~g~(map\_bt~f~Tip)~=~map\_bt~(g~{\isasymcirc}~f)~Tip\isanewline
~2.~{\isasymAnd}a~F.\isanewline
~~~~~~{\isasymforall}x.~map\_bt~g~(map\_bt~f~(F~x))~=~map\_bt~(g~{\isasymcirc}~f)~(F~x)~{\isasymLongrightarrow}\isanewline
~~~~~~map\_bt~g~(map\_bt~f~(Branch~a~F))~=~map\_bt~(g~{\isasymcirc}~f)~(Branch~a~F)%
\end{isabelle}%
\end{isamarkuptext}%
\end{isabellebody}%
%%% Local Variables:
%%% mode: latex
%%% TeX-master: "root"
%%% End:
