%
\begin{isabellebody}%
\def\isabellecontext{CTLind}%
%
\isamarkupsubsection{CTL revisited%
}
%
\begin{isamarkuptext}%
\label{sec:CTL-revisited}
The purpose of this section is twofold: we want to demonstrate
some of the induction principles and heuristics discussed above and we want to
show how inductive definitions can simplify proofs.
In \S\ref{sec:CTL} we gave a fairly involved proof of the correctness of a
model checker for CTL. In particular the proof of the
\isa{infinity{\isacharunderscore}lemma} on the way to \isa{AF{\isacharunderscore}lemma{\isadigit{2}}} is not as
simple as one might intuitively expect, due to the \isa{SOME} operator
involved. Below we give a simpler proof of \isa{AF{\isacharunderscore}lemma{\isadigit{2}}}
based on an auxiliary inductive definition.

Let us call a (finite or infinite) path \emph{\isa{A}-avoiding} if it does
not touch any node in the set \isa{A}. Then \isa{AF{\isacharunderscore}lemma{\isadigit{2}}} says
that if no infinite path from some state \isa{s} is \isa{A}-avoiding,
then \isa{s\ {\isasymin}\ lfp\ {\isacharparenleft}af\ A{\isacharparenright}}. We prove this by inductively defining the set
\isa{Avoid\ s\ A} of states reachable from \isa{s} by a finite \isa{A}-avoiding path:
% Second proof of opposite direction, directly by well-founded induction
% on the initial segment of M that avoids A.%
\end{isamarkuptext}%
\isacommand{consts}\ Avoid\ {\isacharcolon}{\isacharcolon}\ {\isachardoublequote}state\ {\isasymRightarrow}\ state\ set\ {\isasymRightarrow}\ state\ set{\isachardoublequote}\isanewline
\isacommand{inductive}\ {\isachardoublequote}Avoid\ s\ A{\isachardoublequote}\isanewline
\isakeyword{intros}\ {\isachardoublequote}s\ {\isasymin}\ Avoid\ s\ A{\isachardoublequote}\isanewline
\ \ \ \ \ \ \ {\isachardoublequote}{\isasymlbrakk}\ t\ {\isasymin}\ Avoid\ s\ A{\isacharsemicolon}\ t\ {\isasymnotin}\ A{\isacharsemicolon}\ {\isacharparenleft}t{\isacharcomma}u{\isacharparenright}\ {\isasymin}\ M\ {\isasymrbrakk}\ {\isasymLongrightarrow}\ u\ {\isasymin}\ Avoid\ s\ A{\isachardoublequote}%
\begin{isamarkuptext}%
It is easy to see that for any infinite \isa{A}-avoiding path \isa{f}
with \isa{f\ {\isadigit{0}}\ {\isasymin}\ Avoid\ s\ A} there is an infinite \isa{A}-avoiding path
starting with \isa{s} because (by definition of \isa{Avoid}) there is a
finite \isa{A}-avoiding path from \isa{s} to \isa{f\ {\isadigit{0}}}.
The proof is by induction on \isa{f\ {\isadigit{0}}\ {\isasymin}\ Avoid\ s\ A}. However,
this requires the following
reformulation, as explained in \S\ref{sec:ind-var-in-prems} above;
the \isa{rule{\isacharunderscore}format} directive undoes the reformulation after the proof.%
\end{isamarkuptext}%
\isacommand{lemma}\ ex{\isacharunderscore}infinite{\isacharunderscore}path{\isacharbrackleft}rule{\isacharunderscore}format{\isacharbrackright}{\isacharcolon}\isanewline
\ \ {\isachardoublequote}t\ {\isasymin}\ Avoid\ s\ A\ \ {\isasymLongrightarrow}\isanewline
\ \ \ {\isasymforall}f{\isasymin}Paths\ t{\isachardot}\ {\isacharparenleft}{\isasymforall}i{\isachardot}\ f\ i\ {\isasymnotin}\ A{\isacharparenright}\ {\isasymlongrightarrow}\ {\isacharparenleft}{\isasymexists}p{\isasymin}Paths\ s{\isachardot}\ {\isasymforall}i{\isachardot}\ p\ i\ {\isasymnotin}\ A{\isacharparenright}{\isachardoublequote}\isanewline
\isacommand{apply}{\isacharparenleft}erule\ Avoid{\isachardot}induct{\isacharparenright}\isanewline
\ \isacommand{apply}{\isacharparenleft}blast{\isacharparenright}\isanewline
\isacommand{apply}{\isacharparenleft}clarify{\isacharparenright}\isanewline
\isacommand{apply}{\isacharparenleft}drule{\isacharunderscore}tac\ x\ {\isacharequal}\ {\isachardoublequote}{\isasymlambda}i{\isachardot}\ case\ i\ of\ {\isadigit{0}}\ {\isasymRightarrow}\ t\ {\isacharbar}\ Suc\ i\ {\isasymRightarrow}\ f\ i{\isachardoublequote}\ \isakeyword{in}\ bspec{\isacharparenright}\isanewline
\isacommand{apply}{\isacharparenleft}simp{\isacharunderscore}all\ add{\isacharcolon}Paths{\isacharunderscore}def\ split{\isacharcolon}nat{\isachardot}split{\isacharparenright}\isanewline
\isacommand{done}%
\begin{isamarkuptext}%
\noindent
The base case (\isa{t\ {\isacharequal}\ s}) is trivial (\isa{blast}).
In the induction step, we have an infinite \isa{A}-avoiding path \isa{f}
starting from \isa{u}, a successor of \isa{t}. Now we simply instantiate
the \isa{{\isasymforall}f{\isasymin}Paths\ t} in the induction hypothesis by the path starting with
\isa{t} and continuing with \isa{f}. That is what the above $\lambda$-term
expresses. That fact that this is a path starting with \isa{t} and that
the instantiated induction hypothesis implies the conclusion is shown by
simplification.

Now we come to the key lemma. It says that if \isa{t} can be reached by a
finite \isa{A}-avoiding path from \isa{s}, then \isa{t\ {\isasymin}\ lfp\ {\isacharparenleft}af\ A{\isacharparenright}},
provided there is no infinite \isa{A}-avoiding path starting from \isa{s}.%
\end{isamarkuptext}%
\isacommand{lemma}\ Avoid{\isacharunderscore}in{\isacharunderscore}lfp{\isacharbrackleft}rule{\isacharunderscore}format{\isacharparenleft}no{\isacharunderscore}asm{\isacharparenright}{\isacharbrackright}{\isacharcolon}\isanewline
\ \ {\isachardoublequote}{\isasymforall}p{\isasymin}Paths\ s{\isachardot}\ {\isasymexists}i{\isachardot}\ p\ i\ {\isasymin}\ A\ {\isasymLongrightarrow}\ t\ {\isasymin}\ Avoid\ s\ A\ {\isasymlongrightarrow}\ t\ {\isasymin}\ lfp{\isacharparenleft}af\ A{\isacharparenright}{\isachardoublequote}%
\begin{isamarkuptxt}%
\noindent
The trick is not to induct on \isa{t\ {\isasymin}\ Avoid\ s\ A}, as already the base
case would be a problem, but to proceed by well-founded induction \isa{t}. Hence \isa{t\ {\isasymin}\ Avoid\ s\ A} needs to be brought into the conclusion as
well, which the directive \isa{rule{\isacharunderscore}format} undoes at the end (see below).
But induction with respect to which well-founded relation? The restriction
of \isa{M} to \isa{Avoid\ s\ A}:
\begin{isabelle}%
\ \ \ \ \ {\isacharbraceleft}{\isacharparenleft}y{\isacharcomma}\ x{\isacharparenright}{\isachardot}\ {\isacharparenleft}x{\isacharcomma}\ y{\isacharparenright}\ {\isasymin}\ M\ {\isasymand}\ x\ {\isasymin}\ Avoid\ s\ A\ {\isasymand}\ y\ {\isasymin}\ Avoid\ s\ A\ {\isasymand}\ x\ {\isasymnotin}\ A{\isacharbraceright}%
\end{isabelle}
As we shall see in a moment, the absence of infinite \isa{A}-avoiding paths
starting from \isa{s} implies well-foundedness of this relation. For the
moment we assume this and proceed with the induction:%
\end{isamarkuptxt}%
\isacommand{apply}{\isacharparenleft}subgoal{\isacharunderscore}tac\isanewline
\ \ {\isachardoublequote}wf{\isacharbraceleft}{\isacharparenleft}y{\isacharcomma}x{\isacharparenright}{\isachardot}\ {\isacharparenleft}x{\isacharcomma}y{\isacharparenright}{\isasymin}M\ {\isasymand}\ x\ {\isasymin}\ Avoid\ s\ A\ {\isasymand}\ y\ {\isasymin}\ Avoid\ s\ A\ {\isasymand}\ x\ {\isasymnotin}\ A{\isacharbraceright}{\isachardoublequote}{\isacharparenright}\isanewline
\ \isacommand{apply}{\isacharparenleft}erule{\isacharunderscore}tac\ a\ {\isacharequal}\ t\ \isakeyword{in}\ wf{\isacharunderscore}induct{\isacharparenright}\isanewline
\ \isacommand{apply}{\isacharparenleft}clarsimp{\isacharparenright}%
\begin{isamarkuptxt}%
\noindent
Now can assume additionally (induction hypothesis) that if \isa{t\ {\isasymnotin}\ A}
then all successors of \isa{t} that are in \isa{Avoid\ s\ A} are in
\isa{lfp\ {\isacharparenleft}af\ A{\isacharparenright}}. To prove the actual goal we unfold \isa{lfp} once. Now
we have to prove that \isa{t} is in \isa{A} or all successors of \isa{t} are in \isa{lfp\ {\isacharparenleft}af\ A{\isacharparenright}}. If \isa{t} is not in \isa{A}, the second
\isa{Avoid}-rule implies that all successors of \isa{t} are in
\isa{Avoid\ s\ A} (because we also assume \isa{t\ {\isasymin}\ Avoid\ s\ A}), and
hence, by the induction hypothesis, all successors of \isa{t} are indeed in
\isa{lfp\ {\isacharparenleft}af\ A{\isacharparenright}}. Mechanically:%
\end{isamarkuptxt}%
\ \isacommand{apply}{\isacharparenleft}rule\ ssubst\ {\isacharbrackleft}OF\ lfp{\isacharunderscore}unfold{\isacharbrackleft}OF\ mono{\isacharunderscore}af{\isacharbrackright}{\isacharbrackright}{\isacharparenright}\isanewline
\ \isacommand{apply}{\isacharparenleft}simp\ only{\isacharcolon}\ af{\isacharunderscore}def{\isacharparenright}\isanewline
\ \isacommand{apply}{\isacharparenleft}blast\ intro{\isacharcolon}Avoid{\isachardot}intros{\isacharparenright}%
\begin{isamarkuptxt}%
Having proved the main goal we return to the proof obligation that the above
relation is indeed well-founded. This is proved by contraposition: we assume
the relation is not well-founded. Thus there exists an infinite \isa{A}-avoiding path all in \isa{Avoid\ s\ A}, by theorem
\isa{wf{\isacharunderscore}iff{\isacharunderscore}no{\isacharunderscore}infinite{\isacharunderscore}down{\isacharunderscore}chain}:
\begin{isabelle}%
\ \ \ \ \ wf\ r\ {\isacharequal}\ {\isacharparenleft}{\isasymnot}\ {\isacharparenleft}{\isasymexists}f{\isachardot}\ {\isasymforall}i{\isachardot}\ {\isacharparenleft}f\ {\isacharparenleft}Suc\ i{\isacharparenright}{\isacharcomma}\ f\ i{\isacharparenright}\ {\isasymin}\ r{\isacharparenright}{\isacharparenright}%
\end{isabelle}
From lemma \isa{ex{\isacharunderscore}infinite{\isacharunderscore}path} the existence of an infinite
\isa{A}-avoiding path starting in \isa{s} follows, just as required for
the contraposition.%
\end{isamarkuptxt}%
\isacommand{apply}{\isacharparenleft}erule\ contrapos{\isacharunderscore}pp{\isacharparenright}\isanewline
\isacommand{apply}{\isacharparenleft}simp\ add{\isacharcolon}wf{\isacharunderscore}iff{\isacharunderscore}no{\isacharunderscore}infinite{\isacharunderscore}down{\isacharunderscore}chain{\isacharparenright}\isanewline
\isacommand{apply}{\isacharparenleft}erule\ exE{\isacharparenright}\isanewline
\isacommand{apply}{\isacharparenleft}rule\ ex{\isacharunderscore}infinite{\isacharunderscore}path{\isacharparenright}\isanewline
\isacommand{apply}{\isacharparenleft}auto\ simp\ add{\isacharcolon}Paths{\isacharunderscore}def{\isacharparenright}\isanewline
\isacommand{done}%
\begin{isamarkuptext}%
The \isa{{\isacharparenleft}no{\isacharunderscore}asm{\isacharparenright}} modifier of the \isa{rule{\isacharunderscore}format} directive means
that the assumption is left unchanged---otherwise the \isa{{\isasymforall}p} is turned
into a \isa{{\isasymAnd}p}, which would complicate matters below. As it is,
\isa{Avoid{\isacharunderscore}in{\isacharunderscore}lfp} is now
\begin{isabelle}%
\ \ \ \ \ {\isasymlbrakk}{\isasymforall}p{\isasymin}Paths\ s{\isachardot}\ {\isasymexists}i{\isachardot}\ p\ i\ {\isasymin}\ A{\isacharsemicolon}\ t\ {\isasymin}\ Avoid\ s\ A{\isasymrbrakk}\ {\isasymLongrightarrow}\ t\ {\isasymin}\ lfp\ {\isacharparenleft}af\ A{\isacharparenright}%
\end{isabelle}
The main theorem is simply the corollary where \isa{t\ {\isacharequal}\ s},
in which case the assumption \isa{t\ {\isasymin}\ Avoid\ s\ A} is trivially true
by the first \isa{Avoid}-rule). Isabelle confirms this:%
\end{isamarkuptext}%
\isacommand{theorem}\ AF{\isacharunderscore}lemma{\isadigit{2}}{\isacharcolon}\isanewline
\ \ {\isachardoublequote}{\isacharbraceleft}s{\isachardot}\ {\isasymforall}p\ {\isasymin}\ Paths\ s{\isachardot}\ {\isasymexists}\ i{\isachardot}\ p\ i\ {\isasymin}\ A{\isacharbraceright}\ {\isasymsubseteq}\ lfp{\isacharparenleft}af\ A{\isacharparenright}{\isachardoublequote}\isanewline
\isacommand{by}{\isacharparenleft}auto\ elim{\isacharcolon}Avoid{\isacharunderscore}in{\isacharunderscore}lfp\ intro{\isacharcolon}Avoid{\isachardot}intros{\isacharparenright}\isanewline
\isanewline
\end{isabellebody}%
%%% Local Variables:
%%% mode: latex
%%% TeX-master: "root"
%%% End:
