%
\begin{isabellebody}%
\def\isabellecontext{PDL}%
%
\isamarkupsubsection{Propositional Dynamic Logic---PDL%
}
%
\begin{isamarkuptext}%
\index{PDL|(}
The formulae of PDL are built up from atomic propositions via the customary
propositional connectives of negation and conjunction and the two temporal
connectives \isa{AX} and \isa{EF}. Since formulae are essentially
syntax trees, they are naturally modelled as a datatype:%
\end{isamarkuptext}%
\isacommand{datatype}\ formula\ {\isacharequal}\ Atom\ atom\isanewline
\ \ \ \ \ \ \ \ \ \ \ \ \ \ \ \ \ \ {\isacharbar}\ Neg\ formula\isanewline
\ \ \ \ \ \ \ \ \ \ \ \ \ \ \ \ \ \ {\isacharbar}\ And\ formula\ formula\isanewline
\ \ \ \ \ \ \ \ \ \ \ \ \ \ \ \ \ \ {\isacharbar}\ AX\ formula\isanewline
\ \ \ \ \ \ \ \ \ \ \ \ \ \ \ \ \ \ {\isacharbar}\ EF\ formula%
\begin{isamarkuptext}%
\noindent
This is almost the same as in the boolean expression case study in
\S\ref{sec:boolex}.

The meaning of these formulae is given by saying which formula is true in
which state:%
\end{isamarkuptext}%
\isacommand{consts}\ valid\ {\isacharcolon}{\isacharcolon}\ {\isachardoublequote}state\ {\isasymRightarrow}\ formula\ {\isasymRightarrow}\ bool{\isachardoublequote}\ \ \ {\isacharparenleft}{\isachardoublequote}{\isacharparenleft}{\isacharunderscore}\ {\isasymTurnstile}\ {\isacharunderscore}{\isacharparenright}{\isachardoublequote}\ {\isacharbrackleft}{\isadigit{8}}{\isadigit{0}}{\isacharcomma}{\isadigit{8}}{\isadigit{0}}{\isacharbrackright}\ {\isadigit{8}}{\isadigit{0}}{\isacharparenright}%
\begin{isamarkuptext}%
\noindent
The syntax annotation allows us to write \isa{s\ {\isasymTurnstile}\ f} instead of
\hbox{\isa{valid\ s\ f}}.

\smallskip
The definition of \isa{{\isasymTurnstile}} is by recursion over the syntax:%
\end{isamarkuptext}%
\isacommand{primrec}\isanewline
{\isachardoublequote}s\ {\isasymTurnstile}\ Atom\ a\ \ {\isacharequal}\ {\isacharparenleft}a\ {\isasymin}\ L\ s{\isacharparenright}{\isachardoublequote}\isanewline
{\isachardoublequote}s\ {\isasymTurnstile}\ Neg\ f\ \ \ {\isacharequal}\ {\isacharparenleft}{\isasymnot}{\isacharparenleft}s\ {\isasymTurnstile}\ f{\isacharparenright}{\isacharparenright}{\isachardoublequote}\isanewline
{\isachardoublequote}s\ {\isasymTurnstile}\ And\ f\ g\ {\isacharequal}\ {\isacharparenleft}s\ {\isasymTurnstile}\ f\ {\isasymand}\ s\ {\isasymTurnstile}\ g{\isacharparenright}{\isachardoublequote}\isanewline
{\isachardoublequote}s\ {\isasymTurnstile}\ AX\ f\ \ \ \ {\isacharequal}\ {\isacharparenleft}{\isasymforall}t{\isachardot}\ {\isacharparenleft}s{\isacharcomma}t{\isacharparenright}\ {\isasymin}\ M\ {\isasymlongrightarrow}\ t\ {\isasymTurnstile}\ f{\isacharparenright}{\isachardoublequote}\isanewline
{\isachardoublequote}s\ {\isasymTurnstile}\ EF\ f\ \ \ \ {\isacharequal}\ {\isacharparenleft}{\isasymexists}t{\isachardot}\ {\isacharparenleft}s{\isacharcomma}t{\isacharparenright}\ {\isasymin}\ M\isactrlsup {\isacharasterisk}\ {\isasymand}\ t\ {\isasymTurnstile}\ f{\isacharparenright}{\isachardoublequote}%
\begin{isamarkuptext}%
\noindent
The first three equations should be self-explanatory. The temporal formula
\isa{AX\ f} means that \isa{f} is true in all next states whereas
\isa{EF\ f} means that there exists some future state in which \isa{f} is
true. The future is expressed via \isa{\isactrlsup {\isacharasterisk}}, the reflexive transitive
closure. Because of reflexivity, the future includes the present.

Now we come to the model checker itself. It maps a formula into the set of
states where the formula is true and is defined by recursion over the syntax,
too:%
\end{isamarkuptext}%
\isacommand{consts}\ mc\ {\isacharcolon}{\isacharcolon}\ {\isachardoublequote}formula\ {\isasymRightarrow}\ state\ set{\isachardoublequote}\isanewline
\isacommand{primrec}\isanewline
{\isachardoublequote}mc{\isacharparenleft}Atom\ a{\isacharparenright}\ \ {\isacharequal}\ {\isacharbraceleft}s{\isachardot}\ a\ {\isasymin}\ L\ s{\isacharbraceright}{\isachardoublequote}\isanewline
{\isachardoublequote}mc{\isacharparenleft}Neg\ f{\isacharparenright}\ \ \ {\isacharequal}\ {\isacharminus}mc\ f{\isachardoublequote}\isanewline
{\isachardoublequote}mc{\isacharparenleft}And\ f\ g{\isacharparenright}\ {\isacharequal}\ mc\ f\ {\isasyminter}\ mc\ g{\isachardoublequote}\isanewline
{\isachardoublequote}mc{\isacharparenleft}AX\ f{\isacharparenright}\ \ \ \ {\isacharequal}\ {\isacharbraceleft}s{\isachardot}\ {\isasymforall}t{\isachardot}\ {\isacharparenleft}s{\isacharcomma}t{\isacharparenright}\ {\isasymin}\ M\ \ {\isasymlongrightarrow}\ t\ {\isasymin}\ mc\ f{\isacharbraceright}{\isachardoublequote}\isanewline
{\isachardoublequote}mc{\isacharparenleft}EF\ f{\isacharparenright}\ \ \ \ {\isacharequal}\ lfp{\isacharparenleft}{\isasymlambda}T{\isachardot}\ mc\ f\ {\isasymunion}\ {\isacharparenleft}M{\isasyminverse}\ {\isacharbackquote}{\isacharbackquote}\ T{\isacharparenright}{\isacharparenright}{\isachardoublequote}%
\begin{isamarkuptext}%
\noindent
Only the equation for \isa{EF} deserves some comments. Remember that the
postfix \isa{{\isasyminverse}} and the infix \isa{{\isacharbackquote}{\isacharbackquote}} are predefined and denote the
converse of a relation and the image of a set under a relation.  Thus
\isa{M{\isasyminverse}\ {\isacharbackquote}{\isacharbackquote}\ T} is the set of all predecessors of \isa{T} and the least
fixed point (\isa{lfp}) of \isa{{\isasymlambda}T{\isachardot}\ mc\ f\ {\isasymunion}\ M{\isasyminverse}\ {\isacharbackquote}{\isacharbackquote}\ T} is the least set
\isa{T} containing \isa{mc\ f} and all predecessors of \isa{T}. If you
find it hard to see that \isa{mc\ {\isacharparenleft}EF\ f{\isacharparenright}} contains exactly those states from
which there is a path to a state where \isa{f} is true, do not worry---that
will be proved in a moment.

First we prove monotonicity of the function inside \isa{lfp}
in order to make sure it really has a least fixed point.%
\end{isamarkuptext}%
\isacommand{lemma}\ mono{\isacharunderscore}ef{\isacharcolon}\ {\isachardoublequote}mono{\isacharparenleft}{\isasymlambda}T{\isachardot}\ A\ {\isasymunion}\ {\isacharparenleft}M{\isasyminverse}\ {\isacharbackquote}{\isacharbackquote}\ T{\isacharparenright}{\isacharparenright}{\isachardoublequote}\isanewline
\isacommand{apply}{\isacharparenleft}rule\ monoI{\isacharparenright}\isanewline
\isacommand{apply}\ blast\isanewline
\isacommand{done}%
\begin{isamarkuptext}%
\noindent
Now we can relate model checking and semantics. For the \isa{EF} case we need
a separate lemma:%
\end{isamarkuptext}%
\isacommand{lemma}\ EF{\isacharunderscore}lemma{\isacharcolon}\isanewline
\ \ {\isachardoublequote}lfp{\isacharparenleft}{\isasymlambda}T{\isachardot}\ A\ {\isasymunion}\ {\isacharparenleft}M{\isasyminverse}\ {\isacharbackquote}{\isacharbackquote}\ T{\isacharparenright}{\isacharparenright}\ {\isacharequal}\ {\isacharbraceleft}s{\isachardot}\ {\isasymexists}t{\isachardot}\ {\isacharparenleft}s{\isacharcomma}t{\isacharparenright}\ {\isasymin}\ M\isactrlsup {\isacharasterisk}\ {\isasymand}\ t\ {\isasymin}\ A{\isacharbraceright}{\isachardoublequote}%
\begin{isamarkuptxt}%
\noindent
The equality is proved in the canonical fashion by proving that each set
includes the other; the inclusion is shown pointwise:%
\end{isamarkuptxt}%
\isacommand{apply}{\isacharparenleft}rule\ equalityI{\isacharparenright}\isanewline
\ \isacommand{apply}{\isacharparenleft}rule\ subsetI{\isacharparenright}\isanewline
\ \isacommand{apply}{\isacharparenleft}simp{\isacharparenright}%
\begin{isamarkuptxt}%
\noindent
Simplification leaves us with the following first subgoal
\begin{isabelle}%
\ {\isadigit{1}}{\isachardot}\ {\isasymAnd}s{\isachardot}\ s\ {\isasymin}\ lfp\ {\isacharparenleft}{\isasymlambda}T{\isachardot}\ A\ {\isasymunion}\ M{\isasyminverse}\ {\isacharbackquote}{\isacharbackquote}\ T{\isacharparenright}\ {\isasymLongrightarrow}\ {\isasymexists}t{\isachardot}\ {\isacharparenleft}s{\isacharcomma}\ t{\isacharparenright}\ {\isasymin}\ M\isactrlsup {\isacharasterisk}\ {\isasymand}\ t\ {\isasymin}\ A%
\end{isabelle}
which is proved by \isa{lfp}-induction:%
\end{isamarkuptxt}%
\ \isacommand{apply}{\isacharparenleft}erule\ lfp{\isacharunderscore}induct{\isacharparenright}\isanewline
\ \ \isacommand{apply}{\isacharparenleft}rule\ mono{\isacharunderscore}ef{\isacharparenright}\isanewline
\ \isacommand{apply}{\isacharparenleft}simp{\isacharparenright}%
\begin{isamarkuptxt}%
\noindent
Having disposed of the monotonicity subgoal,
simplification leaves us with the following main goal
\begin{isabelle}
\ {\isadigit{1}}{\isachardot}\ {\isasymAnd}x{\isachardot}\ x\ {\isasymin}\ A\ {\isasymor}\isanewline
\ \ \ \ \ \ \ \ \ x\ {\isasymin}\ M{\isasyminverse}\ {\isacharbackquote}{\isacharbackquote}\ {\isacharparenleft}lfp\ {\isacharparenleft}\dots{\isacharparenright}\ {\isasyminter}\ {\isacharbraceleft}x{\isachardot}\ {\isasymexists}t{\isachardot}\ {\isacharparenleft}x{\isacharcomma}\ t{\isacharparenright}\ {\isasymin}\ M\isactrlsup {\isacharasterisk}\ {\isasymand}\ t\ {\isasymin}\ A{\isacharbraceright}{\isacharparenright}\isanewline
\ \ \ \ \ \ \ \ {\isasymLongrightarrow}\ {\isasymexists}t{\isachardot}\ {\isacharparenleft}x{\isacharcomma}\ t{\isacharparenright}\ {\isasymin}\ M\isactrlsup {\isacharasterisk}\ {\isasymand}\ t\ {\isasymin}\ A
\end{isabelle}
which is proved by \isa{blast} with the help of transitivity of \isa{\isactrlsup {\isacharasterisk}}:%
\end{isamarkuptxt}%
\ \isacommand{apply}{\isacharparenleft}blast\ intro{\isacharcolon}\ rtrancl{\isacharunderscore}trans{\isacharparenright}%
\begin{isamarkuptxt}%
We now return to the second set inclusion subgoal, which is again proved
pointwise:%
\end{isamarkuptxt}%
\isacommand{apply}{\isacharparenleft}rule\ subsetI{\isacharparenright}\isanewline
\isacommand{apply}{\isacharparenleft}simp{\isacharcomma}\ clarify{\isacharparenright}%
\begin{isamarkuptxt}%
\noindent
After simplification and clarification we are left with
\begin{isabelle}%
\ {\isadigit{1}}{\isachardot}\ {\isasymAnd}x\ t{\isachardot}\ {\isasymlbrakk}{\isacharparenleft}x{\isacharcomma}\ t{\isacharparenright}\ {\isasymin}\ M\isactrlsup {\isacharasterisk}{\isacharsemicolon}\ t\ {\isasymin}\ A{\isasymrbrakk}\ {\isasymLongrightarrow}\ x\ {\isasymin}\ lfp\ {\isacharparenleft}{\isasymlambda}T{\isachardot}\ A\ {\isasymunion}\ M{\isasyminverse}\ {\isacharbackquote}{\isacharbackquote}\ T{\isacharparenright}%
\end{isabelle}
This goal is proved by induction on \isa{{\isacharparenleft}s{\isacharcomma}\ t{\isacharparenright}\ {\isasymin}\ M\isactrlsup {\isacharasterisk}}. But since the model
checker works backwards (from \isa{t} to \isa{s}), we cannot use the
induction theorem \isa{rtrancl{\isacharunderscore}induct} because it works in the
forward direction. Fortunately the converse induction theorem
\isa{converse{\isacharunderscore}rtrancl{\isacharunderscore}induct} already exists:
\begin{isabelle}%
\ \ \ \ \ {\isasymlbrakk}{\isacharparenleft}a{\isacharcomma}\ b{\isacharparenright}\ {\isasymin}\ r\isactrlsup {\isacharasterisk}{\isacharsemicolon}\ P\ b{\isacharsemicolon}\isanewline
\ \ \ \ \ \ \ \ {\isasymAnd}y\ z{\isachardot}\ {\isasymlbrakk}{\isacharparenleft}y{\isacharcomma}\ z{\isacharparenright}\ {\isasymin}\ r{\isacharsemicolon}\ {\isacharparenleft}z{\isacharcomma}\ b{\isacharparenright}\ {\isasymin}\ r\isactrlsup {\isacharasterisk}{\isacharsemicolon}\ P\ z{\isasymrbrakk}\ {\isasymLongrightarrow}\ P\ y{\isasymrbrakk}\isanewline
\ \ \ \ \ {\isasymLongrightarrow}\ P\ a%
\end{isabelle}
It says that if \isa{{\isacharparenleft}a{\isacharcomma}\ b{\isacharparenright}\ {\isasymin}\ r\isactrlsup {\isacharasterisk}} and we know \isa{P\ b} then we can infer
\isa{P\ a} provided each step backwards from a predecessor \isa{z} of
\isa{b} preserves \isa{P}.%
\end{isamarkuptxt}%
\isacommand{apply}{\isacharparenleft}erule\ converse{\isacharunderscore}rtrancl{\isacharunderscore}induct{\isacharparenright}%
\begin{isamarkuptxt}%
\noindent
The base case
\begin{isabelle}%
\ {\isadigit{1}}{\isachardot}\ {\isasymAnd}x\ t{\isachardot}\ t\ {\isasymin}\ A\ {\isasymLongrightarrow}\ t\ {\isasymin}\ lfp\ {\isacharparenleft}{\isasymlambda}T{\isachardot}\ A\ {\isasymunion}\ M{\isasyminverse}\ {\isacharbackquote}{\isacharbackquote}\ T{\isacharparenright}%
\end{isabelle}
is solved by unrolling \isa{lfp} once%
\end{isamarkuptxt}%
\ \isacommand{apply}{\isacharparenleft}rule\ ssubst{\isacharbrackleft}OF\ lfp{\isacharunderscore}unfold{\isacharbrackleft}OF\ mono{\isacharunderscore}ef{\isacharbrackright}{\isacharbrackright}{\isacharparenright}%
\begin{isamarkuptxt}%
\begin{isabelle}%
\ {\isadigit{1}}{\isachardot}\ {\isasymAnd}x\ t{\isachardot}\ t\ {\isasymin}\ A\ {\isasymLongrightarrow}\ t\ {\isasymin}\ A\ {\isasymunion}\ M{\isasyminverse}\ {\isacharbackquote}{\isacharbackquote}\ lfp\ {\isacharparenleft}{\isasymlambda}T{\isachardot}\ A\ {\isasymunion}\ M{\isasyminverse}\ {\isacharbackquote}{\isacharbackquote}\ T{\isacharparenright}%
\end{isabelle}
and disposing of the resulting trivial subgoal automatically:%
\end{isamarkuptxt}%
\ \isacommand{apply}{\isacharparenleft}blast{\isacharparenright}%
\begin{isamarkuptxt}%
\noindent
The proof of the induction step is identical to the one for the base case:%
\end{isamarkuptxt}%
\isacommand{apply}{\isacharparenleft}rule\ ssubst{\isacharbrackleft}OF\ lfp{\isacharunderscore}unfold{\isacharbrackleft}OF\ mono{\isacharunderscore}ef{\isacharbrackright}{\isacharbrackright}{\isacharparenright}\isanewline
\isacommand{apply}{\isacharparenleft}blast{\isacharparenright}\isanewline
\isacommand{done}%
\begin{isamarkuptext}%
The main theorem is proved in the familiar manner: induction followed by
\isa{auto} augmented with the lemma as a simplification rule.%
\end{isamarkuptext}%
\isacommand{theorem}\ {\isachardoublequote}mc\ f\ {\isacharequal}\ {\isacharbraceleft}s{\isachardot}\ s\ {\isasymTurnstile}\ f{\isacharbraceright}{\isachardoublequote}\isanewline
\isacommand{apply}{\isacharparenleft}induct{\isacharunderscore}tac\ f{\isacharparenright}\isanewline
\isacommand{apply}{\isacharparenleft}auto\ simp\ add{\isacharcolon}EF{\isacharunderscore}lemma{\isacharparenright}\isanewline
\isacommand{done}%
\begin{isamarkuptext}%
\begin{exercise}
\isa{AX} has a dual operator \isa{EN}\footnote{We cannot use the customary \isa{EX}
as that is the ASCII equivalent of \isa{{\isasymexists}}}
(``there exists a next state such that'') with the intended semantics
\begin{isabelle}%
\ \ \ \ \ s\ {\isasymTurnstile}\ EN\ f\ {\isacharequal}\ {\isacharparenleft}{\isasymexists}t{\isachardot}\ {\isacharparenleft}s{\isacharcomma}\ t{\isacharparenright}\ {\isasymin}\ M\ {\isasymand}\ t\ {\isasymTurnstile}\ f{\isacharparenright}%
\end{isabelle}
Fortunately, \isa{EN\ f} can already be expressed as a PDL formula. How?

Show that the semantics for \isa{EF} satisfies the following recursion equation:
\begin{isabelle}%
\ \ \ \ \ s\ {\isasymTurnstile}\ EF\ f\ {\isacharequal}\ {\isacharparenleft}s\ {\isasymTurnstile}\ f\ {\isasymor}\ s\ {\isasymTurnstile}\ EN\ {\isacharparenleft}EF\ f{\isacharparenright}{\isacharparenright}%
\end{isabelle}
\end{exercise}
\index{PDL|)}%
\end{isamarkuptext}%
\end{isabellebody}%
%%% Local Variables:
%%% mode: latex
%%% TeX-master: "root"
%%% End:
