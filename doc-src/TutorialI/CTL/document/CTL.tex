%
\begin{isabellebody}%
\def\isabellecontext{CTL}%
%
\isamarkupsubsection{Computation tree logic---CTL%
}
%
\begin{isamarkuptext}%
\label{sec:CTL}
The semantics of PDL only needs transitive reflexive closure.
Let us now be a bit more adventurous and introduce a new temporal operator
that goes beyond transitive reflexive closure. We extend the datatype
\isa{formula} by a new constructor%
\end{isamarkuptext}%
\ \ \ \ \ \ \ \ \ \ \ \ \ \ \ \ \ \ {\isacharbar}\ AF\ formula%
\begin{isamarkuptext}%
\noindent
which stands for "always in the future":
on all paths, at some point the formula holds. Formalizing the notion of an infinite path is easy
in HOL: it is simply a function from \isa{nat} to \isa{state}.%
\end{isamarkuptext}%
\isacommand{constdefs}\ Paths\ {\isacharcolon}{\isacharcolon}\ {\isachardoublequote}state\ {\isasymRightarrow}\ {\isacharparenleft}nat\ {\isasymRightarrow}\ state{\isacharparenright}set{\isachardoublequote}\isanewline
\ \ \ \ \ \ \ \ \ {\isachardoublequote}Paths\ s\ {\isasymequiv}\ {\isacharbraceleft}p{\isachardot}\ s\ {\isacharequal}\ p\ {\isadigit{0}}\ {\isasymand}\ {\isacharparenleft}{\isasymforall}i{\isachardot}\ {\isacharparenleft}p\ i{\isacharcomma}\ p{\isacharparenleft}i{\isacharplus}{\isadigit{1}}{\isacharparenright}{\isacharparenright}\ {\isasymin}\ M{\isacharparenright}{\isacharbraceright}{\isachardoublequote}%
\begin{isamarkuptext}%
\noindent
This definition allows a very succinct statement of the semantics of \isa{AF}:
\footnote{Do not be mislead: neither datatypes nor recursive functions can be
extended by new constructors or equations. This is just a trick of the
presentation. In reality one has to define a new datatype and a new function.}%
\end{isamarkuptext}%
{\isachardoublequote}s\ {\isasymTurnstile}\ AF\ f\ \ \ \ {\isacharequal}\ {\isacharparenleft}{\isasymforall}p\ {\isasymin}\ Paths\ s{\isachardot}\ {\isasymexists}i{\isachardot}\ p\ i\ {\isasymTurnstile}\ f{\isacharparenright}{\isachardoublequote}%
\begin{isamarkuptext}%
\noindent
Model checking \isa{AF} involves a function which
is just complicated enough to warrant a separate definition:%
\end{isamarkuptext}%
\isacommand{constdefs}\ af\ {\isacharcolon}{\isacharcolon}\ {\isachardoublequote}state\ set\ {\isasymRightarrow}\ state\ set\ {\isasymRightarrow}\ state\ set{\isachardoublequote}\isanewline
\ \ \ \ \ \ \ \ \ {\isachardoublequote}af\ A\ T\ {\isasymequiv}\ A\ {\isasymunion}\ {\isacharbraceleft}s{\isachardot}\ {\isasymforall}t{\isachardot}\ {\isacharparenleft}s{\isacharcomma}\ t{\isacharparenright}\ {\isasymin}\ M\ {\isasymlongrightarrow}\ t\ {\isasymin}\ T{\isacharbraceright}{\isachardoublequote}%
\begin{isamarkuptext}%
\noindent
Now we define \isa{mc\ {\isacharparenleft}AF\ f{\isacharparenright}} as the least set \isa{T} that contains
\isa{mc\ f} and all states all of whose direct successors are in \isa{T}:%
\end{isamarkuptext}%
{\isachardoublequote}mc{\isacharparenleft}AF\ f{\isacharparenright}\ \ \ \ {\isacharequal}\ lfp{\isacharparenleft}af{\isacharparenleft}mc\ f{\isacharparenright}{\isacharparenright}{\isachardoublequote}%
\begin{isamarkuptext}%
\noindent
Because \isa{af} is monotone in its second argument (and also its first, but
that is irrelevant) \isa{af\ A} has a least fixed point:%
\end{isamarkuptext}%
\isacommand{lemma}\ mono{\isacharunderscore}af{\isacharcolon}\ {\isachardoublequote}mono{\isacharparenleft}af\ A{\isacharparenright}{\isachardoublequote}\isanewline
\isacommand{apply}{\isacharparenleft}simp\ add{\isacharcolon}\ mono{\isacharunderscore}def\ af{\isacharunderscore}def{\isacharparenright}\isanewline
\isacommand{apply}\ blast\isanewline
\isacommand{done}%
\begin{isamarkuptext}%
All we need to prove now is that \isa{mc} and \isa{{\isasymTurnstile}}
agree for \isa{AF}, i.e.\ that \isa{mc\ {\isacharparenleft}AF\ f{\isacharparenright}\ {\isacharequal}\ {\isacharbraceleft}s{\isachardot}\ s\ {\isasymTurnstile}\ AF\ f{\isacharbraceright}}. This time we prove the two containments separately, starting
with the easy one:%
\end{isamarkuptext}%
\isacommand{theorem}\ AF{\isacharunderscore}lemma{\isadigit{1}}{\isacharcolon}\isanewline
\ \ {\isachardoublequote}lfp{\isacharparenleft}af\ A{\isacharparenright}\ {\isasymsubseteq}\ {\isacharbraceleft}s{\isachardot}\ {\isasymforall}\ p\ {\isasymin}\ Paths\ s{\isachardot}\ {\isasymexists}\ i{\isachardot}\ p\ i\ {\isasymin}\ A{\isacharbraceright}{\isachardoublequote}%
\begin{isamarkuptxt}%
\noindent
In contrast to the analogous property for \isa{EF}, and just
for a change, we do not use fixed point induction but a weaker theorem,
\isa{lfp{\isacharunderscore}lowerbound}:
\begin{isabelle}%
\ \ \ \ \ f\ S\ {\isasymsubseteq}\ S\ {\isasymLongrightarrow}\ lfp\ f\ {\isasymsubseteq}\ S%
\end{isabelle}
The instance of the premise \isa{f\ S\ {\isasymsubseteq}\ S} is proved pointwise,
a decision that clarification takes for us:%
\end{isamarkuptxt}%
\isacommand{apply}{\isacharparenleft}rule\ lfp{\isacharunderscore}lowerbound{\isacharparenright}\isanewline
\isacommand{apply}{\isacharparenleft}clarsimp\ simp\ add{\isacharcolon}\ af{\isacharunderscore}def\ Paths{\isacharunderscore}def{\isacharparenright}%
\begin{isamarkuptxt}%
\begin{isabelle}%
\ {\isadigit{1}}{\isachardot}\ {\isasymAnd}p{\isachardot}\ {\isasymlbrakk}p\ {\isadigit{0}}\ {\isasymin}\ A\ {\isasymor}\isanewline
\ \ \ \ \ \ \ \ \ {\isacharparenleft}{\isasymforall}t{\isachardot}\ {\isacharparenleft}p\ {\isadigit{0}}{\isacharcomma}\ t{\isacharparenright}\ {\isasymin}\ M\ {\isasymlongrightarrow}\isanewline
\ \ \ \ \ \ \ \ \ \ \ \ \ \ {\isacharparenleft}{\isasymforall}p{\isachardot}\ t\ {\isacharequal}\ p\ {\isadigit{0}}\ {\isasymand}\ {\isacharparenleft}{\isasymforall}i{\isachardot}\ {\isacharparenleft}p\ i{\isacharcomma}\ p\ {\isacharparenleft}Suc\ i{\isacharparenright}{\isacharparenright}\ {\isasymin}\ M{\isacharparenright}\ {\isasymlongrightarrow}\isanewline
\ \ \ \ \ \ \ \ \ \ \ \ \ \ \ \ \ \ \ {\isacharparenleft}{\isasymexists}i{\isachardot}\ p\ i\ {\isasymin}\ A{\isacharparenright}{\isacharparenright}{\isacharparenright}{\isacharsemicolon}\isanewline
\ \ \ \ \ \ \ \ \ \ \ {\isasymforall}i{\isachardot}\ {\isacharparenleft}p\ i{\isacharcomma}\ p\ {\isacharparenleft}Suc\ i{\isacharparenright}{\isacharparenright}\ {\isasymin}\ M{\isasymrbrakk}\isanewline
\ \ \ \ \ \ \ \ {\isasymLongrightarrow}\ {\isasymexists}i{\isachardot}\ p\ i\ {\isasymin}\ A%
\end{isabelle}
Now we eliminate the disjunction. The case \isa{p\ {\isadigit{0}}\ {\isasymin}\ A} is trivial:%
\end{isamarkuptxt}%
\isacommand{apply}{\isacharparenleft}erule\ disjE{\isacharparenright}\isanewline
\ \isacommand{apply}{\isacharparenleft}blast{\isacharparenright}%
\begin{isamarkuptxt}%
\noindent
In the other case we set \isa{t} to \isa{p\ {\isadigit{1}}} and simplify matters:%
\end{isamarkuptxt}%
\isacommand{apply}{\isacharparenleft}erule{\isacharunderscore}tac\ x\ {\isacharequal}\ {\isachardoublequote}p\ {\isadigit{1}}{\isachardoublequote}\ \isakeyword{in}\ allE{\isacharparenright}\isanewline
\isacommand{apply}{\isacharparenleft}clarsimp{\isacharparenright}%
\begin{isamarkuptxt}%
\begin{isabelle}%
\ {\isadigit{1}}{\isachardot}\ {\isasymAnd}p{\isachardot}\ {\isasymlbrakk}{\isasymforall}i{\isachardot}\ {\isacharparenleft}p\ i{\isacharcomma}\ p\ {\isacharparenleft}Suc\ i{\isacharparenright}{\isacharparenright}\ {\isasymin}\ M{\isacharsemicolon}\isanewline
\ \ \ \ \ \ \ \ \ \ \ {\isasymforall}pa{\isachardot}\ p\ {\isadigit{1}}\ {\isacharequal}\ pa\ {\isadigit{0}}\ {\isasymand}\ {\isacharparenleft}{\isasymforall}i{\isachardot}\ {\isacharparenleft}pa\ i{\isacharcomma}\ pa\ {\isacharparenleft}Suc\ i{\isacharparenright}{\isacharparenright}\ {\isasymin}\ M{\isacharparenright}\ {\isasymlongrightarrow}\isanewline
\ \ \ \ \ \ \ \ \ \ \ \ \ \ \ \ {\isacharparenleft}{\isasymexists}i{\isachardot}\ pa\ i\ {\isasymin}\ A{\isacharparenright}{\isasymrbrakk}\isanewline
\ \ \ \ \ \ \ \ {\isasymLongrightarrow}\ {\isasymexists}i{\isachardot}\ p\ i\ {\isasymin}\ A%
\end{isabelle}
It merely remains to set \isa{pa} to \isa{{\isasymlambda}i{\isachardot}\ p\ {\isacharparenleft}i\ {\isacharplus}\ {\isadigit{1}}{\isacharparenright}}, i.e.\ \isa{p} without its
first element. The rest is practically automatic:%
\end{isamarkuptxt}%
\isacommand{apply}{\isacharparenleft}erule{\isacharunderscore}tac\ x\ {\isacharequal}\ {\isachardoublequote}{\isasymlambda}i{\isachardot}\ p{\isacharparenleft}i{\isacharplus}{\isadigit{1}}{\isacharparenright}{\isachardoublequote}\ \isakeyword{in}\ allE{\isacharparenright}\isanewline
\isacommand{apply}\ simp\isanewline
\isacommand{apply}\ blast\isanewline
\isacommand{done}%
\begin{isamarkuptext}%
The opposite containment is proved by contradiction: if some state
\isa{s} is not in \isa{lfp\ {\isacharparenleft}af\ A{\isacharparenright}}, then we can construct an
infinite \isa{A}-avoiding path starting from \isa{s}. The reason is
that by unfolding \isa{lfp} we find that if \isa{s} is not in
\isa{lfp\ {\isacharparenleft}af\ A{\isacharparenright}}, then \isa{s} is not in \isa{A} and there is a
direct successor of \isa{s} that is again not in \isa{lfp\ {\isacharparenleft}af\ A{\isacharparenright}}. Iterating this argument yields the promised infinite
\isa{A}-avoiding path. Let us formalize this sketch.

The one-step argument in the above sketch%
\end{isamarkuptext}%
\isacommand{lemma}\ not{\isacharunderscore}in{\isacharunderscore}lfp{\isacharunderscore}afD{\isacharcolon}\isanewline
\ {\isachardoublequote}s\ {\isasymnotin}\ lfp{\isacharparenleft}af\ A{\isacharparenright}\ {\isasymLongrightarrow}\ s\ {\isasymnotin}\ A\ {\isasymand}\ {\isacharparenleft}{\isasymexists}\ t{\isachardot}\ {\isacharparenleft}s{\isacharcomma}t{\isacharparenright}{\isasymin}M\ {\isasymand}\ t\ {\isasymnotin}\ lfp{\isacharparenleft}af\ A{\isacharparenright}{\isacharparenright}{\isachardoublequote}\isanewline
\isacommand{apply}{\isacharparenleft}erule\ contrapos{\isacharunderscore}np{\isacharparenright}\isanewline
\isacommand{apply}{\isacharparenleft}rule\ ssubst{\isacharbrackleft}OF\ lfp{\isacharunderscore}unfold{\isacharbrackleft}OF\ mono{\isacharunderscore}af{\isacharbrackright}{\isacharbrackright}{\isacharparenright}\isanewline
\isacommand{apply}{\isacharparenleft}simp\ add{\isacharcolon}af{\isacharunderscore}def{\isacharparenright}\isanewline
\isacommand{done}%
\begin{isamarkuptext}%
\noindent
is proved by a variant of contraposition:
assume the negation of the conclusion and prove \isa{s\ {\isasymin}\ lfp\ {\isacharparenleft}af\ A{\isacharparenright}}.
Unfolding \isa{lfp} once and
simplifying with the definition of \isa{af} finishes the proof.

Now we iterate this process. The following construction of the desired
path is parameterized by a predicate \isa{P} that should hold along the path:%
\end{isamarkuptext}%
\isacommand{consts}\ path\ {\isacharcolon}{\isacharcolon}\ {\isachardoublequote}state\ {\isasymRightarrow}\ {\isacharparenleft}state\ {\isasymRightarrow}\ bool{\isacharparenright}\ {\isasymRightarrow}\ {\isacharparenleft}nat\ {\isasymRightarrow}\ state{\isacharparenright}{\isachardoublequote}\isanewline
\isacommand{primrec}\isanewline
{\isachardoublequote}path\ s\ P\ {\isadigit{0}}\ {\isacharequal}\ s{\isachardoublequote}\isanewline
{\isachardoublequote}path\ s\ P\ {\isacharparenleft}Suc\ n{\isacharparenright}\ {\isacharequal}\ {\isacharparenleft}SOME\ t{\isachardot}\ {\isacharparenleft}path\ s\ P\ n{\isacharcomma}t{\isacharparenright}\ {\isasymin}\ M\ {\isasymand}\ P\ t{\isacharparenright}{\isachardoublequote}%
\begin{isamarkuptext}%
\noindent
Element \isa{n\ {\isacharplus}\ {\isadigit{1}}} on this path is some arbitrary successor
\isa{t} of element \isa{n} such that \isa{P\ t} holds.  Remember that \isa{SOME\ t{\isachardot}\ R\ t}
is some arbitrary but fixed \isa{t} such that \isa{R\ t} holds (see \S\ref{sec:SOME}). Of
course, such a \isa{t} may in general not exist, but that is of no
concern to us since we will only use \isa{path} in such cases where a
suitable \isa{t} does exist.

Let us show that if each state \isa{s} that satisfies \isa{P}
has a successor that again satisfies \isa{P}, then there exists an infinite \isa{P}-path:%
\end{isamarkuptext}%
\isacommand{lemma}\ infinity{\isacharunderscore}lemma{\isacharcolon}\isanewline
\ \ {\isachardoublequote}{\isasymlbrakk}\ P\ s{\isacharsemicolon}\ {\isasymforall}s{\isachardot}\ P\ s\ {\isasymlongrightarrow}\ {\isacharparenleft}{\isasymexists}\ t{\isachardot}\ {\isacharparenleft}s{\isacharcomma}t{\isacharparenright}\ {\isasymin}\ M\ {\isasymand}\ P\ t{\isacharparenright}\ {\isasymrbrakk}\ {\isasymLongrightarrow}\isanewline
\ \ \ {\isasymexists}p{\isasymin}Paths\ s{\isachardot}\ {\isasymforall}i{\isachardot}\ P{\isacharparenleft}p\ i{\isacharparenright}{\isachardoublequote}%
\begin{isamarkuptxt}%
\noindent
First we rephrase the conclusion slightly because we need to prove both the path property
and the fact that \isa{P} holds simultaneously:%
\end{isamarkuptxt}%
\isacommand{apply}{\isacharparenleft}subgoal{\isacharunderscore}tac\ {\isachardoublequote}{\isasymexists}p{\isachardot}\ s\ {\isacharequal}\ p\ {\isadigit{0}}\ {\isasymand}\ {\isacharparenleft}{\isasymforall}i{\isachardot}\ {\isacharparenleft}p\ i{\isacharcomma}p{\isacharparenleft}i{\isacharplus}{\isadigit{1}}{\isacharparenright}{\isacharparenright}\ {\isasymin}\ M\ {\isasymand}\ P{\isacharparenleft}p\ i{\isacharparenright}{\isacharparenright}{\isachardoublequote}{\isacharparenright}%
\begin{isamarkuptxt}%
\noindent
From this proposition the original goal follows easily:%
\end{isamarkuptxt}%
\ \isacommand{apply}{\isacharparenleft}simp\ add{\isacharcolon}Paths{\isacharunderscore}def{\isacharcomma}\ blast{\isacharparenright}%
\begin{isamarkuptxt}%
\noindent
The new subgoal is proved by providing the witness \isa{path\ s\ P} for \isa{p}:%
\end{isamarkuptxt}%
\isacommand{apply}{\isacharparenleft}rule{\isacharunderscore}tac\ x\ {\isacharequal}\ {\isachardoublequote}path\ s\ P{\isachardoublequote}\ \isakeyword{in}\ exI{\isacharparenright}\isanewline
\isacommand{apply}{\isacharparenleft}clarsimp{\isacharparenright}%
\begin{isamarkuptxt}%
\noindent
After simplification and clarification the subgoal has the following compact form
\begin{isabelle}%
\ {\isadigit{1}}{\isachardot}\ {\isasymAnd}i{\isachardot}\ {\isasymlbrakk}P\ s{\isacharsemicolon}\ {\isasymforall}s{\isachardot}\ P\ s\ {\isasymlongrightarrow}\ {\isacharparenleft}{\isasymexists}t{\isachardot}\ {\isacharparenleft}s{\isacharcomma}\ t{\isacharparenright}\ {\isasymin}\ M\ {\isasymand}\ P\ t{\isacharparenright}{\isasymrbrakk}\isanewline
\ \ \ \ \ \ \ \ {\isasymLongrightarrow}\ {\isacharparenleft}path\ s\ P\ i{\isacharcomma}\ SOME\ t{\isachardot}\ {\isacharparenleft}path\ s\ P\ i{\isacharcomma}\ t{\isacharparenright}\ {\isasymin}\ M\ {\isasymand}\ P\ t{\isacharparenright}\ {\isasymin}\ M\ {\isasymand}\isanewline
\ \ \ \ \ \ \ \ \ \ \ P\ {\isacharparenleft}path\ s\ P\ i{\isacharparenright}%
\end{isabelle}
and invites a proof by induction on \isa{i}:%
\end{isamarkuptxt}%
\isacommand{apply}{\isacharparenleft}induct{\isacharunderscore}tac\ i{\isacharparenright}\isanewline
\ \isacommand{apply}{\isacharparenleft}simp{\isacharparenright}%
\begin{isamarkuptxt}%
\noindent
After simplification the base case boils down to
\begin{isabelle}%
\ {\isadigit{1}}{\isachardot}\ {\isasymlbrakk}P\ s{\isacharsemicolon}\ {\isasymforall}s{\isachardot}\ P\ s\ {\isasymlongrightarrow}\ {\isacharparenleft}{\isasymexists}t{\isachardot}\ {\isacharparenleft}s{\isacharcomma}\ t{\isacharparenright}\ {\isasymin}\ M\ {\isasymand}\ P\ t{\isacharparenright}{\isasymrbrakk}\isanewline
\ \ \ \ {\isasymLongrightarrow}\ {\isacharparenleft}s{\isacharcomma}\ SOME\ t{\isachardot}\ {\isacharparenleft}s{\isacharcomma}\ t{\isacharparenright}\ {\isasymin}\ M\ {\isasymand}\ P\ t{\isacharparenright}\ {\isasymin}\ M%
\end{isabelle}
The conclusion looks exceedingly trivial: after all, \isa{t} is chosen such that \isa{{\isacharparenleft}s{\isacharcomma}\ t{\isacharparenright}\ {\isasymin}\ M}
holds. However, we first have to show that such a \isa{t} actually exists! This reasoning
is embodied in the theorem \isa{someI{\isadigit{2}}{\isacharunderscore}ex}:
\begin{isabelle}%
\ \ \ \ \ {\isasymlbrakk}{\isasymexists}a{\isachardot}\ {\isacharquery}P\ a{\isacharsemicolon}\ {\isasymAnd}x{\isachardot}\ {\isacharquery}P\ x\ {\isasymLongrightarrow}\ {\isacharquery}Q\ x{\isasymrbrakk}\ {\isasymLongrightarrow}\ {\isacharquery}Q\ {\isacharparenleft}SOME\ x{\isachardot}\ {\isacharquery}P\ x{\isacharparenright}%
\end{isabelle}
When we apply this theorem as an introduction rule, \isa{{\isacharquery}P\ x} becomes
\isa{{\isacharparenleft}s{\isacharcomma}\ x{\isacharparenright}\ {\isasymin}\ M\ {\isasymand}\ P\ x} and \isa{{\isacharquery}Q\ x} becomes \isa{{\isacharparenleft}s{\isacharcomma}\ x{\isacharparenright}\ {\isasymin}\ M} and we have to prove
two subgoals: \isa{{\isasymexists}a{\isachardot}\ {\isacharparenleft}s{\isacharcomma}\ a{\isacharparenright}\ {\isasymin}\ M\ {\isasymand}\ P\ a}, which follows from the assumptions, and
\isa{{\isacharparenleft}s{\isacharcomma}\ x{\isacharparenright}\ {\isasymin}\ M\ {\isasymand}\ P\ x\ {\isasymLongrightarrow}\ {\isacharparenleft}s{\isacharcomma}\ x{\isacharparenright}\ {\isasymin}\ M}, which is trivial. Thus it is not surprising that
\isa{fast} can prove the base case quickly:%
\end{isamarkuptxt}%
\ \isacommand{apply}{\isacharparenleft}fast\ intro{\isacharcolon}someI{\isadigit{2}}{\isacharunderscore}ex{\isacharparenright}%
\begin{isamarkuptxt}%
\noindent
What is worth noting here is that we have used \isa{fast} rather than
\isa{blast}.  The reason is that \isa{blast} would fail because it cannot
cope with \isa{someI{\isadigit{2}}{\isacharunderscore}ex}: unifying its conclusion with the current
subgoal is nontrivial because of the nested schematic variables. For
efficiency reasons \isa{blast} does not even attempt such unifications.
Although \isa{fast} can in principle cope with complicated unification
problems, in practice the number of unifiers arising is often prohibitive and
the offending rule may need to be applied explicitly rather than
automatically. This is what happens in the step case.

The induction step is similar, but more involved, because now we face nested
occurrences of \isa{SOME}. As a result, \isa{fast} is no longer able to
solve the subgoal and we apply \isa{someI{\isadigit{2}}{\isacharunderscore}ex} by hand.  We merely
show the proof commands but do not describe the details:%
\end{isamarkuptxt}%
\isacommand{apply}{\isacharparenleft}simp{\isacharparenright}\isanewline
\isacommand{apply}{\isacharparenleft}rule\ someI{\isadigit{2}}{\isacharunderscore}ex{\isacharparenright}\isanewline
\ \isacommand{apply}{\isacharparenleft}blast{\isacharparenright}\isanewline
\isacommand{apply}{\isacharparenleft}rule\ someI{\isadigit{2}}{\isacharunderscore}ex{\isacharparenright}\isanewline
\ \isacommand{apply}{\isacharparenleft}blast{\isacharparenright}\isanewline
\isacommand{apply}{\isacharparenleft}blast{\isacharparenright}\isanewline
\isacommand{done}%
\begin{isamarkuptext}%
Function \isa{path} has fulfilled its purpose now and can be forgotten
about. It was merely defined to provide the witness in the proof of the
\isa{infinity{\isacharunderscore}lemma}. Aficionados of minimal proofs might like to know
that we could have given the witness without having to define a new function:
the term
\begin{isabelle}%
\ \ \ \ \ nat{\isacharunderscore}rec\ s\ {\isacharparenleft}{\isasymlambda}n\ t{\isachardot}\ SOME\ u{\isachardot}\ {\isacharparenleft}t{\isacharcomma}\ u{\isacharparenright}\ {\isasymin}\ M\ {\isasymand}\ P\ u{\isacharparenright}%
\end{isabelle}
is extensionally equal to \isa{path\ s\ P},
where \isa{nat{\isacharunderscore}rec} is the predefined primitive recursor on \isa{nat}, whose defining
equations we omit.%
\end{isamarkuptext}%
%
\begin{isamarkuptext}%
At last we can prove the opposite direction of \isa{AF{\isacharunderscore}lemma{\isadigit{1}}}:%
\end{isamarkuptext}%
\isacommand{theorem}\ AF{\isacharunderscore}lemma{\isadigit{2}}{\isacharcolon}\ {\isachardoublequote}{\isacharbraceleft}s{\isachardot}\ {\isasymforall}\ p\ {\isasymin}\ Paths\ s{\isachardot}\ {\isasymexists}\ i{\isachardot}\ p\ i\ {\isasymin}\ A{\isacharbraceright}\ {\isasymsubseteq}\ lfp{\isacharparenleft}af\ A{\isacharparenright}{\isachardoublequote}%
\begin{isamarkuptxt}%
\noindent
The proof is again pointwise and then by contraposition:%
\end{isamarkuptxt}%
\isacommand{apply}{\isacharparenleft}rule\ subsetI{\isacharparenright}\isanewline
\isacommand{apply}{\isacharparenleft}erule\ contrapos{\isacharunderscore}pp{\isacharparenright}\isanewline
\isacommand{apply}\ simp%
\begin{isamarkuptxt}%
\begin{isabelle}%
\ {\isadigit{1}}{\isachardot}\ {\isasymAnd}x{\isachardot}\ x\ {\isasymnotin}\ lfp\ {\isacharparenleft}af\ A{\isacharparenright}\ {\isasymLongrightarrow}\ {\isasymexists}p{\isasymin}Paths\ x{\isachardot}\ {\isasymforall}i{\isachardot}\ p\ i\ {\isasymnotin}\ A%
\end{isabelle}
Applying the \isa{infinity{\isacharunderscore}lemma} as a destruction rule leaves two subgoals, the second
premise of \isa{infinity{\isacharunderscore}lemma} and the original subgoal:%
\end{isamarkuptxt}%
\isacommand{apply}{\isacharparenleft}drule\ infinity{\isacharunderscore}lemma{\isacharparenright}%
\begin{isamarkuptxt}%
\begin{isabelle}%
\ {\isadigit{1}}{\isachardot}\ {\isasymAnd}x{\isachardot}\ {\isasymforall}s{\isachardot}\ s\ {\isasymnotin}\ lfp\ {\isacharparenleft}af\ A{\isacharparenright}\ {\isasymlongrightarrow}\ {\isacharparenleft}{\isasymexists}t{\isachardot}\ {\isacharparenleft}s{\isacharcomma}\ t{\isacharparenright}\ {\isasymin}\ M\ {\isasymand}\ t\ {\isasymnotin}\ lfp\ {\isacharparenleft}af\ A{\isacharparenright}{\isacharparenright}\isanewline
\ {\isadigit{2}}{\isachardot}\ {\isasymAnd}x{\isachardot}\ {\isasymexists}p{\isasymin}Paths\ x{\isachardot}\ {\isasymforall}i{\isachardot}\ p\ i\ {\isasymnotin}\ lfp\ {\isacharparenleft}af\ A{\isacharparenright}\ {\isasymLongrightarrow}\isanewline
\ \ \ \ \ \ \ \ {\isasymexists}p{\isasymin}Paths\ x{\isachardot}\ {\isasymforall}i{\isachardot}\ p\ i\ {\isasymnotin}\ A%
\end{isabelle}
Both are solved automatically:%
\end{isamarkuptxt}%
\ \isacommand{apply}{\isacharparenleft}auto\ dest{\isacharcolon}not{\isacharunderscore}in{\isacharunderscore}lfp{\isacharunderscore}afD{\isacharparenright}\isanewline
\isacommand{done}%
\begin{isamarkuptext}%
If you found the above proofs somewhat complicated we recommend you read
\S\ref{sec:CTL-revisited} where we shown how inductive definitions lead to
simpler arguments.

The main theorem is proved as for PDL, except that we also derive the
necessary equality \isa{lfp{\isacharparenleft}af\ A{\isacharparenright}\ {\isacharequal}\ {\isachardot}{\isachardot}{\isachardot}} by combining
\isa{AF{\isacharunderscore}lemma{\isadigit{1}}} and \isa{AF{\isacharunderscore}lemma{\isadigit{2}}} on the spot:%
\end{isamarkuptext}%
\isacommand{theorem}\ {\isachardoublequote}mc\ f\ {\isacharequal}\ {\isacharbraceleft}s{\isachardot}\ s\ {\isasymTurnstile}\ f{\isacharbraceright}{\isachardoublequote}\isanewline
\isacommand{apply}{\isacharparenleft}induct{\isacharunderscore}tac\ f{\isacharparenright}\isanewline
\isacommand{apply}{\isacharparenleft}auto\ simp\ add{\isacharcolon}\ EF{\isacharunderscore}lemma\ equalityI{\isacharbrackleft}OF\ AF{\isacharunderscore}lemma{\isadigit{1}}\ AF{\isacharunderscore}lemma{\isadigit{2}}{\isacharbrackright}{\isacharparenright}\isanewline
\isacommand{done}%
\begin{isamarkuptext}%
The above language is not quite CTL\@. The latter also includes an
until-operator \isa{EU\ f\ g} with semantics ``there exist a path
where \isa{f} is true until \isa{g} becomes true''. With the help
of an auxiliary function%
\end{isamarkuptext}%
\isacommand{consts}\ until{\isacharcolon}{\isacharcolon}\ {\isachardoublequote}state\ set\ {\isasymRightarrow}\ state\ set\ {\isasymRightarrow}\ state\ {\isasymRightarrow}\ state\ list\ {\isasymRightarrow}\ bool{\isachardoublequote}\isanewline
\isacommand{primrec}\isanewline
{\isachardoublequote}until\ A\ B\ s\ {\isacharbrackleft}{\isacharbrackright}\ \ \ \ {\isacharequal}\ {\isacharparenleft}s\ {\isasymin}\ B{\isacharparenright}{\isachardoublequote}\isanewline
{\isachardoublequote}until\ A\ B\ s\ {\isacharparenleft}t{\isacharhash}p{\isacharparenright}\ {\isacharequal}\ {\isacharparenleft}s\ {\isasymin}\ A\ {\isasymand}\ {\isacharparenleft}s{\isacharcomma}t{\isacharparenright}\ {\isasymin}\ M\ {\isasymand}\ until\ A\ B\ t\ p{\isacharparenright}{\isachardoublequote}%
\begin{isamarkuptext}%
\noindent
the semantics of \isa{EU} is straightforward:
\begin{isabelle}%
\ \ \ \ \ s\ {\isasymTurnstile}\ EU\ f\ g\ {\isacharequal}\ {\isacharparenleft}{\isasymexists}p{\isachardot}\ until\ A\ B\ s\ p{\isacharparenright}%
\end{isabelle}
Note that \isa{EU} is not definable in terms of the other operators!

Model checking \isa{EU} is again a least fixed point construction:
\begin{isabelle}%
\ \ \ \ \ mc{\isacharparenleft}EU\ f\ g{\isacharparenright}\ {\isacharequal}\ lfp{\isacharparenleft}{\isasymlambda}T{\isachardot}\ mc\ g\ {\isasymunion}\ mc\ f\ {\isasyminter}\ {\isacharparenleft}M{\isasyminverse}\ {\isacharbackquote}{\isacharbackquote}\ T{\isacharparenright}{\isacharparenright}%
\end{isabelle}

\begin{exercise}
Extend the datatype of formulae by the above until operator
and prove the equivalence between semantics and model checking, i.e.\ that
\begin{isabelle}%
\ \ \ \ \ mc\ {\isacharparenleft}EU\ f\ g{\isacharparenright}\ {\isacharequal}\ {\isacharbraceleft}s{\isachardot}\ s\ {\isasymTurnstile}\ EU\ f\ g{\isacharbraceright}%
\end{isabelle}
%For readability you may want to annotate {term EU} with its customary syntax
%{text[display]"| EU formula formula    E[_ U _]"}
%which enables you to read and write {text"E[f U g]"} instead of {term"EU f g"}.
\end{exercise}
For more CTL exercises see, for example, \cite{Huth-Ryan-book}.%
\end{isamarkuptext}%
%
\begin{isamarkuptext}%
Let us close this section with a few words about the executability of our model checkers.
It is clear that if all sets are finite, they can be represented as lists and the usual
set operations are easily implemented. Only \isa{lfp} requires a little thought.
Fortunately the HOL library proves that in the case of finite sets and a monotone \isa{F},
\isa{lfp\ F} can be computed by iterated application of \isa{F} to \isa{{\isacharbraceleft}{\isacharbraceright}} until
a fixed point is reached. It is actually possible to generate executable functional programs
from HOL definitions, but that is beyond the scope of the tutorial.%
\end{isamarkuptext}%
\end{isabellebody}%
%%% Local Variables:
%%% mode: latex
%%% TeX-master: "root"
%%% End:
