%
\begin{isabellebody}%
\def\isabellecontext{Classes}%
%
\isadelimtheory
%
\endisadelimtheory
%
\isatagtheory
\isacommand{theory}\isamarkupfalse%
\ Classes\isanewline
\isakeyword{imports}\ Main\ Setup\isanewline
\isakeyword{begin}%
\endisatagtheory
{\isafoldtheory}%
%
\isadelimtheory
%
\endisadelimtheory
%
\isamarkupchapter{Haskell-style classes with Isabelle/Isar%
}
\isamarkuptrue%
%
\isamarkupsection{Introduction%
}
\isamarkuptrue%
%
\begin{isamarkuptext}%
Type classes were introduces by Wadler and Blott \cite{wadler89how}
  into the Haskell language, to allow for a reasonable implementation
  of overloading\footnote{throughout this tutorial, we are referring
  to classical Haskell 1.0 type classes, not considering
  later additions in expressiveness}.
  As a canonical example, a polymorphic equality function
  \isa{eq\ {\isasymColon}\ {\isasymalpha}\ {\isasymRightarrow}\ {\isasymalpha}\ {\isasymRightarrow}\ bool} which is overloaded on different
  types for \isa{{\isasymalpha}}, which is achieved by splitting introduction
  of the \isa{eq} function from its overloaded definitions by means
  of \isa{class} and \isa{instance} declarations:

  \begin{quote}

  \noindent\isa{class\ eq\ where}\footnote{syntax here is a kind of isabellized Haskell} \\
  \hspace*{2ex}\isa{eq\ {\isasymColon}\ {\isasymalpha}\ {\isasymRightarrow}\ {\isasymalpha}\ {\isasymRightarrow}\ bool}

  \medskip\noindent\isa{instance\ nat\ {\isasymColon}\ eq\ where} \\
  \hspace*{2ex}\isa{eq\ {\isadigit{0}}\ {\isadigit{0}}\ {\isacharequal}\ True} \\
  \hspace*{2ex}\isa{eq\ {\isadigit{0}}\ {\isacharunderscore}\ {\isacharequal}\ False} \\
  \hspace*{2ex}\isa{eq\ {\isacharunderscore}\ {\isadigit{0}}\ {\isacharequal}\ False} \\
  \hspace*{2ex}\isa{eq\ {\isacharparenleft}Suc\ n{\isacharparenright}\ {\isacharparenleft}Suc\ m{\isacharparenright}\ {\isacharequal}\ eq\ n\ m}

  \medskip\noindent\\isa{instance\ {\isacharparenleft}{\isasymalpha}{\isasymColon}eq{\isacharcomma}\ {\isasymbeta}{\isasymColon}eq{\isacharparenright}\ pair\ {\isasymColon}\ eq\ where} \\
  \hspace*{2ex}\isa{eq\ {\isacharparenleft}x{\isadigit{1}}{\isacharcomma}\ y{\isadigit{1}}{\isacharparenright}\ {\isacharparenleft}x{\isadigit{2}}{\isacharcomma}\ y{\isadigit{2}}{\isacharparenright}\ {\isacharequal}\ eq\ x{\isadigit{1}}\ x{\isadigit{2}}\ {\isasymand}\ eq\ y{\isadigit{1}}\ y{\isadigit{2}}}

  \medskip\noindent\isa{class\ ord\ extends\ eq\ where} \\
  \hspace*{2ex}\isa{less{\isacharunderscore}eq\ {\isasymColon}\ {\isasymalpha}\ {\isasymRightarrow}\ {\isasymalpha}\ {\isasymRightarrow}\ bool} \\
  \hspace*{2ex}\isa{less\ {\isasymColon}\ {\isasymalpha}\ {\isasymRightarrow}\ {\isasymalpha}\ {\isasymRightarrow}\ bool}

  \end{quote}

  \noindent Type variables are annotated with (finitely many) classes;
  these annotations are assertions that a particular polymorphic type
  provides definitions for overloaded functions.

  Indeed, type classes not only allow for simple overloading
  but form a generic calculus, an instance of order-sorted
  algebra \cite{Nipkow-Prehofer:1993,nipkow-sorts93,Wenzel:1997:TPHOL}.

  From a software engeneering point of view, type classes
  roughly correspond to interfaces in object-oriented languages like Java;
  so, it is naturally desirable that type classes do not only
  provide functions (class parameters) but also state specifications
  implementations must obey.  For example, the \isa{class\ eq}
  above could be given the following specification, demanding that
  \isa{class\ eq} is an equivalence relation obeying reflexivity,
  symmetry and transitivity:

  \begin{quote}

  \noindent\isa{class\ eq\ where} \\
  \hspace*{2ex}\isa{eq\ {\isasymColon}\ {\isasymalpha}\ {\isasymRightarrow}\ {\isasymalpha}\ {\isasymRightarrow}\ bool} \\
  \isa{satisfying} \\
  \hspace*{2ex}\isa{refl{\isacharcolon}\ eq\ x\ x} \\
  \hspace*{2ex}\isa{sym{\isacharcolon}\ eq\ x\ y\ {\isasymlongleftrightarrow}\ eq\ x\ y} \\
  \hspace*{2ex}\isa{trans{\isacharcolon}\ eq\ x\ y\ {\isasymand}\ eq\ y\ z\ {\isasymlongrightarrow}\ eq\ x\ z}

  \end{quote}

  \noindent From a theoretic point of view, type classes are lightweight
  modules; Haskell type classes may be emulated by
  SML functors \cite{classes_modules}. 
  Isabelle/Isar offers a discipline of type classes which brings
  all those aspects together:

  \begin{enumerate}
    \item specifying abstract parameters together with
       corresponding specifications,
    \item instantiating those abstract parameters by a particular
       type
    \item in connection with a ``less ad-hoc'' approach to overloading,
    \item with a direct link to the Isabelle module system
      (aka locales \cite{kammueller-locales}).
  \end{enumerate}

  \noindent Isar type classes also directly support code generation
  in a Haskell like fashion.

  This tutorial demonstrates common elements of structured specifications
  and abstract reasoning with type classes by the algebraic hierarchy of
  semigroups, monoids and groups.  Our background theory is that of
  Isabelle/HOL \cite{isa-tutorial}, for which some
  familiarity is assumed.

  Here we merely present the look-and-feel for end users.
  Internally, those are mapped to more primitive Isabelle concepts.
  See \cite{Haftmann-Wenzel:2006:classes} for more detail.%
\end{isamarkuptext}%
\isamarkuptrue%
%
\isamarkupsection{A simple algebra example \label{sec:example}%
}
\isamarkuptrue%
%
\isamarkupsubsection{Class definition%
}
\isamarkuptrue%
%
\begin{isamarkuptext}%
Depending on an arbitrary type \isa{{\isasymalpha}}, class \isa{semigroup} introduces a binary operator \isa{{\isacharparenleft}{\isasymotimes}{\isacharparenright}} that is
  assumed to be associative:%
\end{isamarkuptext}%
\isamarkuptrue%
%
\isadelimquote
%
\endisadelimquote
%
\isatagquote
\isacommand{class}\isamarkupfalse%
\ semigroup\ {\isacharequal}\ type\ {\isacharplus}\isanewline
\ \ \isakeyword{fixes}\ mult\ {\isacharcolon}{\isacharcolon}\ {\isachardoublequoteopen}{\isasymalpha}\ {\isasymRightarrow}\ {\isasymalpha}\ {\isasymRightarrow}\ {\isasymalpha}{\isachardoublequoteclose}\ \ \ \ {\isacharparenleft}\isakeyword{infixl}\ {\isachardoublequoteopen}{\isasymotimes}{\isachardoublequoteclose}\ {\isadigit{7}}{\isadigit{0}}{\isacharparenright}\isanewline
\ \ \isakeyword{assumes}\ assoc{\isacharcolon}\ {\isachardoublequoteopen}{\isacharparenleft}x\ {\isasymotimes}\ y{\isacharparenright}\ {\isasymotimes}\ z\ {\isacharequal}\ x\ {\isasymotimes}\ {\isacharparenleft}y\ {\isasymotimes}\ z{\isacharparenright}{\isachardoublequoteclose}%
\endisatagquote
{\isafoldquote}%
%
\isadelimquote
%
\endisadelimquote
%
\begin{isamarkuptext}%
\noindent This \hyperlink{command.class}{\mbox{\isa{\isacommand{class}}}} specification consists of two
  parts: the \qn{operational} part names the class parameter
  (\hyperlink{element.fixes}{\mbox{\isa{\isakeyword{fixes}}}}), the \qn{logical} part specifies properties on them
  (\hyperlink{element.assumes}{\mbox{\isa{\isakeyword{assumes}}}}).  The local \hyperlink{element.fixes}{\mbox{\isa{\isakeyword{fixes}}}} and
  \hyperlink{element.assumes}{\mbox{\isa{\isakeyword{assumes}}}} are lifted to the theory toplevel,
  yielding the global
  parameter \isa{{\isachardoublequote}mult\ {\isasymColon}\ {\isasymalpha}{\isasymColon}semigroup\ {\isasymRightarrow}\ {\isasymalpha}\ {\isasymRightarrow}\ {\isasymalpha}{\isachardoublequote}} and the
  global theorem \hyperlink{fact.semigroup.assoc:}{\mbox{\isa{semigroup{\isachardot}assoc{\isacharcolon}}}}~\isa{{\isachardoublequote}{\isasymAnd}x\ y\ z\ {\isasymColon}\ {\isasymalpha}{\isasymColon}semigroup{\isachardot}\ {\isacharparenleft}x\ {\isasymotimes}\ y{\isacharparenright}\ {\isasymotimes}\ z\ {\isacharequal}\ x\ {\isasymotimes}\ {\isacharparenleft}y\ {\isasymotimes}\ z{\isacharparenright}{\isachardoublequote}}.%
\end{isamarkuptext}%
\isamarkuptrue%
%
\isamarkupsubsection{Class instantiation \label{sec:class_inst}%
}
\isamarkuptrue%
%
\begin{isamarkuptext}%
The concrete type \isa{int} is made a \isa{semigroup}
  instance by providing a suitable definition for the class parameter
  \isa{{\isacharparenleft}{\isasymotimes}{\isacharparenright}} and a proof for the specification of \hyperlink{fact.assoc}{\mbox{\isa{assoc}}}.
  This is accomplished by the \hyperlink{command.instantiation}{\mbox{\isa{\isacommand{instantiation}}}} target:%
\end{isamarkuptext}%
\isamarkuptrue%
%
\isadelimquote
%
\endisadelimquote
%
\isatagquote
\isacommand{instantiation}\isamarkupfalse%
\ int\ {\isacharcolon}{\isacharcolon}\ semigroup\isanewline
\isakeyword{begin}\isanewline
\isanewline
\isacommand{definition}\isamarkupfalse%
\isanewline
\ \ mult{\isacharunderscore}int{\isacharunderscore}def{\isacharcolon}\ {\isachardoublequoteopen}i\ {\isasymotimes}\ j\ {\isacharequal}\ i\ {\isacharplus}\ {\isacharparenleft}j{\isasymColon}int{\isacharparenright}{\isachardoublequoteclose}\isanewline
\isanewline
\isacommand{instance}\isamarkupfalse%
\ \isacommand{proof}\isamarkupfalse%
\isanewline
\ \ \isacommand{fix}\isamarkupfalse%
\ i\ j\ k\ {\isacharcolon}{\isacharcolon}\ int\ \isacommand{have}\isamarkupfalse%
\ {\isachardoublequoteopen}{\isacharparenleft}i\ {\isacharplus}\ j{\isacharparenright}\ {\isacharplus}\ k\ {\isacharequal}\ i\ {\isacharplus}\ {\isacharparenleft}j\ {\isacharplus}\ k{\isacharparenright}{\isachardoublequoteclose}\ \isacommand{by}\isamarkupfalse%
\ simp\isanewline
\ \ \isacommand{then}\isamarkupfalse%
\ \isacommand{show}\isamarkupfalse%
\ {\isachardoublequoteopen}{\isacharparenleft}i\ {\isasymotimes}\ j{\isacharparenright}\ {\isasymotimes}\ k\ {\isacharequal}\ i\ {\isasymotimes}\ {\isacharparenleft}j\ {\isasymotimes}\ k{\isacharparenright}{\isachardoublequoteclose}\isanewline
\ \ \ \ \isacommand{unfolding}\isamarkupfalse%
\ mult{\isacharunderscore}int{\isacharunderscore}def\ \isacommand{{\isachardot}}\isamarkupfalse%
\isanewline
\isacommand{qed}\isamarkupfalse%
\isanewline
\isanewline
\isacommand{end}\isamarkupfalse%
%
\endisatagquote
{\isafoldquote}%
%
\isadelimquote
%
\endisadelimquote
%
\begin{isamarkuptext}%
\noindent \hyperlink{command.instantiation}{\mbox{\isa{\isacommand{instantiation}}}} allows to define class parameters
  at a particular instance using common specification tools (here,
  \hyperlink{command.definition}{\mbox{\isa{\isacommand{definition}}}}).  The concluding \hyperlink{command.instance}{\mbox{\isa{\isacommand{instance}}}}
  opens a proof that the given parameters actually conform
  to the class specification.  Note that the first proof step
  is the \hyperlink{method.default}{\mbox{\isa{default}}} method,
  which for such instance proofs maps to the \hyperlink{method.intro-classes}{\mbox{\isa{intro{\isacharunderscore}classes}}} method.
  This boils down an instance judgement to the relevant primitive
  proof goals and should conveniently always be the first method applied
  in an instantiation proof.

  From now on, the type-checker will consider \isa{int}
  as a \isa{semigroup} automatically, i.e.\ any general results
  are immediately available on concrete instances.

  \medskip Another instance of \isa{semigroup} are the natural numbers:%
\end{isamarkuptext}%
\isamarkuptrue%
%
\isadelimquote
%
\endisadelimquote
%
\isatagquote
\isacommand{instantiation}\isamarkupfalse%
\ nat\ {\isacharcolon}{\isacharcolon}\ semigroup\isanewline
\isakeyword{begin}\isanewline
\isanewline
\isacommand{primrec}\isamarkupfalse%
\ mult{\isacharunderscore}nat\ \isakeyword{where}\isanewline
\ \ {\isachardoublequoteopen}{\isacharparenleft}{\isadigit{0}}{\isasymColon}nat{\isacharparenright}\ {\isasymotimes}\ n\ {\isacharequal}\ n{\isachardoublequoteclose}\isanewline
\ \ {\isacharbar}\ {\isachardoublequoteopen}Suc\ m\ {\isasymotimes}\ n\ {\isacharequal}\ Suc\ {\isacharparenleft}m\ {\isasymotimes}\ n{\isacharparenright}{\isachardoublequoteclose}\isanewline
\isanewline
\isacommand{instance}\isamarkupfalse%
\ \isacommand{proof}\isamarkupfalse%
\isanewline
\ \ \isacommand{fix}\isamarkupfalse%
\ m\ n\ q\ {\isacharcolon}{\isacharcolon}\ nat\ \isanewline
\ \ \isacommand{show}\isamarkupfalse%
\ {\isachardoublequoteopen}m\ {\isasymotimes}\ n\ {\isasymotimes}\ q\ {\isacharequal}\ m\ {\isasymotimes}\ {\isacharparenleft}n\ {\isasymotimes}\ q{\isacharparenright}{\isachardoublequoteclose}\isanewline
\ \ \ \ \isacommand{by}\isamarkupfalse%
\ {\isacharparenleft}induct\ m{\isacharparenright}\ auto\isanewline
\isacommand{qed}\isamarkupfalse%
\isanewline
\isanewline
\isacommand{end}\isamarkupfalse%
%
\endisatagquote
{\isafoldquote}%
%
\isadelimquote
%
\endisadelimquote
%
\begin{isamarkuptext}%
\noindent Note the occurence of the name \isa{mult{\isacharunderscore}nat}
  in the primrec declaration;  by default, the local name of
  a class operation \isa{f} to instantiate on type constructor
  \isa{{\isasymkappa}} are mangled as \isa{f{\isacharunderscore}{\isasymkappa}}.  In case of uncertainty,
  these names may be inspected using the \hyperlink{command.print-context}{\mbox{\isa{\isacommand{print{\isacharunderscore}context}}}} command
  or the corresponding ProofGeneral button.%
\end{isamarkuptext}%
\isamarkuptrue%
%
\isamarkupsubsection{Lifting and parametric types%
}
\isamarkuptrue%
%
\begin{isamarkuptext}%
Overloaded definitions giving on class instantiation
  may include recursion over the syntactic structure of types.
  As a canonical example, we model product semigroups
  using our simple algebra:%
\end{isamarkuptext}%
\isamarkuptrue%
%
\isadelimquote
%
\endisadelimquote
%
\isatagquote
\isacommand{instantiation}\isamarkupfalse%
\ {\isacharasterisk}\ {\isacharcolon}{\isacharcolon}\ {\isacharparenleft}semigroup{\isacharcomma}\ semigroup{\isacharparenright}\ semigroup\isanewline
\isakeyword{begin}\isanewline
\isanewline
\isacommand{definition}\isamarkupfalse%
\isanewline
\ \ mult{\isacharunderscore}prod{\isacharunderscore}def{\isacharcolon}\ {\isachardoublequoteopen}p\isactrlisub {\isadigit{1}}\ {\isasymotimes}\ p\isactrlisub {\isadigit{2}}\ {\isacharequal}\ {\isacharparenleft}fst\ p\isactrlisub {\isadigit{1}}\ {\isasymotimes}\ fst\ p\isactrlisub {\isadigit{2}}{\isacharcomma}\ snd\ p\isactrlisub {\isadigit{1}}\ {\isasymotimes}\ snd\ p\isactrlisub {\isadigit{2}}{\isacharparenright}{\isachardoublequoteclose}\isanewline
\isanewline
\isacommand{instance}\isamarkupfalse%
\ \isacommand{proof}\isamarkupfalse%
\isanewline
\ \ \isacommand{fix}\isamarkupfalse%
\ p\isactrlisub {\isadigit{1}}\ p\isactrlisub {\isadigit{2}}\ p\isactrlisub {\isadigit{3}}\ {\isacharcolon}{\isacharcolon}\ {\isachardoublequoteopen}{\isasymalpha}{\isasymColon}semigroup\ {\isasymtimes}\ {\isasymbeta}{\isasymColon}semigroup{\isachardoublequoteclose}\isanewline
\ \ \isacommand{show}\isamarkupfalse%
\ {\isachardoublequoteopen}p\isactrlisub {\isadigit{1}}\ {\isasymotimes}\ p\isactrlisub {\isadigit{2}}\ {\isasymotimes}\ p\isactrlisub {\isadigit{3}}\ {\isacharequal}\ p\isactrlisub {\isadigit{1}}\ {\isasymotimes}\ {\isacharparenleft}p\isactrlisub {\isadigit{2}}\ {\isasymotimes}\ p\isactrlisub {\isadigit{3}}{\isacharparenright}{\isachardoublequoteclose}\isanewline
\ \ \ \ \isacommand{unfolding}\isamarkupfalse%
\ mult{\isacharunderscore}prod{\isacharunderscore}def\ \isacommand{by}\isamarkupfalse%
\ {\isacharparenleft}simp\ add{\isacharcolon}\ assoc{\isacharparenright}\isanewline
\isacommand{qed}\isamarkupfalse%
\ \ \ \ \ \ \isanewline
\isanewline
\isacommand{end}\isamarkupfalse%
%
\endisatagquote
{\isafoldquote}%
%
\isadelimquote
%
\endisadelimquote
%
\begin{isamarkuptext}%
\noindent Associativity from product semigroups is
  established using
  the definition of \isa{{\isacharparenleft}{\isasymotimes}{\isacharparenright}} on products and the hypothetical
  associativity of the type components;  these hypotheses
  are facts due to the \isa{semigroup} constraints imposed
  on the type components by the \hyperlink{command.instance}{\mbox{\isa{\isacommand{instance}}}} proposition.
  Indeed, this pattern often occurs with parametric types
  and type classes.%
\end{isamarkuptext}%
\isamarkuptrue%
%
\isamarkupsubsection{Subclassing%
}
\isamarkuptrue%
%
\begin{isamarkuptext}%
We define a subclass \isa{monoidl} (a semigroup with a left-hand neutral)
  by extending \isa{semigroup}
  with one additional parameter \isa{neutral} together
  with its property:%
\end{isamarkuptext}%
\isamarkuptrue%
%
\isadelimquote
%
\endisadelimquote
%
\isatagquote
\isacommand{class}\isamarkupfalse%
\ monoidl\ {\isacharequal}\ semigroup\ {\isacharplus}\isanewline
\ \ \isakeyword{fixes}\ neutral\ {\isacharcolon}{\isacharcolon}\ {\isachardoublequoteopen}{\isasymalpha}{\isachardoublequoteclose}\ {\isacharparenleft}{\isachardoublequoteopen}{\isasymone}{\isachardoublequoteclose}{\isacharparenright}\isanewline
\ \ \isakeyword{assumes}\ neutl{\isacharcolon}\ {\isachardoublequoteopen}{\isasymone}\ {\isasymotimes}\ x\ {\isacharequal}\ x{\isachardoublequoteclose}%
\endisatagquote
{\isafoldquote}%
%
\isadelimquote
%
\endisadelimquote
%
\begin{isamarkuptext}%
\noindent Again, we prove some instances, by
  providing suitable parameter definitions and proofs for the
  additional specifications.  Observe that instantiations
  for types with the same arity may be simultaneous:%
\end{isamarkuptext}%
\isamarkuptrue%
%
\isadelimquote
%
\endisadelimquote
%
\isatagquote
\isacommand{instantiation}\isamarkupfalse%
\ nat\ \isakeyword{and}\ int\ {\isacharcolon}{\isacharcolon}\ monoidl\isanewline
\isakeyword{begin}\isanewline
\isanewline
\isacommand{definition}\isamarkupfalse%
\isanewline
\ \ neutral{\isacharunderscore}nat{\isacharunderscore}def{\isacharcolon}\ {\isachardoublequoteopen}{\isasymone}\ {\isacharequal}\ {\isacharparenleft}{\isadigit{0}}{\isasymColon}nat{\isacharparenright}{\isachardoublequoteclose}\isanewline
\isanewline
\isacommand{definition}\isamarkupfalse%
\isanewline
\ \ neutral{\isacharunderscore}int{\isacharunderscore}def{\isacharcolon}\ {\isachardoublequoteopen}{\isasymone}\ {\isacharequal}\ {\isacharparenleft}{\isadigit{0}}{\isasymColon}int{\isacharparenright}{\isachardoublequoteclose}\isanewline
\isanewline
\isacommand{instance}\isamarkupfalse%
\ \isacommand{proof}\isamarkupfalse%
\isanewline
\ \ \isacommand{fix}\isamarkupfalse%
\ n\ {\isacharcolon}{\isacharcolon}\ nat\isanewline
\ \ \isacommand{show}\isamarkupfalse%
\ {\isachardoublequoteopen}{\isasymone}\ {\isasymotimes}\ n\ {\isacharequal}\ n{\isachardoublequoteclose}\isanewline
\ \ \ \ \isacommand{unfolding}\isamarkupfalse%
\ neutral{\isacharunderscore}nat{\isacharunderscore}def\ \isacommand{by}\isamarkupfalse%
\ simp\isanewline
\isacommand{next}\isamarkupfalse%
\isanewline
\ \ \isacommand{fix}\isamarkupfalse%
\ k\ {\isacharcolon}{\isacharcolon}\ int\isanewline
\ \ \isacommand{show}\isamarkupfalse%
\ {\isachardoublequoteopen}{\isasymone}\ {\isasymotimes}\ k\ {\isacharequal}\ k{\isachardoublequoteclose}\isanewline
\ \ \ \ \isacommand{unfolding}\isamarkupfalse%
\ neutral{\isacharunderscore}int{\isacharunderscore}def\ mult{\isacharunderscore}int{\isacharunderscore}def\ \isacommand{by}\isamarkupfalse%
\ simp\isanewline
\isacommand{qed}\isamarkupfalse%
\isanewline
\isanewline
\isacommand{end}\isamarkupfalse%
\isanewline
\isanewline
\isacommand{instantiation}\isamarkupfalse%
\ {\isacharasterisk}\ {\isacharcolon}{\isacharcolon}\ {\isacharparenleft}monoidl{\isacharcomma}\ monoidl{\isacharparenright}\ monoidl\isanewline
\isakeyword{begin}\isanewline
\isanewline
\isacommand{definition}\isamarkupfalse%
\isanewline
\ \ neutral{\isacharunderscore}prod{\isacharunderscore}def{\isacharcolon}\ {\isachardoublequoteopen}{\isasymone}\ {\isacharequal}\ {\isacharparenleft}{\isasymone}{\isacharcomma}\ {\isasymone}{\isacharparenright}{\isachardoublequoteclose}\isanewline
\isanewline
\isacommand{instance}\isamarkupfalse%
\ \isacommand{proof}\isamarkupfalse%
\isanewline
\ \ \isacommand{fix}\isamarkupfalse%
\ p\ {\isacharcolon}{\isacharcolon}\ {\isachardoublequoteopen}{\isasymalpha}{\isasymColon}monoidl\ {\isasymtimes}\ {\isasymbeta}{\isasymColon}monoidl{\isachardoublequoteclose}\isanewline
\ \ \isacommand{show}\isamarkupfalse%
\ {\isachardoublequoteopen}{\isasymone}\ {\isasymotimes}\ p\ {\isacharequal}\ p{\isachardoublequoteclose}\isanewline
\ \ \ \ \isacommand{unfolding}\isamarkupfalse%
\ neutral{\isacharunderscore}prod{\isacharunderscore}def\ mult{\isacharunderscore}prod{\isacharunderscore}def\ \isacommand{by}\isamarkupfalse%
\ {\isacharparenleft}simp\ add{\isacharcolon}\ neutl{\isacharparenright}\isanewline
\isacommand{qed}\isamarkupfalse%
\isanewline
\isanewline
\isacommand{end}\isamarkupfalse%
%
\endisatagquote
{\isafoldquote}%
%
\isadelimquote
%
\endisadelimquote
%
\begin{isamarkuptext}%
\noindent Fully-fledged monoids are modelled by another subclass
  which does not add new parameters but tightens the specification:%
\end{isamarkuptext}%
\isamarkuptrue%
%
\isadelimquote
%
\endisadelimquote
%
\isatagquote
\isacommand{class}\isamarkupfalse%
\ monoid\ {\isacharequal}\ monoidl\ {\isacharplus}\isanewline
\ \ \isakeyword{assumes}\ neutr{\isacharcolon}\ {\isachardoublequoteopen}x\ {\isasymotimes}\ {\isasymone}\ {\isacharequal}\ x{\isachardoublequoteclose}\isanewline
\isanewline
\isacommand{instantiation}\isamarkupfalse%
\ nat\ \isakeyword{and}\ int\ {\isacharcolon}{\isacharcolon}\ monoid\ \isanewline
\isakeyword{begin}\isanewline
\isanewline
\isacommand{instance}\isamarkupfalse%
\ \isacommand{proof}\isamarkupfalse%
\isanewline
\ \ \isacommand{fix}\isamarkupfalse%
\ n\ {\isacharcolon}{\isacharcolon}\ nat\isanewline
\ \ \isacommand{show}\isamarkupfalse%
\ {\isachardoublequoteopen}n\ {\isasymotimes}\ {\isasymone}\ {\isacharequal}\ n{\isachardoublequoteclose}\isanewline
\ \ \ \ \isacommand{unfolding}\isamarkupfalse%
\ neutral{\isacharunderscore}nat{\isacharunderscore}def\ \isacommand{by}\isamarkupfalse%
\ {\isacharparenleft}induct\ n{\isacharparenright}\ simp{\isacharunderscore}all\isanewline
\isacommand{next}\isamarkupfalse%
\isanewline
\ \ \isacommand{fix}\isamarkupfalse%
\ k\ {\isacharcolon}{\isacharcolon}\ int\isanewline
\ \ \isacommand{show}\isamarkupfalse%
\ {\isachardoublequoteopen}k\ {\isasymotimes}\ {\isasymone}\ {\isacharequal}\ k{\isachardoublequoteclose}\isanewline
\ \ \ \ \isacommand{unfolding}\isamarkupfalse%
\ neutral{\isacharunderscore}int{\isacharunderscore}def\ mult{\isacharunderscore}int{\isacharunderscore}def\ \isacommand{by}\isamarkupfalse%
\ simp\isanewline
\isacommand{qed}\isamarkupfalse%
\isanewline
\isanewline
\isacommand{end}\isamarkupfalse%
\isanewline
\isanewline
\isacommand{instantiation}\isamarkupfalse%
\ {\isacharasterisk}\ {\isacharcolon}{\isacharcolon}\ {\isacharparenleft}monoid{\isacharcomma}\ monoid{\isacharparenright}\ monoid\isanewline
\isakeyword{begin}\isanewline
\isanewline
\isacommand{instance}\isamarkupfalse%
\ \isacommand{proof}\isamarkupfalse%
\ \isanewline
\ \ \isacommand{fix}\isamarkupfalse%
\ p\ {\isacharcolon}{\isacharcolon}\ {\isachardoublequoteopen}{\isasymalpha}{\isasymColon}monoid\ {\isasymtimes}\ {\isasymbeta}{\isasymColon}monoid{\isachardoublequoteclose}\isanewline
\ \ \isacommand{show}\isamarkupfalse%
\ {\isachardoublequoteopen}p\ {\isasymotimes}\ {\isasymone}\ {\isacharequal}\ p{\isachardoublequoteclose}\isanewline
\ \ \ \ \isacommand{unfolding}\isamarkupfalse%
\ neutral{\isacharunderscore}prod{\isacharunderscore}def\ mult{\isacharunderscore}prod{\isacharunderscore}def\ \isacommand{by}\isamarkupfalse%
\ {\isacharparenleft}simp\ add{\isacharcolon}\ neutr{\isacharparenright}\isanewline
\isacommand{qed}\isamarkupfalse%
\isanewline
\isanewline
\isacommand{end}\isamarkupfalse%
%
\endisatagquote
{\isafoldquote}%
%
\isadelimquote
%
\endisadelimquote
%
\begin{isamarkuptext}%
\noindent To finish our small algebra example, we add a \isa{group} class
  with a corresponding instance:%
\end{isamarkuptext}%
\isamarkuptrue%
%
\isadelimquote
%
\endisadelimquote
%
\isatagquote
\isacommand{class}\isamarkupfalse%
\ group\ {\isacharequal}\ monoidl\ {\isacharplus}\isanewline
\ \ \isakeyword{fixes}\ inverse\ {\isacharcolon}{\isacharcolon}\ {\isachardoublequoteopen}{\isasymalpha}\ {\isasymRightarrow}\ {\isasymalpha}{\isachardoublequoteclose}\ \ \ \ {\isacharparenleft}{\isachardoublequoteopen}{\isacharparenleft}{\isacharunderscore}{\isasymdiv}{\isacharparenright}{\isachardoublequoteclose}\ {\isacharbrackleft}{\isadigit{1}}{\isadigit{0}}{\isadigit{0}}{\isadigit{0}}{\isacharbrackright}\ {\isadigit{9}}{\isadigit{9}}{\isadigit{9}}{\isacharparenright}\isanewline
\ \ \isakeyword{assumes}\ invl{\isacharcolon}\ {\isachardoublequoteopen}x{\isasymdiv}\ {\isasymotimes}\ x\ {\isacharequal}\ {\isasymone}{\isachardoublequoteclose}\isanewline
\isanewline
\isacommand{instantiation}\isamarkupfalse%
\ int\ {\isacharcolon}{\isacharcolon}\ group\isanewline
\isakeyword{begin}\isanewline
\isanewline
\isacommand{definition}\isamarkupfalse%
\isanewline
\ \ inverse{\isacharunderscore}int{\isacharunderscore}def{\isacharcolon}\ {\isachardoublequoteopen}i{\isasymdiv}\ {\isacharequal}\ {\isacharminus}\ {\isacharparenleft}i{\isasymColon}int{\isacharparenright}{\isachardoublequoteclose}\isanewline
\isanewline
\isacommand{instance}\isamarkupfalse%
\ \isacommand{proof}\isamarkupfalse%
\isanewline
\ \ \isacommand{fix}\isamarkupfalse%
\ i\ {\isacharcolon}{\isacharcolon}\ int\isanewline
\ \ \isacommand{have}\isamarkupfalse%
\ {\isachardoublequoteopen}{\isacharminus}i\ {\isacharplus}\ i\ {\isacharequal}\ {\isadigit{0}}{\isachardoublequoteclose}\ \isacommand{by}\isamarkupfalse%
\ simp\isanewline
\ \ \isacommand{then}\isamarkupfalse%
\ \isacommand{show}\isamarkupfalse%
\ {\isachardoublequoteopen}i{\isasymdiv}\ {\isasymotimes}\ i\ {\isacharequal}\ {\isasymone}{\isachardoublequoteclose}\isanewline
\ \ \ \ \isacommand{unfolding}\isamarkupfalse%
\ mult{\isacharunderscore}int{\isacharunderscore}def\ neutral{\isacharunderscore}int{\isacharunderscore}def\ inverse{\isacharunderscore}int{\isacharunderscore}def\ \isacommand{{\isachardot}}\isamarkupfalse%
\isanewline
\isacommand{qed}\isamarkupfalse%
\isanewline
\isanewline
\isacommand{end}\isamarkupfalse%
%
\endisatagquote
{\isafoldquote}%
%
\isadelimquote
%
\endisadelimquote
%
\isamarkupsection{Type classes as locales%
}
\isamarkuptrue%
%
\isamarkupsubsection{A look behind the scene%
}
\isamarkuptrue%
%
\begin{isamarkuptext}%
The example above gives an impression how Isar type classes work
  in practice.  As stated in the introduction, classes also provide
  a link to Isar's locale system.  Indeed, the logical core of a class
  is nothing else than a locale:%
\end{isamarkuptext}%
\isamarkuptrue%
%
\isadelimquote
%
\endisadelimquote
%
\isatagquote
\isacommand{class}\isamarkupfalse%
\ idem\ {\isacharequal}\ type\ {\isacharplus}\isanewline
\ \ \isakeyword{fixes}\ f\ {\isacharcolon}{\isacharcolon}\ {\isachardoublequoteopen}{\isasymalpha}\ {\isasymRightarrow}\ {\isasymalpha}{\isachardoublequoteclose}\isanewline
\ \ \isakeyword{assumes}\ idem{\isacharcolon}\ {\isachardoublequoteopen}f\ {\isacharparenleft}f\ x{\isacharparenright}\ {\isacharequal}\ f\ x{\isachardoublequoteclose}%
\endisatagquote
{\isafoldquote}%
%
\isadelimquote
%
\endisadelimquote
%
\begin{isamarkuptext}%
\noindent essentially introduces the locale%
\end{isamarkuptext}%
\isamarkuptrue%
%
\isadeliminvisible
\ %
\endisadeliminvisible
%
\isataginvisible
\isacommand{setup}\isamarkupfalse%
\ {\isacharverbatimopen}\ Sign{\isachardot}add{\isacharunderscore}path\ {\isachardoublequote}foo{\isachardoublequote}\ {\isacharverbatimclose}%
\endisataginvisible
{\isafoldinvisible}%
%
\isadeliminvisible
%
\endisadeliminvisible
\isanewline
%
\isadelimquote
\isanewline
%
\endisadelimquote
%
\isatagquote
\isacommand{locale}\isamarkupfalse%
\ idem\ {\isacharequal}\isanewline
\ \ \isakeyword{fixes}\ f\ {\isacharcolon}{\isacharcolon}\ {\isachardoublequoteopen}{\isasymalpha}\ {\isasymRightarrow}\ {\isasymalpha}{\isachardoublequoteclose}\isanewline
\ \ \isakeyword{assumes}\ idem{\isacharcolon}\ {\isachardoublequoteopen}f\ {\isacharparenleft}f\ x{\isacharparenright}\ {\isacharequal}\ f\ x{\isachardoublequoteclose}%
\endisatagquote
{\isafoldquote}%
%
\isadelimquote
%
\endisadelimquote
%
\begin{isamarkuptext}%
\noindent together with corresponding constant(s):%
\end{isamarkuptext}%
\isamarkuptrue%
%
\isadelimquote
%
\endisadelimquote
%
\isatagquote
\isacommand{consts}\isamarkupfalse%
\ f\ {\isacharcolon}{\isacharcolon}\ {\isachardoublequoteopen}{\isasymalpha}\ {\isasymRightarrow}\ {\isasymalpha}{\isachardoublequoteclose}%
\endisatagquote
{\isafoldquote}%
%
\isadelimquote
%
\endisadelimquote
%
\begin{isamarkuptext}%
\noindent The connection to the type system is done by means
  of a primitive axclass%
\end{isamarkuptext}%
\isamarkuptrue%
%
\isadelimquote
%
\endisadelimquote
%
\isatagquote
\isacommand{axclass}\isamarkupfalse%
\ idem\ {\isacharless}\ type\isanewline
\ \ idem{\isacharcolon}\ {\isachardoublequoteopen}f\ {\isacharparenleft}f\ x{\isacharparenright}\ {\isacharequal}\ f\ x{\isachardoublequoteclose}%
\endisatagquote
{\isafoldquote}%
%
\isadelimquote
%
\endisadelimquote
%
\begin{isamarkuptext}%
\noindent together with a corresponding interpretation:%
\end{isamarkuptext}%
\isamarkuptrue%
%
\isadelimquote
%
\endisadelimquote
%
\isatagquote
\isacommand{interpretation}\isamarkupfalse%
\ idem{\isacharunderscore}class{\isacharcolon}\isanewline
\ \ idem\ {\isacharbrackleft}{\isachardoublequoteopen}f\ {\isasymColon}\ {\isacharparenleft}{\isasymalpha}{\isasymColon}idem{\isacharparenright}\ {\isasymRightarrow}\ {\isasymalpha}{\isachardoublequoteclose}{\isacharbrackright}\isanewline
\isacommand{by}\isamarkupfalse%
\ unfold{\isacharunderscore}locales\ {\isacharparenleft}rule\ idem{\isacharparenright}%
\endisatagquote
{\isafoldquote}%
%
\isadelimquote
%
\endisadelimquote
%
\begin{isamarkuptext}%
\noindent This gives you at hand the full power of the Isabelle module system;
  conclusions in locale \isa{idem} are implicitly propagated
  to class \isa{idem}.%
\end{isamarkuptext}%
\isamarkuptrue%
%
\isadeliminvisible
\ %
\endisadeliminvisible
%
\isataginvisible
\isacommand{setup}\isamarkupfalse%
\ {\isacharverbatimopen}\ Sign{\isachardot}parent{\isacharunderscore}path\ {\isacharverbatimclose}%
\endisataginvisible
{\isafoldinvisible}%
%
\isadeliminvisible
%
\endisadeliminvisible
%
\isamarkupsubsection{Abstract reasoning%
}
\isamarkuptrue%
%
\begin{isamarkuptext}%
Isabelle locales enable reasoning at a general level, while results
  are implicitly transferred to all instances.  For example, we can
  now establish the \isa{left{\isacharunderscore}cancel} lemma for groups, which
  states that the function \isa{{\isacharparenleft}x\ {\isasymotimes}{\isacharparenright}} is injective:%
\end{isamarkuptext}%
\isamarkuptrue%
%
\isadelimquote
%
\endisadelimquote
%
\isatagquote
\isacommand{lemma}\isamarkupfalse%
\ {\isacharparenleft}\isakeyword{in}\ group{\isacharparenright}\ left{\isacharunderscore}cancel{\isacharcolon}\ {\isachardoublequoteopen}x\ {\isasymotimes}\ y\ {\isacharequal}\ x\ {\isasymotimes}\ z\ {\isasymlongleftrightarrow}\ y\ {\isacharequal}\ z{\isachardoublequoteclose}\isanewline
\isacommand{proof}\isamarkupfalse%
\isanewline
\ \ \isacommand{assume}\isamarkupfalse%
\ {\isachardoublequoteopen}x\ {\isasymotimes}\ y\ {\isacharequal}\ x\ {\isasymotimes}\ z{\isachardoublequoteclose}\isanewline
\ \ \isacommand{then}\isamarkupfalse%
\ \isacommand{have}\isamarkupfalse%
\ {\isachardoublequoteopen}x{\isasymdiv}\ {\isasymotimes}\ {\isacharparenleft}x\ {\isasymotimes}\ y{\isacharparenright}\ {\isacharequal}\ x{\isasymdiv}\ {\isasymotimes}\ {\isacharparenleft}x\ {\isasymotimes}\ z{\isacharparenright}{\isachardoublequoteclose}\ \isacommand{by}\isamarkupfalse%
\ simp\isanewline
\ \ \isacommand{then}\isamarkupfalse%
\ \isacommand{have}\isamarkupfalse%
\ {\isachardoublequoteopen}{\isacharparenleft}x{\isasymdiv}\ {\isasymotimes}\ x{\isacharparenright}\ {\isasymotimes}\ y\ {\isacharequal}\ {\isacharparenleft}x{\isasymdiv}\ {\isasymotimes}\ x{\isacharparenright}\ {\isasymotimes}\ z{\isachardoublequoteclose}\ \isacommand{using}\isamarkupfalse%
\ assoc\ \isacommand{by}\isamarkupfalse%
\ simp\isanewline
\ \ \isacommand{then}\isamarkupfalse%
\ \isacommand{show}\isamarkupfalse%
\ {\isachardoublequoteopen}y\ {\isacharequal}\ z{\isachardoublequoteclose}\ \isacommand{using}\isamarkupfalse%
\ neutl\ \isakeyword{and}\ invl\ \isacommand{by}\isamarkupfalse%
\ simp\isanewline
\isacommand{next}\isamarkupfalse%
\isanewline
\ \ \isacommand{assume}\isamarkupfalse%
\ {\isachardoublequoteopen}y\ {\isacharequal}\ z{\isachardoublequoteclose}\isanewline
\ \ \isacommand{then}\isamarkupfalse%
\ \isacommand{show}\isamarkupfalse%
\ {\isachardoublequoteopen}x\ {\isasymotimes}\ y\ {\isacharequal}\ x\ {\isasymotimes}\ z{\isachardoublequoteclose}\ \isacommand{by}\isamarkupfalse%
\ simp\isanewline
\isacommand{qed}\isamarkupfalse%
%
\endisatagquote
{\isafoldquote}%
%
\isadelimquote
%
\endisadelimquote
%
\begin{isamarkuptext}%
\noindent Here the \qt{\hyperlink{keyword.in}{\mbox{\isa{\isakeyword{in}}}} \isa{group}} target specification
  indicates that the result is recorded within that context for later
  use.  This local theorem is also lifted to the global one \hyperlink{fact.group.left-cancel:}{\mbox{\isa{group{\isachardot}left{\isacharunderscore}cancel{\isacharcolon}}}} \isa{{\isachardoublequote}{\isasymAnd}x\ y\ z\ {\isasymColon}\ {\isasymalpha}{\isasymColon}group{\isachardot}\ x\ {\isasymotimes}\ y\ {\isacharequal}\ x\ {\isasymotimes}\ z\ {\isasymlongleftrightarrow}\ y\ {\isacharequal}\ z{\isachardoublequote}}.  Since type \isa{int} has been made an instance of
  \isa{group} before, we may refer to that fact as well: \isa{{\isachardoublequote}{\isasymAnd}x\ y\ z\ {\isasymColon}\ int{\isachardot}\ x\ {\isasymotimes}\ y\ {\isacharequal}\ x\ {\isasymotimes}\ z\ {\isasymlongleftrightarrow}\ y\ {\isacharequal}\ z{\isachardoublequote}}.%
\end{isamarkuptext}%
\isamarkuptrue%
%
\isamarkupsubsection{Derived definitions%
}
\isamarkuptrue%
%
\begin{isamarkuptext}%
Isabelle locales support a concept of local definitions
  in locales:%
\end{isamarkuptext}%
\isamarkuptrue%
%
\isadelimquote
%
\endisadelimquote
%
\isatagquote
\isacommand{primrec}\isamarkupfalse%
\ {\isacharparenleft}\isakeyword{in}\ monoid{\isacharparenright}\ pow{\isacharunderscore}nat\ {\isacharcolon}{\isacharcolon}\ {\isachardoublequoteopen}nat\ {\isasymRightarrow}\ {\isasymalpha}\ {\isasymRightarrow}\ {\isasymalpha}{\isachardoublequoteclose}\ \isakeyword{where}\isanewline
\ \ {\isachardoublequoteopen}pow{\isacharunderscore}nat\ {\isadigit{0}}\ x\ {\isacharequal}\ {\isasymone}{\isachardoublequoteclose}\isanewline
\ \ {\isacharbar}\ {\isachardoublequoteopen}pow{\isacharunderscore}nat\ {\isacharparenleft}Suc\ n{\isacharparenright}\ x\ {\isacharequal}\ x\ {\isasymotimes}\ pow{\isacharunderscore}nat\ n\ x{\isachardoublequoteclose}%
\endisatagquote
{\isafoldquote}%
%
\isadelimquote
%
\endisadelimquote
%
\begin{isamarkuptext}%
\noindent If the locale \isa{group} is also a class, this local
  definition is propagated onto a global definition of
  \isa{{\isachardoublequote}pow{\isacharunderscore}nat\ {\isasymColon}\ nat\ {\isasymRightarrow}\ {\isasymalpha}{\isasymColon}monoid\ {\isasymRightarrow}\ {\isasymalpha}{\isasymColon}monoid{\isachardoublequote}}
  with corresponding theorems

  \isa{pow{\isacharunderscore}nat\ {\isadigit{0}}\ x\ {\isacharequal}\ {\isasymone}\isasep\isanewline%
pow{\isacharunderscore}nat\ {\isacharparenleft}Suc\ n{\isacharparenright}\ x\ {\isacharequal}\ x\ {\isasymotimes}\ pow{\isacharunderscore}nat\ n\ x}.

  \noindent As you can see from this example, for local
  definitions you may use any specification tool
  which works together with locales (e.g. \cite{krauss2006}).%
\end{isamarkuptext}%
\isamarkuptrue%
%
\isamarkupsubsection{A functor analogy%
}
\isamarkuptrue%
%
\begin{isamarkuptext}%
We introduced Isar classes by analogy to type classes
  functional programming;  if we reconsider this in the
  context of what has been said about type classes and locales,
  we can drive this analogy further by stating that type
  classes essentially correspond to functors which have
  a canonical interpretation as type classes.
  Anyway, there is also the possibility of other interpretations.
  For example, also \isa{list}s form a monoid with
  \isa{append} and \isa{{\isacharbrackleft}{\isacharbrackright}} as operations, but it
  seems inappropriate to apply to lists
  the same operations as for genuinely algebraic types.
  In such a case, we simply can do a particular interpretation
  of monoids for lists:%
\end{isamarkuptext}%
\isamarkuptrue%
%
\isadelimquote
%
\endisadelimquote
%
\isatagquote
\isacommand{interpretation}\isamarkupfalse%
\ list{\isacharunderscore}monoid{\isacharcolon}\ monoid\ {\isacharbrackleft}append\ {\isachardoublequoteopen}{\isacharbrackleft}{\isacharbrackright}{\isachardoublequoteclose}{\isacharbrackright}\isanewline
\ \ \isacommand{by}\isamarkupfalse%
\ unfold{\isacharunderscore}locales\ auto%
\endisatagquote
{\isafoldquote}%
%
\isadelimquote
%
\endisadelimquote
%
\begin{isamarkuptext}%
\noindent This enables us to apply facts on monoids
  to lists, e.g. \isa{{\isacharbrackleft}{\isacharbrackright}\ {\isacharat}\ x\ {\isacharequal}\ x}.

  When using this interpretation pattern, it may also
  be appropriate to map derived definitions accordingly:%
\end{isamarkuptext}%
\isamarkuptrue%
%
\isadelimquote
%
\endisadelimquote
%
\isatagquote
\isacommand{primrec}\isamarkupfalse%
\ replicate\ {\isacharcolon}{\isacharcolon}\ {\isachardoublequoteopen}nat\ {\isasymRightarrow}\ {\isasymalpha}\ list\ {\isasymRightarrow}\ {\isasymalpha}\ list{\isachardoublequoteclose}\ \isakeyword{where}\isanewline
\ \ {\isachardoublequoteopen}replicate\ {\isadigit{0}}\ {\isacharunderscore}\ {\isacharequal}\ {\isacharbrackleft}{\isacharbrackright}{\isachardoublequoteclose}\isanewline
\ \ {\isacharbar}\ {\isachardoublequoteopen}replicate\ {\isacharparenleft}Suc\ n{\isacharparenright}\ xs\ {\isacharequal}\ xs\ {\isacharat}\ replicate\ n\ xs{\isachardoublequoteclose}\isanewline
\isanewline
\isacommand{interpretation}\isamarkupfalse%
\ list{\isacharunderscore}monoid{\isacharcolon}\ monoid\ {\isacharbrackleft}append\ {\isachardoublequoteopen}{\isacharbrackleft}{\isacharbrackright}{\isachardoublequoteclose}{\isacharbrackright}\ \isakeyword{where}\isanewline
\ \ {\isachardoublequoteopen}monoid{\isachardot}pow{\isacharunderscore}nat\ append\ {\isacharbrackleft}{\isacharbrackright}\ {\isacharequal}\ replicate{\isachardoublequoteclose}\isanewline
\isacommand{proof}\isamarkupfalse%
\ {\isacharminus}\isanewline
\ \ \isacommand{interpret}\isamarkupfalse%
\ monoid\ {\isacharbrackleft}append\ {\isachardoublequoteopen}{\isacharbrackleft}{\isacharbrackright}{\isachardoublequoteclose}{\isacharbrackright}\ \isacommand{{\isachardot}{\isachardot}}\isamarkupfalse%
\isanewline
\ \ \isacommand{show}\isamarkupfalse%
\ {\isachardoublequoteopen}monoid{\isachardot}pow{\isacharunderscore}nat\ append\ {\isacharbrackleft}{\isacharbrackright}\ {\isacharequal}\ replicate{\isachardoublequoteclose}\isanewline
\ \ \isacommand{proof}\isamarkupfalse%
\isanewline
\ \ \ \ \isacommand{fix}\isamarkupfalse%
\ n\isanewline
\ \ \ \ \isacommand{show}\isamarkupfalse%
\ {\isachardoublequoteopen}monoid{\isachardot}pow{\isacharunderscore}nat\ append\ {\isacharbrackleft}{\isacharbrackright}\ n\ {\isacharequal}\ replicate\ n{\isachardoublequoteclose}\isanewline
\ \ \ \ \ \ \isacommand{by}\isamarkupfalse%
\ {\isacharparenleft}induct\ n{\isacharparenright}\ auto\isanewline
\ \ \isacommand{qed}\isamarkupfalse%
\isanewline
\isacommand{qed}\isamarkupfalse%
\ intro{\isacharunderscore}locales%
\endisatagquote
{\isafoldquote}%
%
\isadelimquote
%
\endisadelimquote
%
\isamarkupsubsection{Additional subclass relations%
}
\isamarkuptrue%
%
\begin{isamarkuptext}%
Any \isa{group} is also a \isa{monoid};  this
  can be made explicit by claiming an additional
  subclass relation,
  together with a proof of the logical difference:%
\end{isamarkuptext}%
\isamarkuptrue%
%
\isadelimquote
%
\endisadelimquote
%
\isatagquote
\isacommand{subclass}\isamarkupfalse%
\ {\isacharparenleft}\isakeyword{in}\ group{\isacharparenright}\ monoid\isanewline
\isacommand{proof}\isamarkupfalse%
\ unfold{\isacharunderscore}locales\isanewline
\ \ \isacommand{fix}\isamarkupfalse%
\ x\isanewline
\ \ \isacommand{from}\isamarkupfalse%
\ invl\ \isacommand{have}\isamarkupfalse%
\ {\isachardoublequoteopen}x{\isasymdiv}\ {\isasymotimes}\ x\ {\isacharequal}\ {\isasymone}{\isachardoublequoteclose}\ \isacommand{by}\isamarkupfalse%
\ simp\isanewline
\ \ \isacommand{with}\isamarkupfalse%
\ assoc\ {\isacharbrackleft}symmetric{\isacharbrackright}\ neutl\ invl\ \isacommand{have}\isamarkupfalse%
\ {\isachardoublequoteopen}x{\isasymdiv}\ {\isasymotimes}\ {\isacharparenleft}x\ {\isasymotimes}\ {\isasymone}{\isacharparenright}\ {\isacharequal}\ x{\isasymdiv}\ {\isasymotimes}\ x{\isachardoublequoteclose}\ \isacommand{by}\isamarkupfalse%
\ simp\isanewline
\ \ \isacommand{with}\isamarkupfalse%
\ left{\isacharunderscore}cancel\ \isacommand{show}\isamarkupfalse%
\ {\isachardoublequoteopen}x\ {\isasymotimes}\ {\isasymone}\ {\isacharequal}\ x{\isachardoublequoteclose}\ \isacommand{by}\isamarkupfalse%
\ simp\isanewline
\isacommand{qed}\isamarkupfalse%
%
\endisatagquote
{\isafoldquote}%
%
\isadelimquote
%
\endisadelimquote
%
\begin{isamarkuptext}%
\noindent The logical proof is carried out on the locale level
  and thus conveniently is opened using the \isa{unfold{\isacharunderscore}locales}
  method which only leaves the logical differences still
  open to proof to the user.  Afterwards it is propagated
  to the type system, making \isa{group} an instance of
  \isa{monoid} by adding an additional edge
  to the graph of subclass relations
  (cf.\ \figref{fig:subclass}).

  \begin{figure}[htbp]
   \begin{center}
     \small
     \unitlength 0.6mm
     \begin{picture}(40,60)(0,0)
       \put(20,60){\makebox(0,0){\isa{semigroup}}}
       \put(20,40){\makebox(0,0){\isa{monoidl}}}
       \put(00,20){\makebox(0,0){\isa{monoid}}}
       \put(40,00){\makebox(0,0){\isa{group}}}
       \put(20,55){\vector(0,-1){10}}
       \put(15,35){\vector(-1,-1){10}}
       \put(25,35){\vector(1,-3){10}}
     \end{picture}
     \hspace{8em}
     \begin{picture}(40,60)(0,0)
       \put(20,60){\makebox(0,0){\isa{semigroup}}}
       \put(20,40){\makebox(0,0){\isa{monoidl}}}
       \put(00,20){\makebox(0,0){\isa{monoid}}}
       \put(40,00){\makebox(0,0){\isa{group}}}
       \put(20,55){\vector(0,-1){10}}
       \put(15,35){\vector(-1,-1){10}}
       \put(05,15){\vector(3,-1){30}}
     \end{picture}
     \caption{Subclass relationship of monoids and groups:
        before and after establishing the relationship
        \isa{group\ {\isasymsubseteq}\ monoid};  transitive edges left out.}
     \label{fig:subclass}
   \end{center}
  \end{figure}

  For illustration, a derived definition
  in \isa{group} which uses \isa{pow{\isacharunderscore}nat}:%
\end{isamarkuptext}%
\isamarkuptrue%
%
\isadelimquote
%
\endisadelimquote
%
\isatagquote
\isacommand{definition}\isamarkupfalse%
\ {\isacharparenleft}\isakeyword{in}\ group{\isacharparenright}\ pow{\isacharunderscore}int\ {\isacharcolon}{\isacharcolon}\ {\isachardoublequoteopen}int\ {\isasymRightarrow}\ {\isasymalpha}\ {\isasymRightarrow}\ {\isasymalpha}{\isachardoublequoteclose}\ \isakeyword{where}\isanewline
\ \ {\isachardoublequoteopen}pow{\isacharunderscore}int\ k\ x\ {\isacharequal}\ {\isacharparenleft}if\ k\ {\isachargreater}{\isacharequal}\ {\isadigit{0}}\isanewline
\ \ \ \ then\ pow{\isacharunderscore}nat\ {\isacharparenleft}nat\ k{\isacharparenright}\ x\isanewline
\ \ \ \ else\ {\isacharparenleft}pow{\isacharunderscore}nat\ {\isacharparenleft}nat\ {\isacharparenleft}{\isacharminus}\ k{\isacharparenright}{\isacharparenright}\ x{\isacharparenright}{\isasymdiv}{\isacharparenright}{\isachardoublequoteclose}%
\endisatagquote
{\isafoldquote}%
%
\isadelimquote
%
\endisadelimquote
%
\begin{isamarkuptext}%
\noindent yields the global definition of
  \isa{{\isachardoublequote}pow{\isacharunderscore}int\ {\isasymColon}\ int\ {\isasymRightarrow}\ {\isasymalpha}{\isasymColon}group\ {\isasymRightarrow}\ {\isasymalpha}{\isasymColon}group{\isachardoublequote}}
  with the corresponding theorem \isa{pow{\isacharunderscore}int\ k\ x\ {\isacharequal}\ {\isacharparenleft}if\ {\isadigit{0}}\ {\isasymle}\ k\ then\ pow{\isacharunderscore}nat\ {\isacharparenleft}nat\ k{\isacharparenright}\ x\ else\ {\isacharparenleft}pow{\isacharunderscore}nat\ {\isacharparenleft}nat\ {\isacharparenleft}{\isacharminus}\ k{\isacharparenright}{\isacharparenright}\ x{\isacharparenright}{\isasymdiv}{\isacharparenright}}.%
\end{isamarkuptext}%
\isamarkuptrue%
%
\isamarkupsubsection{A note on syntax%
}
\isamarkuptrue%
%
\begin{isamarkuptext}%
As a commodity, class context syntax allows to refer
  to local class operations and their global counterparts
  uniformly;  type inference resolves ambiguities.  For example:%
\end{isamarkuptext}%
\isamarkuptrue%
%
\isadelimquote
%
\endisadelimquote
%
\isatagquote
\isacommand{context}\isamarkupfalse%
\ semigroup\isanewline
\isakeyword{begin}\isanewline
\isanewline
\isacommand{term}\isamarkupfalse%
\ {\isachardoublequoteopen}x\ {\isasymotimes}\ y{\isachardoublequoteclose}\ %
\isamarkupcmt{example 1%
}
\isanewline
\isacommand{term}\isamarkupfalse%
\ {\isachardoublequoteopen}{\isacharparenleft}x{\isasymColon}nat{\isacharparenright}\ {\isasymotimes}\ y{\isachardoublequoteclose}\ %
\isamarkupcmt{example 2%
}
\isanewline
\isanewline
\isacommand{end}\isamarkupfalse%
\isanewline
\isanewline
\isacommand{term}\isamarkupfalse%
\ {\isachardoublequoteopen}x\ {\isasymotimes}\ y{\isachardoublequoteclose}\ %
\isamarkupcmt{example 3%
}
%
\endisatagquote
{\isafoldquote}%
%
\isadelimquote
%
\endisadelimquote
%
\begin{isamarkuptext}%
\noindent Here in example 1, the term refers to the local class operation
  \isa{mult\ {\isacharbrackleft}{\isasymalpha}{\isacharbrackright}}, whereas in example 2 the type constraint
  enforces the global class operation \isa{mult\ {\isacharbrackleft}nat{\isacharbrackright}}.
  In the global context in example 3, the reference is
  to the polymorphic global class operation \isa{mult\ {\isacharbrackleft}{\isacharquery}{\isasymalpha}\ {\isasymColon}\ semigroup{\isacharbrackright}}.%
\end{isamarkuptext}%
\isamarkuptrue%
%
\isamarkupsection{Type classes and code generation%
}
\isamarkuptrue%
%
\begin{isamarkuptext}%
Turning back to the first motivation for type classes,
  namely overloading, it is obvious that overloading
  stemming from \hyperlink{command.class}{\mbox{\isa{\isacommand{class}}}} statements and
  \hyperlink{command.instantiation}{\mbox{\isa{\isacommand{instantiation}}}}
  targets naturally maps to Haskell type classes.
  The code generator framework \cite{isabelle-codegen} 
  takes this into account.  Concerning target languages
  lacking type classes (e.g.~SML), type classes
  are implemented by explicit dictionary construction.
  As example, let's go back to the power function:%
\end{isamarkuptext}%
\isamarkuptrue%
%
\isadelimquote
%
\endisadelimquote
%
\isatagquote
\isacommand{definition}\isamarkupfalse%
\ example\ {\isacharcolon}{\isacharcolon}\ int\ \isakeyword{where}\isanewline
\ \ {\isachardoublequoteopen}example\ {\isacharequal}\ pow{\isacharunderscore}int\ {\isadigit{1}}{\isadigit{0}}\ {\isacharparenleft}{\isacharminus}{\isadigit{2}}{\isacharparenright}{\isachardoublequoteclose}%
\endisatagquote
{\isafoldquote}%
%
\isadelimquote
%
\endisadelimquote
%
\begin{isamarkuptext}%
\noindent This maps to Haskell as:%
\end{isamarkuptext}%
\isamarkuptrue%
%
\isadelimquote
%
\endisadelimquote
%
\isatagquote
%
\begin{isamarkuptext}%
\isaverbatim%
\noindent%
\verb|module Example where {|\newline%
\newline%
\newline%
\verb|data Nat = Suc Nat |\verb,|,\verb| Zero_nat;|\newline%
\newline%
\verb|nat_aux :: Integer -> Nat -> Nat;|\newline%
\verb|nat_aux i n = (if i <= 0 then n else nat_aux (i - 1) (Suc n));|\newline%
\newline%
\verb|nat :: Integer -> Nat;|\newline%
\verb|nat i = nat_aux i Zero_nat;|\newline%
\newline%
\verb|class Semigroup a where {|\newline%
\verb|  mult :: a -> a -> a;|\newline%
\verb|};|\newline%
\newline%
\verb|class (Semigroup a) => Monoidl a where {|\newline%
\verb|  neutral :: a;|\newline%
\verb|};|\newline%
\newline%
\verb|class (Monoidl a) => Monoid a where {|\newline%
\verb|};|\newline%
\newline%
\verb|class (Monoid a) => Group a where {|\newline%
\verb|  inverse :: a -> a;|\newline%
\verb|};|\newline%
\newline%
\verb|inverse_int :: Integer -> Integer;|\newline%
\verb|inverse_int i = negate i;|\newline%
\newline%
\verb|neutral_int :: Integer;|\newline%
\verb|neutral_int = 0;|\newline%
\newline%
\verb|mult_int :: Integer -> Integer -> Integer;|\newline%
\verb|mult_int i j = i + j;|\newline%
\newline%
\verb|instance Semigroup Integer where {|\newline%
\verb|  mult = mult_int;|\newline%
\verb|};|\newline%
\newline%
\verb|instance Monoidl Integer where {|\newline%
\verb|  neutral = neutral_int;|\newline%
\verb|};|\newline%
\newline%
\verb|instance Monoid Integer where {|\newline%
\verb|};|\newline%
\newline%
\verb|instance Group Integer where {|\newline%
\verb|  inverse = inverse_int;|\newline%
\verb|};|\newline%
\newline%
\verb|pow_nat :: forall a. (Monoid a) => Nat -> a -> a;|\newline%
\verb|pow_nat Zero_nat x = neutral;|\newline%
\verb|pow_nat (Suc n) x = mult x (pow_nat n x);|\newline%
\newline%
\verb|pow_int :: forall a. (Group a) => Integer -> a -> a;|\newline%
\verb|pow_int k x =|\newline%
\verb|  (if 0 <= k then pow_nat (nat k) x|\newline%
\verb|    else inverse (pow_nat (nat (negate k)) x));|\newline%
\newline%
\verb|example :: Integer;|\newline%
\verb|example = pow_int 10 (-2);|\newline%
\newline%
\verb|}|%
\end{isamarkuptext}%
\isamarkuptrue%
%
\endisatagquote
{\isafoldquote}%
%
\isadelimquote
%
\endisadelimquote
%
\begin{isamarkuptext}%
\noindent The whole code in SML with explicit dictionary passing:%
\end{isamarkuptext}%
\isamarkuptrue%
%
\isadelimquote
%
\endisadelimquote
%
\isatagquote
%
\begin{isamarkuptext}%
\isaverbatim%
\noindent%
\verb|structure Example = |\newline%
\verb|struct|\newline%
\newline%
\verb|datatype nat = Suc of nat |\verb,|,\verb| Zero_nat;|\newline%
\newline%
\verb|fun nat_aux i n =|\newline%
\verb|  (if IntInf.<= (i, (0 : IntInf.int)) then n|\newline%
\verb|    else nat_aux (IntInf.- (i, (1 : IntInf.int))) (Suc n));|\newline%
\newline%
\verb|fun nat i = nat_aux i Zero_nat;|\newline%
\newline%
\verb|type 'a semigroup = {mult : 'a -> 'a -> 'a};|\newline%
\verb|fun mult (A_:'a semigroup) = #mult A_;|\newline%
\newline%
\verb|type 'a monoidl =|\newline%
\verb|  {Classes__semigroup_monoidl : 'a semigroup, neutral : 'a};|\newline%
\verb|fun semigroup_monoidl (A_:'a monoidl) = #Classes__semigroup_monoidl A_;|\newline%
\verb|fun neutral (A_:'a monoidl) = #neutral A_;|\newline%
\newline%
\verb|type 'a monoid = {Classes__monoidl_monoid : 'a monoidl};|\newline%
\verb|fun monoidl_monoid (A_:'a monoid) = #Classes__monoidl_monoid A_;|\newline%
\newline%
\verb|type 'a group = {Classes__monoid_group : 'a monoid, inverse : 'a -> 'a};|\newline%
\verb|fun monoid_group (A_:'a group) = #Classes__monoid_group A_;|\newline%
\verb|fun inverse (A_:'a group) = #inverse A_;|\newline%
\newline%
\verb|fun inverse_int i = IntInf.~ i;|\newline%
\newline%
\verb|val neutral_int : IntInf.int = (0 : IntInf.int);|\newline%
\newline%
\verb|fun mult_int i j = IntInf.+ (i, j);|\newline%
\newline%
\verb|val semigroup_int = {mult = mult_int} : IntInf.int semigroup;|\newline%
\newline%
\verb|val monoidl_int =|\newline%
\verb|  {Classes__semigroup_monoidl = semigroup_int, neutral = neutral_int} :|\newline%
\verb|  IntInf.int monoidl;|\newline%
\newline%
\verb|val monoid_int = {Classes__monoidl_monoid = monoidl_int} :|\newline%
\verb|  IntInf.int monoid;|\newline%
\newline%
\verb|val group_int =|\newline%
\verb|  {Classes__monoid_group = monoid_int, inverse = inverse_int} :|\newline%
\verb|  IntInf.int group;|\newline%
\newline%
\verb|fun pow_nat A_ Zero_nat x = neutral (monoidl_monoid A_)|\newline%
\verb|  |\verb,|,\verb| pow_nat A_ (Suc n) x =|\newline%
\verb|    mult ((semigroup_monoidl o monoidl_monoid) A_) x (pow_nat A_ n x);|\newline%
\newline%
\verb|fun pow_int A_ k x =|\newline%
\verb|  (if IntInf.<= ((0 : IntInf.int), k)|\newline%
\verb|    then pow_nat (monoid_group A_) (nat k) x|\newline%
\verb|    else inverse A_ (pow_nat (monoid_group A_) (nat (IntInf.~ k)) x));|\newline%
\newline%
\verb|val example : IntInf.int =|\newline%
\verb|  pow_int group_int (10 : IntInf.int) (~2 : IntInf.int);|\newline%
\newline%
\verb|end; (*struct Example*)|%
\end{isamarkuptext}%
\isamarkuptrue%
%
\endisatagquote
{\isafoldquote}%
%
\isadelimquote
%
\endisadelimquote
%
\isadelimtheory
%
\endisadelimtheory
%
\isatagtheory
\isacommand{end}\isamarkupfalse%
%
\endisatagtheory
{\isafoldtheory}%
%
\isadelimtheory
%
\endisadelimtheory
\isanewline
\end{isabellebody}%
%%% Local Variables:
%%% mode: latex
%%% TeX-master: "root"
%%% End:
