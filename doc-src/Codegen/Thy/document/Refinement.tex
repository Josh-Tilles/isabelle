%
\begin{isabellebody}%
\def\isabellecontext{Refinement}%
%
\isadelimtheory
%
\endisadelimtheory
%
\isatagtheory
\isacommand{theory}\isamarkupfalse%
\ Refinement\isanewline
\isakeyword{imports}\ Setup\isanewline
\isakeyword{begin}%
\endisatagtheory
{\isafoldtheory}%
%
\isadelimtheory
%
\endisadelimtheory
%
\isamarkupsection{Program and datatype refinement \label{sec:refinement}%
}
\isamarkuptrue%
%
\begin{isamarkuptext}%
Code generation by shallow embedding (cf.~\secref{sec:principle})
  allows to choose code equations and datatype constructors freely,
  given that some very basic syntactic properties are met; this
  flexibility opens up mechanisms for refinement which allow to extend
  the scope and quality of generated code dramatically.%
\end{isamarkuptext}%
\isamarkuptrue%
%
\isamarkupsubsection{Program refinement%
}
\isamarkuptrue%
%
\begin{isamarkuptext}%
Program refinement works by choosing appropriate code equations
  explicitly (cf.~\label{sec:equations}); as example, we use Fibonacci
  numbers:%
\end{isamarkuptext}%
\isamarkuptrue%
%
\isadelimquote
%
\endisadelimquote
%
\isatagquote
\isacommand{fun}\isamarkupfalse%
\ fib\ {\isacharcolon}{\isacharcolon}\ {\isachardoublequoteopen}nat\ {\isasymRightarrow}\ nat{\isachardoublequoteclose}\ \isakeyword{where}\isanewline
\ \ \ \ {\isachardoublequoteopen}fib\ {\isadigit{0}}\ {\isacharequal}\ {\isadigit{0}}{\isachardoublequoteclose}\isanewline
\ \ {\isacharbar}\ {\isachardoublequoteopen}fib\ {\isacharparenleft}Suc\ {\isadigit{0}}{\isacharparenright}\ {\isacharequal}\ Suc\ {\isadigit{0}}{\isachardoublequoteclose}\isanewline
\ \ {\isacharbar}\ {\isachardoublequoteopen}fib\ {\isacharparenleft}Suc\ {\isacharparenleft}Suc\ n{\isacharparenright}{\isacharparenright}\ {\isacharequal}\ fib\ n\ {\isacharplus}\ fib\ {\isacharparenleft}Suc\ n{\isacharparenright}{\isachardoublequoteclose}%
\endisatagquote
{\isafoldquote}%
%
\isadelimquote
%
\endisadelimquote
%
\begin{isamarkuptext}%
\noindent The runtime of the corresponding code grows exponential due
  to two recursive calls:%
\end{isamarkuptext}%
\isamarkuptrue%
%
\isadelimquote
%
\endisadelimquote
%
\isatagquote
%
\begin{isamarkuptext}%
\isatypewriter%
\noindent%
\hspace*{0pt}fib ::~Example.Nat -> Example.Nat;\\
\hspace*{0pt}fib Example.Zero{\char95}nat = Example.Zero{\char95}nat;\\
\hspace*{0pt}fib (Example.Suc Example.Zero{\char95}nat) = Example.Suc Example.Zero{\char95}nat;\\
\hspace*{0pt}fib (Example.Suc (Example.Suc n)) =\\
\hspace*{0pt} ~Example.plus{\char95}nat (Example.fib n) (Example.fib (Example.Suc n));%
\end{isamarkuptext}%
\isamarkuptrue%
%
\endisatagquote
{\isafoldquote}%
%
\isadelimquote
%
\endisadelimquote
%
\begin{isamarkuptext}%
\noindent A more efficient implementation would use dynamic
  programming, e.g.~sharing of common intermediate results between
  recursive calls.  This idea is expressed by an auxiliary operation
  which computes a Fibonacci number and its successor simultaneously:%
\end{isamarkuptext}%
\isamarkuptrue%
%
\isadelimquote
%
\endisadelimquote
%
\isatagquote
\isacommand{definition}\isamarkupfalse%
\ fib{\isacharunderscore}step\ {\isacharcolon}{\isacharcolon}\ {\isachardoublequoteopen}nat\ {\isasymRightarrow}\ nat\ {\isasymtimes}\ nat{\isachardoublequoteclose}\ \isakeyword{where}\isanewline
\ \ {\isachardoublequoteopen}fib{\isacharunderscore}step\ n\ {\isacharequal}\ {\isacharparenleft}fib\ {\isacharparenleft}Suc\ n{\isacharparenright}{\isacharcomma}\ fib\ n{\isacharparenright}{\isachardoublequoteclose}%
\endisatagquote
{\isafoldquote}%
%
\isadelimquote
%
\endisadelimquote
%
\begin{isamarkuptext}%
\noindent This operation can be implemented by recursion using
  dynamic programming:%
\end{isamarkuptext}%
\isamarkuptrue%
%
\isadelimquote
%
\endisadelimquote
%
\isatagquote
\isacommand{lemma}\isamarkupfalse%
\ {\isacharbrackleft}code{\isacharbrackright}{\isacharcolon}\isanewline
\ \ {\isachardoublequoteopen}fib{\isacharunderscore}step\ {\isadigit{0}}\ {\isacharequal}\ {\isacharparenleft}Suc\ {\isadigit{0}}{\isacharcomma}\ {\isadigit{0}}{\isacharparenright}{\isachardoublequoteclose}\isanewline
\ \ {\isachardoublequoteopen}fib{\isacharunderscore}step\ {\isacharparenleft}Suc\ n{\isacharparenright}\ {\isacharequal}\ {\isacharparenleft}let\ {\isacharparenleft}m{\isacharcomma}\ q{\isacharparenright}\ {\isacharequal}\ fib{\isacharunderscore}step\ n\ in\ {\isacharparenleft}m\ {\isacharplus}\ q{\isacharcomma}\ m{\isacharparenright}{\isacharparenright}{\isachardoublequoteclose}\isanewline
\ \ \isacommand{by}\isamarkupfalse%
\ {\isacharparenleft}simp{\isacharunderscore}all\ add{\isacharcolon}\ fib{\isacharunderscore}step{\isacharunderscore}def{\isacharparenright}%
\endisatagquote
{\isafoldquote}%
%
\isadelimquote
%
\endisadelimquote
%
\begin{isamarkuptext}%
\noindent What remains is to implement \isa{fib} by \isa{fib{\isacharunderscore}step} as follows:%
\end{isamarkuptext}%
\isamarkuptrue%
%
\isadelimquote
%
\endisadelimquote
%
\isatagquote
\isacommand{lemma}\isamarkupfalse%
\ {\isacharbrackleft}code{\isacharbrackright}{\isacharcolon}\isanewline
\ \ {\isachardoublequoteopen}fib\ {\isadigit{0}}\ {\isacharequal}\ {\isadigit{0}}{\isachardoublequoteclose}\isanewline
\ \ {\isachardoublequoteopen}fib\ {\isacharparenleft}Suc\ n{\isacharparenright}\ {\isacharequal}\ fst\ {\isacharparenleft}fib{\isacharunderscore}step\ n{\isacharparenright}{\isachardoublequoteclose}\isanewline
\ \ \isacommand{by}\isamarkupfalse%
\ {\isacharparenleft}simp{\isacharunderscore}all\ add{\isacharcolon}\ fib{\isacharunderscore}step{\isacharunderscore}def{\isacharparenright}%
\endisatagquote
{\isafoldquote}%
%
\isadelimquote
%
\endisadelimquote
%
\begin{isamarkuptext}%
\noindent The resulting code shows only linear growth of runtime:%
\end{isamarkuptext}%
\isamarkuptrue%
%
\isadelimquote
%
\endisadelimquote
%
\isatagquote
%
\begin{isamarkuptext}%
\isatypewriter%
\noindent%
\hspace*{0pt}fib{\char95}step ::~Example.Nat -> (Example.Nat,~Example.Nat);\\
\hspace*{0pt}fib{\char95}step (Example.Suc n) =\\
\hspace*{0pt} ~let {\char123}\\
\hspace*{0pt} ~~~(m,~q) = Example.fib{\char95}step n;\\
\hspace*{0pt} ~{\char125}~in (Example.plus{\char95}nat m q,~m);\\
\hspace*{0pt}fib{\char95}step Example.Zero{\char95}nat =\\
\hspace*{0pt} ~(Example.Suc Example.Zero{\char95}nat,~Example.Zero{\char95}nat);\\
\hspace*{0pt}\\
\hspace*{0pt}fib ::~Example.Nat -> Example.Nat;\\
\hspace*{0pt}fib (Example.Suc n) = fst (Example.fib{\char95}step n);\\
\hspace*{0pt}fib Example.Zero{\char95}nat = Example.Zero{\char95}nat;%
\end{isamarkuptext}%
\isamarkuptrue%
%
\endisatagquote
{\isafoldquote}%
%
\isadelimquote
%
\endisadelimquote
%
\isamarkupsubsection{Datatype refinement%
}
\isamarkuptrue%
%
\begin{isamarkuptext}%
Selecting specific code equations \emph{and} datatype constructors
  leads to datatype refinement.  As an example, we will develop an
  alternative representation of the queue example given in
  \secref{sec:queue_example}.  The amortised representation is
  convenient for generating code but exposes its \qt{implementation}
  details, which may be cumbersome when proving theorems about it.
  Therefore, here is a simple, straightforward representation of
  queues:%
\end{isamarkuptext}%
\isamarkuptrue%
%
\isadelimquote
%
\endisadelimquote
%
\isatagquote
\isacommand{datatype}\isamarkupfalse%
\ {\isacharprime}a\ queue\ {\isacharequal}\ Queue\ {\isachardoublequoteopen}{\isacharprime}a\ list{\isachardoublequoteclose}\isanewline
\isanewline
\isacommand{definition}\isamarkupfalse%
\ empty\ {\isacharcolon}{\isacharcolon}\ {\isachardoublequoteopen}{\isacharprime}a\ queue{\isachardoublequoteclose}\ \isakeyword{where}\isanewline
\ \ {\isachardoublequoteopen}empty\ {\isacharequal}\ Queue\ {\isacharbrackleft}{\isacharbrackright}{\isachardoublequoteclose}\isanewline
\isanewline
\isacommand{primrec}\isamarkupfalse%
\ enqueue\ {\isacharcolon}{\isacharcolon}\ {\isachardoublequoteopen}{\isacharprime}a\ {\isasymRightarrow}\ {\isacharprime}a\ queue\ {\isasymRightarrow}\ {\isacharprime}a\ queue{\isachardoublequoteclose}\ \isakeyword{where}\isanewline
\ \ {\isachardoublequoteopen}enqueue\ x\ {\isacharparenleft}Queue\ xs{\isacharparenright}\ {\isacharequal}\ Queue\ {\isacharparenleft}xs\ {\isacharat}\ {\isacharbrackleft}x{\isacharbrackright}{\isacharparenright}{\isachardoublequoteclose}\isanewline
\isanewline
\isacommand{fun}\isamarkupfalse%
\ dequeue\ {\isacharcolon}{\isacharcolon}\ {\isachardoublequoteopen}{\isacharprime}a\ queue\ {\isasymRightarrow}\ {\isacharprime}a\ option\ {\isasymtimes}\ {\isacharprime}a\ queue{\isachardoublequoteclose}\ \isakeyword{where}\isanewline
\ \ \ \ {\isachardoublequoteopen}dequeue\ {\isacharparenleft}Queue\ {\isacharbrackleft}{\isacharbrackright}{\isacharparenright}\ {\isacharequal}\ {\isacharparenleft}None{\isacharcomma}\ Queue\ {\isacharbrackleft}{\isacharbrackright}{\isacharparenright}{\isachardoublequoteclose}\isanewline
\ \ {\isacharbar}\ {\isachardoublequoteopen}dequeue\ {\isacharparenleft}Queue\ {\isacharparenleft}x\ {\isacharhash}\ xs{\isacharparenright}{\isacharparenright}\ {\isacharequal}\ {\isacharparenleft}Some\ x{\isacharcomma}\ Queue\ xs{\isacharparenright}{\isachardoublequoteclose}%
\endisatagquote
{\isafoldquote}%
%
\isadelimquote
%
\endisadelimquote
%
\begin{isamarkuptext}%
\noindent This we can use directly for proving;  for executing,
  we provide an alternative characterisation:%
\end{isamarkuptext}%
\isamarkuptrue%
%
\isadelimquote
%
\endisadelimquote
%
\isatagquote
\isacommand{definition}\isamarkupfalse%
\ AQueue\ {\isacharcolon}{\isacharcolon}\ {\isachardoublequoteopen}{\isacharprime}a\ list\ {\isasymRightarrow}\ {\isacharprime}a\ list\ {\isasymRightarrow}\ {\isacharprime}a\ queue{\isachardoublequoteclose}\ \isakeyword{where}\isanewline
\ \ {\isachardoublequoteopen}AQueue\ xs\ ys\ {\isacharequal}\ Queue\ {\isacharparenleft}ys\ {\isacharat}\ rev\ xs{\isacharparenright}{\isachardoublequoteclose}\isanewline
\isanewline
\isacommand{code{\isacharunderscore}datatype}\isamarkupfalse%
\ AQueue%
\endisatagquote
{\isafoldquote}%
%
\isadelimquote
%
\endisadelimquote
%
\begin{isamarkuptext}%
\noindent Here we define a \qt{constructor} \isa{AQueue} which
  is defined in terms of \isa{Queue} and interprets its arguments
  according to what the \emph{content} of an amortised queue is supposed
  to be.

  The prerequisite for datatype constructors is only syntactical: a
  constructor must be of type \isa{{\isasymtau}\ {\isacharequal}\ {\isasymdots}\ {\isasymRightarrow}\ {\isasymkappa}\ {\isasymalpha}\isactrlisub {\isadigit{1}}\ {\isasymdots}\ {\isasymalpha}\isactrlisub n} where \isa{{\isacharbraceleft}{\isasymalpha}\isactrlisub {\isadigit{1}}{\isacharcomma}\ {\isasymdots}{\isacharcomma}\ {\isasymalpha}\isactrlisub n{\isacharbraceright}} is exactly the set of \emph{all} type variables in
  \isa{{\isasymtau}}; then \isa{{\isasymkappa}} is its corresponding datatype.  The
  HOL datatype package by default registers any new datatype with its
  constructors, but this may be changed using \indexdef{}{command}{code\_datatype}\hypertarget{command.code-datatype}{\hyperlink{command.code-datatype}{\mbox{\isa{\isacommand{code{\isacharunderscore}datatype}}}}}; the currently chosen constructors can be inspected
  using the \hyperlink{command.print-codesetup}{\mbox{\isa{\isacommand{print{\isacharunderscore}codesetup}}}} command.

  Equipped with this, we are able to prove the following equations
  for our primitive queue operations which \qt{implement} the simple
  queues in an amortised fashion:%
\end{isamarkuptext}%
\isamarkuptrue%
%
\isadelimquote
%
\endisadelimquote
%
\isatagquote
\isacommand{lemma}\isamarkupfalse%
\ empty{\isacharunderscore}AQueue\ {\isacharbrackleft}code{\isacharbrackright}{\isacharcolon}\isanewline
\ \ {\isachardoublequoteopen}empty\ {\isacharequal}\ AQueue\ {\isacharbrackleft}{\isacharbrackright}\ {\isacharbrackleft}{\isacharbrackright}{\isachardoublequoteclose}\isanewline
\ \ \isacommand{unfolding}\isamarkupfalse%
\ AQueue{\isacharunderscore}def\ empty{\isacharunderscore}def\ \isacommand{by}\isamarkupfalse%
\ simp\isanewline
\isanewline
\isacommand{lemma}\isamarkupfalse%
\ enqueue{\isacharunderscore}AQueue\ {\isacharbrackleft}code{\isacharbrackright}{\isacharcolon}\isanewline
\ \ {\isachardoublequoteopen}enqueue\ x\ {\isacharparenleft}AQueue\ xs\ ys{\isacharparenright}\ {\isacharequal}\ AQueue\ {\isacharparenleft}x\ {\isacharhash}\ xs{\isacharparenright}\ ys{\isachardoublequoteclose}\isanewline
\ \ \isacommand{unfolding}\isamarkupfalse%
\ AQueue{\isacharunderscore}def\ \isacommand{by}\isamarkupfalse%
\ simp\isanewline
\isanewline
\isacommand{lemma}\isamarkupfalse%
\ dequeue{\isacharunderscore}AQueue\ {\isacharbrackleft}code{\isacharbrackright}{\isacharcolon}\isanewline
\ \ {\isachardoublequoteopen}dequeue\ {\isacharparenleft}AQueue\ xs\ {\isacharbrackleft}{\isacharbrackright}{\isacharparenright}\ {\isacharequal}\isanewline
\ \ \ \ {\isacharparenleft}if\ xs\ {\isacharequal}\ {\isacharbrackleft}{\isacharbrackright}\ then\ {\isacharparenleft}None{\isacharcomma}\ AQueue\ {\isacharbrackleft}{\isacharbrackright}\ {\isacharbrackleft}{\isacharbrackright}{\isacharparenright}\isanewline
\ \ \ \ else\ dequeue\ {\isacharparenleft}AQueue\ {\isacharbrackleft}{\isacharbrackright}\ {\isacharparenleft}rev\ xs{\isacharparenright}{\isacharparenright}{\isacharparenright}{\isachardoublequoteclose}\isanewline
\ \ {\isachardoublequoteopen}dequeue\ {\isacharparenleft}AQueue\ xs\ {\isacharparenleft}y\ {\isacharhash}\ ys{\isacharparenright}{\isacharparenright}\ {\isacharequal}\ {\isacharparenleft}Some\ y{\isacharcomma}\ AQueue\ xs\ ys{\isacharparenright}{\isachardoublequoteclose}\isanewline
\ \ \isacommand{unfolding}\isamarkupfalse%
\ AQueue{\isacharunderscore}def\ \isacommand{by}\isamarkupfalse%
\ simp{\isacharunderscore}all%
\endisatagquote
{\isafoldquote}%
%
\isadelimquote
%
\endisadelimquote
%
\begin{isamarkuptext}%
\noindent For completeness, we provide a substitute for the
  \isa{case} combinator on queues:%
\end{isamarkuptext}%
\isamarkuptrue%
%
\isadelimquote
%
\endisadelimquote
%
\isatagquote
\isacommand{lemma}\isamarkupfalse%
\ queue{\isacharunderscore}case{\isacharunderscore}AQueue\ {\isacharbrackleft}code{\isacharbrackright}{\isacharcolon}\isanewline
\ \ {\isachardoublequoteopen}queue{\isacharunderscore}case\ f\ {\isacharparenleft}AQueue\ xs\ ys{\isacharparenright}\ {\isacharequal}\ f\ {\isacharparenleft}ys\ {\isacharat}\ rev\ xs{\isacharparenright}{\isachardoublequoteclose}\isanewline
\ \ \isacommand{unfolding}\isamarkupfalse%
\ AQueue{\isacharunderscore}def\ \isacommand{by}\isamarkupfalse%
\ simp%
\endisatagquote
{\isafoldquote}%
%
\isadelimquote
%
\endisadelimquote
%
\begin{isamarkuptext}%
\noindent The resulting code looks as expected:%
\end{isamarkuptext}%
\isamarkuptrue%
%
\isadelimquote
%
\endisadelimquote
%
\isatagquote
%
\begin{isamarkuptext}%
\isatypewriter%
\noindent%
\hspace*{0pt}structure Example :~sig\\
\hspace*{0pt} ~val id :~'a -> 'a\\
\hspace*{0pt} ~val fold :~('a -> 'b -> 'b) -> 'a list -> 'b -> 'b\\
\hspace*{0pt} ~val rev :~'a list -> 'a list\\
\hspace*{0pt} ~val null :~'a list -> bool\\
\hspace*{0pt} ~datatype 'a queue = AQueue of 'a list * 'a list\\
\hspace*{0pt} ~val empty :~'a queue\\
\hspace*{0pt} ~val dequeue :~'a queue -> 'a option * 'a queue\\
\hspace*{0pt} ~val enqueue :~'a -> 'a queue -> 'a queue\\
\hspace*{0pt}end = struct\\
\hspace*{0pt}\\
\hspace*{0pt}fun id x = (fn xa => xa) x;\\
\hspace*{0pt}\\
\hspace*{0pt}fun fold f [] = id\\
\hspace*{0pt} ~| fold f (x ::~xs) = fold f xs o f x;\\
\hspace*{0pt}\\
\hspace*{0pt}fun rev xs = fold (fn a => fn b => a ::~b) xs [];\\
\hspace*{0pt}\\
\hspace*{0pt}fun null [] = true\\
\hspace*{0pt} ~| null (x ::~xs) = false;\\
\hspace*{0pt}\\
\hspace*{0pt}datatype 'a queue = AQueue of 'a list * 'a list;\\
\hspace*{0pt}\\
\hspace*{0pt}val empty :~'a queue = AQueue ([],~[]);\\
\hspace*{0pt}\\
\hspace*{0pt}fun dequeue (AQueue (xs,~y ::~ys)) = (SOME y,~AQueue (xs,~ys))\\
\hspace*{0pt} ~| dequeue (AQueue (xs,~[])) =\\
\hspace*{0pt} ~~~(if null xs then (NONE,~AQueue ([],~[]))\\
\hspace*{0pt} ~~~~~else dequeue (AQueue ([],~rev xs)));\\
\hspace*{0pt}\\
\hspace*{0pt}fun enqueue x (AQueue (xs,~ys)) = AQueue (x ::~xs,~ys);\\
\hspace*{0pt}\\
\hspace*{0pt}end;~(*struct Example*)%
\end{isamarkuptext}%
\isamarkuptrue%
%
\endisatagquote
{\isafoldquote}%
%
\isadelimquote
%
\endisadelimquote
%
\begin{isamarkuptext}%
The same techniques can also be applied to types which are not
  specified as datatypes, e.g.~type \isa{int} is originally specified
  as quotient type by means of \indexdef{}{command}{typedef}\hypertarget{command.typedef}{\hyperlink{command.typedef}{\mbox{\isa{\isacommand{typedef}}}}}, but for code
  generation constants allowing construction of binary numeral values
  are used as constructors for \isa{int}.

  This approach however fails if the representation of a type demands
  invariants; this issue is discussed in the next section.%
\end{isamarkuptext}%
\isamarkuptrue%
%
\isamarkupsubsection{Datatype refinement involving invariants%
}
\isamarkuptrue%
%
\begin{isamarkuptext}%
Datatype representation involving invariants require a dedicated
  setup for the type and its primitive operations.  As a running
  example, we implement a type \isa{{\isacharprime}a\ dlist} of list consisting
  of distinct elements.

  The first step is to decide on which representation the abstract
  type (in our example \isa{{\isacharprime}a\ dlist}) should be implemented.
  Here we choose \isa{{\isacharprime}a\ list}.  Then a conversion from the concrete
  type to the abstract type must be specified, here:%
\end{isamarkuptext}%
\isamarkuptrue%
%
\isadelimquote
%
\endisadelimquote
%
\isatagquote
%
\begin{isamarkuptext}%
\isa{Dlist\ {\isasymColon}\ {\isacharprime}a\ list\ {\isasymRightarrow}\ {\isacharprime}a\ dlist}%
\end{isamarkuptext}%
\isamarkuptrue%
%
\endisatagquote
{\isafoldquote}%
%
\isadelimquote
%
\endisadelimquote
%
\begin{isamarkuptext}%
\noindent Next follows the specification of a suitable \emph{projection},
  i.e.~a conversion from abstract to concrete type:%
\end{isamarkuptext}%
\isamarkuptrue%
%
\isadelimquote
%
\endisadelimquote
%
\isatagquote
%
\begin{isamarkuptext}%
\isa{list{\isacharunderscore}of{\isacharunderscore}dlist\ {\isasymColon}\ {\isacharprime}a\ dlist\ {\isasymRightarrow}\ {\isacharprime}a\ list}%
\end{isamarkuptext}%
\isamarkuptrue%
%
\endisatagquote
{\isafoldquote}%
%
\isadelimquote
%
\endisadelimquote
%
\begin{isamarkuptext}%
\noindent This projection must be specified such that the following
  \emph{abstract datatype certificate} can be proven:%
\end{isamarkuptext}%
\isamarkuptrue%
%
\isadelimquote
%
\endisadelimquote
%
\isatagquote
\isacommand{lemma}\isamarkupfalse%
\ {\isacharbrackleft}code\ abstype{\isacharbrackright}{\isacharcolon}\isanewline
\ \ {\isachardoublequoteopen}Dlist\ {\isacharparenleft}list{\isacharunderscore}of{\isacharunderscore}dlist\ dxs{\isacharparenright}\ {\isacharequal}\ dxs{\isachardoublequoteclose}\isanewline
\ \ \isacommand{by}\isamarkupfalse%
\ {\isacharparenleft}fact\ Dlist{\isacharunderscore}list{\isacharunderscore}of{\isacharunderscore}dlist{\isacharparenright}%
\endisatagquote
{\isafoldquote}%
%
\isadelimquote
%
\endisadelimquote
%
\begin{isamarkuptext}%
\noindent Note that so far the invariant on representations
  (\isa{distinct\ {\isasymColon}\ {\isacharprime}a\ list\ {\isasymRightarrow}\ bool}) has never been mentioned explicitly:
  the invariant is only referred to implicitly: all values in
  set \isa{{\isacharbraceleft}xs{\isachardot}\ list{\isacharunderscore}of{\isacharunderscore}dlist\ {\isacharparenleft}Dlist\ xs{\isacharparenright}\ {\isacharequal}\ xs{\isacharbraceright}} are invariant,
  and in our example this is exactly \isa{{\isacharbraceleft}xs{\isachardot}\ distinct\ xs{\isacharbraceright}}.
  
  The primitive operations on \isa{{\isacharprime}a\ dlist} are specified
  indirectly using the projection \isa{list{\isacharunderscore}of{\isacharunderscore}dlist}.  For
  the empty \isa{dlist}, \isa{Dlist{\isachardot}empty}, we finally want
  the code equation%
\end{isamarkuptext}%
\isamarkuptrue%
%
\isadelimquote
%
\endisadelimquote
%
\isatagquote
%
\begin{isamarkuptext}%
\isa{Dlist{\isachardot}empty\ {\isacharequal}\ Dlist\ {\isacharbrackleft}{\isacharbrackright}}%
\end{isamarkuptext}%
\isamarkuptrue%
%
\endisatagquote
{\isafoldquote}%
%
\isadelimquote
%
\endisadelimquote
%
\begin{isamarkuptext}%
\noindent This we have to prove indirectly as follows:%
\end{isamarkuptext}%
\isamarkuptrue%
%
\isadelimquote
%
\endisadelimquote
%
\isatagquote
\isacommand{lemma}\isamarkupfalse%
\ {\isacharbrackleft}code\ abstract{\isacharbrackright}{\isacharcolon}\isanewline
\ \ {\isachardoublequoteopen}list{\isacharunderscore}of{\isacharunderscore}dlist\ Dlist{\isachardot}empty\ {\isacharequal}\ {\isacharbrackleft}{\isacharbrackright}{\isachardoublequoteclose}\isanewline
\ \ \isacommand{by}\isamarkupfalse%
\ {\isacharparenleft}fact\ list{\isacharunderscore}of{\isacharunderscore}dlist{\isacharunderscore}empty{\isacharparenright}%
\endisatagquote
{\isafoldquote}%
%
\isadelimquote
%
\endisadelimquote
%
\begin{isamarkuptext}%
\noindent This equation logically encodes both the desired code
  equation and that the expression \isa{Dlist} is applied to obeys
  the implicit invariant.  Equations for insertion and removal are
  similar:%
\end{isamarkuptext}%
\isamarkuptrue%
%
\isadelimquote
%
\endisadelimquote
%
\isatagquote
\isacommand{lemma}\isamarkupfalse%
\ {\isacharbrackleft}code\ abstract{\isacharbrackright}{\isacharcolon}\isanewline
\ \ {\isachardoublequoteopen}list{\isacharunderscore}of{\isacharunderscore}dlist\ {\isacharparenleft}Dlist{\isachardot}insert\ x\ dxs{\isacharparenright}\ {\isacharequal}\ List{\isachardot}insert\ x\ {\isacharparenleft}list{\isacharunderscore}of{\isacharunderscore}dlist\ dxs{\isacharparenright}{\isachardoublequoteclose}\isanewline
\ \ \isacommand{by}\isamarkupfalse%
\ {\isacharparenleft}fact\ list{\isacharunderscore}of{\isacharunderscore}dlist{\isacharunderscore}insert{\isacharparenright}\isanewline
\isanewline
\isacommand{lemma}\isamarkupfalse%
\ {\isacharbrackleft}code\ abstract{\isacharbrackright}{\isacharcolon}\isanewline
\ \ {\isachardoublequoteopen}list{\isacharunderscore}of{\isacharunderscore}dlist\ {\isacharparenleft}Dlist{\isachardot}remove\ x\ dxs{\isacharparenright}\ {\isacharequal}\ remove{\isadigit{1}}\ x\ {\isacharparenleft}list{\isacharunderscore}of{\isacharunderscore}dlist\ dxs{\isacharparenright}{\isachardoublequoteclose}\isanewline
\ \ \isacommand{by}\isamarkupfalse%
\ {\isacharparenleft}fact\ list{\isacharunderscore}of{\isacharunderscore}dlist{\isacharunderscore}remove{\isacharparenright}%
\endisatagquote
{\isafoldquote}%
%
\isadelimquote
%
\endisadelimquote
%
\begin{isamarkuptext}%
\noindent Then the corresponding code is as follows:%
\end{isamarkuptext}%
\isamarkuptrue%
%
\isadelimquote
%
\endisadelimquote
%
\isatagquote
%
\begin{isamarkuptext}%
\isatypewriter%
\noindent%
\hspace*{0pt}module Example where {\char123}\\
\hspace*{0pt}\\
\hspace*{0pt}newtype Dlist a = Dlist [a];\\
\hspace*{0pt}\\
\hspace*{0pt}empty ::~forall a.~Example.Dlist a;\\
\hspace*{0pt}empty = Example.Dlist [];\\
\hspace*{0pt}\\
\hspace*{0pt}member ::~forall a.~(Eq a) => [a] -> a -> Bool;\\
\hspace*{0pt}member [] y = False;\\
\hspace*{0pt}member (x :~xs) y = x == y || Example.member xs y;\\
\hspace*{0pt}\\
\hspace*{0pt}inserta ::~forall a.~(Eq a) => a -> [a] -> [a];\\
\hspace*{0pt}inserta x xs = (if Example.member xs x then xs else x :~xs);\\
\hspace*{0pt}\\
\hspace*{0pt}list{\char95}of{\char95}dlist ::~forall a.~Example.Dlist a -> [a];\\
\hspace*{0pt}list{\char95}of{\char95}dlist (Example.Dlist x) = x;\\
\hspace*{0pt}\\
\hspace*{0pt}insert ::~forall a.~(Eq a) => a -> Example.Dlist a -> Example.Dlist a;\\
\hspace*{0pt}insert x dxs =\\
\hspace*{0pt} ~Example.Dlist (Example.inserta x (Example.list{\char95}of{\char95}dlist dxs));\\
\hspace*{0pt}\\
\hspace*{0pt}remove1 ::~forall a.~(Eq a) => a -> [a] -> [a];\\
\hspace*{0pt}remove1 x [] = [];\\
\hspace*{0pt}remove1 x (y :~xs) = (if x == y then xs else y :~Example.remove1 x xs);\\
\hspace*{0pt}\\
\hspace*{0pt}remove ::~forall a.~(Eq a) => a -> Example.Dlist a -> Example.Dlist a;\\
\hspace*{0pt}remove x dxs =\\
\hspace*{0pt} ~Example.Dlist (Example.remove1 x (Example.list{\char95}of{\char95}dlist dxs));\\
\hspace*{0pt}\\
\hspace*{0pt}{\char125}%
\end{isamarkuptext}%
\isamarkuptrue%
%
\endisatagquote
{\isafoldquote}%
%
\isadelimquote
%
\endisadelimquote
%
\begin{isamarkuptext}%
Typical data structures implemented by representations involving
  invariants are available in the library, e.g.~theories \hyperlink{theory.Fset}{\mbox{\isa{Fset}}} and \hyperlink{theory.Mapping}{\mbox{\isa{Mapping}}} specify sets (type \isa{{\isacharprime}a\ fset}) and
  key-value-mappings (type \isa{{\isacharparenleft}{\isacharprime}a{\isacharcomma}\ {\isacharprime}b{\isacharparenright}\ mapping}) respectively;
  these can be implemented by distinct lists as presented here as
  example (theory \hyperlink{theory.Dlist}{\mbox{\isa{Dlist}}}) and red-black-trees respectively
  (theory \hyperlink{theory.RBT}{\mbox{\isa{RBT}}}).%
\end{isamarkuptext}%
\isamarkuptrue%
%
\isadelimtheory
%
\endisadelimtheory
%
\isatagtheory
\isacommand{end}\isamarkupfalse%
%
\endisatagtheory
{\isafoldtheory}%
%
\isadelimtheory
%
\endisadelimtheory
\isanewline
\end{isabellebody}%
%%% Local Variables:
%%% mode: latex
%%% TeX-master: "root"
%%% End:
