%
\begin{isabellebody}%
\def\isabellecontext{Evaluation}%
%
\isadelimtheory
%
\endisadelimtheory
%
\isatagtheory
\isacommand{theory}\isamarkupfalse%
\ Evaluation\isanewline
\isakeyword{imports}\ Setup\isanewline
\isakeyword{begin}%
\endisatagtheory
{\isafoldtheory}%
%
\isadelimtheory
%
\endisadelimtheory
%
\isamarkupsection{Evaluation%
}
\isamarkuptrue%
%
\begin{isamarkuptext}%
Introduction%
\end{isamarkuptext}%
\isamarkuptrue%
%
\isamarkupsubsection{Evaluation techniques%
}
\isamarkuptrue%
%
\begin{isamarkuptext}%
simplifier%
\end{isamarkuptext}%
\isamarkuptrue%
%
\begin{isamarkuptext}%
nbe%
\end{isamarkuptext}%
\isamarkuptrue%
%
\begin{isamarkuptext}%
eval target: SML standalone vs. Isabelle/SML, example, soundness%
\end{isamarkuptext}%
\isamarkuptrue%
%
\isamarkupsubsection{Dynamic evaluation%
}
\isamarkuptrue%
%
\begin{isamarkuptext}%
value (three variants)%
\end{isamarkuptext}%
\isamarkuptrue%
%
\begin{isamarkuptext}%
methods (three variants)%
\end{isamarkuptext}%
\isamarkuptrue%
%
\begin{isamarkuptext}%
corresponding ML interfaces%
\end{isamarkuptext}%
\isamarkuptrue%
%
\isamarkupsubsection{Static evaluation%
}
\isamarkuptrue%
%
\begin{isamarkuptext}%
code_simp, nbe (tbd), Eval (tbd, in simple fashion)%
\end{isamarkuptext}%
\isamarkuptrue%
%
\begin{isamarkuptext}%
hand-written: code antiquotation%
\end{isamarkuptext}%
\isamarkuptrue%
%
\isamarkupsubsection{Hybrid techniques%
}
\isamarkuptrue%
%
\begin{isamarkuptext}%
code reflect runtime%
\end{isamarkuptext}%
\isamarkuptrue%
%
\begin{isamarkuptext}%
code reflect external%
\end{isamarkuptext}%
\isamarkuptrue%
%
\begin{isamarkuptext}%
FIXME here the old sections follow%
\end{isamarkuptext}%
\isamarkuptrue%
%
\isamarkupsubsection{Evaluation oracle%
}
\isamarkuptrue%
%
\begin{isamarkuptext}%
Code generation may also be used to \emph{evaluate} expressions
  (using \isa{SML} as target language of course).
  For instance, the \indexdef{}{command}{value}\hypertarget{command.value}{\hyperlink{command.value}{\mbox{\isa{\isacommand{value}}}}} reduces an expression to a
  normal form with respect to the underlying code equations:%
\end{isamarkuptext}%
\isamarkuptrue%
%
\isadelimquote
%
\endisadelimquote
%
\isatagquote
\isacommand{value}\isamarkupfalse%
\ {\isachardoublequoteopen}{\isadigit{4}}{\isadigit{2}}\ {\isacharslash}\ {\isacharparenleft}{\isadigit{1}}{\isadigit{2}}\ {\isacharcolon}{\isacharcolon}\ rat{\isacharparenright}{\isachardoublequoteclose}%
\endisatagquote
{\isafoldquote}%
%
\isadelimquote
%
\endisadelimquote
%
\begin{isamarkuptext}%
\noindent will display \isa{{\isadigit{7}}\ {\isacharslash}\ {\isadigit{2}}}.

  The \hyperlink{method.eval}{\mbox{\isa{eval}}} method tries to reduce a goal by code generation to \isa{True}
  and solves it in that case, but fails otherwise:%
\end{isamarkuptext}%
\isamarkuptrue%
%
\isadelimquote
%
\endisadelimquote
%
\isatagquote
\isacommand{lemma}\isamarkupfalse%
\ {\isachardoublequoteopen}{\isadigit{4}}{\isadigit{2}}\ {\isacharslash}\ {\isacharparenleft}{\isadigit{1}}{\isadigit{2}}\ {\isacharcolon}{\isacharcolon}\ rat{\isacharparenright}\ {\isacharequal}\ {\isadigit{7}}\ {\isacharslash}\ {\isadigit{2}}{\isachardoublequoteclose}\isanewline
\ \ \isacommand{by}\isamarkupfalse%
\ eval%
\endisatagquote
{\isafoldquote}%
%
\isadelimquote
%
\endisadelimquote
%
\begin{isamarkuptext}%
\noindent The soundness of the \hyperlink{method.eval}{\mbox{\isa{eval}}} method depends crucially 
  on the correctness of the code generator;  this is one of the reasons
  why you should not use adaptation (see \secref{sec:adaptation}) frivolously.%
\end{isamarkuptext}%
\isamarkuptrue%
%
\isamarkupsubsubsection{Code antiquotation%
}
\isamarkuptrue%
%
\begin{isamarkuptext}%
In scenarios involving techniques like reflection it is quite common
  that code generated from a theory forms the basis for implementing
  a proof procedure in \isa{SML}.  To facilitate interfacing of generated code
  with system code, the code generator provides a \isa{code} antiquotation:%
\end{isamarkuptext}%
\isamarkuptrue%
%
\isadelimquote
%
\endisadelimquote
%
\isatagquote
\isacommand{datatype}\isamarkupfalse%
\ form\ {\isacharequal}\ T\ {\isacharbar}\ F\ {\isacharbar}\ And\ form\ form\ {\isacharbar}\ Or\ form\ form\ %
\endisatagquote
{\isafoldquote}%
%
\isadelimquote
%
\endisadelimquote
%
\isadelimquotett
\ %
\endisadelimquotett
%
\isatagquotett
\isacommand{ML}\isamarkupfalse%
\ {\isacharverbatimopen}\isanewline
\ \ fun\ eval{\isacharunderscore}form\ %
\isaantiq
code\ T%
\endisaantiq
\ {\isacharequal}\ true\isanewline
\ \ \ \ {\isacharbar}\ eval{\isacharunderscore}form\ %
\isaantiq
code\ F%
\endisaantiq
\ {\isacharequal}\ false\isanewline
\ \ \ \ {\isacharbar}\ eval{\isacharunderscore}form\ {\isacharparenleft}%
\isaantiq
code\ And%
\endisaantiq
\ {\isacharparenleft}p{\isacharcomma}\ q{\isacharparenright}{\isacharparenright}\ {\isacharequal}\isanewline
\ \ \ \ \ \ \ \ eval{\isacharunderscore}form\ p\ andalso\ eval{\isacharunderscore}form\ q\isanewline
\ \ \ \ {\isacharbar}\ eval{\isacharunderscore}form\ {\isacharparenleft}%
\isaantiq
code\ Or%
\endisaantiq
\ {\isacharparenleft}p{\isacharcomma}\ q{\isacharparenright}{\isacharparenright}\ {\isacharequal}\isanewline
\ \ \ \ \ \ \ \ eval{\isacharunderscore}form\ p\ orelse\ eval{\isacharunderscore}form\ q{\isacharsemicolon}\isanewline
{\isacharverbatimclose}%
\endisatagquotett
{\isafoldquotett}%
%
\isadelimquotett
%
\endisadelimquotett
%
\begin{isamarkuptext}%
\noindent \isa{code} takes as argument the name of a constant;  after the
  whole \isa{SML} is read, the necessary code is generated transparently
  and the corresponding constant names are inserted.  This technique also
  allows to use pattern matching on constructors stemming from compiled
  \isa{datatypes}.

  For a less simplistic example, theory \hyperlink{theory.Ferrack}{\mbox{\isa{Ferrack}}} is
  a good reference.%
\end{isamarkuptext}%
\isamarkuptrue%
%
\isadelimtheory
%
\endisadelimtheory
%
\isatagtheory
\isacommand{end}\isamarkupfalse%
%
\endisatagtheory
{\isafoldtheory}%
%
\isadelimtheory
%
\endisadelimtheory
\isanewline
\end{isabellebody}%
%%% Local Variables:
%%% mode: latex
%%% TeX-master: "root"
%%% End:
