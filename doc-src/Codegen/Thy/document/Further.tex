%
\begin{isabellebody}%
\def\isabellecontext{Further}%
%
\isadelimtheory
%
\endisadelimtheory
%
\isatagtheory
\isacommand{theory}\isamarkupfalse%
\ Further\isanewline
\isakeyword{imports}\ Setup\isanewline
\isakeyword{begin}%
\endisatagtheory
{\isafoldtheory}%
%
\isadelimtheory
%
\endisadelimtheory
%
\isamarkupsection{Further issues \label{sec:further}%
}
\isamarkuptrue%
%
\isamarkupsubsection{Further reading%
}
\isamarkuptrue%
%
\begin{isamarkuptext}%
Do dive deeper into the issue of code generation, you should visit
  the Isabelle/Isar Reference Manual \cite{isabelle-isar-ref} which
  contains exhaustive syntax diagrams.%
\end{isamarkuptext}%
\isamarkuptrue%
%
\isamarkupsubsection{Modules%
}
\isamarkuptrue%
%
\begin{isamarkuptext}%
When invoking the \hyperlink{command.export-code}{\mbox{\isa{\isacommand{export{\isacharunderscore}code}}}} command it is possible to leave
  out the \hyperlink{keyword.module-name}{\mbox{\isa{\isakeyword{module{\isacharunderscore}name}}}} part;  then code is distributed over
  different modules, where the module name space roughly is induced
  by the \isa{Isabelle} theory name space.

  Then sometimes the awkward situation occurs that dependencies between
  definitions introduce cyclic dependencies between modules, which in the
  \isa{Haskell} world leaves you to the mercy of the \isa{Haskell} implementation
  you are using,  while for \isa{SML}/\isa{OCaml} code generation is not possible.

  A solution is to declare module names explicitly.
  Let use assume the three cyclically dependent
  modules are named \emph{A}, \emph{B} and \emph{C}.
  Then, by stating%
\end{isamarkuptext}%
\isamarkuptrue%
%
\isadelimquote
%
\endisadelimquote
%
\isatagquote
\isacommand{code{\isacharunderscore}modulename}\isamarkupfalse%
\ SML\isanewline
\ \ A\ ABC\isanewline
\ \ B\ ABC\isanewline
\ \ C\ ABC%
\endisatagquote
{\isafoldquote}%
%
\isadelimquote
%
\endisadelimquote
%
\begin{isamarkuptext}%
we explicitly map all those modules on \emph{ABC},
  resulting in an ad-hoc merge of this three modules
  at serialisation time.%
\end{isamarkuptext}%
\isamarkuptrue%
%
\isamarkupsubsection{Evaluation oracle%
}
\isamarkuptrue%
%
\begin{isamarkuptext}%
Code generation may also be used to \emph{evaluate} expressions
  (using \isa{SML} as target language of course).
  For instance, the \hyperlink{command.value}{\mbox{\isa{\isacommand{value}}}} allows to reduce an expression to a
  normal form with respect to the underlying code equations:%
\end{isamarkuptext}%
\isamarkuptrue%
%
\isadelimquote
%
\endisadelimquote
%
\isatagquote
\isacommand{value}\isamarkupfalse%
\ {\isachardoublequoteopen}{\isadigit{4}}{\isadigit{2}}\ {\isacharslash}\ {\isacharparenleft}{\isadigit{1}}{\isadigit{2}}\ {\isacharcolon}{\isacharcolon}\ rat{\isacharparenright}{\isachardoublequoteclose}%
\endisatagquote
{\isafoldquote}%
%
\isadelimquote
%
\endisadelimquote
%
\begin{isamarkuptext}%
\noindent will display \isa{{\isadigit{7}}\ {\isacharslash}\ {\isadigit{2}}}.

  The \hyperlink{method.eval}{\mbox{\isa{eval}}} method tries to reduce a goal by code generation to \isa{True}
  and solves it in that case, but fails otherwise:%
\end{isamarkuptext}%
\isamarkuptrue%
%
\isadelimquote
%
\endisadelimquote
%
\isatagquote
\isacommand{lemma}\isamarkupfalse%
\ {\isachardoublequoteopen}{\isadigit{4}}{\isadigit{2}}\ {\isacharslash}\ {\isacharparenleft}{\isadigit{1}}{\isadigit{2}}\ {\isacharcolon}{\isacharcolon}\ rat{\isacharparenright}\ {\isacharequal}\ {\isadigit{7}}\ {\isacharslash}\ {\isadigit{2}}{\isachardoublequoteclose}\isanewline
\ \ \isacommand{by}\isamarkupfalse%
\ eval%
\endisatagquote
{\isafoldquote}%
%
\isadelimquote
%
\endisadelimquote
%
\begin{isamarkuptext}%
\noindent The soundness of the \hyperlink{method.eval}{\mbox{\isa{eval}}} method depends crucially 
  on the correctness of the code generator;  this is one of the reasons
  why you should not use adaptation (see \secref{sec:adaptation}) frivolously.%
\end{isamarkuptext}%
\isamarkuptrue%
%
\isamarkupsubsection{Code antiquotation%
}
\isamarkuptrue%
%
\begin{isamarkuptext}%
In scenarios involving techniques like reflection it is quite common
  that code generated from a theory forms the basis for implementing
  a proof procedure in \isa{SML}.  To facilitate interfacing of generated code
  with system code, the code generator provides a \isa{code} antiquotation:%
\end{isamarkuptext}%
\isamarkuptrue%
%
\isadelimquote
%
\endisadelimquote
%
\isatagquote
\isacommand{datatype}\isamarkupfalse%
\ form\ {\isacharequal}\ T\ {\isacharbar}\ F\ {\isacharbar}\ And\ form\ form\ {\isacharbar}\ Or\ form\ form\ %
\endisatagquote
{\isafoldquote}%
%
\isadelimquote
%
\endisadelimquote
%
\isadelimquotett
\ %
\endisadelimquotett
%
\isatagquotett
\isacommand{ML}\isamarkupfalse%
\ {\isacharverbatimopen}\isanewline
\ \ fun\ eval{\isacharunderscore}form\ %
\isaantiq
code\ T%
\endisaantiq
\ {\isacharequal}\ true\isanewline
\ \ \ \ {\isacharbar}\ eval{\isacharunderscore}form\ %
\isaantiq
code\ F%
\endisaantiq
\ {\isacharequal}\ false\isanewline
\ \ \ \ {\isacharbar}\ eval{\isacharunderscore}form\ {\isacharparenleft}%
\isaantiq
code\ And%
\endisaantiq
\ {\isacharparenleft}p{\isacharcomma}\ q{\isacharparenright}{\isacharparenright}\ {\isacharequal}\isanewline
\ \ \ \ \ \ \ \ eval{\isacharunderscore}form\ p\ andalso\ eval{\isacharunderscore}form\ q\isanewline
\ \ \ \ {\isacharbar}\ eval{\isacharunderscore}form\ {\isacharparenleft}%
\isaantiq
code\ Or%
\endisaantiq
\ {\isacharparenleft}p{\isacharcomma}\ q{\isacharparenright}{\isacharparenright}\ {\isacharequal}\isanewline
\ \ \ \ \ \ \ \ eval{\isacharunderscore}form\ p\ orelse\ eval{\isacharunderscore}form\ q{\isacharsemicolon}\isanewline
{\isacharverbatimclose}%
\endisatagquotett
{\isafoldquotett}%
%
\isadelimquotett
%
\endisadelimquotett
%
\begin{isamarkuptext}%
\noindent \isa{code} takes as argument the name of a constant;  after the
  whole \isa{SML} is read, the necessary code is generated transparently
  and the corresponding constant names are inserted.  This technique also
  allows to use pattern matching on constructors stemming from compiled
  \isa{datatypes}.

  For a less simplistic example, theory \hyperlink{theory.Ferrack}{\mbox{\isa{Ferrack}}} is
  a good reference.%
\end{isamarkuptext}%
\isamarkuptrue%
%
\isamarkupsubsection{Imperative data structures%
}
\isamarkuptrue%
%
\begin{isamarkuptext}%
If you consider imperative data structures as inevitable for a specific
  application, you should consider
  \emph{Imperative Functional Programming with Isabelle/HOL}
  (\cite{bulwahn-et-al:2008:imperative});
  the framework described there is available in theory \hyperlink{theory.Imperative-HOL}{\mbox{\isa{Imperative{\isacharunderscore}HOL}}}.%
\end{isamarkuptext}%
\isamarkuptrue%
%
\isadelimtheory
%
\endisadelimtheory
%
\isatagtheory
\isacommand{end}\isamarkupfalse%
%
\endisatagtheory
{\isafoldtheory}%
%
\isadelimtheory
%
\endisadelimtheory
\isanewline
\end{isabellebody}%
%%% Local Variables:
%%% mode: latex
%%% TeX-master: "root"
%%% End:
