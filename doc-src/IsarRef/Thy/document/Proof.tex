%
\begin{isabellebody}%
\def\isabellecontext{Proof}%
%
\isadelimtheory
\isanewline
\isanewline
%
\endisadelimtheory
%
\isatagtheory
\isacommand{theory}\isamarkupfalse%
\ Proof\isanewline
\isakeyword{imports}\ Main\isanewline
\isakeyword{begin}%
\endisatagtheory
{\isafoldtheory}%
%
\isadelimtheory
%
\endisadelimtheory
%
\isamarkupchapter{Proofs%
}
\isamarkuptrue%
%
\begin{isamarkuptext}%
Proof commands perform transitions of Isar/VM machine
  configurations, which are block-structured, consisting of a stack of
  nodes with three main components: logical proof context, current
  facts, and open goals.  Isar/VM transitions are \emph{typed}
  according to the following three different modes of operation:

  \begin{descr}

  \item [\isa{{\isachardoublequote}proof{\isacharparenleft}prove{\isacharparenright}{\isachardoublequote}}] means that a new goal has just been
  stated that is now to be \emph{proven}; the next command may refine
  it by some proof method, and enter a sub-proof to establish the
  actual result.

  \item [\isa{{\isachardoublequote}proof{\isacharparenleft}state{\isacharparenright}{\isachardoublequote}}] is like a nested theory mode: the
  context may be augmented by \emph{stating} additional assumptions,
  intermediate results etc.

  \item [\isa{{\isachardoublequote}proof{\isacharparenleft}chain{\isacharparenright}{\isachardoublequote}}] is intermediate between \isa{{\isachardoublequote}proof{\isacharparenleft}state{\isacharparenright}{\isachardoublequote}} and \isa{{\isachardoublequote}proof{\isacharparenleft}prove{\isacharparenright}{\isachardoublequote}}: existing facts (i.e.\
  the contents of the special ``\indexref{}{fact}{this}\mbox{\isa{this}}'' register) have been
  just picked up in order to be used when refining the goal claimed
  next.

  \end{descr}

  The proof mode indicator may be read as a verb telling the writer
  what kind of operation may be performed next.  The corresponding
  typings of proof commands restricts the shape of well-formed proof
  texts to particular command sequences.  So dynamic arrangements of
  commands eventually turn out as static texts of a certain structure.
  \Appref{ap:refcard} gives a simplified grammar of the overall
  (extensible) language emerging that way.%
\end{isamarkuptext}%
\isamarkuptrue%
%
\isamarkupsection{Context elements \label{sec:proof-context}%
}
\isamarkuptrue%
%
\begin{isamarkuptext}%
\begin{matharray}{rcl}
    \indexdef{}{command}{fix}\mbox{\isa{\isacommand{fix}}} & : & \isartrans{proof(state)}{proof(state)} \\
    \indexdef{}{command}{assume}\mbox{\isa{\isacommand{assume}}} & : & \isartrans{proof(state)}{proof(state)} \\
    \indexdef{}{command}{presume}\mbox{\isa{\isacommand{presume}}} & : & \isartrans{proof(state)}{proof(state)} \\
    \indexdef{}{command}{def}\mbox{\isa{\isacommand{def}}} & : & \isartrans{proof(state)}{proof(state)} \\
  \end{matharray}

  The logical proof context consists of fixed variables and
  assumptions.  The former closely correspond to Skolem constants, or
  meta-level universal quantification as provided by the Isabelle/Pure
  logical framework.  Introducing some \emph{arbitrary, but fixed}
  variable via ``\mbox{\isa{\isacommand{fix}}}~\isa{x}'' results in a local value
  that may be used in the subsequent proof as any other variable or
  constant.  Furthermore, any result \isa{{\isachardoublequote}{\isasymturnstile}\ {\isasymphi}{\isacharbrackleft}x{\isacharbrackright}{\isachardoublequote}} exported from
  the context will be universally closed wrt.\ \isa{x} at the
  outermost level: \isa{{\isachardoublequote}{\isasymturnstile}\ {\isasymAnd}x{\isachardot}\ {\isasymphi}{\isacharbrackleft}x{\isacharbrackright}{\isachardoublequote}} (this is expressed in normal
  form using Isabelle's meta-variables).

  Similarly, introducing some assumption \isa{{\isasymchi}} has two effects.
  On the one hand, a local theorem is created that may be used as a
  fact in subsequent proof steps.  On the other hand, any result
  \isa{{\isachardoublequote}{\isasymchi}\ {\isasymturnstile}\ {\isasymphi}{\isachardoublequote}} exported from the context becomes conditional wrt.\
  the assumption: \isa{{\isachardoublequote}{\isasymturnstile}\ {\isasymchi}\ {\isasymLongrightarrow}\ {\isasymphi}{\isachardoublequote}}.  Thus, solving an enclosing goal
  using such a result would basically introduce a new subgoal stemming
  from the assumption.  How this situation is handled depends on the
  version of assumption command used: while \mbox{\isa{\isacommand{assume}}}
  insists on solving the subgoal by unification with some premise of
  the goal, \mbox{\isa{\isacommand{presume}}} leaves the subgoal unchanged in order
  to be proved later by the user.

  Local definitions, introduced by ``\mbox{\isa{\isacommand{def}}}~\isa{{\isachardoublequote}x\ {\isasymequiv}\ t{\isachardoublequote}}'', are achieved by combining ``\mbox{\isa{\isacommand{fix}}}~\isa{x}'' with
  another version of assumption that causes any hypothetical equation
  \isa{{\isachardoublequote}x\ {\isasymequiv}\ t{\isachardoublequote}} to be eliminated by the reflexivity rule.  Thus,
  exporting some result \isa{{\isachardoublequote}x\ {\isasymequiv}\ t\ {\isasymturnstile}\ {\isasymphi}{\isacharbrackleft}x{\isacharbrackright}{\isachardoublequote}} yields \isa{{\isachardoublequote}{\isasymturnstile}\ {\isasymphi}{\isacharbrackleft}t{\isacharbrackright}{\isachardoublequote}}.

  \begin{rail}
    'fix' (vars + 'and')
    ;
    ('assume' | 'presume') (props + 'and')
    ;
    'def' (def + 'and')
    ;
    def: thmdecl? \\ name ('==' | equiv) term termpat?
    ;
  \end{rail}

  \begin{descr}
  
  \item [\mbox{\isa{\isacommand{fix}}}~\isa{x}] introduces a local variable
  \isa{x} that is \emph{arbitrary, but fixed.}
  
  \item [\mbox{\isa{\isacommand{assume}}}~\isa{{\isachardoublequote}a{\isacharcolon}\ {\isasymphi}{\isachardoublequote}} and \mbox{\isa{\isacommand{presume}}}~\isa{{\isachardoublequote}a{\isacharcolon}\ {\isasymphi}{\isachardoublequote}}] introduce a local fact \isa{{\isachardoublequote}{\isasymphi}\ {\isasymturnstile}\ {\isasymphi}{\isachardoublequote}} by
  assumption.  Subsequent results applied to an enclosing goal (e.g.\
  by \indexref{}{command}{show}\mbox{\isa{\isacommand{show}}}) are handled as follows: \mbox{\isa{\isacommand{assume}}} expects to be able to unify with existing premises in the
  goal, while \mbox{\isa{\isacommand{presume}}} leaves \isa{{\isasymphi}} as new subgoals.
  
  Several lists of assumptions may be given (separated by
  \indexref{}{keyword}{and}\mbox{\isa{\isakeyword{and}}}; the resulting list of current facts consists
  of all of these concatenated.
  
  \item [\mbox{\isa{\isacommand{def}}}~\isa{{\isachardoublequote}x\ {\isasymequiv}\ t{\isachardoublequote}}] introduces a local
  (non-polymorphic) definition.  In results exported from the context,
  \isa{x} is replaced by \isa{t}.  Basically, ``\mbox{\isa{\isacommand{def}}}~\isa{{\isachardoublequote}x\ {\isasymequiv}\ t{\isachardoublequote}}'' abbreviates ``\mbox{\isa{\isacommand{fix}}}~\isa{x}~\mbox{\isa{\isacommand{assume}}}~\isa{{\isachardoublequote}x\ {\isasymequiv}\ t{\isachardoublequote}}'', with the resulting
  hypothetical equation solved by reflexivity.
  
  The default name for the definitional equation is \isa{x{\isacharunderscore}def}.
  Several simultaneous definitions may be given at the same time.

  \end{descr}

  The special name \indexref{}{fact}{prems}\mbox{\isa{prems}} refers to all assumptions of the
  current context as a list of theorems.  This feature should be used
  with great care!  It is better avoided in final proof texts.%
\end{isamarkuptext}%
\isamarkuptrue%
%
\isamarkupsection{Facts and forward chaining%
}
\isamarkuptrue%
%
\begin{isamarkuptext}%
\begin{matharray}{rcl}
    \indexdef{}{command}{note}\mbox{\isa{\isacommand{note}}} & : & \isartrans{proof(state)}{proof(state)} \\
    \indexdef{}{command}{then}\mbox{\isa{\isacommand{then}}} & : & \isartrans{proof(state)}{proof(chain)} \\
    \indexdef{}{command}{from}\mbox{\isa{\isacommand{from}}} & : & \isartrans{proof(state)}{proof(chain)} \\
    \indexdef{}{command}{with}\mbox{\isa{\isacommand{with}}} & : & \isartrans{proof(state)}{proof(chain)} \\
    \indexdef{}{command}{using}\mbox{\isa{\isacommand{using}}} & : & \isartrans{proof(prove)}{proof(prove)} \\
    \indexdef{}{command}{unfolding}\mbox{\isa{\isacommand{unfolding}}} & : & \isartrans{proof(prove)}{proof(prove)} \\
  \end{matharray}

  New facts are established either by assumption or proof of local
  statements.  Any fact will usually be involved in further proofs,
  either as explicit arguments of proof methods, or when forward
  chaining towards the next goal via \mbox{\isa{\isacommand{then}}} (and variants);
  \mbox{\isa{\isacommand{from}}} and \mbox{\isa{\isacommand{with}}} are composite forms
  involving \mbox{\isa{\isacommand{note}}}.  The \mbox{\isa{\isacommand{using}}} elements
  augments the collection of used facts \emph{after} a goal has been
  stated.  Note that the special theorem name \indexref{}{fact}{this}\mbox{\isa{this}} refers
  to the most recently established facts, but only \emph{before}
  issuing a follow-up claim.

  \begin{rail}
    'note' (thmdef? thmrefs + 'and')
    ;
    ('from' | 'with' | 'using' | 'unfolding') (thmrefs + 'and')
    ;
  \end{rail}

  \begin{descr}

  \item [\mbox{\isa{\isacommand{note}}}~\isa{{\isachardoublequote}a\ {\isacharequal}\ b\isactrlsub {\isadigit{1}}\ {\isasymdots}\ b\isactrlsub n{\isachardoublequote}}]
  recalls existing facts \isa{{\isachardoublequote}b\isactrlsub {\isadigit{1}}{\isacharcomma}\ {\isasymdots}{\isacharcomma}\ b\isactrlsub n{\isachardoublequote}}, binding
  the result as \isa{a}.  Note that attributes may be involved as
  well, both on the left and right hand sides.

  \item [\mbox{\isa{\isacommand{then}}}] indicates forward chaining by the current
  facts in order to establish the goal to be claimed next.  The
  initial proof method invoked to refine that will be offered the
  facts to do ``anything appropriate'' (see also
  \secref{sec:proof-steps}).  For example, method \indexref{}{method}{rule}\mbox{\isa{rule}}
  (see \secref{sec:pure-meth-att}) would typically do an elimination
  rather than an introduction.  Automatic methods usually insert the
  facts into the goal state before operation.  This provides a simple
  scheme to control relevance of facts in automated proof search.
  
  \item [\mbox{\isa{\isacommand{from}}}~\isa{b}] abbreviates ``\mbox{\isa{\isacommand{note}}}~\isa{b}~\mbox{\isa{\isacommand{then}}}''; thus \mbox{\isa{\isacommand{then}}} is
  equivalent to ``\mbox{\isa{\isacommand{from}}}~\isa{this}''.
  
  \item [\mbox{\isa{\isacommand{with}}}~\isa{{\isachardoublequote}b\isactrlsub {\isadigit{1}}\ {\isasymdots}\ b\isactrlsub n{\isachardoublequote}}]
  abbreviates ``\mbox{\isa{\isacommand{from}}}~\isa{{\isachardoublequote}b\isactrlsub {\isadigit{1}}\ {\isasymdots}\ b\isactrlsub n\ {\isasymAND}\ this{\isachardoublequote}}''; thus the forward chaining is from earlier facts together
  with the current ones.
  
  \item [\mbox{\isa{\isacommand{using}}}~\isa{{\isachardoublequote}b\isactrlsub {\isadigit{1}}\ {\isasymdots}\ b\isactrlsub n{\isachardoublequote}}] augments
  the facts being currently indicated for use by a subsequent
  refinement step (such as \indexref{}{command}{apply}\mbox{\isa{\isacommand{apply}}} or \indexref{}{command}{proof}\mbox{\isa{\isacommand{proof}}}).
  
  \item [\mbox{\isa{\isacommand{unfolding}}}~\isa{{\isachardoublequote}b\isactrlsub {\isadigit{1}}\ {\isasymdots}\ b\isactrlsub n{\isachardoublequote}}] is
  structurally similar to \mbox{\isa{\isacommand{using}}}, but unfolds definitional
  equations \isa{{\isachardoublequote}b\isactrlsub {\isadigit{1}}{\isacharcomma}\ {\isasymdots}\ b\isactrlsub n{\isachardoublequote}} throughout the goal state
  and facts.

  \end{descr}

  Forward chaining with an empty list of theorems is the same as not
  chaining at all.  Thus ``\mbox{\isa{\isacommand{from}}}~\isa{nothing}'' has no
  effect apart from entering \isa{{\isachardoublequote}prove{\isacharparenleft}chain{\isacharparenright}{\isachardoublequote}} mode, since
  \indexref{}{fact}{nothing}\mbox{\isa{nothing}} is bound to the empty list of theorems.

  Basic proof methods (such as \indexref{}{method}{rule}\mbox{\isa{rule}}) expect multiple
  facts to be given in their proper order, corresponding to a prefix
  of the premises of the rule involved.  Note that positions may be
  easily skipped using something like \mbox{\isa{\isacommand{from}}}~\isa{{\isachardoublequote}{\isacharunderscore}\ {\isasymAND}\ a\ {\isasymAND}\ b{\isachardoublequote}}, for example.  This involves the trivial rule
  \isa{{\isachardoublequote}PROP\ {\isasympsi}\ {\isasymLongrightarrow}\ PROP\ {\isasympsi}{\isachardoublequote}}, which is bound in Isabelle/Pure as
  ``\indexref{}{fact}{\_}\mbox{\isa{{\isacharunderscore}}}'' (underscore).

  Automated methods (such as \mbox{\isa{simp}} or \mbox{\isa{auto}}) just
  insert any given facts before their usual operation.  Depending on
  the kind of procedure involved, the order of facts is less
  significant here.%
\end{isamarkuptext}%
\isamarkuptrue%
%
\isamarkupsection{Goal statements \label{sec:goals}%
}
\isamarkuptrue%
%
\begin{isamarkuptext}%
\begin{matharray}{rcl}
    \indexdef{}{command}{lemma}\mbox{\isa{\isacommand{lemma}}} & : & \isartrans{local{\dsh}theory}{proof(prove)} \\
    \indexdef{}{command}{theorem}\mbox{\isa{\isacommand{theorem}}} & : & \isartrans{local{\dsh}theory}{proof(prove)} \\
    \indexdef{}{command}{corollary}\mbox{\isa{\isacommand{corollary}}} & : & \isartrans{local{\dsh}theory}{proof(prove)} \\
    \indexdef{}{command}{have}\mbox{\isa{\isacommand{have}}} & : & \isartrans{proof(state) ~|~ proof(chain)}{proof(prove)} \\
    \indexdef{}{command}{show}\mbox{\isa{\isacommand{show}}} & : & \isartrans{proof(state) ~|~ proof(chain)}{proof(prove)} \\
    \indexdef{}{command}{hence}\mbox{\isa{\isacommand{hence}}} & : & \isartrans{proof(state)}{proof(prove)} \\
    \indexdef{}{command}{thus}\mbox{\isa{\isacommand{thus}}} & : & \isartrans{proof(state)}{proof(prove)} \\
    \indexdef{}{command}{print\_statement}\mbox{\isa{\isacommand{print{\isacharunderscore}statement}}}\isa{{\isachardoublequote}\isactrlsup {\isacharasterisk}{\isachardoublequote}} & : & \isarkeep{theory~|~proof} \\
  \end{matharray}

  From a theory context, proof mode is entered by an initial goal
  command such as \mbox{\isa{\isacommand{lemma}}}, \mbox{\isa{\isacommand{theorem}}}, or
  \mbox{\isa{\isacommand{corollary}}}.  Within a proof, new claims may be
  introduced locally as well; four variants are available here to
  indicate whether forward chaining of facts should be performed
  initially (via \indexref{}{command}{then}\mbox{\isa{\isacommand{then}}}), and whether the final result
  is meant to solve some pending goal.

  Goals may consist of multiple statements, resulting in a list of
  facts eventually.  A pending multi-goal is internally represented as
  a meta-level conjunction (printed as \isa{{\isachardoublequote}{\isacharampersand}{\isacharampersand}{\isachardoublequote}}), which is usually
  split into the corresponding number of sub-goals prior to an initial
  method application, via \indexref{}{command}{proof}\mbox{\isa{\isacommand{proof}}}
  (\secref{sec:proof-steps}) or \indexref{}{command}{apply}\mbox{\isa{\isacommand{apply}}}
  (\secref{sec:tactic-commands}).  The \indexref{}{method}{induct}\mbox{\isa{induct}} method
  covered in \secref{sec:cases-induct} acts on multiple claims
  simultaneously.

  Claims at the theory level may be either in short or long form.  A
  short goal merely consists of several simultaneous propositions
  (often just one).  A long goal includes an explicit context
  specification for the subsequent conclusion, involving local
  parameters and assumptions.  Here the role of each part of the
  statement is explicitly marked by separate keywords (see also
  \secref{sec:locale}); the local assumptions being introduced here
  are available as \indexref{}{fact}{assms}\mbox{\isa{assms}} in the proof.  Moreover, there
  are two kinds of conclusions: \indexdef{}{element}{shows}\mbox{\isa{\isakeyword{shows}}} states several
  simultaneous propositions (essentially a big conjunction), while
  \indexdef{}{element}{obtains}\mbox{\isa{\isakeyword{obtains}}} claims several simultaneous simultaneous
  contexts of (essentially a big disjunction of eliminated parameters
  and assumptions, cf.\ \secref{sec:obtain}).

  \begin{rail}
    ('lemma' | 'theorem' | 'corollary') target? (goal | longgoal)
    ;
    ('have' | 'show' | 'hence' | 'thus') goal
    ;
    'print\_statement' modes? thmrefs
    ;
  
    goal: (props + 'and')
    ;
    longgoal: thmdecl? (contextelem *) conclusion
    ;
    conclusion: 'shows' goal | 'obtains' (parname? case + '|')
    ;
    case: (vars + 'and') 'where' (props + 'and')
    ;
  \end{rail}

  \begin{descr}
  
  \item [\mbox{\isa{\isacommand{lemma}}}~\isa{{\isachardoublequote}a{\isacharcolon}\ {\isasymphi}{\isachardoublequote}}] enters proof mode with
  \isa{{\isasymphi}} as main goal, eventually resulting in some fact \isa{{\isachardoublequote}{\isasymturnstile}\ {\isasymphi}{\isachardoublequote}} to be put back into the target context.  An additional
  \railnonterm{context} specification may build up an initial proof
  context for the subsequent claim; this includes local definitions
  and syntax as well, see the definition of \mbox{\isa{contextelem}} in
  \secref{sec:locale}.
  
  \item [\mbox{\isa{\isacommand{theorem}}}~\isa{{\isachardoublequote}a{\isacharcolon}\ {\isasymphi}{\isachardoublequote}} and \mbox{\isa{\isacommand{corollary}}}~\isa{{\isachardoublequote}a{\isacharcolon}\ {\isasymphi}{\isachardoublequote}}] are essentially the same as \mbox{\isa{\isacommand{lemma}}}~\isa{{\isachardoublequote}a{\isacharcolon}\ {\isasymphi}{\isachardoublequote}}, but the facts are internally marked as
  being of a different kind.  This discrimination acts like a formal
  comment.
  
  \item [\mbox{\isa{\isacommand{have}}}~\isa{{\isachardoublequote}a{\isacharcolon}\ {\isasymphi}{\isachardoublequote}}] claims a local goal,
  eventually resulting in a fact within the current logical context.
  This operation is completely independent of any pending sub-goals of
  an enclosing goal statements, so \mbox{\isa{\isacommand{have}}} may be freely
  used for experimental exploration of potential results within a
  proof body.
  
  \item [\mbox{\isa{\isacommand{show}}}~\isa{{\isachardoublequote}a{\isacharcolon}\ {\isasymphi}{\isachardoublequote}}] is like \mbox{\isa{\isacommand{have}}}~\isa{{\isachardoublequote}a{\isacharcolon}\ {\isasymphi}{\isachardoublequote}} plus a second stage to refine some pending
  sub-goal for each one of the finished result, after having been
  exported into the corresponding context (at the head of the
  sub-proof of this \mbox{\isa{\isacommand{show}}} command).
  
  To accommodate interactive debugging, resulting rules are printed
  before being applied internally.  Even more, interactive execution
  of \mbox{\isa{\isacommand{show}}} predicts potential failure and displays the
  resulting error as a warning beforehand.  Watch out for the
  following message:

  %FIXME proper antiquitation
  \begin{ttbox}
  Problem! Local statement will fail to solve any pending goal
  \end{ttbox}
  
  \item [\mbox{\isa{\isacommand{hence}}}] abbreviates ``\mbox{\isa{\isacommand{then}}}~\mbox{\isa{\isacommand{have}}}'', i.e.\ claims a local goal to be proven by forward
  chaining the current facts.  Note that \mbox{\isa{\isacommand{hence}}} is also
  equivalent to ``\mbox{\isa{\isacommand{from}}}~\isa{this}~\mbox{\isa{\isacommand{have}}}''.
  
  \item [\mbox{\isa{\isacommand{thus}}}] abbreviates ``\mbox{\isa{\isacommand{then}}}~\mbox{\isa{\isacommand{show}}}''.  Note that \mbox{\isa{\isacommand{thus}}} is also equivalent to
  ``\mbox{\isa{\isacommand{from}}}~\isa{this}~\mbox{\isa{\isacommand{show}}}''.
  
  \item [\mbox{\isa{\isacommand{print{\isacharunderscore}statement}}}~\isa{a}] prints facts from the
  current theory or proof context in long statement form, according to
  the syntax for \mbox{\isa{\isacommand{lemma}}} given above.

  \end{descr}

  Any goal statement causes some term abbreviations (such as
  \indexref{}{variable}{?thesis}\mbox{\isa{{\isacharquery}thesis}}) to be bound automatically, see also
  \secref{sec:term-abbrev}.  Furthermore, the local context of a
  (non-atomic) goal is provided via the \indexref{}{case}{rule\_context}\mbox{\isa{rule{\isacharunderscore}context}} case.

  The optional case names of \indexref{}{element}{obtains}\mbox{\isa{\isakeyword{obtains}}} have a twofold
  meaning: (1) during the of this claim they refer to the the local
  context introductions, (2) the resulting rule is annotated
  accordingly to support symbolic case splits when used with the
  \indexref{}{method}{cases}\mbox{\isa{cases}} method (cf.  \secref{sec:cases-induct}).

  \medskip

  \begin{warn}
    Isabelle/Isar suffers theory-level goal statements to contain
    \emph{unbound schematic variables}, although this does not conform
    to the aim of human-readable proof documents!  The main problem
    with schematic goals is that the actual outcome is usually hard to
    predict, depending on the behavior of the proof methods applied
    during the course of reasoning.  Note that most semi-automated
    methods heavily depend on several kinds of implicit rule
    declarations within the current theory context.  As this would
    also result in non-compositional checking of sub-proofs,
    \emph{local goals} are not allowed to be schematic at all.
    Nevertheless, schematic goals do have their use in Prolog-style
    interactive synthesis of proven results, usually by stepwise
    refinement via emulation of traditional Isabelle tactic scripts
    (see also \secref{sec:tactic-commands}).  In any case, users
    should know what they are doing.
  \end{warn}%
\end{isamarkuptext}%
\isamarkuptrue%
%
\isamarkupsection{Initial and terminal proof steps \label{sec:proof-steps}%
}
\isamarkuptrue%
%
\begin{isamarkuptext}%
\begin{matharray}{rcl}
    \indexdef{}{command}{proof}\mbox{\isa{\isacommand{proof}}} & : & \isartrans{proof(prove)}{proof(state)} \\
    \indexdef{}{command}{qed}\mbox{\isa{\isacommand{qed}}} & : & \isartrans{proof(state)}{proof(state) ~|~ theory} \\
    \indexdef{}{command}{by}\mbox{\isa{\isacommand{by}}} & : & \isartrans{proof(prove)}{proof(state) ~|~ theory} \\
    \indexdef{}{command}{..}\mbox{\isa{\isacommand{{\isachardot}{\isachardot}}}} & : & \isartrans{proof(prove)}{proof(state) ~|~ theory} \\
    \indexdef{}{command}{.}\mbox{\isa{\isacommand{{\isachardot}}}} & : & \isartrans{proof(prove)}{proof(state) ~|~ theory} \\
    \indexdef{}{command}{sorry}\mbox{\isa{\isacommand{sorry}}} & : & \isartrans{proof(prove)}{proof(state) ~|~ theory} \\
  \end{matharray}

  Arbitrary goal refinement via tactics is considered harmful.
  Structured proof composition in Isar admits proof methods to be
  invoked in two places only.

  \begin{enumerate}

  \item An \emph{initial} refinement step \indexref{}{command}{proof}\mbox{\isa{\isacommand{proof}}}~\isa{{\isachardoublequote}m\isactrlsub {\isadigit{1}}{\isachardoublequote}} reduces a newly stated goal to a number
  of sub-goals that are to be solved later.  Facts are passed to
  \isa{{\isachardoublequote}m\isactrlsub {\isadigit{1}}{\isachardoublequote}} for forward chaining, if so indicated by \isa{{\isachardoublequote}proof{\isacharparenleft}chain{\isacharparenright}{\isachardoublequote}} mode.
  
  \item A \emph{terminal} conclusion step \indexref{}{command}{qed}\mbox{\isa{\isacommand{qed}}}~\isa{{\isachardoublequote}m\isactrlsub {\isadigit{2}}{\isachardoublequote}} is intended to solve remaining goals.  No facts are
  passed to \isa{{\isachardoublequote}m\isactrlsub {\isadigit{2}}{\isachardoublequote}}.

  \end{enumerate}

  The only other (proper) way to affect pending goals in a proof body
  is by \indexref{}{command}{show}\mbox{\isa{\isacommand{show}}}, which involves an explicit statement of
  what is to be solved eventually.  Thus we avoid the fundamental
  problem of unstructured tactic scripts that consist of numerous
  consecutive goal transformations, with invisible effects.

  \medskip As a general rule of thumb for good proof style, initial
  proof methods should either solve the goal completely, or constitute
  some well-understood reduction to new sub-goals.  Arbitrary
  automatic proof tools that are prone leave a large number of badly
  structured sub-goals are no help in continuing the proof document in
  an intelligible manner.

  Unless given explicitly by the user, the default initial method is
  ``\indexref{}{method}{rule}\mbox{\isa{rule}}'', which applies a single standard elimination
  or introduction rule according to the topmost symbol involved.
  There is no separate default terminal method.  Any remaining goals
  are always solved by assumption in the very last step.

  \begin{rail}
    'proof' method?
    ;
    'qed' method?
    ;
    'by' method method?
    ;
    ('.' | '..' | 'sorry')
    ;
  \end{rail}

  \begin{descr}
  
  \item [\mbox{\isa{\isacommand{proof}}}~\isa{{\isachardoublequote}m\isactrlsub {\isadigit{1}}{\isachardoublequote}}] refines the goal by
  proof method \isa{{\isachardoublequote}m\isactrlsub {\isadigit{1}}{\isachardoublequote}}; facts for forward chaining are
  passed if so indicated by \isa{{\isachardoublequote}proof{\isacharparenleft}chain{\isacharparenright}{\isachardoublequote}} mode.
  
  \item [\mbox{\isa{\isacommand{qed}}}~\isa{{\isachardoublequote}m\isactrlsub {\isadigit{2}}{\isachardoublequote}}] refines any remaining
  goals by proof method \isa{{\isachardoublequote}m\isactrlsub {\isadigit{2}}{\isachardoublequote}} and concludes the
  sub-proof by assumption.  If the goal had been \isa{{\isachardoublequote}show{\isachardoublequote}} (or
  \isa{{\isachardoublequote}thus{\isachardoublequote}}), some pending sub-goal is solved as well by the rule
  resulting from the result \emph{exported} into the enclosing goal
  context.  Thus \isa{{\isachardoublequote}qed{\isachardoublequote}} may fail for two reasons: either \isa{{\isachardoublequote}m\isactrlsub {\isadigit{2}}{\isachardoublequote}} fails, or the resulting rule does not fit to any
  pending goal\footnote{This includes any additional ``strong''
  assumptions as introduced by \mbox{\isa{\isacommand{assume}}}.} of the enclosing
  context.  Debugging such a situation might involve temporarily
  changing \mbox{\isa{\isacommand{show}}} into \mbox{\isa{\isacommand{have}}}, or weakening the
  local context by replacing occurrences of \mbox{\isa{\isacommand{assume}}} by
  \mbox{\isa{\isacommand{presume}}}.
  
  \item [\mbox{\isa{\isacommand{by}}}~\isa{{\isachardoublequote}m\isactrlsub {\isadigit{1}}\ m\isactrlsub {\isadigit{2}}{\isachardoublequote}}] is a
  \emph{terminal proof}\index{proof!terminal}; it abbreviates
  \mbox{\isa{\isacommand{proof}}}~\isa{{\isachardoublequote}m\isactrlsub {\isadigit{1}}{\isachardoublequote}}~\isa{{\isachardoublequote}qed{\isachardoublequote}}~\isa{{\isachardoublequote}m\isactrlsub {\isadigit{2}}{\isachardoublequote}}, but with backtracking across both methods.  Debugging
  an unsuccessful \mbox{\isa{\isacommand{by}}}~\isa{{\isachardoublequote}m\isactrlsub {\isadigit{1}}\ m\isactrlsub {\isadigit{2}}{\isachardoublequote}}
  command can be done by expanding its definition; in many cases
  \mbox{\isa{\isacommand{proof}}}~\isa{{\isachardoublequote}m\isactrlsub {\isadigit{1}}{\isachardoublequote}} (or even \isa{{\isachardoublequote}apply{\isachardoublequote}}~\isa{{\isachardoublequote}m\isactrlsub {\isadigit{1}}{\isachardoublequote}}) is already sufficient to see the
  problem.

  \item [``\mbox{\isa{\isacommand{{\isachardot}{\isachardot}}}}''] is a \emph{default
  proof}\index{proof!default}; it abbreviates \mbox{\isa{\isacommand{by}}}~\isa{{\isachardoublequote}rule{\isachardoublequote}}.

  \item [``\mbox{\isa{\isacommand{{\isachardot}}}}''] is a \emph{trivial
  proof}\index{proof!trivial}; it abbreviates \mbox{\isa{\isacommand{by}}}~\isa{{\isachardoublequote}this{\isachardoublequote}}.
  
  \item [\mbox{\isa{\isacommand{sorry}}}] is a \emph{fake proof}\index{proof!fake}
  pretending to solve the pending claim without further ado.  This
  only works in interactive development, or if the \verb|quick_and_dirty| flag is enabled (in ML).  Facts emerging from fake
  proofs are not the real thing.  Internally, each theorem container
  is tainted by an oracle invocation, which is indicated as ``\isa{{\isachardoublequote}{\isacharbrackleft}{\isacharbang}{\isacharbrackright}{\isachardoublequote}}'' in the printed result.
  
  The most important application of \mbox{\isa{\isacommand{sorry}}} is to support
  experimentation and top-down proof development.

  \end{descr}%
\end{isamarkuptext}%
\isamarkuptrue%
%
\isamarkupsection{Fundamental methods and attributes \label{sec:pure-meth-att}%
}
\isamarkuptrue%
%
\begin{isamarkuptext}%
The following proof methods and attributes refer to basic logical
  operations of Isar.  Further methods and attributes are provided by
  several generic and object-logic specific tools and packages (see
  \chref{ch:gen-tools} and \chref{ch:hol}).

  \begin{matharray}{rcl}
    \indexdef{}{method}{-}\mbox{\isa{{\isacharminus}}} & : & \isarmeth \\
    \indexdef{}{method}{fact}\mbox{\isa{fact}} & : & \isarmeth \\
    \indexdef{}{method}{assumption}\mbox{\isa{assumption}} & : & \isarmeth \\
    \indexdef{}{method}{this}\mbox{\isa{this}} & : & \isarmeth \\
    \indexdef{}{method}{rule}\mbox{\isa{rule}} & : & \isarmeth \\
    \indexdef{}{method}{iprover}\mbox{\isa{iprover}} & : & \isarmeth \\[0.5ex]
    \indexdef{}{attribute}{intro}\mbox{\isa{intro}} & : & \isaratt \\
    \indexdef{}{attribute}{elim}\mbox{\isa{elim}} & : & \isaratt \\
    \indexdef{}{attribute}{dest}\mbox{\isa{dest}} & : & \isaratt \\
    \indexdef{}{attribute}{rule}\mbox{\isa{rule}} & : & \isaratt \\[0.5ex]
    \indexdef{}{attribute}{OF}\mbox{\isa{OF}} & : & \isaratt \\
    \indexdef{}{attribute}{of}\mbox{\isa{of}} & : & \isaratt \\
    \indexdef{}{attribute}{where}\mbox{\isa{where}} & : & \isaratt \\
  \end{matharray}

  \begin{rail}
    'fact' thmrefs?
    ;
    'rule' thmrefs?
    ;
    'iprover' ('!' ?) (rulemod *)
    ;
    rulemod: ('intro' | 'elim' | 'dest') ((('!' | () | '?') nat?) | 'del') ':' thmrefs
    ;
    ('intro' | 'elim' | 'dest') ('!' | () | '?') nat?
    ;
    'rule' 'del'
    ;
    'OF' thmrefs
    ;
    'of' insts ('concl' ':' insts)?
    ;
    'where' ((name | var | typefree | typevar) '=' (type | term) * 'and')
    ;
  \end{rail}

  \begin{descr}
  
  \item [``\mbox{\isa{{\isacharminus}}}'' (minus)] does nothing but insert the
  forward chaining facts as premises into the goal.  Note that command
  \indexref{}{command}{proof}\mbox{\isa{\isacommand{proof}}} without any method actually performs a single
  reduction step using the \indexref{}{method}{rule}\mbox{\isa{rule}} method; thus a plain
  \emph{do-nothing} proof step would be ``\mbox{\isa{\isacommand{proof}}}~\isa{{\isachardoublequote}{\isacharminus}{\isachardoublequote}}'' rather than \mbox{\isa{\isacommand{proof}}} alone.
  
  \item [\mbox{\isa{fact}}~\isa{{\isachardoublequote}a\isactrlsub {\isadigit{1}}\ {\isasymdots}\ a\isactrlsub n{\isachardoublequote}}] composes
  some fact from \isa{{\isachardoublequote}a\isactrlsub {\isadigit{1}}{\isacharcomma}\ {\isasymdots}{\isacharcomma}\ a\isactrlsub n{\isachardoublequote}} (or implicitly from
  the current proof context) modulo unification of schematic type and
  term variables.  The rule structure is not taken into account, i.e.\
  meta-level implication is considered atomic.  This is the same
  principle underlying literal facts (cf.\ \secref{sec:syn-att}):
  ``\mbox{\isa{\isacommand{have}}}~\isa{{\isachardoublequote}{\isasymphi}{\isachardoublequote}}~\mbox{\isa{\isacommand{by}}}~\isa{fact}'' is
  equivalent to ``\mbox{\isa{\isacommand{note}}}~\verb|`|\isa{{\isasymphi}}\verb|`|'' provided that \isa{{\isachardoublequote}{\isasymturnstile}\ {\isasymphi}{\isachardoublequote}} is an instance of some known
  \isa{{\isachardoublequote}{\isasymturnstile}\ {\isasymphi}{\isachardoublequote}} in the proof context.
  
  \item [\mbox{\isa{assumption}}] solves some goal by a single assumption
  step.  All given facts are guaranteed to participate in the
  refinement; this means there may be only 0 or 1 in the first place.
  Recall that \mbox{\isa{\isacommand{qed}}} (\secref{sec:proof-steps}) already
  concludes any remaining sub-goals by assumption, so structured
  proofs usually need not quote the \mbox{\isa{assumption}} method at
  all.
  
  \item [\mbox{\isa{this}}] applies all of the current facts directly as
  rules.  Recall that ``\mbox{\isa{\isacommand{{\isachardot}}}}'' (dot) abbreviates ``\mbox{\isa{\isacommand{by}}}~\isa{this}''.
  
  \item [\mbox{\isa{rule}}~\isa{{\isachardoublequote}a\isactrlsub {\isadigit{1}}\ {\isasymdots}\ a\isactrlsub n{\isachardoublequote}}] applies some
  rule given as argument in backward manner; facts are used to reduce
  the rule before applying it to the goal.  Thus \mbox{\isa{rule}}
  without facts is plain introduction, while with facts it becomes
  elimination.
  
  When no arguments are given, the \mbox{\isa{rule}} method tries to pick
  appropriate rules automatically, as declared in the current context
  using the \mbox{\isa{intro}}, \mbox{\isa{elim}}, \mbox{\isa{dest}}
  attributes (see below).  This is the default behavior of \mbox{\isa{\isacommand{proof}}} and ``\mbox{\isa{\isacommand{{\isachardot}{\isachardot}}}}'' (double-dot) steps (see
  \secref{sec:proof-steps}).
  
  \item [\mbox{\isa{iprover}}] performs intuitionistic proof search,
  depending on specifically declared rules from the context, or given
  as explicit arguments.  Chained facts are inserted into the goal
  before commencing proof search; ``\mbox{\isa{iprover}}\isa{{\isachardoublequote}{\isacharbang}{\isachardoublequote}}'' 
  means to include the current \mbox{\isa{prems}} as well.
  
  Rules need to be classified as \mbox{\isa{intro}}, \mbox{\isa{elim}}, or \mbox{\isa{dest}}; here the ``\isa{{\isachardoublequote}{\isacharbang}{\isachardoublequote}}'' indicator
  refers to ``safe'' rules, which may be applied aggressively (without
  considering back-tracking later).  Rules declared with ``\isa{{\isachardoublequote}{\isacharquery}{\isachardoublequote}}'' are ignored in proof search (the single-step \mbox{\isa{rule}}
  method still observes these).  An explicit weight annotation may be
  given as well; otherwise the number of rule premises will be taken
  into account here.
  
  \item [\mbox{\isa{intro}}, \mbox{\isa{elim}}, and \mbox{\isa{dest}}]
  declare introduction, elimination, and destruct rules, to be used
  with the \mbox{\isa{rule}} and \mbox{\isa{iprover}} methods.  Note that
  the latter will ignore rules declared with ``\isa{{\isachardoublequote}{\isacharquery}{\isachardoublequote}}'', while
  ``\isa{{\isachardoublequote}{\isacharbang}{\isachardoublequote}}''  are used most aggressively.
  
  The classical reasoner (see \secref{sec:classical}) introduces its
  own variants of these attributes; use qualified names to access the
  present versions of Isabelle/Pure, i.e.\ \mbox{\isa{Pure{\isachardot}intro}}.
  
  \item [\mbox{\isa{rule}}~\isa{del}] undeclares introduction,
  elimination, or destruct rules.
  
  \item [\mbox{\isa{OF}}~\isa{{\isachardoublequote}a\isactrlsub {\isadigit{1}}\ {\isasymdots}\ a\isactrlsub n{\isachardoublequote}}] applies some
  theorem to all of the given rules \isa{{\isachardoublequote}a\isactrlsub {\isadigit{1}}{\isacharcomma}\ {\isasymdots}{\isacharcomma}\ a\isactrlsub n{\isachardoublequote}}
  (in parallel).  This corresponds to the \verb|"op MRS"| operation in
  ML, but note the reversed order.  Positions may be effectively
  skipped by including ``\isa{{\isacharunderscore}}'' (underscore) as argument.
  
  \item [\mbox{\isa{of}}~\isa{{\isachardoublequote}t\isactrlsub {\isadigit{1}}\ {\isasymdots}\ t\isactrlsub n{\isachardoublequote}}] performs
  positional instantiation of term variables.  The terms \isa{{\isachardoublequote}t\isactrlsub {\isadigit{1}}{\isacharcomma}\ {\isasymdots}{\isacharcomma}\ t\isactrlsub n{\isachardoublequote}} are substituted for any schematic
  variables occurring in a theorem from left to right; ``\isa{{\isacharunderscore}}'' (underscore) indicates to skip a position.  Arguments following
  a ``\mbox{\isa{\isakeyword{concl}}}\isa{{\isachardoublequote}{\isacharcolon}{\isachardoublequote}}'' specification refer to positions
  of the conclusion of a rule.
  
  \item [\mbox{\isa{where}}~\isa{{\isachardoublequote}x\isactrlsub {\isadigit{1}}\ {\isacharequal}\ t\isactrlsub {\isadigit{1}}\ {\isasymAND}\ {\isasymdots}\ x\isactrlsub n\ {\isacharequal}\ t\isactrlsub n{\isachardoublequote}}] performs named instantiation of schematic
  type and term variables occurring in a theorem.  Schematic variables
  have to be specified on the left-hand side (e.g.\ \isa{{\isachardoublequote}{\isacharquery}x{\isadigit{1}}{\isachardot}{\isadigit{3}}{\isachardoublequote}}).
  The question mark may be omitted if the variable name is a plain
  identifier without index.  As type instantiations are inferred from
  term instantiations, explicit type instantiations are seldom
  necessary.

  \end{descr}%
\end{isamarkuptext}%
\isamarkuptrue%
%
\isamarkupsection{Term abbreviations \label{sec:term-abbrev}%
}
\isamarkuptrue%
%
\begin{isamarkuptext}%
\begin{matharray}{rcl}
    \indexdef{}{command}{let}\mbox{\isa{\isacommand{let}}} & : & \isartrans{proof(state)}{proof(state)} \\
    \indexdef{}{keyword}{is}\mbox{\isa{\isakeyword{is}}} & : & syntax \\
  \end{matharray}

  Abbreviations may be either bound by explicit \mbox{\isa{\isacommand{let}}}~\isa{{\isachardoublequote}p\ {\isasymequiv}\ t{\isachardoublequote}} statements, or by annotating assumptions or
  goal statements with a list of patterns ``\isa{{\isachardoublequote}{\isacharparenleft}{\isasymIS}\ p\isactrlsub {\isadigit{1}}\ {\isasymdots}\ p\isactrlsub n{\isacharparenright}{\isachardoublequote}}''.  In both cases, higher-order matching is invoked to
  bind extra-logical term variables, which may be either named
  schematic variables of the form \isa{{\isacharquery}x}, or nameless dummies
  ``\mbox{\isa{{\isacharunderscore}}}'' (underscore). Note that in the \mbox{\isa{\isacommand{let}}}
  form the patterns occur on the left-hand side, while the \mbox{\isa{\isakeyword{is}}} patterns are in postfix position.

  Polymorphism of term bindings is handled in Hindley-Milner style,
  similar to ML.  Type variables referring to local assumptions or
  open goal statements are \emph{fixed}, while those of finished
  results or bound by \mbox{\isa{\isacommand{let}}} may occur in \emph{arbitrary}
  instances later.  Even though actual polymorphism should be rarely
  used in practice, this mechanism is essential to achieve proper
  incremental type-inference, as the user proceeds to build up the
  Isar proof text from left to right.

  \medskip Term abbreviations are quite different from local
  definitions as introduced via \mbox{\isa{\isacommand{def}}} (see
  \secref{sec:proof-context}).  The latter are visible within the
  logic as actual equations, while abbreviations disappear during the
  input process just after type checking.  Also note that \mbox{\isa{\isacommand{def}}} does not support polymorphism.

  \begin{rail}
    'let' ((term + 'and') '=' term + 'and')
    ;  
  \end{rail}

  The syntax of \mbox{\isa{\isakeyword{is}}} patterns follows \railnonterm{termpat}
  or \railnonterm{proppat} (see \secref{sec:term-decls}).

  \begin{descr}

  \item [\mbox{\isa{\isacommand{let}}}~\isa{{\isachardoublequote}p\isactrlsub {\isadigit{1}}\ {\isacharequal}\ t\isactrlsub {\isadigit{1}}\ {\isasymAND}\ {\isasymdots}\ p\isactrlsub n\ {\isacharequal}\ t\isactrlsub n{\isachardoublequote}}] binds any text variables in patterns \isa{{\isachardoublequote}p\isactrlsub {\isadigit{1}}{\isacharcomma}\ {\isasymdots}{\isacharcomma}\ p\isactrlsub n{\isachardoublequote}} by simultaneous higher-order matching
  against terms \isa{{\isachardoublequote}t\isactrlsub {\isadigit{1}}{\isacharcomma}\ {\isasymdots}{\isacharcomma}\ t\isactrlsub n{\isachardoublequote}}.

  \item [\isa{{\isachardoublequote}{\isacharparenleft}{\isasymIS}\ p\isactrlsub {\isadigit{1}}\ {\isasymdots}\ p\isactrlsub n{\isacharparenright}{\isachardoublequote}}] resembles \mbox{\isa{\isacommand{let}}}, but matches \isa{{\isachardoublequote}p\isactrlsub {\isadigit{1}}{\isacharcomma}\ {\isasymdots}{\isacharcomma}\ p\isactrlsub n{\isachardoublequote}} against the
  preceding statement.  Also note that \mbox{\isa{\isakeyword{is}}} is not a
  separate command, but part of others (such as \mbox{\isa{\isacommand{assume}}},
  \mbox{\isa{\isacommand{have}}} etc.).

  \end{descr}

  Some \emph{implicit} term abbreviations\index{term abbreviations}
  for goals and facts are available as well.  For any open goal,
  \indexref{}{variable}{thesis}\mbox{\isa{thesis}} refers to its object-level statement,
  abstracted over any meta-level parameters (if present).  Likewise,
  \indexref{}{variable}{this}\mbox{\isa{this}} is bound for fact statements resulting from
  assumptions or finished goals.  In case \mbox{\isa{this}} refers to
  an object-logic statement that is an application \isa{{\isachardoublequote}f\ t{\isachardoublequote}}, then
  \isa{t} is bound to the special text variable ``\mbox{\isa{{\isasymdots}}}''
  (three dots).  The canonical application of this convenience are
  calculational proofs (see \secref{sec:calculation}).%
\end{isamarkuptext}%
\isamarkuptrue%
%
\isamarkupsection{Block structure%
}
\isamarkuptrue%
%
\begin{isamarkuptext}%
\begin{matharray}{rcl}
    \indexdef{}{command}{next}\mbox{\isa{\isacommand{next}}} & : & \isartrans{proof(state)}{proof(state)} \\
    \indexdef{}{command}{\{}\mbox{\isa{\isacommand{{\isacharbraceleft}}}} & : & \isartrans{proof(state)}{proof(state)} \\
    \indexdef{}{command}{\}}\mbox{\isa{\isacommand{{\isacharbraceright}}}} & : & \isartrans{proof(state)}{proof(state)} \\
  \end{matharray}

  While Isar is inherently block-structured, opening and closing
  blocks is mostly handled rather casually, with little explicit
  user-intervention.  Any local goal statement automatically opens
  \emph{two} internal blocks, which are closed again when concluding
  the sub-proof (by \mbox{\isa{\isacommand{qed}}} etc.).  Sections of different
  context within a sub-proof may be switched via \mbox{\isa{\isacommand{next}}},
  which is just a single block-close followed by block-open again.
  The effect of \mbox{\isa{\isacommand{next}}} is to reset the local proof context;
  there is no goal focus involved here!

  For slightly more advanced applications, there are explicit block
  parentheses as well.  These typically achieve a stronger forward
  style of reasoning.

  \begin{descr}

  \item [\mbox{\isa{\isacommand{next}}}] switches to a fresh block within a
  sub-proof, resetting the local context to the initial one.

  \item [\mbox{\isa{\isacommand{{\isacharbraceleft}}}} and \mbox{\isa{\isacommand{{\isacharbraceright}}}}] explicitly open and close
  blocks.  Any current facts pass through ``\mbox{\isa{\isacommand{{\isacharbraceleft}}}}''
  unchanged, while ``\mbox{\isa{\isacommand{{\isacharbraceright}}}}'' causes any result to be
  \emph{exported} into the enclosing context.  Thus fixed variables
  are generalized, assumptions discharged, and local definitions
  unfolded (cf.\ \secref{sec:proof-context}).  There is no difference
  of \mbox{\isa{\isacommand{assume}}} and \mbox{\isa{\isacommand{presume}}} in this mode of
  forward reasoning --- in contrast to plain backward reasoning with
  the result exported at \mbox{\isa{\isacommand{show}}} time.

  \end{descr}%
\end{isamarkuptext}%
\isamarkuptrue%
%
\isamarkupsection{Emulating tactic scripts \label{sec:tactic-commands}%
}
\isamarkuptrue%
%
\begin{isamarkuptext}%
The Isar provides separate commands to accommodate tactic-style
  proof scripts within the same system.  While being outside the
  orthodox Isar proof language, these might come in handy for
  interactive exploration and debugging, or even actual tactical proof
  within new-style theories (to benefit from document preparation, for
  example).  See also \secref{sec:tactics} for actual tactics, that
  have been encapsulated as proof methods.  Proper proof methods may
  be used in scripts, too.

  \begin{matharray}{rcl}
    \indexdef{}{command}{apply}\mbox{\isa{\isacommand{apply}}}\isa{{\isachardoublequote}\isactrlsup {\isacharasterisk}{\isachardoublequote}} & : & \isartrans{proof(prove)}{proof(prove)} \\
    \indexdef{}{command}{apply\_end}\mbox{\isa{\isacommand{apply{\isacharunderscore}end}}}\isa{{\isachardoublequote}\isactrlsup {\isacharasterisk}{\isachardoublequote}} & : & \isartrans{proof(state)}{proof(state)} \\
    \indexdef{}{command}{done}\mbox{\isa{\isacommand{done}}}\isa{{\isachardoublequote}\isactrlsup {\isacharasterisk}{\isachardoublequote}} & : & \isartrans{proof(prove)}{proof(state)} \\
    \indexdef{}{command}{defer}\mbox{\isa{\isacommand{defer}}}\isa{{\isachardoublequote}\isactrlsup {\isacharasterisk}{\isachardoublequote}} & : & \isartrans{proof}{proof} \\
    \indexdef{}{command}{prefer}\mbox{\isa{\isacommand{prefer}}}\isa{{\isachardoublequote}\isactrlsup {\isacharasterisk}{\isachardoublequote}} & : & \isartrans{proof}{proof} \\
    \indexdef{}{command}{back}\mbox{\isa{\isacommand{back}}}\isa{{\isachardoublequote}\isactrlsup {\isacharasterisk}{\isachardoublequote}} & : & \isartrans{proof}{proof} \\
  \end{matharray}

  \begin{rail}
    ( 'apply' | 'apply\_end' ) method
    ;
    'defer' nat?
    ;
    'prefer' nat
    ;
  \end{rail}

  \begin{descr}

  \item [\mbox{\isa{\isacommand{apply}}}~\isa{m}] applies proof method \isa{m}
  in initial position, but unlike \mbox{\isa{\isacommand{proof}}} it retains
  ``\isa{{\isachardoublequote}proof{\isacharparenleft}prove{\isacharparenright}{\isachardoublequote}}'' mode.  Thus consecutive method
  applications may be given just as in tactic scripts.
  
  Facts are passed to \isa{m} as indicated by the goal's
  forward-chain mode, and are \emph{consumed} afterwards.  Thus any
  further \mbox{\isa{\isacommand{apply}}} command would always work in a purely
  backward manner.
  
  \item [\mbox{\isa{\isacommand{apply{\isacharunderscore}end}}}~\isa{{\isachardoublequote}m{\isachardoublequote}}] applies proof method
  \isa{m} as if in terminal position.  Basically, this simulates a
  multi-step tactic script for \mbox{\isa{\isacommand{qed}}}, but may be given
  anywhere within the proof body.
  
  No facts are passed to \mbox{\isa{m}} here.  Furthermore, the static
  context is that of the enclosing goal (as for actual \mbox{\isa{\isacommand{qed}}}).  Thus the proof method may not refer to any assumptions
  introduced in the current body, for example.
  
  \item [\mbox{\isa{\isacommand{done}}}] completes a proof script, provided that
  the current goal state is solved completely.  Note that actual
  structured proof commands (e.g.\ ``\mbox{\isa{\isacommand{{\isachardot}}}}'' or \mbox{\isa{\isacommand{sorry}}}) may be used to conclude proof scripts as well.

  \item [\mbox{\isa{\isacommand{defer}}}~\isa{n} and \mbox{\isa{\isacommand{prefer}}}~\isa{n}] shuffle the list of pending goals: \mbox{\isa{\isacommand{defer}}} puts off
  sub-goal \isa{n} to the end of the list (\isa{{\isachardoublequote}n\ {\isacharequal}\ {\isadigit{1}}{\isachardoublequote}} by
  default), while \mbox{\isa{\isacommand{prefer}}} brings sub-goal \isa{n} to the
  front.
  
  \item [\mbox{\isa{\isacommand{back}}}] does back-tracking over the result
  sequence of the latest proof command.  Basically, any proof command
  may return multiple results.
  
  \end{descr}

  Any proper Isar proof method may be used with tactic script commands
  such as \mbox{\isa{\isacommand{apply}}}.  A few additional emulations of actual
  tactics are provided as well; these would be never used in actual
  structured proofs, of course.%
\end{isamarkuptext}%
\isamarkuptrue%
%
\isamarkupsection{Omitting proofs%
}
\isamarkuptrue%
%
\begin{isamarkuptext}%
\begin{matharray}{rcl}
    \indexdef{}{command}{oops}\mbox{\isa{\isacommand{oops}}} & : & \isartrans{proof}{theory} \\
  \end{matharray}

  The \mbox{\isa{\isacommand{oops}}} command discontinues the current proof
  attempt, while considering the partial proof text as properly
  processed.  This is conceptually quite different from ``faking''
  actual proofs via \indexref{}{command}{sorry}\mbox{\isa{\isacommand{sorry}}} (see
  \secref{sec:proof-steps}): \mbox{\isa{\isacommand{oops}}} does not observe the
  proof structure at all, but goes back right to the theory level.
  Furthermore, \mbox{\isa{\isacommand{oops}}} does not produce any result theorem
  --- there is no intended claim to be able to complete the proof
  anyhow.

  A typical application of \mbox{\isa{\isacommand{oops}}} is to explain Isar proofs
  \emph{within} the system itself, in conjunction with the document
  preparation tools of Isabelle described in \cite{isabelle-sys}.
  Thus partial or even wrong proof attempts can be discussed in a
  logically sound manner.  Note that the Isabelle {\LaTeX} macros can
  be easily adapted to print something like ``\isa{{\isachardoublequote}{\isasymdots}{\isachardoublequote}}'' instead of
  the keyword ``\mbox{\isa{\isacommand{oops}}}''.

  \medskip The \mbox{\isa{\isacommand{oops}}} command is undo-able, unlike
  \indexref{}{command}{kill}\mbox{\isa{\isacommand{kill}}} (see \secref{sec:history}).  The effect is to
  get back to the theory just before the opening of the proof.%
\end{isamarkuptext}%
\isamarkuptrue%
%
\isamarkupsection{Generalized elimination \label{sec:obtain}%
}
\isamarkuptrue%
%
\begin{isamarkuptext}%
\begin{matharray}{rcl}
    \indexdef{}{command}{obtain}\mbox{\isa{\isacommand{obtain}}} & : & \isartrans{proof(state)}{proof(prove)} \\
    \indexdef{}{command}{guess}\mbox{\isa{\isacommand{guess}}}\isa{{\isachardoublequote}\isactrlsup {\isacharasterisk}{\isachardoublequote}} & : & \isartrans{proof(state)}{proof(prove)} \\
  \end{matharray}

  Generalized elimination means that additional elements with certain
  properties may be introduced in the current context, by virtue of a
  locally proven ``soundness statement''.  Technically speaking, the
  \mbox{\isa{\isacommand{obtain}}} language element is like a declaration of
  \mbox{\isa{\isacommand{fix}}} and \mbox{\isa{\isacommand{assume}}} (see also see
  \secref{sec:proof-context}), together with a soundness proof of its
  additional claim.  According to the nature of existential reasoning,
  assumptions get eliminated from any result exported from the context
  later, provided that the corresponding parameters do \emph{not}
  occur in the conclusion.

  \begin{rail}
    'obtain' parname? (vars + 'and') 'where' (props + 'and')
    ;
    'guess' (vars + 'and')
    ;
  \end{rail}

  The derived Isar command \mbox{\isa{\isacommand{obtain}}} is defined as follows
  (where \isa{{\isachardoublequote}b\isactrlsub {\isadigit{1}}{\isacharcomma}\ {\isasymdots}{\isacharcomma}\ b\isactrlsub k{\isachardoublequote}} shall refer to (optional)
  facts indicated for forward chaining).
  \begin{matharray}{l}
    \isa{{\isachardoublequote}{\isasymlangle}using\ b\isactrlsub {\isadigit{1}}\ {\isasymdots}\ b\isactrlsub k{\isasymrangle}{\isachardoublequote}}~~\mbox{\isa{\isacommand{obtain}}}~\isa{{\isachardoublequote}x\isactrlsub {\isadigit{1}}\ {\isasymdots}\ x\isactrlsub m\ {\isasymWHERE}\ a{\isacharcolon}\ {\isasymphi}\isactrlsub {\isadigit{1}}\ {\isasymdots}\ {\isasymphi}\isactrlsub n\ \ {\isasymlangle}proof{\isasymrangle}\ {\isasymequiv}{\isachardoublequote}} \\[1ex]
    \quad \mbox{\isa{\isacommand{have}}}~\isa{{\isachardoublequote}{\isasymAnd}thesis{\isachardot}\ {\isacharparenleft}{\isasymAnd}x\isactrlsub {\isadigit{1}}\ {\isasymdots}\ x\isactrlsub m{\isachardot}\ {\isasymphi}\isactrlsub {\isadigit{1}}\ {\isasymLongrightarrow}\ {\isasymdots}\ {\isasymphi}\isactrlsub n\ {\isasymLongrightarrow}\ thesis{\isacharparenright}\ {\isasymLongrightarrow}\ thesis{\isachardoublequote}} \\
    \quad \mbox{\isa{\isacommand{proof}}}~\isa{succeed} \\
    \qquad \mbox{\isa{\isacommand{fix}}}~\isa{thesis} \\
    \qquad \mbox{\isa{\isacommand{assume}}}~\isa{{\isachardoublequote}that\ {\isacharbrackleft}Pure{\isachardot}intro{\isacharquery}{\isacharbrackright}{\isacharcolon}\ {\isasymAnd}x\isactrlsub {\isadigit{1}}\ {\isasymdots}\ x\isactrlsub m{\isachardot}\ {\isasymphi}\isactrlsub {\isadigit{1}}\ {\isasymLongrightarrow}\ {\isasymdots}\ {\isasymphi}\isactrlsub n\ {\isasymLongrightarrow}\ thesis{\isachardoublequote}} \\
    \qquad \mbox{\isa{\isacommand{then}}}~\mbox{\isa{\isacommand{show}}}~\isa{thesis} \\
    \quad\qquad \mbox{\isa{\isacommand{apply}}}~\isa{{\isacharminus}} \\
    \quad\qquad \mbox{\isa{\isacommand{using}}}~\isa{{\isachardoublequote}b\isactrlsub {\isadigit{1}}\ {\isasymdots}\ b\isactrlsub k\ \ {\isasymlangle}proof{\isasymrangle}{\isachardoublequote}} \\
    \quad \mbox{\isa{\isacommand{qed}}} \\
    \quad \mbox{\isa{\isacommand{fix}}}~\isa{{\isachardoublequote}x\isactrlsub {\isadigit{1}}\ {\isasymdots}\ x\isactrlsub m{\isachardoublequote}}~\mbox{\isa{\isacommand{assume}}}\isa{{\isachardoublequote}\isactrlsup {\isacharasterisk}\ a{\isacharcolon}\ {\isasymphi}\isactrlsub {\isadigit{1}}\ {\isasymdots}\ {\isasymphi}\isactrlsub n{\isachardoublequote}} \\
  \end{matharray}

  Typically, the soundness proof is relatively straight-forward, often
  just by canonical automated tools such as ``\mbox{\isa{\isacommand{by}}}~\isa{simp}'' or ``\mbox{\isa{\isacommand{by}}}~\isa{blast}''.  Accordingly, the
  ``\isa{that}'' reduction above is declared as simplification and
  introduction rule.

  In a sense, \mbox{\isa{\isacommand{obtain}}} represents at the level of Isar
  proofs what would be meta-logical existential quantifiers and
  conjunctions.  This concept has a broad range of useful
  applications, ranging from plain elimination (or introduction) of
  object-level existential and conjunctions, to elimination over
  results of symbolic evaluation of recursive definitions, for
  example.  Also note that \mbox{\isa{\isacommand{obtain}}} without parameters acts
  much like \mbox{\isa{\isacommand{have}}}, where the result is treated as a
  genuine assumption.

  An alternative name to be used instead of ``\isa{that}'' above may
  be given in parentheses.

  \medskip The improper variant \mbox{\isa{\isacommand{guess}}} is similar to
  \mbox{\isa{\isacommand{obtain}}}, but derives the obtained statement from the
  course of reasoning!  The proof starts with a fixed goal \isa{thesis}.  The subsequent proof may refine this to anything of the
  form like \isa{{\isachardoublequote}{\isasymAnd}x\isactrlsub {\isadigit{1}}\ {\isasymdots}\ x\isactrlsub m{\isachardot}\ {\isasymphi}\isactrlsub {\isadigit{1}}\ {\isasymLongrightarrow}\ {\isasymdots}\ {\isasymphi}\isactrlsub n\ {\isasymLongrightarrow}\ thesis{\isachardoublequote}}, but must not introduce new subgoals.  The
  final goal state is then used as reduction rule for the obtain
  scheme described above.  Obtained parameters \isa{{\isachardoublequote}x\isactrlsub {\isadigit{1}}{\isacharcomma}\ {\isasymdots}{\isacharcomma}\ x\isactrlsub m{\isachardoublequote}} are marked as internal by default, which prevents the
  proof context from being polluted by ad-hoc variables.  The variable
  names and type constraints given as arguments for \mbox{\isa{\isacommand{guess}}}
  specify a prefix of obtained parameters explicitly in the text.

  It is important to note that the facts introduced by \mbox{\isa{\isacommand{obtain}}} and \mbox{\isa{\isacommand{guess}}} may not be polymorphic: any
  type-variables occurring here are fixed in the present context!%
\end{isamarkuptext}%
\isamarkuptrue%
%
\isamarkupsection{Calculational reasoning \label{sec:calculation}%
}
\isamarkuptrue%
%
\begin{isamarkuptext}%
\begin{matharray}{rcl}
    \indexdef{}{command}{also}\mbox{\isa{\isacommand{also}}} & : & \isartrans{proof(state)}{proof(state)} \\
    \indexdef{}{command}{finally}\mbox{\isa{\isacommand{finally}}} & : & \isartrans{proof(state)}{proof(chain)} \\
    \indexdef{}{command}{moreover}\mbox{\isa{\isacommand{moreover}}} & : & \isartrans{proof(state)}{proof(state)} \\
    \indexdef{}{command}{ultimately}\mbox{\isa{\isacommand{ultimately}}} & : & \isartrans{proof(state)}{proof(chain)} \\
    \indexdef{}{command}{print\_trans\_rules}\mbox{\isa{\isacommand{print{\isacharunderscore}trans{\isacharunderscore}rules}}}\isa{{\isachardoublequote}\isactrlsup {\isacharasterisk}{\isachardoublequote}} & : & \isarkeep{theory~|~proof} \\
    \mbox{\isa{trans}} & : & \isaratt \\
    \mbox{\isa{sym}} & : & \isaratt \\
    \mbox{\isa{symmetric}} & : & \isaratt \\
  \end{matharray}

  Calculational proof is forward reasoning with implicit application
  of transitivity rules (such those of \isa{{\isachardoublequote}{\isacharequal}{\isachardoublequote}}, \isa{{\isachardoublequote}{\isasymle}{\isachardoublequote}},
  \isa{{\isachardoublequote}{\isacharless}{\isachardoublequote}}).  Isabelle/Isar maintains an auxiliary fact register
  \indexref{}{fact}{calculation}\mbox{\isa{calculation}} for accumulating results obtained by
  transitivity composed with the current result.  Command \mbox{\isa{\isacommand{also}}} updates \mbox{\isa{calculation}} involving \mbox{\isa{this}}, while
  \mbox{\isa{\isacommand{finally}}} exhibits the final \mbox{\isa{calculation}} by
  forward chaining towards the next goal statement.  Both commands
  require valid current facts, i.e.\ may occur only after commands
  that produce theorems such as \mbox{\isa{\isacommand{assume}}}, \mbox{\isa{\isacommand{note}}}, or some finished proof of \mbox{\isa{\isacommand{have}}}, \mbox{\isa{\isacommand{show}}} etc.  The \mbox{\isa{\isacommand{moreover}}} and \mbox{\isa{\isacommand{ultimately}}}
  commands are similar to \mbox{\isa{\isacommand{also}}} and \mbox{\isa{\isacommand{finally}}},
  but only collect further results in \mbox{\isa{calculation}} without
  applying any rules yet.

  Also note that the implicit term abbreviation ``\isa{{\isachardoublequote}{\isasymdots}{\isachardoublequote}}'' has
  its canonical application with calculational proofs.  It refers to
  the argument of the preceding statement. (The argument of a curried
  infix expression happens to be its right-hand side.)

  Isabelle/Isar calculations are implicitly subject to block structure
  in the sense that new threads of calculational reasoning are
  commenced for any new block (as opened by a local goal, for
  example).  This means that, apart from being able to nest
  calculations, there is no separate \emph{begin-calculation} command
  required.

  \medskip The Isar calculation proof commands may be defined as
  follows:\footnote{We suppress internal bookkeeping such as proper
  handling of block-structure.}

  \begin{matharray}{rcl}
    \mbox{\isa{\isacommand{also}}}\isa{{\isachardoublequote}\isactrlsub {\isadigit{0}}{\isachardoublequote}} & \equiv & \mbox{\isa{\isacommand{note}}}~\isa{{\isachardoublequote}calculation\ {\isacharequal}\ this{\isachardoublequote}} \\
    \mbox{\isa{\isacommand{also}}}\isa{{\isachardoublequote}\isactrlsub n\isactrlsub {\isacharplus}\isactrlsub {\isadigit{1}}{\isachardoublequote}} & \equiv & \mbox{\isa{\isacommand{note}}}~\isa{{\isachardoublequote}calculation\ {\isacharequal}\ trans\ {\isacharbrackleft}OF\ calculation\ this{\isacharbrackright}{\isachardoublequote}} \\[0.5ex]
    \mbox{\isa{\isacommand{finally}}} & \equiv & \mbox{\isa{\isacommand{also}}}~\mbox{\isa{\isacommand{from}}}~\isa{calculation} \\[0.5ex]
    \mbox{\isa{\isacommand{moreover}}} & \equiv & \mbox{\isa{\isacommand{note}}}~\isa{{\isachardoublequote}calculation\ {\isacharequal}\ calculation\ this{\isachardoublequote}} \\
    \mbox{\isa{\isacommand{ultimately}}} & \equiv & \mbox{\isa{\isacommand{moreover}}}~\mbox{\isa{\isacommand{from}}}~\isa{calculation} \\
  \end{matharray}

  \begin{rail}
    ('also' | 'finally') ('(' thmrefs ')')?
    ;
    'trans' (() | 'add' | 'del')
    ;
  \end{rail}

  \begin{descr}

  \item [\mbox{\isa{\isacommand{also}}}~\isa{{\isachardoublequote}{\isacharparenleft}a\isactrlsub {\isadigit{1}}\ {\isasymdots}\ a\isactrlsub n{\isacharparenright}{\isachardoublequote}}]
  maintains the auxiliary \mbox{\isa{calculation}} register as follows.
  The first occurrence of \mbox{\isa{\isacommand{also}}} in some calculational
  thread initializes \mbox{\isa{calculation}} by \mbox{\isa{this}}. Any
  subsequent \mbox{\isa{\isacommand{also}}} on the same level of block-structure
  updates \mbox{\isa{calculation}} by some transitivity rule applied to
  \mbox{\isa{calculation}} and \mbox{\isa{this}} (in that order).  Transitivity
  rules are picked from the current context, unless alternative rules
  are given as explicit arguments.

  \item [\mbox{\isa{\isacommand{finally}}}~\isa{{\isachardoublequote}{\isacharparenleft}a\isactrlsub {\isadigit{1}}\ {\isasymdots}\ a\isactrlsub n{\isacharparenright}{\isachardoublequote}}]
  maintaining \mbox{\isa{calculation}} in the same way as \mbox{\isa{\isacommand{also}}}, and concludes the current calculational thread.  The final
  result is exhibited as fact for forward chaining towards the next
  goal. Basically, \mbox{\isa{\isacommand{finally}}} just abbreviates \mbox{\isa{\isacommand{also}}}~\mbox{\isa{\isacommand{from}}}~\mbox{\isa{calculation}}.  Typical idioms for
  concluding calculational proofs are ``\mbox{\isa{\isacommand{finally}}}~\mbox{\isa{\isacommand{show}}}~\isa{{\isacharquery}thesis}~\mbox{\isa{\isacommand{{\isachardot}}}}'' and ``\mbox{\isa{\isacommand{finally}}}~\mbox{\isa{\isacommand{have}}}~\isa{{\isasymphi}}~\mbox{\isa{\isacommand{{\isachardot}}}}''.

  \item [\mbox{\isa{\isacommand{moreover}}} and \mbox{\isa{\isacommand{ultimately}}}] are
  analogous to \mbox{\isa{\isacommand{also}}} and \mbox{\isa{\isacommand{finally}}}, but collect
  results only, without applying rules.

  \item [\mbox{\isa{\isacommand{print{\isacharunderscore}trans{\isacharunderscore}rules}}}] prints the list of
  transitivity rules (for calculational commands \mbox{\isa{\isacommand{also}}} and
  \mbox{\isa{\isacommand{finally}}}) and symmetry rules (for the \mbox{\isa{symmetric}} operation and single step elimination patters) of the
  current context.

  \item [\mbox{\isa{trans}}] declares theorems as transitivity rules.

  \item [\mbox{\isa{sym}}] declares symmetry rules, as well as
  \mbox{\isa{Pure{\isachardot}elim{\isacharquery}}} rules.

  \item [\mbox{\isa{symmetric}}] resolves a theorem with some rule
  declared as \mbox{\isa{sym}} in the current context.  For example,
  ``\mbox{\isa{\isacommand{assume}}}~\isa{{\isachardoublequote}{\isacharbrackleft}symmetric{\isacharbrackright}{\isacharcolon}\ x\ {\isacharequal}\ y{\isachardoublequote}}'' produces a
  swapped fact derived from that assumption.

  In structured proof texts it is often more appropriate to use an
  explicit single-step elimination proof, such as ``\mbox{\isa{\isacommand{assume}}}~\isa{{\isachardoublequote}x\ {\isacharequal}\ y{\isachardoublequote}}~\mbox{\isa{\isacommand{then}}}~\mbox{\isa{\isacommand{have}}}~\isa{{\isachardoublequote}y\ {\isacharequal}\ x{\isachardoublequote}}~\mbox{\isa{\isacommand{{\isachardot}{\isachardot}}}}''.

  \end{descr}%
\end{isamarkuptext}%
\isamarkuptrue%
%
\isadelimtheory
%
\endisadelimtheory
%
\isatagtheory
\isacommand{end}\isamarkupfalse%
%
\endisatagtheory
{\isafoldtheory}%
%
\isadelimtheory
%
\endisadelimtheory
\isanewline
\end{isabellebody}%
%%% Local Variables:
%%% mode: latex
%%% TeX-master: "root"
%%% End:
