%
\begin{isabellebody}%
\def\isabellecontext{Document{\isacharunderscore}Preparation}%
%
\isadelimtheory
%
\endisadelimtheory
%
\isatagtheory
\isacommand{theory}\isamarkupfalse%
\ Document{\isacharunderscore}Preparation\isanewline
\isakeyword{imports}\ Main\isanewline
\isakeyword{begin}%
\endisatagtheory
{\isafoldtheory}%
%
\isadelimtheory
%
\endisadelimtheory
%
\isamarkupchapter{Document preparation \label{ch:document-prep}%
}
\isamarkuptrue%
%
\begin{isamarkuptext}%
Isabelle/Isar provides a simple document preparation system
  based on regular {PDF-\LaTeX} technology, with full support for
  hyper-links and bookmarks.  Thus the results are well suited for WWW
  browsing and as printed copies.

  \medskip Isabelle generates {\LaTeX} output while running a
  \emph{logic session} (see also \cite{isabelle-sys}).  Getting
  started with a working configuration for common situations is quite
  easy by using the Isabelle \verb|mkdir| and \verb|make|
  tools.  First invoke
\begin{ttbox}
  isabelle mkdir Foo
\end{ttbox}
  to initialize a separate directory for session \verb|Foo| (it
  is safe to experiment, since \verb|isabelle mkdir| never
  overwrites existing files).  Ensure that \verb|Foo/ROOT.ML|
  holds ML commands to load all theories required for this session;
  furthermore \verb|Foo/document/root.tex| should include any
  special {\LaTeX} macro packages required for your document (the
  default is usually sufficient as a start).

  The session is controlled by a separate \verb|IsaMakefile|
  (with crude source dependencies by default).  This file is located
  one level up from the \verb|Foo| directory location.  Now
  invoke
\begin{ttbox}
  isabelle make Foo
\end{ttbox}
  to run the \verb|Foo| session, with browser information and
  document preparation enabled.  Unless any errors are reported by
  Isabelle or {\LaTeX}, the output will appear inside the directory
  defined by the \verb|ISABELLE_BROWSER_INFO| setting (as
  reported by the batch job in verbose mode).

  \medskip You may also consider to tune the \verb|usedir|
  options in \verb|IsaMakefile|, for example to switch the output
  format between \verb|pdf| and \verb|dvi|, or activate the
  \verb|-D| option to retain a second copy of the generated
  {\LaTeX} sources (for manual inspection or separate runs of
  \hyperlink{executable.latex}{\mbox{\isa{\isatt{latex}}}}).

  \medskip See \emph{The Isabelle System Manual} \cite{isabelle-sys}
  for further details on Isabelle logic sessions and theory
  presentation.  The Isabelle/HOL tutorial \cite{isabelle-hol-book}
  also covers theory presentation to some extent.%
\end{isamarkuptext}%
\isamarkuptrue%
%
\isamarkupsection{Markup commands \label{sec:markup}%
}
\isamarkuptrue%
%
\begin{isamarkuptext}%
\begin{matharray}{rcl}
    \indexdef{}{command}{header}\hypertarget{command.header}{\hyperlink{command.header}{\mbox{\isa{\isacommand{header}}}}} & : & \isa{{\isachardoublequote}toplevel\ {\isasymrightarrow}\ toplevel{\isachardoublequote}} \\[0.5ex]
    \indexdef{}{command}{chapter}\hypertarget{command.chapter}{\hyperlink{command.chapter}{\mbox{\isa{\isacommand{chapter}}}}} & : & \isa{{\isachardoublequote}local{\isacharunderscore}theory\ {\isasymrightarrow}\ local{\isacharunderscore}theory{\isachardoublequote}} \\
    \indexdef{}{command}{section}\hypertarget{command.section}{\hyperlink{command.section}{\mbox{\isa{\isacommand{section}}}}} & : & \isa{{\isachardoublequote}local{\isacharunderscore}theory\ {\isasymrightarrow}\ local{\isacharunderscore}theory{\isachardoublequote}} \\
    \indexdef{}{command}{subsection}\hypertarget{command.subsection}{\hyperlink{command.subsection}{\mbox{\isa{\isacommand{subsection}}}}} & : & \isa{{\isachardoublequote}local{\isacharunderscore}theory\ {\isasymrightarrow}\ local{\isacharunderscore}theory{\isachardoublequote}} \\
    \indexdef{}{command}{subsubsection}\hypertarget{command.subsubsection}{\hyperlink{command.subsubsection}{\mbox{\isa{\isacommand{subsubsection}}}}} & : & \isa{{\isachardoublequote}local{\isacharunderscore}theory\ {\isasymrightarrow}\ local{\isacharunderscore}theory{\isachardoublequote}} \\
    \indexdef{}{command}{text}\hypertarget{command.text}{\hyperlink{command.text}{\mbox{\isa{\isacommand{text}}}}} & : & \isa{{\isachardoublequote}local{\isacharunderscore}theory\ {\isasymrightarrow}\ local{\isacharunderscore}theory{\isachardoublequote}} \\
    \indexdef{}{command}{text\_raw}\hypertarget{command.text-raw}{\hyperlink{command.text-raw}{\mbox{\isa{\isacommand{text{\isacharunderscore}raw}}}}} & : & \isa{{\isachardoublequote}local{\isacharunderscore}theory\ {\isasymrightarrow}\ local{\isacharunderscore}theory{\isachardoublequote}} \\[0.5ex]
    \indexdef{}{command}{sect}\hypertarget{command.sect}{\hyperlink{command.sect}{\mbox{\isa{\isacommand{sect}}}}} & : & \isa{{\isachardoublequote}proof\ {\isasymrightarrow}\ proof{\isachardoublequote}} \\
    \indexdef{}{command}{subsect}\hypertarget{command.subsect}{\hyperlink{command.subsect}{\mbox{\isa{\isacommand{subsect}}}}} & : & \isa{{\isachardoublequote}proof\ {\isasymrightarrow}\ proof{\isachardoublequote}} \\
    \indexdef{}{command}{subsubsect}\hypertarget{command.subsubsect}{\hyperlink{command.subsubsect}{\mbox{\isa{\isacommand{subsubsect}}}}} & : & \isa{{\isachardoublequote}proof\ {\isasymrightarrow}\ proof{\isachardoublequote}} \\
    \indexdef{}{command}{txt}\hypertarget{command.txt}{\hyperlink{command.txt}{\mbox{\isa{\isacommand{txt}}}}} & : & \isa{{\isachardoublequote}proof\ {\isasymrightarrow}\ proof{\isachardoublequote}} \\
    \indexdef{}{command}{txt\_raw}\hypertarget{command.txt-raw}{\hyperlink{command.txt-raw}{\mbox{\isa{\isacommand{txt{\isacharunderscore}raw}}}}} & : & \isa{{\isachardoublequote}proof\ {\isasymrightarrow}\ proof{\isachardoublequote}} \\
  \end{matharray}

  Markup commands provide a structured way to insert text into the
  document generated from a theory.  Each markup command takes a
  single \hyperlink{syntax.text}{\mbox{\isa{text}}} argument, which is passed as argument to a
  corresponding {\LaTeX} macro.  The default macros provided by
  \hyperlink{file.~~/lib/texinputs/isabelle.sty}{\mbox{\isa{\isatt{{\isachartilde}{\isachartilde}{\isacharslash}lib{\isacharslash}texinputs{\isacharslash}isabelle{\isachardot}sty}}}} can be redefined according
  to the needs of the underlying document and {\LaTeX} styles.

  Note that formal comments (\secref{sec:comments}) are similar to
  markup commands, but have a different status within Isabelle/Isar
  syntax.

  \begin{rail}
    ('chapter' | 'section' | 'subsection' | 'subsubsection' | 'text') target? text
    ;
    ('header' | 'text\_raw' | 'sect' | 'subsect' | 'subsubsect' | 'txt' | 'txt\_raw') text
    ;
  \end{rail}

  \begin{description}

  \item \hyperlink{command.header}{\mbox{\isa{\isacommand{header}}}} provides plain text markup just preceding
  the formal beginning of a theory.  The corresponding {\LaTeX} macro
  is \verb|\isamarkupheader|, which acts like \hyperlink{command.section}{\mbox{\isa{\isacommand{section}}}} by default.
  
  \item \hyperlink{command.chapter}{\mbox{\isa{\isacommand{chapter}}}}, \hyperlink{command.section}{\mbox{\isa{\isacommand{section}}}}, \hyperlink{command.subsection}{\mbox{\isa{\isacommand{subsection}}}},
  and \hyperlink{command.subsubsection}{\mbox{\isa{\isacommand{subsubsection}}}} mark chapter and section headings
  within the main theory body or local theory targets.  The
  corresponding {\LaTeX} macros are \verb|\isamarkupchapter|,
  \verb|\isamarkupsection|, \verb|\isamarkupsubsection| etc.

  \item \hyperlink{command.sect}{\mbox{\isa{\isacommand{sect}}}}, \hyperlink{command.subsect}{\mbox{\isa{\isacommand{subsect}}}}, and \hyperlink{command.subsubsect}{\mbox{\isa{\isacommand{subsubsect}}}}
  mark section headings within proofs.  The corresponding {\LaTeX}
  macros are \verb|\isamarkupsect|, \verb|\isamarkupsubsect| etc.

  \item \hyperlink{command.text}{\mbox{\isa{\isacommand{text}}}} and \hyperlink{command.txt}{\mbox{\isa{\isacommand{txt}}}} specify paragraphs of plain
  text.  This corresponds to a {\LaTeX} environment \verb|\begin{isamarkuptext}| \isa{{\isachardoublequote}{\isasymdots}{\isachardoublequote}} \verb|\end{isamarkuptext}| etc.

  \item \hyperlink{command.text-raw}{\mbox{\isa{\isacommand{text{\isacharunderscore}raw}}}} and \hyperlink{command.txt-raw}{\mbox{\isa{\isacommand{txt{\isacharunderscore}raw}}}} insert {\LaTeX}
  source into the output, without additional markup.  Thus the full
  range of document manipulations becomes available, at the risk of
  messing up document output.

  \end{description}

  Except for \hyperlink{command.text-raw}{\mbox{\isa{\isacommand{text{\isacharunderscore}raw}}}} and \hyperlink{command.txt-raw}{\mbox{\isa{\isacommand{txt{\isacharunderscore}raw}}}}, the text
  passed to any of the above markup commands may refer to formal
  entities via \emph{document antiquotations}, see also
  \secref{sec:antiq}.  These are interpreted in the present theory or
  proof context, or the named \isa{{\isachardoublequote}target{\isachardoublequote}}.

  \medskip The proof markup commands closely resemble those for theory
  specifications, but have a different formal status and produce
  different {\LaTeX} macros.  The default definitions coincide for
  analogous commands such as \hyperlink{command.section}{\mbox{\isa{\isacommand{section}}}} and \hyperlink{command.sect}{\mbox{\isa{\isacommand{sect}}}}.%
\end{isamarkuptext}%
\isamarkuptrue%
%
\isamarkupsection{Document Antiquotations \label{sec:antiq}%
}
\isamarkuptrue%
%
\begin{isamarkuptext}%
\begin{matharray}{rcl}
    \indexdef{}{antiquotation}{theory}\hypertarget{antiquotation.theory}{\hyperlink{antiquotation.theory}{\mbox{\isa{theory}}}} & : & \isa{antiquotation} \\
    \indexdef{}{antiquotation}{thm}\hypertarget{antiquotation.thm}{\hyperlink{antiquotation.thm}{\mbox{\isa{thm}}}} & : & \isa{antiquotation} \\
    \indexdef{}{antiquotation}{lemma}\hypertarget{antiquotation.lemma}{\hyperlink{antiquotation.lemma}{\mbox{\isa{lemma}}}} & : & \isa{antiquotation} \\
    \indexdef{}{antiquotation}{prop}\hypertarget{antiquotation.prop}{\hyperlink{antiquotation.prop}{\mbox{\isa{prop}}}} & : & \isa{antiquotation} \\
    \indexdef{}{antiquotation}{term}\hypertarget{antiquotation.term}{\hyperlink{antiquotation.term}{\mbox{\isa{term}}}} & : & \isa{antiquotation} \\
    \indexdef{}{antiquotation}{term\_type}\hypertarget{antiquotation.term-type}{\hyperlink{antiquotation.term-type}{\mbox{\isa{term{\isacharunderscore}type}}}} & : & \isa{antiquotation} \\
    \indexdef{}{antiquotation}{typeof}\hypertarget{antiquotation.typeof}{\hyperlink{antiquotation.typeof}{\mbox{\isa{typeof}}}} & : & \isa{antiquotation} \\
    \indexdef{}{antiquotation}{const}\hypertarget{antiquotation.const}{\hyperlink{antiquotation.const}{\mbox{\isa{const}}}} & : & \isa{antiquotation} \\
    \indexdef{}{antiquotation}{abbrev}\hypertarget{antiquotation.abbrev}{\hyperlink{antiquotation.abbrev}{\mbox{\isa{abbrev}}}} & : & \isa{antiquotation} \\
    \indexdef{}{antiquotation}{typ}\hypertarget{antiquotation.typ}{\hyperlink{antiquotation.typ}{\mbox{\isa{typ}}}} & : & \isa{antiquotation} \\
    \indexdef{}{antiquotation}{text}\hypertarget{antiquotation.text}{\hyperlink{antiquotation.text}{\mbox{\isa{text}}}} & : & \isa{antiquotation} \\
    \indexdef{}{antiquotation}{goals}\hypertarget{antiquotation.goals}{\hyperlink{antiquotation.goals}{\mbox{\isa{goals}}}} & : & \isa{antiquotation} \\
    \indexdef{}{antiquotation}{subgoals}\hypertarget{antiquotation.subgoals}{\hyperlink{antiquotation.subgoals}{\mbox{\isa{subgoals}}}} & : & \isa{antiquotation} \\
    \indexdef{}{antiquotation}{prf}\hypertarget{antiquotation.prf}{\hyperlink{antiquotation.prf}{\mbox{\isa{prf}}}} & : & \isa{antiquotation} \\
    \indexdef{}{antiquotation}{full\_prf}\hypertarget{antiquotation.full-prf}{\hyperlink{antiquotation.full-prf}{\mbox{\isa{full{\isacharunderscore}prf}}}} & : & \isa{antiquotation} \\
    \indexdef{}{antiquotation}{ML}\hypertarget{antiquotation.ML}{\hyperlink{antiquotation.ML}{\mbox{\isa{ML}}}} & : & \isa{antiquotation} \\
    \indexdef{}{antiquotation}{ML\_type}\hypertarget{antiquotation.ML-type}{\hyperlink{antiquotation.ML-type}{\mbox{\isa{ML{\isacharunderscore}type}}}} & : & \isa{antiquotation} \\
    \indexdef{}{antiquotation}{ML\_struct}\hypertarget{antiquotation.ML-struct}{\hyperlink{antiquotation.ML-struct}{\mbox{\isa{ML{\isacharunderscore}struct}}}} & : & \isa{antiquotation} \\
  \end{matharray}

  The overall content of an Isabelle/Isar theory may alternate between
  formal and informal text.  The main body consists of formal
  specification and proof commands, interspersed with markup commands
  (\secref{sec:markup}) or document comments (\secref{sec:comments}).
  The argument of markup commands quotes informal text to be printed
  in the resulting document, but may again refer to formal entities
  via \emph{document antiquotations}.

  For example, embedding of ``\isa{{\isacharat}{\isacharbraceleft}term\ {\isacharbrackleft}show{\isacharunderscore}types{\isacharbrackright}\ {\isachardoublequote}f\ x\ {\isacharequal}\ a\ {\isacharplus}\ x{\isachardoublequote}{\isacharbraceright}}''
  within a text block makes
  \isa{{\isacharparenleft}f{\isasymColon}{\isacharprime}a\ {\isasymRightarrow}\ {\isacharprime}a{\isacharparenright}\ {\isacharparenleft}x{\isasymColon}{\isacharprime}a{\isacharparenright}\ {\isacharequal}\ {\isacharparenleft}a{\isasymColon}{\isacharprime}a{\isacharparenright}\ {\isacharplus}\ x} appear in the final {\LaTeX} document.

  Antiquotations usually spare the author tedious typing of logical
  entities in full detail.  Even more importantly, some degree of
  consistency-checking between the main body of formal text and its
  informal explanation is achieved, since terms and types appearing in
  antiquotations are checked within the current theory or proof
  context.

  \begin{rail}
    atsign lbrace antiquotation rbrace
    ;

    antiquotation:
      'theory' options name |
      'thm' options styles thmrefs |
      'lemma' options prop 'by' method |
      'prop' options styles prop |
      'term' options styles term |
      'term_type' options styles term |
      'typeof' options styles term |
      'const' options term |
      'abbrev' options term |
      'typ' options type |
      'text' options name |
      'goals' options |
      'subgoals' options |
      'prf' options thmrefs |
      'full\_prf' options thmrefs |
      'ML' options name |
      'ML\_type' options name |
      'ML\_struct' options name
    ;
    options: '[' (option * ',') ']'
    ;
    option: name | name '=' name
    ;
    styles: '(' (style + ',') ')'
    ;
    style: (name +)
    ;
  \end{rail}

  Note that the syntax of antiquotations may \emph{not} include source
  comments \verb|(*|~\isa{{\isachardoublequote}{\isasymdots}{\isachardoublequote}}~\verb|*)| nor verbatim
  text \verb|{|\verb|*|~\isa{{\isachardoublequote}{\isasymdots}{\isachardoublequote}}~\verb|*|\verb|}|.

  \begin{description}
  
  \item \isa{{\isachardoublequote}{\isacharat}{\isacharbraceleft}theory\ A{\isacharbraceright}{\isachardoublequote}} prints the name \isa{{\isachardoublequote}A{\isachardoublequote}}, which is
  guaranteed to refer to a valid ancestor theory in the current
  context.

  \item \isa{{\isachardoublequote}{\isacharat}{\isacharbraceleft}thm\ a\isactrlsub {\isadigit{1}}\ {\isasymdots}\ a\isactrlsub n{\isacharbraceright}{\isachardoublequote}} prints theorems \isa{{\isachardoublequote}a\isactrlsub {\isadigit{1}}\ {\isasymdots}\ a\isactrlsub n{\isachardoublequote}}.
  Full fact expressions are allowed here, including attributes
  (\secref{sec:syn-att}).

  \item \isa{{\isachardoublequote}{\isacharat}{\isacharbraceleft}prop\ {\isasymphi}{\isacharbraceright}{\isachardoublequote}} prints a well-typed proposition \isa{{\isachardoublequote}{\isasymphi}{\isachardoublequote}}.

  \item \isa{{\isachardoublequote}{\isacharat}{\isacharbraceleft}lemma\ {\isasymphi}\ by\ m{\isacharbraceright}{\isachardoublequote}} proves a well-typed proposition
  \isa{{\isachardoublequote}{\isasymphi}{\isachardoublequote}} by method \isa{m} and prints the original \isa{{\isachardoublequote}{\isasymphi}{\isachardoublequote}}.

  \item \isa{{\isachardoublequote}{\isacharat}{\isacharbraceleft}term\ t{\isacharbraceright}{\isachardoublequote}} prints a well-typed term \isa{{\isachardoublequote}t{\isachardoublequote}}.

  \item \isa{{\isachardoublequote}{\isacharat}{\isacharbraceleft}term{\isacharunderscore}type\ t{\isacharbraceright}{\isachardoublequote}} prints a well-typed term \isa{{\isachardoublequote}t{\isachardoublequote}}
  annotated with its type.

  \item \isa{{\isachardoublequote}{\isacharat}{\isacharbraceleft}typeof\ t{\isacharbraceright}{\isachardoublequote}} prints the type of a well-typed term
  \isa{{\isachardoublequote}t{\isachardoublequote}}.

  \item \isa{{\isachardoublequote}{\isacharat}{\isacharbraceleft}const\ c{\isacharbraceright}{\isachardoublequote}} prints a logical or syntactic constant
  \isa{{\isachardoublequote}c{\isachardoublequote}}.
  
  \item \isa{{\isachardoublequote}{\isacharat}{\isacharbraceleft}abbrev\ c\ x\isactrlsub {\isadigit{1}}\ {\isasymdots}\ x\isactrlsub n{\isacharbraceright}{\isachardoublequote}} prints a constant abbreviation
  \isa{{\isachardoublequote}c\ x\isactrlsub {\isadigit{1}}\ {\isasymdots}\ x\isactrlsub n\ {\isasymequiv}\ rhs{\isachardoublequote}} as defined in the current context.
  \item \isa{{\isachardoublequote}{\isacharat}{\isacharbraceleft}typ\ {\isasymtau}{\isacharbraceright}{\isachardoublequote}} prints a well-formed type \isa{{\isachardoublequote}{\isasymtau}{\isachardoublequote}}.
  
  \item \isa{{\isachardoublequote}{\isacharat}{\isacharbraceleft}text\ s{\isacharbraceright}{\isachardoublequote}} prints uninterpreted source text \isa{s}.  This is particularly useful to print portions of text according
  to the Isabelle document style, without demanding well-formedness,
  e.g.\ small pieces of terms that should not be parsed or
  type-checked yet.

  \item \isa{{\isachardoublequote}{\isacharat}{\isacharbraceleft}goals{\isacharbraceright}{\isachardoublequote}} prints the current \emph{dynamic} goal
  state.  This is mainly for support of tactic-emulation scripts
  within Isar.  Presentation of goal states does not conform to the
  idea of human-readable proof documents!

  When explaining proofs in detail it is usually better to spell out
  the reasoning via proper Isar proof commands, instead of peeking at
  the internal machine configuration.
  
  \item \isa{{\isachardoublequote}{\isacharat}{\isacharbraceleft}subgoals{\isacharbraceright}{\isachardoublequote}} is similar to \isa{{\isachardoublequote}{\isacharat}{\isacharbraceleft}goals{\isacharbraceright}{\isachardoublequote}}, but
  does not print the main goal.
  
  \item \isa{{\isachardoublequote}{\isacharat}{\isacharbraceleft}prf\ a\isactrlsub {\isadigit{1}}\ {\isasymdots}\ a\isactrlsub n{\isacharbraceright}{\isachardoublequote}} prints the (compact) proof terms
  corresponding to the theorems \isa{{\isachardoublequote}a\isactrlsub {\isadigit{1}}\ {\isasymdots}\ a\isactrlsub n{\isachardoublequote}}. Note that this
  requires proof terms to be switched on for the current logic
  session.
  
  \item \isa{{\isachardoublequote}{\isacharat}{\isacharbraceleft}full{\isacharunderscore}prf\ a\isactrlsub {\isadigit{1}}\ {\isasymdots}\ a\isactrlsub n{\isacharbraceright}{\isachardoublequote}} is like \isa{{\isachardoublequote}{\isacharat}{\isacharbraceleft}prf\ a\isactrlsub {\isadigit{1}}\ {\isasymdots}\ a\isactrlsub n{\isacharbraceright}{\isachardoublequote}}, but prints the full proof terms, i.e.\ also displays
  information omitted in the compact proof term, which is denoted by
  ``\isa{{\isacharunderscore}}'' placeholders there.
  
  \item \isa{{\isachardoublequote}{\isacharat}{\isacharbraceleft}ML\ s{\isacharbraceright}{\isachardoublequote}}, \isa{{\isachardoublequote}{\isacharat}{\isacharbraceleft}ML{\isacharunderscore}type\ s{\isacharbraceright}{\isachardoublequote}}, and \isa{{\isachardoublequote}{\isacharat}{\isacharbraceleft}ML{\isacharunderscore}struct\ s{\isacharbraceright}{\isachardoublequote}} check text \isa{s} as ML value, type, and
  structure, respectively.  The source is printed verbatim.

  \end{description}%
\end{isamarkuptext}%
\isamarkuptrue%
%
\isamarkupsubsubsection{Styled antiquotations%
}
\isamarkuptrue%
%
\begin{isamarkuptext}%
The antiquotations \isa{thm}, \isa{prop} and \isa{term} admit an extra \emph{style} specification to modify the
  printed result.  A style is specified by a name with a possibly
  empty number of arguments;  multiple styles can be sequenced with
  commas.  The following standard styles are available:

  \begin{description}
  
  \item \isa{lhs} extracts the first argument of any application
  form with at least two arguments --- typically meta-level or
  object-level equality, or any other binary relation.
  
  \item \isa{rhs} is like \isa{lhs}, but extracts the second
  argument.
  
  \item \isa{{\isachardoublequote}concl{\isachardoublequote}} extracts the conclusion \isa{C} from a rule
  in Horn-clause normal form \isa{{\isachardoublequote}A\isactrlsub {\isadigit{1}}\ {\isasymLongrightarrow}\ {\isasymdots}\ A\isactrlsub n\ {\isasymLongrightarrow}\ C{\isachardoublequote}}.
  
  \item \isa{{\isachardoublequote}prem{\isachardoublequote}} \isa{n} extract premise number
  \isa{{\isachardoublequote}n{\isachardoublequote}} from from a rule in Horn-clause
  normal form \isa{{\isachardoublequote}A\isactrlsub {\isadigit{1}}\ {\isasymLongrightarrow}\ {\isasymdots}\ A\isactrlsub n\ {\isasymLongrightarrow}\ C{\isachardoublequote}}

  \end{description}%
\end{isamarkuptext}%
\isamarkuptrue%
%
\isamarkupsubsubsection{General options%
}
\isamarkuptrue%
%
\begin{isamarkuptext}%
The following options are available to tune the printed output
  of antiquotations.  Note that many of these coincide with global ML
  flags of the same names.

  \begin{description}

  \item \indexdef{}{antiquotation option}{show\_types}\hypertarget{antiquotation option.show-types}{\hyperlink{antiquotation option.show-types}{\mbox{\isa{show{\isacharunderscore}types}}}}~\isa{{\isachardoublequote}{\isacharequal}\ bool{\isachardoublequote}} and
  \indexdef{}{antiquotation option}{show\_sorts}\hypertarget{antiquotation option.show-sorts}{\hyperlink{antiquotation option.show-sorts}{\mbox{\isa{show{\isacharunderscore}sorts}}}}~\isa{{\isachardoublequote}{\isacharequal}\ bool{\isachardoublequote}} control
  printing of explicit type and sort constraints.

  \item \indexdef{}{antiquotation option}{show\_structs}\hypertarget{antiquotation option.show-structs}{\hyperlink{antiquotation option.show-structs}{\mbox{\isa{show{\isacharunderscore}structs}}}}~\isa{{\isachardoublequote}{\isacharequal}\ bool{\isachardoublequote}}
  controls printing of implicit structures.

  \item \indexdef{}{antiquotation option}{long\_names}\hypertarget{antiquotation option.long-names}{\hyperlink{antiquotation option.long-names}{\mbox{\isa{long{\isacharunderscore}names}}}}~\isa{{\isachardoublequote}{\isacharequal}\ bool{\isachardoublequote}} forces
  names of types and constants etc.\ to be printed in their fully
  qualified internal form.

  \item \indexdef{}{antiquotation option}{short\_names}\hypertarget{antiquotation option.short-names}{\hyperlink{antiquotation option.short-names}{\mbox{\isa{short{\isacharunderscore}names}}}}~\isa{{\isachardoublequote}{\isacharequal}\ bool{\isachardoublequote}}
  forces names of types and constants etc.\ to be printed unqualified.
  Note that internalizing the output again in the current context may
  well yield a different result.

  \item \indexdef{}{antiquotation option}{unique\_names}\hypertarget{antiquotation option.unique-names}{\hyperlink{antiquotation option.unique-names}{\mbox{\isa{unique{\isacharunderscore}names}}}}~\isa{{\isachardoublequote}{\isacharequal}\ bool{\isachardoublequote}}
  determines whether the printed version of qualified names should be
  made sufficiently long to avoid overlap with names declared further
  back.  Set to \isa{false} for more concise output.

  \item \indexdef{}{antiquotation option}{eta\_contract}\hypertarget{antiquotation option.eta-contract}{\hyperlink{antiquotation option.eta-contract}{\mbox{\isa{eta{\isacharunderscore}contract}}}}~\isa{{\isachardoublequote}{\isacharequal}\ bool{\isachardoublequote}}
  prints terms in \isa{{\isasymeta}}-contracted form.

  \item \indexdef{}{antiquotation option}{display}\hypertarget{antiquotation option.display}{\hyperlink{antiquotation option.display}{\mbox{\isa{display}}}}~\isa{{\isachardoublequote}{\isacharequal}\ bool{\isachardoublequote}} indicates
  if the text is to be output as multi-line ``display material'',
  rather than a small piece of text without line breaks (which is the
  default).

  In this mode the embedded entities are printed in the same style as
  the main theory text.

  \item \indexdef{}{antiquotation option}{break}\hypertarget{antiquotation option.break}{\hyperlink{antiquotation option.break}{\mbox{\isa{break}}}}~\isa{{\isachardoublequote}{\isacharequal}\ bool{\isachardoublequote}} controls
  line breaks in non-display material.

  \item \indexdef{}{antiquotation option}{quotes}\hypertarget{antiquotation option.quotes}{\hyperlink{antiquotation option.quotes}{\mbox{\isa{quotes}}}}~\isa{{\isachardoublequote}{\isacharequal}\ bool{\isachardoublequote}} indicates
  if the output should be enclosed in double quotes.

  \item \indexdef{}{antiquotation option}{mode}\hypertarget{antiquotation option.mode}{\hyperlink{antiquotation option.mode}{\mbox{\isa{mode}}}}~\isa{{\isachardoublequote}{\isacharequal}\ name{\isachardoublequote}} adds \isa{name} to the print mode to be used for presentation.  Note that the
  standard setup for {\LaTeX} output is already present by default,
  including the modes \isa{latex} and \isa{xsymbols}.

  \item \indexdef{}{antiquotation option}{margin}\hypertarget{antiquotation option.margin}{\hyperlink{antiquotation option.margin}{\mbox{\isa{margin}}}}~\isa{{\isachardoublequote}{\isacharequal}\ nat{\isachardoublequote}} and
  \indexdef{}{antiquotation option}{indent}\hypertarget{antiquotation option.indent}{\hyperlink{antiquotation option.indent}{\mbox{\isa{indent}}}}~\isa{{\isachardoublequote}{\isacharequal}\ nat{\isachardoublequote}} change the margin
  or indentation for pretty printing of display material.

  \item \indexdef{}{antiquotation option}{goals\_limit}\hypertarget{antiquotation option.goals-limit}{\hyperlink{antiquotation option.goals-limit}{\mbox{\isa{goals{\isacharunderscore}limit}}}}~\isa{{\isachardoublequote}{\isacharequal}\ nat{\isachardoublequote}}
  determines the maximum number of goals to be printed (for goal-based
  antiquotation).

  \item \indexdef{}{antiquotation option}{source}\hypertarget{antiquotation option.source}{\hyperlink{antiquotation option.source}{\mbox{\isa{source}}}}~\isa{{\isachardoublequote}{\isacharequal}\ bool{\isachardoublequote}} prints the
  original source text of the antiquotation arguments, rather than its
  internal representation.  Note that formal checking of
  \hyperlink{antiquotation.thm}{\mbox{\isa{thm}}}, \hyperlink{antiquotation.term}{\mbox{\isa{term}}}, etc. is still
  enabled; use the \hyperlink{antiquotation.text}{\mbox{\isa{text}}} antiquotation for unchecked
  output.

  Regular \isa{{\isachardoublequote}term{\isachardoublequote}} and \isa{{\isachardoublequote}typ{\isachardoublequote}} antiquotations with \isa{{\isachardoublequote}source\ {\isacharequal}\ false{\isachardoublequote}} involve a full round-trip from the original source
  to an internalized logical entity back to a source form, according
  to the syntax of the current context.  Thus the printed output is
  not under direct control of the author, it may even fluctuate a bit
  as the underlying theory is changed later on.

  In contrast, \indexdef{}{antiquotation option}{source}\hypertarget{antiquotation option.source}{\hyperlink{antiquotation option.source}{\mbox{\isa{source}}}}~\isa{{\isachardoublequote}{\isacharequal}\ true{\isachardoublequote}}
  admits direct printing of the given source text, with the desirable
  well-formedness check in the background, but without modification of
  the printed text.

  \end{description}

  For boolean flags, ``\isa{{\isachardoublequote}name\ {\isacharequal}\ true{\isachardoublequote}}'' may be abbreviated as
  ``\isa{name}''.  All of the above flags are disabled by default,
  unless changed from ML, say in the \verb|ROOT.ML| of the
  logic session.%
\end{isamarkuptext}%
\isamarkuptrue%
%
\isamarkupsection{Markup via command tags \label{sec:tags}%
}
\isamarkuptrue%
%
\begin{isamarkuptext}%
Each Isabelle/Isar command may be decorated by additional
  presentation tags, to indicate some modification in the way it is
  printed in the document.

  \indexouternonterm{tags}
  \begin{rail}
    tags: ( tag * )
    ;
    tag: '\%' (ident | string)
  \end{rail}

  Some tags are pre-declared for certain classes of commands, serving
  as default markup if no tags are given in the text:

  \medskip
  \begin{tabular}{ll}
    \isa{{\isachardoublequote}theory{\isachardoublequote}} & theory begin/end \\
    \isa{{\isachardoublequote}proof{\isachardoublequote}} & all proof commands \\
    \isa{{\isachardoublequote}ML{\isachardoublequote}} & all commands involving ML code \\
  \end{tabular}

  \medskip The Isabelle document preparation system
  \cite{isabelle-sys} allows tagged command regions to be presented
  specifically, e.g.\ to fold proof texts, or drop parts of the text
  completely.

  For example ``\hyperlink{command.by}{\mbox{\isa{\isacommand{by}}}}~\isa{{\isachardoublequote}{\isacharpercent}invisible\ auto{\isachardoublequote}}'' causes
  that piece of proof to be treated as \isa{invisible} instead of
  \isa{{\isachardoublequote}proof{\isachardoublequote}} (the default), which may be shown or hidden
  depending on the document setup.  In contrast, ``\hyperlink{command.by}{\mbox{\isa{\isacommand{by}}}}~\isa{{\isachardoublequote}{\isacharpercent}visible\ auto{\isachardoublequote}}'' forces this text to be shown
  invariably.

  Explicit tag specifications within a proof apply to all subsequent
  commands of the same level of nesting.  For example, ``\hyperlink{command.proof}{\mbox{\isa{\isacommand{proof}}}}~\isa{{\isachardoublequote}{\isacharpercent}visible\ {\isasymdots}{\isachardoublequote}}~\hyperlink{command.qed}{\mbox{\isa{\isacommand{qed}}}}'' forces the whole
  sub-proof to be typeset as \isa{visible} (unless some of its parts
  are tagged differently).

  \medskip Command tags merely produce certain markup environments for
  type-setting.  The meaning of these is determined by {\LaTeX}
  macros, as defined in \hyperlink{file.~~/lib/texinputs/isabelle.sty}{\mbox{\isa{\isatt{{\isachartilde}{\isachartilde}{\isacharslash}lib{\isacharslash}texinputs{\isacharslash}isabelle{\isachardot}sty}}}} or
  by the document author.  The Isabelle document preparation tools
  also provide some high-level options to specify the meaning of
  arbitrary tags to ``keep'', ``drop'', or ``fold'' the corresponding
  parts of the text.  Logic sessions may also specify ``document
  versions'', where given tags are interpreted in some particular way.
  Again see \cite{isabelle-sys} for further details.%
\end{isamarkuptext}%
\isamarkuptrue%
%
\isamarkupsection{Draft presentation%
}
\isamarkuptrue%
%
\begin{isamarkuptext}%
\begin{matharray}{rcl}
    \indexdef{}{command}{display\_drafts}\hypertarget{command.display-drafts}{\hyperlink{command.display-drafts}{\mbox{\isa{\isacommand{display{\isacharunderscore}drafts}}}}}\isa{{\isachardoublequote}\isactrlsup {\isacharasterisk}{\isachardoublequote}} & : & \isa{{\isachardoublequote}any\ {\isasymrightarrow}{\isachardoublequote}} \\
    \indexdef{}{command}{print\_drafts}\hypertarget{command.print-drafts}{\hyperlink{command.print-drafts}{\mbox{\isa{\isacommand{print{\isacharunderscore}drafts}}}}}\isa{{\isachardoublequote}\isactrlsup {\isacharasterisk}{\isachardoublequote}} & : & \isa{{\isachardoublequote}any\ {\isasymrightarrow}{\isachardoublequote}} \\
  \end{matharray}

  \begin{rail}
    ('display\_drafts' | 'print\_drafts') (name +)
    ;
  \end{rail}

  \begin{description}

  \item \hyperlink{command.display-drafts}{\mbox{\isa{\isacommand{display{\isacharunderscore}drafts}}}}~\isa{paths} and \hyperlink{command.print-drafts}{\mbox{\isa{\isacommand{print{\isacharunderscore}drafts}}}}~\isa{paths} perform simple output of a given list
  of raw source files.  Only those symbols that do not require
  additional {\LaTeX} packages are displayed properly, everything else
  is left verbatim.

  \end{description}%
\end{isamarkuptext}%
\isamarkuptrue%
%
\isadelimtheory
%
\endisadelimtheory
%
\isatagtheory
\isacommand{end}\isamarkupfalse%
%
\endisatagtheory
{\isafoldtheory}%
%
\isadelimtheory
%
\endisadelimtheory
\isanewline
\end{isabellebody}%
%%% Local Variables:
%%% mode: latex
%%% TeX-master: "root"
%%% End:
