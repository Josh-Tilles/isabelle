%
\begin{isabellebody}%
\def\isabellecontext{Introduction}%
%
\isadelimtheory
%
\endisadelimtheory
%
\isatagtheory
\isacommand{theory}\isamarkupfalse%
\ Introduction\isanewline
\isakeyword{imports}\ Main\isanewline
\isakeyword{begin}%
\endisatagtheory
{\isafoldtheory}%
%
\isadelimtheory
%
\endisadelimtheory
%
\isamarkupchapter{Introduction%
}
\isamarkuptrue%
%
\isamarkupsection{Overview%
}
\isamarkuptrue%
%
\begin{isamarkuptext}%
The \emph{Isabelle} system essentially provides a generic
  infrastructure for building deductive systems (programmed in
  Standard ML), with a special focus on interactive theorem proving in
  higher-order logics.  Many years ago, even end-users would refer to
  certain ML functions (goal commands, tactics, tacticals etc.) to
  pursue their everyday theorem proving tasks.
  
  In contrast \emph{Isar} provides an interpreted language environment
  of its own, which has been specifically tailored for the needs of
  theory and proof development.  Compared to raw ML, the Isabelle/Isar
  top-level provides a more robust and comfortable development
  platform, with proper support for theory development graphs, managed
  transactions with unlimited undo etc.  The Isabelle/Isar version of
  the \emph{Proof~General} user interface
  \cite{proofgeneral,Aspinall:TACAS:2000} provides a decent front-end
  for interactive theory and proof development in this advanced
  theorem proving environment, even though it is somewhat biased
  towards old-style proof scripts.

  \medskip Apart from the technical advances over bare-bones ML
  programming, the main purpose of the Isar language is to provide a
  conceptually different view on machine-checked proofs
  \cite{Wenzel:1999:TPHOL,Wenzel-PhD}.  \emph{Isar} stands for
  \emph{Intelligible semi-automated reasoning}.  Drawing from both the
  traditions of informal mathematical proof texts and high-level
  programming languages, Isar offers a versatile environment for
  structured formal proof documents.  Thus properly written Isar
  proofs become accessible to a broader audience than unstructured
  tactic scripts (which typically only provide operational information
  for the machine).  Writing human-readable proof texts certainly
  requires some additional efforts by the writer to achieve a good
  presentation, both of formal and informal parts of the text.  On the
  other hand, human-readable formal texts gain some value in their own
  right, independently of the mechanic proof-checking process.

  Despite its grand design of structured proof texts, Isar is able to
  assimilate the old tactical style as an ``improper'' sub-language.
  This provides an easy upgrade path for existing tactic scripts, as
  well as some means for interactive experimentation and debugging of
  structured proofs.  Isabelle/Isar supports a broad range of proof
  styles, both readable and unreadable ones.

  \medskip The generic Isabelle/Isar framework (see
  \chref{ch:isar-framework}) works reasonably well for any Isabelle
  object-logic that conforms to the natural deduction view of the
  Isabelle/Pure framework.  Specific language elements introduced by
  the major object-logics are described in \chref{ch:hol}
  (Isabelle/HOL), \chref{ch:holcf} (Isabelle/HOLCF), and \chref{ch:zf}
  (Isabelle/ZF).  The main language elements are already provided by
  the Isabelle/Pure framework. Nevertheless, examples given in the
  generic parts will usually refer to Isabelle/HOL as well.

  \medskip Isar commands may be either \emph{proper} document
  constructors, or \emph{improper commands}.  Some proof methods and
  attributes introduced later are classified as improper as well.
  Improper Isar language elements, which are marked by ``\isa{{\isachardoublequote}\isactrlsup {\isacharasterisk}{\isachardoublequote}}'' in the subsequent chapters; they are often helpful
  when developing proof documents, but their use is discouraged for
  the final human-readable outcome.  Typical examples are diagnostic
  commands that print terms or theorems according to the current
  context; other commands emulate old-style tactical theorem proving.%
\end{isamarkuptext}%
\isamarkuptrue%
%
\isadelimtheory
%
\endisadelimtheory
%
\isatagtheory
\isacommand{end}\isamarkupfalse%
%
\endisatagtheory
{\isafoldtheory}%
%
\isadelimtheory
%
\endisadelimtheory
\isanewline
\end{isabellebody}%
%%% Local Variables:
%%% mode: latex
%%% TeX-master: "root"
%%% End:
