%
\begin{isabellebody}%
\def\isabellecontext{Inner{\isacharunderscore}Syntax}%
%
\isadelimtheory
%
\endisadelimtheory
%
\isatagtheory
\isacommand{theory}\isamarkupfalse%
\ Inner{\isacharunderscore}Syntax\isanewline
\isakeyword{imports}\ Main\isanewline
\isakeyword{begin}%
\endisatagtheory
{\isafoldtheory}%
%
\isadelimtheory
%
\endisadelimtheory
%
\isamarkupchapter{Inner syntax --- the term language \label{ch:inner-syntax}%
}
\isamarkuptrue%
%
\isamarkupsection{Printing logical entities%
}
\isamarkuptrue%
%
\isamarkupsubsection{Diagnostic commands%
}
\isamarkuptrue%
%
\begin{isamarkuptext}%
\begin{matharray}{rcl}
    \indexdef{}{command}{typ}\hypertarget{command.typ}{\hyperlink{command.typ}{\mbox{\isa{\isacommand{typ}}}}}\isa{{\isachardoublequote}\isactrlsup {\isacharasterisk}{\isachardoublequote}} & : & \isa{{\isachardoublequote}context\ {\isasymrightarrow}{\isachardoublequote}} \\
    \indexdef{}{command}{term}\hypertarget{command.term}{\hyperlink{command.term}{\mbox{\isa{\isacommand{term}}}}}\isa{{\isachardoublequote}\isactrlsup {\isacharasterisk}{\isachardoublequote}} & : & \isa{{\isachardoublequote}context\ {\isasymrightarrow}{\isachardoublequote}} \\
    \indexdef{}{command}{prop}\hypertarget{command.prop}{\hyperlink{command.prop}{\mbox{\isa{\isacommand{prop}}}}}\isa{{\isachardoublequote}\isactrlsup {\isacharasterisk}{\isachardoublequote}} & : & \isa{{\isachardoublequote}context\ {\isasymrightarrow}{\isachardoublequote}} \\
    \indexdef{}{command}{thm}\hypertarget{command.thm}{\hyperlink{command.thm}{\mbox{\isa{\isacommand{thm}}}}}\isa{{\isachardoublequote}\isactrlsup {\isacharasterisk}{\isachardoublequote}} & : & \isa{{\isachardoublequote}context\ {\isasymrightarrow}{\isachardoublequote}} \\
    \indexdef{}{command}{prf}\hypertarget{command.prf}{\hyperlink{command.prf}{\mbox{\isa{\isacommand{prf}}}}}\isa{{\isachardoublequote}\isactrlsup {\isacharasterisk}{\isachardoublequote}} & : & \isa{{\isachardoublequote}context\ {\isasymrightarrow}{\isachardoublequote}} \\
    \indexdef{}{command}{full\_prf}\hypertarget{command.full-prf}{\hyperlink{command.full-prf}{\mbox{\isa{\isacommand{full{\isacharunderscore}prf}}}}}\isa{{\isachardoublequote}\isactrlsup {\isacharasterisk}{\isachardoublequote}} & : & \isa{{\isachardoublequote}context\ {\isasymrightarrow}{\isachardoublequote}} \\
    \indexdef{}{command}{pr}\hypertarget{command.pr}{\hyperlink{command.pr}{\mbox{\isa{\isacommand{pr}}}}}\isa{{\isachardoublequote}\isactrlsup {\isacharasterisk}{\isachardoublequote}} & : & \isa{{\isachardoublequote}any\ {\isasymrightarrow}{\isachardoublequote}} \\
  \end{matharray}

  These diagnostic commands assist interactive development by printing
  internal logical entities in a human-readable fashion.

  \begin{rail}
    'typ' modes? type
    ;
    'term' modes? term
    ;
    'prop' modes? prop
    ;
    'thm' modes? thmrefs
    ;
    ( 'prf' | 'full\_prf' ) modes? thmrefs?
    ;
    'pr' modes? nat? (',' nat)?
    ;

    modes: '(' (name + ) ')'
    ;
  \end{rail}

  \begin{description}

  \item \hyperlink{command.typ}{\mbox{\isa{\isacommand{typ}}}}~\isa{{\isasymtau}} reads and prints types of the
  meta-logic according to the current theory or proof context.

  \item \hyperlink{command.term}{\mbox{\isa{\isacommand{term}}}}~\isa{t} and \hyperlink{command.prop}{\mbox{\isa{\isacommand{prop}}}}~\isa{{\isasymphi}}
  read, type-check and print terms or propositions according to the
  current theory or proof context; the inferred type of \isa{t} is
  output as well.  Note that these commands are also useful in
  inspecting the current environment of term abbreviations.

  \item \hyperlink{command.thm}{\mbox{\isa{\isacommand{thm}}}}~\isa{{\isachardoublequote}a\isactrlsub {\isadigit{1}}\ {\isasymdots}\ a\isactrlsub n{\isachardoublequote}} retrieves
  theorems from the current theory or proof context.  Note that any
  attributes included in the theorem specifications are applied to a
  temporary context derived from the current theory or proof; the
  result is discarded, i.e.\ attributes involved in \isa{{\isachardoublequote}a\isactrlsub {\isadigit{1}}{\isacharcomma}\ {\isasymdots}{\isacharcomma}\ a\isactrlsub n{\isachardoublequote}} do not have any permanent effect.

  \item \hyperlink{command.prf}{\mbox{\isa{\isacommand{prf}}}} displays the (compact) proof term of the
  current proof state (if present), or of the given theorems. Note
  that this requires proof terms to be switched on for the current
  object logic (see the ``Proof terms'' section of the Isabelle
  reference manual for information on how to do this).

  \item \hyperlink{command.full-prf}{\mbox{\isa{\isacommand{full{\isacharunderscore}prf}}}} is like \hyperlink{command.prf}{\mbox{\isa{\isacommand{prf}}}}, but displays
  the full proof term, i.e.\ also displays information omitted in the
  compact proof term, which is denoted by ``\isa{{\isacharunderscore}}'' placeholders
  there.

  \item \hyperlink{command.pr}{\mbox{\isa{\isacommand{pr}}}}~\isa{{\isachardoublequote}goals{\isacharcomma}\ prems{\isachardoublequote}} prints the current
  proof state (if present), including the proof context, current facts
  and goals.  The optional limit arguments affect the number of goals
  and premises to be displayed, which is initially 10 for both.
  Omitting limit values leaves the current setting unchanged.

  \end{description}

  All of the diagnostic commands above admit a list of \isa{modes}
  to be specified, which is appended to the current print mode (see
  also \cite{isabelle-ref}).  Thus the output behavior may be modified
  according particular print mode features.  For example, \hyperlink{command.pr}{\mbox{\isa{\isacommand{pr}}}}~\isa{{\isachardoublequote}{\isacharparenleft}latex\ xsymbols{\isacharparenright}{\isachardoublequote}} would print the current proof state
  with mathematical symbols and special characters represented in
  {\LaTeX} source, according to the Isabelle style
  \cite{isabelle-sys}.

  Note that antiquotations (cf.\ \secref{sec:antiq}) provide a more
  systematic way to include formal items into the printed text
  document.%
\end{isamarkuptext}%
\isamarkuptrue%
%
\isamarkupsubsection{Details of printed content%
}
\isamarkuptrue%
%
\begin{isamarkuptext}%
\begin{mldecls} 
    \indexdef{}{ML}{show\_types}\verb|show_types: bool Unsynchronized.ref| & default \verb|false| \\
    \indexdef{}{ML}{show\_sorts}\verb|show_sorts: bool Unsynchronized.ref| & default \verb|false| \\
    \indexdef{}{ML}{show\_consts}\verb|show_consts: bool Config.T| & default \verb|false| \\
    \indexdef{}{ML}{long\_names}\verb|long_names: bool Unsynchronized.ref| & default \verb|false| \\
    \indexdef{}{ML}{short\_names}\verb|short_names: bool Unsynchronized.ref| & default \verb|false| \\
    \indexdef{}{ML}{unique\_names}\verb|unique_names: bool Unsynchronized.ref| & default \verb|true| \\
    \indexdef{}{ML}{show\_brackets}\verb|show_brackets: bool Unsynchronized.ref| & default \verb|false| \\
    \indexdef{}{ML}{eta\_contract}\verb|eta_contract: bool Unsynchronized.ref| & default \verb|true| \\
    \indexdef{}{ML}{goals\_limit}\verb|goals_limit: int Unsynchronized.ref| & default \verb|10| \\
    \indexdef{}{ML}{Proof.show\_main\_goal}\verb|Proof.show_main_goal: bool Unsynchronized.ref| & default \verb|false| \\
    \indexdef{}{ML}{show\_hyps}\verb|show_hyps: bool Unsynchronized.ref| & default \verb|false| \\
    \indexdef{}{ML}{show\_tags}\verb|show_tags: bool Unsynchronized.ref| & default \verb|false| \\
    \indexdef{}{ML}{show\_question\_marks}\verb|show_question_marks: bool Config.T| & default \verb|true| \\
  \end{mldecls}

  These global ML variables control the detail of information that is
  displayed for types, terms, theorems, goals etc.

  In interactive sessions, the user interface usually manages these
  global parameters of the Isabelle process, even with some concept of
  persistence.  Nonetheless it is occasionally useful to manipulate ML
  variables directly, e.g.\ using \hyperlink{command.ML-val}{\mbox{\isa{\isacommand{ML{\isacharunderscore}val}}}} or \hyperlink{command.ML-command}{\mbox{\isa{\isacommand{ML{\isacharunderscore}command}}}}.

  Batch-mode logic sessions may be configured by putting appropriate
  ML text directly into the \verb|ROOT.ML| file.

  \begin{description}

  \item \verb|show_types| and \verb|show_sorts| control printing of type
  constraints for term variables, and sort constraints for type
  variables.  By default, neither of these are shown in output.  If
  \verb|show_sorts| is set to \verb|true|, types are always shown as
  well.

  Note that displaying types and sorts may explain why a polymorphic
  inference rule fails to resolve with some goal, or why a rewrite
  rule does not apply as expected.

  \item \verb|show_consts| controls printing of types of constants when
  displaying a goal state.

  Note that the output can be enormous, because polymorphic constants
  often occur at several different type instances.

  \item \verb|long_names|, \verb|short_names|, and \verb|unique_names|
  control the way of printing fully qualified internal names in
  external form.  See also \secref{sec:antiq} for the document
  antiquotation options of the same names.

  \item \verb|show_brackets| controls bracketing in pretty printed
  output.  If set to \verb|true|, all sub-expressions of the pretty
  printing tree will be parenthesized, even if this produces malformed
  term syntax!  This crude way of showing the internal structure of
  pretty printed entities may occasionally help to diagnose problems
  with operator priorities, for example.

  \item \verb|eta_contract| controls \isa{{\isachardoublequote}{\isasymeta}{\isachardoublequote}}-contracted printing of
  terms.

  The \isa{{\isasymeta}}-contraction law asserts \isa{{\isachardoublequote}{\isacharparenleft}{\isasymlambda}x{\isachardot}\ f\ x{\isacharparenright}\ {\isasymequiv}\ f{\isachardoublequote}},
  provided \isa{x} is not free in \isa{f}.  It asserts
  \emph{extensionality} of functions: \isa{{\isachardoublequote}f\ {\isasymequiv}\ g{\isachardoublequote}} if \isa{{\isachardoublequote}f\ x\ {\isasymequiv}\ g\ x{\isachardoublequote}} for all \isa{x}.  Higher-order unification frequently puts
  terms into a fully \isa{{\isasymeta}}-expanded form.  For example, if \isa{F} has type \isa{{\isachardoublequote}{\isacharparenleft}{\isasymtau}\ {\isasymRightarrow}\ {\isasymtau}{\isacharparenright}\ {\isasymRightarrow}\ {\isasymtau}{\isachardoublequote}} then its expanded form is \isa{{\isachardoublequote}{\isasymlambda}h{\isachardot}\ F\ {\isacharparenleft}{\isasymlambda}x{\isachardot}\ h\ x{\isacharparenright}{\isachardoublequote}}.

  Setting \verb|eta_contract| makes Isabelle perform \isa{{\isasymeta}}-contractions before printing, so that \isa{{\isachardoublequote}{\isasymlambda}h{\isachardot}\ F\ {\isacharparenleft}{\isasymlambda}x{\isachardot}\ h\ x{\isacharparenright}{\isachardoublequote}}
  appears simply as \isa{F}.

  Note that the distinction between a term and its \isa{{\isasymeta}}-expanded
  form occasionally matters.  While higher-order resolution and
  rewriting operate modulo \isa{{\isachardoublequote}{\isasymalpha}{\isasymbeta}{\isasymeta}{\isachardoublequote}}-conversion, some other tools
  might look at terms more discretely.

  \item \verb|goals_limit| controls the maximum number of subgoals to
  be shown in goal output.

  \item \verb|Proof.show_main_goal| controls whether the main result to
  be proven should be displayed.  This information might be relevant
  for schematic goals, to inspect the current claim that has been
  synthesized so far.

  \item \verb|show_hyps| controls printing of implicit hypotheses of
  local facts.  Normally, only those hypotheses are displayed that are
  \emph{not} covered by the assumptions of the current context: this
  situation indicates a fault in some tool being used.

  By setting \verb|show_hyps| to \verb|true|, output of \emph{all}
  hypotheses can be enforced, which is occasionally useful for
  diagnostic purposes.

  \item \verb|show_tags| controls printing of extra annotations within
  theorems, such as internal position information, or the case names
  being attached by the attribute \hyperlink{attribute.case-names}{\mbox{\isa{case{\isacharunderscore}names}}}.

  Note that the \hyperlink{attribute.tagged}{\mbox{\isa{tagged}}} and \hyperlink{attribute.untagged}{\mbox{\isa{untagged}}}
  attributes provide low-level access to the collection of tags
  associated with a theorem.

  \item \verb|show_question_marks| controls printing of question marks
  for schematic variables, such as \isa{{\isacharquery}x}.  Only the leading
  question mark is affected, the remaining text is unchanged
  (including proper markup for schematic variables that might be
  relevant for user interfaces).

  \end{description}%
\end{isamarkuptext}%
\isamarkuptrue%
%
\isamarkupsubsection{Printing limits%
}
\isamarkuptrue%
%
\begin{isamarkuptext}%
\begin{mldecls}
    \indexdef{}{ML}{Pretty.margin\_default}\verb|Pretty.margin_default: int Unsynchronized.ref| \\
    \indexdef{}{ML}{print\_depth}\verb|print_depth: int -> unit| \\
  \end{mldecls}

  These ML functions set limits for pretty printed text.

  \begin{description}

  \item \verb|Pretty.margin_default| indicates the global default for
  the right margin of the built-in pretty printer, with initial value
  76.  Note that user-interfaces typically control margins
  automatically when resizing windows, or even bypass the formatting
  engine of Isabelle/ML altogether and do it within the front end via
  Isabelle/Scala.

  \item \verb|print_depth|~\isa{n} limits the printing depth of the
  ML toplevel pretty printer; the precise effect depends on the ML
  compiler and run-time system.  Typically \isa{n} should be less
  than 10.  Bigger values such as 100--1000 are useful for debugging.

  \end{description}%
\end{isamarkuptext}%
\isamarkuptrue%
%
\isamarkupsection{Mixfix annotations%
}
\isamarkuptrue%
%
\begin{isamarkuptext}%
Mixfix annotations specify concrete \emph{inner syntax} of
  Isabelle types and terms.  Locally fixed parameters in toplevel
  theorem statements, locale specifications etc.\ also admit mixfix
  annotations.

  \indexouternonterm{infix}\indexouternonterm{mixfix}\indexouternonterm{structmixfix}
  \begin{rail}
    infix: '(' ('infix' | 'infixl' | 'infixr') string nat ')'
    ;
    mixfix: infix | '(' string prios? nat? ')' | '(' 'binder' string prios? nat ')'
    ;
    structmixfix: mixfix | '(' 'structure' ')'
    ;

    prios: '[' (nat + ',') ']'
    ;
  \end{rail}

  Here the \railtok{string} specifications refer to the actual mixfix
  template, which may include literal text, spacing, blocks, and
  arguments (denoted by ``\isa{{\isacharunderscore}}''); the special symbol
  ``\verb|\<index>|'' (printed as ``\isa{{\isachardoublequote}{\isasymindex}{\isachardoublequote}}'') represents an index
  argument that specifies an implicit structure reference (see also
  \secref{sec:locale}).  Infix and binder declarations provide common
  abbreviations for particular mixfix declarations.  So in practice,
  mixfix templates mostly degenerate to literal text for concrete
  syntax, such as ``\verb|++|'' for an infix symbol.

  \medskip In full generality, mixfix declarations work as follows.
  Suppose a constant \isa{{\isachardoublequote}c\ {\isacharcolon}{\isacharcolon}\ {\isasymtau}\isactrlsub {\isadigit{1}}\ {\isasymRightarrow}\ {\isasymdots}\ {\isasymtau}\isactrlsub n\ {\isasymRightarrow}\ {\isasymtau}{\isachardoublequote}} is
  annotated by \isa{{\isachardoublequote}{\isacharparenleft}mixfix\ {\isacharbrackleft}p\isactrlsub {\isadigit{1}}{\isacharcomma}\ {\isasymdots}{\isacharcomma}\ p\isactrlsub n{\isacharbrackright}\ p{\isacharparenright}{\isachardoublequote}}, where \isa{{\isachardoublequote}mixfix{\isachardoublequote}} is a string \isa{{\isachardoublequote}d\isactrlsub {\isadigit{0}}\ {\isacharunderscore}\ d\isactrlsub {\isadigit{1}}\ {\isacharunderscore}\ {\isasymdots}\ {\isacharunderscore}\ d\isactrlsub n{\isachardoublequote}} consisting of
  delimiters that surround argument positions as indicated by
  underscores.

  Altogether this determines a production for a context-free priority
  grammar, where for each argument \isa{{\isachardoublequote}i{\isachardoublequote}} the syntactic category
  is determined by \isa{{\isachardoublequote}{\isasymtau}\isactrlsub i{\isachardoublequote}} (with priority \isa{{\isachardoublequote}p\isactrlsub i{\isachardoublequote}}), and
  the result category is determined from \isa{{\isachardoublequote}{\isasymtau}{\isachardoublequote}} (with
  priority \isa{{\isachardoublequote}p{\isachardoublequote}}).  Priority specifications are optional, with
  default 0 for arguments and 1000 for the result.

  Since \isa{{\isachardoublequote}{\isasymtau}{\isachardoublequote}} may be again a function type, the constant
  type scheme may have more argument positions than the mixfix
  pattern.  Printing a nested application \isa{{\isachardoublequote}c\ t\isactrlsub {\isadigit{1}}\ {\isasymdots}\ t\isactrlsub m{\isachardoublequote}} for
  \isa{{\isachardoublequote}m\ {\isachargreater}\ n{\isachardoublequote}} works by attaching concrete notation only to the
  innermost part, essentially by printing \isa{{\isachardoublequote}{\isacharparenleft}c\ t\isactrlsub {\isadigit{1}}\ {\isasymdots}\ t\isactrlsub n{\isacharparenright}\ {\isasymdots}\ t\isactrlsub m{\isachardoublequote}}
  instead.  If a term has fewer arguments than specified in the mixfix
  template, the concrete syntax is ignored.

  \medskip A mixfix template may also contain additional directives
  for pretty printing, notably spaces, blocks, and breaks.  The
  general template format is a sequence over any of the following
  entities.

  \begin{description}

  \item \isa{{\isachardoublequote}d{\isachardoublequote}} is a delimiter, namely a non-empty sequence of
  characters other than the following special characters:

  \smallskip
  \begin{tabular}{ll}
    \verb|'| & single quote \\
    \verb|_| & underscore \\
    \isa{{\isachardoublequote}{\isasymindex}{\isachardoublequote}} & index symbol \\
    \verb|(| & open parenthesis \\
    \verb|)| & close parenthesis \\
    \verb|/| & slash \\
  \end{tabular}
  \medskip

  \item \verb|'| escapes the special meaning of these
  meta-characters, producing a literal version of the following
  character, unless that is a blank.

  A single quote followed by a blank separates delimiters, without
  affecting printing, but input tokens may have additional white space
  here.

  \item \verb|_| is an argument position, which stands for a
  certain syntactic category in the underlying grammar.

  \item \isa{{\isachardoublequote}{\isasymindex}{\isachardoublequote}} is an indexed argument position; this is the place
  where implicit structure arguments can be attached.

  \item \isa{{\isachardoublequote}s{\isachardoublequote}} is a non-empty sequence of spaces for printing.
  This and the following specifications do not affect parsing at all.

  \item \verb|(|\isa{n} opens a pretty printing block.  The
  optional number specifies how much indentation to add when a line
  break occurs within the block.  If the parenthesis is not followed
  by digits, the indentation defaults to 0.  A block specified via
  \verb|(00| is unbreakable.

  \item \verb|)| closes a pretty printing block.

  \item \verb|//| forces a line break.

  \item \verb|/|\isa{s} allows a line break.  Here \isa{s}
  stands for the string of spaces (zero or more) right after the
  slash.  These spaces are printed if the break is \emph{not} taken.

  \end{description}

  For example, the template \verb|(_ +/ _)| specifies an infix
  operator.  There are two argument positions; the delimiter
  \verb|+| is preceded by a space and followed by a space or
  line break; the entire phrase is a pretty printing block.

  The general idea of pretty printing with blocks and breaks is also
  described in \cite{paulson-ml2}.%
\end{isamarkuptext}%
\isamarkuptrue%
%
\isamarkupsection{Explicit notation%
}
\isamarkuptrue%
%
\begin{isamarkuptext}%
\begin{matharray}{rcll}
    \indexdef{}{command}{type\_notation}\hypertarget{command.type-notation}{\hyperlink{command.type-notation}{\mbox{\isa{\isacommand{type{\isacharunderscore}notation}}}}} & : & \isa{{\isachardoublequote}local{\isacharunderscore}theory\ {\isasymrightarrow}\ local{\isacharunderscore}theory{\isachardoublequote}} \\
    \indexdef{}{command}{no\_type\_notation}\hypertarget{command.no-type-notation}{\hyperlink{command.no-type-notation}{\mbox{\isa{\isacommand{no{\isacharunderscore}type{\isacharunderscore}notation}}}}} & : & \isa{{\isachardoublequote}local{\isacharunderscore}theory\ {\isasymrightarrow}\ local{\isacharunderscore}theory{\isachardoublequote}} \\
    \indexdef{}{command}{notation}\hypertarget{command.notation}{\hyperlink{command.notation}{\mbox{\isa{\isacommand{notation}}}}} & : & \isa{{\isachardoublequote}local{\isacharunderscore}theory\ {\isasymrightarrow}\ local{\isacharunderscore}theory{\isachardoublequote}} \\
    \indexdef{}{command}{no\_notation}\hypertarget{command.no-notation}{\hyperlink{command.no-notation}{\mbox{\isa{\isacommand{no{\isacharunderscore}notation}}}}} & : & \isa{{\isachardoublequote}local{\isacharunderscore}theory\ {\isasymrightarrow}\ local{\isacharunderscore}theory{\isachardoublequote}} \\
    \indexdef{}{command}{write}\hypertarget{command.write}{\hyperlink{command.write}{\mbox{\isa{\isacommand{write}}}}} & : & \isa{{\isachardoublequote}proof{\isacharparenleft}state{\isacharparenright}\ {\isasymrightarrow}\ proof{\isacharparenleft}state{\isacharparenright}{\isachardoublequote}} \\
  \end{matharray}

  \begin{rail}
    ('type\_notation' | 'no\_type\_notation') target? mode? \\ (nameref mixfix + 'and')
    ;
    ('notation' | 'no\_notation') target? mode? \\ (nameref structmixfix + 'and')
    ;
    'write' mode? (nameref structmixfix + 'and')
    ;
  \end{rail}

  \begin{description}

  \item \hyperlink{command.type-notation}{\mbox{\isa{\isacommand{type{\isacharunderscore}notation}}}}~\isa{{\isachardoublequote}c\ {\isacharparenleft}mx{\isacharparenright}{\isachardoublequote}} associates mixfix
  syntax with an existing type constructor.  The arity of the
  constructor is retrieved from the context.
  
  \item \hyperlink{command.no-type-notation}{\mbox{\isa{\isacommand{no{\isacharunderscore}type{\isacharunderscore}notation}}}} is similar to \hyperlink{command.type-notation}{\mbox{\isa{\isacommand{type{\isacharunderscore}notation}}}}, but removes the specified syntax annotation from
  the present context.

  \item \hyperlink{command.notation}{\mbox{\isa{\isacommand{notation}}}}~\isa{{\isachardoublequote}c\ {\isacharparenleft}mx{\isacharparenright}{\isachardoublequote}} associates mixfix
  syntax with an existing constant or fixed variable.  The type
  declaration of the given entity is retrieved from the context.
  
  \item \hyperlink{command.no-notation}{\mbox{\isa{\isacommand{no{\isacharunderscore}notation}}}} is similar to \hyperlink{command.notation}{\mbox{\isa{\isacommand{notation}}}},
  but removes the specified syntax annotation from the present
  context.

  \item \hyperlink{command.write}{\mbox{\isa{\isacommand{write}}}} is similar to \hyperlink{command.notation}{\mbox{\isa{\isacommand{notation}}}}, but
  works within an Isar proof body.

  \end{description}

  Note that the more primitive commands \hyperlink{command.syntax}{\mbox{\isa{\isacommand{syntax}}}} and
  \hyperlink{command.no-syntax}{\mbox{\isa{\isacommand{no{\isacharunderscore}syntax}}}} (\secref{sec:syn-trans}) provide raw access
  to the syntax tables of a global theory.%
\end{isamarkuptext}%
\isamarkuptrue%
%
\isamarkupsection{The Pure syntax \label{sec:pure-syntax}%
}
\isamarkuptrue%
%
\isamarkupsubsection{Priority grammars \label{sec:priority-grammar}%
}
\isamarkuptrue%
%
\begin{isamarkuptext}%
A context-free grammar consists of a set of \emph{terminal
  symbols}, a set of \emph{nonterminal symbols} and a set of
  \emph{productions}.  Productions have the form \isa{{\isachardoublequote}A\ {\isacharequal}\ {\isasymgamma}{\isachardoublequote}},
  where \isa{A} is a nonterminal and \isa{{\isasymgamma}} is a string of
  terminals and nonterminals.  One designated nonterminal is called
  the \emph{root symbol}.  The language defined by the grammar
  consists of all strings of terminals that can be derived from the
  root symbol by applying productions as rewrite rules.

  The standard Isabelle parser for inner syntax uses a \emph{priority
  grammar}.  Each nonterminal is decorated by an integer priority:
  \isa{{\isachardoublequote}A\isactrlsup {\isacharparenleft}\isactrlsup p\isactrlsup {\isacharparenright}{\isachardoublequote}}.  In a derivation, \isa{{\isachardoublequote}A\isactrlsup {\isacharparenleft}\isactrlsup p\isactrlsup {\isacharparenright}{\isachardoublequote}} may be rewritten
  using a production \isa{{\isachardoublequote}A\isactrlsup {\isacharparenleft}\isactrlsup q\isactrlsup {\isacharparenright}\ {\isacharequal}\ {\isasymgamma}{\isachardoublequote}} only if \isa{{\isachardoublequote}p\ {\isasymle}\ q{\isachardoublequote}}.  Any
  priority grammar can be translated into a normal context-free
  grammar by introducing new nonterminals and productions.

  \medskip Formally, a set of context free productions \isa{G}
  induces a derivation relation \isa{{\isachardoublequote}{\isasymlongrightarrow}\isactrlsub G{\isachardoublequote}} as follows.  Let \isa{{\isasymalpha}} and \isa{{\isasymbeta}} denote strings of terminal or nonterminal symbols.
  Then \isa{{\isachardoublequote}{\isasymalpha}\ A\isactrlsup {\isacharparenleft}\isactrlsup p\isactrlsup {\isacharparenright}\ {\isasymbeta}\ {\isasymlongrightarrow}\isactrlsub G\ {\isasymalpha}\ {\isasymgamma}\ {\isasymbeta}{\isachardoublequote}} holds if and only if \isa{G}
  contains some production \isa{{\isachardoublequote}A\isactrlsup {\isacharparenleft}\isactrlsup q\isactrlsup {\isacharparenright}\ {\isacharequal}\ {\isasymgamma}{\isachardoublequote}} for \isa{{\isachardoublequote}p\ {\isasymle}\ q{\isachardoublequote}}.

  \medskip The following grammar for arithmetic expressions
  demonstrates how binding power and associativity of operators can be
  enforced by priorities.

  \begin{center}
  \begin{tabular}{rclr}
  \isa{{\isachardoublequote}A\isactrlsup {\isacharparenleft}\isactrlsup {\isadigit{1}}\isactrlsup {\isadigit{0}}\isactrlsup {\isadigit{0}}\isactrlsup {\isadigit{0}}\isactrlsup {\isacharparenright}{\isachardoublequote}} & \isa{{\isachardoublequote}{\isacharequal}{\isachardoublequote}} & \verb|(| \isa{{\isachardoublequote}A\isactrlsup {\isacharparenleft}\isactrlsup {\isadigit{0}}\isactrlsup {\isacharparenright}{\isachardoublequote}} \verb|)| \\
  \isa{{\isachardoublequote}A\isactrlsup {\isacharparenleft}\isactrlsup {\isadigit{1}}\isactrlsup {\isadigit{0}}\isactrlsup {\isadigit{0}}\isactrlsup {\isadigit{0}}\isactrlsup {\isacharparenright}{\isachardoublequote}} & \isa{{\isachardoublequote}{\isacharequal}{\isachardoublequote}} & \verb|0| \\
  \isa{{\isachardoublequote}A\isactrlsup {\isacharparenleft}\isactrlsup {\isadigit{0}}\isactrlsup {\isacharparenright}{\isachardoublequote}} & \isa{{\isachardoublequote}{\isacharequal}{\isachardoublequote}} & \isa{{\isachardoublequote}A\isactrlsup {\isacharparenleft}\isactrlsup {\isadigit{0}}\isactrlsup {\isacharparenright}{\isachardoublequote}} \verb|+| \isa{{\isachardoublequote}A\isactrlsup {\isacharparenleft}\isactrlsup {\isadigit{1}}\isactrlsup {\isacharparenright}{\isachardoublequote}} \\
  \isa{{\isachardoublequote}A\isactrlsup {\isacharparenleft}\isactrlsup {\isadigit{2}}\isactrlsup {\isacharparenright}{\isachardoublequote}} & \isa{{\isachardoublequote}{\isacharequal}{\isachardoublequote}} & \isa{{\isachardoublequote}A\isactrlsup {\isacharparenleft}\isactrlsup {\isadigit{3}}\isactrlsup {\isacharparenright}{\isachardoublequote}} \verb|*| \isa{{\isachardoublequote}A\isactrlsup {\isacharparenleft}\isactrlsup {\isadigit{2}}\isactrlsup {\isacharparenright}{\isachardoublequote}} \\
  \isa{{\isachardoublequote}A\isactrlsup {\isacharparenleft}\isactrlsup {\isadigit{3}}\isactrlsup {\isacharparenright}{\isachardoublequote}} & \isa{{\isachardoublequote}{\isacharequal}{\isachardoublequote}} & \verb|-| \isa{{\isachardoublequote}A\isactrlsup {\isacharparenleft}\isactrlsup {\isadigit{3}}\isactrlsup {\isacharparenright}{\isachardoublequote}} \\
  \end{tabular}
  \end{center}
  The choice of priorities determines that \verb|-| binds
  tighter than \verb|*|, which binds tighter than \verb|+|.  Furthermore \verb|+| associates to the left and
  \verb|*| to the right.

  \medskip For clarity, grammars obey these conventions:
  \begin{itemize}

  \item All priorities must lie between 0 and 1000.

  \item Priority 0 on the right-hand side and priority 1000 on the
  left-hand side may be omitted.

  \item The production \isa{{\isachardoublequote}A\isactrlsup {\isacharparenleft}\isactrlsup p\isactrlsup {\isacharparenright}\ {\isacharequal}\ {\isasymalpha}{\isachardoublequote}} is written as \isa{{\isachardoublequote}A\ {\isacharequal}\ {\isasymalpha}\ {\isacharparenleft}p{\isacharparenright}{\isachardoublequote}}, i.e.\ the priority of the left-hand side actually appears in
  a column on the far right.

  \item Alternatives are separated by \isa{{\isachardoublequote}{\isacharbar}{\isachardoublequote}}.

  \item Repetition is indicated by dots \isa{{\isachardoublequote}{\isacharparenleft}{\isasymdots}{\isacharparenright}{\isachardoublequote}} in an informal
  but obvious way.

  \end{itemize}

  Using these conventions, the example grammar specification above
  takes the form:
  \begin{center}
  \begin{tabular}{rclc}
    \isa{A} & \isa{{\isachardoublequote}{\isacharequal}{\isachardoublequote}} & \verb|(| \isa{A} \verb|)| \\
              & \isa{{\isachardoublequote}{\isacharbar}{\isachardoublequote}} & \verb|0| & \qquad\qquad \\
              & \isa{{\isachardoublequote}{\isacharbar}{\isachardoublequote}} & \isa{A} \verb|+| \isa{{\isachardoublequote}A\isactrlsup {\isacharparenleft}\isactrlsup {\isadigit{1}}\isactrlsup {\isacharparenright}{\isachardoublequote}} & \isa{{\isachardoublequote}{\isacharparenleft}{\isadigit{0}}{\isacharparenright}{\isachardoublequote}} \\
              & \isa{{\isachardoublequote}{\isacharbar}{\isachardoublequote}} & \isa{{\isachardoublequote}A\isactrlsup {\isacharparenleft}\isactrlsup {\isadigit{3}}\isactrlsup {\isacharparenright}{\isachardoublequote}} \verb|*| \isa{{\isachardoublequote}A\isactrlsup {\isacharparenleft}\isactrlsup {\isadigit{2}}\isactrlsup {\isacharparenright}{\isachardoublequote}} & \isa{{\isachardoublequote}{\isacharparenleft}{\isadigit{2}}{\isacharparenright}{\isachardoublequote}} \\
              & \isa{{\isachardoublequote}{\isacharbar}{\isachardoublequote}} & \verb|-| \isa{{\isachardoublequote}A\isactrlsup {\isacharparenleft}\isactrlsup {\isadigit{3}}\isactrlsup {\isacharparenright}{\isachardoublequote}} & \isa{{\isachardoublequote}{\isacharparenleft}{\isadigit{3}}{\isacharparenright}{\isachardoublequote}} \\
  \end{tabular}
  \end{center}%
\end{isamarkuptext}%
\isamarkuptrue%
%
\isamarkupsubsection{The Pure grammar%
}
\isamarkuptrue%
%
\begin{isamarkuptext}%
The priority grammar of the \isa{{\isachardoublequote}Pure{\isachardoublequote}} theory is defined as follows:

  %FIXME syntax for "index" (?)
  %FIXME "op" versions of ==> etc. (?)

  \begin{center}
  \begin{supertabular}{rclr}

  \indexdef{inner}{syntax}{any}\hypertarget{syntax.inner.any}{\hyperlink{syntax.inner.any}{\mbox{\isa{any}}}} & = & \isa{{\isachardoublequote}prop\ \ {\isacharbar}\ \ logic{\isachardoublequote}} \\\\

  \indexdef{inner}{syntax}{prop}\hypertarget{syntax.inner.prop}{\hyperlink{syntax.inner.prop}{\mbox{\isa{prop}}}} & = & \verb|(| \isa{prop} \verb|)| \\
    & \isa{{\isachardoublequote}{\isacharbar}{\isachardoublequote}} & \isa{{\isachardoublequote}prop\isactrlsup {\isacharparenleft}\isactrlsup {\isadigit{4}}\isactrlsup {\isacharparenright}{\isachardoublequote}} \verb|::| \isa{type} & \isa{{\isachardoublequote}{\isacharparenleft}{\isadigit{3}}{\isacharparenright}{\isachardoublequote}} \\
    & \isa{{\isachardoublequote}{\isacharbar}{\isachardoublequote}} & \isa{{\isachardoublequote}any\isactrlsup {\isacharparenleft}\isactrlsup {\isadigit{3}}\isactrlsup {\isacharparenright}{\isachardoublequote}} \verb|=?=| \isa{{\isachardoublequote}any\isactrlsup {\isacharparenleft}\isactrlsup {\isadigit{2}}\isactrlsup {\isacharparenright}{\isachardoublequote}} & \isa{{\isachardoublequote}{\isacharparenleft}{\isadigit{2}}{\isacharparenright}{\isachardoublequote}} \\
    & \isa{{\isachardoublequote}{\isacharbar}{\isachardoublequote}} & \isa{{\isachardoublequote}any\isactrlsup {\isacharparenleft}\isactrlsup {\isadigit{3}}\isactrlsup {\isacharparenright}{\isachardoublequote}} \verb|==| \isa{{\isachardoublequote}any\isactrlsup {\isacharparenleft}\isactrlsup {\isadigit{2}}\isactrlsup {\isacharparenright}{\isachardoublequote}} & \isa{{\isachardoublequote}{\isacharparenleft}{\isadigit{2}}{\isacharparenright}{\isachardoublequote}} \\
    & \isa{{\isachardoublequote}{\isacharbar}{\isachardoublequote}} & \isa{{\isachardoublequote}any\isactrlsup {\isacharparenleft}\isactrlsup {\isadigit{3}}\isactrlsup {\isacharparenright}{\isachardoublequote}} \isa{{\isachardoublequote}{\isasymequiv}{\isachardoublequote}} \isa{{\isachardoublequote}any\isactrlsup {\isacharparenleft}\isactrlsup {\isadigit{2}}\isactrlsup {\isacharparenright}{\isachardoublequote}} & \isa{{\isachardoublequote}{\isacharparenleft}{\isadigit{2}}{\isacharparenright}{\isachardoublequote}} \\
    & \isa{{\isachardoublequote}{\isacharbar}{\isachardoublequote}} & \isa{{\isachardoublequote}prop\isactrlsup {\isacharparenleft}\isactrlsup {\isadigit{3}}\isactrlsup {\isacharparenright}{\isachardoublequote}} \verb|&&&| \isa{{\isachardoublequote}prop\isactrlsup {\isacharparenleft}\isactrlsup {\isadigit{2}}\isactrlsup {\isacharparenright}{\isachardoublequote}} & \isa{{\isachardoublequote}{\isacharparenleft}{\isadigit{2}}{\isacharparenright}{\isachardoublequote}} \\
    & \isa{{\isachardoublequote}{\isacharbar}{\isachardoublequote}} & \isa{{\isachardoublequote}prop\isactrlsup {\isacharparenleft}\isactrlsup {\isadigit{2}}\isactrlsup {\isacharparenright}{\isachardoublequote}} \verb|==>| \isa{{\isachardoublequote}prop\isactrlsup {\isacharparenleft}\isactrlsup {\isadigit{1}}\isactrlsup {\isacharparenright}{\isachardoublequote}} & \isa{{\isachardoublequote}{\isacharparenleft}{\isadigit{1}}{\isacharparenright}{\isachardoublequote}} \\
    & \isa{{\isachardoublequote}{\isacharbar}{\isachardoublequote}} & \isa{{\isachardoublequote}prop\isactrlsup {\isacharparenleft}\isactrlsup {\isadigit{2}}\isactrlsup {\isacharparenright}{\isachardoublequote}} \isa{{\isachardoublequote}{\isasymLongrightarrow}{\isachardoublequote}} \isa{{\isachardoublequote}prop\isactrlsup {\isacharparenleft}\isactrlsup {\isadigit{1}}\isactrlsup {\isacharparenright}{\isachardoublequote}} & \isa{{\isachardoublequote}{\isacharparenleft}{\isadigit{1}}{\isacharparenright}{\isachardoublequote}} \\
    & \isa{{\isachardoublequote}{\isacharbar}{\isachardoublequote}} & \verb|[|\verb,|,\verb|| \isa{prop} \verb|;| \isa{{\isachardoublequote}{\isasymdots}{\isachardoublequote}} \verb|;| \isa{prop} \verb||\verb,|,\verb|]| \verb|==>| \isa{{\isachardoublequote}prop\isactrlsup {\isacharparenleft}\isactrlsup {\isadigit{1}}\isactrlsup {\isacharparenright}{\isachardoublequote}} & \isa{{\isachardoublequote}{\isacharparenleft}{\isadigit{1}}{\isacharparenright}{\isachardoublequote}} \\
    & \isa{{\isachardoublequote}{\isacharbar}{\isachardoublequote}} & \isa{{\isachardoublequote}{\isasymlbrakk}{\isachardoublequote}} \isa{prop} \verb|;| \isa{{\isachardoublequote}{\isasymdots}{\isachardoublequote}} \verb|;| \isa{prop} \isa{{\isachardoublequote}{\isasymrbrakk}{\isachardoublequote}} \isa{{\isachardoublequote}{\isasymLongrightarrow}{\isachardoublequote}} \isa{{\isachardoublequote}prop\isactrlsup {\isacharparenleft}\isactrlsup {\isadigit{1}}\isactrlsup {\isacharparenright}{\isachardoublequote}} & \isa{{\isachardoublequote}{\isacharparenleft}{\isadigit{1}}{\isacharparenright}{\isachardoublequote}} \\
    & \isa{{\isachardoublequote}{\isacharbar}{\isachardoublequote}} & \verb|!!| \isa{idts} \verb|.| \isa{prop} & \isa{{\isachardoublequote}{\isacharparenleft}{\isadigit{0}}{\isacharparenright}{\isachardoublequote}} \\
    & \isa{{\isachardoublequote}{\isacharbar}{\isachardoublequote}} & \isa{{\isachardoublequote}{\isasymAnd}{\isachardoublequote}} \isa{idts} \verb|.| \isa{prop} & \isa{{\isachardoublequote}{\isacharparenleft}{\isadigit{0}}{\isacharparenright}{\isachardoublequote}} \\
    & \isa{{\isachardoublequote}{\isacharbar}{\isachardoublequote}} & \verb|OFCLASS| \verb|(| \isa{type} \verb|,| \isa{logic} \verb|)| \\
    & \isa{{\isachardoublequote}{\isacharbar}{\isachardoublequote}} & \verb|SORT_CONSTRAINT| \verb|(| \isa{type} \verb|)| \\
    & \isa{{\isachardoublequote}{\isacharbar}{\isachardoublequote}} & \verb|TERM| \isa{logic} \\
    & \isa{{\isachardoublequote}{\isacharbar}{\isachardoublequote}} & \verb|PROP| \isa{aprop} \\\\

  \indexdef{inner}{syntax}{aprop}\hypertarget{syntax.inner.aprop}{\hyperlink{syntax.inner.aprop}{\mbox{\isa{aprop}}}} & = & \verb|(| \isa{aprop} \verb|)| \\
    & \isa{{\isachardoublequote}{\isacharbar}{\isachardoublequote}} & \isa{{\isachardoublequote}id\ \ {\isacharbar}\ \ longid\ \ {\isacharbar}\ \ var\ \ {\isacharbar}\ \ {\isachardoublequote}}\verb|_|\isa{{\isachardoublequote}\ \ {\isacharbar}\ \ {\isachardoublequote}}\verb|...| \\
    & \isa{{\isachardoublequote}{\isacharbar}{\isachardoublequote}} & \verb|CONST| \isa{{\isachardoublequote}id\ \ {\isacharbar}\ \ {\isachardoublequote}}\verb|CONST| \isa{{\isachardoublequote}longid{\isachardoublequote}} \\
    & \isa{{\isachardoublequote}{\isacharbar}{\isachardoublequote}} & \isa{{\isachardoublequote}logic\isactrlsup {\isacharparenleft}\isactrlsup {\isadigit{1}}\isactrlsup {\isadigit{0}}\isactrlsup {\isadigit{0}}\isactrlsup {\isadigit{0}}\isactrlsup {\isacharparenright}\ \ any\isactrlsup {\isacharparenleft}\isactrlsup {\isadigit{1}}\isactrlsup {\isadigit{0}}\isactrlsup {\isadigit{0}}\isactrlsup {\isadigit{0}}\isactrlsup {\isacharparenright}\ {\isasymdots}\ any\isactrlsup {\isacharparenleft}\isactrlsup {\isadigit{1}}\isactrlsup {\isadigit{0}}\isactrlsup {\isadigit{0}}\isactrlsup {\isadigit{0}}\isactrlsup {\isacharparenright}{\isachardoublequote}} & \isa{{\isachardoublequote}{\isacharparenleft}{\isadigit{9}}{\isadigit{9}}{\isadigit{9}}{\isacharparenright}{\isachardoublequote}} \\\\

  \indexdef{inner}{syntax}{logic}\hypertarget{syntax.inner.logic}{\hyperlink{syntax.inner.logic}{\mbox{\isa{logic}}}} & = & \verb|(| \isa{logic} \verb|)| \\
    & \isa{{\isachardoublequote}{\isacharbar}{\isachardoublequote}} & \isa{{\isachardoublequote}logic\isactrlsup {\isacharparenleft}\isactrlsup {\isadigit{4}}\isactrlsup {\isacharparenright}{\isachardoublequote}} \verb|::| \isa{type} & \isa{{\isachardoublequote}{\isacharparenleft}{\isadigit{3}}{\isacharparenright}{\isachardoublequote}} \\
    & \isa{{\isachardoublequote}{\isacharbar}{\isachardoublequote}} & \isa{{\isachardoublequote}id\ \ {\isacharbar}\ \ longid\ \ {\isacharbar}\ \ var\ \ {\isacharbar}\ \ {\isachardoublequote}}\verb|_|\isa{{\isachardoublequote}\ \ {\isacharbar}\ \ {\isachardoublequote}}\verb|...| \\
    & \isa{{\isachardoublequote}{\isacharbar}{\isachardoublequote}} & \verb|CONST| \isa{{\isachardoublequote}id\ \ {\isacharbar}\ \ {\isachardoublequote}}\verb|CONST| \isa{{\isachardoublequote}longid{\isachardoublequote}} \\
    & \isa{{\isachardoublequote}{\isacharbar}{\isachardoublequote}} & \isa{{\isachardoublequote}logic\isactrlsup {\isacharparenleft}\isactrlsup {\isadigit{1}}\isactrlsup {\isadigit{0}}\isactrlsup {\isadigit{0}}\isactrlsup {\isadigit{0}}\isactrlsup {\isacharparenright}\ \ any\isactrlsup {\isacharparenleft}\isactrlsup {\isadigit{1}}\isactrlsup {\isadigit{0}}\isactrlsup {\isadigit{0}}\isactrlsup {\isadigit{0}}\isactrlsup {\isacharparenright}\ {\isasymdots}\ any\isactrlsup {\isacharparenleft}\isactrlsup {\isadigit{1}}\isactrlsup {\isadigit{0}}\isactrlsup {\isadigit{0}}\isactrlsup {\isadigit{0}}\isactrlsup {\isacharparenright}{\isachardoublequote}} & \isa{{\isachardoublequote}{\isacharparenleft}{\isadigit{9}}{\isadigit{9}}{\isadigit{9}}{\isacharparenright}{\isachardoublequote}} \\
    & \isa{{\isachardoublequote}{\isacharbar}{\isachardoublequote}} & \verb|%| \isa{pttrns} \verb|.| \isa{{\isachardoublequote}any\isactrlsup {\isacharparenleft}\isactrlsup {\isadigit{3}}\isactrlsup {\isacharparenright}{\isachardoublequote}} & \isa{{\isachardoublequote}{\isacharparenleft}{\isadigit{3}}{\isacharparenright}{\isachardoublequote}} \\
    & \isa{{\isachardoublequote}{\isacharbar}{\isachardoublequote}} & \isa{{\isasymlambda}} \isa{pttrns} \verb|.| \isa{{\isachardoublequote}any\isactrlsup {\isacharparenleft}\isactrlsup {\isadigit{3}}\isactrlsup {\isacharparenright}{\isachardoublequote}} & \isa{{\isachardoublequote}{\isacharparenleft}{\isadigit{3}}{\isacharparenright}{\isachardoublequote}} \\
    & \isa{{\isachardoublequote}{\isacharbar}{\isachardoublequote}} & \verb|TYPE| \verb|(| \isa{type} \verb|)| \\\\

  \indexdef{inner}{syntax}{idt}\hypertarget{syntax.inner.idt}{\hyperlink{syntax.inner.idt}{\mbox{\isa{idt}}}} & = & \verb|(| \isa{idt} \verb|)|\isa{{\isachardoublequote}\ \ {\isacharbar}\ \ id\ \ {\isacharbar}\ \ {\isachardoublequote}}\verb|_| \\
    & \isa{{\isachardoublequote}{\isacharbar}{\isachardoublequote}} & \isa{id} \verb|::| \isa{type} & \isa{{\isachardoublequote}{\isacharparenleft}{\isadigit{0}}{\isacharparenright}{\isachardoublequote}} \\
    & \isa{{\isachardoublequote}{\isacharbar}{\isachardoublequote}} & \verb|_| \verb|::| \isa{type} & \isa{{\isachardoublequote}{\isacharparenleft}{\isadigit{0}}{\isacharparenright}{\isachardoublequote}} \\\\

  \indexdef{inner}{syntax}{idts}\hypertarget{syntax.inner.idts}{\hyperlink{syntax.inner.idts}{\mbox{\isa{idts}}}} & = & \isa{{\isachardoublequote}idt\ \ {\isacharbar}\ \ idt\isactrlsup {\isacharparenleft}\isactrlsup {\isadigit{1}}\isactrlsup {\isacharparenright}\ idts{\isachardoublequote}} & \isa{{\isachardoublequote}{\isacharparenleft}{\isadigit{0}}{\isacharparenright}{\isachardoublequote}} \\\\

  \indexdef{inner}{syntax}{pttrn}\hypertarget{syntax.inner.pttrn}{\hyperlink{syntax.inner.pttrn}{\mbox{\isa{pttrn}}}} & = & \isa{idt} \\\\

  \indexdef{inner}{syntax}{pttrns}\hypertarget{syntax.inner.pttrns}{\hyperlink{syntax.inner.pttrns}{\mbox{\isa{pttrns}}}} & = & \isa{{\isachardoublequote}pttrn\ \ {\isacharbar}\ \ pttrn\isactrlsup {\isacharparenleft}\isactrlsup {\isadigit{1}}\isactrlsup {\isacharparenright}\ pttrns{\isachardoublequote}} & \isa{{\isachardoublequote}{\isacharparenleft}{\isadigit{0}}{\isacharparenright}{\isachardoublequote}} \\\\

  \indexdef{inner}{syntax}{type}\hypertarget{syntax.inner.type}{\hyperlink{syntax.inner.type}{\mbox{\isa{type}}}} & = & \verb|(| \isa{type} \verb|)| \\
    & \isa{{\isachardoublequote}{\isacharbar}{\isachardoublequote}} & \isa{{\isachardoublequote}tid\ \ {\isacharbar}\ \ tvar\ \ {\isacharbar}\ \ {\isachardoublequote}}\verb|_| \\
    & \isa{{\isachardoublequote}{\isacharbar}{\isachardoublequote}} & \isa{{\isachardoublequote}tid{\isachardoublequote}} \verb|::| \isa{{\isachardoublequote}sort\ \ {\isacharbar}\ \ tvar\ \ {\isachardoublequote}}\verb|::| \isa{{\isachardoublequote}sort\ \ {\isacharbar}\ \ {\isachardoublequote}}\verb|_| \verb|::| \isa{{\isachardoublequote}sort{\isachardoublequote}} \\
    & \isa{{\isachardoublequote}{\isacharbar}{\isachardoublequote}} & \isa{{\isachardoublequote}id\ \ {\isacharbar}\ \ type\isactrlsup {\isacharparenleft}\isactrlsup {\isadigit{1}}\isactrlsup {\isadigit{0}}\isactrlsup {\isadigit{0}}\isactrlsup {\isadigit{0}}\isactrlsup {\isacharparenright}\ id\ \ {\isacharbar}\ \ {\isachardoublequote}}\verb|(| \isa{type} \verb|,| \isa{{\isachardoublequote}{\isasymdots}{\isachardoublequote}} \verb|,| \isa{type} \verb|)| \isa{id} \\
    & \isa{{\isachardoublequote}{\isacharbar}{\isachardoublequote}} & \isa{{\isachardoublequote}longid\ \ {\isacharbar}\ \ type\isactrlsup {\isacharparenleft}\isactrlsup {\isadigit{1}}\isactrlsup {\isadigit{0}}\isactrlsup {\isadigit{0}}\isactrlsup {\isadigit{0}}\isactrlsup {\isacharparenright}\ longid{\isachardoublequote}} \\
    & \isa{{\isachardoublequote}{\isacharbar}{\isachardoublequote}} & \verb|(| \isa{type} \verb|,| \isa{{\isachardoublequote}{\isasymdots}{\isachardoublequote}} \verb|,| \isa{type} \verb|)| \isa{longid} \\
    & \isa{{\isachardoublequote}{\isacharbar}{\isachardoublequote}} & \isa{{\isachardoublequote}type\isactrlsup {\isacharparenleft}\isactrlsup {\isadigit{1}}\isactrlsup {\isacharparenright}{\isachardoublequote}} \verb|=>| \isa{type} & \isa{{\isachardoublequote}{\isacharparenleft}{\isadigit{0}}{\isacharparenright}{\isachardoublequote}} \\
    & \isa{{\isachardoublequote}{\isacharbar}{\isachardoublequote}} & \isa{{\isachardoublequote}type\isactrlsup {\isacharparenleft}\isactrlsup {\isadigit{1}}\isactrlsup {\isacharparenright}{\isachardoublequote}} \isa{{\isachardoublequote}{\isasymRightarrow}{\isachardoublequote}} \isa{type} & \isa{{\isachardoublequote}{\isacharparenleft}{\isadigit{0}}{\isacharparenright}{\isachardoublequote}} \\
    & \isa{{\isachardoublequote}{\isacharbar}{\isachardoublequote}} & \verb|[| \isa{type} \verb|,| \isa{{\isachardoublequote}{\isasymdots}{\isachardoublequote}} \verb|,| \isa{type} \verb|]| \verb|=>| \isa{type} & \isa{{\isachardoublequote}{\isacharparenleft}{\isadigit{0}}{\isacharparenright}{\isachardoublequote}} \\
    & \isa{{\isachardoublequote}{\isacharbar}{\isachardoublequote}} & \verb|[| \isa{type} \verb|,| \isa{{\isachardoublequote}{\isasymdots}{\isachardoublequote}} \verb|,| \isa{type} \verb|]| \isa{{\isachardoublequote}{\isasymRightarrow}{\isachardoublequote}} \isa{type} & \isa{{\isachardoublequote}{\isacharparenleft}{\isadigit{0}}{\isacharparenright}{\isachardoublequote}} \\\\

  \indexdef{inner}{syntax}{sort}\hypertarget{syntax.inner.sort}{\hyperlink{syntax.inner.sort}{\mbox{\isa{sort}}}} & = & \isa{{\isachardoublequote}id\ \ {\isacharbar}\ \ longid\ \ {\isacharbar}\ \ {\isachardoublequote}}\verb|{}| \\
    & \isa{{\isachardoublequote}{\isacharbar}{\isachardoublequote}} & \verb|{| \isa{{\isachardoublequote}{\isacharparenleft}id\ {\isacharbar}\ longid{\isacharparenright}{\isachardoublequote}} \verb|,| \isa{{\isachardoublequote}{\isasymdots}{\isachardoublequote}} \verb|,| \isa{{\isachardoublequote}{\isacharparenleft}id\ {\isacharbar}\ longid{\isacharparenright}{\isachardoublequote}} \verb|}| \\
  \end{supertabular}
  \end{center}

  \medskip Here literal terminals are printed \verb|verbatim|;
  see also \secref{sec:inner-lex} for further token categories of the
  inner syntax.  The meaning of the nonterminals defined by the above
  grammar is as follows:

  \begin{description}

  \item \indexref{inner}{syntax}{any}\hyperlink{syntax.inner.any}{\mbox{\isa{any}}} denotes any term.

  \item \indexref{inner}{syntax}{prop}\hyperlink{syntax.inner.prop}{\mbox{\isa{prop}}} denotes meta-level propositions,
  which are terms of type \isa{prop}.  The syntax of such formulae of
  the meta-logic is carefully distinguished from usual conventions for
  object-logics.  In particular, plain \isa{{\isachardoublequote}{\isasymlambda}{\isachardoublequote}}-term notation is
  \emph{not} recognized as \hyperlink{syntax.inner.prop}{\mbox{\isa{prop}}}.

  \item \indexref{inner}{syntax}{aprop}\hyperlink{syntax.inner.aprop}{\mbox{\isa{aprop}}} denotes atomic propositions, which
  are embedded into regular \hyperlink{syntax.inner.prop}{\mbox{\isa{prop}}} by means of an
  explicit \verb|PROP| token.

  Terms of type \isa{prop} with non-constant head, e.g.\ a plain
  variable, are printed in this form.  Constants that yield type \isa{prop} are expected to provide their own concrete syntax; otherwise
  the printed version will appear like \hyperlink{syntax.inner.logic}{\mbox{\isa{logic}}} and
  cannot be parsed again as \hyperlink{syntax.inner.prop}{\mbox{\isa{prop}}}.

  \item \indexref{inner}{syntax}{logic}\hyperlink{syntax.inner.logic}{\mbox{\isa{logic}}} denotes arbitrary terms of a
  logical type, excluding type \isa{prop}.  This is the main
  syntactic category of object-logic entities, covering plain \isa{{\isasymlambda}}-term notation (variables, abstraction, application), plus
  anything defined by the user.

  When specifying notation for logical entities, all logical types
  (excluding \isa{prop}) are \emph{collapsed} to this single category
  of \hyperlink{syntax.inner.logic}{\mbox{\isa{logic}}}.

  \item \indexref{inner}{syntax}{idt}\hyperlink{syntax.inner.idt}{\mbox{\isa{idt}}} denotes identifiers, possibly
  constrained by types.

  \item \indexref{inner}{syntax}{idts}\hyperlink{syntax.inner.idts}{\mbox{\isa{idts}}} denotes a sequence of \indexref{inner}{syntax}{idt}\hyperlink{syntax.inner.idt}{\mbox{\isa{idt}}}.  This is the most basic category for variables in
  iterated binders, such as \isa{{\isachardoublequote}{\isasymlambda}{\isachardoublequote}} or \isa{{\isachardoublequote}{\isasymAnd}{\isachardoublequote}}.

  \item \indexref{inner}{syntax}{pttrn}\hyperlink{syntax.inner.pttrn}{\mbox{\isa{pttrn}}} and \indexref{inner}{syntax}{pttrns}\hyperlink{syntax.inner.pttrns}{\mbox{\isa{pttrns}}}
  denote patterns for abstraction, cases bindings etc.  In Pure, these
  categories start as a merely copy of \hyperlink{syntax.inner.idt}{\mbox{\isa{idt}}} and
  \hyperlink{syntax.inner.idts}{\mbox{\isa{idts}}}, respectively.  Object-logics may add
  additional productions for binding forms.

  \item \indexref{inner}{syntax}{type}\hyperlink{syntax.inner.type}{\mbox{\isa{type}}} denotes types of the meta-logic.

  \item \indexref{inner}{syntax}{sort}\hyperlink{syntax.inner.sort}{\mbox{\isa{sort}}} denotes meta-level sorts.

  \end{description}

  Here are some further explanations of certain syntax features.

  \begin{itemize}

  \item In \hyperlink{syntax.inner.idts}{\mbox{\isa{idts}}}, note that \isa{{\isachardoublequote}x\ {\isacharcolon}{\isacharcolon}\ nat\ y{\isachardoublequote}} is
  parsed as \isa{{\isachardoublequote}x\ {\isacharcolon}{\isacharcolon}\ {\isacharparenleft}nat\ y{\isacharparenright}{\isachardoublequote}}, treating \isa{y} like a type
  constructor applied to \isa{nat}.  To avoid this interpretation,
  write \isa{{\isachardoublequote}{\isacharparenleft}x\ {\isacharcolon}{\isacharcolon}\ nat{\isacharparenright}\ y{\isachardoublequote}} with explicit parentheses.

  \item Similarly, \isa{{\isachardoublequote}x\ {\isacharcolon}{\isacharcolon}\ nat\ y\ {\isacharcolon}{\isacharcolon}\ nat{\isachardoublequote}} is parsed as \isa{{\isachardoublequote}x\ {\isacharcolon}{\isacharcolon}\ {\isacharparenleft}nat\ y\ {\isacharcolon}{\isacharcolon}\ nat{\isacharparenright}{\isachardoublequote}}.  The correct form is \isa{{\isachardoublequote}{\isacharparenleft}x\ {\isacharcolon}{\isacharcolon}\ nat{\isacharparenright}\ {\isacharparenleft}y\ {\isacharcolon}{\isacharcolon}\ nat{\isacharparenright}{\isachardoublequote}}, or \isa{{\isachardoublequote}{\isacharparenleft}x\ {\isacharcolon}{\isacharcolon}\ nat{\isacharparenright}\ y\ {\isacharcolon}{\isacharcolon}\ nat{\isachardoublequote}} if \isa{y} is last in the
  sequence of identifiers.

  \item Type constraints for terms bind very weakly.  For example,
  \isa{{\isachardoublequote}x\ {\isacharless}\ y\ {\isacharcolon}{\isacharcolon}\ nat{\isachardoublequote}} is normally parsed as \isa{{\isachardoublequote}{\isacharparenleft}x\ {\isacharless}\ y{\isacharparenright}\ {\isacharcolon}{\isacharcolon}\ nat{\isachardoublequote}}, unless \isa{{\isachardoublequote}{\isacharless}{\isachardoublequote}} has a very low priority, in which case the
  input is likely to be ambiguous.  The correct form is \isa{{\isachardoublequote}x\ {\isacharless}\ {\isacharparenleft}y\ {\isacharcolon}{\isacharcolon}\ nat{\isacharparenright}{\isachardoublequote}}.

  \item Constraints may be either written with two literal colons
  ``\verb|::|'' or the double-colon symbol \verb|\<Colon>|,
  which actually looks exactly the same in some {\LaTeX} styles.

  \item Dummy variables (written as underscore) may occur in different
  roles.

  \begin{description}

  \item A type ``\isa{{\isachardoublequote}{\isacharunderscore}{\isachardoublequote}}'' or ``\isa{{\isachardoublequote}{\isacharunderscore}\ {\isacharcolon}{\isacharcolon}\ sort{\isachardoublequote}}'' acts like an
  anonymous inference parameter, which is filled-in according to the
  most general type produced by the type-checking phase.

  \item A bound ``\isa{{\isachardoublequote}{\isacharunderscore}{\isachardoublequote}}'' refers to a vacuous abstraction, where
  the body does not refer to the binding introduced here.  As in the
  term \isa{{\isachardoublequote}{\isasymlambda}x\ {\isacharunderscore}{\isachardot}\ x{\isachardoublequote}}, which is \isa{{\isachardoublequote}{\isasymalpha}{\isachardoublequote}}-equivalent to \isa{{\isachardoublequote}{\isasymlambda}x\ y{\isachardot}\ x{\isachardoublequote}}.

  \item A free ``\isa{{\isachardoublequote}{\isacharunderscore}{\isachardoublequote}}'' refers to an implicit outer binding.
  Higher definitional packages usually allow forms like \isa{{\isachardoublequote}f\ x\ {\isacharunderscore}\ {\isacharequal}\ x{\isachardoublequote}}.

  \item A schematic ``\isa{{\isachardoublequote}{\isacharunderscore}{\isachardoublequote}}'' (within a term pattern, see
  \secref{sec:term-decls}) refers to an anonymous variable that is
  implicitly abstracted over its context of locally bound variables.
  For example, this allows pattern matching of \isa{{\isachardoublequote}{\isacharbraceleft}x{\isachardot}\ f\ x\ {\isacharequal}\ g\ x{\isacharbraceright}{\isachardoublequote}} against \isa{{\isachardoublequote}{\isacharbraceleft}x{\isachardot}\ {\isacharunderscore}\ {\isacharequal}\ {\isacharunderscore}{\isacharbraceright}{\isachardoublequote}}, or even \isa{{\isachardoublequote}{\isacharbraceleft}{\isacharunderscore}{\isachardot}\ {\isacharunderscore}\ {\isacharequal}\ {\isacharunderscore}{\isacharbraceright}{\isachardoublequote}} by
  using both bound and schematic dummies.

  \end{description}

  \item The three literal dots ``\verb|...|'' may be also
  written as ellipsis symbol \verb|\<dots>|.  In both cases this
  refers to a special schematic variable, which is bound in the
  context.  This special term abbreviation works nicely with
  calculational reasoning (\secref{sec:calculation}).

  \end{itemize}%
\end{isamarkuptext}%
\isamarkuptrue%
%
\isamarkupsection{Lexical matters \label{sec:inner-lex}%
}
\isamarkuptrue%
%
\begin{isamarkuptext}%
The inner lexical syntax vaguely resembles the outer one
  (\secref{sec:outer-lex}), but some details are different.  There are
  two main categories of inner syntax tokens:

  \begin{enumerate}

  \item \emph{delimiters} --- the literal tokens occurring in
  productions of the given priority grammar (cf.\
  \secref{sec:priority-grammar});

  \item \emph{named tokens} --- various categories of identifiers etc.

  \end{enumerate}

  Delimiters override named tokens and may thus render certain
  identifiers inaccessible.  Sometimes the logical context admits
  alternative ways to refer to the same entity, potentially via
  qualified names.

  \medskip The categories for named tokens are defined once and for
  all as follows, reusing some categories of the outer token syntax
  (\secref{sec:outer-lex}).

  \begin{center}
  \begin{supertabular}{rcl}
    \indexdef{inner}{syntax}{id}\hypertarget{syntax.inner.id}{\hyperlink{syntax.inner.id}{\mbox{\isa{id}}}} & = & \indexref{}{syntax}{ident}\hyperlink{syntax.ident}{\mbox{\isa{ident}}} \\
    \indexdef{inner}{syntax}{longid}\hypertarget{syntax.inner.longid}{\hyperlink{syntax.inner.longid}{\mbox{\isa{longid}}}} & = & \indexref{}{syntax}{longident}\hyperlink{syntax.longident}{\mbox{\isa{longident}}} \\
    \indexdef{inner}{syntax}{var}\hypertarget{syntax.inner.var}{\hyperlink{syntax.inner.var}{\mbox{\isa{var}}}} & = & \indexref{}{syntax}{var}\hyperlink{syntax.var}{\mbox{\isa{var}}} \\
    \indexdef{inner}{syntax}{tid}\hypertarget{syntax.inner.tid}{\hyperlink{syntax.inner.tid}{\mbox{\isa{tid}}}} & = & \indexref{}{syntax}{typefree}\hyperlink{syntax.typefree}{\mbox{\isa{typefree}}} \\
    \indexdef{inner}{syntax}{tvar}\hypertarget{syntax.inner.tvar}{\hyperlink{syntax.inner.tvar}{\mbox{\isa{tvar}}}} & = & \indexref{}{syntax}{typevar}\hyperlink{syntax.typevar}{\mbox{\isa{typevar}}} \\
    \indexdef{inner}{syntax}{num}\hypertarget{syntax.inner.num}{\hyperlink{syntax.inner.num}{\mbox{\isa{num}}}} & = & \indexref{}{syntax}{nat}\hyperlink{syntax.nat}{\mbox{\isa{nat}}}\isa{{\isachardoublequote}\ \ {\isacharbar}\ \ {\isachardoublequote}}\verb|-|\indexref{}{syntax}{nat}\hyperlink{syntax.nat}{\mbox{\isa{nat}}} \\
    \indexdef{inner}{syntax}{float\_token}\hypertarget{syntax.inner.float-token}{\hyperlink{syntax.inner.float-token}{\mbox{\isa{float{\isacharunderscore}token}}}} & = & \indexref{}{syntax}{nat}\hyperlink{syntax.nat}{\mbox{\isa{nat}}}\verb|.|\indexref{}{syntax}{nat}\hyperlink{syntax.nat}{\mbox{\isa{nat}}}\isa{{\isachardoublequote}\ \ {\isacharbar}\ \ {\isachardoublequote}}\verb|-|\indexref{}{syntax}{nat}\hyperlink{syntax.nat}{\mbox{\isa{nat}}}\verb|.|\indexref{}{syntax}{nat}\hyperlink{syntax.nat}{\mbox{\isa{nat}}} \\
    \indexdef{inner}{syntax}{xnum}\hypertarget{syntax.inner.xnum}{\hyperlink{syntax.inner.xnum}{\mbox{\isa{xnum}}}} & = & \verb|#|\indexref{}{syntax}{nat}\hyperlink{syntax.nat}{\mbox{\isa{nat}}}\isa{{\isachardoublequote}\ \ {\isacharbar}\ \ {\isachardoublequote}}\verb|#-|\indexref{}{syntax}{nat}\hyperlink{syntax.nat}{\mbox{\isa{nat}}} \\

    \indexdef{inner}{syntax}{xstr}\hypertarget{syntax.inner.xstr}{\hyperlink{syntax.inner.xstr}{\mbox{\isa{xstr}}}} & = & \verb|''| \isa{{\isachardoublequote}{\isasymdots}{\isachardoublequote}} \verb|''| \\
  \end{supertabular}
  \end{center}

  The token categories \hyperlink{syntax.inner.num}{\mbox{\isa{num}}}, \hyperlink{syntax.inner.float-token}{\mbox{\isa{float{\isacharunderscore}token}}}, \hyperlink{syntax.inner.xnum}{\mbox{\isa{xnum}}}, and \hyperlink{syntax.inner.xstr}{\mbox{\isa{xstr}}} are
  not used in Pure.  Object-logics may implement numerals and string
  constants by adding appropriate syntax declarations, together with
  some translation functions (e.g.\ see Isabelle/HOL).

  The derived categories \indexdef{inner}{syntax}{num\_const}\hypertarget{syntax.inner.num-const}{\hyperlink{syntax.inner.num-const}{\mbox{\isa{num{\isacharunderscore}const}}}} and
  \indexdef{inner}{syntax}{float\_const}\hypertarget{syntax.inner.float-const}{\hyperlink{syntax.inner.float-const}{\mbox{\isa{float{\isacharunderscore}const}}}} provide robust access to \hyperlink{syntax.inner.num}{\mbox{\isa{num}}}, and \hyperlink{syntax.inner.float-token}{\mbox{\isa{float{\isacharunderscore}token}}}, respectively: the
  syntax tree holds a syntactic constant instead of a free variable.%
\end{isamarkuptext}%
\isamarkuptrue%
%
\isamarkupsection{Syntax and translations \label{sec:syn-trans}%
}
\isamarkuptrue%
%
\begin{isamarkuptext}%
\begin{matharray}{rcl}
    \indexdef{}{command}{nonterminals}\hypertarget{command.nonterminals}{\hyperlink{command.nonterminals}{\mbox{\isa{\isacommand{nonterminals}}}}} & : & \isa{{\isachardoublequote}theory\ {\isasymrightarrow}\ theory{\isachardoublequote}} \\
    \indexdef{}{command}{syntax}\hypertarget{command.syntax}{\hyperlink{command.syntax}{\mbox{\isa{\isacommand{syntax}}}}} & : & \isa{{\isachardoublequote}theory\ {\isasymrightarrow}\ theory{\isachardoublequote}} \\
    \indexdef{}{command}{no\_syntax}\hypertarget{command.no-syntax}{\hyperlink{command.no-syntax}{\mbox{\isa{\isacommand{no{\isacharunderscore}syntax}}}}} & : & \isa{{\isachardoublequote}theory\ {\isasymrightarrow}\ theory{\isachardoublequote}} \\
    \indexdef{}{command}{translations}\hypertarget{command.translations}{\hyperlink{command.translations}{\mbox{\isa{\isacommand{translations}}}}} & : & \isa{{\isachardoublequote}theory\ {\isasymrightarrow}\ theory{\isachardoublequote}} \\
    \indexdef{}{command}{no\_translations}\hypertarget{command.no-translations}{\hyperlink{command.no-translations}{\mbox{\isa{\isacommand{no{\isacharunderscore}translations}}}}} & : & \isa{{\isachardoublequote}theory\ {\isasymrightarrow}\ theory{\isachardoublequote}} \\
  \end{matharray}

  \begin{rail}
    'nonterminals' (name +)
    ;
    ('syntax' | 'no\_syntax') mode? (constdecl +)
    ;
    ('translations' | 'no\_translations') (transpat ('==' | '=>' | '<=' | rightleftharpoons | rightharpoonup | leftharpoondown) transpat +)
    ;

    mode: ('(' ( name | 'output' | name 'output' ) ')')
    ;
    transpat: ('(' nameref ')')? string
    ;
  \end{rail}

  \begin{description}
  
  \item \hyperlink{command.nonterminals}{\mbox{\isa{\isacommand{nonterminals}}}}~\isa{c} declares a type
  constructor \isa{c} (without arguments) to act as purely syntactic
  type: a nonterminal symbol of the inner syntax.

  \item \hyperlink{command.syntax}{\mbox{\isa{\isacommand{syntax}}}}~\isa{{\isachardoublequote}{\isacharparenleft}mode{\isacharparenright}\ decls{\isachardoublequote}} is similar to
  \hyperlink{command.consts}{\mbox{\isa{\isacommand{consts}}}}~\isa{decls}, except that the actual logical
  signature extension is omitted.  Thus the context free grammar of
  Isabelle's inner syntax may be augmented in arbitrary ways,
  independently of the logic.  The \isa{mode} argument refers to the
  print mode that the grammar rules belong; unless the \indexref{}{keyword}{output}\hyperlink{keyword.output}{\mbox{\isa{\isakeyword{output}}}} indicator is given, all productions are added both to the
  input and output grammar.
  
  \item \hyperlink{command.no-syntax}{\mbox{\isa{\isacommand{no{\isacharunderscore}syntax}}}}~\isa{{\isachardoublequote}{\isacharparenleft}mode{\isacharparenright}\ decls{\isachardoublequote}} removes grammar
  declarations (and translations) resulting from \isa{decls}, which
  are interpreted in the same manner as for \hyperlink{command.syntax}{\mbox{\isa{\isacommand{syntax}}}} above.
  
  \item \hyperlink{command.translations}{\mbox{\isa{\isacommand{translations}}}}~\isa{rules} specifies syntactic
  translation rules (i.e.\ macros): parse~/ print rules (\isa{{\isachardoublequote}{\isasymrightleftharpoons}{\isachardoublequote}}),
  parse rules (\isa{{\isachardoublequote}{\isasymrightharpoonup}{\isachardoublequote}}), or print rules (\isa{{\isachardoublequote}{\isasymleftharpoondown}{\isachardoublequote}}).
  Translation patterns may be prefixed by the syntactic category to be
  used for parsing; the default is \isa{logic}.
  
  \item \hyperlink{command.no-translations}{\mbox{\isa{\isacommand{no{\isacharunderscore}translations}}}}~\isa{rules} removes syntactic
  translation rules, which are interpreted in the same manner as for
  \hyperlink{command.translations}{\mbox{\isa{\isacommand{translations}}}} above.

  \end{description}%
\end{isamarkuptext}%
\isamarkuptrue%
%
\isamarkupsection{Syntax translation functions \label{sec:tr-funs}%
}
\isamarkuptrue%
%
\begin{isamarkuptext}%
\begin{matharray}{rcl}
    \indexdef{}{command}{parse\_ast\_translation}\hypertarget{command.parse-ast-translation}{\hyperlink{command.parse-ast-translation}{\mbox{\isa{\isacommand{parse{\isacharunderscore}ast{\isacharunderscore}translation}}}}} & : & \isa{{\isachardoublequote}theory\ {\isasymrightarrow}\ theory{\isachardoublequote}} \\
    \indexdef{}{command}{parse\_translation}\hypertarget{command.parse-translation}{\hyperlink{command.parse-translation}{\mbox{\isa{\isacommand{parse{\isacharunderscore}translation}}}}} & : & \isa{{\isachardoublequote}theory\ {\isasymrightarrow}\ theory{\isachardoublequote}} \\
    \indexdef{}{command}{print\_translation}\hypertarget{command.print-translation}{\hyperlink{command.print-translation}{\mbox{\isa{\isacommand{print{\isacharunderscore}translation}}}}} & : & \isa{{\isachardoublequote}theory\ {\isasymrightarrow}\ theory{\isachardoublequote}} \\
    \indexdef{}{command}{typed\_print\_translation}\hypertarget{command.typed-print-translation}{\hyperlink{command.typed-print-translation}{\mbox{\isa{\isacommand{typed{\isacharunderscore}print{\isacharunderscore}translation}}}}} & : & \isa{{\isachardoublequote}theory\ {\isasymrightarrow}\ theory{\isachardoublequote}} \\
    \indexdef{}{command}{print\_ast\_translation}\hypertarget{command.print-ast-translation}{\hyperlink{command.print-ast-translation}{\mbox{\isa{\isacommand{print{\isacharunderscore}ast{\isacharunderscore}translation}}}}} & : & \isa{{\isachardoublequote}theory\ {\isasymrightarrow}\ theory{\isachardoublequote}} \\
  \end{matharray}

  \begin{rail}
  ( 'parse\_ast\_translation' | 'parse\_translation' | 'print\_translation' |
    'typed\_print\_translation' | 'print\_ast\_translation' ) ('(advanced)')? text
  ;
  \end{rail}

  Syntax translation functions written in ML admit almost arbitrary
  manipulations of Isabelle's inner syntax.  Any of the above commands
  have a single \railqtok{text} argument that refers to an ML
  expression of appropriate type, which are as follows by default:

%FIXME proper antiquotations
\begin{ttbox}
val parse_ast_translation   : (string * (ast list -> ast)) list
val parse_translation       : (string * (term list -> term)) list
val print_translation       : (string * (term list -> term)) list
val typed_print_translation :
  (string * (bool -> typ -> term list -> term)) list
val print_ast_translation   : (string * (ast list -> ast)) list
\end{ttbox}

  If the \isa{{\isachardoublequote}{\isacharparenleft}advanced{\isacharparenright}{\isachardoublequote}} option is given, the corresponding
  translation functions may depend on the current theory or proof
  context.  This allows to implement advanced syntax mechanisms, as
  translations functions may refer to specific theory declarations or
  auxiliary proof data.

  See also \cite{isabelle-ref} for more information on the general
  concept of syntax transformations in Isabelle.

%FIXME proper antiquotations
\begin{ttbox}
val parse_ast_translation:
  (string * (Proof.context -> ast list -> ast)) list
val parse_translation:
  (string * (Proof.context -> term list -> term)) list
val print_translation:
  (string * (Proof.context -> term list -> term)) list
val typed_print_translation:
  (string * (Proof.context -> bool -> typ -> term list -> term)) list
val print_ast_translation:
  (string * (Proof.context -> ast list -> ast)) list
\end{ttbox}%
\end{isamarkuptext}%
\isamarkuptrue%
%
\isamarkupsection{Inspecting the syntax%
}
\isamarkuptrue%
%
\begin{isamarkuptext}%
\begin{matharray}{rcl}
    \indexdef{}{command}{print\_syntax}\hypertarget{command.print-syntax}{\hyperlink{command.print-syntax}{\mbox{\isa{\isacommand{print{\isacharunderscore}syntax}}}}}\isa{{\isachardoublequote}\isactrlsup {\isacharasterisk}{\isachardoublequote}} & : & \isa{{\isachardoublequote}context\ {\isasymrightarrow}{\isachardoublequote}} \\
  \end{matharray}

  \begin{description}

  \item \hyperlink{command.print-syntax}{\mbox{\isa{\isacommand{print{\isacharunderscore}syntax}}}} prints the inner syntax of the
  current context.  The output can be quite large; the most important
  sections are explained below.

  \begin{description}

  \item \isa{{\isachardoublequote}lexicon{\isachardoublequote}} lists the delimiters of the inner token
  language; see \secref{sec:inner-lex}.

  \item \isa{{\isachardoublequote}prods{\isachardoublequote}} lists the productions of the underlying
  priority grammar; see \secref{sec:priority-grammar}.

  The nonterminal \isa{{\isachardoublequote}A\isactrlsup {\isacharparenleft}\isactrlsup p\isactrlsup {\isacharparenright}{\isachardoublequote}} is rendered in plain text as \isa{{\isachardoublequote}A{\isacharbrackleft}p{\isacharbrackright}{\isachardoublequote}}; delimiters are quoted.  Many productions have an extra
  \isa{{\isachardoublequote}{\isasymdots}\ {\isacharequal}{\isachargreater}\ name{\isachardoublequote}}.  These names later become the heads of parse
  trees; they also guide the pretty printer.

  Productions without such parse tree names are called \emph{copy
  productions}.  Their right-hand side must have exactly one
  nonterminal symbol (or named token).  The parser does not create a
  new parse tree node for copy productions, but simply returns the
  parse tree of the right-hand symbol.

  If the right-hand side of a copy production consists of a single
  nonterminal without any delimiters, then it is called a \emph{chain
  production}.  Chain productions act as abbreviations: conceptually,
  they are removed from the grammar by adding new productions.
  Priority information attached to chain productions is ignored; only
  the dummy value \isa{{\isachardoublequote}{\isacharminus}{\isadigit{1}}{\isachardoublequote}} is displayed.

  \item \isa{{\isachardoublequote}print\ modes{\isachardoublequote}} lists the alternative print modes
  provided by this grammar; see \secref{sec:print-modes}.

  \item \isa{{\isachardoublequote}parse{\isacharunderscore}rules{\isachardoublequote}} and \isa{{\isachardoublequote}print{\isacharunderscore}rules{\isachardoublequote}} relate to
  syntax translations (macros); see \secref{sec:syn-trans}.

  \item \isa{{\isachardoublequote}parse{\isacharunderscore}ast{\isacharunderscore}translation{\isachardoublequote}} and \isa{{\isachardoublequote}print{\isacharunderscore}ast{\isacharunderscore}translation{\isachardoublequote}} list sets of constants that invoke
  translation functions for abstract syntax trees, which are only
  required in very special situations; see \secref{sec:tr-funs}.

  \item \isa{{\isachardoublequote}parse{\isacharunderscore}translation{\isachardoublequote}} and \isa{{\isachardoublequote}print{\isacharunderscore}translation{\isachardoublequote}}
  list the sets of constants that invoke regular translation
  functions; see \secref{sec:tr-funs}.

  \end{description}
  
  \end{description}%
\end{isamarkuptext}%
\isamarkuptrue%
%
\isadelimtheory
%
\endisadelimtheory
%
\isatagtheory
\isacommand{end}\isamarkupfalse%
%
\endisatagtheory
{\isafoldtheory}%
%
\isadelimtheory
%
\endisadelimtheory
\isanewline
\end{isabellebody}%
%%% Local Variables:
%%% mode: latex
%%% TeX-master: "root"
%%% End:
