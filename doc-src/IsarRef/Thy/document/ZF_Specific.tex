%
\begin{isabellebody}%
\def\isabellecontext{ZF{\isacharunderscore}Specific}%
%
\isadelimtheory
%
\endisadelimtheory
%
\isatagtheory
\isacommand{theory}\isamarkupfalse%
\ ZF{\isacharunderscore}Specific\isanewline
\isakeyword{imports}\ Main\isanewline
\isakeyword{begin}%
\endisatagtheory
{\isafoldtheory}%
%
\isadelimtheory
%
\endisadelimtheory
%
\isamarkupchapter{Isabelle/ZF \label{ch:zf}%
}
\isamarkuptrue%
%
\isamarkupsection{Type checking%
}
\isamarkuptrue%
%
\begin{isamarkuptext}%
The ZF logic is essentially untyped, so the concept of ``type
  checking'' is performed as logical reasoning about set-membership
  statements.  A special method assists users in this task; a version
  of this is already declared as a ``solver'' in the standard
  Simplifier setup.

  \begin{matharray}{rcl}
    \indexdef{ZF}{command}{print\_tcset}\hypertarget{command.ZF.print-tcset}{\hyperlink{command.ZF.print-tcset}{\mbox{\isa{\isacommand{print{\isacharunderscore}tcset}}}}}\isa{{\isachardoublequote}\isactrlsup {\isacharasterisk}{\isachardoublequote}} & : & \isa{{\isachardoublequote}context\ {\isasymrightarrow}{\isachardoublequote}} \\
    \indexdef{ZF}{method}{typecheck}\hypertarget{method.ZF.typecheck}{\hyperlink{method.ZF.typecheck}{\mbox{\isa{typecheck}}}} & : & \isa{method} \\
    \indexdef{ZF}{attribute}{TC}\hypertarget{attribute.ZF.TC}{\hyperlink{attribute.ZF.TC}{\mbox{\isa{TC}}}} & : & \isa{attribute} \\
  \end{matharray}

  \begin{rail}
    'TC' (() | 'add' | 'del')
    ;
  \end{rail}

  \begin{description}
  
  \item \hyperlink{command.ZF.print-tcset}{\mbox{\isa{\isacommand{print{\isacharunderscore}tcset}}}} prints the collection of
  typechecking rules of the current context.
  
  \item \hyperlink{method.ZF.typecheck}{\mbox{\isa{typecheck}}} attempts to solve any pending
  type-checking problems in subgoals.
  
  \item \hyperlink{attribute.ZF.TC}{\mbox{\isa{TC}}} adds or deletes type-checking rules from
  the context.

  \end{description}%
\end{isamarkuptext}%
\isamarkuptrue%
%
\isamarkupsection{(Co)Inductive sets and datatypes%
}
\isamarkuptrue%
%
\isamarkupsubsection{Set definitions%
}
\isamarkuptrue%
%
\begin{isamarkuptext}%
In ZF everything is a set.  The generic inductive package also
  provides a specific view for ``datatype'' specifications.
  Coinductive definitions are available in both cases, too.

  \begin{matharray}{rcl}
    \indexdef{ZF}{command}{inductive}\hypertarget{command.ZF.inductive}{\hyperlink{command.ZF.inductive}{\mbox{\isa{\isacommand{inductive}}}}} & : & \isa{{\isachardoublequote}theory\ {\isasymrightarrow}\ theory{\isachardoublequote}} \\
    \indexdef{ZF}{command}{coinductive}\hypertarget{command.ZF.coinductive}{\hyperlink{command.ZF.coinductive}{\mbox{\isa{\isacommand{coinductive}}}}} & : & \isa{{\isachardoublequote}theory\ {\isasymrightarrow}\ theory{\isachardoublequote}} \\
    \indexdef{ZF}{command}{datatype}\hypertarget{command.ZF.datatype}{\hyperlink{command.ZF.datatype}{\mbox{\isa{\isacommand{datatype}}}}} & : & \isa{{\isachardoublequote}theory\ {\isasymrightarrow}\ theory{\isachardoublequote}} \\
    \indexdef{ZF}{command}{codatatype}\hypertarget{command.ZF.codatatype}{\hyperlink{command.ZF.codatatype}{\mbox{\isa{\isacommand{codatatype}}}}} & : & \isa{{\isachardoublequote}theory\ {\isasymrightarrow}\ theory{\isachardoublequote}} \\
  \end{matharray}

  \begin{rail}
    ('inductive' | 'coinductive') domains intros hints
    ;

    domains: 'domains' (term + '+') ('<=' | subseteq) term
    ;
    intros: 'intros' (thmdecl? prop +)
    ;
    hints: monos? condefs? typeintros? typeelims?
    ;
    monos: ('monos' thmrefs)?
    ;
    condefs: ('con_defs' thmrefs)?
    ;
    typeintros: ('type_intros' thmrefs)?
    ;
    typeelims: ('type_elims' thmrefs)?
    ;
  \end{rail}

  In the following syntax specification \isa{{\isachardoublequote}monos{\isachardoublequote}}, \isa{typeintros}, and \isa{typeelims} are the same as above.

  \begin{rail}
    ('datatype' | 'codatatype') domain? (dtspec + 'and') hints
    ;

    domain: ('<=' | subseteq) term
    ;
    dtspec: term '=' (con + '|')
    ;
    con: name ('(' (term ',' +) ')')?  
    ;
    hints: monos? typeintros? typeelims?
    ;
  \end{rail}

  See \cite{isabelle-ZF} for further information on inductive
  definitions in ZF, but note that this covers the old-style theory
  format.%
\end{isamarkuptext}%
\isamarkuptrue%
%
\isamarkupsubsection{Primitive recursive functions%
}
\isamarkuptrue%
%
\begin{isamarkuptext}%
\begin{matharray}{rcl}
    \indexdef{ZF}{command}{primrec}\hypertarget{command.ZF.primrec}{\hyperlink{command.ZF.primrec}{\mbox{\isa{\isacommand{primrec}}}}} & : & \isa{{\isachardoublequote}theory\ {\isasymrightarrow}\ theory{\isachardoublequote}} \\
  \end{matharray}

  \begin{rail}
    'primrec' (thmdecl? prop +)
    ;
  \end{rail}%
\end{isamarkuptext}%
\isamarkuptrue%
%
\isamarkupsubsection{Cases and induction: emulating tactic scripts%
}
\isamarkuptrue%
%
\begin{isamarkuptext}%
The following important tactical tools of Isabelle/ZF have been
  ported to Isar.  These should not be used in proper proof texts.

  \begin{matharray}{rcl}
    \indexdef{ZF}{method}{case\_tac}\hypertarget{method.ZF.case-tac}{\hyperlink{method.ZF.case-tac}{\mbox{\isa{case{\isacharunderscore}tac}}}}\isa{{\isachardoublequote}\isactrlsup {\isacharasterisk}{\isachardoublequote}} & : & \isa{method} \\
    \indexdef{ZF}{method}{induct\_tac}\hypertarget{method.ZF.induct-tac}{\hyperlink{method.ZF.induct-tac}{\mbox{\isa{induct{\isacharunderscore}tac}}}}\isa{{\isachardoublequote}\isactrlsup {\isacharasterisk}{\isachardoublequote}} & : & \isa{method} \\
    \indexdef{ZF}{method}{ind\_cases}\hypertarget{method.ZF.ind-cases}{\hyperlink{method.ZF.ind-cases}{\mbox{\isa{ind{\isacharunderscore}cases}}}}\isa{{\isachardoublequote}\isactrlsup {\isacharasterisk}{\isachardoublequote}} & : & \isa{method} \\
    \indexdef{ZF}{command}{inductive\_cases}\hypertarget{command.ZF.inductive-cases}{\hyperlink{command.ZF.inductive-cases}{\mbox{\isa{\isacommand{inductive{\isacharunderscore}cases}}}}} & : & \isa{{\isachardoublequote}theory\ {\isasymrightarrow}\ theory{\isachardoublequote}} \\
  \end{matharray}

  \begin{rail}
    ('case_tac' | 'induct_tac') goalspec? name
    ;
    indcases (prop +)
    ;
    inductivecases (thmdecl? (prop +) + 'and')
    ;
  \end{rail}%
\end{isamarkuptext}%
\isamarkuptrue%
%
\isadelimtheory
%
\endisadelimtheory
%
\isatagtheory
\isacommand{end}\isamarkupfalse%
%
\endisatagtheory
{\isafoldtheory}%
%
\isadelimtheory
%
\endisadelimtheory
\isanewline
\end{isabellebody}%
%%% Local Variables:
%%% mode: latex
%%% TeX-master: "root"
%%% End:
