
\chapter{Object-logic Specific Elements}\label{ch:logics}

\section{General logic setup}\label{sec:object-logic}

\indexisarcmd{judgment}
\indexisarmeth{atomize}\indexisaratt{atomize}
\indexisaratt{rule-format}\indexisaratt{rulify}

\begin{matharray}{rcl}
  \isarcmd{judgment} & : & \isartrans{theory}{theory} \\
  atomize & : & \isarmeth \\
  atomize & : & \isaratt \\
  rule_format & : & \isaratt \\
  rulify & : & \isaratt \\
\end{matharray}

The very starting point for any Isabelle object-logic is a ``truth judgment''
that links object-level statements to the meta-logic (with its minimal
language of $prop$ that covers universal quantification $\Forall$ and
implication $\Imp$).

Common object-logics are sufficiently expressive to internalize rule
statements over $\Forall$ and $\Imp$ within their own language.  This is
useful in certain situations where a rule needs to be viewed as an atomic
statement from the meta-level perspective, e.g.\ $\All x x \in A \Imp P(x)$
versus $\forall x \in A. P(x)$.

From the following language elements, only the $atomize$ method and
$rule_format$ attribute are occasionally required by end-users, the rest is
for those who need to setup their own object-logic.  In the latter case
existing formulations of Isabelle/FOL or Isabelle/HOL may be taken as
realistic examples.

Generic tools may refer to the information provided by object-logic
declarations internally.

\railalias{ruleformat}{rule\_format}
\railterm{ruleformat}

\begin{rail}
  'judgment' constdecl
  ;
  'atomize' ('(' 'full' ')')?
  ;
  ruleformat ('(' 'noasm' ')')?
  ;
\end{rail}

\begin{descr}
  
\item [$\isarkeyword{judgment}~c::\sigma~~(mx)$] declares constant $c$ as the
  truth judgment of the current object-logic.  Its type $\sigma$ should
  specify a coercion of the category of object-level propositions to $prop$ of
  the Pure meta-logic; the mixfix annotation $(mx)$ would typically just link
  the object language (internally of syntactic category $logic$) with that of
  $prop$.  Only one $\isarkeyword{judgment}$ declaration may be given in any
  theory development.
  
\item [$atomize$] (as a method) rewrites any non-atomic premises of a
  sub-goal, using the meta-level equations declared via $atomize$ (as an
  attribute) beforehand.  As a result, heavily nested goals become amenable to
  fundamental operations such as resolution (cf.\ the $rule$ method) and
  proof-by-assumption (cf.\ $assumption$).  Giving the ``$(full)$'' option
  here means to turn the whole subgoal into an object-statement (if possible),
  including the outermost parameters and assumptions as well.

  A typical collection of $atomize$ rules for a particular object-logic would
  provide an internalization for each of the connectives of $\Forall$, $\Imp$,
  and $\equiv$.  Meta-level conjunction expressed in the manner of minimal
  higher-order logic as $\All{\PROP\,C} (A \Imp B \Imp \PROP\,C) \Imp PROP\,C$
  should be covered as well (this is particularly important for locales, see
  \S\ref{sec:locale}).

\item [$rule_format$] rewrites a theorem by the equalities declared as
  $rulify$ rules in the current object-logic.  By default, the result is fully
  normalized, including assumptions and conclusions at any depth.  The
  $no_asm$ option restricts the transformation to the conclusion of a rule.

  In common object-logics (HOL, FOL, ZF), the effect of $rule_format$ is to
  replace (bounded) universal quantification ($\forall$) and implication
  ($\imp$) by the corresponding rule statements over $\Forall$ and $\Imp$.

\end{descr}


\section{HOL}

\subsection{Primitive types}\label{sec:hol-typedef}

\indexisarcmdof{HOL}{typedecl}\indexisarcmdof{HOL}{typedef}
\begin{matharray}{rcl}
  \isarcmd{typedecl} & : & \isartrans{theory}{theory} \\
  \isarcmd{typedef} & : & \isartrans{theory}{proof(prove)} \\
\end{matharray}

\begin{rail}
  'typedecl' typespec infix?
  ;
  'typedef' parname? abstype '=' repset
  ;

  abstype: typespec infix?
  ;
  repset: term ('morphisms' name name)?
  ;
\end{rail}

\begin{descr}
  
\item [$\isarkeyword{typedecl}~(\vec\alpha)t$] is similar to the original
  $\isarkeyword{typedecl}$ of Isabelle/Pure (see \S\ref{sec:types-pure}), but
  also declares type arity $t :: (type, \dots, type) type$, making $t$ an
  actual HOL type constructor.
  
\item [$\isarkeyword{typedef}~(\vec\alpha)t = A$] sets up a goal stating
  non-emptiness of the set $A$.  After finishing the proof, the theory will be
  augmented by a Gordon/HOL-style type definition, which establishes a
  bijection between the representing set $A$ and the new type $t$.
  
  Technically, $\isarkeyword{typedef}$ defines both a type $t$ and a set (term
  constant) of the same name (an alternative base name may be given in
  parentheses).  The injection from type to set is called $Rep_t$, its inverse
  $Abs_t$ (this may be changed via an explicit $\isarkeyword{morphisms}$
  declaration).
  
  Theorems $Rep_t$, $Rep_t_inverse$, and $Abs_t_inverse$ provide the most
  basic characterization as a corresponding injection/surjection pair (in both
  directions).  Rules $Rep_t_inject$ and $Abs_t_inject$ provide a slightly
  more convenient view on the injectivity part, suitable for automated proof
  tools (e.g.\ in $simp$ or $iff$ declarations).  Rules
  $Rep_t_cases/Rep_t_induct$, and $Abs_t_cases/Abs_t_induct$ provide
  alternative views on surjectivity; these are already declared as set or type
  rules for the generic $cases$ and $induct$ methods.
\end{descr}

Note that raw type declarations are rarely used in practice; the main
application is with experimental (or even axiomatic!) theory fragments.
Instead of primitive HOL type definitions, user-level theories usually refer
to higher-level packages such as $\isarkeyword{record}$ (see
\S\ref{sec:hol-record}) or $\isarkeyword{datatype}$ (see
\S\ref{sec:hol-datatype}).


\subsection{Adhoc tuples}

\indexisarattof{HOL}{split-format}
\begin{matharray}{rcl}
  split_format^* & : & \isaratt \\
\end{matharray}

\railalias{splitformat}{split\_format}
\railterm{splitformat}

\begin{rail}
  splitformat (((name *) + 'and') | ('(' 'complete' ')'))
  ;
\end{rail}

\begin{descr}
  
\item [$split_format~\vec p@1 \dots \vec p@n$] puts expressions of low-level
  tuple types into canonical form as specified by the arguments given; $\vec
  p@i$ refers to occurrences in premise $i$ of the rule.  The $(complete)$
  option causes \emph{all} arguments in function applications to be
  represented canonically according to their tuple type structure.

  Note that these operations tend to invent funny names for new local
  parameters to be introduced.

\end{descr}


\subsection{Records}\label{sec:hol-record}

In principle, records merely generalize the concept of tuples, where
components may be addressed by labels instead of just position.  The logical
infrastructure of records in Isabelle/HOL is slightly more advanced, though,
supporting truly extensible record schemes.  This admits operations that are
polymorphic with respect to record extension, yielding ``object-oriented''
effects like (single) inheritance.  See also \cite{NaraschewskiW-TPHOLs98} for
more details on object-oriented verification and record subtyping in HOL.


\subsubsection{Basic concepts}

Isabelle/HOL supports both \emph{fixed} and \emph{schematic} records at the
level of terms and types.  The notation is as follows:

\begin{center}
\begin{tabular}{l|l|l}
  & record terms & record types \\ \hline
  fixed & $\record{x = a\fs y = b}$ & $\record{x \ty A\fs y \ty B}$ \\
  schematic & $\record{x = a\fs y = b\fs \more = m}$ &
    $\record{x \ty A\fs y \ty B\fs \more \ty M}$ \\
\end{tabular}
\end{center}

\noindent The ASCII representation of $\record{x = a}$ is \texttt{(| x = a |)}.

A fixed record $\record{x = a\fs y = b}$ has field $x$ of value $a$ and field
$y$ of value $b$.  The corresponding type is $\record{x \ty A\fs y \ty B}$,
assuming that $a \ty A$ and $b \ty B$.

A record scheme like $\record{x = a\fs y = b\fs \more = m}$ contains fields
$x$ and $y$ as before, but also possibly further fields as indicated by the
``$\more$'' notation (which is actually part of the syntax).  The improper
field ``$\more$'' of a record scheme is called the \emph{more part}.
Logically it is just a free variable, which is occasionally referred to as
``row variable'' in the literature.  The more part of a record scheme may be
instantiated by zero or more further components.  For example, the previous
scheme may get instantiated to $\record{x = a\fs y = b\fs z = c\fs \more =
  m'}$, where $m'$ refers to a different more part.  Fixed records are special
instances of record schemes, where ``$\more$'' is properly terminated by the
$() :: unit$ element.  Actually, $\record{x = a\fs y = b}$ is just an
abbreviation for $\record{x = a\fs y = b\fs \more = ()}$.

\medskip

Two key observations make extensible records in a simply typed language like
HOL feasible:
\begin{enumerate}
\item the more part is internalized, as a free term or type variable,
\item field names are externalized, they cannot be accessed within the logic
  as first-class values.
\end{enumerate}

\medskip

In Isabelle/HOL record types have to be defined explicitly, fixing their field
names and types, and their (optional) parent record (see
{\S}\ref{sec:hol-record-def}).  Afterwards, records may be formed using above
syntax, while obeying the canonical order of fields as given by their
declaration.  The record package provides several standard operations like
selectors and updates (see {\S}\ref{sec:hol-record-ops}).  The common setup
for various generic proof tools enable succinct reasoning patterns (see
{\S}\ref{sec:hol-record-thms}).  See also the Isabelle/HOL tutorial
\cite{isabelle-hol-book} for further instructions on using records in
practice.


\subsubsection{Record specifications}\label{sec:hol-record-def}

\indexisarcmdof{HOL}{record}
\begin{matharray}{rcl}
  \isarcmd{record} & : & \isartrans{theory}{theory} \\
\end{matharray}

\begin{rail}
  'record' typespec '=' (type '+')? (constdecl +)
  ;
\end{rail}

\begin{descr}
\item [$\isarkeyword{record}~(\vec\alpha)t = \tau + \vec c :: \vec\sigma$]
  defines extensible record type $(\vec\alpha)t$, derived from the optional
  parent record $\tau$ by adding new field components $\vec c :: \vec\sigma$.

  The type variables of $\tau$ and $\vec\sigma$ need to be covered by the
  (distinct) parameters $\vec\alpha$.  Type constructor $t$ has to be new,
  while $\tau$ needs to specify an instance of an existing record type.  At
  least one new field $\vec c$ has to be specified.  Basically, field names
  need to belong to a unique record.  This is not a real restriction in
  practice, since fields are qualified by the record name internally.

  The parent record specification $\tau$ is optional; if omitted $t$ becomes a
  root record.  The hierarchy of all records declared within a theory context
  forms a forest structure, i.e.\ a set of trees starting with a root record
  each.  There is no way to merge multiple parent records!

  For convenience, $(\vec\alpha) \, t$ is made a type abbreviation for the
  fixed record type $\record{\vec c \ty \vec\sigma}$, likewise is
  $(\vec\alpha, \zeta) \, t_scheme$ made an abbreviation for $\record{\vec c
    \ty \vec\sigma\fs \more \ty \zeta}$.

\end{descr}

\subsubsection{Record operations}\label{sec:hol-record-ops}

Any record definition of the form presented above produces certain standard
operations.  Selectors and updates are provided for any field, including the
improper one ``$more$''.  There are also cumulative record constructor
functions.  To simplify the presentation below, we assume for now that
$(\vec\alpha) \, t$ is a root record with fields $\vec c \ty \vec\sigma$.

\medskip \textbf{Selectors} and \textbf{updates} are available for any field
(including ``$more$''):
\begin{matharray}{lll}
  c@i & \ty & \record{\vec c \ty \vec \sigma, \more \ty \zeta} \To \sigma@i \\
  c@i_update & \ty & \sigma@i \To \record{\vec c \ty \vec\sigma, \more \ty \zeta} \To
    \record{\vec c \ty \vec\sigma, \more \ty \zeta}
\end{matharray}

There is special syntax for application of updates: $r \, \record{x \asn a}$
abbreviates term $x_update \, a \, r$.  Further notation for repeated updates
is also available: $r \, \record{x \asn a} \, \record{y \asn b} \, \record{z
  \asn c}$ may be written $r \, \record{x \asn a\fs y \asn b\fs z \asn c}$.
Note that because of postfix notation the order of fields shown here is
reverse than in the actual term.  Since repeated updates are just function
applications, fields may be freely permuted in $\record{x \asn a\fs y \asn
  b\fs z \asn c}$, as far as logical equality is concerned.  Thus
commutativity of independent updates can be proven within the logic for any
two fields, but not as a general theorem.

\medskip The \textbf{make} operation provides a cumulative record constructor
function:
\begin{matharray}{lll}
  t{\dtt}make & \ty & \vec\sigma \To \record{\vec c \ty \vec \sigma} \\
\end{matharray}

\medskip We now reconsider the case of non-root records, which are derived of
some parent.  In general, the latter may depend on another parent as well,
resulting in a list of \emph{ancestor records}.  Appending the lists of fields
of all ancestors results in a certain field prefix.  The record package
automatically takes care of this by lifting operations over this context of
ancestor fields.  Assuming that $(\vec\alpha) \, t$ has ancestor fields $\vec
b \ty \vec\rho$, the above record operations will get the following types:
\begin{matharray}{lll}
  c@i & \ty & \record{\vec b \ty \vec\rho, \vec c \ty \vec\sigma, \more \ty
    \zeta} \To \sigma@i \\
  c@i_update & \ty & \sigma@i \To
    \record{\vec b \ty \vec\rho, \vec c \ty \vec\sigma, \more \ty \zeta} \To
    \record{\vec b \ty \vec\rho, \vec c \ty \vec\sigma, \more \ty \zeta} \\
  t{\dtt}make & \ty & \vec\rho \To \vec\sigma \To
    \record{\vec b \ty \vec\rho, \vec c \ty \vec \sigma} \\
\end{matharray}
\noindent

\medskip Some further operations address the extension aspect of a derived
record scheme specifically: $fields$ produces a record fragment consisting of
exactly the new fields introduced here (the result may serve as a more part
elsewhere); $extend$ takes a fixed record and adds a given more part;
$truncate$ restricts a record scheme to a fixed record.

\begin{matharray}{lll}
  t{\dtt}fields & \ty & \vec\sigma \To \record{\vec c \ty \vec \sigma} \\
  t{\dtt}extend & \ty & \record{\vec d \ty \vec \rho, \vec c \ty \vec\sigma} \To
    \zeta \To \record{\vec d \ty \vec \rho, \vec c \ty \vec\sigma, \more \ty \zeta} \\
  t{\dtt}truncate & \ty & \record{\vec d \ty \vec \rho, \vec c \ty \vec\sigma, \more \ty \zeta} \To
    \record{\vec d \ty \vec \rho, \vec c \ty \vec\sigma} \\
\end{matharray}

\noindent Note that $t{\dtt}make$ and $t{\dtt}fields$ actually coincide for root records.


\subsubsection{Derived rules and proof tools}\label{sec:hol-record-thms}

The record package proves several results internally, declaring these facts to
appropriate proof tools.  This enables users to reason about record structures
quite conveniently.  Assume that $t$ is a record type as specified above.

\begin{enumerate}
  
\item Standard conversions for selectors or updates applied to record
  constructor terms are made part of the default Simplifier context; thus
  proofs by reduction of basic operations merely require the $simp$ method
  without further arguments.  These rules are available as $t{\dtt}simps$,
  too.
  
\item Selectors applied to updated records are automatically reduced by an
  internal simplification procedure, which is also part of the standard
  Simplifier setup.

\item Inject equations of a form analogous to $((x, y) = (x', y')) \equiv x=x'
  \conj y=y'$ are declared to the Simplifier and Classical Reasoner as $iff$
  rules.  These rules are available as $t{\dtt}iffs$.

\item The introduction rule for record equality analogous to $x~r = x~r' \Imp
  y~r = y~r' \Imp \dots \Imp r = r'$ is declared to the Simplifier, and as the
  basic rule context as ``$intro?$''.  The rule is called $t{\dtt}equality$.

\item Representations of arbitrary record expressions as canonical constructor
  terms are provided both in $cases$ and $induct$ format (cf.\ the generic
  proof methods of the same name, \S\ref{sec:cases-induct}).  Several
  variations are available, for fixed records, record schemes, more parts etc.
  
  The generic proof methods are sufficiently smart to pick the most sensible
  rule according to the type of the indicated record expression: users just
  need to apply something like ``$(cases~r)$'' to a certain proof problem.

\item The derived record operations $t{\dtt}make$, $t{\dtt}fields$,
  $t{\dtt}extend$, $t{\dtt}truncate$ are \emph{not} treated automatically, but
  usually need to be expanded by hand, using the collective fact
  $t{\dtt}defs$.

\end{enumerate}


\subsection{Datatypes}\label{sec:hol-datatype}

\indexisarcmdof{HOL}{datatype}\indexisarcmdof{HOL}{rep-datatype}
\begin{matharray}{rcl}
  \isarcmd{datatype} & : & \isartrans{theory}{theory} \\
  \isarcmd{rep_datatype} & : & \isartrans{theory}{theory} \\
\end{matharray}

\railalias{repdatatype}{rep\_datatype}
\railterm{repdatatype}

\begin{rail}
  'datatype' (dtspec + 'and')
  ;
  repdatatype (name *) dtrules
  ;

  dtspec: parname? typespec infix? '=' (cons + '|')
  ;
  cons: name (type *) mixfix?
  ;
  dtrules: 'distinct' thmrefs 'inject' thmrefs 'induction' thmrefs
\end{rail}

\begin{descr}
\item [$\isarkeyword{datatype}$] defines inductive datatypes in HOL.
\item [$\isarkeyword{rep_datatype}$] represents existing types as inductive
  ones, generating the standard infrastructure of derived concepts (primitive
  recursion etc.).
\end{descr}

The induction and exhaustion theorems generated provide case names according
to the constructors involved, while parameters are named after the types (see
also \S\ref{sec:cases-induct}).

See \cite{isabelle-HOL} for more details on datatypes, but beware of the
old-style theory syntax being used there!  Apart from proper proof methods for
case-analysis and induction, there are also emulations of ML tactics
\texttt{case_tac} and \texttt{induct_tac} available, see
\S\ref{sec:hol-induct-tac} or \S\ref{sec:zf-induct-tac}; these admit to refer
directly to the internal structure of subgoals (including internally bound
parameters).


\subsection{Recursive functions}\label{sec:recursion}

\indexisarcmdof{HOL}{primrec}\indexisarcmdof{HOL}{recdef}\indexisarcmdof{HOL}{recdef-tc}
\begin{matharray}{rcl}
  \isarcmd{primrec} & : & \isartrans{theory}{theory} \\
  \isarcmd{recdef} & : & \isartrans{theory}{theory} \\
  \isarcmd{recdef_tc}^* & : & \isartrans{theory}{proof(prove)} \\
\end{matharray}

\railalias{recdefsimp}{recdef\_simp}
\railterm{recdefsimp}

\railalias{recdefcong}{recdef\_cong}
\railterm{recdefcong}

\railalias{recdefwf}{recdef\_wf}
\railterm{recdefwf}

\railalias{recdeftc}{recdef\_tc}
\railterm{recdeftc}

\begin{rail}
  'primrec' parname? (equation +)
  ;
  'recdef' ('(' 'permissive' ')')? \\ name term (prop +) hints?
  ;
  recdeftc thmdecl? tc
  ;

  equation: thmdecl? prop
  ;
  hints: '(' 'hints' (recdefmod *) ')'
  ;
  recdefmod: ((recdefsimp | recdefcong | recdefwf) (() | 'add' | 'del') ':' thmrefs) | clasimpmod
  ;
  tc: nameref ('(' nat ')')?
  ;
\end{rail}

\begin{descr}
  
\item [$\isarkeyword{primrec}$] defines primitive recursive functions over
  datatypes, see also \cite{isabelle-HOL}.
  
\item [$\isarkeyword{recdef}$] defines general well-founded recursive
  functions (using the TFL package), see also \cite{isabelle-HOL}.  The
  ``$(permissive)$'' option tells TFL to recover from failed proof attempts,
  returning unfinished results.  The $recdef_simp$, $recdef_cong$, and
  $recdef_wf$ hints refer to auxiliary rules to be used in the internal
  automated proof process of TFL.  Additional $clasimpmod$ declarations (cf.\ 
  \S\ref{sec:clasimp}) may be given to tune the context of the Simplifier
  (cf.\ \S\ref{sec:simplifier}) and Classical reasoner (cf.\ 
  \S\ref{sec:classical}).
  
\item [$\isarkeyword{recdef_tc}~c~(i)$] recommences the proof for leftover
  termination condition number $i$ (default $1$) as generated by a
  $\isarkeyword{recdef}$ definition of constant $c$.
  
  Note that in most cases, $\isarkeyword{recdef}$ is able to finish its
  internal proofs without manual intervention.

\end{descr}

Both kinds of recursive definitions accommodate reasoning by induction (cf.\
\S\ref{sec:cases-induct}): rule $c\mathord{.}induct$ (where $c$ is the name of
the function definition) refers to a specific induction rule, with parameters
named according to the user-specified equations.  Case names of
$\isarkeyword{primrec}$ are that of the datatypes involved, while those of
$\isarkeyword{recdef}$ are numbered (starting from $1$).

The equations provided by these packages may be referred later as theorem list
$f{\dtt}simps$, where $f$ is the (collective) name of the functions defined.
Individual equations may be named explicitly as well; note that for
$\isarkeyword{recdef}$ each specification given by the user may result in
several theorems.

\medskip Hints for $\isarkeyword{recdef}$ may be also declared globally, using
the following attributes.

\indexisarattof{HOL}{recdef-simp}\indexisarattof{HOL}{recdef-cong}\indexisarattof{HOL}{recdef-wf}
\begin{matharray}{rcl}
  recdef_simp & : & \isaratt \\
  recdef_cong & : & \isaratt \\
  recdef_wf & : & \isaratt \\
\end{matharray}

\railalias{recdefsimp}{recdef\_simp}
\railterm{recdefsimp}

\railalias{recdefcong}{recdef\_cong}
\railterm{recdefcong}

\railalias{recdefwf}{recdef\_wf}
\railterm{recdefwf}

\begin{rail}
  (recdefsimp | recdefcong | recdefwf) (() | 'add' | 'del')
  ;
\end{rail}


\subsection{(Co)Inductive sets}\label{sec:hol-inductive}

\indexisarcmdof{HOL}{inductive}\indexisarcmdof{HOL}{coinductive}\indexisarattof{HOL}{mono}
\begin{matharray}{rcl}
  \isarcmd{inductive} & : & \isartrans{theory}{theory} \\
  \isarcmd{coinductive} & : & \isartrans{theory}{theory} \\
  mono & : & \isaratt \\
\end{matharray}

\begin{rail}
  ('inductive' | 'coinductive') sets intros monos?
  ;
  'mono' (() | 'add' | 'del')
  ;

  sets: (term +)
  ;
  intros: 'intros' (thmdecl? prop +)
  ;
  monos: 'monos' thmrefs
  ;
\end{rail}

\begin{descr}
\item [$\isarkeyword{inductive}$ and $\isarkeyword{coinductive}$] define
  (co)inductive sets from the given introduction rules.
\item [$mono$] declares monotonicity rules.  These rule are involved in the
  automated monotonicity proof of $\isarkeyword{inductive}$.
\end{descr}

See \cite{isabelle-HOL} for further information on inductive definitions in
HOL, but note that this covers the old-style theory format.


\subsection{Arithmetic proof support}

\indexisarmethof{HOL}{arith}\indexisarattof{HOL}{arith-split}
\begin{matharray}{rcl}
  arith & : & \isarmeth \\
  arith_split & : & \isaratt \\
\end{matharray}

\begin{rail}
  'arith' '!'?
  ;
\end{rail}

The $arith$ method decides linear arithmetic problems (on types $nat$, $int$,
$real$).  Any current facts are inserted into the goal before running the
procedure.  The ``!''~argument causes the full context of assumptions to be
included.  The $arith_split$ attribute declares case split rules to be
expanded before the arithmetic procedure is invoked.

Note that a simpler (but faster) version of arithmetic reasoning is already
performed by the Simplifier.


\subsection{Cases and induction: emulating tactic scripts}\label{sec:hol-induct-tac}

The following important tactical tools of Isabelle/HOL have been ported to
Isar.  These should be never used in proper proof texts!

\indexisarmethof{HOL}{case-tac}\indexisarmethof{HOL}{induct-tac}
\indexisarmethof{HOL}{ind-cases}\indexisarcmdof{HOL}{inductive-cases}
\begin{matharray}{rcl}
  case_tac^* & : & \isarmeth \\
  induct_tac^* & : & \isarmeth \\
  ind_cases^* & : & \isarmeth \\
  \isarcmd{inductive_cases} & : & \isartrans{theory}{theory} \\
\end{matharray}

\railalias{casetac}{case\_tac}
\railterm{casetac}

\railalias{inducttac}{induct\_tac}
\railterm{inducttac}

\railalias{indcases}{ind\_cases}
\railterm{indcases}

\railalias{inductivecases}{inductive\_cases}
\railterm{inductivecases}

\begin{rail}
  casetac goalspec? term rule?
  ;
  inducttac goalspec? (insts * 'and') rule?
  ;
  indcases (prop +)
  ;
  inductivecases (thmdecl? (prop +) + 'and')
  ;

  rule: ('rule' ':' thmref)
  ;
\end{rail}

\begin{descr}
\item [$case_tac$ and $induct_tac$] admit to reason about inductive datatypes
  only (unless an alternative rule is given explicitly).  Furthermore,
  $case_tac$ does a classical case split on booleans; $induct_tac$ allows only
  variables to be given as instantiation.  These tactic emulations feature
  both goal addressing and dynamic instantiation.  Note that named rule cases
  are \emph{not} provided as would be by the proper $induct$ and $cases$ proof
  methods (see \S\ref{sec:cases-induct}).
  
\item [$ind_cases$ and $\isarkeyword{inductive_cases}$] provide an interface
  to the internal \texttt{mk_cases} operation.  Rules are simplified in an
  unrestricted forward manner.

  While $ind_cases$ is a proof method to apply the result immediately as
  elimination rules, $\isarkeyword{inductive_cases}$ provides case split
  theorems at the theory level for later use,
\end{descr}


\subsection{Executable code}

Isabelle/Pure provides a generic infrastructure to support code generation
from executable specifications, both functional and relational programs.
Isabelle/HOL instantiates these mechanisms in a way that is amenable to
end-user applications.  See \cite{isabelle-HOL} for further information (this
actually covers the new-style theory format as well).

\indexisarcmd{generate-code}\indexisarcmd{consts-code}\indexisarcmd{types-code}
\indexisaratt{code}

\begin{matharray}{rcl}
  \isarcmd{generate_code} & : & \isartrans{theory}{theory} \\
  \isarcmd{consts_code} & : & \isartrans{theory}{theory} \\
  \isarcmd{types_code} & : & \isartrans{theory}{theory} \\  
  code & : & \isaratt \\
\end{matharray}

\railalias{generatecode}{generate\_code}
\railterm{generatecode}

\railalias{constscode}{consts\_code}
\railterm{constscode}

\railalias{typescode}{types\_code}
\railterm{typescode}

\begin{rail}
  generatecode ( () | '(' name ')') ((name '=' term) +)
  ;
  constscode (name ('::' type)? template +)
  ;
  typescode (name template +)
  ;
  template: '(' string ')'
  ;

  'code' (name)?
  ;
\end{rail}


\section{HOLCF}

\subsection{Mixfix syntax for continuous operations}

\indexisarcmdof{HOLCF}{consts}\indexisarcmdof{HOLCF}{constdefs}

\begin{matharray}{rcl}
  \isarcmd{consts} & : & \isartrans{theory}{theory} \\
  \isarcmd{constdefs} & : & \isartrans{theory}{theory} \\
\end{matharray}

HOLCF provides a separate type for continuous functions $\alpha \rightarrow
\beta$, with an explicit application operator $f \cdot x$.  Isabelle mixfix
syntax normally refers directly to the pure meta-level function type $\alpha
\To \beta$, with application $f\,x$.

The HOLCF variants of $\CONSTS$ and $\CONSTDEFS$ have the same outer syntax as
the pure versions (cf.\ \S\ref{sec:consts}).  Internally, declarations
involving continuous function types are treated specifically, transforming the
syntax template accordingly and generating syntax translation rules for the
abstract and concrete representation of application.

The behavior for plain meta-level function types is unchanged.  Mixed
continuous and meta-level application is \emph{not} supported.

\subsection{Recursive domains}

\indexisarcmdof{HOLCF}{domain}
\begin{matharray}{rcl}
  \isarcmd{domain} & : & \isartrans{theory}{theory} \\
\end{matharray}

\begin{rail}
  'domain' parname? (dmspec + 'and')
  ;

  dmspec: typespec '=' (cons + '|')
  ;
  cons: name (type *) mixfix?
  ;
  dtrules: 'distinct' thmrefs 'inject' thmrefs 'induction' thmrefs
\end{rail}

Recursive domains in HOLCF are analogous to datatypes in classical HOL (cf.\ 
\S\ref{sec:hol-datatype}).  Mutual recursion is supported, but no nesting nor
arbitrary branching.  Domain constructors may be strict (default) or lazy, the
latter admits to introduce infinitary objects in the typical LCF manner (e.g.\ 
lazy lists).  See also \cite{MuellerNvOS99} for a general discussion of HOLCF
domains.


\section{ZF}

\subsection{Type checking}

The ZF logic is essentially untyped, so the concept of ``type checking'' is
performed as logical reasoning about set-membership statements.  A special
method assists users in this task; a version of this is already declared as a
``solver'' in the standard Simplifier setup.

\indexisarcmd{print-tcset}\indexisaratt{typecheck}\indexisaratt{TC}

\begin{matharray}{rcl}
  \isarcmd{print_tcset}^* & : & \isarkeep{theory~|~proof} \\
  typecheck & : & \isarmeth \\
  TC & : & \isaratt \\
\end{matharray}

\begin{rail}
  'TC' (() | 'add' | 'del')
  ;
\end{rail}

\begin{descr}
  
\item [$\isarcmd{print_tcset}$] prints the collection of typechecking rules of
  the current context.
  
  Note that the component built into the Simplifier only knows about those
  rules being declared globally in the theory!
  
\item [$typecheck$] attempts to solve any pending type-checking problems in
  subgoals.
  
\item [$TC$] adds or deletes type-checking rules from the context.

\end{descr}


\subsection{(Co)Inductive sets and datatypes}

\subsubsection{Set definitions}

In ZF everything is a set.  The generic inductive package also provides a
specific view for ``datatype'' specifications.  Coinductive definitions are
available in both cases, too.

\indexisarcmdof{ZF}{inductive}\indexisarcmdof{ZF}{coinductive}
\indexisarcmdof{ZF}{datatype}\indexisarcmdof{ZF}{codatatype}
\begin{matharray}{rcl}
  \isarcmd{inductive} & : & \isartrans{theory}{theory} \\
  \isarcmd{coinductive} & : & \isartrans{theory}{theory} \\
  \isarcmd{datatype} & : & \isartrans{theory}{theory} \\
  \isarcmd{codatatype} & : & \isartrans{theory}{theory} \\
\end{matharray}

\railalias{CONDEFS}{con\_defs}
\railterm{CONDEFS}

\railalias{TYPEINTROS}{type\_intros}
\railterm{TYPEINTROS}

\railalias{TYPEELIMS}{type\_elims}
\railterm{TYPEELIMS}

\begin{rail}
  ('inductive' | 'coinductive') domains intros hints
  ;

  domains: 'domains' (term + '+') ('<=' | subseteq) term
  ;
  intros: 'intros' (thmdecl? prop +)
  ;
  hints: monos? condefs? typeintros? typeelims?
  ;
  monos: ('monos' thmrefs)?
  ;
  condefs: (CONDEFS thmrefs)?
  ;
  typeintros: (TYPEINTROS thmrefs)?
  ;
  typeelims: (TYPEELIMS thmrefs)?
  ;
\end{rail}

In the following diagram $monos$, $typeintros$, and $typeelims$ are the same
as above.

\begin{rail}
  ('datatype' | 'codatatype') domain? (dtspec + 'and') hints
  ;

  domain: ('<=' | subseteq) term
  ;
  dtspec: term '=' (con + '|')
  ;
  con: name ('(' (term ',' +) ')')?  
  ;
  hints: monos? typeintros? typeelims?
  ;
\end{rail}

See \cite{isabelle-ZF} for further information on inductive definitions in
HOL, but note that this covers the old-style theory format.


\subsubsection{Primitive recursive functions}

\indexisarcmdof{ZF}{primrec}
\begin{matharray}{rcl}
  \isarcmd{primrec} & : & \isartrans{theory}{theory} \\
\end{matharray}

\begin{rail}
  'primrec' (thmdecl? prop +)
  ;
\end{rail}


\subsubsection{Cases and induction: emulating tactic scripts}\label{sec:zf-induct-tac}

The following important tactical tools of Isabelle/ZF have been ported to
Isar.  These should be never used in proper proof texts!

\indexisarmethof{ZF}{case-tac}\indexisarmethof{ZF}{induct-tac}
\indexisarmethof{ZF}{ind-cases}\indexisarcmdof{ZF}{inductive-cases}
\begin{matharray}{rcl}
  case_tac^* & : & \isarmeth \\
  induct_tac^* & : & \isarmeth \\
  ind_cases^* & : & \isarmeth \\
  \isarcmd{inductive_cases} & : & \isartrans{theory}{theory} \\
\end{matharray}

\begin{rail}
  (casetac | inducttac) goalspec? name
  ;
  indcases (prop +)
  ;
  inductivecases (thmdecl? (prop +) + 'and')
  ;
\end{rail}


%%% Local Variables:
%%% mode: latex
%%% TeX-master: "isar-ref"
%%% End:
