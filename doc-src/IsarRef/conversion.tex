
\chapter{Isabelle/Isar Conversion Guide}\label{ap:conv}

Subsequently, we give a few practical hints on working in a mixed environment
of old Isabelle ML proof scripts and new Isabelle/Isar theories.  There are
basically three ways to cope with this issue.
\begin{enumerate}
\item Do not convert old sources at all, but communicate directly at the level
  of \emph{internal} theory and theorem values.
\item Port old-style theory files to new-style ones (very easy), and ML proof
  scripts to Isar tactic-emulation scripts (quite easy).
\item Actually redo ML proof scripts as human-readable Isar proof texts
  (probably hard, depending who wrote the original scripts).
\end{enumerate}


\section{No conversion}

Internally, Isabelle is able to handle both old and new-style theories at the
same time; the theory loader automatically detects the input format.  In any
case, the results are certain internal ML values of type \texttt{theory} and
\texttt{thm}.  These may be accessed from either classic Isabelle or
Isabelle/Isar, provided that some minimal precautions are observed.


\subsection{Referring to theorem and theory values}

\begin{ttbox}
thm         : xstring -> thm
thms        : xstring -> thm list
the_context : unit -> theory
theory      : string -> theory
\end{ttbox}

These functions provide general means to refer to logical objects from ML.
Old-style theories used to emit many ML bindings of theorems and theories, but
this is no longer done in new-style Isabelle/Isar theories.

\begin{descr}
\item [$\mathtt{thm}~name$ and $\mathtt{thms}~name$] retrieve theorems stored
  in the current theory context, including any ancestor node.
  
  The convention of old-style theories was to bind any theorem as an ML value
  as well.  New-style theories no longer do this, so ML code may require
  \texttt{thm~"foo"} rather than just \texttt{foo}.
  
\item [$\mathtt{the_context()}$] refers to the current theory context.
  
  Old-style theories often use the ML binding \texttt{thy}, which is
  dynamically created by the ML code generated from old theory source.  This
  is no longer the recommended way in any case!  Function \texttt{the_context}
  should be used for old scripts as well.
  
\item [$\mathtt{theory}~name$] retrieves the named theory from the global
  theory-loader database.
  
  The ML code generated from old-style theories would include an ML binding
  $name\mathtt{.thy}$ as part of an ML structure.
\end{descr}


\subsection{Storing theorem values}

\begin{ttbox}
qed        : string -> unit
bind_thm   : string * thm -> unit
bind_thms  : string * thm list -> unit
\end{ttbox}

ML proof scripts have to be well-behaved by storing theorems properly within
the current theory context, in order to enable new-style theories to retrieve
these these later.

\begin{descr}
\item [$\mathtt{qed}~name$] is the canonical way to conclude a proof script in
  ML.  This already manages entry in the theorem database of the current
  theory context.
\item [$\mathtt{bind_thm}~(name, thm)$ and $\mathtt{bind_thms}~(name, thms)$]
  store theorems that have been produced in ML in an ad-hoc manner.
\end{descr}

Note that the original ``LCF-system'' approach of binding theorem values on
the ML toplevel only has long been given up in Isabelle!  Despite of this, old
legacy proof scripts occasionally contain code such as \texttt{val foo =
  result();} which is ill-behaved in several respects.  Apart from preventing
access from Isar theories, it also omits the result from the WWW presentation,
for example.


\subsection{ML declarations in Isar}

\begin{matharray}{rcl}
  \isarcmd{ML} & : & \isartrans{\cdot}{\cdot} \\
  \isarcmd{ML_setup} & : & \isartrans{theory}{theory} \\
\end{matharray}

Isabelle/Isar theories may contain ML declarations as well.  For example, an
old-style theorem binding may be mimicked as follows.
\[
\isarkeyword{ML}~\{*~\mbox{\texttt{val foo = thm "foo"}}~*\}
\]
Note that this command cannot be undone, so invalid theorem bindings in ML may
persist.  Also note that the current theory may not be modified; use
$\isarkeyword{ML_setup}$ for declarations that act on the current context.


\section{Porting theories and proof scripts}\label{sec:port-scripts}

Porting legacy theory and ML files to proper Isabelle/Isar theories has
several advantages.  For example, the Proof~General user interface
\cite{proofgeneral} for Isabelle/Isar is more robust and more comfortable to
use than the version for classic Isabelle.  This is due to the fact that the
generic ML toplevel has been replaced by a separate Isar interaction loop,
with full control over input synchronization and error conditions.

Furthermore, the Isabelle document preparation system (see also
\cite{isabelle-sys}) only works properly with new-style theories.  Output of
old-style sources is at the level of individual characters (and symbols),
without proper document markup as in Isabelle/Isar theories.


\subsection{Theories}

Basically, the Isabelle/Isar theory syntax is a proper superset of the classic
one.  Only a few quirks and legacy problems have been eliminated, resulting in
simpler rules and less special cases.  The main changes of theory syntax are
as follows.

\begin{itemize}
\item Quoted strings may contain arbitrary white space, and span several lines
  without requiring \verb,\,\,\dots\verb,\, escapes.
\item Names may always be quoted.
  
  The old syntax would occasionally demand plain identifiers vs.\ quoted
  strings to accommodate certain syntactic features.
\item Types and terms have to be \emph{atomic} as far as the theory syntax is
  concerned; this typically requires quoting of input strings, e.g.\ ``$x +
  y$''.
  
  The old theory syntax used to fake part of the syntax of types in order to
  require less quoting in common cases; this was hard to predict, though.  On
  the other hand, Isar does not require quotes for simple terms, such as plain
  identifiers $x$, numerals $1$, or symbols $\forall$ (input as
  \verb,\<forall>,).
\item Theorem declarations require an explicit colon to separate the name from
  the statement (the name is usually optional).  Cf.\ the syntax of $\DEFS$ in
  \S\ref{sec:consts}, or $\THEOREMNAME$ in \S\ref{sec:axms-thms}.
\end{itemize}

Note that Isabelle/Isar error messages are usually quite explicit about the
problem at hand.  So in cases of doubt, input syntax may be just as well tried
out interactively.


\subsection{Goal statements}\label{sec:conv-goal}

\subsubsection{Simple goals}

In ML the canonical a goal statement together with a complete proof script is
as follows:
\begin{ttbox}
 Goal "\(\phi\)";
 by \(tac@1\);
   \(\vdots\)
 qed "\(name\)";
\end{ttbox}
This form may be turned into an Isar tactic-emulation script like this:
\begin{matharray}{l}
  \LEMMA{name}{\texttt"{\phi}\texttt"} \\
  \quad \APPLY{meth@1} \\
  \qquad \vdots \\
  \quad \DONE \\
\end{matharray}
Note that the main statement may be $\THEOREMNAME$ as well.  See
\S\ref{sec:conv-tac} for further details on how to convert actual tactic
expressions into proof methods.

\medskip Classic Isabelle provides many variant forms of goal commands, see
also \cite{isabelle-ref} for further details.  The second most common one is
\texttt{Goalw}, which expands definitions before commencing the actual proof
script.
\begin{ttbox}
 Goalw [\(def@1\), \(\dots\)] "\(\phi\)";
\end{ttbox}
This may be replaced by using the $unfold$ proof method explicitly.
\begin{matharray}{l}
\LEMMA{name}{\texttt"{\phi}\texttt"} \\
\quad \APPLY{unfold~def@1~\dots} \\
\end{matharray}


\subsubsection{Deriving rules}

Deriving non-atomic meta-level propositions requires special precautions in
classic Isabelle: the primitive \texttt{goal} command decomposes a statement
into the atomic conclusion and a list of assumptions, which are exhibited as
ML values of type \texttt{thm}.  After the proof is finished, these premises
are discharged again, resulting in the original rule statement.

In contrast, Isabelle/Isar does \emph{not} require any special treatment of
non-atomic statements --- assumptions and goals may be arbitrarily complex.
While this would basically require to proceed by structured proof, decomposing
the main problem into sub-problems according to the basic Isar scheme of
backward reasoning, the old tactic-style procedure may be mimicked as follows.
The general ML goal statement for derived rules looks like this:
\begin{ttbox}
 val [\(prem@1\), \dots] = goal "\(\phi@1 \Imp \dots \Imp \psi\)";
 by \(tac@1\);
   \(\vdots\)
\end{ttbox}
This form may be turned into a tactic-emulation script that is wrapped in an
Isar text to get access to the premises as local facts.
\begin{matharray}{l}
  \LEMMA{}{\texttt"{\phi@1 \Imp \dots \Imp \psi}\texttt"} \\
  \PROOF{}~- \\
  \quad \ASSUME{prem@1}{\phi@1}~\AND~\dots \\
  \quad \SHOW{}{\Var{thesis}} \\
  \qquad \APPLY{meth@1} \\
  \qquad\quad \vdots \\
  \qquad \DONE \\
  \QED{} \\
\end{matharray}
Note that the above scheme repeats the text of premises $\phi@1$, \dots, while
the conclusion $\psi$ is referenced via the automatic text abbreviation
$\Var{thesis}$.  The assumption context may be invoked in a less verbose
manner as well, using ``$\CASE{antecedent}$'' instead of
``$\ASSUME{prem@1}{\phi@1}~\AND~\dots$'' above.  Then the list of \emph{all}
premises is bound to the name $antecedent$; Isar does not provide a way to
destruct lists into single items, though.

\medskip In practice, actual rules are often rather direct consequences of
corresponding atomic statements, typically stemming from the definition of a
new concept.  In that case, the general scheme for deriving rules may be
greatly simplified, using one of the standard automated proof tools, such ad
$simp$, $blast$, or $auto$.  This would work as follows:
\begin{matharray}{l}
  \LEMMA{}{\texttt"{\phi@1 \Imp \dots \Imp \psi}\texttt"} \\
  \quad \APPLY{unfold~defs} \quad \CMT{or ``$\APPLY{insert~facts}$''} \\
  \quad \APPLY{blast} \\
  \quad \DONE \\
\end{matharray}
Note that classic Isabelle would support this form only in the special case
where $\phi@1$, \dots are atomic statements (when using the standard
\texttt{Goal} command).  Otherwise the special treatment of rules would be
applied, disturbing this simple setup.

\medskip Occasionally, derived rules would be established by first proving an
appropriate atomic statement (using $\forall$ and $\longrightarrow$ of the
object-logic), and putting the final result into ``rule format''.  In classic
Isabelle this would usually proceed as follows:
\begin{ttbox}
 Goal "\(\phi\)";
 by \(tac@1\);
   \(\vdots\)
 qed_spec_mp "\(name\)";
\end{ttbox}
The operation performed by \texttt{qed_spec_mp} is also performed by the Isar
attribute ``$rule_format$'', see also \S\ref{sec:rule-format}.  Thus the
corresponding Isar text may look like this:
\begin{matharray}{l}
  \LEMMA{name~[rule_format]}{\texttt"{\phi}\texttt"} \\
  \quad \APPLY{meth@1} \\
  \qquad \vdots \\
  \quad \DONE \\
\end{matharray}
Note plain ``$rule_format$'' actually performs a slightly different operation:
it fully replaces object-level implication and universal quantification
throughout the whole result statement.  This is the right thing in most cases.
For historical reasons, \texttt{qed_spec_mp} would only operate on the
conclusion; one may get this exact behavior by using
``$rule_format~(no_asm)$'' instead.

\subsection{Tactics}\label{sec:conv-tac}

Isar Proof methods closely resemble traditional tactics, when used in
unstructured sequences of $\APPLYNAME$ commands (cf.\ \S\ref{sec:conv-goal}).
Isabelle/Isar provides emulations for all major ML tactics of classic Isabelle
--- mostly for the sake of easy porting of existing developments, as actual
Isar proof texts would demand much less diversity of proof methods.

Unlike tactic expressions in ML, Isar proof methods provide proper concrete
syntax for additional arguments, options, modifiers etc.  Thus a typical
method text is usually more concise than the corresponding ML tactic.
Furthermore, the Isar versions of classic Isabelle tactics often cover several
variant forms by a single method with separate options to tune the behavior.
For example, method $simp$ replaces all of \texttt{simp_tac}~/
\texttt{asm_simp_tac}~/ \texttt{full_simp_tac}~/ \texttt{asm_full_simp_tac},
there is also concrete syntax for augmenting the Simplifier context (the
current ``simpset'') in a handsome way.


\subsubsection{Resolution tactics}

Classic Isabelle provides several variant forms of tactics for single-step
rule applications (based on higher-order resolution).  The space of resolution
tactics has the following main dimensions.
\begin{enumerate}
\item The ``mode'' of resolution: intro, elim, destruct, or forward (e.g.\ 
  \texttt{resolve_tac}, \texttt{eresolve_tac}, \texttt{dresolve_tac},
  \texttt{forward_tac}).
\item Optional explicit instantiation (e.g.\ \texttt{resolve_tac} vs.\ 
  \texttt{res_inst_tac}).
\item Abbreviations for singleton arguments (e.g.\ \texttt{resolve_tac} vs.\ 
  \texttt{rtac}).
\end{enumerate}

Basically, the set of Isar tactic emulations $rule_tac$, $erule_tac$,
$drule_tac$, $frule_tac$ (see \S\ref{sec:tactics}) would be sufficient to
cover the four modes, either with or without instantiation, and either with
single or multiple arguments.  Although it is more convenient in most cases to
use the plain $rule$ method (see \S\ref{sec:pure-meth-att}), or any of its
``improper'' variants $erule$, $drule$, $frule$ (see
\S\ref{sec:misc-methods}).  Note that explicit goal addressing is only
supported by the actual $rule_tac$ version.

With this in mind, plain resolution tactics may be ported as follows.
\begin{matharray}{lll}
  \texttt{rtac}~a~1 & & rule~a \\
  \texttt{resolve_tac}~[a@1, \dots]~1 & & rule~a@1~\dots \\
  \texttt{res_inst_tac}~[(x@1, t@1), \dots]~a~1 & &
  rule_tac~x@1 = t@1~and~\dots~in~a \\[0.5ex]
  \texttt{rtac}~a~i & & rule_tac~[i]~a \\
  \texttt{resolve_tac}~[a@1, \dots]~i & & rule_tac~[i]~a@1~\dots \\
  \texttt{res_inst_tac}~[(x@1, t@1), \dots]~a~i & &
  rule_tac~[i]~x@1 = t@1~and~\dots~in~a \\
\end{matharray}

Note that explicit goal addressing may be usually avoided by changing the
order of subgoals with $\isarkeyword{defer}$ or $\isarkeyword{prefer}$ (see
\S\ref{sec:tactic-commands}).

\medskip Some further (less frequently used) combinations of basic resolution
tactics may be expressed as follows.
\begin{matharray}{lll}
  \texttt{ares_tac}~[a@1, \dots]~1 & & assumption~|~rule~a@1~\dots \\
  \texttt{eatac}~a~n~1 & & erule~(n)~a \\
  \texttt{datac}~a~n~1 & & drule~(n)~a \\
  \texttt{fatac}~a~n~1 & & frule~(n)~a \\
\end{matharray}


\subsubsection{Simplifier tactics}

The main Simplifier tactics \texttt{Simp_tac}, \texttt{simp_tac} and variants
(cf.\ \cite{isabelle-ref}) are all covered by the $simp$ and $simp_all$
methods (see \S\ref{sec:simp}).  Note that there is no individual goal
addressing available, simplification acts either on the first goal ($simp$)
or all goals ($simp_all$).

\begin{matharray}{lll}
  \texttt{Asm_full_simp_tac 1} & & simp \\
  \texttt{ALLGOALS Asm_full_simp_tac} & & simp_all \\[0.5ex]
  \texttt{Simp_tac 1} & & simp~(no_asm) \\
  \texttt{Asm_simp_tac 1} & & simp~(no_asm_simp) \\
  \texttt{Full_simp_tac 1} & & simp~(no_asm_use) \\
\end{matharray}

Isar also provides separate method modifier syntax for augmenting the
Simplifier context (see \S\ref{sec:simp}), which is known as the ``simpset''
in ML.  A typical ML expression with simpset changes looks like this:
\begin{ttbox}
asm_full_simp_tac (simpset () addsimps [\(a@1\), \(\dots\)] delsimps [\(b@1\), \(\dots\)]) 1
\end{ttbox}
The corresponding Isar text is as follows:
\[
simp~add:~a@1~\dots~del:~b@1~\dots
\]
Global declarations of Simplifier rules (e.g.\ \texttt{Addsimps}) are covered
by application of attributes, see \S\ref{sec:conv-decls} for more information.


\subsubsection{Classical Reasoner tactics}

The Classical Reasoner provides a rather large number of variations of
automated tactics, such as \texttt{Blast_tac}, \texttt{Fast_tac},
\texttt{Clarify_tac} etc.\ (see \cite{isabelle-ref}).  The corresponding Isar
methods usually share the same base name, such as $blast$, $fast$, $clarify$
etc.\ (see \S\ref{sec:classical-auto}).

Similar to the Simplifier, there is separate method modifier syntax for
augmenting the Classical Reasoner context, which is known as the ``claset'' in
ML.  A typical ML expression with claset changes looks like this:
\begin{ttbox}
blast_tac (claset () addIs [\(a@1\), \(\dots\)] addSEs [\(b@1\), \(\dots\)]) 1
\end{ttbox}
The corresponding Isar text is as follows:
\[
blast~intro:~a@1~\dots~elim!:~b@1~\dots
\]
Global declarations of Classical Reasoner rules (e.g.\ \texttt{AddIs}) are
covered by application of attributes, see \S\ref{sec:conv-decls} for more
information.


\subsubsection{Miscellaneous tactics}

There are a few additional tactics defined in various theories of
Isabelle/HOL, some of these also in Isabelle/FOL or Isabelle/ZF.  The most
common ones of these may be ported to Isar as follows.

\begin{matharray}{lll}
  \texttt{stac}~a~1 & & subst~a \\
  \texttt{hyp_subst_tac}~1 & & hypsubst \\
  \texttt{strip_tac}~1 & \approx & intro~strip \\
  \texttt{split_all_tac}~1 & & simp~(no_asm_simp)~only:~split_tupled_all \\
                         & \approx & simp~only:~split_tupled_all \\
                         & \ll & clarify \\
\end{matharray}


\subsubsection{Tacticals}

Classic Isabelle provides a huge amount of tacticals for combination and
modification of existing tactics.  This has been greatly reduced in Isar,
providing the bare minimum of combinators only: ``$,$'' (sequential
composition), ``$|$'' (alternative choices), ``$?$'' (try), ``$+$'' (repeat at
least once).  These are usually sufficient in practice; if all fails,
arbitrary ML tactic code may be invoked via the $tactic$ method (see
\S\ref{sec:tactics}).

\medskip Common ML tacticals may be expressed directly in Isar as follows:
\begin{matharray}{lll}
tac@1~\texttt{THEN}~tac@2 & & meth@1, meth@2 \\
tac@1~\texttt{ORELSE}~tac@2 & & meth@1~|~meth@2 \\
\texttt{TRY}~tac & & meth? \\
\texttt{REPEAT1}~tac & & meth+ \\
\texttt{REPEAT}~tac & & (meth+)? \\
\texttt{EVERY}~[tac@1, \dots] & & meth@1, \dots \\
\texttt{FIRST}~[tac@1, \dots] & & meth@1~|~\dots \\
\end{matharray}

\medskip \texttt{CHANGED} (see \cite{isabelle-ref}) is usually not required in
Isar, since most basic proof methods already fail unless there is an actual
change in the goal state.  Nevertheless, ``$?$'' (try) may be used to accept
\emph{unchanged} results as well.

\medskip \texttt{ALLGOALS}, \texttt{SOMEGOAL} etc.\ (see \cite{isabelle-ref})
are not available in Isar, since there is no direct goal addressing.
Nevertheless, some basic methods address all goals internally, notably
$simp_all$ (see \S\ref{sec:simp}).  Also note that \texttt{ALLGOALS} may be
often replaced by ``$+$'' (repeat at least once), although this usually has a
different operational behavior, such as solving goals in a different order.

\medskip Iterated resolution, such as
\texttt{REPEAT~(FIRSTGOAL~(resolve_tac~$\dots$))}, is usually better expressed
using the $intro$ and $elim$ methods of Isar (see
\S\ref{sec:classical-basic}).


\subsection{Declarations and ad-hoc operations}\label{sec:conv-decls}

Apart from proof commands and tactic expressions, almost all of the remaining
ML code occurring in legacy proof scripts are either global context
declarations (such as \texttt{Addsimps}) or ad-hoc operations on theorems
(such as \texttt{RS}).  In Isar both of these are covered by theorem
expressions with \emph{attributes}.

\medskip Theorem operations may be attached as attributes in the very place
where theorems are referenced, say within a method argument.  The subsequent
common ML combinators may be expressed directly in Isar as follows.
\begin{matharray}{lll}
  thm@1~\texttt{RS}~thm@2 & & thm@1~[THEN~thm@2] \\
  thm@1~\texttt{RSN}~(i, thm@2) & & thm@1~[THEN~[i]~thm@2] \\
  thm@1~\texttt{COMP}~thm@2 & & thm@1~[COMP~thm@2] \\
  \relax[thm@1, \dots]~\texttt{MRS}~thm & & thm~[OF~thm@1~\dots] \\
  \texttt{read_instantiate}~[(x@1, t@1), \dots]~thm & & thm~[where~x@1 = t@1~and~\dots] \\
  \texttt{standard}~thm & & thm~[standard] \\
  \texttt{make_elim}~thm & & thm~[elim_format] \\
\end{matharray}

Note that $OF$ is often more readable as $THEN$; likewise positional
instantiation with $of$ is more handsome than $where$.

The special ML command \texttt{qed_spec_mp} of Isabelle/HOL and FOL may be
replaced by passing the result of a proof through $rule_format$.

\medskip Global ML declarations may be expressed using the $\DECLARE$ command
(see \S\ref{sec:tactic-commands}) together with appropriate attributes.  The
most common ones are as follows.
\begin{matharray}{lll}
  \texttt{Addsimps}~[thm] & & \DECLARE~thm~[simp] \\
  \texttt{Delsimps}~[thm] & & \DECLARE~thm~[simp~del] \\
  \texttt{Addsplits}~[thm] & & \DECLARE~thm~[split] \\
  \texttt{Delsplits}~[thm] & & \DECLARE~thm~[split~del] \\[0.5ex]
  \texttt{AddIs}~[thm] & & \DECLARE~thm~[intro] \\
  \texttt{AddEs}~[thm] & & \DECLARE~thm~[elim] \\
  \texttt{AddDs}~[thm] & & \DECLARE~thm~[dest] \\
  \texttt{AddSIs}~[thm] & & \DECLARE~thm~[intro!] \\
  \texttt{AddSEs}~[thm] & & \DECLARE~thm~[elim!] \\
  \texttt{AddSDs}~[thm] & & \DECLARE~thm~[dest!] \\[0.5ex]
  \texttt{AddIffs}~[thm] & & \DECLARE~thm~[iff] \\
\end{matharray}
Note that explicit $\DECLARE$ commands are actually needed only very rarely;
Isar admits to declare theorems on-the-fly wherever they emerge.  Consider the
following ML idiom:
\begin{ttbox}
Goal "\(\phi\)";
 \(\vdots\)
qed "name";
Addsimps [name];
\end{ttbox}
This may be expressed more concisely in Isar like this:
\begin{matharray}{l}
  \LEMMA{name~[simp]}{\phi} \\
  \quad\vdots
\end{matharray}
The $name$ may be even omitted, although this would make it difficult to
declare the theorem otherwise later (e.g.\ as $[simp~del]$).


\section{Converting to actual Isar proof texts}

Porting legacy ML proof scripts into Isar tactic emulation scripts (see
\S\ref{sec:port-scripts}) is mainly a technical issue, since the basic
representation of formal ``proof script'' is preserved.  In contrast,
converting existing Isabelle developments into actual human-readably Isar
proof texts is more involved, due to the fundamental change of the underlying
paradigm.

This issue is comparable to that of converting programs written in a low-level
programming languages (say Assembler) into higher-level ones (say Haskell).
In order to accomplish this, one needs a working knowledge of the target
language, as well an understanding of the \emph{original} idea of the piece of
code expressed in the low-level language.

As far as Isar proofs are concerned, it is usually much easier to re-use only
definitions and the main statements directly, while following the arrangement
of proof scripts only very loosely.  Ideally, one would also have some
\emph{informal} proof outlines available for guidance as well.  In the worst
case, obscure proof scripts would have to be re-engineered by tracing forth
and backwards, and by educated guessing!

\medskip This is a possible schedule to embark on actual conversion of legacy
proof scripts into Isar proof texts.
\begin{enumerate}
\item Port ML scripts to Isar tactic emulation scripts (see
  \S\ref{sec:port-scripts}).
\item Get sufficiently acquainted with Isabelle/Isar proof
  development.\footnote{As there is still no Isar tutorial around, it is best
    to look at existing Isar examples, see also \S\ref{sec:isar-howto}.}
\item Recover the proof structure of a few important theorems.
\item Rephrase the original intention of the course of reasoning in terms of
  Isar proof language elements.
\end{enumerate}

Certainly, rewriting formal reasoning in Isar requires much additional effort.
On the other hand, one gains a human-readable representation of
machine-checked formal proof.  Depending on the context of application, this
might be even indispensable to start with!


%%% Local Variables: 
%%% mode: latex
%%% TeX-master: "isar-ref"
%%% End: 
