
\chapter{Common Isar elements}

FIXME $*$ indicates \emph{improper commands}

\section{Theory commands}

\subsection{Defining theories}

\indexisarcmd{theory}\indexisarcmd{end}\indexisarcmd{context}
\begin{matharray}{rcl}
  \isarcmd{theory} & : & \isartrans{\cdot}{theory} \\
  \isarcmd{context}^* & : & \isartrans{\cdot}{theory} \\
  \isarcmd{end} & : & \isartrans{theory}{\cdot} \\
\end{matharray}

Isabelle/Isar ``new-style'' theories are either defined via theory files or
interactively.\footnote{In contrast, ``old-style'' Isabelle theories support
  batch processing only, with only the ML proof script part suitable for
  interaction.}

The first command of any theory has to be $\THEORY$, starting a new theory
based on the merge of existing ones.  In interactive experiments, the theory
context may be changed by $\CONTEXT$ without creating a new theory.  In both
cases the concluding command is $\END$, which has to be the very last one of
any proper theory file.

\begin{rail}
  'theory' name '=' (name + '+') filespecs? ':'
  ;
  'context' name
  ;
  'end'
  ;;

  filespecs : 'files' ((name | '(' name ')') +);
\end{rail}

\begin{description}
\item [$\THEORY~A = B@1 + \cdots + B@n$] commences a new theory $A$ based on
  existing ones $B@1 + \cdots + B@n$.  Note that Isabelle's theory loader
  system ensures that any of the base theories are properly loaded (and fully
  up-to-date when $\THEORY$ is executed interactively).
  
\item [$\CONTEXT~B$] enters existing theory context $B$, basically in
  read-only mode, so only a limited set of commands may be performed.  Just as
  for $\THEORY$, the theory loader ensures that $B$ is loaded and up-to-date.

\item [$\END$] concludes the current theory definition of context switch.
\end{description}


\subsection{Formal comments}

\indexisarcmd{title}\indexisarcmd{chapter}\indexisarcmd{subsection}\indexisarcmd{subsubsection}
\indexisarcmd{text}\indexisarcmd{txt}
\begin{matharray}{rcl}
  \isarcmd{title} & : & \isartrans{theory}{theory} \\
  \isarcmd{chapter} & : & \isartrans{theory}{theory} \\
  \isarcmd{subsection} & : & \isartrans{theory}{theory} \\
  \isarcmd{subsubsection} & : & \isartrans{theory}{theory} \\
  \isarcmd{text} & : & \isartrans{theory}{theory} \\
  \isarcmd{txt} & : & \isartrans{proof(state)}{proof(state)} \\
\end{matharray}

There are several commands to include \emph{formal comments} in theory and
proof specification.  In contrast to source-level comments
\verb|(*|\dots\verb|*)|, which are stripped at the lexical level, any text
given as formal comment is meant to be part of the actual document.
Consequently, it would be included in the final printed version.

Apart from plain prose, formal comments may also refer to logical entities of
the current theory context (types, terms, theorems etc.).  Proper processing
of the text would then include some further consistency checks with the items
declared in the current theory, e.g.\ type-checking of included terms.
\footnote{The current version of Isabelle/Isar does not process formal
  comments in any such way.  This will be available as part of the automatic
  theory and proof document preparation system (via (PDF)LaTeX) that is
  planned for the near future.}

\begin{rail}
  'title' text text? text?
  ;
  'chapter' text
  ;
  'subsection' text
  ;
  'subsubsection' text
  ;
  'text' text
  ;
  'txt' text
  ;
\end{rail}

\begin{description}
\item [$\isarkeyword{title}~title~author~date$] specifies the document title
  just as in typical LaTeX documents.
\item [$\isarkeyword{chapter}~text$, $\isarkeyword{subsection}~text$,
  $\isarkeyword{subsubsection}~text$] specify chapter and subsection headings.
\item [$\TEXT~text$] specifies an actual body of prose text, including
  references to formal entities.\footnote{The latter feature is not yet
    exploited in any way.}
\item [$\TXT~text$] is similar to $\TEXT$, but may appear within proofs.
\end{description}


\subsection{Type classes and sorts}\label{sec:classes}

\indexisarcmd{classes}\indexisarcmd{classrel}\indexisarcmd{defaultsort}
\begin{matharray}{rcl}
  \isarcmd{classes} & : & \isartrans{theory}{theory} \\
  \isarcmd{classrel} & : & \isartrans{theory}{theory} \\
  \isarcmd{defaultsort} & : & \isartrans{theory}{theory} \\
\end{matharray}

\begin{rail}
  'classes' (classdecl +)
  ;
  'classrel' nameref '<' nameref comment?
  ;
  'defaultsort' sort comment?
  ;
\end{rail}

\begin{description}
\item [$\isarkeyword{classes}~c<cs ~\dots$] declares class $c$ to be a
  subclass of existing classes $cs$.  Cyclic class structures are ruled out.
\item [$\isarkeyword{classrel}~c@1<c@2$] states a subclass relation between
  existing classes $c@1$ and $c@2$.  This is done axiomatically!  The
  $\isarkeyword{instance}$ command provides a way introduce proven class
  relations (see \S\ref{sec:axclass}).
\item [$\isarkeyword{defaultsort}~s$] makes sort $s$ the new default sort for
  any type variables input without sort constraints.  Typically, the default
  sort would be only changed when defining new logics.
\end{description}


\subsection{Types}

\indexisarcmd{typedecl}\indexisarcmd{types}\indexisarcmd{nonterminals}\indexisarcmd{arities}
\begin{matharray}{rcl}
  \isarcmd{types} & : & \isartrans{theory}{theory} \\
  \isarcmd{typedecl} & : & \isartrans{theory}{theory} \\
  \isarcmd{nonterminals} & : & \isartrans{theory}{theory} \\
  \isarcmd{arities} & : & \isartrans{theory}{theory} \\
\end{matharray}

\begin{rail}
  'types' (typespec '=' type infix? comment? +)
  ;
  'typedecl' typespec infix? comment?
  ;
  'nonterminals' (name +) comment?
  ;
  'arities' (nameref '::' arity comment? +)
  ;
\end{rail}

\begin{description}
\item [$\TYPES~(\vec\alpha)t = \tau~\dots$] introduces \emph{type synonym}
  $(\vec\alpha)t$ for existing type $\tau$.  Unlike actual type definitions,
  as are available in Isabelle/HOL for example, type synonyms are just purely
  syntactic abbreviations, without any logical significance.  Internally, type
  synonyms are fully expanded, as may be observed when printing terms or
  theorems.
\item [$\isarkeyword{typedecl}~(\vec\alpha)t$] declares a new type constructor
  $t$, intended as an actual logical type.  Note that some logics such as
  Isabelle/HOL provide their own version of $\isarkeyword{typedecl}$.
\item [$\isarkeyword{nonterminals}~c~\dots$] declares $0$-ary type
  constructors $c$ to act as purely syntactic types, i.e.\ nonterminal symbols
  of Isabelle's inner syntax of terms or types.
\item [$\isarkeyword{arities}~t::(\vec s)s~\dots$] augments Isabelle's
  order-sorted signature of types by new type constructor arities.  This is
  done axiomatically!  The $\isarkeyword{instance}$ command provides a way
  introduce proven type arities (see \S\ref{sec:axclass}).
\end{description}


\subsection{Constants and simple definitions}

\indexisarcmd{consts}\indexisarcmd{defs}\indexisarcmd{constdefs}
\begin{matharray}{rcl}
  \isarcmd{consts} & : & \isartrans{theory}{theory} \\
  \isarcmd{defs} & : & \isartrans{theory}{theory} \\
  \isarcmd{constdefs} & : & \isartrans{theory}{theory} \\
\end{matharray}

\begin{rail}
  'consts' (constdecl +)
  ;
  'defs' (thmdecl? prop comment? +)
  ;
  'constdefs' (constdecl prop comment? +)
  ;

  constdecl: name '::' type mixfix? comment?
  ;
\end{rail}

\begin{description}
\item [$\CONSTS~c::\tau~\dots$] declares constant $c$ to have any instance of
  type scheme $\tau$.  The optional mixfix annotations may attach concrete
  syntax to the constant.
\item [$\DEFS~name: eqn~\dots$] introduces $eqn$ as a definitional axiom for
  some existing constant.  See \cite[\S6]{isabelle-ref} for more details on
  the form of equations admitted as constant definitions.
\item [$\isarkeyword{constdefs}~c::\tau~eqn~\dots$] combines constant
  declarations and definitions, using canonical name $c_def$ for the
  definitional axiom.
\end{description}


\subsection{Concrete syntax}

\indexisarcmd{syntax}\indexisarcmd{translations}
\begin{matharray}{rcl}
  \isarcmd{syntax} & : & \isartrans{theory}{theory} \\
  \isarcmd{translations} & : & \isartrans{theory}{theory} \\
\end{matharray}

\begin{rail}
  'syntax' ('(' name 'output'? ')')? (constdecl +)
  ;
  'translations' (transpat ('==' | '=>' | '<=') transpat comment? +)
  ;
  transpat: ('(' nameref ')')? string
  ;
\end{rail}

\begin{description}
\item [$\isarkeyword{syntax}~mode~decls$] is similar to $\CONSTS~decls$,
  except the actual logical signature extension.  Thus the context free
  grammar of Isabelle's inner syntax may be augmented in arbitrary ways.  The
  $mode$ argument refers to the print mode that the grammar rules belong;
  unless there is the \texttt{output} flag given, all productions are added
  both to the input and output grammar.
\item [$\isarkeyword{translations}~rule~\dots$] specifies syntactic
  translation rules (macros): parse/print rules (\texttt{==}), parse rules
  (\texttt{=>}), print rules (\texttt{<=}).  Translation patterns may be
  prefixed by the syntactic category to be used for parsing; the default is
  \texttt{logic}.
\end{description}


\subsection{Axioms and theorems}

\indexisarcmd{axioms}\indexisarcmd{theorems}\indexisarcmd{lemmas}
\begin{matharray}{rcl}
  \isarcmd{axioms} & : & \isartrans{theory}{theory} \\
  \isarcmd{theorems} & : & \isartrans{theory}{theory} \\
  \isarcmd{lemmas} & : & \isartrans{theory}{theory} \\
\end{matharray}

\begin{rail}
  'axioms' (axmdecl prop comment? +)
  ;
  ('theorems' | 'lemmas') thmdef? thmrefs
  ;
\end{rail}

\begin{description}
\item [$\isarkeyword{axioms}~name: \phi~\dots$] introduces arbitrary
  statements as logical axioms.  In fact, axioms are ``axiomatic theorems'',
  and may be referred as any other theorems later.
  
  Axioms are usually only introduced when declaring new logical systems.
  Everyday work is normally done the hard way, with proper definitions and
  actual theorems.
\item [$\isarkeyword{theorems}~name = thms$] stores lists of existing theorems
  as $name$.  Typical applications would also involve attributes to augment
  the default simpset, for example.
\item [$\isarkeyword{lemmas}$] is similar to $\isarkeyword{theorems}$, but
  tags the results as ``lemma''.
\end{description}


\subsection{Manipulating name spaces}

\indexisarcmd{global}\indexisarcmd{local}\indexisarcmd{path}
\begin{matharray}{rcl}
  \isarcmd{global} & : & \isartrans{theory}{theory} \\
  \isarcmd{local} & : & \isartrans{theory}{theory} \\
  \isarcmd{path} & : & \isartrans{theory}{theory} \\
\end{matharray}

\begin{rail}
  'global'
  ;
  'local'
  ;
  'path' nameref
  ;
\end{rail}

\begin{description}
\item [$\isarkeyword{global}$] FIXME
\item [$\isarkeyword{local}$] FIXME
\item [$\isarkeyword{path}~name$] FIXME
\end{description}


\subsection{Incorporating ML code}

\indexisarcmd{use}\indexisarcmd{ML}\indexisarcmd{setup}
\begin{matharray}{rcl}
  \isarcmd{use} & : & \isartrans{\cdot}{\cdot} \\
  \isarcmd{ML} & : & \isartrans{\cdot}{\cdot} \\
  \isarcmd{setup} & : & \isartrans{\cdot}{\cdot} \\
\end{matharray}

\begin{rail}
  'use' name
  ;
  'ML' text
  ;
  'setup' text
  ;
\end{rail}

\begin{description}
\item [$\isarkeyword{use}~file$] FIXME
\item [$\isarkeyword{ML}~text$] FIXME
\item [$\isarkeyword{setup}~text$] FIXME
\end{description}


\subsection{ML translation functions}

\indexisarcmd{parse_ast_translation}\indexisarcmd{parse_translation}
\indexisarcmd{print_translation}\indexisarcmd{typed_print_translation}
\indexisarcmd{print_ast_translation}\indexisarcmd{token_translation}
\begin{matharray}{rcl}
  \isarcmd{parse_ast_translation} & : & \isartrans{theory}{theory} \\
  \isarcmd{parse_translation} & : & \isartrans{theory}{theory} \\
  \isarcmd{print_translation} & : & \isartrans{theory}{theory} \\
  \isarcmd{typed_print_translation} & : & \isartrans{theory}{theory} \\
  \isarcmd{print_ast_translation} & : & \isartrans{theory}{theory} \\
  \isarcmd{token_translation} & : & \isartrans{theory}{theory} \\
\end{matharray}

Syntax translation functions written in ML admit almost arbitrary
manipulations of Isabelle's inner syntax.  Any of the above commands have a
single \railqtoken{text} argument that refers to an ML expression of
appropriate type.  See \cite[\S8]{isabelle-ref} for more information on syntax
transformations.


\subsection{Oracles}

\indexisarcmd{oracle}
\begin{matharray}{rcl}
  \isarcmd{oracle} & : & \isartrans{theory}{theory} \\
\end{matharray}

\begin{rail}
  'oracle' name '=' text comment?
  ;
\end{rail}

\begin{description}
\item [$\isarkeyword{oracle}~name=text$] FIXME
\end{description}


\section{Proof commands}

\subsection{Goal statements}

\indexisarcmd{}
\begin{matharray}{rcl}
  \isarcmd{theorem} & : & \isartrans{theory}{proof(prove)} \\
  \isarcmd{lemma} & : & \isartrans{theory}{proof(prove)} \\
  \isarcmd{have} & : & \isartrans{proof(state)}{proof(prove)} \\
  \isarcmd{show} & : & \isartrans{proof(state)}{proof(prove)} \\
  \isarcmd{hence} & : & \isartrans{proof(state)}{proof(prove)} \\
  \isarcmd{thus} & : & \isartrans{proof(state)}{proof(prove)} \\
\end{matharray}

\begin{rail}
  ('theorem' | 'lemma') goal
  ;
  ('have' | 'show' | 'hence' | 'thus') goal
  ;

  goal: thmdecl? proppat comment?
  ;
\end{rail}

\begin{description}
\item [$\THEOREM{name}{\phi}$] enters proof mode with $\phi$ as main goal,
  eventually resulting in some theorem $\turn \phi$, which stored in the
  theory.
\item [$\LEMMANAME$] is similar to $\THEOREMNAME$, but tags the result as
  ``lemma''.
\item [$\HAVE{name}{\phi}$] FIXME
\item [$\SHOW{name}{\phi}$] FIXME
\item [$\HENCE{name}{\phi}$] FIXME
\item [$\THUS{name}{\phi}$] FIXME
\end{description}


\subsection{Initial and terminal proof steps}

\indexisarcmd{proof}\indexisarcmd{qed}\indexisarcmd{by}
\indexisarcmd{.}\indexisarcmd{..}\indexisarcmd{sorry}
\begin{matharray}{rcl}
  \isarcmd{proof} & : & \isartrans{proof(prove)}{proof(state)} \\
  \isarcmd{qed} & : & \isartrans{proof(state)}{proof(state) ~|~ theory} \\
  \isarcmd{by} & : & \isartrans{proof(prove)}{proof(state) ~|~ theory} \\
  \isarcmd{.} & : & \isartrans{proof(prove)}{proof(state) ~|~ theory} \\
  \isarcmd{..} & : & \isartrans{proof(prove)}{proof(state) ~|~ theory} \\
  \isarcmd{sorry} & : & \isartrans{proof(prove)}{proof(state) ~|~ theory} \\
\end{matharray}

\begin{rail}
  'proof' interest? meth? comment?
  ;
  'qed' meth? comment?
  ;
  'by' meth meth? comment?
  ;
  ('.' | '..' | 'sorry') comment?
  ;

  meth: method interest?
  ;
\end{rail}

\begin{description}
\item [$ $] FIXME
\end{description}


\subsection{Facts and forward chaining}

\indexisarcmd{note}\indexisarcmd{then}\indexisarcmd{from}\indexisarcmd{with}
\begin{matharray}{rcl}
  \isarcmd{note} & : & \isartrans{proof(state)}{proof(state)} \\
  \isarcmd{then} & : & \isartrans{proof(state)}{proof(chain)} \\
  \isarcmd{from} & : & \isartrans{proof(state)}{proof(chain)} \\
  \isarcmd{with} & : & \isartrans{proof(state)}{proof(chain)} \\
\end{matharray}

\begin{rail}
  'note' thmdef? thmrefs comment?
  ;
  'then' comment?
  ;
  ('from' | 'with') thmrefs comment?
  ;
\end{rail}

\begin{description}
\item [$ $] FIXME
\end{description}


\subsection{Proof context}

\indexisarcmd{fix}\indexisarcmd{assume}\indexisarcmd{presume}\indexisarcmd{def}\indexisarcmd{let}
\begin{matharray}{rcl}
  \isarcmd{fix} & : & \isartrans{proof(state)}{proof(state)} \\
  \isarcmd{assume} & : & \isartrans{proof(state)}{proof(state)} \\
  \isarcmd{presume} & : & \isartrans{proof(state)}{proof(state)} \\
  \isarcmd{def} & : & \isartrans{proof(state)}{proof(state)} \\
  \isarcmd{let} & : & \isartrans{proof(state)}{proof(state)} \\
\end{matharray}

\begin{rail}
  'fix' (var +) comment?
  ;
  ('assume' | 'presume') thmdecl? (proppat +) comment?
  ;
  'def' thmdecl? var '==' termpat comment?
  ;
  'let' ((term + 'as') '=' term comment? + 'and')
  ;

  var: name ('::' type)?
  ;
\end{rail}

\begin{description}
\item [$ $] FIXME
\end{description}


\subsection{Block structure}

\indexisarcmd{next}\indexisarcmd{\{\{}\indexisarcmd{\}\}}
\begin{matharray}{rcl}
  \isarcmd{next} & : & \isartrans{proof(state)}{proof(state)} \\
  \isarcmd{\{\{} & : & \isartrans{proof(state)}{proof(state)} \\
  \isarcmd{\}\}} & : & \isartrans{proof(state)}{proof(state)} \\
\end{matharray}

\begin{rail}
  llbrace
  ;
  rrbrace
  ;
  'next'
  ;
\end{rail}

\begin{description}
\item [$ $] FIXME
\end{description}


\subsection{Calculational proof}

\indexisarcmd{also}\indexisarcmd{finally}
\begin{matharray}{rcl}
  \isarcmd{also} & : & \isartrans{proof(state)}{proof(state)} \\
  \isarcmd{finally} & : & \isartrans{proof(state)}{proof(chain)} \\
\end{matharray}

\begin{rail}
  ('also' | 'finally') transrules? comment?
  ;

  transrules: '(' thmrefs ')' interest?
  ;
\end{rail}

\begin{description}
\item [$ $] FIXME
\end{description}



\subsection{Improper proof steps}

\indexisarcmd{apply}\indexisarcmd{then_apply}\indexisarcmd{back}
\begin{matharray}{rcl}
  \isarcmd{apply}^* & : & \isartrans{proof}{proof} \\
  \isarcmd{then_apply}^* & : & \isartrans{proof}{proof} \\
  \isarcmd{back}^* & : & \isartrans{proof}{proof} \\
\end{matharray}

\railalias{thenapply}{then\_apply}
\railterm{thenapply}

\begin{rail}
  'apply' method
  ;
  thenapply method
  ;
  'back'
  ;
\end{rail}

\begin{description}
\item [$ $] FIXME
\end{description}


\section{Other commands}

\subsection{Diagnostics}

\indexisarcmd{typ}\indexisarcmd{term}\indexisarcmd{prop}\indexisarcmd{thm}
\begin{matharray}{rcl}
  \isarcmd{typ} & : & \isarkeep{theory~|~proof} \\
  \isarcmd{term} & : & \isarkeep{theory~|~proof} \\
  \isarcmd{prop} & : & \isarkeep{theory~|~proof} \\
  \isarcmd{thm} & : & \isarkeep{theory~|~proof} \\
\end{matharray}

\begin{rail}
  'typ' type
  ;
  'term' term
  ;
  'prop' prop
  ;
  'thm' thmrefs
  ;
\end{rail}

\begin{description}
\item [$\isarkeyword{typ}~\tau$, $\isarkeyword{term}~t$,
  $\isarkeyword{prop}~\phi$] read and print types / terms / propositions
  according to the current theory or proof context.
\item [$\isarkeyword{thm}~thms$] retrieves lists of theorems from the current
  theory or proof context.  Note that any attributes included in the theorem
  specifications are applied to a temporary proof context derived from the
  current theory or proof; the resulting context is discarded.
\end{description}


\subsection{System operations}

\indexisarcmd{cd}\indexisarcmd{pwd}\indexisarcmd{use_thy}\indexisarcmd{use_thy_only}
\indexisarcmd{update_thy}\indexisarcmd{update_thy_only}
\begin{matharray}{rcl}
  \isarcmd{cd} & : & \isarkeep{\cdot} \\
  \isarcmd{pwd} & : & \isarkeep{\cdot} \\
  \isarcmd{use_thy} & : & \isarkeep{\cdot} \\
  \isarcmd{use_thy_only} & : & \isarkeep{\cdot} \\
  \isarcmd{update_thy} & : & \isarkeep{\cdot} \\
  \isarcmd{update_thy_only} & : & \isarkeep{\cdot} \\
\end{matharray}

\begin{description}
\item [$\isarkeyword{cd}~name$] changes the current directory of the Isabelle
  process.
\item [$\isarkeyword{pwd}~$] prints the current working directory.
\item [$\isarkeyword{use_thy}~name$, $\isarkeyword{use_thy_only}~name$,
  $\isarkeyword{update_thy}~name$, $\isarkeyword{update_thy_only}~name$] load
  theory files.  These commands are exactly the same as the corresponding ML
  functions (see also \cite[\S1 and \S6]{isabelle-ref}).
\end{description}


%%% Local Variables: 
%%% mode: latex
%%% TeX-master: "isar-ref"
%%% End: 
