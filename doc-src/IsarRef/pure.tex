
\chapter{Basic Isar Elements}\label{ch:pure-syntax}

Subsequently, we introduce the main part of the basic Isar theory and proof
commands as provided by Isabelle/Pure.  Chapter~\ref{ch:gen-tools} describes
further Isar elements as provided by generic tools and packages (such as the
Simplifier) that are either part of Pure Isabelle, or pre-loaded by most
object logics.  See chapter~\ref{ch:hol-tools} for actual object-logic
specific elements (for Isabelle/HOL).

\medskip

Isar commands may be either \emph{proper} document constructors, or
\emph{improper commands} (indicated by $^*$).  Some proof methods and
attributes introduced later are classified as improper as well.  Improper Isar
language elements might be helpful when developing proof documents, while
their use is strongly discouraged for the final version.  Typical examples are
diagnostic commands that print terms or theorems according to the current
context; other commands even emulate old-style tactical theorem proving, which
facilitates porting of legacy proof scripts.


\section{Theory commands}

\subsection{Defining theories}\label{sec:begin-thy}

\indexisarcmd{theory}\indexisarcmd{end}\indexisarcmd{context}
\begin{matharray}{rcl}
  \isarcmd{theory} & : & \isartrans{\cdot}{theory} \\
  \isarcmd{context}^* & : & \isartrans{\cdot}{theory} \\
  \isarcmd{end} & : & \isartrans{theory}{\cdot} \\
\end{matharray}

Isabelle/Isar ``new-style'' theories are either defined via theory files or
interactively.  Both actual theory specifications and proofs are handled
uniformly --- occasionally definitional mechanisms even require some explicit
proof as well.  In contrast, ``old-style'' Isabelle theories support batch
processing only, with the proof scripts collected in separate ML files.

The first command of any theory has to be $\THEORY$, starting a new theory
based on the merge of existing ones.  The theory context may be also changed
by $\CONTEXT$ without creating a new theory.  In both cases, $\END$ concludes
the theory development; it has to be the very last command of any proper
theory file.

\begin{rail}
  'theory' name '=' (name + '+') filespecs? ':'
  ;
  'context' name
  ;
  'end'
  ;;

  filespecs: 'files' ((name | parname) +);
\end{rail}

\begin{descr}
\item [$\THEORY~A = B@1 + \cdots + B@n$] commences a new theory $A$ based on
  existing ones $B@1 + \cdots + B@n$.  Isabelle's theory loader system ensures
  that any of the base theories are properly loaded (and fully up-to-date when
  $\THEORY$ is executed interactively).  The optional $\isarkeyword{files}$
  specification declares additional dependencies on ML files.  Unless put in
  parentheses, any file will be loaded immediately via $\isarcmd{use}$ (see
  also \S\ref{sec:ML}).  The optional ML file \texttt{$A$.ML} that may be
  associated with any theory should \emph{not} be included in
  $\isarkeyword{files}$.
  
\item [$\CONTEXT~B$] enters an existing theory context $B$, basically in
  read-only mode, so only a limited set of commands may be performed.  Just as
  for $\THEORY$, the theory loader ensures that $B$ is loaded and up-to-date.
  
\item [$\END$] concludes the current theory definition or context switch.
  Note that this command cannot be undone, instead the theory definition
  itself has to be retracted.
\end{descr}


\subsection{Formal comments}\label{sec:formal-cmt-thy}

\indexisarcmd{title}\indexisarcmd{chapter}\indexisarcmd{section}\indexisarcmd{subsection}
\indexisarcmd{subsubsection}\indexisarcmd{text}
\begin{matharray}{rcl}
  \isarcmd{title} & : & \isartrans{theory}{theory} \\
  \isarcmd{chapter} & : & \isartrans{theory}{theory} \\
  \isarcmd{section} & : & \isartrans{theory}{theory} \\
  \isarcmd{subsection} & : & \isartrans{theory}{theory} \\
  \isarcmd{subsubsection} & : & \isartrans{theory}{theory} \\
  \isarcmd{text} & : & \isartrans{theory}{theory} \\
\end{matharray}

There are several commands to include \emph{formal comments} in theory
specification (a few more are available for proofs, see
\S\ref{sec:formal-cmt-prf}).  In contrast to source-level comments
\verb|(*|\dots\verb|*)|, which are stripped at the lexical level, any text
given as formal comment is meant to be part of the actual document.
Consequently, it would be included in the final printed version.

Apart from plain prose, formal comments may also refer to logical entities of
the theory context (types, terms, theorems etc.).  Proper processing of the
text would then include some further consistency checks with the items
declared in the current theory, e.g.\ type-checking of included
terms.\footnote{The current version of Isabelle/Isar does not process formal
  comments in any such way.  This will be available as part of the automatic
  theory and proof document preparation system (using (PDF){\LaTeX}) that is
  planned for the near future.}

\begin{rail}
  'title' text text? text?
  ;
  ('chapter' | 'section' | 'subsection' | 'subsubsection' | 'text') text
  ;
\end{rail}

\begin{descr}
\item [$\isarkeyword{title}~title~author~date$] specifies the document title
  just as in typical {\LaTeX} documents.
\item [$\isarkeyword{chapter}$, $\isarkeyword{section}$,
  $\isarkeyword{subsection}$, and $\isarkeyword{subsubsection}$] mark chapter
  and section headings.
\item [$\TEXT$] specifies an actual body of prose text, including references
  to formal entities.\footnote{The latter feature is not yet exploited.
    Nevertheless, any text of the form \texttt{\at\ttlbrace\dots\ttrbrace}
    should be considered as reserved for future use.}
\end{descr}


\subsection{Type classes and sorts}\label{sec:classes}

\indexisarcmd{classes}\indexisarcmd{classrel}\indexisarcmd{defaultsort}
\begin{matharray}{rcl}
  \isarcmd{classes} & : & \isartrans{theory}{theory} \\
  \isarcmd{classrel} & : & \isartrans{theory}{theory} \\
  \isarcmd{defaultsort} & : & \isartrans{theory}{theory} \\
\end{matharray}

\begin{rail}
  'classes' (classdecl comment? +)
  ;
  'classrel' nameref '<' nameref comment?
  ;
  'defaultsort' sort comment?
  ;
\end{rail}

\begin{descr}
\item [$\isarkeyword{classes}~c<\vec c$] declares class $c$ to be a subclass
  of existing classes $\vec c$.  Cyclic class structures are ruled out.
\item [$\isarkeyword{classrel}~c@1<c@2$] states a subclass relation between
  existing classes $c@1$ and $c@2$.  This is done axiomatically!  The
  $\isarkeyword{instance}$ command (see \S\ref{sec:axclass}) provides a way
  introduce proven class relations.
\item [$\isarkeyword{defaultsort}~s$] makes sort $s$ the new default sort for
  any type variables input without sort constraints.  Usually, the default
  sort would be only changed when defining new logics.
\end{descr}


\subsection{Primitive types and type abbreviations}\label{sec:types-pure}

\indexisarcmd{typedecl}\indexisarcmd{types}\indexisarcmd{nonterminals}\indexisarcmd{arities}
\begin{matharray}{rcl}
  \isarcmd{types} & : & \isartrans{theory}{theory} \\
  \isarcmd{typedecl} & : & \isartrans{theory}{theory} \\
  \isarcmd{nonterminals} & : & \isartrans{theory}{theory} \\
  \isarcmd{arities} & : & \isartrans{theory}{theory} \\
\end{matharray}

\begin{rail}
  'types' (typespec '=' type infix? comment? +)
  ;
  'typedecl' typespec infix? comment?
  ;
  'nonterminals' (name +) comment?
  ;
  'arities' (nameref '::' arity comment? +)
  ;
\end{rail}

\begin{descr}
\item [$\TYPES~(\vec\alpha)t = \tau$] introduces \emph{type synonym}
  $(\vec\alpha)t$ for existing type $\tau$.  Unlike actual type definitions,
  as are available in Isabelle/HOL for example, type synonyms are just purely
  syntactic abbreviations, without any logical significance.  Internally, type
  synonyms are fully expanded, as may be observed when printing terms or
  theorems.
\item [$\isarkeyword{typedecl}~(\vec\alpha)t$] declares a new type constructor
  $t$, intended as an actual logical type.  Note that some logics such as
  Isabelle/HOL provide their own version of $\isarkeyword{typedecl}$.
\item [$\isarkeyword{nonterminals}~\vec c$] declares $0$-ary type constructors
  $\vec c$ to act as purely syntactic types, i.e.\ nonterminal symbols of
  Isabelle's inner syntax of terms or types.
\item [$\isarkeyword{arities}~t::(\vec s)s$] augments Isabelle's order-sorted
  signature of types by new type constructor arities.  This is done
  axiomatically!  The $\isarkeyword{instance}$ command (see
  \S\ref{sec:axclass}) provides a way introduce proven type arities.
\end{descr}


\subsection{Constants and simple definitions}

\indexisarcmd{consts}\indexisarcmd{defs}\indexisarcmd{constdefs}\indexoutertoken{constdecl}
\begin{matharray}{rcl}
  \isarcmd{consts} & : & \isartrans{theory}{theory} \\
  \isarcmd{defs} & : & \isartrans{theory}{theory} \\
  \isarcmd{constdefs} & : & \isartrans{theory}{theory} \\
\end{matharray}

\begin{rail}
  'consts' (constdecl +)
  ;
  'defs' (thmdecl? prop comment? +)
  ;
  'constdefs' (constdecl prop comment? +)
  ;

  constdecl: name '::' type mixfix? comment?
  ;
\end{rail}

\begin{descr}
\item [$\CONSTS~c::\sigma$] declares constant $c$ to have any instance of type
  scheme $\sigma$.  The optional mixfix annotations may attach concrete syntax
  constants.
\item [$\DEFS~name: eqn$] introduces $eqn$ as a definitional axiom for some
  existing constant.  See \cite[\S6]{isabelle-ref} for more details on the
  form of equations admitted as constant definitions.
\item [$\isarkeyword{constdefs}~c::\sigma~eqn$] combines declarations and
  definitions of constants, using canonical name $c_def$ for the definitional
  axiom.
\end{descr}


\subsection{Syntax and translations}

\indexisarcmd{syntax}\indexisarcmd{translations}
\begin{matharray}{rcl}
  \isarcmd{syntax} & : & \isartrans{theory}{theory} \\
  \isarcmd{translations} & : & \isartrans{theory}{theory} \\
\end{matharray}

\begin{rail}
  'syntax' ('(' name 'output'? ')')? (constdecl +)
  ;
  'translations' (transpat ('==' | '=>' | '<=') transpat comment? +)
  ;
  transpat: ('(' nameref ')')? string
  ;
\end{rail}

\begin{descr}
\item [$\isarkeyword{syntax}~(mode)~decls$] is similar to $\CONSTS~decls$,
  except that the actual logical signature extension is omitted.  Thus the
  context free grammar of Isabelle's inner syntax may be augmented in
  arbitrary ways, independently of the logic.  The $mode$ argument refers to
  the print mode that the grammar rules belong; unless there is the
  \texttt{output} flag given, all productions are added both to the input and
  output grammar.
\item [$\isarkeyword{translations}~rules$] specifies syntactic translation
  rules (also known as \emph{macros}): parse/print rules (\texttt{==}), parse
  rules (\texttt{=>}), or print rules (\texttt{<=}).  Translation patterns may
  be prefixed by the syntactic category to be used for parsing; the default is
  \texttt{logic}.
\end{descr}


\subsection{Axioms and theorems}

\indexisarcmd{axioms}\indexisarcmd{theorems}\indexisarcmd{lemmas}
\begin{matharray}{rcl}
  \isarcmd{axioms} & : & \isartrans{theory}{theory} \\
  \isarcmd{theorems} & : & \isartrans{theory}{theory} \\
  \isarcmd{lemmas} & : & \isartrans{theory}{theory} \\
\end{matharray}

\begin{rail}
  'axioms' (axmdecl prop comment? +)
  ;
  ('theorems' | 'lemmas') thmdef? thmrefs
  ;
\end{rail}

\begin{descr}
\item [$\isarkeyword{axioms}~a: \phi$] introduces arbitrary statements as
  logical axioms.  In fact, axioms are ``axiomatic theorems'', and may be
  referred later just as any other theorem.
  
  Axioms are usually only introduced when declaring new logical systems.
  Everyday work is typically done the hard way, with proper definitions and
  actual theorems.
\item [$\isarkeyword{theorems}~a = \vec b$] stores lists of existing theorems.
  Typical applications would also involve attributes, to augment the default
  Simplifier context, for example.
\item [$\isarkeyword{lemmas}$] is similar to $\isarkeyword{theorems}$, but
  tags the results as ``lemma''.
\end{descr}


\subsection{Name spaces}

\indexisarcmd{global}\indexisarcmd{local}
\begin{matharray}{rcl}
  \isarcmd{global} & : & \isartrans{theory}{theory} \\
  \isarcmd{local} & : & \isartrans{theory}{theory} \\
\end{matharray}

Isabelle organizes any kind of names (of types, constants, theorems etc.)  by
hierarchically structured name spaces.  Normally the user never has to control
the behavior of name space entry by hand, yet the following commands provide
some way to do so.

\begin{descr}
\item [$\isarkeyword{global}$ and $\isarkeyword{local}$] change the current
  name declaration mode.  Initially, theories start in $\isarkeyword{local}$
  mode, causing all names to be automatically qualified by the theory name.
  Changing this to $\isarkeyword{global}$ causes all names to be declared as
  base names only, until $\isarkeyword{local}$ is declared again.
\end{descr}


\subsection{Incorporating ML code}\label{sec:ML}

\indexisarcmd{use}\indexisarcmd{ML}\indexisarcmd{setup}
\begin{matharray}{rcl}
  \isarcmd{use} & : & \isartrans{\cdot}{\cdot} \\
  \isarcmd{ML} & : & \isartrans{\cdot}{\cdot} \\
  \isarcmd{setup} & : & \isartrans{theory}{theory} \\
\end{matharray}

\begin{rail}
  'use' name
  ;
  'ML' text
  ;
  'setup' text
  ;
\end{rail}

\begin{descr}
\item [$\isarkeyword{use}~file$] reads and executes ML commands from $file$.
  The current theory context (if present) is passed down to the ML session.
  Furthermore, the file name is checked with the $\isarkeyword{files}$
  dependency declaration given in the theory header (see also
  \S\ref{sec:begin-thy}).

\item [$\isarkeyword{ML}~text$] reads and executes ML commands from $text$.
  The theory context is passed just as for $\isarkeyword{use}$.

\item [$\isarkeyword{setup}~text$] changes the current theory context by
  applying setup functions $text$, which has to be an ML expression of type
  $(theory \to theory)~list$.  The $\isarkeyword{setup}$ command is the usual
  way to initialize object-logic specific tools and packages written in ML.
\end{descr}


\subsection{Syntax translation functions}

\indexisarcmd{parse-ast-translation}\indexisarcmd{parse-translation}
\indexisarcmd{print-translation}\indexisarcmd{typed-print-translation}
\indexisarcmd{print-ast-translation}\indexisarcmd{token-translation}
\begin{matharray}{rcl}
  \isarcmd{parse_ast_translation} & : & \isartrans{theory}{theory} \\
  \isarcmd{parse_translation} & : & \isartrans{theory}{theory} \\
  \isarcmd{print_translation} & : & \isartrans{theory}{theory} \\
  \isarcmd{typed_print_translation} & : & \isartrans{theory}{theory} \\
  \isarcmd{print_ast_translation} & : & \isartrans{theory}{theory} \\
  \isarcmd{token_translation} & : & \isartrans{theory}{theory} \\
\end{matharray}

Syntax translation functions written in ML admit almost arbitrary
manipulations of Isabelle's inner syntax.  Any of the above commands have a
single \railqtoken{text} argument that refers to an ML expression of
appropriate type.  See \cite[\S8]{isabelle-ref} for more information on syntax
transformations.


\subsection{Oracles}

\indexisarcmd{oracle}
\begin{matharray}{rcl}
  \isarcmd{oracle} & : & \isartrans{theory}{theory} \\
\end{matharray}

Oracles provide an interface to external reasoning systems, without giving up
control completely --- each theorem carries a derivation object recording any
oracle invocation.  See \cite[\S6]{isabelle-ref} for more information.

\begin{rail}
  'oracle' name '=' text comment?
  ;
\end{rail}

\begin{descr}
\item [$\isarkeyword{oracle}~name=text$] declares oracle $name$ to be ML
  function $text$, which has to be of type $Sign\mathord.sg \times
  Object\mathord.T \to term$.
\end{descr}


\section{Proof commands}

Proof commands provide transitions of Isar/VM machine configurations, which
are block-structured, consisting of a stack of nodes with three main
components: logical proof context, current facts, and open goals.  Isar/VM
transitions are \emph{typed} according to the following three three different
modes of operation:
\begin{descr}
\item [$proof(prove)$] means that a new goal has just been stated that is now
  to be \emph{proven}; the next command may refine it by some proof method
  ($\approx$ tactic), and enter a sub-proof to establish the final result.
\item [$proof(state)$] is like an internal theory mode: the context may be
  augmented by \emph{stating} additional assumptions, intermediate result etc.
\item [$proof(chain)$] is an intermediate mode between $proof(state)$ and
  $proof(prove)$: existing facts (i.e.\ the contents of $this$) have been just
  picked up in order to use them when refining the goal claimed next.
\end{descr}


\subsection{Formal comments}\label{sec:formal-cmt-prf}

\indexisarcmd{sect}\indexisarcmd{subsect}\indexisarcmd{subsect}\indexisarcmd{txt}
\begin{matharray}{rcl}
  \isarcmd{sect} & : & \isartrans{proof(state)}{proof(state)} \\
  \isarcmd{subsect} & : & \isartrans{proof(state)}{proof(state)} \\
  \isarcmd{subsubsect} & : & \isartrans{proof(state)}{proof(state)} \\
  \isarcmd{txt} & : & \isartrans{proof(state)}{proof(state)} \\
\end{matharray}

These formal comments in proof mode closely correspond to the ones of theory
mode (see \S\ref{sec:formal-cmt-thy}).

\begin{rail}
  ('sect' | 'subsect' | 'subsubsect' | 'txt') text
  ;
\end{rail}


\subsection{Proof context}\label{sec:proof-context}

\indexisarcmd{fix}\indexisarcmd{assume}\indexisarcmd{presume}\indexisarcmd{def}
\begin{matharray}{rcl}
  \isarcmd{fix} & : & \isartrans{proof(state)}{proof(state)} \\
  \isarcmd{assume} & : & \isartrans{proof(state)}{proof(state)} \\
  \isarcmd{presume} & : & \isartrans{proof(state)}{proof(state)} \\
  \isarcmd{def} & : & \isartrans{proof(state)}{proof(state)} \\
\end{matharray}

The logical proof context consists of fixed variables and assumptions.  The
former closely correspond to Skolem constants, or meta-level universal
quantification as provided by the Isabelle/Pure logical framework.
Introducing some \emph{arbitrary, but fixed} variable via $\FIX x$ results in
a local object that may be used in the subsequent proof as any other variable
or constant.  Furthermore, any result $\phi[x]$ exported from the current
context will be universally closed wrt.\ $x$ at the outermost level (this is
expressed using Isabelle's meta-variables).

Similarly, introducing some assumption $\chi$ has two effects.  On the one
hand, a local theorem is created that may be used as a fact in subsequent
proof steps.  On the other hand, any result $\phi$ exported from the context
becomes conditional wrt.\ the assumption.  Thus, solving an enclosing goal
using such a result would basically introduce a new subgoal stemming from the
assumption.  How this situation is handled depends on the actual version of
assumption command used: while $\ASSUMENAME$ solves the subgoal by unifying
with some premise of the goal, $\PRESUMENAME$ leaves the subgoal unchanged to
be proved later by the user.

Local definitions, introduced by $\DEF{}{x \equiv t}$, are achieved by
combining $\FIX x$ with another kind of assumption that causes any
hypothetical equation $x = t$ to be eliminated by reflexivity.  Thus,
exporting some result $\phi[x]$ simply yields $\phi[t]$.

\begin{rail}
  'fix' (vars + 'and') comment?
  ;
  ('assume' | 'presume') (assm comment? + 'and')
  ;
  'def' thmdecl? \\ var '==' term termpat? comment?
  ;

  var: name ('::' type)?
  ;
  vars: (name+) ('::' type)?
  ;
  assm: thmdecl? (prop proppat? +)
  ;
\end{rail}

\begin{descr}
\item [$\FIX{x}$] introduces a local \emph{arbitrary, but fixed} variable $x$.
\item [$\ASSUME{a}{\Phi}$ and $\PRESUME{a}{\Phi}$] introduce local theorems
  $\Phi$ by assumption.  Subsequent results applied to an enclosing goal
  (e.g.\ via $\SHOWNAME$) are handled as follows: $\ASSUMENAME$ expects to be
  able to unify with existing premises in the goal, while $\PRESUMENAME$
  leaves $\Phi$ as new subgoals.
  
  Several lists of assumptions may be given (separated by
  $\isarkeyword{and}$); the resulting list of facts consists of all of these
  concatenated.
\item [$\DEF{a}{x \equiv t}$] introduces a local (non-polymorphic) definition.
  In results exported from the context, $x$ is replaced by $t$.  Basically,
  $\DEF{}{x \equiv t}$ abbreviates $\FIX{x}~\PRESUME{}{x \equiv t}$, with the
  resulting hypothetical equation solved by reflexivity.
  
  The default name for the definitional equation is $x_def$.
\end{descr}

The special theorem name $prems$\indexisarthm{prems} refers to all current
assumptions.


\subsection{Facts and forward chaining}

\indexisarcmd{note}\indexisarcmd{then}\indexisarcmd{from}\indexisarcmd{with}
\begin{matharray}{rcl}
  \isarcmd{note} & : & \isartrans{proof(state)}{proof(state)} \\
  \isarcmd{then} & : & \isartrans{proof(state)}{proof(chain)} \\
  \isarcmd{from} & : & \isartrans{proof(state)}{proof(chain)} \\
  \isarcmd{with} & : & \isartrans{proof(state)}{proof(chain)} \\
\end{matharray}

New facts are established either by assumption or proof of local statements.
Any fact will usually be involved in further proofs, either as explicit
arguments of proof methods or when forward chaining towards the next goal via
$\THEN$ (and variants).  Note that the special theorem name
$this$.\indexisarthm{this} refers to the most recently established facts.
\begin{rail}
  'note' thmdef? thmrefs comment?
  ;
  'then' comment?
  ;
  ('from' | 'with') thmrefs comment?
  ;
\end{rail}

\begin{descr}
\item [$\NOTE{a}{\vec b}$] recalls existing facts $\vec b$, binding the result
  as $a$.  Note that attributes may be involved as well, both on the left and
  right hand sides.
\item [$\THEN$] indicates forward chaining by the current facts in order to
  establish the goal claimed next.  The initial proof method invoked to refine
  that will be offered these facts to do ``anything appropriate'' (see also
  \S\ref{sec:proof-steps}).  For example, method $rule$ (see
  \S\ref{sec:pure-meth}) would do an elimination rather than an introduction.
\item [$\FROM{\vec b}$] abbreviates $\NOTE{}{\vec b}~\THEN$; thus $\THEN$ is
  equivalent to $\FROM{this}$.
\item [$\WITH{\vec b}$] abbreviates $\FROM{\vec b~facts}$; thus the forward
  chaining is from earlier facts together with the current ones.
\end{descr}

Basic proof methods (such as $rule$, see \S\ref{sec:pure-meth}) expect
multiple facts to be given in proper order, corresponding to a prefix of the
premises of the rule involved.  Note that positions may be easily skipped
using a form like $\FROM{\text{\texttt{_}}~a~b}$, for example.  This involves
the rule $\PROP\psi \Imp \PROP\psi$, which is bound in Isabelle/Pure as
``\texttt{_}'' (underscore).\indexisarthm{_@\texttt{_}}


\subsection{Goal statements}

\indexisarcmd{theorem}\indexisarcmd{lemma}
\indexisarcmd{have}\indexisarcmd{show}\indexisarcmd{hence}\indexisarcmd{thus}
\begin{matharray}{rcl}
  \isarcmd{theorem} & : & \isartrans{theory}{proof(prove)} \\
  \isarcmd{lemma} & : & \isartrans{theory}{proof(prove)} \\
  \isarcmd{have} & : & \isartrans{proof(state)}{proof(prove)} \\
  \isarcmd{show} & : & \isartrans{proof(state)}{proof(prove)} \\
  \isarcmd{hence} & : & \isartrans{proof(state)}{proof(prove)} \\
  \isarcmd{thus} & : & \isartrans{proof(state)}{proof(prove)} \\
\end{matharray}

Proof mode is entered from theory mode by initial goal commands $\THEOREMNAME$
and $\LEMMANAME$.  New local goals may be claimed within proof mode: four
variants are available, indicating whether the result is meant to solve some
pending goal and whether forward chaining is employed.

\begin{rail}
  ('theorem' | 'lemma') goal
  ;
  ('have' | 'show' | 'hence' | 'thus') goal
  ;

  goal: thmdecl? proppat comment?
  ;
\end{rail}

\begin{descr}
\item [$\THEOREM{a}{\phi}$] enters proof mode with $\phi$ as main goal,
  eventually resulting in some theorem $\turn \phi$, which will be stored in
  the theory.
\item [$\LEMMANAME$] is similar to $\THEOREMNAME$, but tags the result as
  ``lemma''.
\item [$\HAVE{a}{\phi}$] claims a local goal, eventually resulting in a
  theorem with the current assumption context as hypotheses.
\item [$\SHOW{a}{\phi}$] is similar to $\HAVE{a}{\phi}$, but solves some
  pending goal with the result \emph{exported} into the corresponding context.
\item [$\HENCE{a}{\phi}$] abbreviates $\THEN~\HAVE{a}{\phi}$, i.e.\ claims a
  local goal to be proven by forward chaining the current facts.  Note that
  $\HENCENAME$ is equivalent to $\FROM{this}~\HAVENAME$.
\item [$\THUS{a}{\phi}$] abbreviates $\THEN~\SHOW{a}{\phi}$.  Note that
  $\THUSNAME$ is equivalent to $\FROM{this}~\SHOWNAME$.
\end{descr}


\subsection{Initial and terminal proof steps}\label{sec:proof-steps}

\indexisarcmd{proof}\indexisarcmd{qed}\indexisarcmd{by}
\indexisarcmd{.}\indexisarcmd{..}\indexisarcmd{sorry}
\begin{matharray}{rcl}
  \isarcmd{proof} & : & \isartrans{proof(prove)}{proof(state)} \\
  \isarcmd{qed} & : & \isartrans{proof(state)}{proof(state) ~|~ theory} \\
  \isarcmd{by} & : & \isartrans{proof(prove)}{proof(state) ~|~ theory} \\
  \isarcmd{.\,.} & : & \isartrans{proof(prove)}{proof(state) ~|~ theory} \\
  \isarcmd{.} & : & \isartrans{proof(prove)}{proof(state) ~|~ theory} \\
  \isarcmd{sorry} & : & \isartrans{proof(prove)}{proof(state) ~|~ theory} \\
\end{matharray}

Arbitrary goal refinement via tactics is considered harmful.  Consequently the
Isar framework admits proof methods to be invoked in two places only.
\begin{enumerate}
\item An \emph{initial} refinement step $\PROOF{m@1}$ reduces a newly stated
  goal to a number of sub-goals that are to be solved later.  Facts are passed
  to $m@1$ for forward chaining if so indicated by $proof(chain)$ mode.
  
\item A \emph{terminal} conclusion step $\QED{m@2}$ solves any remaining goals
  completely.  No facts are passed to $m@2$.
\end{enumerate}

The only other proper way to affect pending goals is by $\SHOWNAME$ (or
$\THUSNAME$), which involves an explicit statement of what is to be solved.

\medskip

Also note that initial proof methods should either solve the goal completely,
or constitute some well-understood deterministic reduction to new sub-goals.
Arbitrary automatic proof tools that are prone leave a large number of badly
structured sub-goals are no help in continuing the proof document in any
intelligible way.  A much better technique would be to $\SHOWNAME$ some
non-trivial reduction as an explicit rule, which is solved completely by some
automated method, and then applied to some pending goal.

\medskip

Unless given explicitly by the user, the default initial method is
``$default$'', which is usually set up to apply a single standard elimination
or introduction rule according to the topmost symbol involved.  There is no
default terminal method; in any case the final step is to solve all remaining
goals by assumption.

\begin{rail}
  'proof' interest? meth? comment?
  ;
  'qed' meth? comment?
  ;
  'by' meth meth? comment?
  ;
  ('.' | '..' | 'sorry') comment?
  ;

  meth: method interest?
  ;
\end{rail}

\begin{descr}
\item [$\PROOF{m@1}$] refines the goal by proof method $m@1$; facts for
  forward chaining are passed if so indicated by $proof(chain)$ mode.
\item [$\QED{m@2}$] refines any remaining goals by proof method $m@2$ and
  concludes the sub-proof.  If the goal had been $\SHOWNAME$ (or $\THUSNAME$),
  some pending sub-goal is solved as well by the rule resulting from the
  result exported to the enclosing goal context.  Thus $\QEDNAME$ may fail for
  two reasons: either $m@2$ fails to solve all remaining goals completely, or
  the resulting rule does not resolve with any enclosing goal.  Debugging such
  a situation might involve temporarily changing $\SHOWNAME$ into $\HAVENAME$,
  or softening the local context by replacing $\ASSUMENAME$ by $\PRESUMENAME$.
\item [$\BYY{m@1}{m@2}$] is a \emph{terminal proof}; it abbreviates
  $\PROOF{m@1}~\QED{m@2}$, with automatic backtracking across both methods.
  Debugging an unsuccessful $\BYY{m@1}{m@2}$ commands might be done by simply
  expanding the abbreviation by hand; note that $\PROOF{m@1}$ is usually
  sufficient to see what is going wrong.
\item [``$\DDOT$''] is a \emph{default proof}; it abbreviates $\BY{default}$.
\item [``$\DOT$''] is a \emph{trivial proof}; it abbreviates
  $\BY{assumption}$.
\item [$\isarkeyword{sorry}$] is a \emph{fake proof}; provided that
  \texttt{quick_and_dirty} is enabled, $\isarkeyword{sorry}$ pretends to solve
  the goal without further ado.  Of course, the result is a fake theorem only,
  involving some oracle in its internal derivation object (this is indicated
  as $[!]$ in the printed result).  The main application of
  $\isarkeyword{sorry}$ is to support top-down proof development.
\end{descr}


\subsection{Improper proof steps}

The following commands emulate unstructured tactic scripts to some extent.
While these are anathema for writing proper Isar proof documents, they might
come in handy for exploring and debugging.

\indexisarcmd{apply}\indexisarcmd{then-apply}\indexisarcmd{back}
\begin{matharray}{rcl}
  \isarcmd{apply}^* & : & \isartrans{proof}{proof} \\
  \isarcmd{then_apply}^* & : & \isartrans{proof}{proof} \\
  \isarcmd{back}^* & : & \isartrans{proof}{proof} \\
\end{matharray}

\railalias{thenapply}{then\_apply}
\railterm{thenapply}

\begin{rail}
  'apply' method
  ;
  thenapply method
  ;
  'back'
  ;
\end{rail}

\begin{descr}
\item [$\isarkeyword{apply}~(m)$] applies proof method $m$ in the
  plain-old-tactic sense.  Facts for forward chaining are ignored.
\item [$\isarkeyword{then_apply}~(m)$] is similar to $\isarkeyword{apply}$,
  but observes the goal's facts.
\item [$\isarkeyword{back}$] does back-tracking over the result sequence of
  the latest proof command.\footnote{Unlike the ML function \texttt{back}
    \cite{isabelle-ref}, the Isar command does not search upwards for further
    branch points.} Basically, any proof command may return multiple results.
\end{descr}


\subsection{Term abbreviations}\label{sec:term-abbrev}

\indexisarcmd{let}
\begin{matharray}{rcl}
  \isarcmd{let} & : & \isartrans{proof(state)}{proof(state)} \\
  \isarkeyword{is} & : & syntax \\
\end{matharray}

Abbreviations may be either bound by explicit $\LET{p \equiv t}$ statements,
or by annotating assumptions or goal statements ($\ASSUMENAME$, $\SHOWNAME$
etc.) with a list of patterns $\IS{p@1 \dots p@n}$.  In both cases,
higher-order matching is applied to bind extra-logical text
variables\index{text variables}, which may be either of the form $\VVar{x}$
(token class \railtoken{textvar}, see \S\ref{sec:lex-syntax}) or nameless
dummies ``\texttt{_}'' (underscore).\index{dummy variables} Note that in the
$\LETNAME$ form the patterns occur on the left-hand side, while the $\ISNAME$
patterns are in postfix position.

Term abbreviations are quite different from actual local definitions as
introduced via $\DEFNAME$ (see \S\ref{sec:proof-context}).  The latter are
visible within the logic as actual equations, while abbreviations disappear
during the input process just after type checking.

\begin{rail}
  'let' ((term + 'as') '=' term comment? + 'and')
  ;  
\end{rail}

The syntax of $\ISNAME$ patterns follows \railnonterm{termpat} or
\railnonterm{proppat} (see \S\ref{sec:term-pats}).

\begin{descr}
\item [$\LET{\vec p = \vec t}$] binds any text variables in patters $\vec p$
  by simultaneous higher-order matching against terms $\vec t$.
\item [$\IS{\vec p}$] resembles $\LETNAME$, but matches $\vec p$ against the
  preceding statement.  Also note that $\ISNAME$ is not a separate command,
  but part of others (such as $\ASSUMENAME$, $\HAVENAME$ etc.).
\end{descr}

A few \emph{automatic} term abbreviations\index{automatic abbreviation} for
goals and facts are available as well.  For any open goal,
$\VVar{thesis_prop}$ refers to the full proposition (which may be a rule),
$\VVar{thesis_concl}$ to its (atomic) conclusion, and $\VVar{thesis}$ to its
object-logical statement.  The latter two abstract over any meta-level
parameters.

Facts (i.e.\ assumptions and finished goals) that have an application $f(x)$
as object-logic statement get $x$ bound to the special text variable
``$\dots$'' (three dots).  The canonical application of this feature are
calculational proofs (see \S\ref{sec:calculation}).


\subsection{Block structure}

\indexisarcmd{next}\indexisarcmd{\{\{}\indexisarcmd{\}\}}
\begin{matharray}{rcl}
  \isarcmd{next} & : & \isartrans{proof(state)}{proof(state)} \\
  \isarcmd{\{\{} & : & \isartrans{proof(state)}{proof(state)} \\
  \isarcmd{\}\}} & : & \isartrans{proof(state)}{proof(state)} \\
\end{matharray}

While Isar is inherently block-structured, opening and closing blocks is
mostly handled rather casually, with little explicit user-intervention.  Any
local goal statement automatically opens \emph{two} blocks, which are closed
again when concluding the sub-proof (by $\QEDNAME$ etc.).  Sections of
different context within a sub-proof are typically switched via
$\isarkeyword{next}$, which is just a single block-close followed by
block-open again.  Thus the effect of $\isarkeyword{next}$ is to reset the
proof context to that of the head of the sub-proof.  Note that there is no
goal focus involved here!

For slightly more advanced applications, there are explicit block parentheses
as well.  These typically achieve a strong forward style of reasoning.

\begin{descr}
\item [$\isarkeyword{next}$] switches to a fresh block within a sub-proof,
  resetting the context to the initial one.
\item [$\isarkeyword{\{\{}$ and $\isarkeyword{\}\}}$] explicitly open and
  close blocks.  Any current facts pass through $\isarkeyword{\{\{}$
  unchanged, while $\isarkeyword{\}\}}$ causes them to be \emph{exported} into
  the enclosing context.  Thus fixed variables are generalized, assumptions
  discharged, and local definitions eliminated.  There is no difference of
  $\ASSUMENAME$ and $\PRESUMENAME$ here.
\end{descr}


\section{Other commands}

\subsection{Diagnostics}

\indexisarcmd{typ}\indexisarcmd{term}\indexisarcmd{prop}\indexisarcmd{thm}
\begin{matharray}{rcl}
  \isarcmd{typ} & : & \isarkeep{theory~|~proof} \\
  \isarcmd{term} & : & \isarkeep{theory~|~proof} \\
  \isarcmd{prop} & : & \isarkeep{theory~|~proof} \\
  \isarcmd{thm} & : & \isarkeep{theory~|~proof} \\
\end{matharray}

These commands are not part of the actual Isabelle/Isar syntax, but assist
interactive development.  Also note that $undo$ does not apply here, since the
theory or proof configuration is not changed.

\begin{rail}
  'typ' type
  ;
  'term' term
  ;
  'prop' prop
  ;
  'thm' thmrefs
  ;
\end{rail}

\begin{descr}
\item [$\isarkeyword{typ}~\tau$, $\isarkeyword{term}~t$,
  $\isarkeyword{prop}~\phi$] read and print types / terms / propositions
  according to the current theory or proof context.
\item [$\isarkeyword{thm}~thms$] retrieves lists of theorems from the current
  theory or proof context.  Note that any attributes included in the theorem
  specifications are applied to a temporary context derived from the current
  theory or proof; the result is discarded, i.e.\ attributes involved in
  $thms$ only have a temporary effect.
\end{descr}


\subsection{System operations}

\indexisarcmd{cd}\indexisarcmd{pwd}\indexisarcmd{use-thy}\indexisarcmd{use-thy-only}
\indexisarcmd{update-thy}\indexisarcmd{update-thy-only}
\begin{matharray}{rcl}
  \isarcmd{cd} & : & \isarkeep{\cdot} \\
  \isarcmd{pwd} & : & \isarkeep{\cdot} \\
  \isarcmd{use_thy} & : & \isarkeep{\cdot} \\
  \isarcmd{use_thy_only} & : & \isarkeep{\cdot} \\
  \isarcmd{update_thy} & : & \isarkeep{\cdot} \\
  \isarcmd{update_thy_only} & : & \isarkeep{\cdot} \\
\end{matharray}

\begin{descr}
\item [$\isarkeyword{cd}~name$] changes the current directory of the Isabelle
  process.
\item [$\isarkeyword{pwd}~$] prints the current working directory.
\item [$\isarkeyword{use_thy}$, $\isarkeyword{use_thy_only}$,
  $\isarkeyword{update_thy}$, and $\isarkeyword{update_thy_only}$] load some
  theory given as $name$ argument.  These commands are exactly the same as the
  corresponding ML functions (see also \cite[\S1,\S6]{isabelle-ref}).  Note
  that both the ML and Isar versions may load new- and old-style theories
  alike.
\end{descr}

Note that these system commands are scarcely used when working with
Proof~General, since loading of theories is done fully automatic.


%%% Local Variables: 
%%% mode: latex
%%% TeX-master: "isar-ref"
%%% End: 
