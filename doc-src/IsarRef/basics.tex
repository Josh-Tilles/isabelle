
\chapter{Basic Concepts}\label{ch:basics}

Isabelle/Isar offers two main improvements over classic Isabelle:
\begin{enumerate}
\item A new \emph{theory format}, often referred to as ``new-style theories'',
  supporting interactive development with unlimited undo operation.
\item A formal \emph{proof language} language designed to support
  \emph{intelligible} semi-automated reasoning.  Rather than putting together
  tactic scripts, the author is enabled to express the reasoning in way that
  is close to mathematical practice.
\end{enumerate}

The Isar proof language is embedded into the new theory format as a proper
sub-language.  Proof mode is entered by stating some $\THEOREMNAME$ or
$\LEMMANAME$ at the theory levels, and left with the final end of proof.  Some
theory extension mechanisms require proof as well, such as the HOL
$\isarkeyword{typedef}$.

New-style theory files may still be associated with an ML file consisting of
plain old tactic scripts.  Generally, migration between the two formats is
made relatively easy, and users may start to benefit from interactive theory
development even before they have any idea of the Isar proof language.


\section{The Isar proof language}

This rather important section has not been written yet!  Refer to
\cite{Wenzel:1999:TPHOL} for the time being.

\subsection{Commands}

\subsubsection{Isar primitives}

\subsubsection{Derived elements}


\subsection{Methods}

\subsection{Attributes}


%%% Local Variables: 
%%% mode: latex
%%% TeX-master: "isar-ref"
%%% End: 
