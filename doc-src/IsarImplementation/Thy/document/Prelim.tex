%
\begin{isabellebody}%
\def\isabellecontext{Prelim}%
%
\isadelimtheory
%
\endisadelimtheory
%
\isatagtheory
\isacommand{theory}\isamarkupfalse%
\ Prelim\isanewline
\isakeyword{imports}\ Base\isanewline
\isakeyword{begin}%
\endisatagtheory
{\isafoldtheory}%
%
\isadelimtheory
%
\endisadelimtheory
%
\isamarkupchapter{Preliminaries%
}
\isamarkuptrue%
%
\isamarkupsection{Contexts \label{sec:context}%
}
\isamarkuptrue%
%
\begin{isamarkuptext}%
A logical context represents the background that is required for
  formulating statements and composing proofs.  It acts as a medium to
  produce formal content, depending on earlier material (declarations,
  results etc.).

  For example, derivations within the Isabelle/Pure logic can be
  described as a judgment \isa{{\isasymGamma}\ {\isasymturnstile}\isactrlsub {\isasymTheta}\ {\isasymphi}}, which means that a
  proposition \isa{{\isasymphi}} is derivable from hypotheses \isa{{\isasymGamma}}
  within the theory \isa{{\isasymTheta}}.  There are logical reasons for
  keeping \isa{{\isasymTheta}} and \isa{{\isasymGamma}} separate: theories can be
  liberal about supporting type constructors and schematic
  polymorphism of constants and axioms, while the inner calculus of
  \isa{{\isasymGamma}\ {\isasymturnstile}\ {\isasymphi}} is strictly limited to Simple Type Theory (with
  fixed type variables in the assumptions).

  \medskip Contexts and derivations are linked by the following key
  principles:

  \begin{itemize}

  \item Transfer: monotonicity of derivations admits results to be
  transferred into a \emph{larger} context, i.e.\ \isa{{\isasymGamma}\ {\isasymturnstile}\isactrlsub {\isasymTheta}\ {\isasymphi}} implies \isa{{\isasymGamma}{\isacharprime}\ {\isasymturnstile}\isactrlsub {\isasymTheta}\isactrlsub {\isacharprime}\ {\isasymphi}} for contexts \isa{{\isasymTheta}{\isacharprime}\ {\isasymsupseteq}\ {\isasymTheta}} and \isa{{\isasymGamma}{\isacharprime}\ {\isasymsupseteq}\ {\isasymGamma}}.

  \item Export: discharge of hypotheses admits results to be exported
  into a \emph{smaller} context, i.e.\ \isa{{\isasymGamma}{\isacharprime}\ {\isasymturnstile}\isactrlsub {\isasymTheta}\ {\isasymphi}}
  implies \isa{{\isasymGamma}\ {\isasymturnstile}\isactrlsub {\isasymTheta}\ {\isasymDelta}\ {\isasymLongrightarrow}\ {\isasymphi}} where \isa{{\isasymGamma}{\isacharprime}\ {\isasymsupseteq}\ {\isasymGamma}} and
  \isa{{\isasymDelta}\ {\isacharequal}\ {\isasymGamma}{\isacharprime}\ {\isacharminus}\ {\isasymGamma}}.  Note that \isa{{\isasymTheta}} remains unchanged here,
  only the \isa{{\isasymGamma}} part is affected.

  \end{itemize}

  \medskip By modeling the main characteristics of the primitive
  \isa{{\isasymTheta}} and \isa{{\isasymGamma}} above, and abstracting over any
  particular logical content, we arrive at the fundamental notions of
  \emph{theory context} and \emph{proof context} in Isabelle/Isar.
  These implement a certain policy to manage arbitrary \emph{context
  data}.  There is a strongly-typed mechanism to declare new kinds of
  data at compile time.

  The internal bootstrap process of Isabelle/Pure eventually reaches a
  stage where certain data slots provide the logical content of \isa{{\isasymTheta}} and \isa{{\isasymGamma}} sketched above, but this does not stop there!
  Various additional data slots support all kinds of mechanisms that
  are not necessarily part of the core logic.

  For example, there would be data for canonical introduction and
  elimination rules for arbitrary operators (depending on the
  object-logic and application), which enables users to perform
  standard proof steps implicitly (cf.\ the \isa{rule} method
  \cite{isabelle-isar-ref}).

  \medskip Thus Isabelle/Isar is able to bring forth more and more
  concepts successively.  In particular, an object-logic like
  Isabelle/HOL continues the Isabelle/Pure setup by adding specific
  components for automated reasoning (classical reasoner, tableau
  prover, structured induction etc.) and derived specification
  mechanisms (inductive predicates, recursive functions etc.).  All of
  this is ultimately based on the generic data management by theory
  and proof contexts introduced here.%
\end{isamarkuptext}%
\isamarkuptrue%
%
\isamarkupsubsection{Theory context \label{sec:context-theory}%
}
\isamarkuptrue%
%
\begin{isamarkuptext}%
A \emph{theory} is a data container with explicit name and
  unique identifier.  Theories are related by a (nominal) sub-theory
  relation, which corresponds to the dependency graph of the original
  construction; each theory is derived from a certain sub-graph of
  ancestor theories.  To this end, the system maintains a set of
  symbolic ``identification stamps'' within each theory.

  In order to avoid the full-scale overhead of explicit sub-theory
  identification of arbitrary intermediate stages, a theory is
  switched into \isa{draft} mode under certain circumstances.  A
  draft theory acts like a linear type, where updates invalidate
  earlier versions.  An invalidated draft is called \emph{stale}.

  The \isa{checkpoint} operation produces a safe stepping stone
  that will survive the next update without becoming stale: both the
  old and the new theory remain valid and are related by the
  sub-theory relation.  Checkpointing essentially recovers purely
  functional theory values, at the expense of some extra internal
  bookkeeping.

  The \isa{copy} operation produces an auxiliary version that has
  the same data content, but is unrelated to the original: updates of
  the copy do not affect the original, neither does the sub-theory
  relation hold.

  The \isa{merge} operation produces the least upper bound of two
  theories, which actually degenerates into absorption of one theory
  into the other (according to the nominal sub-theory relation).

  The \isa{begin} operation starts a new theory by importing
  several parent theories and entering a special mode of nameless
  incremental updates, until the final \isa{end} operation is
  performed.

  \medskip The example in \figref{fig:ex-theory} below shows a theory
  graph derived from \isa{Pure}, with theory \isa{Length}
  importing \isa{Nat} and \isa{List}.  The body of \isa{Length} consists of a sequence of updates, working mostly on
  drafts internally, while transaction boundaries of Isar top-level
  commands (\secref{sec:isar-toplevel}) are guaranteed to be safe
  checkpoints.

  \begin{figure}[htb]
  \begin{center}
  \begin{tabular}{rcccl}
        &            & \isa{Pure} \\
        &            & \isa{{\isasymdown}} \\
        &            & \isa{FOL} \\
        & $\swarrow$ &              & $\searrow$ & \\
  \isa{Nat} &    &              &            & \isa{List} \\
        & $\searrow$ &              & $\swarrow$ \\
        &            & \isa{Length} \\
        &            & \multicolumn{3}{l}{~~\hyperlink{keyword.imports}{\mbox{\isa{\isakeyword{imports}}}}} \\
        &            & \multicolumn{3}{l}{~~\hyperlink{keyword.begin}{\mbox{\isa{\isakeyword{begin}}}}} \\
        &            & $\vdots$~~ \\
        &            & \isa{{\isasymbullet}}~~ \\
        &            & $\vdots$~~ \\
        &            & \isa{{\isasymbullet}}~~ \\
        &            & $\vdots$~~ \\
        &            & \multicolumn{3}{l}{~~\hyperlink{command.end}{\mbox{\isa{\isacommand{end}}}}} \\
  \end{tabular}
  \caption{A theory definition depending on ancestors}\label{fig:ex-theory}
  \end{center}
  \end{figure}

  \medskip There is a separate notion of \emph{theory reference} for
  maintaining a live link to an evolving theory context: updates on
  drafts are propagated automatically.  Dynamic updating stops when
  the next \isa{checkpoint} is reached.

  Derived entities may store a theory reference in order to indicate
  the formal context from which they are derived.  This implicitly
  assumes monotonic reasoning, because the referenced context may
  become larger without further notice.%
\end{isamarkuptext}%
\isamarkuptrue%
%
\isadelimmlref
%
\endisadelimmlref
%
\isatagmlref
%
\begin{isamarkuptext}%
\begin{mldecls}
  \indexdef{}{ML type}{theory}\verb|type theory| \\
  \indexdef{}{ML}{Theory.eq\_thy}\verb|Theory.eq_thy: theory * theory -> bool| \\
  \indexdef{}{ML}{Theory.subthy}\verb|Theory.subthy: theory * theory -> bool| \\
  \indexdef{}{ML}{Theory.checkpoint}\verb|Theory.checkpoint: theory -> theory| \\
  \indexdef{}{ML}{Theory.copy}\verb|Theory.copy: theory -> theory| \\
  \indexdef{}{ML}{Theory.merge}\verb|Theory.merge: theory * theory -> theory| \\
  \indexdef{}{ML}{Theory.begin\_theory}\verb|Theory.begin_theory: string -> theory list -> theory| \\
  \indexdef{}{ML}{Theory.parents\_of}\verb|Theory.parents_of: theory -> theory list| \\
  \indexdef{}{ML}{Theory.ancestors\_of}\verb|Theory.ancestors_of: theory -> theory list| \\
  \end{mldecls}
  \begin{mldecls}
  \indexdef{}{ML type}{theory\_ref}\verb|type theory_ref| \\
  \indexdef{}{ML}{Theory.deref}\verb|Theory.deref: theory_ref -> theory| \\
  \indexdef{}{ML}{Theory.check\_thy}\verb|Theory.check_thy: theory -> theory_ref| \\
  \end{mldecls}

  \begin{description}

  \item Type \verb|theory| represents theory contexts.  This is
  essentially a linear type, with explicit runtime checking.
  Primitive theory operations destroy the original version, which then
  becomes ``stale''.  This can be prevented by explicit checkpointing,
  which the system does at least at the boundary of toplevel command
  transactions \secref{sec:isar-toplevel}.

  \item \verb|Theory.eq_thy|~\isa{{\isacharparenleft}thy\isactrlsub {\isadigit{1}}{\isacharcomma}\ thy\isactrlsub {\isadigit{2}}{\isacharparenright}} check strict
  identity of two theories.

  \item \verb|Theory.subthy|~\isa{{\isacharparenleft}thy\isactrlsub {\isadigit{1}}{\isacharcomma}\ thy\isactrlsub {\isadigit{2}}{\isacharparenright}} compares theories
  according to the intrinsic graph structure of the construction.
  This sub-theory relation is a nominal approximation of inclusion
  (\isa{{\isasymsubseteq}}) of the corresponding content (according to the
  semantics of the ML modules that implement the data).

  \item \verb|Theory.checkpoint|~\isa{thy} produces a safe
  stepping stone in the linear development of \isa{thy}.  This
  changes the old theory, but the next update will result in two
  related, valid theories.

  \item \verb|Theory.copy|~\isa{thy} produces a variant of \isa{thy} with the same data.  The copy is not related to the original,
  but the original is unchanged.

  \item \verb|Theory.merge|~\isa{{\isacharparenleft}thy\isactrlsub {\isadigit{1}}{\isacharcomma}\ thy\isactrlsub {\isadigit{2}}{\isacharparenright}} absorbs one theory
  into the other, without changing \isa{thy\isactrlsub {\isadigit{1}}} or \isa{thy\isactrlsub {\isadigit{2}}}.
  This version of ad-hoc theory merge fails for unrelated theories!

  \item \verb|Theory.begin_theory|~\isa{name\ parents} constructs
  a new theory based on the given parents.  This ML function is
  normally not invoked directly.

  \item \verb|Theory.parents_of|~\isa{thy} returns the direct
  ancestors of \isa{thy}.

  \item \verb|Theory.ancestors_of|~\isa{thy} returns all
  ancestors of \isa{thy} (not including \isa{thy} itself).

  \item Type \verb|theory_ref| represents a sliding reference to
  an always valid theory; updates on the original are propagated
  automatically.

  \item \verb|Theory.deref|~\isa{thy{\isacharunderscore}ref} turns a \verb|theory_ref| into an \verb|theory| value.  As the referenced
  theory evolves monotonically over time, later invocations of \verb|Theory.deref| may refer to a larger context.

  \item \verb|Theory.check_thy|~\isa{thy} produces a \verb|theory_ref| from a valid \verb|theory| value.

  \end{description}%
\end{isamarkuptext}%
\isamarkuptrue%
%
\endisatagmlref
{\isafoldmlref}%
%
\isadelimmlref
%
\endisadelimmlref
%
\isadelimmlantiq
%
\endisadelimmlantiq
%
\isatagmlantiq
%
\begin{isamarkuptext}%
\begin{matharray}{rcl}
  \indexdef{}{ML antiquotation}{theory}\hypertarget{ML antiquotation.theory}{\hyperlink{ML antiquotation.theory}{\mbox{\isa{theory}}}} & : & \isa{ML{\isacharunderscore}antiquotation} \\
  \indexdef{}{ML antiquotation}{theory\_ref}\hypertarget{ML antiquotation.theory-ref}{\hyperlink{ML antiquotation.theory-ref}{\mbox{\isa{theory{\isacharunderscore}ref}}}} & : & \isa{ML{\isacharunderscore}antiquotation} \\
  \end{matharray}

  \begin{rail}
  ('theory' | 'theory\_ref') nameref?
  ;
  \end{rail}

  \begin{description}

  \item \isa{{\isacharat}{\isacharbraceleft}theory{\isacharbraceright}} refers to the background theory of the
  current context --- as abstract value.

  \item \isa{{\isacharat}{\isacharbraceleft}theory\ A{\isacharbraceright}} refers to an explicitly named ancestor
  theory \isa{A} of the background theory of the current context
  --- as abstract value.

  \item \isa{{\isacharat}{\isacharbraceleft}theory{\isacharunderscore}ref{\isacharbraceright}} is similar to \isa{{\isacharat}{\isacharbraceleft}theory{\isacharbraceright}}, but
  produces a \verb|theory_ref| via \verb|Theory.check_thy| as
  explained above.

  \end{description}%
\end{isamarkuptext}%
\isamarkuptrue%
%
\endisatagmlantiq
{\isafoldmlantiq}%
%
\isadelimmlantiq
%
\endisadelimmlantiq
%
\isamarkupsubsection{Proof context \label{sec:context-proof}%
}
\isamarkuptrue%
%
\begin{isamarkuptext}%
A proof context is a container for pure data with a
  back-reference to the theory from which it is derived.  The \isa{init} operation creates a proof context from a given theory.
  Modifications to draft theories are propagated to the proof context
  as usual, but there is also an explicit \isa{transfer} operation
  to force resynchronization with more substantial updates to the
  underlying theory.

  Entities derived in a proof context need to record logical
  requirements explicitly, since there is no separate context
  identification or symbolic inclusion as for theories.  For example,
  hypotheses used in primitive derivations (cf.\ \secref{sec:thms})
  are recorded separately within the sequent \isa{{\isasymGamma}\ {\isasymturnstile}\ {\isasymphi}}, just to
  make double sure.  Results could still leak into an alien proof
  context due to programming errors, but Isabelle/Isar includes some
  extra validity checks in critical positions, notably at the end of a
  sub-proof.

  Proof contexts may be manipulated arbitrarily, although the common
  discipline is to follow block structure as a mental model: a given
  context is extended consecutively, and results are exported back
  into the original context.  Note that an Isar proof state models
  block-structured reasoning explicitly, using a stack of proof
  contexts internally.  For various technical reasons, the background
  theory of an Isar proof state must not be changed while the proof is
  still under construction!%
\end{isamarkuptext}%
\isamarkuptrue%
%
\isadelimmlref
%
\endisadelimmlref
%
\isatagmlref
%
\begin{isamarkuptext}%
\begin{mldecls}
  \indexdef{}{ML type}{Proof.context}\verb|type Proof.context| \\
  \indexdef{}{ML}{ProofContext.init\_global}\verb|ProofContext.init_global: theory -> Proof.context| \\
  \indexdef{}{ML}{ProofContext.theory\_of}\verb|ProofContext.theory_of: Proof.context -> theory| \\
  \indexdef{}{ML}{ProofContext.transfer}\verb|ProofContext.transfer: theory -> Proof.context -> Proof.context| \\
  \end{mldecls}

  \begin{description}

  \item Type \verb|Proof.context| represents proof contexts.
  Elements of this type are essentially pure values, with a sliding
  reference to the background theory.

  \item \verb|ProofContext.init_global|~\isa{thy} produces a proof context
  derived from \isa{thy}, initializing all data.

  \item \verb|ProofContext.theory_of|~\isa{ctxt} selects the
  background theory from \isa{ctxt}, dereferencing its internal
  \verb|theory_ref|.

  \item \verb|ProofContext.transfer|~\isa{thy\ ctxt} promotes the
  background theory of \isa{ctxt} to the super theory \isa{thy}.

  \end{description}%
\end{isamarkuptext}%
\isamarkuptrue%
%
\endisatagmlref
{\isafoldmlref}%
%
\isadelimmlref
%
\endisadelimmlref
%
\isadelimmlantiq
%
\endisadelimmlantiq
%
\isatagmlantiq
%
\begin{isamarkuptext}%
\begin{matharray}{rcl}
  \indexdef{}{ML antiquotation}{context}\hypertarget{ML antiquotation.context}{\hyperlink{ML antiquotation.context}{\mbox{\isa{context}}}} & : & \isa{ML{\isacharunderscore}antiquotation} \\
  \end{matharray}

  \begin{description}

  \item \isa{{\isacharat}{\isacharbraceleft}context{\isacharbraceright}} refers to \emph{the} context at
  compile-time --- as abstract value.  Independently of (local) theory
  or proof mode, this always produces a meaningful result.

  This is probably the most common antiquotation in interactive
  experimentation with ML inside Isar.

  \end{description}%
\end{isamarkuptext}%
\isamarkuptrue%
%
\endisatagmlantiq
{\isafoldmlantiq}%
%
\isadelimmlantiq
%
\endisadelimmlantiq
%
\isamarkupsubsection{Generic contexts \label{sec:generic-context}%
}
\isamarkuptrue%
%
\begin{isamarkuptext}%
A generic context is the disjoint sum of either a theory or proof
  context.  Occasionally, this enables uniform treatment of generic
  context data, typically extra-logical information.  Operations on
  generic contexts include the usual injections, partial selections,
  and combinators for lifting operations on either component of the
  disjoint sum.

  Moreover, there are total operations \isa{theory{\isacharunderscore}of} and \isa{proof{\isacharunderscore}of} to convert a generic context into either kind: a theory
  can always be selected from the sum, while a proof context might
  have to be constructed by an ad-hoc \isa{init} operation, which
  incurs a small runtime overhead.%
\end{isamarkuptext}%
\isamarkuptrue%
%
\isadelimmlref
%
\endisadelimmlref
%
\isatagmlref
%
\begin{isamarkuptext}%
\begin{mldecls}
  \indexdef{}{ML type}{Context.generic}\verb|type Context.generic| \\
  \indexdef{}{ML}{Context.theory\_of}\verb|Context.theory_of: Context.generic -> theory| \\
  \indexdef{}{ML}{Context.proof\_of}\verb|Context.proof_of: Context.generic -> Proof.context| \\
  \end{mldecls}

  \begin{description}

  \item Type \verb|Context.generic| is the direct sum of \verb|theory| and \verb|Proof.context|, with the datatype
  constructors \verb|Context.Theory| and \verb|Context.Proof|.

  \item \verb|Context.theory_of|~\isa{context} always produces a
  theory from the generic \isa{context}, using \verb|ProofContext.theory_of| as required.

  \item \verb|Context.proof_of|~\isa{context} always produces a
  proof context from the generic \isa{context}, using \verb|ProofContext.init_global| as required (note that this re-initializes the
  context data with each invocation).

  \end{description}%
\end{isamarkuptext}%
\isamarkuptrue%
%
\endisatagmlref
{\isafoldmlref}%
%
\isadelimmlref
%
\endisadelimmlref
%
\isamarkupsubsection{Context data \label{sec:context-data}%
}
\isamarkuptrue%
%
\begin{isamarkuptext}%
The main purpose of theory and proof contexts is to manage
  arbitrary (pure) data.  New data types can be declared incrementally
  at compile time.  There are separate declaration mechanisms for any
  of the three kinds of contexts: theory, proof, generic.

  \paragraph{Theory data} declarations need to implement the following
  SML signature:

  \medskip
  \begin{tabular}{ll}
  \isa{{\isasymtype}\ T} & representing type \\
  \isa{{\isasymval}\ empty{\isacharcolon}\ T} & empty default value \\
  \isa{{\isasymval}\ extend{\isacharcolon}\ T\ {\isasymrightarrow}\ T} & re-initialize on import \\
  \isa{{\isasymval}\ merge{\isacharcolon}\ T\ {\isasymtimes}\ T\ {\isasymrightarrow}\ T} & join on import \\
  \end{tabular}
  \medskip

  The \isa{empty} value acts as initial default for \emph{any}
  theory that does not declare actual data content; \isa{extend}
  is acts like a unitary version of \isa{merge}.

  Implementing \isa{merge} can be tricky.  The general idea is
  that \isa{merge\ {\isacharparenleft}data\isactrlsub {\isadigit{1}}{\isacharcomma}\ data\isactrlsub {\isadigit{2}}{\isacharparenright}} inserts those parts of \isa{data\isactrlsub {\isadigit{2}}} into \isa{data\isactrlsub {\isadigit{1}}} that are not yet present, while
  keeping the general order of things.  The \verb|Library.merge|
  function on plain lists may serve as canonical template.

  Particularly note that shared parts of the data must not be
  duplicated by naive concatenation, or a theory graph that is like a
  chain of diamonds would cause an exponential blowup!

  \paragraph{Proof context data} declarations need to implement the
  following SML signature:

  \medskip
  \begin{tabular}{ll}
  \isa{{\isasymtype}\ T} & representing type \\
  \isa{{\isasymval}\ init{\isacharcolon}\ theory\ {\isasymrightarrow}\ T} & produce initial value \\
  \end{tabular}
  \medskip

  The \isa{init} operation is supposed to produce a pure value
  from the given background theory and should be somehow
  ``immediate''.  Whenever a proof context is initialized, which
  happens frequently, the the system invokes the \isa{init}
  operation of \emph{all} theory data slots ever declared.  This also
  means that one needs to be economic about the total number of proof
  data declarations in the system, i.e.\ each ML module should declare
  at most one, sometimes two data slots for its internal use.
  Repeated data declarations to simulate a record type should be
  avoided!

  \paragraph{Generic data} provides a hybrid interface for both theory
  and proof data.  The \isa{init} operation for proof contexts is
  predefined to select the current data value from the background
  theory.

  \bigskip Any of the above data declarations over type \isa{T}
  result in an ML structure with the following signature:

  \medskip
  \begin{tabular}{ll}
  \isa{get{\isacharcolon}\ context\ {\isasymrightarrow}\ T} \\
  \isa{put{\isacharcolon}\ T\ {\isasymrightarrow}\ context\ {\isasymrightarrow}\ context} \\
  \isa{map{\isacharcolon}\ {\isacharparenleft}T\ {\isasymrightarrow}\ T{\isacharparenright}\ {\isasymrightarrow}\ context\ {\isasymrightarrow}\ context} \\
  \end{tabular}
  \medskip

  These other operations provide exclusive access for the particular
  kind of context (theory, proof, or generic context).  This interface
  observes the ML discipline for types and scopes: there is no other
  way to access the corresponding data slot of a context.  By keeping
  these operations private, an Isabelle/ML module may maintain
  abstract values authentically.%
\end{isamarkuptext}%
\isamarkuptrue%
%
\isadelimmlref
%
\endisadelimmlref
%
\isatagmlref
%
\begin{isamarkuptext}%
\begin{mldecls}
  \indexdef{}{ML functor}{Theory\_Data}\verb|functor Theory_Data| \\
  \indexdef{}{ML functor}{Proof\_Data}\verb|functor Proof_Data| \\
  \indexdef{}{ML functor}{Generic\_Data}\verb|functor Generic_Data| \\
  \end{mldecls}

  \begin{description}

  \item \verb|Theory_Data|\isa{{\isacharparenleft}spec{\isacharparenright}} declares data for
  type \verb|theory| according to the specification provided as
  argument structure.  The resulting structure provides data init and
  access operations as described above.

  \item \verb|Proof_Data|\isa{{\isacharparenleft}spec{\isacharparenright}} is analogous to
  \verb|Theory_Data| for type \verb|Proof.context|.

  \item \verb|Generic_Data|\isa{{\isacharparenleft}spec{\isacharparenright}} is analogous to
  \verb|Theory_Data| for type \verb|Context.generic|.

  \end{description}%
\end{isamarkuptext}%
\isamarkuptrue%
%
\endisatagmlref
{\isafoldmlref}%
%
\isadelimmlref
%
\endisadelimmlref
%
\isadelimmlex
%
\endisadelimmlex
%
\isatagmlex
%
\begin{isamarkuptext}%
The following artificial example demonstrates theory
  data: we maintain a set of terms that are supposed to be wellformed
  wrt.\ the enclosing theory.  The public interface is as follows:%
\end{isamarkuptext}%
\isamarkuptrue%
%
\endisatagmlex
{\isafoldmlex}%
%
\isadelimmlex
%
\endisadelimmlex
%
\isadelimML
%
\endisadelimML
%
\isatagML
\isacommand{ML}\isamarkupfalse%
\ {\isacharverbatimopen}\isanewline
\ \ signature\ WELLFORMED{\isacharunderscore}TERMS\ {\isacharequal}\isanewline
\ \ sig\isanewline
\ \ \ \ val\ get{\isacharcolon}\ theory\ {\isacharminus}{\isachargreater}\ term\ list\isanewline
\ \ \ \ val\ add{\isacharcolon}\ term\ {\isacharminus}{\isachargreater}\ theory\ {\isacharminus}{\isachargreater}\ theory\isanewline
\ \ end{\isacharsemicolon}\isanewline
{\isacharverbatimclose}%
\endisatagML
{\isafoldML}%
%
\isadelimML
%
\endisadelimML
%
\begin{isamarkuptext}%
The implementation uses private theory data internally, and
  only exposes an operation that involves explicit argument checking
  wrt.\ the given theory.%
\end{isamarkuptext}%
\isamarkuptrue%
%
\isadelimML
%
\endisadelimML
%
\isatagML
\isacommand{ML}\isamarkupfalse%
\ {\isacharverbatimopen}\isanewline
\ \ structure\ Wellformed{\isacharunderscore}Terms{\isacharcolon}\ WELLFORMED{\isacharunderscore}TERMS\ {\isacharequal}\isanewline
\ \ struct\isanewline
\isanewline
\ \ structure\ Terms\ {\isacharequal}\ Theory{\isacharunderscore}Data\isanewline
\ \ {\isacharparenleft}\isanewline
\ \ \ \ type\ T\ {\isacharequal}\ term\ Ord{\isacharunderscore}List{\isachardot}T{\isacharsemicolon}\isanewline
\ \ \ \ val\ empty\ {\isacharequal}\ {\isacharbrackleft}{\isacharbrackright}{\isacharsemicolon}\isanewline
\ \ \ \ val\ extend\ {\isacharequal}\ I{\isacharsemicolon}\isanewline
\ \ \ \ fun\ merge\ {\isacharparenleft}ts{\isadigit{1}}{\isacharcomma}\ ts{\isadigit{2}}{\isacharparenright}\ {\isacharequal}\isanewline
\ \ \ \ \ \ Ord{\isacharunderscore}List{\isachardot}union\ Term{\isacharunderscore}Ord{\isachardot}fast{\isacharunderscore}term{\isacharunderscore}ord\ ts{\isadigit{1}}\ ts{\isadigit{2}}{\isacharsemicolon}\isanewline
\ \ {\isacharparenright}{\isacharsemicolon}\isanewline
\isanewline
\ \ val\ get\ {\isacharequal}\ Terms{\isachardot}get{\isacharsemicolon}\isanewline
\isanewline
\ \ fun\ add\ raw{\isacharunderscore}t\ thy\ {\isacharequal}\isanewline
\ \ \ \ let\isanewline
\ \ \ \ \ \ val\ t\ {\isacharequal}\ Sign{\isachardot}cert{\isacharunderscore}term\ thy\ raw{\isacharunderscore}t{\isacharsemicolon}\isanewline
\ \ \ \ in\isanewline
\ \ \ \ \ \ Terms{\isachardot}map\ {\isacharparenleft}Ord{\isacharunderscore}List{\isachardot}insert\ Term{\isacharunderscore}Ord{\isachardot}fast{\isacharunderscore}term{\isacharunderscore}ord\ t{\isacharparenright}\ thy\isanewline
\ \ \ \ end{\isacharsemicolon}\isanewline
\isanewline
\ \ end{\isacharsemicolon}\isanewline
{\isacharverbatimclose}%
\endisatagML
{\isafoldML}%
%
\isadelimML
%
\endisadelimML
%
\begin{isamarkuptext}%
Type \verb|term Ord_List.T| is used for reasonably
  efficient representation of a set of terms: all operations are
  linear in the number of stored elements.  Here we assume that users
  of this module do not care about the declaration order, since that
  data structure forces its own arrangement of elements.

  Observe how the \verb|merge| operation joins the data slots of
  the two constituents: \verb|Ord_List.union| prevents duplication of
  common data from different branches, thus avoiding the danger of
  exponential blowup.  Plain list append etc.\ must never be used for
  theory data merges!

  \medskip Our intended invariant is achieved as follows:
  \begin{enumerate}

  \item \verb|Wellformed_Terms.add| only admits terms that have passed
  the \verb|Sign.cert_term| check of the given theory at that point.

  \item Wellformedness in the sense of \verb|Sign.cert_term| is
  monotonic wrt.\ the sub-theory relation.  So our data can move
  upwards in the hierarchy (via extension or merges), and maintain
  wellformedness without further checks.

  \end{enumerate}

  Note that all basic operations of the inference kernel (which
  includes \verb|Sign.cert_term|) observe this monotonicity principle,
  but other user-space tools don't.  For example, fully-featured
  type-inference via \verb|Syntax.check_term| (cf.\
  \secref{sec:term-check}) is not necessarily monotonic wrt.\ the
  background theory, since constraints of term constants can be
  modified by later declarations, for example.

  In most cases, user-space context data does not have to take such
  invariants too seriously.  The situation is different in the
  implementation of the inference kernel itself, which uses the very
  same data mechanisms for types, constants, axioms etc.%
\end{isamarkuptext}%
\isamarkuptrue%
%
\isamarkupsubsection{Configuration options \label{sec:config-options}%
}
\isamarkuptrue%
%
\begin{isamarkuptext}%
A \emph{configuration option} is a named optional value of
  some basic type (Boolean, integer, string) that is stored in the
  context.  It is a simple application of general context data
  (\secref{sec:context-data}) that is sufficiently common to justify
  customized setup, which includes some concrete declarations for
  end-users using existing notation for attributes (cf.\
  \secref{sec:attributes}).

  For example, the predefined configuration option \hyperlink{attribute.show-types}{\mbox{\isa{show{\isacharunderscore}types}}} controls output of explicit type constraints for
  variables in printed terms (cf.\ \secref{sec:read-print}).  Its
  value can be modified within Isar text like this:%
\end{isamarkuptext}%
\isamarkuptrue%
\isacommand{declare}\isamarkupfalse%
\ {\isacharbrackleft}{\isacharbrackleft}show{\isacharunderscore}types\ {\isacharequal}\ false{\isacharbrackright}{\isacharbrackright}\isanewline
\ \ %
\isamarkupcmt{declaration within (local) theory context%
}
\isanewline
\isanewline
\isacommand{example{\isacharunderscore}proof}\isamarkupfalse%
\isanewline
%
\isadelimproof
\ \ %
\endisadelimproof
%
\isatagproof
\isacommand{note}\isamarkupfalse%
\ {\isacharbrackleft}{\isacharbrackleft}show{\isacharunderscore}types\ {\isacharequal}\ true{\isacharbrackright}{\isacharbrackright}\isanewline
\ \ \ \ %
\isamarkupcmt{declaration within proof (forward mode)%
}
%
\endisatagproof
{\isafoldproof}%
%
\isadelimproof
\isanewline
%
\endisadelimproof
\ \ \isacommand{term}\isamarkupfalse%
\ x\isanewline
%
\isadelimproof
\isanewline
\ \ %
\endisadelimproof
%
\isatagproof
\isacommand{have}\isamarkupfalse%
\ {\isachardoublequoteopen}x\ {\isacharequal}\ x{\isachardoublequoteclose}\isanewline
\ \ \ \ \isacommand{using}\isamarkupfalse%
\ {\isacharbrackleft}{\isacharbrackleft}show{\isacharunderscore}types\ {\isacharequal}\ false{\isacharbrackright}{\isacharbrackright}\isanewline
\ \ \ \ \ \ %
\isamarkupcmt{declaration within proof (backward mode)%
}
\isanewline
\ \ \ \ \isacommand{{\isachardot}{\isachardot}}\isamarkupfalse%
\isanewline
\isacommand{qed}\isamarkupfalse%
%
\endisatagproof
{\isafoldproof}%
%
\isadelimproof
%
\endisadelimproof
%
\begin{isamarkuptext}%
Configuration options that are not set explicitly hold a
  default value that can depend on the application context.  This
  allows to retrieve the value from another slot within the context,
  or fall back on a global preference mechanism, for example.

  The operations to declare configuration options and get/map their
  values are modeled as direct replacements for historic global
  references, only that the context is made explicit.  This allows
  easy configuration of tools, without relying on the execution order
  as required for old-style mutable references.%
\end{isamarkuptext}%
\isamarkuptrue%
%
\isadelimmlref
%
\endisadelimmlref
%
\isatagmlref
%
\begin{isamarkuptext}%
\begin{mldecls}
  \indexdef{}{ML}{Config.get}\verb|Config.get: Proof.context -> 'a Config.T -> 'a| \\
  \indexdef{}{ML}{Config.map}\verb|Config.map: 'a Config.T -> ('a -> 'a) -> Proof.context -> Proof.context| \\
  \indexdef{}{ML}{Attrib.config\_bool}\verb|Attrib.config_bool: string -> (Context.generic -> bool) ->|\isasep\isanewline%
\verb|  bool Config.T * (theory -> theory)| \\
  \indexdef{}{ML}{Attrib.config\_int}\verb|Attrib.config_int: string -> (Context.generic -> int) ->|\isasep\isanewline%
\verb|  int Config.T * (theory -> theory)| \\
  \indexdef{}{ML}{Attrib.config\_string}\verb|Attrib.config_string: string -> (Context.generic -> string) ->|\isasep\isanewline%
\verb|  string Config.T * (theory -> theory)| \\
  \end{mldecls}

  \begin{description}

  \item \verb|Config.get|~\isa{ctxt\ config} gets the value of
  \isa{config} in the given context.

  \item \verb|Config.map|~\isa{config\ f\ ctxt} updates the context
  by updating the value of \isa{config}.

  \item \isa{{\isacharparenleft}config{\isacharcomma}\ setup{\isacharparenright}\ {\isacharequal}}~\verb|Attrib.config_bool|~\isa{name\ default} creates a named configuration option of type
  \verb|bool|, with the given \isa{default} depending on the
  application context.  The resulting \isa{config} can be used to
  get/map its value in a given context.  The \isa{setup} function
  needs to be applied to the theory initially, in order to make
  concrete declaration syntax available to the user.

  \item \verb|Attrib.config_int| and \verb|Attrib.config_string| work
  like \verb|Attrib.config_bool|, but for types \verb|int| and
  \verb|string|, respectively.

  \end{description}%
\end{isamarkuptext}%
\isamarkuptrue%
%
\endisatagmlref
{\isafoldmlref}%
%
\isadelimmlref
%
\endisadelimmlref
%
\isadelimmlex
%
\endisadelimmlex
%
\isatagmlex
%
\begin{isamarkuptext}%
The following example shows how to declare and use a
  Boolean configuration option called \isa{my{\isacharunderscore}flag} with constant
  default value \verb|false|.%
\end{isamarkuptext}%
\isamarkuptrue%
%
\endisatagmlex
{\isafoldmlex}%
%
\isadelimmlex
%
\endisadelimmlex
%
\isadelimML
%
\endisadelimML
%
\isatagML
\isacommand{ML}\isamarkupfalse%
\ {\isacharverbatimopen}\isanewline
\ \ val\ {\isacharparenleft}my{\isacharunderscore}flag{\isacharcomma}\ my{\isacharunderscore}flag{\isacharunderscore}setup{\isacharparenright}\ {\isacharequal}\isanewline
\ \ \ \ Attrib{\isachardot}config{\isacharunderscore}bool\ {\isachardoublequote}my{\isacharunderscore}flag{\isachardoublequote}\ {\isacharparenleft}K\ false{\isacharparenright}\isanewline
{\isacharverbatimclose}\isanewline
\isacommand{setup}\isamarkupfalse%
\ my{\isacharunderscore}flag{\isacharunderscore}setup%
\endisatagML
{\isafoldML}%
%
\isadelimML
%
\endisadelimML
%
\begin{isamarkuptext}%
Now the user can refer to \hyperlink{attribute.my-flag}{\mbox{\isa{my{\isacharunderscore}flag}}} in
  declarations, while we can retrieve the current value from the
  context via \verb|Config.get|.%
\end{isamarkuptext}%
\isamarkuptrue%
%
\isadelimML
%
\endisadelimML
%
\isatagML
\isacommand{ML{\isacharunderscore}val}\isamarkupfalse%
\ {\isacharverbatimopen}\ %
\isaantiq
assert%
\endisaantiq
\ {\isacharparenleft}Config{\isachardot}get\ %
\isaantiq
context%
\endisaantiq
\ my{\isacharunderscore}flag\ {\isacharequal}\ false{\isacharparenright}\ {\isacharverbatimclose}%
\endisatagML
{\isafoldML}%
%
\isadelimML
%
\endisadelimML
\isanewline
\isanewline
\isacommand{declare}\isamarkupfalse%
\ {\isacharbrackleft}{\isacharbrackleft}my{\isacharunderscore}flag\ {\isacharequal}\ true{\isacharbrackright}{\isacharbrackright}\isanewline
%
\isadelimML
\isanewline
%
\endisadelimML
%
\isatagML
\isacommand{ML{\isacharunderscore}val}\isamarkupfalse%
\ {\isacharverbatimopen}\ %
\isaantiq
assert%
\endisaantiq
\ {\isacharparenleft}Config{\isachardot}get\ %
\isaantiq
context%
\endisaantiq
\ my{\isacharunderscore}flag\ {\isacharequal}\ true{\isacharparenright}\ {\isacharverbatimclose}%
\endisatagML
{\isafoldML}%
%
\isadelimML
\isanewline
%
\endisadelimML
\isanewline
\isacommand{example{\isacharunderscore}proof}\isamarkupfalse%
\isanewline
%
\isadelimproof
\ \ %
\endisadelimproof
%
\isatagproof
\isacommand{{\isacharbraceleft}}\isamarkupfalse%
\isanewline
\ \ \ \ \isacommand{note}\isamarkupfalse%
\ {\isacharbrackleft}{\isacharbrackleft}my{\isacharunderscore}flag\ {\isacharequal}\ false{\isacharbrackright}{\isacharbrackright}%
\endisatagproof
{\isafoldproof}%
%
\isadelimproof
\isanewline
%
\endisadelimproof
%
\isadelimML
\ \ \ \ %
\endisadelimML
%
\isatagML
\isacommand{ML{\isacharunderscore}val}\isamarkupfalse%
\ {\isacharverbatimopen}\ %
\isaantiq
assert%
\endisaantiq
\ {\isacharparenleft}Config{\isachardot}get\ %
\isaantiq
context%
\endisaantiq
\ my{\isacharunderscore}flag\ {\isacharequal}\ false{\isacharparenright}\ {\isacharverbatimclose}%
\endisatagML
{\isafoldML}%
%
\isadelimML
\isanewline
%
\endisadelimML
%
\isadelimproof
\ \ %
\endisadelimproof
%
\isatagproof
\isacommand{{\isacharbraceright}}\isamarkupfalse%
%
\endisatagproof
{\isafoldproof}%
%
\isadelimproof
\isanewline
%
\endisadelimproof
%
\isadelimML
\ \ %
\endisadelimML
%
\isatagML
\isacommand{ML{\isacharunderscore}val}\isamarkupfalse%
\ {\isacharverbatimopen}\ %
\isaantiq
assert%
\endisaantiq
\ {\isacharparenleft}Config{\isachardot}get\ %
\isaantiq
context%
\endisaantiq
\ my{\isacharunderscore}flag\ {\isacharequal}\ true{\isacharparenright}\ {\isacharverbatimclose}%
\endisatagML
{\isafoldML}%
%
\isadelimML
\isanewline
%
\endisadelimML
%
\isadelimproof
%
\endisadelimproof
%
\isatagproof
\isacommand{qed}\isamarkupfalse%
%
\endisatagproof
{\isafoldproof}%
%
\isadelimproof
%
\endisadelimproof
%
\isamarkupsection{Names \label{sec:names}%
}
\isamarkuptrue%
%
\begin{isamarkuptext}%
In principle, a name is just a string, but there are various
  conventions for representing additional structure.  For example,
  ``\isa{Foo{\isachardot}bar{\isachardot}baz}'' is considered as a long name consisting of
  qualifier \isa{Foo{\isachardot}bar} and base name \isa{baz}.  The
  individual constituents of a name may have further substructure,
  e.g.\ the string ``\verb,\,\verb,<alpha>,'' encodes as a single
  symbol.

  \medskip Subsequently, we shall introduce specific categories of
  names.  Roughly speaking these correspond to logical entities as
  follows:
  \begin{itemize}

  \item Basic names (\secref{sec:basic-name}): free and bound
  variables.

  \item Indexed names (\secref{sec:indexname}): schematic variables.

  \item Long names (\secref{sec:long-name}): constants of any kind
  (type constructors, term constants, other concepts defined in user
  space).  Such entities are typically managed via name spaces
  (\secref{sec:name-space}).

  \end{itemize}%
\end{isamarkuptext}%
\isamarkuptrue%
%
\isamarkupsubsection{Strings of symbols \label{sec:symbols}%
}
\isamarkuptrue%
%
\begin{isamarkuptext}%
A \emph{symbol} constitutes the smallest textual unit in
  Isabelle --- raw ML characters are normally not encountered at all!
  Isabelle strings consist of a sequence of symbols, represented as a
  packed string or an exploded list of strings.  Each symbol is in
  itself a small string, which has either one of the following forms:

  \begin{enumerate}

  \item a single ASCII character ``\isa{c}'', for example
  ``\verb,a,'',

  \item a codepoint according to UTF8 (non-ASCII byte sequence),

  \item a regular symbol ``\verb,\,\verb,<,\isa{ident}\verb,>,'',
  for example ``\verb,\,\verb,<alpha>,'',

  \item a control symbol ``\verb,\,\verb,<^,\isa{ident}\verb,>,'',
  for example ``\verb,\,\verb,<^bold>,'',

  \item a raw symbol ``\verb,\,\verb,<^raw:,\isa{text}\verb,>,''
  where \isa{text} consists of printable characters excluding
  ``\verb,.,'' and ``\verb,>,'', for example
  ``\verb,\,\verb,<^raw:$\sum_{i = 1}^n$>,'',

  \item a numbered raw control symbol ``\verb,\,\verb,<^raw,\isa{n}\verb,>, where \isa{n} consists of digits, for example
  ``\verb,\,\verb,<^raw42>,''.

  \end{enumerate}

  The \isa{ident} syntax for symbol names is \isa{letter\ {\isacharparenleft}letter\ {\isacharbar}\ digit{\isacharparenright}\isactrlsup {\isacharasterisk}}, where \isa{letter\ {\isacharequal}\ A{\isachardot}{\isachardot}Za{\isachardot}{\isachardot}z} and \isa{digit\ {\isacharequal}\ {\isadigit{0}}{\isachardot}{\isachardot}{\isadigit{9}}}.  There are infinitely many regular symbols and
  control symbols, but a fixed collection of standard symbols is
  treated specifically.  For example, ``\verb,\,\verb,<alpha>,'' is
  classified as a letter, which means it may occur within regular
  Isabelle identifiers.

  The character set underlying Isabelle symbols is 7-bit ASCII, but
  8-bit character sequences are passed-through unchanged.  Unicode/UCS
  data in UTF-8 encoding is processed in a non-strict fashion, such
  that well-formed code sequences are recognized
  accordingly.\footnote{Note that ISO-Latin-1 differs from UTF-8 only
  in some special punctuation characters that even have replacements
  within the standard collection of Isabelle symbols.  Text consisting
  of ASCII plus accented letters can be processed in either encoding.}
  Unicode provides its own collection of mathematical symbols, but
  within the core Isabelle/ML world there is no link to the standard
  collection of Isabelle regular symbols.

  \medskip Output of Isabelle symbols depends on the print mode
  (\secref{print-mode}).  For example, the standard {\LaTeX} setup of
  the Isabelle document preparation system would present
  ``\verb,\,\verb,<alpha>,'' as \isa{{\isasymalpha}}, and
  ``\verb,\,\verb,<^bold>,\verb,\,\verb,<alpha>,'' as \isa{\isactrlbold {\isasymalpha}}.  On-screen rendering usually works by mapping a finite
  subset of Isabelle symbols to suitable Unicode characters.%
\end{isamarkuptext}%
\isamarkuptrue%
%
\isadelimmlref
%
\endisadelimmlref
%
\isatagmlref
%
\begin{isamarkuptext}%
\begin{mldecls}
  \indexdef{}{ML type}{Symbol.symbol}\verb|type Symbol.symbol = string| \\
  \indexdef{}{ML}{Symbol.explode}\verb|Symbol.explode: string -> Symbol.symbol list| \\
  \indexdef{}{ML}{Symbol.is\_letter}\verb|Symbol.is_letter: Symbol.symbol -> bool| \\
  \indexdef{}{ML}{Symbol.is\_digit}\verb|Symbol.is_digit: Symbol.symbol -> bool| \\
  \indexdef{}{ML}{Symbol.is\_quasi}\verb|Symbol.is_quasi: Symbol.symbol -> bool| \\
  \indexdef{}{ML}{Symbol.is\_blank}\verb|Symbol.is_blank: Symbol.symbol -> bool| \\
  \end{mldecls}
  \begin{mldecls}
  \indexdef{}{ML type}{Symbol.sym}\verb|type Symbol.sym| \\
  \indexdef{}{ML}{Symbol.decode}\verb|Symbol.decode: Symbol.symbol -> Symbol.sym| \\
  \end{mldecls}

  \begin{description}

  \item Type \verb|Symbol.symbol| represents individual Isabelle
  symbols.

  \item \verb|Symbol.explode|~\isa{str} produces a symbol list
  from the packed form.  This function supersedes \verb|String.explode| for virtually all purposes of manipulating text in
  Isabelle!\footnote{The runtime overhead for exploded strings is
  mainly that of the list structure: individual symbols that happen to
  be a singleton string do not require extra memory in Poly/ML.}

  \item \verb|Symbol.is_letter|, \verb|Symbol.is_digit|, \verb|Symbol.is_quasi|, \verb|Symbol.is_blank| classify standard
  symbols according to fixed syntactic conventions of Isabelle, cf.\
  \cite{isabelle-isar-ref}.

  \item Type \verb|Symbol.sym| is a concrete datatype that
  represents the different kinds of symbols explicitly, with
  constructors \verb|Symbol.Char|, \verb|Symbol.Sym|, \verb|Symbol.UTF8|, \verb|Symbol.Ctrl|, \verb|Symbol.Raw|.

  \item \verb|Symbol.decode| converts the string representation of a
  symbol into the datatype version.

  \end{description}

  \paragraph{Historical note.} In the original SML90 standard the
  primitive ML type \verb|char| did not exists, and the basic \verb|explode: string -> string list| operation would produce a list of
  singleton strings as in Isabelle/ML today.  When SML97 came out,
  Isabelle did not adopt its slightly anachronistic 8-bit characters,
  but the idea of exploding a string into a list of small strings was
  extended to ``symbols'' as explained above.  Thus Isabelle sources
  can refer to an infinite store of user-defined symbols, without
  having to worry about the multitude of Unicode encodings.%
\end{isamarkuptext}%
\isamarkuptrue%
%
\endisatagmlref
{\isafoldmlref}%
%
\isadelimmlref
%
\endisadelimmlref
%
\isamarkupsubsection{Basic names \label{sec:basic-name}%
}
\isamarkuptrue%
%
\begin{isamarkuptext}%
A \emph{basic name} essentially consists of a single Isabelle
  identifier.  There are conventions to mark separate classes of basic
  names, by attaching a suffix of underscores: one underscore means
  \emph{internal name}, two underscores means \emph{Skolem name},
  three underscores means \emph{internal Skolem name}.

  For example, the basic name \isa{foo} has the internal version
  \isa{foo{\isacharunderscore}}, with Skolem versions \isa{foo{\isacharunderscore}{\isacharunderscore}} and \isa{foo{\isacharunderscore}{\isacharunderscore}{\isacharunderscore}}, respectively.

  These special versions provide copies of the basic name space, apart
  from anything that normally appears in the user text.  For example,
  system generated variables in Isar proof contexts are usually marked
  as internal, which prevents mysterious names like \isa{xaa} to
  appear in human-readable text.

  \medskip Manipulating binding scopes often requires on-the-fly
  renamings.  A \emph{name context} contains a collection of already
  used names.  The \isa{declare} operation adds names to the
  context.

  The \isa{invents} operation derives a number of fresh names from
  a given starting point.  For example, the first three names derived
  from \isa{a} are \isa{a}, \isa{b}, \isa{c}.

  The \isa{variants} operation produces fresh names by
  incrementing tentative names as base-26 numbers (with digits \isa{a{\isachardot}{\isachardot}z}) until all clashes are resolved.  For example, name \isa{foo} results in variants \isa{fooa}, \isa{foob}, \isa{fooc}, \dots, \isa{fooaa}, \isa{fooab} etc.; each renaming
  step picks the next unused variant from this sequence.%
\end{isamarkuptext}%
\isamarkuptrue%
%
\isadelimmlref
%
\endisadelimmlref
%
\isatagmlref
%
\begin{isamarkuptext}%
\begin{mldecls}
  \indexdef{}{ML}{Name.internal}\verb|Name.internal: string -> string| \\
  \indexdef{}{ML}{Name.skolem}\verb|Name.skolem: string -> string| \\
  \end{mldecls}
  \begin{mldecls}
  \indexdef{}{ML type}{Name.context}\verb|type Name.context| \\
  \indexdef{}{ML}{Name.context}\verb|Name.context: Name.context| \\
  \indexdef{}{ML}{Name.declare}\verb|Name.declare: string -> Name.context -> Name.context| \\
  \indexdef{}{ML}{Name.invents}\verb|Name.invents: Name.context -> string -> int -> string list| \\
  \indexdef{}{ML}{Name.variants}\verb|Name.variants: string list -> Name.context -> string list * Name.context| \\
  \end{mldecls}
  \begin{mldecls}
  \indexdef{}{ML}{Variable.names\_of}\verb|Variable.names_of: Proof.context -> Name.context| \\
  \end{mldecls}

  \begin{description}

  \item \verb|Name.internal|~\isa{name} produces an internal name
  by adding one underscore.

  \item \verb|Name.skolem|~\isa{name} produces a Skolem name by
  adding two underscores.

  \item Type \verb|Name.context| represents the context of already
  used names; the initial value is \verb|Name.context|.

  \item \verb|Name.declare|~\isa{name} enters a used name into the
  context.

  \item \verb|Name.invents|~\isa{context\ name\ n} produces \isa{n} fresh names derived from \isa{name}.

  \item \verb|Name.variants|~\isa{names\ context} produces fresh
  variants of \isa{names}; the result is entered into the context.

  \item \verb|Variable.names_of|~\isa{ctxt} retrieves the context
  of declared type and term variable names.  Projecting a proof
  context down to a primitive name context is occasionally useful when
  invoking lower-level operations.  Regular management of ``fresh
  variables'' is done by suitable operations of structure \verb|Variable|, which is also able to provide an official status of
  ``locally fixed variable'' within the logical environment (cf.\
  \secref{sec:variables}).

  \end{description}%
\end{isamarkuptext}%
\isamarkuptrue%
%
\endisatagmlref
{\isafoldmlref}%
%
\isadelimmlref
%
\endisadelimmlref
%
\isadelimmlex
%
\endisadelimmlex
%
\isatagmlex
%
\begin{isamarkuptext}%
The following simple examples demonstrate how to produce
  fresh names from the initial \verb|Name.context|.%
\end{isamarkuptext}%
\isamarkuptrue%
%
\endisatagmlex
{\isafoldmlex}%
%
\isadelimmlex
%
\endisadelimmlex
%
\isadelimML
%
\endisadelimML
%
\isatagML
\isacommand{ML}\isamarkupfalse%
\ {\isacharverbatimopen}\isanewline
\ \ val\ list{\isadigit{1}}\ {\isacharequal}\ Name{\isachardot}invents\ Name{\isachardot}context\ {\isachardoublequote}a{\isachardoublequote}\ {\isadigit{5}}{\isacharsemicolon}\isanewline
\ \ %
\isaantiq
assert%
\endisaantiq
\ {\isacharparenleft}list{\isadigit{1}}\ {\isacharequal}\ {\isacharbrackleft}{\isachardoublequote}a{\isachardoublequote}{\isacharcomma}\ {\isachardoublequote}b{\isachardoublequote}{\isacharcomma}\ {\isachardoublequote}c{\isachardoublequote}{\isacharcomma}\ {\isachardoublequote}d{\isachardoublequote}{\isacharcomma}\ {\isachardoublequote}e{\isachardoublequote}{\isacharbrackright}{\isacharparenright}{\isacharsemicolon}\isanewline
\isanewline
\ \ val\ list{\isadigit{2}}\ {\isacharequal}\isanewline
\ \ \ \ {\isacharhash}{\isadigit{1}}\ {\isacharparenleft}Name{\isachardot}variants\ {\isacharbrackleft}{\isachardoublequote}x{\isachardoublequote}{\isacharcomma}\ {\isachardoublequote}x{\isachardoublequote}{\isacharcomma}\ {\isachardoublequote}a{\isachardoublequote}{\isacharcomma}\ {\isachardoublequote}a{\isachardoublequote}{\isacharcomma}\ {\isachardoublequote}{\isacharprime}a{\isachardoublequote}{\isacharcomma}\ {\isachardoublequote}{\isacharprime}a{\isachardoublequote}{\isacharbrackright}\ Name{\isachardot}context{\isacharparenright}{\isacharsemicolon}\isanewline
\ \ %
\isaantiq
assert%
\endisaantiq
\ {\isacharparenleft}list{\isadigit{2}}\ {\isacharequal}\ {\isacharbrackleft}{\isachardoublequote}x{\isachardoublequote}{\isacharcomma}\ {\isachardoublequote}xa{\isachardoublequote}{\isacharcomma}\ {\isachardoublequote}a{\isachardoublequote}{\isacharcomma}\ {\isachardoublequote}aa{\isachardoublequote}{\isacharcomma}\ {\isachardoublequote}{\isacharprime}a{\isachardoublequote}{\isacharcomma}\ {\isachardoublequote}{\isacharprime}aa{\isachardoublequote}{\isacharbrackright}{\isacharparenright}{\isacharsemicolon}\isanewline
{\isacharverbatimclose}%
\endisatagML
{\isafoldML}%
%
\isadelimML
%
\endisadelimML
%
\begin{isamarkuptext}%
\medskip The same works reletively to the formal context as
  follows.%
\end{isamarkuptext}%
\isamarkuptrue%
\isacommand{locale}\isamarkupfalse%
\ ex\ {\isacharequal}\ \isakeyword{fixes}\ a\ b\ c\ {\isacharcolon}{\isacharcolon}\ {\isacharprime}a\isanewline
\isakeyword{begin}\isanewline
%
\isadelimML
\isanewline
%
\endisadelimML
%
\isatagML
\isacommand{ML}\isamarkupfalse%
\ {\isacharverbatimopen}\isanewline
\ \ val\ names\ {\isacharequal}\ Variable{\isachardot}names{\isacharunderscore}of\ %
\isaantiq
context%
\endisaantiq
{\isacharsemicolon}\isanewline
\isanewline
\ \ val\ list{\isadigit{1}}\ {\isacharequal}\ Name{\isachardot}invents\ names\ {\isachardoublequote}a{\isachardoublequote}\ {\isadigit{5}}{\isacharsemicolon}\isanewline
\ \ %
\isaantiq
assert%
\endisaantiq
\ {\isacharparenleft}list{\isadigit{1}}\ {\isacharequal}\ {\isacharbrackleft}{\isachardoublequote}d{\isachardoublequote}{\isacharcomma}\ {\isachardoublequote}e{\isachardoublequote}{\isacharcomma}\ {\isachardoublequote}f{\isachardoublequote}{\isacharcomma}\ {\isachardoublequote}g{\isachardoublequote}{\isacharcomma}\ {\isachardoublequote}h{\isachardoublequote}{\isacharbrackright}{\isacharparenright}{\isacharsemicolon}\isanewline
\isanewline
\ \ val\ list{\isadigit{2}}\ {\isacharequal}\isanewline
\ \ \ \ {\isacharhash}{\isadigit{1}}\ {\isacharparenleft}Name{\isachardot}variants\ {\isacharbrackleft}{\isachardoublequote}x{\isachardoublequote}{\isacharcomma}\ {\isachardoublequote}x{\isachardoublequote}{\isacharcomma}\ {\isachardoublequote}a{\isachardoublequote}{\isacharcomma}\ {\isachardoublequote}a{\isachardoublequote}{\isacharcomma}\ {\isachardoublequote}{\isacharprime}a{\isachardoublequote}{\isacharcomma}\ {\isachardoublequote}{\isacharprime}a{\isachardoublequote}{\isacharbrackright}\ names{\isacharparenright}{\isacharsemicolon}\isanewline
\ \ %
\isaantiq
assert%
\endisaantiq
\ {\isacharparenleft}list{\isadigit{2}}\ {\isacharequal}\ {\isacharbrackleft}{\isachardoublequote}x{\isachardoublequote}{\isacharcomma}\ {\isachardoublequote}xa{\isachardoublequote}{\isacharcomma}\ {\isachardoublequote}aa{\isachardoublequote}{\isacharcomma}\ {\isachardoublequote}ab{\isachardoublequote}{\isacharcomma}\ {\isachardoublequote}{\isacharprime}aa{\isachardoublequote}{\isacharcomma}\ {\isachardoublequote}{\isacharprime}ab{\isachardoublequote}{\isacharbrackright}{\isacharparenright}{\isacharsemicolon}\isanewline
{\isacharverbatimclose}%
\endisatagML
{\isafoldML}%
%
\isadelimML
\isanewline
%
\endisadelimML
\isanewline
\isacommand{end}\isamarkupfalse%
%
\isamarkupsubsection{Indexed names \label{sec:indexname}%
}
\isamarkuptrue%
%
\begin{isamarkuptext}%
An \emph{indexed name} (or \isa{indexname}) is a pair of a basic
  name and a natural number.  This representation allows efficient
  renaming by incrementing the second component only.  The canonical
  way to rename two collections of indexnames apart from each other is
  this: determine the maximum index \isa{maxidx} of the first
  collection, then increment all indexes of the second collection by
  \isa{maxidx\ {\isacharplus}\ {\isadigit{1}}}; the maximum index of an empty collection is
  \isa{{\isacharminus}{\isadigit{1}}}.

  Occasionally, basic names are injected into the same pair type of
  indexed names: then \isa{{\isacharparenleft}x{\isacharcomma}\ {\isacharminus}{\isadigit{1}}{\isacharparenright}} is used to encode the basic
  name \isa{x}.

  \medskip Isabelle syntax observes the following rules for
  representing an indexname \isa{{\isacharparenleft}x{\isacharcomma}\ i{\isacharparenright}} as a packed string:

  \begin{itemize}

  \item \isa{{\isacharquery}x} if \isa{x} does not end with a digit and \isa{i\ {\isacharequal}\ {\isadigit{0}}},

  \item \isa{{\isacharquery}xi} if \isa{x} does not end with a digit,

  \item \isa{{\isacharquery}x{\isachardot}i} otherwise.

  \end{itemize}

  Indexnames may acquire large index numbers after several maxidx
  shifts have been applied.  Results are usually normalized towards
  \isa{{\isadigit{0}}} at certain checkpoints, notably at the end of a proof.
  This works by producing variants of the corresponding basic name
  components.  For example, the collection \isa{{\isacharquery}x{\isadigit{1}}{\isacharcomma}\ {\isacharquery}x{\isadigit{7}}{\isacharcomma}\ {\isacharquery}x{\isadigit{4}}{\isadigit{2}}}
  becomes \isa{{\isacharquery}x{\isacharcomma}\ {\isacharquery}xa{\isacharcomma}\ {\isacharquery}xb}.%
\end{isamarkuptext}%
\isamarkuptrue%
%
\isadelimmlref
%
\endisadelimmlref
%
\isatagmlref
%
\begin{isamarkuptext}%
\begin{mldecls}
  \indexdef{}{ML type}{indexname}\verb|type indexname = string * int| \\
  \end{mldecls}

  \begin{description}

  \item Type \verb|indexname| represents indexed names.  This is
  an abbreviation for \verb|string * int|.  The second component
  is usually non-negative, except for situations where \isa{{\isacharparenleft}x{\isacharcomma}\ {\isacharminus}{\isadigit{1}}{\isacharparenright}} is used to inject basic names into this type.  Other negative
  indexes should not be used.

  \end{description}%
\end{isamarkuptext}%
\isamarkuptrue%
%
\endisatagmlref
{\isafoldmlref}%
%
\isadelimmlref
%
\endisadelimmlref
%
\isamarkupsubsection{Long names \label{sec:long-name}%
}
\isamarkuptrue%
%
\begin{isamarkuptext}%
A \emph{long name} consists of a sequence of non-empty name
  components.  The packed representation uses a dot as separator, as
  in ``\isa{A{\isachardot}b{\isachardot}c}''.  The last component is called \emph{base
  name}, the remaining prefix is called \emph{qualifier} (which may be
  empty).  The qualifier can be understood as the access path to the
  named entity while passing through some nested block-structure,
  although our free-form long names do not really enforce any strict
  discipline.

  For example, an item named ``\isa{A{\isachardot}b{\isachardot}c}'' may be understood as
  a local entity \isa{c}, within a local structure \isa{b},
  within a global structure \isa{A}.  In practice, long names
  usually represent 1--3 levels of qualification.  User ML code should
  not make any assumptions about the particular structure of long
  names!

  The empty name is commonly used as an indication of unnamed
  entities, or entities that are not entered into the corresponding
  name space, whenever this makes any sense.  The basic operations on
  long names map empty names again to empty names.%
\end{isamarkuptext}%
\isamarkuptrue%
%
\isadelimmlref
%
\endisadelimmlref
%
\isatagmlref
%
\begin{isamarkuptext}%
\begin{mldecls}
  \indexdef{}{ML}{Long\_Name.base\_name}\verb|Long_Name.base_name: string -> string| \\
  \indexdef{}{ML}{Long\_Name.qualifier}\verb|Long_Name.qualifier: string -> string| \\
  \indexdef{}{ML}{Long\_Name.append}\verb|Long_Name.append: string -> string -> string| \\
  \indexdef{}{ML}{Long\_Name.implode}\verb|Long_Name.implode: string list -> string| \\
  \indexdef{}{ML}{Long\_Name.explode}\verb|Long_Name.explode: string -> string list| \\
  \end{mldecls}

  \begin{description}

  \item \verb|Long_Name.base_name|~\isa{name} returns the base name
  of a long name.

  \item \verb|Long_Name.qualifier|~\isa{name} returns the qualifier
  of a long name.

  \item \verb|Long_Name.append|~\isa{name\isactrlisub {\isadigit{1}}\ name\isactrlisub {\isadigit{2}}} appends two long
  names.

  \item \verb|Long_Name.implode|~\isa{names} and \verb|Long_Name.explode|~\isa{name} convert between the packed string
  representation and the explicit list form of long names.

  \end{description}%
\end{isamarkuptext}%
\isamarkuptrue%
%
\endisatagmlref
{\isafoldmlref}%
%
\isadelimmlref
%
\endisadelimmlref
%
\isamarkupsubsection{Name spaces \label{sec:name-space}%
}
\isamarkuptrue%
%
\begin{isamarkuptext}%
A \isa{name\ space} manages a collection of long names,
  together with a mapping between partially qualified external names
  and fully qualified internal names (in both directions).  Note that
  the corresponding \isa{intern} and \isa{extern} operations
  are mostly used for parsing and printing only!  The \isa{declare} operation augments a name space according to the accesses
  determined by a given binding, and a naming policy from the context.

  \medskip A \isa{binding} specifies details about the prospective
  long name of a newly introduced formal entity.  It consists of a
  base name, prefixes for qualification (separate ones for system
  infrastructure and user-space mechanisms), a slot for the original
  source position, and some additional flags.

  \medskip A \isa{naming} provides some additional details for
  producing a long name from a binding.  Normally, the naming is
  implicit in the theory or proof context.  The \isa{full}
  operation (and its variants for different context types) produces a
  fully qualified internal name to be entered into a name space.  The
  main equation of this ``chemical reaction'' when binding new
  entities in a context is as follows:

  \medskip
  \begin{tabular}{l}
  \isa{binding\ {\isacharplus}\ naming\ {\isasymlongrightarrow}\ long\ name\ {\isacharplus}\ name\ space\ accesses}
  \end{tabular}

  \bigskip As a general principle, there is a separate name space for
  each kind of formal entity, e.g.\ fact, logical constant, type
  constructor, type class.  It is usually clear from the occurrence in
  concrete syntax (or from the scope) which kind of entity a name
  refers to.  For example, the very same name \isa{c} may be used
  uniformly for a constant, type constructor, and type class.

  There are common schemes to name derived entities systematically
  according to the name of the main logical entity involved, e.g.\
  fact \isa{c{\isachardot}intro} for a canonical introduction rule related to
  constant \isa{c}.  This technique of mapping names from one
  space into another requires some care in order to avoid conflicts.
  In particular, theorem names derived from a type constructor or type
  class should get an additional suffix in addition to the usual
  qualification.  This leads to the following conventions for derived
  names:

  \medskip
  \begin{tabular}{ll}
  logical entity & fact name \\\hline
  constant \isa{c} & \isa{c{\isachardot}intro} \\
  type \isa{c} & \isa{c{\isacharunderscore}type{\isachardot}intro} \\
  class \isa{c} & \isa{c{\isacharunderscore}class{\isachardot}intro} \\
  \end{tabular}%
\end{isamarkuptext}%
\isamarkuptrue%
%
\isadelimmlref
%
\endisadelimmlref
%
\isatagmlref
%
\begin{isamarkuptext}%
\begin{mldecls}
  \indexdef{}{ML type}{binding}\verb|type binding| \\
  \indexdef{}{ML}{Binding.empty}\verb|Binding.empty: binding| \\
  \indexdef{}{ML}{Binding.name}\verb|Binding.name: string -> binding| \\
  \indexdef{}{ML}{Binding.qualify}\verb|Binding.qualify: bool -> string -> binding -> binding| \\
  \indexdef{}{ML}{Binding.prefix}\verb|Binding.prefix: bool -> string -> binding -> binding| \\
  \indexdef{}{ML}{Binding.conceal}\verb|Binding.conceal: binding -> binding| \\
  \indexdef{}{ML}{Binding.str\_of}\verb|Binding.str_of: binding -> string| \\
  \end{mldecls}
  \begin{mldecls}
  \indexdef{}{ML type}{Name\_Space.naming}\verb|type Name_Space.naming| \\
  \indexdef{}{ML}{Name\_Space.default\_naming}\verb|Name_Space.default_naming: Name_Space.naming| \\
  \indexdef{}{ML}{Name\_Space.add\_path}\verb|Name_Space.add_path: string -> Name_Space.naming -> Name_Space.naming| \\
  \indexdef{}{ML}{Name\_Space.full\_name}\verb|Name_Space.full_name: Name_Space.naming -> binding -> string| \\
  \end{mldecls}
  \begin{mldecls}
  \indexdef{}{ML type}{Name\_Space.T}\verb|type Name_Space.T| \\
  \indexdef{}{ML}{Name\_Space.empty}\verb|Name_Space.empty: string -> Name_Space.T| \\
  \indexdef{}{ML}{Name\_Space.merge}\verb|Name_Space.merge: Name_Space.T * Name_Space.T -> Name_Space.T| \\
  \indexdef{}{ML}{Name\_Space.declare}\verb|Name_Space.declare: bool -> Name_Space.naming -> binding -> Name_Space.T ->|\isasep\isanewline%
\verb|  string * Name_Space.T| \\
  \indexdef{}{ML}{Name\_Space.intern}\verb|Name_Space.intern: Name_Space.T -> string -> string| \\
  \indexdef{}{ML}{Name\_Space.extern}\verb|Name_Space.extern: Name_Space.T -> string -> string| \\
  \indexdef{}{ML}{Name\_Space.is\_concealed}\verb|Name_Space.is_concealed: Name_Space.T -> string -> bool|
  \end{mldecls}

  \begin{description}

  \item Type \verb|binding| represents the abstract concept of
  name bindings.

  \item \verb|Binding.empty| is the empty binding.

  \item \verb|Binding.name|~\isa{name} produces a binding with base
  name \isa{name}.  Note that this lacks proper source position
  information; see also the ML antiquotation \hyperlink{ML antiquotation.binding}{\mbox{\isa{binding}}}.

  \item \verb|Binding.qualify|~\isa{mandatory\ name\ binding}
  prefixes qualifier \isa{name} to \isa{binding}.  The \isa{mandatory} flag tells if this name component always needs to be
  given in name space accesses --- this is mostly \isa{false} in
  practice.  Note that this part of qualification is typically used in
  derived specification mechanisms.

  \item \verb|Binding.prefix| is similar to \verb|Binding.qualify|, but
  affects the system prefix.  This part of extra qualification is
  typically used in the infrastructure for modular specifications,
  notably ``local theory targets'' (see also \chref{ch:local-theory}).

  \item \verb|Binding.conceal|~\isa{binding} indicates that the
  binding shall refer to an entity that serves foundational purposes
  only.  This flag helps to mark implementation details of
  specification mechanism etc.  Other tools should not depend on the
  particulars of concealed entities (cf.\ \verb|Name_Space.is_concealed|).

  \item \verb|Binding.str_of|~\isa{binding} produces a string
  representation for human-readable output, together with some formal
  markup that might get used in GUI front-ends, for example.

  \item Type \verb|Name_Space.naming| represents the abstract
  concept of a naming policy.

  \item \verb|Name_Space.default_naming| is the default naming policy.
  In a theory context, this is usually augmented by a path prefix
  consisting of the theory name.

  \item \verb|Name_Space.add_path|~\isa{path\ naming} augments the
  naming policy by extending its path component.

  \item \verb|Name_Space.full_name|~\isa{naming\ binding} turns a
  name binding (usually a basic name) into the fully qualified
  internal name, according to the given naming policy.

  \item Type \verb|Name_Space.T| represents name spaces.

  \item \verb|Name_Space.empty|~\isa{kind} and \verb|Name_Space.merge|~\isa{{\isacharparenleft}space\isactrlisub {\isadigit{1}}{\isacharcomma}\ space\isactrlisub {\isadigit{2}}{\isacharparenright}} are the canonical operations for
  maintaining name spaces according to theory data management
  (\secref{sec:context-data}); \isa{kind} is a formal comment
  to characterize the purpose of a name space.

  \item \verb|Name_Space.declare|~\isa{strict\ naming\ bindings\ space} enters a name binding as fully qualified internal name into
  the name space, with external accesses determined by the naming
  policy.

  \item \verb|Name_Space.intern|~\isa{space\ name} internalizes a
  (partially qualified) external name.

  This operation is mostly for parsing!  Note that fully qualified
  names stemming from declarations are produced via \verb|Name_Space.full_name| and \verb|Name_Space.declare|
  (or their derivatives for \verb|theory| and
  \verb|Proof.context|).

  \item \verb|Name_Space.extern|~\isa{space\ name} externalizes a
  (fully qualified) internal name.

  This operation is mostly for printing!  User code should not rely on
  the precise result too much.

  \item \verb|Name_Space.is_concealed|~\isa{space\ name} indicates
  whether \isa{name} refers to a strictly private entity that
  other tools are supposed to ignore!

  \end{description}%
\end{isamarkuptext}%
\isamarkuptrue%
%
\endisatagmlref
{\isafoldmlref}%
%
\isadelimmlref
%
\endisadelimmlref
%
\isadelimmlantiq
%
\endisadelimmlantiq
%
\isatagmlantiq
%
\begin{isamarkuptext}%
\begin{matharray}{rcl}
  \indexdef{}{ML antiquotation}{binding}\hypertarget{ML antiquotation.binding}{\hyperlink{ML antiquotation.binding}{\mbox{\isa{binding}}}} & : & \isa{ML{\isacharunderscore}antiquotation} \\
  \end{matharray}

  \begin{rail}
  'binding' name
  ;
  \end{rail}

  \begin{description}

  \item \isa{{\isacharat}{\isacharbraceleft}binding\ name{\isacharbraceright}} produces a binding with base name
  \isa{name} and the source position taken from the concrete
  syntax of this antiquotation.  In many situations this is more
  appropriate than the more basic \verb|Binding.name| function.

  \end{description}%
\end{isamarkuptext}%
\isamarkuptrue%
%
\endisatagmlantiq
{\isafoldmlantiq}%
%
\isadelimmlantiq
%
\endisadelimmlantiq
%
\isadelimmlex
%
\endisadelimmlex
%
\isatagmlex
%
\begin{isamarkuptext}%
The following example yields the source position of some
  concrete binding inlined into the text.%
\end{isamarkuptext}%
\isamarkuptrue%
%
\endisatagmlex
{\isafoldmlex}%
%
\isadelimmlex
%
\endisadelimmlex
%
\isadelimML
%
\endisadelimML
%
\isatagML
\isacommand{ML}\isamarkupfalse%
\ {\isacharverbatimopen}\ Binding{\isachardot}pos{\isacharunderscore}of\ %
\isaantiq
binding\ here%
\endisaantiq
\ {\isacharverbatimclose}%
\endisatagML
{\isafoldML}%
%
\isadelimML
%
\endisadelimML
%
\begin{isamarkuptext}%
\medskip That position can be also printed in a message as
  follows.%
\end{isamarkuptext}%
\isamarkuptrue%
%
\isadelimML
%
\endisadelimML
%
\isatagML
\isacommand{ML{\isacharunderscore}command}\isamarkupfalse%
\ {\isacharverbatimopen}\isanewline
\ \ writeln\isanewline
\ \ \ \ {\isacharparenleft}{\isachardoublequote}Look\ here{\isachardoublequote}\ {\isacharcircum}\ Position{\isachardot}str{\isacharunderscore}of\ {\isacharparenleft}Binding{\isachardot}pos{\isacharunderscore}of\ %
\isaantiq
binding\ here%
\endisaantiq
{\isacharparenright}{\isacharparenright}\isanewline
{\isacharverbatimclose}%
\endisatagML
{\isafoldML}%
%
\isadelimML
%
\endisadelimML
%
\begin{isamarkuptext}%
This illustrates a key virtue of formalized bindings as
  opposed to raw specifications of base names: the system can use this
  additional information for advanced feedback given to the user.%
\end{isamarkuptext}%
\isamarkuptrue%
%
\isadelimtheory
%
\endisadelimtheory
%
\isatagtheory
\isacommand{end}\isamarkupfalse%
%
\endisatagtheory
{\isafoldtheory}%
%
\isadelimtheory
%
\endisadelimtheory
\isanewline
\end{isabellebody}%
%%% Local Variables:
%%% mode: latex
%%% TeX-master: "root"
%%% End:
