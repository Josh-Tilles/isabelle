%
\begin{isabellebody}%
\def\isabellecontext{ML}%
%
\isadelimtheory
\isanewline
\isanewline
\isanewline
%
\endisadelimtheory
%
\isatagtheory
\isacommand{theory}\isamarkupfalse%
\ {\isachardoublequoteopen}ML{\isachardoublequoteclose}\ \isakeyword{imports}\ base\ \isakeyword{begin}%
\endisatagtheory
{\isafoldtheory}%
%
\isadelimtheory
%
\endisadelimtheory
%
\isamarkupchapter{Aesthetics of ML programming%
}
\isamarkuptrue%
%
\begin{isamarkuptext}%
FIXME style guide, see also
\url{http://caml.inria.fr/resources/doc/guides/guidelines.en.html} and
\url{http://www.cs.cornell.edu/Courses/cs312/2003sp/handouts/style.htm}%
\end{isamarkuptext}%
\isamarkuptrue%
%
\isamarkupchapter{Basic library functions%
}
\isamarkuptrue%
%
\begin{isamarkuptext}%
FIXME beyond the basis library definition%
\end{isamarkuptext}%
\isamarkuptrue%
%
\isamarkupchapter{Cookbook%
}
\isamarkuptrue%
%
\isamarkupsection{Defining a method that depends on declarations in the context%
}
\isamarkuptrue%
%
\begin{isamarkuptext}%
FIXME%
\end{isamarkuptext}%
\isamarkuptrue%
%
\isadelimtheory
%
\endisadelimtheory
%
\isatagtheory
\isacommand{end}\isamarkupfalse%
%
\endisatagtheory
{\isafoldtheory}%
%
\isadelimtheory
%
\endisadelimtheory
\isanewline
\end{isabellebody}%
%%% Local Variables:
%%% mode: latex
%%% TeX-master: "root"
%%% End:
