%
\begin{isabellebody}%
\def\isabellecontext{logic}%
%
\isadelimtheory
\isanewline
\isanewline
\isanewline
%
\endisadelimtheory
%
\isatagtheory
\isacommand{theory}\isamarkupfalse%
\ logic\ \isakeyword{imports}\ base\ \isakeyword{begin}%
\endisatagtheory
{\isafoldtheory}%
%
\isadelimtheory
%
\endisadelimtheory
%
\isamarkupchapter{Primitive logic \label{ch:logic}%
}
\isamarkuptrue%
%
\begin{isamarkuptext}%
The logical foundations of Isabelle/Isar are that of the Pure logic,
  which has been introduced as a natural-deduction framework in
  \cite{paulson700}.  This is essentially the same logic as ``\isa{{\isasymlambda}HOL}'' in the more abstract setting of Pure Type Systems (PTS)
  \cite{Barendregt-Geuvers:2001}, although there are some key
  differences in the specific treatment of simple types in
  Isabelle/Pure.

  Following type-theoretic parlance, the Pure logic consists of three
  levels of \isa{{\isasymlambda}}-calculus with corresponding arrows: \isa{{\isasymRightarrow}} for syntactic function space (terms depending on terms), \isa{{\isasymAnd}} for universal quantification (proofs depending on terms), and
  \isa{{\isasymLongrightarrow}} for implication (proofs depending on proofs).

  Pure derivations are relative to a logical theory, which declares
  type constructors, term constants, and axioms.  Theory declarations
  support schematic polymorphism, which is strictly speaking outside
  the logic.\footnote{Incidently, this is the main logical reason, why
  the theory context \isa{{\isasymTheta}} is separate from the context \isa{{\isasymGamma}} of the core calculus.}%
\end{isamarkuptext}%
\isamarkuptrue%
%
\isamarkupsection{Types \label{sec:types}%
}
\isamarkuptrue%
%
\begin{isamarkuptext}%
The language of types is an uninterpreted order-sorted first-order
  algebra; types are qualified by ordered type classes.

  \medskip A \emph{type class} is an abstract syntactic entity
  declared in the theory context.  The \emph{subclass relation} \isa{c\isactrlisub {\isadigit{1}}\ {\isasymsubseteq}\ c\isactrlisub {\isadigit{2}}} is specified by stating an acyclic
  generating relation; the transitive closure is maintained
  internally.  The resulting relation is an ordering: reflexive,
  transitive, and antisymmetric.

  A \emph{sort} is a list of type classes written as \isa{{\isacharbraceleft}c\isactrlisub {\isadigit{1}}{\isacharcomma}\ {\isasymdots}{\isacharcomma}\ c\isactrlisub m{\isacharbraceright}}, which represents symbolic
  intersection.  Notationally, the curly braces are omitted for
  singleton intersections, i.e.\ any class \isa{c} may be read as
  a sort \isa{{\isacharbraceleft}c{\isacharbraceright}}.  The ordering on type classes is extended to
  sorts according to the meaning of intersections: \isa{{\isacharbraceleft}c\isactrlisub {\isadigit{1}}{\isacharcomma}\ {\isasymdots}\ c\isactrlisub m{\isacharbraceright}\ {\isasymsubseteq}\ {\isacharbraceleft}d\isactrlisub {\isadigit{1}}{\isacharcomma}\ {\isasymdots}{\isacharcomma}\ d\isactrlisub n{\isacharbraceright}} iff
  \isa{{\isasymforall}j{\isachardot}\ {\isasymexists}i{\isachardot}\ c\isactrlisub i\ {\isasymsubseteq}\ d\isactrlisub j}.  The empty intersection
  \isa{{\isacharbraceleft}{\isacharbraceright}} refers to the universal sort, which is the largest
  element wrt.\ the sort order.  The intersections of all (finitely
  many) classes declared in the current theory are the minimal
  elements wrt.\ the sort order.

  \medskip A \emph{fixed type variable} is a pair of a basic name
  (starting with a \isa{{\isacharprime}} character) and a sort constraint.  For
  example, \isa{{\isacharparenleft}{\isacharprime}a{\isacharcomma}\ s{\isacharparenright}} which is usually printed as \isa{{\isasymalpha}\isactrlisub s}.  A \emph{schematic type variable} is a pair of an
  indexname and a sort constraint.  For example, \isa{{\isacharparenleft}{\isacharparenleft}{\isacharprime}a{\isacharcomma}\ {\isadigit{0}}{\isacharparenright}{\isacharcomma}\ s{\isacharparenright}} which is usually printed as \isa{{\isacharquery}{\isasymalpha}\isactrlisub s}.

  Note that \emph{all} syntactic components contribute to the identity
  of type variables, including the sort constraint.  The core logic
  handles type variables with the same name but different sorts as
  different, although some outer layers of the system make it hard to
  produce anything like this.

  A \emph{type constructor} \isa{{\isasymkappa}} is a \isa{k}-ary operator
  on types declared in the theory.  Type constructor application is
  usually written postfix as \isa{{\isacharparenleft}{\isasymalpha}\isactrlisub {\isadigit{1}}{\isacharcomma}\ {\isasymdots}{\isacharcomma}\ {\isasymalpha}\isactrlisub k{\isacharparenright}{\isasymkappa}}.
  For \isa{k\ {\isacharequal}\ {\isadigit{0}}} the argument tuple is omitted, e.g.\ \isa{prop} instead of \isa{{\isacharparenleft}{\isacharparenright}prop}.  For \isa{k\ {\isacharequal}\ {\isadigit{1}}} the
  parentheses are omitted, e.g.\ \isa{{\isasymalpha}\ list} instead of \isa{{\isacharparenleft}{\isasymalpha}{\isacharparenright}list}.  Further notation is provided for specific constructors,
  notably the right-associative infix \isa{{\isasymalpha}\ {\isasymRightarrow}\ {\isasymbeta}} instead of
  \isa{{\isacharparenleft}{\isasymalpha}{\isacharcomma}\ {\isasymbeta}{\isacharparenright}fun}.
  
  A \emph{type} \isa{{\isasymtau}} is defined inductively over type variables
  and type constructors as follows: \isa{{\isasymtau}\ {\isacharequal}\ {\isasymalpha}\isactrlisub s\ {\isacharbar}\ {\isacharquery}{\isasymalpha}\isactrlisub s\ {\isacharbar}\ {\isacharparenleft}{\isasymtau}\isactrlsub {\isadigit{1}}{\isacharcomma}\ {\isasymdots}{\isacharcomma}\ {\isasymtau}\isactrlsub k{\isacharparenright}k}.

  A \emph{type abbreviation} is a syntactic definition \isa{{\isacharparenleft}\isactrlvec {\isasymalpha}{\isacharparenright}{\isasymkappa}\ {\isacharequal}\ {\isasymtau}} of an arbitrary type expression \isa{{\isasymtau}} over
  variables \isa{\isactrlvec {\isasymalpha}}.  Type abbreviations looks like type
  constructors at the surface, but are fully expanded before entering
  the logical core.

  A \emph{type arity} declares the image behavior of a type
  constructor wrt.\ the algebra of sorts: \isa{{\isasymkappa}\ {\isacharcolon}{\isacharcolon}\ {\isacharparenleft}s\isactrlisub {\isadigit{1}}{\isacharcomma}\ {\isasymdots}{\isacharcomma}\ s\isactrlisub k{\isacharparenright}s} means that \isa{{\isacharparenleft}{\isasymtau}\isactrlisub {\isadigit{1}}{\isacharcomma}\ {\isasymdots}{\isacharcomma}\ {\isasymtau}\isactrlisub k{\isacharparenright}{\isasymkappa}} is
  of sort \isa{s} if every argument type \isa{{\isasymtau}\isactrlisub i} is
  of sort \isa{s\isactrlisub i}.  Arity declarations are implicitly
  completed, i.e.\ \isa{{\isasymkappa}\ {\isacharcolon}{\isacharcolon}\ {\isacharparenleft}\isactrlvec s{\isacharparenright}c} entails \isa{{\isasymkappa}\ {\isacharcolon}{\isacharcolon}\ {\isacharparenleft}\isactrlvec s{\isacharparenright}c{\isacharprime}} for any \isa{c{\isacharprime}\ {\isasymsupseteq}\ c}.

  \medskip The sort algebra is always maintained as \emph{coregular},
  which means that type arities are consistent with the subclass
  relation: for each type constructor \isa{{\isasymkappa}} and classes \isa{c\isactrlisub {\isadigit{1}}\ {\isasymsubseteq}\ c\isactrlisub {\isadigit{2}}}, any arity \isa{{\isasymkappa}\ {\isacharcolon}{\isacharcolon}\ {\isacharparenleft}\isactrlvec s\isactrlisub {\isadigit{1}}{\isacharparenright}c\isactrlisub {\isadigit{1}}} has a corresponding arity \isa{{\isasymkappa}\ {\isacharcolon}{\isacharcolon}\ {\isacharparenleft}\isactrlvec s\isactrlisub {\isadigit{2}}{\isacharparenright}c\isactrlisub {\isadigit{2}}} where \isa{\isactrlvec s\isactrlisub {\isadigit{1}}\ {\isasymsubseteq}\ \isactrlvec s\isactrlisub {\isadigit{2}}} holds componentwise.

  The key property of a coregular order-sorted algebra is that sort
  constraints may be always solved in a most general fashion: for each
  type constructor \isa{{\isasymkappa}} and sort \isa{s} there is a most
  general vector of argument sorts \isa{{\isacharparenleft}s\isactrlisub {\isadigit{1}}{\isacharcomma}\ {\isasymdots}{\isacharcomma}\ s\isactrlisub k{\isacharparenright}} such that a type scheme \isa{{\isacharparenleft}{\isasymalpha}\isactrlbsub s\isactrlisub {\isadigit{1}}\isactrlesub {\isacharcomma}\ {\isasymdots}{\isacharcomma}\ {\isasymalpha}\isactrlbsub s\isactrlisub k\isactrlesub {\isacharparenright}{\isasymkappa}} is
  of sort \isa{s}.  Consequently, the unification problem on the
  algebra of types has most general solutions (modulo renaming and
  equivalence of sorts).  Moreover, the usual type-inference algorithm
  will produce primary types as expected \cite{nipkow-prehofer}.%
\end{isamarkuptext}%
\isamarkuptrue%
%
\isadelimmlref
%
\endisadelimmlref
%
\isatagmlref
%
\begin{isamarkuptext}%
\begin{mldecls}
  \indexmltype{class}\verb|type class| \\
  \indexmltype{sort}\verb|type sort| \\
  \indexmltype{arity}\verb|type arity| \\
  \indexmltype{typ}\verb|type typ| \\
  \indexml{fold-atyps}\verb|fold_atyps: (typ -> 'a -> 'a) -> typ -> 'a -> 'a| \\
  \indexml{Sign.subsort}\verb|Sign.subsort: theory -> sort * sort -> bool| \\
  \indexml{Sign.of-sort}\verb|Sign.of_sort: theory -> typ * sort -> bool| \\
  \indexml{Sign.add-types}\verb|Sign.add_types: (bstring * int * mixfix) list -> theory -> theory| \\
  \indexml{Sign.add-tyabbrs-i}\verb|Sign.add_tyabbrs_i: |\isasep\isanewline%
\verb|  (bstring * string list * typ * mixfix) list -> theory -> theory| \\
  \indexml{Sign.primitive-class}\verb|Sign.primitive_class: string * class list -> theory -> theory| \\
  \indexml{Sign.primitive-classrel}\verb|Sign.primitive_classrel: class * class -> theory -> theory| \\
  \indexml{Sign.primitive-arity}\verb|Sign.primitive_arity: arity -> theory -> theory| \\
  \end{mldecls}

  \begin{description}

  \item \verb|class| represents type classes; this is an alias for
  \verb|string|.

  \item \verb|sort| represents sorts; this is an alias for
  \verb|class list|.

  \item \verb|arity| represents type arities; this is an alias for
  triples of the form \isa{{\isacharparenleft}{\isasymkappa}{\isacharcomma}\ \isactrlvec s{\isacharcomma}\ s{\isacharparenright}} for \isa{{\isasymkappa}\ {\isacharcolon}{\isacharcolon}\ {\isacharparenleft}\isactrlvec s{\isacharparenright}s} described above.

  \item \verb|typ| represents types; this is a datatype with
  constructors \verb|TFree|, \verb|TVar|, \verb|Type|.

  \item \verb|fold_atyps|~\isa{f\ {\isasymtau}} iterates function \isa{f}
  over all occurrences of atoms (\verb|TFree| or \verb|TVar|) of \isa{{\isasymtau}}; the type structure is traversed from left to right.

  \item \verb|Sign.subsort|~\isa{thy\ {\isacharparenleft}s\isactrlisub {\isadigit{1}}{\isacharcomma}\ s\isactrlisub {\isadigit{2}}{\isacharparenright}}
  tests the subsort relation \isa{s\isactrlisub {\isadigit{1}}\ {\isasymsubseteq}\ s\isactrlisub {\isadigit{2}}}.

  \item \verb|Sign.of_sort|~\isa{thy\ {\isacharparenleft}{\isasymtau}{\isacharcomma}\ s{\isacharparenright}} tests whether a type
  is of a given sort.

  \item \verb|Sign.add_types|~\isa{{\isacharbrackleft}{\isacharparenleft}{\isasymkappa}{\isacharcomma}\ k{\isacharcomma}\ mx{\isacharparenright}{\isacharcomma}\ {\isasymdots}{\isacharbrackright}} declares new
  type constructors \isa{{\isasymkappa}} with \isa{k} arguments and
  optional mixfix syntax.

  \item \verb|Sign.add_tyabbrs_i|~\isa{{\isacharbrackleft}{\isacharparenleft}{\isasymkappa}{\isacharcomma}\ \isactrlvec {\isasymalpha}{\isacharcomma}\ {\isasymtau}{\isacharcomma}\ mx{\isacharparenright}{\isacharcomma}\ {\isasymdots}{\isacharbrackright}}
  defines a new type abbreviation \isa{{\isacharparenleft}\isactrlvec {\isasymalpha}{\isacharparenright}{\isasymkappa}\ {\isacharequal}\ {\isasymtau}} with
  optional mixfix syntax.

  \item \verb|Sign.primitive_class|~\isa{{\isacharparenleft}c{\isacharcomma}\ {\isacharbrackleft}c\isactrlisub {\isadigit{1}}{\isacharcomma}\ {\isasymdots}{\isacharcomma}\ c\isactrlisub n{\isacharbrackright}{\isacharparenright}} declares new class \isa{c}, together with class
  relations \isa{c\ {\isasymsubseteq}\ c\isactrlisub i}, for \isa{i\ {\isacharequal}\ {\isadigit{1}}{\isacharcomma}\ {\isasymdots}{\isacharcomma}\ n}.

  \item \verb|Sign.primitive_classrel|~\isa{{\isacharparenleft}c\isactrlisub {\isadigit{1}}{\isacharcomma}\ c\isactrlisub {\isadigit{2}}{\isacharparenright}} declares class relation \isa{c\isactrlisub {\isadigit{1}}\ {\isasymsubseteq}\ c\isactrlisub {\isadigit{2}}}.

  \item \verb|Sign.primitive_arity|~\isa{{\isacharparenleft}{\isasymkappa}{\isacharcomma}\ \isactrlvec s{\isacharcomma}\ s{\isacharparenright}} declares
  arity \isa{{\isasymkappa}\ {\isacharcolon}{\isacharcolon}\ {\isacharparenleft}\isactrlvec s{\isacharparenright}s}.

  \end{description}%
\end{isamarkuptext}%
\isamarkuptrue%
%
\endisatagmlref
{\isafoldmlref}%
%
\isadelimmlref
%
\endisadelimmlref
%
\isamarkupsection{Terms \label{sec:terms}%
}
\isamarkuptrue%
%
\begin{isamarkuptext}%
\glossary{Term}{FIXME}

  The language of terms is that of simply-typed \isa{{\isasymlambda}}-calculus
  with de-Bruijn indices for bound variables, and named free variables
  and constants.  Terms with loose bound variables are usually
  considered malformed.  The types of variables and constants is
  stored explicitly at each occurrence in the term.

  \medskip A \emph{bound variable} is a natural number \isa{b},
  which refers to the next binder that is \isa{b} steps upwards
  from the occurrence of \isa{b} (counting from zero).  Bindings
  may be introduced as abstractions within the term, or as a separate
  context (an inside-out list).  This associates each bound variable
  with a type, and a name that is maintained as a comment for parsing
  and printing.  A \emph{loose variables} is a bound variable that is
  outside the current scope of local binders or the context.  For
  example, the de-Bruijn term \isa{{\isasymlambda}\isactrlisub {\isasymtau}{\isachardot}\ {\isasymlambda}\isactrlisub {\isasymtau}{\isachardot}\ {\isadigit{1}}\ {\isacharplus}\ {\isadigit{0}}}
  corresponds to \isa{{\isasymlambda}x\isactrlisub {\isasymtau}{\isachardot}\ {\isasymlambda}y\isactrlisub {\isasymtau}{\isachardot}\ x\ {\isacharplus}\ y} in a named
  representation.  Also note that the very same bound variable may get
  different numbers at different occurrences.

  A \emph{fixed variable} is a pair of a basic name and a type.  For
  example, \isa{{\isacharparenleft}x{\isacharcomma}\ {\isasymtau}{\isacharparenright}} which is usually printed \isa{x\isactrlisub {\isasymtau}}.  A \emph{schematic variable} is a pair of an
  indexname and a type.  For example, \isa{{\isacharparenleft}{\isacharparenleft}x{\isacharcomma}\ {\isadigit{0}}{\isacharparenright}{\isacharcomma}\ {\isasymtau}{\isacharparenright}} which is
  usually printed as \isa{{\isacharquery}x\isactrlisub {\isasymtau}}.

  \medskip A \emph{constant} is a atomic terms consisting of a basic
  name and a type.  Constants are declared in the context as
  polymorphic families \isa{c\ {\isacharcolon}{\isacharcolon}\ {\isasymsigma}}, meaning that any \isa{c\isactrlisub {\isasymtau}} is a valid constant for all substitution instances
  \isa{{\isasymtau}\ {\isasymle}\ {\isasymsigma}}.

  The list of \emph{type arguments} of \isa{c\isactrlisub {\isasymtau}} wrt.\ the
  declaration \isa{c\ {\isacharcolon}{\isacharcolon}\ {\isasymsigma}} is the codomain of the type matcher
  presented in canonical order (according to the left-to-right
  occurrences of type variables in in \isa{{\isasymsigma}}).  Thus \isa{c\isactrlisub {\isasymtau}} can be represented more compactly as \isa{c{\isacharparenleft}{\isasymtau}\isactrlisub {\isadigit{1}}{\isacharcomma}\ {\isasymdots}{\isacharcomma}\ {\isasymtau}\isactrlisub n{\isacharparenright}}.  For example, the instance \isa{plus\isactrlbsub nat\ {\isasymRightarrow}\ nat\ {\isasymRightarrow}\ nat\isactrlesub } of some \isa{plus\ {\isacharcolon}{\isacharcolon}\ {\isasymalpha}\ {\isasymRightarrow}\ {\isasymalpha}\ {\isasymRightarrow}\ {\isasymalpha}} has the singleton list \isa{nat} as type arguments, the
  constant may be represented as \isa{plus{\isacharparenleft}nat{\isacharparenright}}.

  Constant declarations \isa{c\ {\isacharcolon}{\isacharcolon}\ {\isasymsigma}} may contain sort constraints
  for type variables in \isa{{\isasymsigma}}.  These are observed by
  type-inference as expected, but \emph{ignored} by the core logic.
  This means the primitive logic is able to reason with instances of
  polymorphic constants that the user-level type-checker would reject.

  \medskip A \emph{term} \isa{t} is defined inductively over
  variables and constants, with abstraction and application as
  follows: \isa{t\ {\isacharequal}\ b\ {\isacharbar}\ x\isactrlisub {\isasymtau}\ {\isacharbar}\ {\isacharquery}x\isactrlisub {\isasymtau}\ {\isacharbar}\ c\isactrlisub {\isasymtau}\ {\isacharbar}\ {\isasymlambda}\isactrlisub {\isasymtau}{\isachardot}\ t\ {\isacharbar}\ t\isactrlisub {\isadigit{1}}\ t\isactrlisub {\isadigit{2}}}.  Parsing and printing takes
  care of converting between an external representation with named
  bound variables.  Subsequently, we shall use the latter notation
  instead of internal de-Bruijn representation.

  The subsequent inductive relation \isa{t\ {\isacharcolon}{\isacharcolon}\ {\isasymtau}} assigns a
  (unique) type to a term, using the special type constructor \isa{{\isacharparenleft}{\isasymalpha}{\isacharcomma}\ {\isasymbeta}{\isacharparenright}fun}, which is written \isa{{\isasymalpha}\ {\isasymRightarrow}\ {\isasymbeta}}.
  \[
  \infer{\isa{a\isactrlisub {\isasymtau}\ {\isacharcolon}{\isacharcolon}\ {\isasymtau}}}{}
  \qquad
  \infer{\isa{{\isacharparenleft}{\isasymlambda}x\isactrlsub {\isasymtau}{\isachardot}\ t{\isacharparenright}\ {\isacharcolon}{\isacharcolon}\ {\isasymtau}\ {\isasymRightarrow}\ {\isasymsigma}}}{\isa{t\ {\isacharcolon}{\isacharcolon}\ {\isasymsigma}}}
  \qquad
  \infer{\isa{t\ u\ {\isacharcolon}{\isacharcolon}\ {\isasymsigma}}}{\isa{t\ {\isacharcolon}{\isacharcolon}\ {\isasymtau}\ {\isasymRightarrow}\ {\isasymsigma}} & \isa{u\ {\isacharcolon}{\isacharcolon}\ {\isasymtau}}}
  \]
  A \emph{well-typed term} is a term that can be typed according to these rules.

  Typing information can be omitted: type-inference is able to
  reconstruct the most general type of a raw term, while assigning
  most general types to all of its variables and constants.
  Type-inference depends on a context of type constraints for fixed
  variables, and declarations for polymorphic constants.

  The identity of atomic terms consists both of the name and the type.
  Thus different entities \isa{c\isactrlbsub {\isasymtau}\isactrlisub {\isadigit{1}}\isactrlesub } and
  \isa{c\isactrlbsub {\isasymtau}\isactrlisub {\isadigit{2}}\isactrlesub } may well identified by type
  instantiation, by mapping \isa{{\isasymtau}\isactrlisub {\isadigit{1}}} and \isa{{\isasymtau}\isactrlisub {\isadigit{2}}} to the same \isa{{\isasymtau}}.  Although,
  different type instances of constants of the same basic name are
  commonplace, this rarely happens for variables: type-inference
  always demands ``consistent'' type constraints.

  \medskip The \emph{hidden polymorphism} of a term \isa{t\ {\isacharcolon}{\isacharcolon}\ {\isasymsigma}}
  is the set of type variables occurring in \isa{t}, but not in
  \isa{{\isasymsigma}}.  This means that the term implicitly depends on the
  values of various type variables that are not visible in the overall
  type, i.e.\ there are different type instances \isa{t{\isasymvartheta}\ {\isacharcolon}{\isacharcolon}\ {\isasymsigma}} and \isa{t{\isasymvartheta}{\isacharprime}\ {\isacharcolon}{\isacharcolon}\ {\isasymsigma}} with the same type.  This
  slightly pathological situation is apt to cause strange effects.

  \medskip A \emph{term abbreviation} is a syntactic definition \isa{c\isactrlisub {\isasymsigma}\ {\isasymequiv}\ t} of an arbitrary closed term \isa{t} of type
  \isa{{\isasymsigma}} without any hidden polymorphism.  A term abbreviation
  looks like a constant at the surface, but is fully expanded before
  entering the logical core.  Abbreviations are usually reverted when
  printing terms, using rules \isa{t\ {\isasymrightarrow}\ c\isactrlisub {\isasymsigma}} has a
  higher-order term rewrite system.%
\end{isamarkuptext}%
\isamarkuptrue%
%
\isadelimmlref
%
\endisadelimmlref
%
\isatagmlref
%
\begin{isamarkuptext}%
\begin{mldecls}
  \indexmltype{term}\verb|type term| \\
  \indexml{map-aterms}\verb|map_aterms: (term -> term) -> term -> term| \\
  \indexml{fold-aterms}\verb|fold_aterms: (term -> 'a -> 'a) -> term -> 'a -> 'a| \\
  \indexml{fastype-of}\verb|fastype_of: term -> typ| \\
  \indexml{fold-types}\verb|fold_types: (typ -> 'a -> 'a) -> term -> 'a -> 'a| \\
  \end{mldecls}

  \begin{description}

  \item \verb|term| FIXME

  \end{description}%
\end{isamarkuptext}%
\isamarkuptrue%
%
\endisatagmlref
{\isafoldmlref}%
%
\isadelimmlref
%
\endisadelimmlref
%
\isamarkupsection{Theorems \label{sec:thms}%
}
\isamarkuptrue%
%
\begin{isamarkuptext}%
\glossary{Proposition}{A \seeglossary{term} of \seeglossary{type}
  \isa{prop}.  Internally, there is nothing special about
  propositions apart from their type, but the concrete syntax enforces
  a clear distinction.  Propositions are structured via implication
  \isa{A\ {\isasymLongrightarrow}\ B} or universal quantification \isa{{\isasymAnd}x{\isachardot}\ B\ x} ---
  anything else is considered atomic.  The canonical form for
  propositions is that of a \seeglossary{Hereditary Harrop Formula}. FIXME}

  \glossary{Theorem}{A proven proposition within a certain theory and
  proof context, formally \isa{{\isasymGamma}\ {\isasymturnstile}\isactrlsub {\isasymTheta}\ {\isasymphi}}; both contexts are
  rarely spelled out explicitly.  Theorems are usually normalized
  according to the \seeglossary{HHF} format. FIXME}

  \glossary{Fact}{Sometimes used interchangably for
  \seeglossary{theorem}.  Strictly speaking, a list of theorems,
  essentially an extra-logical conjunction.  Facts emerge either as
  local assumptions, or as results of local goal statements --- both
  may be simultaneous, hence the list representation. FIXME}

  \glossary{Schematic variable}{FIXME}

  \glossary{Fixed variable}{A variable that is bound within a certain
  proof context; an arbitrary-but-fixed entity within a portion of
  proof text. FIXME}

  \glossary{Free variable}{Synonymous for \seeglossary{fixed
  variable}. FIXME}

  \glossary{Bound variable}{FIXME}

  \glossary{Variable}{See \seeglossary{schematic variable},
  \seeglossary{fixed variable}, \seeglossary{bound variable}, or
  \seeglossary{type variable}.  The distinguishing feature of
  different variables is their binding scope. FIXME}

  A \emph{proposition} is a well-formed term of type \isa{prop}.
  The connectives of minimal logic are declared as constants of the
  basic theory:

  \smallskip
  \begin{tabular}{ll}
  \isa{all\ {\isacharcolon}{\isacharcolon}\ {\isacharparenleft}{\isasymalpha}\ {\isasymRightarrow}\ prop{\isacharparenright}\ {\isasymRightarrow}\ prop} & universal quantification (binder \isa{{\isasymAnd}}) \\
  \isa{{\isasymLongrightarrow}\ {\isacharcolon}{\isacharcolon}\ prop\ {\isasymRightarrow}\ prop\ {\isasymRightarrow}\ prop} & implication (right associative infix) \\
  \end{tabular}

  \medskip A \emph{theorem} is a proven proposition, depending on a
  collection of assumptions, and axioms from the theory context.  The
  judgment \isa{A\isactrlisub {\isadigit{1}}{\isacharcomma}\ {\isasymdots}{\isacharcomma}\ A\isactrlisub n\ {\isasymturnstile}\ B} is defined
  inductively by the primitive inferences given in
  \figref{fig:prim-rules}; there is a global syntactic restriction
  that the hypotheses may not contain schematic variables.

  \begin{figure}[htb]
  \begin{center}
  \[
  \infer[\isa{{\isacharparenleft}axiom{\isacharparenright}}]{\isa{{\isasymturnstile}\ A}}{\isa{A\ {\isasymin}\ {\isasymTheta}}}
  \qquad
  \infer[\isa{{\isacharparenleft}assume{\isacharparenright}}]{\isa{A\ {\isasymturnstile}\ A}}{}
  \]
  \[
  \infer[\isa{{\isacharparenleft}{\isasymAnd}{\isacharunderscore}intro{\isacharparenright}}]{\isa{{\isasymGamma}\ {\isasymturnstile}\ {\isasymAnd}x{\isachardot}\ b\ x}}{\isa{{\isasymGamma}\ {\isasymturnstile}\ b\ x} & \isa{x\ {\isasymnotin}\ {\isasymGamma}}}
  \qquad
  \infer[\isa{{\isacharparenleft}{\isasymAnd}{\isacharunderscore}elim{\isacharparenright}}]{\isa{{\isasymGamma}\ {\isasymturnstile}\ b\ a}}{\isa{{\isasymGamma}\ {\isasymturnstile}\ {\isasymAnd}x{\isachardot}\ b\ x}}
  \]
  \[
  \infer[\isa{{\isacharparenleft}{\isasymLongrightarrow}{\isacharunderscore}intro{\isacharparenright}}]{\isa{{\isasymGamma}\ {\isacharminus}\ A\ {\isasymturnstile}\ A\ {\isasymLongrightarrow}\ B}}{\isa{{\isasymGamma}\ {\isasymturnstile}\ B}}
  \qquad
  \infer[\isa{{\isacharparenleft}{\isasymLongrightarrow}{\isacharunderscore}elim{\isacharparenright}}]{\isa{{\isasymGamma}\isactrlsub {\isadigit{1}}\ {\isasymunion}\ {\isasymGamma}\isactrlsub {\isadigit{2}}\ {\isasymturnstile}\ B}}{\isa{{\isasymGamma}\isactrlsub {\isadigit{1}}\ {\isasymturnstile}\ A\ {\isasymLongrightarrow}\ B} & \isa{{\isasymGamma}\isactrlsub {\isadigit{2}}\ {\isasymturnstile}\ A}}
  \]
  \caption{Primitive inferences of the Pure logic}\label{fig:prim-rules}
  \end{center}
  \end{figure}

  The introduction and elimination rules for \isa{{\isasymAnd}} and \isa{{\isasymLongrightarrow}} are analogous to formation of (dependently typed) \isa{{\isasymlambda}}-terms representing the underlying proof objects.  Proof terms
  are \emph{irrelevant} in the Pure logic, they may never occur within
  propositions, i.e.\ the \isa{{\isasymLongrightarrow}} arrow of the framework is a
  non-dependent one.

  Also note that fixed parameters as in \isa{{\isasymAnd}{\isacharunderscore}intro} need not be
  recorded in the context \isa{{\isasymGamma}}, since syntactic types are
  always inhabitable.  An ``assumption'' \isa{x\ {\isacharcolon}{\isacharcolon}\ {\isasymtau}} is logically
  vacuous, because \isa{{\isasymtau}} is always non-empty.  This is the deeper
  reason why \isa{{\isasymGamma}} only consists of hypothetical proofs, but no
  hypothetical terms.

  The corresponding proof terms are left implicit in the classic
  ``LCF-approach'', although they could be exploited separately
  \cite{Berghofer-Nipkow:2000}.  The implementation provides a runtime
  option to control the generation of full proof terms.

  \medskip The axiomatization of a theory is implicitly closed by
  forming all instances of type and term variables: \isa{{\isasymturnstile}\ A{\isasymtheta}} for
  any substitution instance of axiom \isa{{\isasymturnstile}\ A}.  By pushing
  substitution through derivations inductively, we get admissible
  substitution rules for theorems shown in \figref{fig:subst-rules}.

  \begin{figure}[htb]
  \begin{center}
  \[
  \infer{\isa{{\isasymGamma}\ {\isasymturnstile}\ B{\isacharbrackleft}{\isacharquery}{\isasymalpha}{\isacharbrackright}}}{\isa{{\isasymGamma}\ {\isasymturnstile}\ B{\isacharbrackleft}{\isasymalpha}{\isacharbrackright}} & \isa{{\isasymalpha}\ {\isasymnotin}\ {\isasymGamma}}}
  \quad
  \infer[\quad\isa{{\isacharparenleft}generalize{\isacharparenright}}]{\isa{{\isasymGamma}\ {\isasymturnstile}\ B{\isacharbrackleft}{\isacharquery}x{\isacharbrackright}}}{\isa{{\isasymGamma}\ {\isasymturnstile}\ B{\isacharbrackleft}x{\isacharbrackright}} & \isa{x\ {\isasymnotin}\ {\isasymGamma}}}
  \]
  \[
  \infer{\isa{{\isasymGamma}\ {\isasymturnstile}\ B{\isacharbrackleft}{\isasymtau}{\isacharbrackright}}}{\isa{{\isasymGamma}\ {\isasymturnstile}\ B{\isacharbrackleft}{\isacharquery}{\isasymalpha}{\isacharbrackright}}}
  \quad
  \infer[\quad\isa{{\isacharparenleft}instantiate{\isacharparenright}}]{\isa{{\isasymGamma}\ {\isasymturnstile}\ B{\isacharbrackleft}t{\isacharbrackright}}}{\isa{{\isasymGamma}\ {\isasymturnstile}\ B{\isacharbrackleft}{\isacharquery}x{\isacharbrackright}}}
  \]
  \caption{Admissible substitution rules}\label{fig:subst-rules}
  \end{center}
  \end{figure}

  Note that \isa{instantiate{\isacharunderscore}term} could be derived using \isa{{\isasymAnd}{\isacharunderscore}intro{\isacharslash}elim}, but this is not how it is implemented.  The type
  instantiation rule is a genuine admissible one, due to the lack of
  true polymorphism in the logic.

  Since \isa{{\isasymGamma}} may never contain any schematic variables, the
  \isa{instantiate} do not require an explicit side-condition.  In
  principle, variables could be substituted in hypotheses as well, but
  this could disrupt monotonicity of the basic calculus: derivations
  could leave the current proof context.

  \medskip The framework also provides builtin equality \isa{{\isasymequiv}},
  which is conceptually axiomatized shown in \figref{fig:equality},
  although the implementation provides derived rules directly:

  \begin{figure}[htb]
  \begin{center}
  \begin{tabular}{ll}
  \isa{{\isasymequiv}\ {\isacharcolon}{\isacharcolon}\ {\isasymalpha}\ {\isasymRightarrow}\ {\isasymalpha}\ {\isasymRightarrow}\ prop} & equality relation (infix) \\
  \isa{{\isasymturnstile}\ {\isacharparenleft}{\isasymlambda}x{\isachardot}\ b\ x{\isacharparenright}\ a\ {\isasymequiv}\ b\ a} & \isa{{\isasymbeta}}-conversion \\
  \isa{{\isasymturnstile}\ x\ {\isasymequiv}\ x} & reflexivity law \\
  \isa{{\isasymturnstile}\ x\ {\isasymequiv}\ y\ {\isasymLongrightarrow}\ P\ x\ {\isasymLongrightarrow}\ P\ y} & substitution law \\
  \isa{{\isasymturnstile}\ {\isacharparenleft}{\isasymAnd}x{\isachardot}\ f\ x\ {\isasymequiv}\ g\ x{\isacharparenright}\ {\isasymLongrightarrow}\ f\ {\isasymequiv}\ g} & extensionality \\
  \isa{{\isasymturnstile}\ {\isacharparenleft}A\ {\isasymLongrightarrow}\ B{\isacharparenright}\ {\isasymLongrightarrow}\ {\isacharparenleft}B\ {\isasymLongrightarrow}\ A{\isacharparenright}\ {\isasymLongrightarrow}\ A\ {\isasymequiv}\ B} & coincidence with equivalence \\
  \end{tabular}
  \caption{Conceptual axiomatization of equality.}\label{fig:equality}
  \end{center}
  \end{figure}

  Since the basic representation of terms already accounts for \isa{{\isasymalpha}}-conversion, Pure equality essentially acts like \isa{{\isasymalpha}{\isasymbeta}{\isasymeta}}-equivalence on terms, while coinciding with bi-implication.


  \medskip Conjunction is defined in Pure as a derived connective, see
  \figref{fig:conjunction}.  This is occasionally useful to represent
  simultaneous statements behind the scenes --- framework conjunction
  is usually not exposed to the user.

  \begin{figure}[htb]
  \begin{center}
  \begin{tabular}{ll}
  \isa{{\isacharampersand}\ {\isacharcolon}{\isacharcolon}\ prop\ {\isasymRightarrow}\ prop\ {\isasymRightarrow}\ prop} & conjunction (hidden) \\
  \isa{{\isasymturnstile}\ A\ {\isacharampersand}\ B\ {\isasymequiv}\ {\isacharparenleft}{\isasymAnd}C{\isachardot}\ {\isacharparenleft}A\ {\isasymLongrightarrow}\ B\ {\isasymLongrightarrow}\ C{\isacharparenright}\ {\isasymLongrightarrow}\ C{\isacharparenright}} \\
  \end{tabular}
  \caption{Definition of conjunction.}\label{fig:equality}
  \end{center}
  \end{figure}

  The definition allows to derive the usual introduction \isa{{\isasymturnstile}\ A\ {\isasymLongrightarrow}\ B\ {\isasymLongrightarrow}\ A\ {\isacharampersand}\ B}, and destructions \isa{A\ {\isacharampersand}\ B\ {\isasymLongrightarrow}\ A} and \isa{A\ {\isacharampersand}\ B\ {\isasymLongrightarrow}\ B}.%
\end{isamarkuptext}%
\isamarkuptrue%
%
\isamarkupsection{Rules \label{sec:rules}%
}
\isamarkuptrue%
%
\begin{isamarkuptext}%
FIXME

  A \emph{rule} is any Pure theorem in HHF normal form; there is a
  separate calculus for rule composition, which is modeled after
  Gentzen's Natural Deduction \cite{Gentzen:1935}, but allows
  rules to be nested arbitrarily, similar to \cite{extensions91}.

  Normally, all theorems accessible to the user are proper rules.
  Low-level inferences are occasional required internally, but the
  result should be always presented in canonical form.  The higher
  interfaces of Isabelle/Isar will always produce proper rules.  It is
  important to maintain this invariant in add-on applications!

  There are two main principles of rule composition: \isa{resolution} (i.e.\ backchaining of rules) and \isa{by{\isacharminus}assumption} (i.e.\ closing a branch); both principles are
  combined in the variants of \isa{elim{\isacharminus}resosultion} and \isa{dest{\isacharminus}resolution}.  Raw \isa{composition} is occasionally
  useful as well, also it is strictly speaking outside of the proper
  rule calculus.

  Rules are treated modulo general higher-order unification, which is
  unification modulo the equational theory of \isa{{\isasymalpha}{\isasymbeta}{\isasymeta}}-conversion
  on \isa{{\isasymlambda}}-terms.  Moreover, propositions are understood modulo
  the (derived) equivalence \isa{{\isacharparenleft}A\ {\isasymLongrightarrow}\ {\isacharparenleft}{\isasymAnd}x{\isachardot}\ B\ x{\isacharparenright}{\isacharparenright}\ {\isasymequiv}\ {\isacharparenleft}{\isasymAnd}x{\isachardot}\ A\ {\isasymLongrightarrow}\ B\ x{\isacharparenright}}.

  This means that any operations within the rule calculus may be
  subject to spontaneous \isa{{\isasymalpha}{\isasymbeta}{\isasymeta}}-HHF conversions.  It is common
  practice not to contract or expand unnecessarily.  Some mechanisms
  prefer an one form, others the opposite, so there is a potential
  danger to produce some oscillation!

  Only few operations really work \emph{modulo} HHF conversion, but
  expect a normal form: quantifiers \isa{{\isasymAnd}} before implications
  \isa{{\isasymLongrightarrow}} at each level of nesting.

\glossary{Hereditary Harrop Formula}{The set of propositions in HHF
format is defined inductively as \isa{H\ {\isacharequal}\ {\isacharparenleft}{\isasymAnd}x\isactrlsup {\isacharasterisk}{\isachardot}\ H\isactrlsup {\isacharasterisk}\ {\isasymLongrightarrow}\ A{\isacharparenright}}, for variables \isa{x} and atomic propositions \isa{A}.
Any proposition may be put into HHF form by normalizing with the rule
\isa{{\isacharparenleft}A\ {\isasymLongrightarrow}\ {\isacharparenleft}{\isasymAnd}x{\isachardot}\ B\ x{\isacharparenright}{\isacharparenright}\ {\isasymequiv}\ {\isacharparenleft}{\isasymAnd}x{\isachardot}\ A\ {\isasymLongrightarrow}\ B\ x{\isacharparenright}}.  In Isabelle, the outermost
quantifier prefix is represented via \seeglossary{schematic
variables}, such that the top-level structure is merely that of a
\seeglossary{Horn Clause}}.

\glossary{HHF}{See \seeglossary{Hereditary Harrop Formula}.}


  \[
  \infer[\isa{{\isacharparenleft}assumption{\isacharparenright}}]{\isa{C{\isasymvartheta}}}
  {\isa{{\isacharparenleft}{\isasymAnd}\isactrlvec x{\isachardot}\ \isactrlvec H\ \isactrlvec x\ {\isasymLongrightarrow}\ A\ \isactrlvec x{\isacharparenright}\ {\isasymLongrightarrow}\ C} & \isa{A{\isasymvartheta}\ {\isacharequal}\ H\isactrlsub i{\isasymvartheta}}~~\text{(for some~\isa{i})}}
  \]


  \[
  \infer[\isa{{\isacharparenleft}compose{\isacharparenright}}]{\isa{\isactrlvec A{\isasymvartheta}\ {\isasymLongrightarrow}\ C{\isasymvartheta}}}
  {\isa{\isactrlvec A\ {\isasymLongrightarrow}\ B} & \isa{B{\isacharprime}\ {\isasymLongrightarrow}\ C} & \isa{B{\isasymvartheta}\ {\isacharequal}\ B{\isacharprime}{\isasymvartheta}}}
  \]


  \[
  \infer[\isa{{\isacharparenleft}{\isasymAnd}{\isacharunderscore}lift{\isacharparenright}}]{\isa{{\isacharparenleft}{\isasymAnd}\isactrlvec x{\isachardot}\ \isactrlvec A\ {\isacharparenleft}{\isacharquery}\isactrlvec a\ \isactrlvec x{\isacharparenright}{\isacharparenright}\ {\isasymLongrightarrow}\ {\isacharparenleft}{\isasymAnd}\isactrlvec x{\isachardot}\ B\ {\isacharparenleft}{\isacharquery}\isactrlvec a\ \isactrlvec x{\isacharparenright}{\isacharparenright}}}{\isa{\isactrlvec A\ {\isacharquery}\isactrlvec a\ {\isasymLongrightarrow}\ B\ {\isacharquery}\isactrlvec a}}
  \]
  \[
  \infer[\isa{{\isacharparenleft}{\isasymLongrightarrow}{\isacharunderscore}lift{\isacharparenright}}]{\isa{{\isacharparenleft}\isactrlvec H\ {\isasymLongrightarrow}\ \isactrlvec A{\isacharparenright}\ {\isasymLongrightarrow}\ {\isacharparenleft}\isactrlvec H\ {\isasymLongrightarrow}\ B{\isacharparenright}}}{\isa{\isactrlvec A\ {\isasymLongrightarrow}\ B}}
  \]

  The \isa{resolve} scheme is now acquired from \isa{{\isasymAnd}{\isacharunderscore}lift},
  \isa{{\isasymLongrightarrow}{\isacharunderscore}lift}, and \isa{compose}.

  \[
  \infer[\isa{{\isacharparenleft}resolution{\isacharparenright}}]
  {\isa{{\isacharparenleft}{\isasymAnd}\isactrlvec x{\isachardot}\ \isactrlvec H\ \isactrlvec x\ {\isasymLongrightarrow}\ \isactrlvec A\ {\isacharparenleft}{\isacharquery}\isactrlvec a\ \isactrlvec x{\isacharparenright}{\isacharparenright}{\isasymvartheta}\ {\isasymLongrightarrow}\ C{\isasymvartheta}}}
  {\begin{tabular}{l}
    \isa{\isactrlvec A\ {\isacharquery}\isactrlvec a\ {\isasymLongrightarrow}\ B\ {\isacharquery}\isactrlvec a} \\
    \isa{{\isacharparenleft}{\isasymAnd}\isactrlvec x{\isachardot}\ \isactrlvec H\ \isactrlvec x\ {\isasymLongrightarrow}\ B{\isacharprime}\ \isactrlvec x{\isacharparenright}\ {\isasymLongrightarrow}\ C} \\
    \isa{{\isacharparenleft}{\isasymlambda}\isactrlvec x{\isachardot}\ B\ {\isacharparenleft}{\isacharquery}\isactrlvec a\ \isactrlvec x{\isacharparenright}{\isacharparenright}{\isasymvartheta}\ {\isacharequal}\ B{\isacharprime}{\isasymvartheta}} \\
   \end{tabular}}
  \]


  FIXME \isa{elim{\isacharunderscore}resolution}, \isa{dest{\isacharunderscore}resolution}%
\end{isamarkuptext}%
\isamarkuptrue%
%
\isadelimtheory
%
\endisadelimtheory
%
\isatagtheory
\isacommand{end}\isamarkupfalse%
%
\endisatagtheory
{\isafoldtheory}%
%
\isadelimtheory
%
\endisadelimtheory
\isanewline
\end{isabellebody}%
%%% Local Variables:
%%% mode: latex
%%% TeX-master: "root"
%%% End:
