%
\begin{isabellebody}%
\def\isabellecontext{logic}%
%
\isadelimtheory
\isanewline
\isanewline
\isanewline
%
\endisadelimtheory
%
\isatagtheory
\isacommand{theory}\isamarkupfalse%
\ logic\ \isakeyword{imports}\ base\ \isakeyword{begin}%
\endisatagtheory
{\isafoldtheory}%
%
\isadelimtheory
%
\endisadelimtheory
%
\isamarkupchapter{Primitive logic \label{ch:logic}%
}
\isamarkuptrue%
%
\isamarkupsection{Variable names%
}
\isamarkuptrue%
%
\begin{isamarkuptext}%
FIXME%
\end{isamarkuptext}%
\isamarkuptrue%
%
\isamarkupsection{Types \label{sec:types}%
}
\isamarkuptrue%
%
\begin{isamarkuptext}%
\glossary{Type class}{FIXME}

  \glossary{Type arity}{FIXME}

  \glossary{Sort}{FIXME}

  FIXME classes and sorts


  \glossary{Type}{FIXME}

  \glossary{Type constructor}{FIXME}

  \glossary{Type variable}{FIXME}

  FIXME simple types%
\end{isamarkuptext}%
\isamarkuptrue%
%
\isamarkupsection{Terms \label{sec:terms}%
}
\isamarkuptrue%
%
\begin{isamarkuptext}%
\glossary{Term}{FIXME}

  FIXME de-Bruijn representation of lambda terms%
\end{isamarkuptext}%
\isamarkuptrue%
%
\begin{isamarkuptext}%
FIXME

\glossary{Schematic polymorphism}{FIXME}

\glossary{Type variable}{FIXME}%
\end{isamarkuptext}%
\isamarkuptrue%
%
\isamarkupsection{Theorems \label{sec:thms}%
}
\isamarkuptrue%
%
\begin{isamarkuptext}%
FIXME

\glossary{Proposition}{A \seeglossary{term} of \seeglossary{type}
\isa{prop}.  Internally, there is nothing special about
propositions apart from their type, but the concrete syntax enforces a
clear distinction.  Propositions are structured via implication \isa{A\ {\isasymLongrightarrow}\ B} or universal quantification \isa{{\isasymAnd}x{\isachardot}\ B\ x} --- anything
else is considered atomic.  The canonical form for propositions is
that of a \seeglossary{Hereditary Harrop Formula}.}

\glossary{Theorem}{A proven proposition within a certain theory and
proof context, formally \isa{{\isasymGamma}\ {\isasymturnstile}\isactrlsub {\isasymTheta}\ {\isasymphi}}; both contexts are
rarely spelled out explicitly.  Theorems are usually normalized
according to the \seeglossary{HHF} format.}

\glossary{Fact}{Sometimes used interchangably for
\seeglossary{theorem}.  Strictly speaking, a list of theorems,
essentially an extra-logical conjunction.  Facts emerge either as
local assumptions, or as results of local goal statements --- both may
be simultaneous, hence the list representation.}

\glossary{Schematic variable}{FIXME}

\glossary{Fixed variable}{A variable that is bound within a certain
proof context; an arbitrary-but-fixed entity within a portion of proof
text.}

\glossary{Free variable}{Synonymous for \seeglossary{fixed variable}.}

\glossary{Bound variable}{FIXME}

\glossary{Variable}{See \seeglossary{schematic variable},
\seeglossary{fixed variable}, \seeglossary{bound variable}, or
\seeglossary{type variable}.  The distinguishing feature of different
variables is their binding scope.}%
\end{isamarkuptext}%
\isamarkuptrue%
%
\isamarkupsubsection{Primitive inferences%
}
\isamarkuptrue%
%
\begin{isamarkuptext}%
FIXME%
\end{isamarkuptext}%
\isamarkuptrue%
%
\isamarkupsubsection{Higher-order resolution%
}
\isamarkuptrue%
%
\begin{isamarkuptext}%
FIXME

\glossary{Hereditary Harrop Formula}{The set of propositions in HHF
format is defined inductively as \isa{H\ {\isacharequal}\ {\isacharparenleft}{\isasymAnd}x\isactrlsup {\isacharasterisk}{\isachardot}\ H\isactrlsup {\isacharasterisk}\ {\isasymLongrightarrow}\ A{\isacharparenright}}, for variables \isa{x} and atomic propositions \isa{A}.
Any proposition may be put into HHF form by normalizing with the rule
\isa{{\isacharparenleft}A\ {\isasymLongrightarrow}\ {\isacharparenleft}{\isasymAnd}x{\isachardot}\ B\ x{\isacharparenright}{\isacharparenright}\ {\isasymequiv}\ {\isacharparenleft}{\isasymAnd}x{\isachardot}\ A\ {\isasymLongrightarrow}\ B\ x{\isacharparenright}}.  In Isabelle, the outermost
quantifier prefix is represented via \seeglossary{schematic
variables}, such that the top-level structure is merely that of a
\seeglossary{Horn Clause}}.

\glossary{HHF}{See \seeglossary{Hereditary Harrop Formula}.}%
\end{isamarkuptext}%
\isamarkuptrue%
%
\isamarkupsubsection{Equational reasoning%
}
\isamarkuptrue%
%
\begin{isamarkuptext}%
FIXME%
\end{isamarkuptext}%
\isamarkuptrue%
%
\isamarkupsection{Low-level specifications%
}
\isamarkuptrue%
%
\begin{isamarkuptext}%
FIXME

\glossary{Constant}{Essentially a \seeglossary{fixed variable} of the
theory context, but slightly more flexible since it may be used at
different type-instances, due to \seeglossary{schematic
polymorphism.}}

\glossary{Axiom}{FIXME}

\glossary{Primitive definition}{FIXME}%
\end{isamarkuptext}%
\isamarkuptrue%
%
\isadelimtheory
%
\endisadelimtheory
%
\isatagtheory
\isacommand{end}\isamarkupfalse%
%
\endisatagtheory
{\isafoldtheory}%
%
\isadelimtheory
%
\endisadelimtheory
\isanewline
\end{isabellebody}%
%%% Local Variables:
%%% mode: latex
%%% TeX-master: "root"
%%% End:
